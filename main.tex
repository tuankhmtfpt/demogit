\documentclass[12pt,a4paper]{book}
\usepackage[left=1.75cm, right=1.75cm, top=2cm, bottom=2cm]{geometry}
\usepackage[utf8]{vietnam}
\usepackage[many]{tcolorbox}
\usepackage[final]{pdfpages}
\usepackage[titles]{tocloft}
\usepackage[solcolor]{ex_test}
\usepackage[hidelinks,unicode]{hyperref}
\usepackage{fontawesome,amsmath,amssymb,amsfonts,enumerate,currfile}
\usepackage{tikz}
\usetikzlibrary{lindenmayersystems}
\usetikzlibrary[shadings]
\usepackage{esvect}
\usepackage{pifont}
\allowdisplaybreaks
\usepackage{titletoc}
\usepackage{verbatim}
\usepackage{titlesec}
\usepackage{titledot}
\usepackage{mathptmx}
%\usepackage{background}
%\usetikzlibrary{backgrounds}
%\backgroundsetup{contents={\scshape\fontfamily{cmr}\selectfont\LaTeX\, by Mathpiad},angle=90,scale=3.75,color=gray!30!white, placement = bottom}
\renewcommand{\baselinestretch}{1}
\renewcommand{\loigiaiEX}{\sffamily\textbf{Lời giải.}}
\renewcommand{\thechapter}{\Roman{chapter}}
\renewcommand{\thefootnote}{\arabic{footnote}}
\renewcommand{\thesection}{\arabic{section}}
\def\qedEX{\ensuremath{\square}}
\renewcommand{\labelenumi}{\alph{enumi})}
\def\sao{\color{black}\small\faCheckCircleO}
\renewcommand\labelitemi{\faGenderless}
\renewcommand\labelitemii{$\bullet$}
\renewcommand{\labelenumi}{\circled{\small\fontfamily{lmss}\bfseries\selectfont\theenumi}}
\colorlet{tuancolor}{black}
%\colorlet{tuancolor}{blue!60!black}
\renewcommand{\ge }{\geqslant}
\renewcommand{\geq }{\geqslant}
\renewcommand{\le }{\leqslant}
\renewcommand{\leq }{\leqslant}
\newcommand{\suy}{\Rightarrow}
\newcommand{\tud}{\Leftrightarrow}
\newcommand{\hoac}[1]{\left[\begin{aligned}#1\end{aligned}\right.}
\newcommand{\heva}[1]{\left\{\begin{aligned}#1\end{aligned}\right.}
\newcommand{\plogo}{\fbox{$\mathcal{PL}$}}
\newcommand{\credha}[2]{#1,\begin{it} #2 \end{it}}
\newcommand{\credhai}[2]{#1,\begin{it} #2. \end{it}}
\newcommand{\credba}[3]{#1,\begin{it} #2, \end{it}#3.}
\newcommand{\credbon}[4]{#1,\begin{it} #2, \end{it}#3, #4.}
\newtheorem{bai}{\sffamily Bài}
\newcommand{\chu}[1]{{\color{tuancolor}\sffamily{\bfseries #1}}}
\newcommand{\nguon}[1]{
\begin{flushright}
\bfseries\sffamily #1
\end{flushright}}
\newcommand{\tron}[1]{\left(\di #1 \right)}
\newcommand{\vuong}[1]{\left[ #1 \right]}
\newcommand{\ddu}[3]{ #1\equiv #2\pmod{#3}}
\def\db{\dbinom}
\def\cm{\textbf{\color{tuancolor}\sffamily Chứng minh. }}
\def\nx{\textbf{\color{tuancolor}\sffamily Nhận xét. }}
\def\lx{\textbf{\color{tuancolor}\sffamily Lưu ý. }}
\def\di{\displaystyle}
\def\bode{\textbf{\color{tuancolor}\sffamily Bổ đề. }}
\def\vr{\textbf{\color{tuancolor}\sffamily Ví dụ. }}
\def\hq{\textbf{\color{tuancolor}\sffamily Hệ quả. }}
\def\tc{\textbf{\color{tuancolor}\sffamily Tính chất. }}
\def\dn{\textbf{\color{tuancolor}\sffamily Định nghĩa. }}
\def\vec{\vv}
\def\overrightarrow{\vv}
\DeclareSymbolFont{symbolsC}{U}{txsyc}{m}{n}
\DeclareMathSymbol{\varparallel}{\mathrel}{symbolsC}{9}
\DeclareMathSymbol{\parallel}{\mathrel}{symbolsC}{9}
\colorlet{tcbcol@back}{tcbcolback}
\colorlet{tcbcol@frame}{tcbcolframe}
\newtcolorbox{luuy}[1][]{enhanced,breakable,
	before skip=2mm,after skip=3mm,
	boxrule=0pt,left=5mm,right=2mm,top=1mm,bottom=1mm,
	colback=gray!10!white,
	colframe=white,
	sharp corners,rounded corners=southeast,arc is angular,arc=3mm,
	underlay={%
		\path[fill=tcbcol@back!80!white] ([yshift=3mm]interior.south east)--++(-0.4,-0.1)--++(0.1,-0.2);
		\path[draw=tcbcol@frame,shorten <=-0.05mm,shorten >=-0.05mm] ([yshift=3mm]interior.south east)--++(-0.4,-0.1)--++(0.1,-0.2);
		\path[fill=tuancolor,draw=none] (interior.south west) rectangle node[white]{\huge\bfseries !}
		([xshift=4mm]interior.north west);
	},
	drop fuzzy shadow, #1}
\newtheorem{kq}{\color{tuancolor}\sffamily Hệ quả}
\newtheorem{btt}{\color{tuancolor}\sffamily Bài}[section]
\newtheorem{tx}{\color{tuancolor}\sffamily Tính chất}[section]
\newtheorem{bx}{\color{tuancolor}\sffamily Ví dụ}
\def\beginboxbx{\begin{tcolorbox}[colframe=gray!15!white,colback=gray!15!white,boxrule=0pt,arc=0.75mm, coltitle=white,breakable]}
\def\endboxbx{\end{tcolorbox}}
\AtBeginEnvironment{bx}{
\beginboxbx
	\renewcommand{\loigiai}[1]{
	\renewcommand{\immini}[2]{
	\setbox\imbox=\vbox{\hbox{##2}}
	\widthimmini=\wd\imbox
	\IMleftright{##1}{##2}
	}
	\endboxbx
	\begin{onlysolution}
	#1
	\end{onlysolution}
	\def\endboxbx{}
	}
}
\AtEndEnvironment{bx}{\endboxbx}
\newtheorem{bd}{\color{tuancolor}\sffamily Bổ đề}
\newtheorem{dx}{\color{tuancolor}\sffamily Định nghĩa}
\newtheorem{gbtt}{\color{tuancolor}\sffamily Bài}[section]
\AtBeginEnvironment{gbtt}{
\beginboxbx
	\renewcommand{\loigiai}[1]{
	\renewcommand{\immini}[2]{
	\setbox\imbox=\vbox{\hbox{##2}}
	\widthimmini=\wd\imbox
	\IMleftright{##1}{##2}
	}
	\endboxbx
	\begin{onlysolution}
	#1
	\end{onlysolution}
	\def\endboxbx{}
	}
}
\AtEndEnvironment{gbtt}{\endboxbx}
\def\beginboxdx{\begin{tcolorbox}
[enhanced jigsaw,breakable,
 colback=gray!15!white,boxrule=0pt,frame hidden,
 borderline west={1.5mm}{-2mm}{tuancolor}]}
\def\endboxdx{\end{tcolorbox}}
\AtBeginEnvironment{dx}{\beginboxdx}
\AtEndEnvironment{dx}{\endboxdx}
\setcounter{secnumdepth}{4}
\setcounter{tocdepth}{2}
\newcounter{CircLabel}
\newcommand*\CircLabel{
	\refstepcounter{CircLabel}
	\hskip1em\llap{\alph{CircLabel})}}
\setlistsEX{label=\CircLabel}
\AtBeginEnvironment{listEX}{\setcounter{CircLabel}{0}}
\AtBeginEnvironment{enumEX}{\setcounter{CircLabel}{0}}
\pagestyle{fancy}
\fancyhf{}
\renewcommand{\headrulewidth}{0pt}
\renewcommand{\footrulewidth}{0pt}
\renewcommand{\sectionmark}[1]{ \markright{#1}{} }
\fancyhead[CE]{\footnotesize\fontfamily{iwona}\selectfont \MakeUppercase{\leftmark}}
\fancyhead[CO]{\footnotesize\fontfamily{iwona}\selectfont \MakeUppercase{\rightmark}}
\cfoot{\sffamily\circEX{\thepage}}
\newtheorem{vd}{Bài}
\setlength{\parindent}{0pt}
\newtheorem{dl}{\sffamily Định lý}
\newenvironment{light}{\begin{tcolorbox}[colback=gray!20!white,boxrule=0pt,arc=0mm, colframe=gray!20!white,breakable]}{\end{tcolorbox}}
\titleformat{\section}
   {\fontfamily{cmss}\selectfont \Large\bfseries\color{tuancolor}}{\thesection.}{0.25em}{
   \setcounter{bx}{0}
   \setcounter{dx}{0}
   }
\titleformat{\subsection}
   {\fontfamily{cmss}\selectfont \large\bfseries\color{tuancolor}}{\thesubsection.}{0.25em}{
   \setcounter{bx}{0}
   \setcounter{dx}{0}
   }
\titleformat{\subsubsection}
   {\fontfamily{cmss}\selectfont \large\bfseries\color{tuancolor}}{\thesubsubsection.}{0.25em}{
   }
\titleformat{\chapter}[display]
  {\centering\sffamily\huge\bfseries\color{tuancolor}}{\chaptertitlename \,\thechapter}{10pt}{\Huge}
\titlespacing*{\chapter}
  {0pt}{20pt}{30pt}
\hypersetup{
     linkcolor=tuancolor,
     filecolor=tuancolor,
     citecolor = black,      
     urlcolor=tuancolor,
     pdftitle={FileSach},
     pdfpagemode=FullScreen,
     }
\renewcommand\cftchapfont{\normalfont}
\renewcommand\cftchappagefont{\normalfont}
\AtBeginDocument{\renewcommand\contentsname{Mục lục}}
\contentsmargin[1cm]{0cm}
\titlecontents{chapter}[0em]{\vskip12pt\bfseries\sffamily}
{\thecontentslabel\enspace}
{\hspace{1.05em}}
{ \hfill\contentspage}[\vskip 6pt]
\titlecontents{section}[1em]{\sffamily}
{\thecontentslabel\enspace}
{}
{\titlerule*[1pc]{.}\quad\contentspage}[\vskip 4pt]
\titlecontents{subsection}[2.7em]{\sffamily}
{\thecontentslabel\enspace}
{}
{\titlerule*[1pc]{.}\quad\contentspage}[\vskip 3pt]
\AtBeginDocument{\renewcommand{\contentsname}{\hfill\sffamily Mục lục \hfill}}
\definecolor{mathpiad}{HTML}{EE4B9C}
\definecolor{mathpiad2}{HTML}{58575A}
\DeclareMathOperator*{\SumInt}{
\mathchoice
  {\ooalign{\color{mathpiad}$\displaystyle\sum$\cr\hidewidth$\color{mathpiad2}\displaystyle\int$\hidewidth\cr}}
  {\ooalign{\raisebox{.14\height}{\scalebox{.7}{$\textstyle\sum$}}\cr\hidewidth$\textstyle\int$\hidewidth\cr}}
  {\ooalign{\raisebox{.2\height}{\scalebox{.6}{$\scriptstyle\sum$}}\cr$\scriptstyle\int$\cr}}
  {\ooalign{\raisebox{.2\height}{\scalebox{.6}{$\scriptstyle\sum$}}\cr$\scriptstyle\int$\cr}}
}
\newcommand*{\Scale}[2][4]{\scalebox{#1}{$#2$}}
\begin{document}
\input{titlepage}
\input{0. Lời nói đầu}
\newpage
\thispagestyle{empty}
\setcounter{page}{0}
\tableofcontents
\chapter{Ước, bội và chia hết}
Trong lí thuyết số, quan hệ chia hết giữa các số nguyên là một trong những quan hệ thứ tự cơ bản và quan trọng nhất. Nó là nền tảng của các quan hệ khác trong số học như quan hệ đồng dư, quan hệ ước, bội. Ta đã được làm quen với vấn đề này kể từ cấp tiểu học, thông qua các phép trong tính bảng cửu chương, hay những dấu hiệu chia hết cho $2,3,5,9.$ Tuy đơn giản song quan hệ chia hết lại có rất nhiều tính chất hay và được thể hiện dưới nhiều dạng phát biểu khác nhau trong các bài toán số học. \\ \\
Chương I của cuốn sách tập trung xây dựng những tính chất cơ bản và thông dụng nhất của quan hệ chia hết thông qua 9 phần
\begin{itemize}
    \item\chu{Phần 1.} Các định nghĩa, tính chất và bài tập cơ bản.
    \item\chu{Phần 2.} Tính chia hết của đa thức cho một số nguyên
    \item\chu{Phần 3.} Đồng dư thức với số mũ lớn.
    \item\chu{Phần 4.} Một số bổ đề đồng dư thức với số mũ nhỏ.
    \item\chu{Phần 5.} Bất đẳng thức trong chia hết.
    \item\chu{Phần 6.} Tính nguyên tố cùng nhau.
    \item\chu{Phần 7.} Phép đặt ước chung đôi một cho ba biến số.
    \item\chu{Phần 8.} Bài toán về các ước của một số nguyên dương.
    \item\chu{Phần 9.} Sự tồn tại trong các bài toán chia hết, ước, bội.
\end{itemize}

\section{Các định nghĩa, tính chất và bài tập cơ bản}
Trong phần này, tác giả xin phép chỉ trình bày lí thuyết và các ví dụ minh họa, mà không đưa ra bất cứ bài tập tự luyện nào. Chúng ta sẽ có mục bài tập tự luyện ở các phần sau.
\subsection{Phép chia hết, phép chia có dư}
\begin{dx}
Cho hai số nguyên $a, b$. Nếu tồn tại số nguyên $q$ sao cho $a=bq$ thì ta nói rằng $a$ chia hết cho $b.$ Ngược lại, nếu không tồn tại số nguyên $q$ sao cho $a=bq$ thì ta nói rằng $a$ không chia hết cho $b.$
\end{dx}
Đối với các phép chia hết kể trên, ta có một vài kí hiệu sau đây.
\begin{enumerate}
    \item $a$ chia hết cho $b$ được kí hiệu là $a\vdots\: b.$
    \item $b$ chia hết $a$ được kí hiệu là $b\mid a.$
    \item $a$ không chia hết cho $b$ được kí hiệu là $a\not\vdots\: b.$
    \item $b$ không chia hết $a$ được kí hiệu là $b\nmid a.$     
\end{enumerate}
Dưới đây là một vài ví dụ minh họa.
\begin{enumerate}
    \item Do $15=3\cdot 5,$ ta nói rằng "$15$ chia hết cho $3$" hoặc "$3$ chia hết $15$", và kí hiệu $15\vdots 3$ hoặc $3 \mid 15.$
    \item Do không tồn tại số nguyên $q$ nào để cho $2q=7,$ ta nói rằng "$7$ không chia hết cho $2$" hoặc "$2$ không chia hết $7$", và kí hiệu $2\nmid 7.$
\end{enumerate}

\begin{dx}
Cho hai số nguyên $a,d,$ trong đó $d\ne 0.$ Khi đó, tồn tại duy nhất các số nguyên $q,r$ sao cho $a=qd+r$ và $0\le r<|d|.$ Ta gọi phép chia $a$ cho $d$ là phép chia có dư, với thương là $q$ và dư là $r.$
\end{dx}

Chẳng hạn, ta nói $19$ chia cho $-5$ được thương là $-3,$ dư là $4,$ bởi vì 
$$19=(-5)\cdot(-3)+4,\quad 0<4<|-5|.$$
Kèm theo đó, chúng ta cũng có một vài tính chất liên quan đến phép chia hết.

\begin{light}
\chu{Các tính chất cơ bản}
\begin{enumerate}
    \item Mọi số nguyên khác 0 luôn chia hết cho chính nó.
    \item Các dấu hiệu chia hết cho $2,3,5,9,10,$ chúng ta đã được học ở cấp học dưới. 
    \item Với $a,b,c$ là các số nguyên, nếu $a$ chia hết cho $b,b$ chia hết cho $c$ thì $a$ chia hết cho $c.$
    \item Với $a,b,c$ là các số nguyên, nếu cả $a$ và $b$ đều chia hết cho $c$ thì tổng, hiệu, tích của $a$ và $b$ cũng chia hết cho $c$ 
    \item  Với $a,b,c,d$ là các số nguyên, nếu $a$ chia hết cho $b$ và $c$ chia hết cho $d$ thì tích $ac$ chia hết cho tích $bd$. 
    \item Với $a,b$ là các số nguyên, nếu $a$ chia hết cho $b$ thì hoặc $a=0,$ hoặc $|a|\ge |b|.$
\end{enumerate}
\chu{Các tính chất nâng cao}    
\begin{enumerate}
    \item Tổng của $n$ số nguyên liên tiếp luôn chia hết cho $n.$
    \item Tích của $n$ số nguyên liên tiếp luôn chia hết cho $n!.$    
    \item Với $a,b$ là các số nguyên phân biệt và $n$ là số tự nhiên, $a^n-b^n$ chia hết cho $a-b.$
    \item Với $a,b$ là các số nguyên và $n$ là số tự nhiên lẻ, $a^n+b^n$ chia hết cho $a+b.$    
\end{enumerate}
\end{light}

\subsubsection*{Ví dụ minh họa}
\begin{bx}
Tìm tất cả các số nguyên dương
\begin{enumerate}[a,]
    \item Có dạng $\overline{12a56c}$, đồng thời chia hết cho $9$ và $5.$
    \item Có dạng $\overline{a432c},$ đồng thời chia hết cho cả $2,5$ và $9.$
\end{enumerate}
\loigiai{
\begin{enumerate}[a,]
    \item Số đã cho chia hết cho $5,$ thế nên $c=0$ hoặc $c=5.$
    \begin{itemize}
        \item Với $c=0,$ do số đã cho chia hết cho $9$ nên tổng các chữ số của nó cũng chia hết cho $9,$ tức là
        $$1+2+a+5+6+0=14+a=18+(a-4)$$
        cũng chia hết cho $9$, lại do $0\le a\le 9$ nên $a=4.$
        \item Với $c=5,$ ta thực hiện tương tự trường hợp trước để chỉ ra $a=8.$
    \end{itemize}
    Tổng kết lại, tất cả các số nguyên dương thỏa yêu cầu là $122560$ và $128565.$
    \item Số đã cho chia hết cho cả $2$ và $5,$ và bắt buộc $c=0.$ Lập luận tương tự như ý a, ta chỉ ra 
    $$a+4+3+2+0=9+a$$
    chia hết cho $9.$ Do $0\ge a>0$ nên bắt buộc $a=9.$\\
    Tổng kết lại, số nguyên dương duy nhất thỏa yêu cầu là $94320.$
\end{enumerate}
}
\end{bx}

\begin{bx}
Tích của bốn số tự nhiên liên tiếp là $255024.$ Tìm bốn số đó.
\loigiai{
Tích đã cho không chia hết cho $5,$ vì thế đây là tích của bốn số tự nhiên liên tiếp không chia hết cho $5.$ Dựa vào nhận xét
$20^3<255024<25^3,$
ta chỉ ra bốn số cần tìm là $21,22,23,24.$
}
\end{bx}

\begin{bx}
Cho hai số tự nhiên $a$ và $b$ tuỳ ý có số dư trong phép chia cho $9$ theo thứ tự là $r_1$ và $r_2$. Chứng minh rằng ${r}_{1}{r}_{2}$ và $ab$ có cùng số dư trong phép chia cho $9.$
\loigiai{
Đặt $a=9a_1+r_1$ và $b=9b_1+r_2.$ Phép đặt này cho ta
$$ab-r_1r_2=\tron{9a_1+r_1}\tron{9b_1+r_2}-r_1r_2=81a_1b_1+9\tron{b_1r_1+a_1r_2}.$$
Do $81a_1b_1$ và $9\tron{b_1r_1+a_1r_2}$ đều chia hết cho $9,$ ta có $ab-r_1r_2$ chia hết cho $9.$ Từ đây, ta suy ra $ab$ và $r_1r_2$ có cùng số dư khi chia cho $9.$ Bài toán được chứng minh.}
\end{bx}

\begin{bx}
Chứng minh rằng với mọi số tự nhiên $n,$ ta luôn có $17^n-2^n\ge 15.$
\loigiai{
Ta đã biết, với $a,b$ là các số nguyên phân biệt và $n$ là số tự nhiên, $a^n-b^n$ chia hết cho $a-b.$ Theo như tính chất kể trên, ta chỉ ra $17^n-2^n$ chia hết cho $15,$ thế nên $17^n-2^n\ge 15.$ Bài toán được chứng minh.}
\end{bx}

\begin{bx} \
\begin{enumerate}[a,]
    \item Cho $A=2+2^{2}+2^{3}+\ldots+2^{60}.$ Chứng minh rằng $A$ chia hết cho $3,7$ và $15.$
    \item Cho $B=3+3^{3}+3^{5}+\ldots+3^{1991}$. Chứng minh rằng $B$ chia hết cho $13$ và $41.$
\end{enumerate}
\loigiai{
\begin{enumerate}[a,]
    \item Ta sẽ chia bài lời giải làm $3$ phần như sau.
    \begin{itemize}
        \item \chu{Chứng minh $A$ chia hết cho $3.$}\\
        Biến đổi $A$, ta thu được
        $$A=2\tron{1+2}+2^3\tron{1+2}+\cdots+2^{59}\tron{1+2}=2\cdot3+2^3\cdot3+\cdots2^{59}\cdot3.$$
        Từ đây, ta suy ra $A$ chia hết cho $3.$
         \item \chu{Chứng minh $A$ chia hết cho $7.$}\\
        Biến đổi $A$, ta thu được
        $$A=2\tron{1+2+2^2}+2^4\tron{1+2+2^2}+\cdots+2^{58}\tron{1+2+2^2}=2\cdot7+2^4\cdot7+\cdots+2^{58}\cdot7.$$
        Từ đây, ta suy ra $A$ chia hết cho $7.$
         \item \chu{Chứng minh $A$ chia hết cho $15.$}\\
        Biến đổi $A$, ta thu được
       $$A=2\tron{1+2+2^2+2^3}+\cdots+2^{57}\tron{1+2+2^2+2^3}=2\cdot15+2^5\cdot15+\cdots+2^{57}\cdot15.$$
        Từ đây, ta suy ra $A$ chia hết cho $15.$
    \end{itemize}
    Như vậy, bài toán đã cho được chứng minh.
    \item Ta sẽ chia bài lời giải làm $2$ phần như sau.
    \begin{itemize}
        \item \chu{Chứng minh $B$ chia hết cho $13.$}\\
        Biến đổi $B$, ta thu được
        $$B=3\tron{1+3^2+3^4}+\cdots+3^{1987}\tron{1+3^2+3^4}=3\cdot91+3^7\cdot91+\cdots+3^{1987}\cdot91.$$
        Vì $91$ chia hết cho $13$, ta suy ra $B$ cũng chia hết cho $13.$
         \item \chu{Chứng minh $B$ chia hết cho $7.$}\\
        Biến đổi $B$, ta thu được
        \begin{align*}
            B&=3\tron{1+3^2+3^4+3^6}+3^9\tron{1+3^2+3^4+3^6}+\cdots+3^{1985}\tron{1+3^2+3^4+3^6}\\
            &=3\cdot820+3^9\cdot820+\cdots3^{1985}\cdot820.
        \end{align*}
         Vì $820$ chia hết cho $41$, ta suy ra $B$ cũng chia hết cho $41.$
    \end{itemize}
    Như vậy, bài toán được chứng minh.
\end{enumerate}}
\end{bx}

\begin{bx}
Cho $a,b,c$ là các số nguyên khác $0$ thỏa mãn điều kiện
$$\left(\frac{1}{a}+\frac{1}{b}+\frac{1}{c}\right)^{2}=\frac{1}{a^{2}}+\frac{1}{b^{2}}+\frac{1}{c^{2}}.$$
Chứng minh rằng $a^3+b^3+c^3$ chia hết cho $3.$
\nguon{Chọn học sinh giỏi lớp 9 thành phố Thanh Hóa 2017}
\loigiai{
Đẳng thức ở giả thiết đã cho tương đương
$$\left(\frac{1}{a}+\frac{1}{b}+\frac{1}{c}\right)^{2}=\frac{1}{a^{2}}+\frac{1}{b^{2}}+\frac{1}{c^{2}} \Leftrightarrow 2\left(\frac{1}{a b}+\frac{1}{b c}+\frac{1}{c a}\right)=0 \Leftrightarrow \frac{a+b+c}{a b c}=0.$$
Vi ${a}, {b}, {c} \neq 0$ nên ${a}+{b}+{c}=0.$ Từ đây, ta lần lượt suy ra
$$
\begin{aligned}
a+b=-c 
&\Rightarrow(a+b)^{3}=(-c)^{3}\\
&\Rightarrow a^{3}+b^{3}+3 a b(a+b)=-c^{3}\\
&\Rightarrow a^{3}+b^{3}+c^{3}=3 a b c.
\end{aligned}$$
Như vậy $a^3+b^3+c^3$ chia hết cho $3.$ Bài toán được chứng minh.}
\end{bx}

\begin{bx}
Cho $x,y,z$ là các số nguyên dương phân biệt. \\
Chứng minh rằng: $(x-y)^{5}+(y-z)^{5}+(z-x)^{5}$ chia hết cho $5(x-y)(y-z)(z-x).$
\loigiai{Đặt ${a}={x}-{y}, b={y}-{z}.$ Phép đặt này cho ta $z-x=a-b,$ và bài toán quy về chứng minh 
$$5 a b(a+b)\mid \vuong{(a+b)^{5}-a^{5}-b^{5}}.$$
Dựa theo phép đặt kể trên, ta nhận thấy rằng
\begin{align*}
    (a+b)^{5}-a^{5}-b^{5}&=5 a^{4} b+10 a^{3} b^{2}+10 a^{2} b^{3}+5 a b^{4}\\&=5 a b\left(a^{3}+2 a^{2} b+2 a b^{2}+b^{3}\right) \\&=5 a b\left[\left(a^{3}+b^{3}\right)+\left(2 a^{2} b+2 a b^{2}\right)\right]\\&=5 a b\left[(a+b)\left(a^{2}-a b+b^{2}\right)+2 a b(a+b)\right] \\ &=5 a b(a+b)\left(a^{2}+a b+b^{2}\right).
\end{align*}
Biến đổi phía trên chính là cơ sở để ta suy ra điều phải chứng minh.}
\end{bx}

\begin{bx}
Cho số nguyên dương $n.$ Chứng minh rằng
\begin{enumerate}[a,]
    \item $A=2n+\underbrace{11 \ldots 1}_{n\text{ chữ số}}$ chia hết cho $3.$
    \item $B=10^n+18n-1$ chia hết cho $27.$
    \item  $C=10^n+72 n-1$ chia hết cho $81.$
\end{enumerate}
\loigiai{
\begin{enumerate}[a,]
    \item Tổng các chữ số của số $\underbrace{11 \ldots 1}_{n\text{ chữ số}}$ là $n.$ Theo đó, số 
    $$A=2n+\underbrace{11 \ldots 1}_{n\text{ chữ số}}$$
    có cùng số dư với $2n+n=3n$ khi chia cho $3.$ Do $3n$ chia hết cho $3,$ bài toán được chứng minh.
    \item Biến đổi  $10^n+18n-1$, ta thu được 
    $$10^n+18n-1=\tron{10^n-1}+18n=\underbrace{99\ldots9}_{n\text{ chữ số}}+18n=9\tron{2n+\underbrace{11 \ldots 1}_{n\text{ chữ số}}}.$$
    Chứng minh hoàn toàn tương tự ý a, ta có $2n+\underbrace{11 \ldots 1}_{n\text{ chữ số}}$ chia hết cho $3.$ \\
    Từ đây, ta suy ra $B$ chia hết cho $27.$ Bài toán được chứng minh.
    \item Ta nhận thấy rằng
    $$C=10^n+72n-1=\tron{10^n-1}+72n=\underbrace{99\ldots9}_{n\text{ chữ số}}+72n=9\tron{8n+\underbrace{11 \ldots 1}_{n\text{ chữ số}}}.$$
    Số $\underbrace{11 \ldots 1}_{n\text{ chữ số}}$ có tổng các chữ số là $n,$ vì thế $8n+\underbrace{11 \ldots 1}_{n\text{ chữ số}}$ có cùng số dư với $8n+n=9n$ khi chia cho $9$. Theo đó $8n+\underbrace{11 \ldots 1}_{n\text{ chữ số}}$ chia hết cho $9,$ hay $C$ chia hết cho $81.$ Bài toán được chứng minh.
\end{enumerate}}
\end{bx}

%nguyệt anh
\begin{bx}
Giả sử 3 số tự nhiên $\overline{abc},\overline{bca},\overline{cab}$ đều chia hết cho $37$. Chứng minh rằng
$$a^{3}+b^{3}+c^{3}-3abc$$
cũng chia hết cho $37$.
\loigiai{
Bạn đọc dễ dàng khai triển hai vế để đưa ra nhận xét rằng
$$a^{3}+b^{3}+c^{3}-3abc=\overline{abc}\tron{c^{2}-ab}+\overline{bca}\tron{a^{2}-bc}+\overline{cab}\tron{b^{2}-ac}$$
Do $\overline{abc},\overline{bca},\overline{cab}$ đều chia hết cho $37$, $a^{3}+b^{3}+c^{3}-3abc$ cũng chia hết cho $37.$ \\Bài toán được chứng minh.}
\end{bx}

\subsection{Ước, bội, ước chung, bội chung}
\begin{dx}
Với hai số nguyên $a,b$ thỏa mãn $a$ chia hết cho $b,$ ta nói $a$ là bội của $b,$ còn $b$ là ước của $a.$
\end{dx}
Chẳng hạn, do $20$ chia hết cho $5,$ ta nói $5$ là ước của $20,$ và $20$ là bội của $5.$

\begin{dx}
Ước chung của hai hay nhiều số là uớc của tất cả các số đó.
\end{dx}
Ví dụ, khi viết tập hợp $A$ các ước của $4$ và tập hợp $B$ các ước của $6,$ ta có
$$
\begin{aligned}
A&=\{-4;-2;-1;1 ; 2 ; 4\}, \\
B&=\{-6;-3;-2;-1;1 ; 2 ; 3 ; 6\}.
\end{aligned}
$$
Các số $-2,\ -1,\ 1$ và $2$ vừa là ước của $4,$ vừa là ước của $6.$ Ta nói chúng là các ước chung của $4$ và $6.$

\begin{dx}
Bội chung của hai hay nhiếu số là bội của tất cả các số dó.
\end{dx}
Ví du, khi viết tập hợp $A$ các bội của $4$ và tập hợp $B$ các bội của $6,$ ta có
\begin{align*}
    A&=\{\ldots-24;-18;-12;-8;-4;0;4;8;12;16;20;24;\ldots\},\\
    B&=\{\ldots;-24;-18;-12;-6;0;6;12;18;24;\ldots\}.
\end{align*}
Các số $\ldots,-24,-12,0,12,24, \ldots$ vừa là bội của $4$, vừa là bội của $6.$\\ Ta nói chúng là các bội chung của $4$ và $6.$
\subsubsection*{Ví dụ minh họa}

\begin{bx} \
\begin{enumerate}[a,]
    \item Tất cả các ước chung nguyên dương nhỏ hơn $210$ của $500$ và $350$.
    \item Tất cả các bội chung nguyên dương nhỏ hơn $1000$ của $48$ và $84$.
\end{enumerate}
\loigiai{
\begin{enumerate}[a,]
    \item Do $\tron{500, 350} = 50$, tập ước chung cần tìm chính là tập ước dương của $50$, và là 
    $$S = \left\{{1;2; 5;10; 25; 50}\right\}.$$
    \item Do $\vuong{48, 84} = 336$, tập bội chung cần tìm chính là tập bội dương nhỏ hơn $1000$ của $336$ và là
    $$S = \left\{336;672\right\}.$$
\end{enumerate}
}
\end{bx}

\begin{bx}
Có bao nhiêu phép chia hết có số bị chia là $784$, đồng thời thương và dư là các số tự nhiên giống nhau gồm hai chữ số?
\loigiai{Gọi số tự nhiên $a$ là thương của phép chia, do đó $a$ cũng là số dư. Từ đây, ta thu được $784=ax+a,$ hay
$$784 = a\tron{x+1},$$
trong đó $x$ là số nguyên dương. Ta nhận $a$ là ước dương  có $2$ chữ số của $78,$ thế nên
$$a\in\left\{14;16;28;49;56;98\right\}.$$ 
Ta sẽ kiểm tra trực tiếp từng trường hợp.
\begin{enumerate}
    \item Với $a=14,$ phép chia thu được là $784\:\div 55=14,$ dư $14.$
    \item Với $a=16,$ phép chia thu được là $784\:\div 48=16,$ dư $16.$    
    \item Với $a=28,49,56,98,$ ta không thu được phép chia nào thỏa mãn.
\end{enumerate}
Như vậy, có tổng cộng $2$ phép chia thỏa mãn yêu cầu.}
\end{bx}

\begin{bx}
Tìm tất cả các phép chia có số bị chia là $1817,$ đồng thời có thương và dư là các số tự nhiên giống nhau.
\loigiai{
Gọi số tự nhiên $a$ là thương của phép chia, do đó $a$ cũng là số dư. Từ đây, ta thu được $1817 = ax + a ,$ hay
$$1817 = a\tron{x+1},$$
trong đó $x$ là số nguyên dương. Ta nhận thấy $a$ là ước dương của $1817,$ thế nên
 $$a\in\left\{1;23;79\right\}.$$ Sau khi kiểm tra từng trường hợp tương tự như bài trên, ta tìm được tổng cộng hai phép chia thỏa mãn yêu cầu.}
\end{bx}

\begin{bx}
Có bao nhiêu cách viết $275$ thành tổng của $n$ số nguyên dương liên tiếp?
\loigiai{
Gọi $n$ số liên tiếp đó là $a+1, a+2, \ldots, a+n$ với $a$ là số tự nhiên. Từ giả thiết, ta có
$$\tron{2a+n+1}n = 275\cdot2 \Leftrightarrow \tron{2a+n+1}n = 550.$$
Ta chú ý rằng $2a+b \geq n$, và ngoài ra
$$ 550 =2 \cdot275 = 5\cdot110= 10\cdot55=11\cdot50 = 22\cdot25.$$
Vậy có $5$ cách viết $275$ thành tổng của $n$ số nguyên dương liên tiếp.
}
\end{bx}

\begin{bx}
Có bao nhiêu cách viết $333$ thành tổng của $n$ số nguyên dương lẻ liên tiếp?
\loigiai{
Gọi $n$ số liên tiếp đó là $a+2, a+4, \ldots, a+2n$ với $a$ là số nguyên lẻ. Từ giả thiết, ta có
$$\tron{2a+2n+2}n = 333\cdot2 \Leftrightarrow \tron{a+n+1}n = 333.$$
Ta nhận thấy rằng $a+n+1 \geq n$, và $333 =3 \cdot11 = 9\cdot37 .$\\
Vậy có $2$ cách viết $333$ thành tổng của $n$ số nguyên dương lẻ liên tiếp.
}
\end{bx}

\begin{bx}
Tìm tất cả các số tự nhiên $x,y$ thỏa mãn đẳng thức
\begin{multicols}{2}
\begin{enumerate}[a,]
    \item $(2 x+1)(y-3)=10.$
    \item $(3 x-2)(2 y-3)=1.$
    \item $x-3=y(x+2).$
    \item $x+6=y(x-1).$
\end{enumerate}
\end{multicols}
\loigiai{\begin{enumerate}[a,]
    \item  Do $2x+1>0$ nên $y-3>0.$ Ta lập bảng giá trị sau
    \begin{center}
        \begin{tabular}{c|c|c|c|c}
            $2x+1$ & $1$ & $2$ & $5$ & $10$ \\
            \hline
            $y-3$ & $10$ & $5$ & $2$ & $1$  \\
            \hline
            $x$ & $0$ & $\notin \mathbb{N}$ & $2$ & $\notin \mathbb{N}$ \\
            \hline
            $y$ & $13$ & $8$ & $5$ & $4$ 
        \end{tabular}
    \end{center}
    Như vậy, có $2$ cặp $\tron{x,y}$ tự nhiên thỏa mãn yêu cầu là $\tron{0,13}$ và $\tron{2,5}.$
    \item Ở ý này, ta chưa biết rõ dấu của $3x-2$ và $2y-3.$ Vì thế, bảng giá trị của ta như sau
    \begin{center}
        \begin{tabular}{c|c|c}
           $3x-2$  & $1$  & $-1$ \\
           \hline
           $2y-3$  & $1$  & $-1$ \\
           \hline
           $x$  & $1$  & $\notin \mathbb{N}$ \\
           \hline
           $y$  & $2$  & 
        \end{tabular}
    \end{center}%
    Như vậy, có duy nhất cặp số tự nhiên thỏa mãn yêu cầu là $\tron{x,y}=\tron{1,2}.$
    \item Phương trình đã cho tương đương với
    $$x-3=y(x+2)\Leftrightarrow x+2-5=y(x+2)\Leftrightarrow (x+2)(y-1)=-5.$$
    Ta có $y-1$ là ước của $5.$ Ta lập bảng giá trị sau
    \begin{center}
        \begin{tabular}{c|c|c|c|c}
            $1-y$ & $1$ & $5$ & $-5$ &$-1$  \\
            \hline
            $y$ & $0$ & $\notin\mathbb{N}$ & $6$ &$2$\\
            \hline
             $x$ & $3$ &  & $\notin \mathbb{N}$ &$\notin \mathbb{N}$
        \end{tabular}
    \end{center}
    Như vậy, có duy nhất một cặp số tự nhiên thỏa mãn yêu cầu là $\tron{x,y}=\tron{0,3}.$
    \item Phương trình đã cho tương đương với
    $$x+6=y(x-1)\Leftrightarrow x-1+7=y(x-1)\Leftrightarrow (y-1)\tron{x-1}=7.$$
    Ta có $y-1$ là ước của $7.$ Ta lập bảng giá trị sau
    \begin{center}
        \begin{tabular}{c|c|c|c|c}
            $1-y$ & $1$ & $7$ & $-7$ &$-1$  \\
            \hline
            $y$ & $0$ & $\notin\mathbb{N}$ & $8$ &$2$\\
            \hline
             $x$ & $\notin \mathbb{N}$ &  & $2$ &$8$
        \end{tabular}
    \end{center}
     Như vậy,  có $2$ cặp $\tron{x,y}$ tự nhiên thỏa mãn yêu cầu là $\tron{8,2}, \tron{2,8}.$
\end{enumerate}}

\end{bx}

\begin{bx}
Tìm tất cả các số tự nhiên $x,y$ thỏa mãn
\begin{multicols}{2}
\begin{enumerate}[a,]
    \item $xy-6x-5y=7.$
    \item $2xy-3x+5y=39.$
\end{enumerate}
\end{multicols}
\loigiai{\begin{enumerate}[a,]
    \item Phương trình đã cho tương đương với
    $$x(y-6)-5y=7\Leftrightarrow x(y-6)-5(y-6)=37\Leftrightarrow (x-5)(y-6)=37.$$
    Ta có $y-6$ là ước của $37.$ Ta lập bảng giá trị sau
    \begin{center}
        \begin{tabular}{c|c|c|c|c}
            $y-6$ & $-37$ & $-1$ & $1$ & $37$ \\
            \hline
             $y$ & $\notin\mathbb{N}$ & $5$ & $7$ & $43$ \\
            \hline
             $x$ && $\notin\mathbb{N}$ & $42$ & $6$
        \end{tabular}
    \end{center}
    Như vậy, có $2$ cặp $\tron{x,y}$ tự nhiên thỏa mãn yêu cầu là $\tron{42,7}, \tron{6,43}.$
     \item Phương trình đã cho tương đương với
     \begin{align*}
     x\tron{2y-3}+5y=39&\Leftrightarrow 2x\tron{2y-3}+10y=78\\&\Leftrightarrow 2x\tron{2y-3}+5\tron{2y-3}=63\\&
         \Leftrightarrow \tron{2x+5}\tron{2y-3}=63.
     \end{align*}
    Ta có $2y-3$ là ước của $63.$ Ta lập bảng giá trị sau
    \begin{center}
        \begin{tabular}{c|c|c|c|c|c|c|c|c|c|c|c|c}
            $2y-3$ & $-63$ & $-21$ & $-9$ & $-7$ & $-3$ &$-1$ & $1$ & $3$ & $7$ & $9$ & $21$ &$63$\\
            \hline
             $y$ & $\notin\mathbb{N}$ & $\notin\mathbb{N}$ & $\notin\mathbb{N}$ & $\notin\mathbb{N}$ & $0$ &$1$ & $2$ & $3$ & $5$ & $6$ & $12$ &$33$\\
            \hline
             $x$ & &  &  & & $\notin\mathbb{N}$ &$\notin\mathbb{N}$ & $29$ & $8$ & $2$ & $1$ & $\notin\mathbb{N}$ &$\notin\mathbb{N}$\\
        \end{tabular}
    \end{center}
    Như vậy, có $4$ cặp $\tron{x,y}$ tự nhiên thỏa mãn yêu cầu là $\tron{1,6}, \tron{2,5},\tron{8,3}$ và $\tron{29,2}.$
\end{enumerate}}
\end{bx}


\begin{bx}
Tìm tất cả các số nguyên dương $x,y$ thỏa mãn
\begin{multicols}{2}
\begin{enumerate}[a,]
    \item $\dfrac{2}{x}+\dfrac{3}{y}=\dfrac{7}{2}.$
    \item $\dfrac{7}{x+1}-\dfrac{2}{2y-1}=-\dfrac{1}{4}.$
\end{enumerate}
\end{multicols}
\loigiai{
\begin{enumerate}[a,]
    \item Nhân hai vế phương trình đã cho với $2xy,$ ta được
    $$4y+6x=7xy\Leftrightarrow y(7x-4)=6x.$$
    Từ đây, ta lần lượt chỉ ra
    $$\tron{7x-4}\mid6x\Rightarrow\tron{7x-4}\mid42x=6(7x-4)+24\Rightarrow\tron{7x-4}\mid 24.$$
    Ta dễ dàng nhận thấy $7x-4\ge0$. Ta có bảng giá trị sau
\begin{center}
   \begin{tabular}{c|c|c|c|c|c|c|c|c}
        $7x-4$ & $1$ & $2$ & $3$ & $4$ & $6$ & $8$ & $12$ & $24$ \\
        \hline
        $x$ & $\notin\mathbb{N}$ & $\notin\mathbb{N}$ & $1$ & $\notin\mathbb{N}$ & $\notin\mathbb{N}$ & $\notin\mathbb{N}$ & $\notin\mathbb{N}$ & $4$\\
        \hline
        $y$ &  &  & $2$ &  &  &  &  & $1$   
    \end{tabular}
\end{center}
Như vây, có $2$ cặp $(x,y)$ nguyên dương thỏa yêu cầu là $(1,2)$ và $(4,1).$
\item Đặt $x+1=u,2y-1=v.$ Phương trình đã cho trở thành
$$\dfrac{7}{u}-\dfrac{2}{v}=-\dfrac{1}{4}\Leftrightarrow 28v-8u=-uv\Leftrightarrow 8u=v\tron{u+28}.$$
Các bước còn lại làm tương tự như ý trên. Kết quả, có hai cặp $(x,y)$ thỏa yêu cầu là $(3,1)$ và $(195,4).$
\end{enumerate}
}
\end{bx}

\begin{bx}
Tìm tất cả các số nguyên $n$ sao cho
\begin{enumerate}[a,]
    \item $4n+11$ chia hết cho $3n+2.$
    \item $n^3+n^2+1$ chia hết cho $n+2.$
    \item $2n^3+8n+1$ chia hết cho $n^2+2.$
\end{enumerate}
\loigiai{
\begin{enumerate}[a,]
    \item Từ giả thiết, ta lần lượt suy ra
    \begin{align*}
        (3n+2)\mid(4n+11)
        &\Rightarrow (3n+2)\mid 3(4n+11)
        \\&\Rightarrow (3n+2)\mid \tron{4(3n+2)+25}
        \\&\Rightarrow (3n+2)\mid 25.
    \end{align*}
    Theo đó, $3n+2$ là ước nguyên của $25.$ Thử trực tiếp, ta tìm được $n=1,n=-1,n=-9.$
    \item Từ giả thiết, ta lần lượt suy ra
    \begin{align*}
        (n+2)\mid\tron{n^3+n^2+1}
        \Rightarrow (n+2)\mid \tron{(n+2)\left(n^2-n+2\right)-3}
        \Rightarrow (n+2)\mid 3.
    \end{align*}    
    Theo đó, $n+2$ là ước nguyên của $3.$ Thử trực tiếp, ta tìm được $n=1,n=-1,n=-3,n=-5.$
    \item Từ giả thiết, ta lần lượt suy ra
    \begin{align*}
        \tron{n^2+2}\mid\tron{2n^3+8n+1}
        &\Rightarrow        \tron{n^2+2}\mid\tron{2n\left(n^2+2\right)+4n+1}\\&
        \Rightarrow \tron{n^2+2}\mid\tron{4n+1}
        \\&\Rightarrow \tron{n^2+2}\mid\tron{4n-1}\tron{4n+1}  
        \\&\Rightarrow\tron{n^2+2}\mid\tron{16n^2+32-33} \\&
        \Rightarrow\tron{n^2+2}\mid33.    
    \end{align*}        
    Theo đó, $n^2+2$ là ước của $33,$ và $n\in\{-3;3;-1;1\}.$ \\
    Thử lại, ta thấy chỉ có $n=-3$ và $n=-1$ thỏa mãn.
\end{enumerate}}
\end{bx}

\begin{bx}
Tìm tất cả các số nguyên dương $n$ sao cho 
\begin{multicols}{2}
\begin{enumerate}[a,]
    \item Phân số $A=\dfrac{n^2-19}{n+5}$ chưa tối giản.
    \item Phân số $A=\dfrac{n^3+8n-2}{n+3}$ tối giản.
\end{enumerate}
\end{multicols}
\loigiai{
\begin{enumerate}[a,]
    \item Ta nhận thấy rằng
    $$A=\dfrac{n^2-19}{n+5}=\dfrac{n^2-25+6}{n+5}=\dfrac{n^2-25}{n+5}+\dfrac{6}{n+5}=n-5+\dfrac{6}{n+5}.$$
    Như vậy, $A$ chưa tối giản khi và chỉ khi $(6,n+5)\ne 1.$ Theo đó, $n+5$ chia hết cho $2$ hoặc $3.$
    \begin{itemize}
        \item Số dư của $n+5$ khi chia cho $6$ phải là một trong ba số $0,2,4,$ do $n+5$ chia hết cho $2.$ Ta suy ra số dư của $n$ khi chia cho $6$ là $1,3$ hoặc $5.$
        \item Số dư của $n+5$ khi chia cho $6$ phải là một trong hai số $0,3,$ do $n+5$ chia hết cho $3.$ Ta suy ra số dư của $n$ khi chia cho $6$ là $1$ hoặc $4.$       
    \end{itemize}
    Tổng kết lại, các số $n$ thỏa yêu cầu có dạng $$6k+1,\ 6k+3,\ 6k+4,\ 6k+5,$$ trong đó $k$ là số tự nhiên nào đó.
    \item Ta nhận thấy rằng 
    $$A= \dfrac{n^3+8n-2}{n+3} = \dfrac{\tron{n^3+27}+\tron{8n+24}-53}{n+3}=n^2-3n+17-\dfrac{53}{n+3}.$$
    Như vậy, $A$ tối giản khi và chỉ khi $\tron{53,n+3}= 1$. Theo đó, $n+3$ không chia hết cho $53$. Ta suy ra số dư của $n$ khi chia cho $53$ khác $53-3=50,$ và đây cũng là các số nguyên dương $n$ thỏa yêu cầu.
\end{enumerate}}
\end{bx}

\begin{bx}
Chứng minh rằng, số ước dương của một số nguyên dương $A$ gồm $n$ ước nguyên tố\footnote{Số nguyên tố là số chỉ có hai ước nguyên dương là 1 và chính nó.} và có phân tích tiêu chuẩn $A=p_1^{k_1}p_2^{k_2}\ldots p_n^{k_n}$ là $\left(k_1+1\right)\left(k_2+1\right)\ldots\left(k_n+1\right).$
\loigiai{
Một ước nguyên dương của số $A$ như đã cho sẽ có dạng
$${p_1}^{m_1}p_2^{m_2}\ldots p_n^{m_n},$$
trong đó, $m_1,m_2,\ldots,m_n$ lần lượt là các số nguyên dương thỏa mãn $m_i\le n_i,i=\overline{1,n}.$\\
Có tất cả $k_1+1$ cách chọn giá trị cho $m_1,$ do $m_1\in \{0;1;2;\ldots;k_1\}.$ Bằng lập luận tương tự, ta chỉ ra số cách chọn một bộ $\tron{m_1,m_2,\ldots,m_n}$ là
$$\tron{k_1+1}\tron{k_2+1}\ldots\tron{k_n+1}.$$
Đây cũng chính là số ước nguyên dương của $A.$
}
\end{bx}

\begin{bx}
Chứng minh số nguyên dương $a$ có lẻ ước khi và chỉ khi $a$ là số chính phương \footnote{Số chính phương là bình phương của một số tự nhiên.}.
\loigiai{
Giả sử tồn tại số nguyên dương $a$ gồm $n$ ước nguyên tố phân biệt thỏa mãn yêu cầu bài toán. Ta đặt
	$$a=p_1^{k_1}p_2^{k_2}\ldots p_n^{k_n}.$$
Theo tính chất đã biết, số ước nguyên dương của $A$ chính là
    $$\left(k_1+1\right)\left(k_2+1\right)\ldots\left(k_n+1\right).$$
Tích bên trên là số lẻ, chứng tỏ các số $k_1,k_2,\ldots,k_n$ đều chẵn. Lần lượt đặt $$k_1=2m_1,k_2=2m_2,\ldots,k_n=2m_n.$$
Các phép đặt như vậy sẽ cho ta
\begin{align*}
    a=p_1^{k_1}p_2^{k_2}\ldots p_n^{k_n}
    =p_1^{2m_1}p_2^{2m_2}\ldots p_n^{2m_n}
    =\left(p_1^{m_1}p_2^{m_2}\ldots p_n^{m_n}\right)^2.
\end{align*}
Chọn $x=p_1^{m_1}p_2^{m_2}\ldots p_n^{m_n},$ và bài toán được chứng minh.
}

\end{bx}

\begin{bx}
	Tìm số tự nhiên nhỏ nhất có
	\begin{multicols}{2}
		\begin{enumerate}[a,]
			\item $9$ ước số.
			\item $12$ ước số.
		\end{enumerate}
	\end{multicols}
	\loigiai{
		\begin{enumerate}[a,]
			\item Giả sử tồn tại số nguyên dương $A$ gồm $n$ ước nguyên tố phân biệt thỏa mãn yêu cầu bài toán. Ta đặt
			$$A=p_1^{k_1}p_2^{k_2}\ldots p_n^{k_n}.$$
			Không mất tổng quát, ta giả sử $k_1\ge k_2\ge \ldots\ge k_n.$ Tổng số ước nguyên dương của $A$ là $9,$ vậy nên
			$$\left(k_1+1\right)\left(k_2+1\right)\ldots\left(k_n+1\right)=9.$$
			Có duy nhất một cách phân tích $9$ thành các thừa số lớn hơn $1,$ đó là $9=3\cdot3.$ Dựa vào đây, ta chia bài toán làm hai trường hợp.
			\begin{itemize}
			    \item \chu{Trường hợp 1.} $k_1+1=k_2+1=3,k_3+1=\ldots=k_n+1=1.$ \\
			    Trường hợp này cho ta $k_1=k_2=2,k_3=k_4=\ldots=k_n=0,$ và ta có
			    $$A=p_1^2p_2^2.$$
			    Do tính nhỏ nhất của $A,$ ta chọn $p_1=2,p_2=3$ (hoặc $p_1=3,p_2=2$). Lúc này, $A=36.$
			    \item \chu{Trường hợp 2.} $k_1+1=9,k_2+1=k_3+1=\ldots=k_n+1=1.$ \\
			    Trường hợp này cho ta $k_1=8,k_3=k_4=\ldots=k_n=0,$ và ta có
			    $$A=p_1^8.$$
			    Do tính nhỏ nhất của $A,$ ta chọn $p_1=2$. Lúc này, $A=256.$
			\end{itemize}
			Dựa vào so sánh $36<256,$ ta tìm ra $A=36.$ Đây chính là kết quả bài toán.
			\item Giả sử tồn tại số nguyên dương $A$ gồm $n$ ước nguyên tố phân biệt thỏa mãn yêu cầu bài toán. Ta đặt
			$$A=p_1^{k_1}p_2^{k_2}\ldots p_n^{k_n}.$$
			Không mất tổng quát, ta giả sử $k_1\ge k_2\ge \ldots\ge k_n.$ 
			Bằng lập luận tương tự như câu \circEX{\textcolor{black}{1}}, ta chia bài toán thành 4 trường hợp
			\begin{itemize}
			    \item \chu{Trường hợp 1.} $A=p_1^2p_2p_3.$ \\
			    Do tính nhỏ nhất của $A$ nên $p_1,p_2,p_3$ chỉ nhận các giá trị $2,3,5.$ Thử lần lượt với $p_1=2,3,5,$ còn $p_2p_3$ là tích hai giá trị còn lại, ta thu được giá trị $A$ nhỏ nhất là $A=2^2\cdot3\cdot5=60.$
			    \item \chu{Trường hợp 2.} $A=p_1^5p_2.$ Lập luận tương tự, ta được $A=2^5\cdot3=96.$
			    \item \chu{Trường hợp 3.} $A=p_1^3p_2^2.$ Lập luận tương tự, ta được $A=2^3\cdot3^2=72.$	\item \chu{Trường hợp 4.} $A=p_1^{11}.$ Lập luận tương tự, ta được $A=2^{11}\cdot3^2=2048.$
	        \end{itemize}
            So sánh kết quả thu được ở các trường hợp, ta được $A=60.$
    \end{enumerate}   
    Tổng kết lại, đáp số bài toán là $A=60.$
	}	
\end{bx}
\begin{bx}
Biết rằng $n$ là số nguyên dương nhận $2$ và $3$ là các ước nguyên tố thỏa mãn $2n$ có $8$ ước dương, $3n$ có $12$ ước dương. Hãy xác định số ước dương của $12n$.
\loigiai{
Đặt $n = 2^{x}3^{y}p_1^{k_1}p_2^{k_2}\ldots p_n^{k_n},$ trong đó $x,y$ là $2$ số tự nhiên và $k_1, k_2,\ldots, k_n$ là các số nguyên dương. Xét số $$3n=  2^{x}3^{y+1}p_1^{k_1}p_2^{k_2}\ldots p_n^{k_n}.$$ 
Do $3n=  2^{x}3^{y+1}p_1^{k_1}p_2^{k_2}\ldots p_n^{k_n}$ có $12$ ước dương, ta có
$$\tron{x+1}\tron{y+2}\tron{k_1+1}\ldots\tron{k_n+1}=12.$$
Một cách tương tự, do $2n=  2^{x+1}3^{y}p_1^{k_1}p_2^{k_2}\ldots p_n^{k_n},$ ta có
$$\tron{x+2}\tron{y+1}\tron{k_1+1}\ldots\tron{k_n+1}=8.$$
Với chú ý rằng $8=2\cdot4=2\cdot2\cdot2$, ta chỉ ra $x+2 \in \left\{{2, 4}\right\},$ và đồng thời
\[\dfrac{\tron{x+2}\tron{y+1}}{\tron{x+1}\tron{y+2} } = \dfrac{8}{12} = \dfrac{2}{3}.\tag{*} \label{na5}\]
Ta xét các trường hợp sau đây.
\begin{itemize}
    \item \chu{Trường hợp 1.} Với $x+2 = 2$ hay là $x=0,$ thế trở lại (\ref{na5}), ta được
    $$\dfrac{2\tron{y+1}}{\tron{y+2}} = \dfrac{2}{3} \Leftrightarrow 3\tron{y+1} = y +2 \Leftrightarrow 2y = -1.$$
    Không tồn tại $y$ nguyên thỏa mãn trường hợp này.
    \item \chu{Trường hợp 2.} Với $x+2 = 4$ hay $x=2,$ thế trở lại (\ref{na5}), ta được
    $$\dfrac{4\tron{y+1}}{3\tron{y+2}} = \dfrac{2}{3} \Leftrightarrow 2\tron{y+1} = y +2 \Leftrightarrow y = 0.$$
    Do đó, $x=2$ và $y=0.$
    %em qua lấy code thôi, không sửa gì đâu
\end{itemize}
Kết luận, số $12n = 2^{x+2}3^{y+1}p_1^{k_1}p_2^{k_2}\ldots p_n^{k_n}$ có số ước dương là
$$\tron{x+3}\tron{y+2}\tron{k_1+1}\ldots\tron{k_n+1}= 20.$$
Bài toán được giải quyết.}
\end{bx}

\begin{luuy}
Bài toán trên vẫn có cách giải quyết tương tự trong trường hợp $2$ và $3$ chưa chắc là ước nguyên tố của $n.$
\end{luuy}

\subsection{Ước chung lớn nhất và bội chung nhỏ nhất}
\begin{dx}
Ước chung lớn nhất của hai hay nhiều số nguyên là phần tử tự nhiên lớn nhất trong tập hợp các ước chung của các số đó.
\end{dx} 
Chẳng hạn, ta biết ước chung lớn nhất của $25$ và $15$ là $5,$ bởi vì
    \begin{itemize}
        \item Tập ước của $25$ là $\{-25;-5;-1;1;5;25\}.$
        \item Tập ước của $15$ là $\{-15;-5;-3;-1;1;3;5;15\}.$   
        \item Tập ước chung của $25$ và $15$ là $\{-5;-1;1;5\},$ và $5$ là phần tử lớn nhất trong này.
    \end{itemize}
Ta kí hiệu $(a,b,c,\ldots)$ thay cho việc gọi ước chung lớn nhất của các số $a,b,c,\ldots.$ Ngoài ra ở một số sách nước ngoài, chúng ta còn bắt gặp kí hiệu là $\gcd(a,b)$, nó bắt nguồn từ thuật ngữ trong tiếng anh "greatest common divisor $-$ gcd" (ước chung lớn nhất). Trong ví dụ trên, ta có $(15,25)=5.$ Về cách xác định ước chung lớn nhất, chúng ta tiến hành theo ba bước.
\begin{light}
\chu{Ba bước xác định ước chung lớn nhất}
\begin{enumerate}
    \item Phân tích mỗi số ra thừa số nguyên tố.
    \item Chọn ra các thừa số nguyên tố \chu{chung}.
    \item Lập tích các thừa số đã chọn, mỗi thừa số lấy với số mũ \chu{nhỏ nhất} của nó. Tích đó là ước chung lớn nhất cần tìm.
\end{enumerate}
\end{light}
Chẳng hạn, với yêu cầu tính $(36,84,168),$ trước hết ta phân tích ba số trên ra thừa số nguyên tố
$$36 =2^{2} \cdot 3^{2},\qquad 84 =2^{2} \cdot 3 \cdot 7,\qquad 168 =2^{3} \cdot 3 \cdot 7.$$
Ta chọn ra các thừa số chung, đó là $2$ và $3$. Số mũ nhỏ nhất của $2$ là $2,$ còn của $3$ là $1.$ Như vậy
$$(36,84,168)=2^{2} \cdot 3=12.$$
Ngoài ra, chúng ta còn có một cách xác định ước chung lớn nhất khác, đó là thuật toán $Euclid.$ 

Để tìm $\left( m,n \right)$ khi $m$  không chia hết cho $n$ ta thực hiện theo các bước
    \begin{itemize}
        \item $m=n{{q}_{1}}+{{r}_{1}},1\le {{r}_{1}}<n$ 
        \item $n={{r}_{1}}{{q}_{2}}+{{r}_{2}},1\le {{r}_{2}}<{{r}_{1}}$ \\
        $\ldots$
        \item ${{r}_{k-2}}={{r}_{k-1}}{{q}_{k}}+{{r}_{k}},1\le {{r}_{k}}<{{r}_{k-1}}$ 
        \item ${{r}_{k-1}}={{r}_{k}}{{q}_{k+1}}+{{r}_{k+1}},{{r}_{k+1}}=0$
    \end{itemize}
Ta thấy rằng, chuỗi đẳng thức này là hữu hạn bởi vì $n>{{r}_{1}}>{{r}_{2}}>...>{{r}_{k}}$. \\
Như vậy ${{r}_{k}}$ là số dư cuối cùng khác 0 trong thuật toán Euclid nên $\left( m,n \right)={{r}_{k}}$.
Hiểu đơn giản giải thuật này sẽ được mô tả ngắn gọn bằng công thức 
\[\left( {a,b} \right) = \left( {b,a - b\cdot \bigg\lfloor {\frac{a}{b}} \bigg\rfloor }. \right)\]

\begin{dx}
Bội chung nhỏ nhất của hai hay nhiều số nguyên là phần tử nguyên dương nhỏ nhất trong tập các bội chung của các số đó.
\end{dx} 
Chẳng hạn, ta biết bội chung lớn nhất của $25$ và $15$ là $75,$ bởi vì
    \begin{itemize}
        \item Tập bội của $25$ là $\{...;-100;-75;-50;-25;0;25;50;75;100;...\}.$
        \item Tập bội của $15$ là $\{...;-90;-75;-60;-45;-30;-15;0;15;30;45;60;75;90;...\}.$   
        \item Tập bội chung của $25$ và $15$ là $\{\ldots;-75;0;75;\ldots\},$ và $75$ là phần tử nguyên dương trong này.
    \end{itemize}
Ta kí hiệu $[a,b,c,\ldots]$ thay cho việc gọi bội chung lớn nhất của các số $a,b,c,\ldots.$ Ngoài ra ở một số sách nước ngoài, chúng ta còn bắt gặp kí hiệu là $\text{lcm} (a,b)$, nó bắt nguồn từ thuật ngữ trong tiếng anh "least common multiple $-$ lcm" (bội chung nhỏ nhất). Trong ví dụ trên, ta có $[15,25]=75.$ Về cách xác định bội chung lớn nhất, chúng ta tiến hành theo ba bước.
\begin{light}
\chu{Ba bước xác định bội chung lớn nhất}
\begin{enumerate}
    \item Phân tích mỗi số ra thừa số nguyên tố,
    \item Chọn ra các thừa số nguyên tố \chu{chung}.
    \item Lập tích các thừa số đã chọn, mỗi thừa số lấy với số mũ \chu{lớn nhất} của nó. Tích đó là ước chung lớn nhất cần tìm.
\end{enumerate}
\end{light}
Chẳng hạn, với yêu cầu tính $[8,18,30]$, trước hết ta phân tích ba số trên ra thừa số nguyên tố
$$8 =2^{3},\qquad18 =2 \cdot 3^{2},\qquad30 =2 \cdot 3 \cdot 5.$$
Chọn ra các thừa số nguyên tố chung và riêng, đó là $2,3,5$. Số mū lớn nhất của $2$ là $3$, số mũ lớn nhất của $3$ là $2,$ trong khi số mũ lớn nhất của $5$ là $1.$ Như vậy
$$[8,18,30]=2^{3} \cdot 3^{2} \cdot 5=360.$$
Đi cùng với những khái niệm và cách xác định liên quan tới ước chung lớn nhất và bội chung nhỏ nhất là một vài tính chất thú vị.

\begin{light}
\chu{Các tính chất về chia hết}
\begin{enumerate}
    \item Với $a,b,m$ là các số nguyên, nếu $ab$ chia hết cho $m$ và $(b,m)=1$ thì $a$ chia hết cho $m.$ Hệ quả là
    \begin{itemize}
        \item Nếu $a$ chia hết cho $m$ thì $(a,m)=m.$
        \item Nếu $a^n$ chia hết cho $p$ (với $n$ là số nguyên dương lớn hơn $1$ và $p$ là số nguyên tố) thì $a$ chia hết cho $p.$
    \end{itemize}
    \item Với $a,m,n$ là các số nguyên, nếu $a$ chia hết cho $m$ và $n$ thì $a$ chia hết cho $(m,n).$ 
    \item Với $a,b,m$ là các số nguyên, nếu $a$ và $b$ cùng chia hết cho $m$ thì $(a,b)$ chia hết cho $m.$    
    \item Với $a,m,n$ là các số nguyên, nếu $a$ chia hết cho cả $m$ và $n$ thì $a$ chia hết cho $[m,n]$.  
\end{enumerate}
\chu{Các tính chất khác}
\begin{enumerate}
    \item Với mọi số nguyên $a,b$ khác $0,$ ta có $(a,1)=1$ và $[a,1]=a.$
    \item Với mọi số nguyên $a,b$ khác $0,$ ta có $(a,b)[a,b]=ab.$    
    \item Với mọi số nguyên $a,b$ khác $0$ và số nguyên $k,$ ta có $(a,b)=k(a,b)$ và $[a,b]=k[a,b].$
    \item Với mọi số nguyên $a,b$ khác $0$ và số nguyên dương $n,$ ta có \[\tron{a^n,b^n}=(a,b)^n,\qquad [a^n,b^n]=[a,b]^n.\]   
\end{enumerate}
\end{light}
\subsubsection*{Ví dụ minh họa}
 
\begin{bx}
Với số tự nhiên $n$ bất kì, hãy chứng minh rằng\\
\begin{minipage}{0.45\textwidth}
\begin{enumerate}[a,]
    \item $(2n+1,3n+2)=1.$
    \item $(4n+1,6n+2)=1.$
\end{enumerate}
\end{minipage}
\begin{minipage}{0.55\textwidth}
\begin{enumerate}
    \setcounter{enumi}{2}
    \item[c,] $\left(n+3,n^2-2n-14\right)=1.$
    \item[d,] $\left(n^2+4n+5,n^3+5n^2-13\right)=1.$
\end{enumerate}
\end{minipage}
\loigiai{
\begin{enumerate}[a,]
    \item Đặt $\tron{2n + 1, 3n + 2}= d$, ta có
    $$\left\{\begin{aligned}
         &d\mid (2n + 1) \\
         &d\mid (3n + 2 )  
    \end{aligned}\right.
    \Rightarrow \left\{\begin{aligned}
         &d\mid 3\tron{2n + 1} \\
         &d\mid 2\tron{3n + 2}   
    \end{aligned}\right.
    \Rightarrow d\mid 2\tron{3n + 2} - 3\tron{2n + 1} = 1.$$
Ta thu được $(2n+1,3n+2)=1.$
    \item Đặt $\tron{4n + 1, 6n + 2}= d$, ta có
    $$\left\{\begin{aligned}
         &d\mid (4n + 1) \\
         &d\mid (6n + 2)   
    \end{aligned}\right.
    \Rightarrow \left\{\begin{aligned}
         &d\mid 3\tron{4n + 1} \\
         &d\mid 2\tron{6n + 2}   
    \end{aligned}\right.
    \Rightarrow d\mid 2\tron{6n + 2} - 3\tron{4n + 1} = 1.
$$
Ta thu được $\tron{4n + 1, 6n + 2}=1.$
    \item Đặt $\tron{n + 3, n^2 - 2n - 14}= d$, ta có
    \begin{align*}
        \left\{\begin{aligned}
         &d\mid (n + 3) \\
         &d\mid \tron{n^2 - 2n - 14}   
    \end{aligned}\right.
    &\Rightarrow \left\{\begin{aligned}
         &d\mid \tron{n^2 + 3n} \\
         &d\mid \tron{n^2 - 2n - 14}   
    \end{aligned}\right.
    \\&\Rightarrow \left\{\begin{aligned}
         &d\mid (n + 3)\\
         &d\mid \tron{n^2 + 3n} -\tron{n^2 - 2n - 14 }  
    \end{aligned}\right.
    \\&\Rightarrow
    \left\{\begin{aligned}
         &d\mid (n + 3) \\
         &d\mid (5n + 14)   
    \end{aligned}\right.
    \\&\Rightarrow d\mid 5\tron{n+3} - \tron{5n + 14} = 1.
    \end{align*}
Ta thu được $\tron{n + 3, n^2 - 2n - 14}=1.$
    \item Đặt $\left(n^2+4n+5,n^3+5n^2-13\right)=d$, ta có
    \begin{align*}
        \left\{\begin{aligned}
         &d\mid \tron{n^2+4n+5} \\
         &d\mid \tron{n^3+5n^2-13}   
    \end{aligned}\right.
    &\Rightarrow \left\{\begin{aligned}
         &d\mid \tron{n^3 + 4n^2+5n} \\
         &d\mid \tron{n^3+5n^2-13}   
    \end{aligned}\right.
    \\&\Rightarrow \left\{\begin{aligned}
         &d\mid \tron{n^2+4n+5}\\
         &d\mid \tron{n^2 - 5n -13} 
    \end{aligned}\right.
    \\&\Rightarrow
    \left\{\begin{aligned}
         &d\mid \tron{n^2+4n+5} \\
         &d\mid (9n + 18)  
    \end{aligned}\right.
    \\&\Rightarrow \left\{\begin{aligned}
         &d\mid \tron{9n^2+36n+45} \\
         &d\mid \tron{9n^2 + 18n}   
    \end{aligned}\right.
    \\&\Rightarrow \left\{\begin{aligned}
         &d\mid (18n+45)\\
         &d\mid (9n + 18) 
    \end{aligned}\right.
    \\&\Rightarrow \left\{\begin{aligned}
         &d\mid (18n+45)\\
         &d\mid (18n + 36) 
    \end{aligned}\right.
    \\&\Rightarrow d \mid (18n+45)-(18n+36)=9.
    \end{align*}
Tới đây, ta xét các trường hợp sau.    
\begin{itemize}
    \item \chu{Trường hợp 1.} Với $d=3$ hoặc $d=9,$ ta có $n^2+4n+5$ chia hết cho $3.$
    \begin{itemize}
        \item Nếu $n+2$ chia cho $3$ dư $1$ hoặc dư $2,$ ta đặt $n=3k\pm 1.$ Ta có
        $$n^2+4n+5=(n+2)^2+1=(3k\pm 1)^2+1=9k^2\pm 6k+2.$$
        Số kể trên chia cho $3$ dư $2,$ mâu thuẫn.
        \item Nếu $n+2$ chia hết cho $3,$ ta có $n^2+4n+5=(n+2)^2+1$ chia cho $3$ dư $1,$ mâu thuẫn.
    \end{itemize}
    \item \chu{Trường hợp 2.} Với $d=1,$ bài toán được chứng minh.
\end{itemize}
\end{enumerate}}
\end{bx}

%bài 67b - Triết và chế thêm 1 bài tương tự theo kiểu xa+yb
\begin{bx}
Cho hai số $a,b$ nguyên tố cùng nhau. Chứng minh rằng
\begin{multicols}{2}
\begin{enumerate}[a,]
    \item $\tron{11a+2b, 18a+5b}$ bằng 1 hoặc 19.
    \item $\tron{25a+7b, 17a +6b}$ bằng 1 hoặc 31.
\end{enumerate}
\end{multicols}
\loigiai{
\begin{enumerate}[a,]
    \item Ta đặt $\tron{11a+2b, 18a+5b}=d$. Ta có
    $$\heva{&d\mid (11a+2b) \\ &d\mid (18a+5b)}\Rightarrow
    \heva{&d\mid 5(11a+2b)\\ &d\mid 2(18a+5b)}\Rightarrow
    d\mid 5(11a+2b)-2(18a+5b)\Rightarrow d\mid 19a.$$
    Ta cũng có thể chỉ ra rằng
    $$\heva{&d\mid (11a+2b) \\ &d\mid (18a+5b)}\Rightarrow
    \heva{&d\mid 18(11a+2b)\\ &d\mid 11(18a+5b)}\Rightarrow
    d\mid 18(11a+2b)-11(18a+5b)\Rightarrow d\mid 19b.$$    
    Hai lập luận kể trên cho ta biết
    $$\heva{&d\mid 19a \\ &d\mid 19b}\Rightarrow d\mid 19(a,b)\Rightarrow d\mid 19.$$
    Như vậy, bài toán đã cho được chứng minh.
    \item Ta đặt $\tron{25a+7b, 17a +6b}=d$. Ta có
    $$\heva{&d \mid (25a+7b)\\ &d\mid (17a +6b)}\Rightarrow \heva{&d\mid 6\tron{25a+7b}\\ &d\mid 7(17a +5b)}\Rightarrow d\mid 6\tron{25a+7b}-7(17a +6b) \Rightarrow d\mid 31a.$$
    Ta cũng có thể chỉ ra rằng
    $$\heva{&d \mid (25a+7b)\\ &d\mid (17a +6b)}\Rightarrow \heva{&d\mid 17\tron{25a+7b}\\ &d\mid 25(17a +5b)}\Rightarrow d\mid 17\tron{25a+7b}-25(17a +6b) \Rightarrow d\mid 31b.$$
   Hai lập luận kể trên cho ta biết
    $$\heva{&d\mid 31a \\ &d\mid 31b}\Rightarrow d\mid 31(a,b)\Rightarrow d\mid 31.$$
    Như vậy, bài toán đã cho được chứng minh.
\end{enumerate}}
\end{bx}

\begin{bx}
Với hai số tự nhiên $a,b$ bất kì thỏa mãn $(a,b)=1,$ hãy chứng minh rằng
\begin{multicols}{2}
\begin{enumerate}[a,]
    \item $(a+b,ab)=1.$
    \item $\left(a^2+b^2,a+b\right)\in \left\{{1,2} \right\}.$
    \item $\left(a^5-b^5,a+b\right) \in \left\{{1,2} \right\}.$
    \item $\left(2^a-1,2^b-1\right)=1.$
\end{enumerate}
\end{multicols}
\loigiai{
\begin{enumerate}[a,]
    \item Ta đặt $\left(a+b,ab\right)=d.$ Phép đặt này cho ta
    $$\heva{&d\mid (a+b)\\&d\mid ab}
    \Rightarrow \heva{&d\mid a(a+b)\\&d\mid b(a+b)\\&d\mid ab}
    \Rightarrow\heva{&d\mid a^2 \\ &d\mid b^2}\Rightarrow d\mid \tron{a^2,b^2}
    \Rightarrow d\mid (a,b)^2\Rightarrow 
    d\mid 1\Rightarrow d=1.$$
    Bài toán được chứng minh.
    \item  Ta đặt $\left(a^2+b^2,a+b\right)=d.$ Phép đặt này cho ta
    \begin{align*}
    \heva{&d\mid \tron{a^2+b^2}\\ &d\mid \tron{a+b}}
    &\Rightarrow
    \heva{&d\mid \tron{a^2+b^2}\\ &d\mid \tron{a+b}\tron{a-b}}    
    \\&\Rightarrow \left\{\begin{aligned}
         d\mid \tron{a^2 + b^2}\\
         d\mid \tron{a^2 - b^2}
    \end{aligned}\right.
    \\&\Rightarrow \left\{\begin{aligned}
         d\mid 2a^2\\
         d\mid 2b^2
    \end{aligned}\right.\\&
    \Rightarrow d\mid \tron{2a^2,2b^2}
    \\&\Rightarrow d\mid 2(a,b)^2\\&\Rightarrow
    d\in\{1;2\}.
    \end{align*}
    Bài toán được chứng minh.
   \item  Đặt $\left(a^5-b^5,a+b\right)=d,$ ta có $d\mid \tron{a^5-b^5}$ và $d\mid (a+b)$. Ngoài ra, ta còn phân tích được
    $$a^5+b^5 = \tron{a+b}\tron{a^4 - a^3b +a^2b^2-ab^3+b^4} \Rightarrow d\mid a^5+b^5.$$
    
    Phân tích trên chỉ ra cho ta
    $$\left\{\begin{aligned}
         d\mid \tron{a^5+b^5}\\
         d\mid \tron{a^5-b^5}
    \end{aligned}\right.
    \Rightarrow \left\{\begin{aligned}
         d\mid 2a^5\\
         d\mid 2b^5
    \end{aligned}\right.
    \Rightarrow d\mid \tron{2a^5,2b^5}
    \Rightarrow d\mid 2(a,b)^5
    \Rightarrow d\mid 2
    \Rightarrow d\in\{1;2\}.$$
    Bài toán được chứng minh.
    \item Đặt $\tron{2^a -1, 2^b -1} = d,$ ta có $d\mid \tron{2^a-1}$ và $d\mid \tron{2^b-1}.$ Ngoài ra, ta còn chứng minh được
    $$2^b-1=2^{b-a}\tron{2^a-1}+2^{b-a}-1.$$
    Như vậy, $2^{b-a}-1$ cũng chia hết cho $d.$ Cách hạ số mũ số bị chia thành hiệu như vậy chính là thuật toán $Euclid.$ Sau hữu hạn các bước hạ, ta chỉ ra
    $$d\mid \tron{2^{(a,b)}-1}.$$
    Do $(a,b)=1,$ ta thu được $d=1.$ Bài toán được chứng minh.
\end{enumerate}
}
\end{bx}

%bài 64 Triết
\begin{bx}
Chứng minh nếu $a,b,c \in \mathbb{Z}$ đôi một nguyên tố cùng nhau thì \[\tron{ab+bc+ca, abc} =1.\]
\loigiai{Dựa vào chú ý $ab+bc+ca=(b+c)a+bc,$ ta nhận thấy rằng
$$(ab+bc+ca,a)=(a,bc).$$
Do $(a, b)=(a, c)=1$  nên $(a, b c)=1 .$
Một cách tương tự, ta chỉ ra 
\[(a b+b c+c a,b)=(a b+b c+c a, c)=1.\]
Như vậy, $(a b+b c+c a, a b c)=1.$ Bài toán được chứng minh.}
\end{bx}

%bài 71b và 72 của Triết, gộp thành 2 ý trong 1 bài
\begin{bx}
Hãy tính
\begin{multicols}{2}
\begin{enumerate}[a,]
    \item $[a,a+2].$
    \item $[a,a+1,a+2].$
\end{enumerate}
\end{multicols}
\loigiai{
\begin{enumerate}[a,]
    \item Ta nhận thấy rằng $\tron{a, a+2} = \heva{1 &\text{ nếu $a$ lẻ}\\ 2 &\text{ nếu $a$ chẵn}.}$\\
    Từ đây, theo tính chất đã biết, ta có 
    $$\vuong{a, a+2}=\dfrac{a\tron{a+2}}{\tron{a,a+2}}= \heva{ a\tron{a+2} &\text{ nếu $a$ lẻ}\\ \dfrac{a\tron{a+2}}{2} &\text{ nếu $a$ chẵn}.}$$
    \item Ta nhận thấy rằng $\tron{a\tron{a+2},a=1}= \tron{a\tron{a+1}+a, a+1}= \tron{a,a+1}= 1.$\\
    Áp dụng ý trên, ta xét các trường hợp dưới đây.
    \begin{itemize}
        \item\chu{Trường hợp 1.} Với $a$ lẻ , ta có $$\vuong{a,a+1,a+2}= \vuong{\vuong{a,a+2},a+1}=\vuong{a\tron{a+2}, a+1}=a\tron{a+1}\tron{a+2}.$$
            \item\chu{Trường hợp 2.} Với $a$ chẵn , ta có $$\vuong{a,a+1,a+2}= \vuong{\vuong{a,a+2},a+1}=\vuong{\dfrac{a\tron{a+2}}{2}, a+1}=\dfrac{a\tron{a+1}\tron{a+2}}{2}.$$
    \end{itemize}
\end{enumerate}
}
\end{bx}

\begin{bx}
Chứng minh rằng $[1,2, \ldots, 2 n]=[n+1, n+2, \ldots, 2 n]$.
\loigiai{Trong $k$ số nguyên liên tiếp, có một và chỉ một số chia hết cho $k$. Do đó, mỗi một số trong các số $$1, 2, 3, \ldots,2n-1, 2 n$$ 
là ước của ít nhất một số trong các số
$$n+1,n+2,n+3,\ldots,2n-1, 2n.$$
Kết quả trên dẫn ta đến điều phải chứng minh.}
\end{bx}


\begin{bx}
Tìm tất cả các số nguyên dương $a,b,c$ thỏa mãn đồng thời các điều kiện
$$(a,20)=b,\qquad (b,15)=c,\qquad (c,a)=5.$$
\nguon{Austrian Mathematical Olympiad Regional Competition 2015}
\loigiai{
Theo giả thiết $\tron{a,20}=b,$ ta suy ra $b\mid 20.$ Kết hợp với $\tron{b,15}=c$, ta có
$$c\mid \tron{20,15}\Rightarrow c\mid 5\Rightarrow c\in\left\{1,5\right\}.$$
Thế từng giá trị của $c$ vào giả thiết $\tron{c,a}=5$, ta nhận được $c=5.$ Nhận xét này cho ta
$$\heva{&b\mid20\\&\tron{b,15}=c}\Rightarrow b\in \left\{5,10,20\right\}.$$
Ta xét từng trường hợp sau.
\begin{enumerate}
    \item Với $b=5,$ thế vào giả thiết, ta thu được $\tron{a,20}=5,$ và $a=5k,$ trong đó $\tron{k,4}=1.$
    \item Với $b=10,$ thế vào giả thiết, ta thu được $\tron{a,20}=10,$ và $a=10k,$ trong đó $\tron{k,2}=1.$
    \item Với $b=20$ thế vào giả thiết, ta thu được $\tron{a,20}=5,$ và $a=20k$ trong đó $k$ nguyên dương tùy ý.
\end{enumerate}}
\end{bx}

%bài 75 Triết
\begin{bx}
Tìm các số nguyên dương $a, b$ thoả mãn đồng thời $(a, b)=15$ và $[a, b]=2835.$
\loigiai{
Từ giả thiết $(a,b)=15$, ta có thể đặt $a=15x, b=15y$ với $\left(x, y\right)=1$. Phép đặt này cho ta
$$[a, b]=\left[15x,15y\right]=15\left[x,y\right]=15xy.$$
Kết hợp biến đổi trên với giả thiết $[a,b]=2835,$ ta có
$$15xy=2835\Rightarrow xy=189=3^{3} \cdot7.$$
Vì $a, b$ có vai trò như nhau nên không mất tính tổng quát, ta giả sử $a \leq b$.\\
Do $\left(x,y\right)=1$ nên $x=1, y=189$ hoặc $x=27, y=7.$ Kiểm tra trực tiếp từng trường hợp, ta kết luận các cặp $(15,2835),(105,405),(405,105),(2835,15)$ là các cặp số thỏa yêu cầu bìa toán.}
\end{bx}

%bài 73 Triết
\begin{bx}
Cho $m, n$ là hai số nguyên dương thoả mãn $(m, n)+[m, n]=m+n$ và $m>n.$ Chứng minh rằng $m$ chia hết cho $n.$
\nguon{Saint Peterburg Mathematical Olympiad 2008}
\loigiai{
Với các số $m,n$ thỏa yêu cầu bài toán, ta có
$$
\heva{
&(m,n)+[m,n]=m+n\\
&(m,n)[m,n]=mn.
}$$
Vì thế, $\tron{m,n}$ và $\vuong{m+n}$ là nghiệm của phương trình ẩn $x$ $$x^2-\tron{m+n}x+mn=0.$$
Ta nhận thấy $m,n$ cũng là cặp nghiệm của phương trình trên. Với việc $m\ne n,$ lập luận trên cho ta
$$\left\{\tron{m,n}, \vuong{m,n}\right\}=\{m,n\}.$$
Dựa trên so sánh $[m,n]>(m,n),$ ta biết được rằng $m=[m,n]$ và $n=(m,n).$ \\Do $[m,n]$ chia hết cho $(m,n),$ bài toán đã cho được chứng minh.}
\end{bx}

%bài 76 Triết và chế thêm 1 ý kiểu như a^3+b^3 hoặc tương tự
\begin{bx}
Tìm các số tự nhiên $a,b$ thỏa mãn 
\begin{enumerate}[a,]
    \item $a^{2}+b^{2}=468$ và $(a, b)+[a, b]=42.$
    \item $a^3+b^3=1512$ và $(a, b)+[a, b]=42.$
\end{enumerate}
\loigiai{
\begin{enumerate}[a,]
\item Ta đặt $a=dx, b=dy$ với $\tron{x,y}=1$. Phép đặt này cho ta
$$\heva{d^2\tron{x^2+y^2}&=468\\
d\tron{1+xy}&=42
}\Rightarrow \dfrac{x^2+y^2}{\tron{1+xy}^2}=\dfrac{13}{7^2}\Rightarrow \dfrac{x^2+y^2}{13}=\dfrac{(xy+1)^2}{7^2}.$$
Ta tiếp tục đặt $k=\dfrac{x^2+y^2}{13}.$ Biến đổi trên kết hợp với phép đặt giúp ta chỉ ra
$$
x^2+y^2=13k^2,\qquad
1+xy=7k
$$
Do $d\cdot7k=d\tron{1+xy}=42,$ ta có $d$ là ước của $6.$ Không mất tính tổng quát, ta giả sử $x\le y.$ \\
Ta xét các trường hợp sau.
\begin{itemize}
    \item\chu{Trường hợp 1.} Với $k=1$, ta thay vào hệ phương trình trên và thu được
    $$\heva{
    x^2+y^2&=13\\
    1+xy&=7
    }\Leftrightarrow \heva{x+y&=5\\ xy&=6}
    \Leftrightarrow \heva{x=2\\y=3.}$$
    \item\chu{Trường hợp 2.} Với $k=2,3,6,$ bằng cách giải hệ như trên, ta không tìm được $x,y$ nguyên dương.
\end{itemize}
Vậy $(a,b)=(12,18)$ và  $(a,b)=(18,12)$ là các cặp số thỏa mãn yêu cầu của đề bài.
\item Ta đặt $a=dx, b=dy$ với $\tron{x,y}=1$. Phép đặt này cho ta
$$\heva{d^3\tron{x^3+y^3}&=1512\\
d\tron{1+xy}&=42
}\Rightarrow \dfrac{x^3+y^3}{\tron{1+xy}^3}=\dfrac{7}{7^3}\Rightarrow\dfrac{x^3+y^3}{7}=\dfrac{\tron{1+xy}^3}{7^3}.$$
Ta tiếp tục đặt $k=\dfrac{x^3+y^3}{7}.$ Biến đổi trên kết hợp với phép đặt giúp ta chỉ ra
$$x^3+y^3=7k^3,\qquad 1+xy=7k.$$
Do $d\cdot7k=d\tron{1+xy}=42,$ ta có $d$ là ước của $6.$ Không mất tính tổng quát, ta giả sử $x\le y.$ \\
Ta xét các trường hợp sau.
\begin{itemize}
    \item\chu{Trường hợp 1.} Với $k=1,2,6,$ thử trực tiếp, ta không tìm được $x,y$ thỏa yêu cầu.
   \item\chu{Trường hợp 2.} Với $k=3$, ta thay vào hệ phương trình và thu được
$\heva{x^3+y^3&=189\\1+xy&=21.}$\\
Hệ trên cho ta $xy=20.$ Ngoài ra 
$$x^3+y^3=\tron{x+y}^3-3xy\tron{x+y}=\tron{x+y}^3-60\tron{x+y}.$$
Giải phương trình
$\tron{x+y}^3-60\tron{x+y}=189$
trong điều kiện $x+y>0,$ ta nhận thấy $x+y=9$.\\ Kết hợp với $xy=20$, ta thu được
$x=4, y=5,$ và khi đó
$$a=2x=8,\quad b=2y=10.$$
\end{itemize}
Như vậy, tất cả các cặp số $(a,b)$ thỏa yêu cầu bài toán là $(8,10)$ và $(10,8).$
\end{enumerate}}


\end{bx}
%bài 77 Triết
\begin{bx}
Cho $a, b \in \mathbb{Z}$. Chứng minh rằng $(a+b,[a, b])=(a, b)$.
\loigiai{Ta đặt $a=dx, b=dy$ với $\tron{x,y}=1$. Phép đặt này cho ta $$(a+b,[a, b])=\left(dx+dx, dxy\right)=d\left(x+y, xy\right).$$
Dựa trên các lập luận $\tron{x+y,x}=\tron{y,x}=1$ và $\tron{x+y,y}=\tron{x,y}=1$, ta suy ra $\tron{x+y,xy}=1$. \\
Bài toán được chứng minh.}
\end{bx}

\subsection{Đồng dư thức}

\begin{dx}
Cho số nguyên dương $m,$ Hai số nguyên $a$ và $b$ được gọi là đồng dư theo modulo $m$ nếu chúng có cùng số dư khi chia cho $m.$ Điều này tương đương với hiệu $a-b$ chia hết cho $m.$
\end{dx}
Ta kí hiệu $a\equiv b\pmod{m}$ trong trường hợp $a$ và $b$ đồng dư theo modulo $m$. Ngược lại, nếu như $a$ không đồng dư $b$ đồng dư theo modulo $m,$ ta kí hiệu $a\not\equiv b\pmod{m}.$ Chẳng hạn
\begin{enumerate}
    \item Do $17$ và $87$ có cùng số dư (là $7$) khi chia cho $10,$ ta kí hiệu $17\equiv 87\pmod{10}.$
    \item Do $25$ và $28$ không có cùng số dư khi chia cho $11,$ ta kí hiệu $25\not\equiv 28\pmod{11}.$
\end{enumerate}
Tiếp theo, chúng ta hãy cùng tìm hiểu một số tính chất của phép toán đồng dư.
\begin{light}
\chu{Các quan hệ tương đương của đồng dư.}
\begin{enumerate}
    \item \chu{Quan hệ phản xạ.} Với mọi số nguyên $a$ và số nguyên $m\ne 0,$ ta luôn có $a\equiv a\pmod{m}.$
    \item \chu{Quan hệ đối xứng.} Với mọi số nguyên $a,b$ và số nguyên $m\ne 0,$ ta có
    $$a\equiv b\pmod{m}\Leftrightarrow b\equiv a\pmod{m}.$$
    \item \chu{Quan hệ bắc cầu.} Với mọi số nguyên $a,b,c$ và số nguyên $m\ne 0,$ ta có
    $$\heva{&a\equiv b\pmod{m} \\ &b\equiv c\pmod{m}}\Rightarrow a\equiv c\pmod{m}.$$
\end{enumerate}
\chu{Các phép toán trong đồng dư.}\\
Phép đồng dư còn có thể cộng, trừ, nhân và nâng lên lũy thừa các đồng dư thức có cùng một modulo. Cụ thể, với giả sử
$$a_1\equiv a_2\pmod{m},\qquad b_1\equiv b_2\pmod{m},$$
ta có các tính chất sau đây.
\begin{enumerate}
    \item $a_{1}+b_{1} \equiv a_{2}+b_{2} \pmod{m}.$
    \item $a_{1}-b_{1} \equiv a_{2}-b_{2} \pmod{m}.$
    \item $a_{1}b_{1} \equiv a_{2}b_{2} \pmod{m}.$
    \item $a_{1}^k \equiv a_{2}^k \pmod{m},$ trong đó $k$ là số tự nhiên tùy ý.
\end{enumerate}
Ngoài ra, đồng dư còn có \chu{luật giản ước}. Cụ thể, với mọi số nguyên $a,b,c,m$ khác $0,$ ta có
$$\heva{&ac\equiv bc\pmod{m}\\&(c,m)=1}\Rightarrow a\equiv b\pmod{m}.$$
\end{light}

Dưới đây là một dạng bài tập minh họa. Loạt bài tập này có thể giải bằng kiến thức về ước, bội, tuy nhiên xử lí chúng bằng đồng dư trông đẹp hơn rất nhiều!

\subsubsection*{Ví dụ minh họa}
\begin{bx}
Tìm số tự nhiên $n$ nhỏ nhất sao cho khi chia $n$ cho $3,4,5,6,$ ta lần lượt nhận được các số dư là $2,3,4,5.$
\loigiai{
Giả sử tồn tại số nguyên dương $n$ thỏa mãn yêu cầu. Ta có
$$\left\{\begin{aligned}
     n&\equiv 2\pmod{3} \\
     n&\equiv 3\pmod{4} \\
     n&\equiv 4\pmod{5} \\
     n&\equiv 5\pmod{6}      
\end{aligned}\right.
\Leftrightarrow \left\{\begin{aligned}
     2n&\equiv 4\pmod{3} \\
     2n&\equiv 6\pmod{4} \\
     2n&\equiv 8\pmod{5} \\
     2n&\equiv 10\pmod{6}      
\end{aligned}\right.
\Leftrightarrow \left\{\begin{aligned}
     2n+2&\equiv 0\pmod{3} \\
     2n+2&\equiv 0\pmod{4} \\
     2n+2&\equiv 0\pmod{5} \\
     2n+2&\equiv 0\pmod{6}.      
\end{aligned}\right.
$$
Dựa vào nhận xét trên, ta suy ra $2n+2$ là số nguyên dương nhỏ nhất chia hết cho $3,4,5,6.$ Bốn số đó đôi một nguyên tố cùng nhau, thế nên
$$2n+2=\vuong{3,4,5,6}=3\cdot 4\cdot 5\cdot 6=360.$$
Từ đây, ta thu được $n=179.$
}
\end{bx}
\begin{bx}
Tìm số nguyên dương $n$ nhỏ nhất, biết rằng khi chia $n$ cho $7,9,11,13$ ta nhận được các số dư tương ứng là $3,4,5,6$.
\nguon{Chuyên Khoa học Tự nhiên 2021}
\loigiai{Giả sử tồn tại số nguyên dương $n$ thỏa mãn yêu cầu. Ta có
$$\left\{\begin{aligned}
     n&\equiv 3\pmod{7} \\
     n&\equiv 4\pmod{9} \\
     n&\equiv 5\pmod{11} \\
     n&\equiv 6\pmod{13}      
\end{aligned}\right.
\Leftrightarrow \left\{\begin{aligned}
     2n&\equiv 6\pmod{7} \\
     2n&\equiv 8\pmod{9} \\
     2n&\equiv 10\pmod{11} \\
     2n&\equiv 12\pmod{13}      
\end{aligned}\right.
\Leftrightarrow \left\{\begin{aligned}
     2n+1&\equiv 0\pmod{7} \\
     2n+1&\equiv 0\pmod{9} \\
     2n+1&\equiv 0\pmod{11} \\
     2n+1&\equiv 0\pmod{13}.   
\end{aligned}\right.
$$
Dựa vào nhận xét trên, ta suy ra $2n+1$ là số nguyên dương nhỏ nhất chia hết cho $7,9,11,13.$ Bốn số đó đôi một nguyên tố cùng nhau, thế nên
$$2n+1=[7,9,11,13]=7\cdot 9\cdot 11\cdot 13=9009.$$
Từ đây, ta thu được $n=4504.$}
\end{bx}


\begin{bx}
Tìm tất cả các số tự nhiên $n$ nhỏ hơn $90,$ sao cho $n$ chia $3$ được dư là $1,$ còn $n$ chia $8$ được dư là $5.$

\loigiai{
Giả sử tồn tại số nguyên dương $n$ thỏa mãn yêu cầu. Ta có
$$\left\{\begin{aligned}
     n&\equiv 1\pmod{3} \\
     n&\equiv 5\pmod{8}   
\end{aligned}\right.
\Leftrightarrow \left\{\begin{aligned}
     n+11&\equiv 0\pmod{3} \\
     n+11&\equiv 0\pmod{8}. 
\end{aligned}\right.
$$
Dựa vào nhận xét trên, ta suy ra $n+11$ là số nguyên dương chia hết cho $3, 8.$ Hai số đó nguyên tố cùng nhau, thế nên tồn tại số nguyên dương $k$ sao cho
$$n+11=\vuong{3, 8}\cdot k= 3\cdot 8 \cdot k = 24k .$$
Từ đây, ta thu được $n=24k - 11,$ lại do $n \leq 90$ nên $n \in \left\{13, 37, 61, 85\right\}.$}
\begin{luuy}
Số $11$ tìm thấy được ở trong phần lập luận được sinh ra từ các bước làm theo cách liệt kê sau.
\begin{enumerate}[\sffamily \bfseries \color{tuancolor} Bước 1. ]
    \item Các số tự nhiên chia $3$ dư $1$ là $1,\ 4,\ 7,\ 10,\ldots.$
    \item Số tự nhiên nhỏ nhất trên dãy trên chia $8$ dư $5$ là $13.$ Ta cần cộng thêm $13$ với $11$ để được kết quả là bội chung nhỏ nhất của $3$ và $8.$
\end{enumerate}
\end{luuy}
\end{bx}

\begin{bx}
Tìm tất cả các số tự nhiên $n$ có ba chữ số, sao cho số dư khi chia $n$ cho $5,8,13$ lần lượt là $3,6,7.$
\loigiai{
Giả sử tồn tại số nguyên dương $n$ thỏa mãn yêu cầu. Ta có
$$\left\{\begin{aligned}
     n&\equiv 3\pmod{5} \\
     n&\equiv 6\pmod{8} \\
     n&\equiv 7\pmod{13}      
\end{aligned}\right.
\Leftrightarrow \left\{\begin{aligned}
     n + 162&\equiv 0\pmod{5} \\
     n + 162&\equiv 0\pmod{8} \\
     n + 162&\equiv 0\pmod{13}.  
\end{aligned}\right.
$$
Dựa vào nhận xét trên, ta suy ra $n+162$ là số nguyên dương chia hết cho $5, 8, 13.$ Ba số đó đôi một nguyên tố cùng nhau, thế nên tồn tại số nguyên dương $k$ sao cho
$$n+162=[5,8, 13]\cdot k=5\cdot 8\cdot 13\cdot k=520k.$$
Từ đây, ta thu được $n=520k - 162,$ lại do $n\leq 999$ nên $n \in \left\{358; 878\right\}.$}
\begin{luuy}
Số $162$ tìm thấy được ở trong phần lập luận được sinh ra từ các bước làm theo cách liệt kê sau.
\begin{enumerate}[\sffamily \bfseries \color{tuancolor} Bước 1. ]
    \item Các số tự nhiên chia $5$ dư $3$ là $3,\ 8,\ 13,\ 18,\ldots.$
    \item Số tự nhiên nhỏ nhất trong dãy trên và chia $8$ dư $6$ là $38.$ Các số vừa chia $5$ dư $3,$ vừa chia $8$ dư $6$ là $38,\ 78,\ 118,\ 158,\ldots$
    \item Số tự nhiên nhỏ nhất trên dãy trên chia $13$ dư $7$ là $358.$ Ta cần cộng thêm $358$ với $162$ để được kết quả là bội chung nhỏ nhất của $5,8,13.$
\end{enumerate}
\end{luuy}
\end{bx}

\begin{bx}
Cho các số nguyên $x,y,z$ thỏa mãn $(x-y)(y-z)(z-x)=x+y+z.$ Chứng minh rằng $x+y+z$ chia hết cho $27.$
\nguon{All Russian Olympiad 1993}
\loigiai{Trong trường hợp $x+y+z$ không chia hết cho $3,$ từ $(x-y)(y-z)(z-x)=x+y+z$ ta suy ra
$$3\nmid (x-y),\quad 3\nmid (y-z),\quad 3\nmid(z-x).$$
Theo đó $x,y,z$ có số dư đôi một khác nhau khi chia cho $3,$ và thế thì
$$x+y+z\equiv 0+1+2\equiv 0\pmod{3},$$
mâu thuẫn với giả sử. Như vậy $x+y+z=(x-y)(y-z)(z-x)$ chia hết cho $3,$ và trong $x,y,z$ hiển nhiên có hai số cùng dư khi chia cho $3.$ Không mất tổng quát, ta giả sử rằng $x\equiv y\pmod 3.$ Giả sử này cho ta
$$0\equiv x+y+z\equiv 2x+z\equiv z-x\pmod{3}.$$
Lập luận trên chứng tỏ $z-x$ chia hết cho $3.$ Kết hợp với giả sử $x-y$ chia hết cho $3$ ở trên, ta suy ra cả $x-y,\ y-z$ và $z-x$ chia hết cho $3,$ và 
$$27\mid (x-y)(y-z)(z-x)=x+y+z.$$
Nhận xét vừa rồi dẫn ta đến điều phải chứng minh.
}
\end{bx}

%nguyệt anh
\begin{bx}
Chứng minh rằng nếu $a,b,c$ là các số nguyên thỏa mãn $a+b+c$ chia hết cho $6$ thì $(a+b)(b+c)(c+a)-2abc$ chia hết cho $6.$
\loigiai{
Giả sử $a,b,c$ là các số lẻ, ta suy ra $a+b+c$ cũng là số lẻ. Điều này mâu thuẫn với giả thiết $a+b+c$ chia hết cho $6$. Do đó giả sử sai nên trong $3$ số $a,b,c$ tồn tại một số chia hết cho $2$.\\
Xét hệ đồng dư modulo $6$, ta có 
$$a+b+c\equiv 0\pmod{6}\Rightarrow \heva{a+b&\equiv -c\pmod{6}\\ b+c &\equiv -a \pmod{6}\\ c+a&\equiv -b \pmod 6.}$$
Từ đây, ta suy ra
$$(a+b)(b+c)(c+a)-2abc\equiv(-c)(-a)(-b)-2abc\equiv-3abc\equiv0\pmod{6}.$$
Như vậy, bài toán được chứng minh.
}
\end{bx}

\section{Tính chia hết của đa thức cho một số nguyên}

Đây là dạng bài tập cơ bản của chia hết. Dạng bài tập này thường xuất hiện trong các đề thi chuyên, đề thi học sinh giỏi dưới dạng một ý nhỏ trong câu số học nhiều ý. Dưới đây là một số ví dụ minh họa.

\subsection*{Ví dụ minh họa}

\begin{bx}
Chứng minh rằng $n\tron{2n^2+7}$ chia hết cho $3$ với mọi số nguyên $n.$
\loigiai{
Ta nhận thấy rằng
$$n\tron{2n^2+7}=n\tron{2n^2-2+9}=n\tron{2n^2-2}+9n=2n(n-1)(n+1)+9n.$$
Do $n(n-1)(n+1)$ là tích ba số nguyên liên tiếp nên nó chia hết cho $3$ (thậm chí là cho $6$). Cùng với đó, $9n$ cũng chia hết cho $3.$ Như vậy
$2n(n-1)(n+1)+9n$
chia hết cho $3,$ và bài toán được chứng minh.}
\end{bx}

\begin{bx}
Cho $a,b$ là hai số chính phương liên tiếp. Chứng minh $ab-a-b+1$ chia hết cho $192.$
\nguon{Chuyên Tin Bình Định 2019}
\loigiai{
Đặt $a=(2n-1)^2,b=(2n+1)^2,$ với $n$ là số nguyên. Phép đặt này cho ta
\begin{align*}
    ab-a-b+1&=(a-1)(b-1)\\&=\tron{(2n-1)^2-1}\tron{(2n+1)^2-1}
    \\&=(2n-2)(2n)(2n)(2n+2)\\&
    =16(n-1)n^2(n+1).
\end{align*}
Tới đây, ta chia bài toán thành các bước làm sau.
\begin{enumerate}[\color{tuancolor}\bf\sffamily Bước 1.]
    \item Ta chứng minh $(n-1)n^2(n+1)$ chia hết cho $4.$\\
    Ta nhận thấy rằng $(n-1)n$ và $n(n+1)$ là hai tích của hai số nguyên liên tiếp. Chúng đều chia hết cho $2,$ thế nên $(n-1)n\cdot n(n+1)=(n-1)n^2(n+1)$ chia hết cho $4.$
    \item Ta chứng minh $(n-1)n^2(n+1)$ chia hết cho $3.$\\
    Đây là điều hiển nhiên, bởi vì $(n-1)n^2(n+1)$ chia hết cho tích $(n-1)n(n+1).$
\end{enumerate}
Dựa theo các lập luận kể trên, ta có $ab-a-b+1$ chia hết cho $16[3,4]=192.$ Chứng minh hoàn tất.}
\end{bx}

\begin{bx}
Cho $a$ và $b$ là các số nguyên dương thỏa mãn $a^2-ab+\dfrac{3}{2}b^2$ chia hết cho $25.$ Chứng minh rằng cả $a$ và $b$ đều chia hết cho $5.$ 
\nguon{Chuyên Tin Bình Thuận 2021}
\loigiai{Với các số nguyên dương $a,b$ thỏa mãn giả thiết, ta có
$$25\mid 4\left(a^2-ab+\dfrac{3}{2}b^2\right)=(2a-b)^2+5b^2.$$
Ta được $2a-b$ chia hết cho $25,$ tức là $2a-b$ chia hết cho $5.$ \\
Tiếp tục sử dụng $25\mid (2a-b)^2+5b^2,$ ta nhận thấy $25$ cũng là ước của $5b^2,$ thế nên $b$ chia hết cho $5.$\\ Kết hợp với $5\mid (2a-b),$ ta có điều phải chứng minh là $a$ và $b$ cùng chia hết cho $5.$}
\end{bx}

\subsection*{Bài tập tự luyện}

\begin{btt}
Cho số tự nhiên $n.$ Chứng minh $n^4-14n^3+71 n^2-154n+120$ chia hết cho $24.$
\end{btt}

\begin{btt}
Với mọi số tự nhiên $n$ lẻ hãy chứng minh rằng $n^{3}+3 n^{2}-n-3$ chia hết cho $48.$
\end{btt}

\begin{btt}
Cho $a,b,c$ là các số nguyên thỏa mãn $a+b+20c=c^3.$ Chứng minh rằng $a^3+b^3+c^3$ chia hết cho $6.$
\nguon{Chuyên Toán Lâm Đồng 2021}
\end{btt}


\begin{btt}
Chứng minh với mọi số nguyên lẻ $n$ thì $n^8-n^6-n^4+n^2$ chia hết cho $5760.$
\end{btt}

\begin{btt}
Chứng minh rằng với mọi số tự nhiên $n,$ ta có $n^2+n+16$ không chia hết cho $49.$
\nguon{Chuyên Toán Hà Nội 2021}
\end{btt}

\begin{btt}
Chứng minh biểu thức $S = {n^3}{\left( {n + 2} \right)^2} + \left( {n + 1} \right)\left( {{n^2} - 5n + 1} \right) - 2n - 1$ chia hết cho $120$ với $n$ là số nguyên.
\nguon{Chuyên Toán Bình Phước 2017 $-$ 2018}
\end{btt}

\begin{btt}
 Cho $x,y$ là hai số nguyên thỏa mãn $x>y>0.$
\begin{enumerate}[a,]
     \item Chứng minh rằng nếu $x^3-y^3$ chia hết cho $3$ thì $x^3-y^3$ chia hết cho $9.$
     \item Tìm tất cả các số nguyên dương $k$ sao cho $x^k-y^k$ chia hết cho $9$ với mọi cặp số nguyên $x,y$ thỏa mãn $xy$ không chia hết cho $9.$
 \end{enumerate}
\end{btt}

\begin{btt}
Cho $x, y$ là các số nguyên sao cho $x^2-2xy-y$ và $xy-2y^2-x$ đều chia hết cho 5. Chứng minh rằng $2x^2+y^2+2x+y$ cũng chia hết cho $5.$
\nguon{Chuyên Khoa học Tự nhiên Hà Nội 2018 $-$ 2019}
\end{btt}

\begin{btt}
Cho hai số nguyên $m,n.$ Chứng minh rằng nếu $5(m+n)^2+mn$ chia hết cho $441$ thì $mn$ chia hết cho $441.$
\nguon{Chuyên Toán Trung học thực hành Đại học Sư phạm thành phố Hồ Chí Minh 2013}
\end{btt}

\begin{btt}
Cho $n$ là số nguyên dương tùy ý. Với mỗi số nguyên dương $k,$ đặt $${{S}_{k}}={{1}^{k}}+{{2}^{k}}+\cdots+{{n}^{k}}.$$
Chứng minh rằng ${{S}_{2019}}$ chia hết cho ${{S}_{1}}$.
\nguon{Chuyên Toán Thanh Hóa 2019}
\end{btt}

\subsection*{Hướng dẫn bài tập tự luyện}

\begin{gbtt}
Cho số tự nhiên $n.$ Chứng minh $n^4-14n^3+71 n^2-154n+120$ chia hết cho $24.$
\loigiai{
Số đã cho được phân tích nhân tử thành
$$n^4-14n^3+71 n^2-154n+120=(n-2)(n-3)(n-4)(n-5).$$
Đây là tích bốn số nguyên liên tiếp, và nó chia hết cho $4!=24.$ Bài toán được chứng minh.}
\end{gbtt}

\begin{gbtt}
Với mọi số tự nhiên $n$ lẻ hãy chứng minh rằng $n^{3}+3 n^{2}-n-3$ chia hết cho $48.$
\loigiai{
Đặt $n=2m+1.$ Phép đặt này cho ta
    $$n^3+3n^2-n-3=(n+3)(n-1)(n+1)=(2m+4)(2m)(2m+2)=8m(m+1)(m+2).$$
Do $m(m+1)(m+2)$ là tích ba số nguyên liên tiếp, nó chia hết cho $3!=6.$\\
Như vậy, số đã cho chia hết cho $8\cdot6=48.$}
\end{gbtt}

\begin{gbtt}
Cho $a,b,c$ là các số nguyên thỏa mãn $a+b+20c=c^3.$ Chứng minh rằng $a^3+b^3+c^3$ chia hết cho $6.$
\nguon{Chuyên Toán Lâm Đồng 2021}
\loigiai{
Ta nhận thấy rằng
\begin{align*}
    a^3+b^3+c^3
    &=a^3+b^3+2c^3-(a+b+20c)
    \\&=(a^3-a)+(b^3-b)+2(c^3-c)-18c
    \\&=a(a-1)(a+1)+b(b-1)(b+1)+2c(c-1)(c+1)-18c.
\end{align*}
Các tích dạng $x^3-x=x(x-1)(x+1)$ chia hết cho $6$ do đây là tích ba số nguyên liên tiếp, và $18c$ cũng chia hết cho $6.$ Từ đây, ta suy ra điều phải chứng minh.}
\end{gbtt}

\begin{gbtt}
Chứng minh với mọi số nguyên lẻ $n$ thì $n^8-n^6-n^4+n^2$ chia hết cho $5760.$
\loigiai{Đặt $n=2k+1.$ Biểu thức đã cho được phân tích nhân tử thành
\begin{align*}
    n^8-n^6-n^4+n^2
    &=n^2(n-1)^2(n+1)^2\left(n^{2}+1\right)
    \\&=(2k+1)^2(2k)^2(2k+2)^2\tron{(2k+1)^2+1}
    \\&=32(2k+1)^2k^2(k+1)^2\tron{k^2+k+1}.
\end{align*}
Tới đây, ta chia bài toán thành các bước làm sau.
\begin{enumerate}[\color{tuancolor}\bf\sffamily Bước 1.]
    \item Ta chứng minh $A=(2k+1)^2k^2(k+1)^2\tron{k^2+k+1}$ chia hết cho $4.$\\
    Ta nhận thấy rằng $k(k+1)$ là tích của hai số nguyên liên tiếp. Tích này chẵn, do đó $k^2(k+1)^2$ chia hết cho $4.$ Ta có $A$ chia hết cho $4$ từ đây. 
    \item Ta chứng minh $A=(2k+1)^2k^2(k+1)^2\tron{k^2+k+1}$ chia hết cho $9.$
    \begin{itemize}
        \item Nếu $k$ chia hết cho $3,$ ta có $k^2$ chia hết cho $9,$ suy ra $A$ cũng chia hết cho $9.$
        \item Nếu $k$ chia $3$ dư $1,$ ta có $(2k+1)^2$ chia hết cho $9,$ suy ra $A$ cũng chia hết cho $9.$  
        \item Nếu $k$ chia $3$ dư $2,$ ta có $(k+2)^2$ chia hết cho $9,$ suy ra $A$ cũng chia hết cho $9.$                
    \end{itemize}
    \item Ta chứng minh $A=(2k+1)^2k^2(k+1)^2\tron{k^2+k+1}$ chia hết cho $5.$\\
    Việc này không khó, với hướng đi là xét số dư của $k$ khi chia cho $5$ rồi chỉ ra một trong bốn thừa số $2k+1,\ k,\ k+1,\ k^2+k+1$ chia hết cho $5.$    
\end{enumerate}
Dựa theo các lập luận kể trên, ta có $n^8-n^6-n^4+n^2$ chia hết cho $32[4,9,5]=5760.$\\ Như vậy, toàn bộ bài toán đã cho được chứng minh.}
\end{gbtt}

\begin{gbtt}
Chứng minh rằng với mọi số tự nhiên $n,$ ta có $n^2+n+16$ không chia hết cho $49.$
\nguon{Chuyên Toán Hà Nội 2021}
\loigiai{
Giả sử tồn tại số tự nhiên $n$ thỏa mãn $49\mid \left(n^2+n+16\right).$ Giả sử này cho ta
    \begin{align*}
    49\mid 4\tron{n^2+n+16}
    &\Rightarrow
    49 \mid \vuong{(2n+1)^2+63}
    \\&\Rightarrow 7\mid\vuong{(2n+1)^2+63}
    \\&\Rightarrow
    7\mid (2n+1)^2
    \\&\Rightarrow
    7\mid (2n+1)
    \\&\Rightarrow
    49\mid (2n+1)^2.
    \end{align*}
Tiếp tục kết hợp điều này với $49 \mid \vuong{(2n+1)^2+63},$ ta có $63$ chia hết cho $49,$ một điều vô lí. \\
Như vậy, giả sử phản chứng là sai, và bài toán được chứng minh.}
\end{gbtt}

\begin{gbtt}
Chứng minh biểu thức $S = {n^3}{\left( {n + 2} \right)^2} + \left( {n + 1} \right)\left( {{n^2} - 5n + 1} \right) - 2n - 1$ chia hết cho $120$ với $n$ là số nguyên.
\nguon{Chuyên Toán Bình Phước 2017 $-$ 2018}
\loigiai{
Ta thấy rằng $120 = 3\cdot5\cdot8.$ Ta chia bài toán thành các bước làm sau
\begin{enumerate}[\color{tuancolor}\bf\sffamily Bước 1.]
    \item Ta chứng minh $S$ chia hết cho $3.$ Biến đổi $S$ ta được
    \[\begin{aligned}
  S &= {n^5} + 5{n^4} + 5{n^3} - 5{n^2} - 6n \\
   &= {n^5} - {n^3} + 6{n^3} + 5\left( {{n^4} - {n^2}} \right) - 6n \\
   &= {n^2}\left( {n - 1} \right)n\left( {n + 1} \right) + 6{n^3} + 5n\left( {n - 1} \right)n\left( {n + 1} \right) - 6n. 
\end{aligned}\]
    Do $3\mid\left( {n - 1} \right)n\left( {n + 1} \right)$ nên ta suy ra $S$ chia hết cho $3.$
    \item Ta chứng minh $S$ chia hết cho $5.$ Biến đổi $S$ ta được
$$S = {n^5} + 5{n^4} + 5{n^3} - 5{n^2} - 6n = {n^5} - n + 5\left( {{n^4} + {n^3} - {n^2} - n} \right).$$
    Công việc còn lại chỉ là chứng minh $n^5-n$ chia hết cho $5.$ Thật vậy
    \[\begin{aligned}
  {n^5} - n &= n\left( {n - 1} \right)\left( {n + 1} \right)\left( {{n^2} + 1} \right) \\
   &= n\left( {n - 1} \right)\left( {n + 1} \right)\left( {{n^2} - 4 + 5} \right) \\
   &= n\left( {n - 1} \right)\left( {n + 1} \right)\left( {{n^2} - 4} \right) + 5n\left( {n - 1} \right)\left( {n + 1} \right) \\
   &= \left( {n - 2} \right)\left( {n - 1} \right)n\left( {n + 1} \right)\left( {n + 2} \right) + 5n\left( {n - 1} \right)\left( {n + 1} \right).
\end{aligned}\]
    Nói tóm lại, $S$ chia hết cho $5.$
    \item Ta chứng minh $S$ chia hết cho $8.$ Biến đổi $S$ ta được
    $$S = {n^5} + 5{n^4} + 5{n^3} - 5{n^2} - 6n = 4{n^3}\left( {n + 1} \right) + n\left( {n + 1} \right)\left( {{n^3} + n - 6} \right).$$
    Tới đây, ta xét các trường hợp sau.
\begin{itemize}
    \item\chu{Trường hợp 1.} Nếu $n$ là số chẵn, ta đặt $n= 2k.$ Ta có
    $$S = 32{k^5} + 80{k^4} + 40{k^3} - 8{k^2} - 12k\left( {k + 1} \right).$$
    Từ đó suy ra $S$ chia hết cho $8$.
    \item\chu{Trường hợp 2.} Nếu $n$ là số lẻ, ta đặt $n = 2k + 1.$  Ta có
    $$S = 4{n^3}\left( {n + 1} \right) + n\left( {n + 1} \right)\left( {{n^3} + n - 6} \right).$$
    Ta có $4{n^3}\left( {n + 1} \right) = 8{\left( {2k + 1} \right)^3}\left( {k + 1} \right)$ chia hết cho 8. Lại có $$n\left( {n + 1} \right)\left( {{n^3} + n - 6} \right) = \left( {2k + 1} \right)\left( {2k + 2} \right)\left( {8{k^3} + 12{k^2} + 8k - 4} \right)$$ chia hết cho 8.
    Do đó $S$ chia hết cho 8.
\end{itemize}
\end{enumerate}
Như vậy với mọi số nguyên $n,$ số $S$ chia hết cho $[3,5,8]=120.$ Bài toán được chứng minh.}
\end{gbtt}

\begin{gbtt}
 Cho $x,y$ là hai số nguyên thỏa mãn $x>y>0.$
\begin{enumerate}[a,]
     \item Chứng minh rằng nếu $x^3-y^3$ chia hết cho $3$ thì $x^3-y^3$ chia hết cho $9.$
     \item Tìm tất cả các số nguyên dương $k$ sao cho $x^k-y^k$ chia hết cho $9$ với mọi cặp số nguyên $x,y$ thỏa mãn $xy$ không chia hết cho $9.$
 \end{enumerate}
\loigiai{
\begin{enumerate}[a,]
    \item Trước hết, ta nhận thấy ${{x}^{3}}-{{y}^{3}}={{\left( x-y \right)}^{3}}-3xy\left( x-y \right)$. Do các số
    \[{{x}^{3}}-{{y}^{3}},\quad 3xy\left( x-y \right)\] 
    chia hết cho $3$ nên suy ra $(x-y)^3$ chia hết cho $3,$ kéo theo $3\mid (x-y).$ Như vậy
    \[{{x}^{3}}-{{y}^{3}}={{\left( x-y \right)}^{3}}-3xy\left( x-y \right)\] chia hết cho $9,$ do cả $(x-y)^3$ và $3(x-y)$ đều chia hết cho $9.$
    \item Nếu $x$ không chia hết cho $3$ thì $(x-1)(x+1)=x^2-1$ chia hết cho $3,$ suy ra 
    $$x^6-1=\tron{x^2-1}\tron{x^4+x^2+1}$$
    chia hết cho $3.$ Chứng minh tương tự, $y^3-1$ cũng chia hết cho $3.$ Áp dụng kết quả câu a, ta có
    \[9\mid\tron{x^6-1},\quad 9\mid\tron{y^6-1}.\] 
    Đặt $k=6a+r,$ trong đó $0\le r\le 5.$ Ta sẽ có 
    \[{{x}^{k}}-{{y}^{k}}={{x}^{6a+r}}-{{y}^{6a+r}}={{x}^{r}}\left( {{x}^{6a}}-1 \right)+{{y}^{r}}\left( {{y}^{6a}}-1 \right)+\left( {{x}^{r}}-{{y}^{r}} \right).\]
    Như vậy, $x^k-y^k$ chia hết cho $9$ khi và chỉ khi $x^r-y^r$ chia hết cho $9.$ Ta sẽ chứng minh $r=0.$ Thật vậy, nếu $r\ne 0,$ ta chọn $x=3,y=1,$ và khi đó $x^r-y^r$ không chia hết cho $9.$ Tóm lại, các số $k$ cần tìm là bội của $6.$
\end{enumerate}
}
\end{gbtt}

\begin{gbtt}
Cho $x, y$ là các số nguyên sao cho $x^2-2xy-y$ và $xy-2y^2-x$ đều chia hết cho 5. Chứng minh rằng $2x^2+y^2+2x+y$ cũng chia hết cho $5.$
\nguon{Chuyên Khoa học Tự nhiên Hà Nội 2018 $-$ 2019}
\loigiai{
Trước tiên ta có số sau đây chia hết cho $5.$ $$\left( {{x}^{2}}-2{x}y-y \right)+\left( xy-2{{y}^{2}}-x \right)={{x}^{2}}-2{{y}^{2}}-xy-x-y=\left( x+y \right)\left( x-2y+1 \right).$$
Tới đây, ta xét các trường hợp sau.
\begin{enumerate}
    \item Nếu $x+y$ chia hết cho $5,$ ta có $x\equiv -y\pmod{5}.$ Kết hợp với giả thiết, ta được
    $$0\equiv x^2-2xy-y\equiv (-y)^2-2(-y)y-y\equiv 3y^2-y=y(3y-1)\pmod{5}.$$
    Ta lại xét các trường hợp nhỏ hơn sau.
    \begin{itemize}
        \item \chu{Trường hợp 1.} Nếu $5\mid y,$ do $5\mid (x+y)$ nên $5\mid x.$ Từ đó
        $$5\mid \tron{2x^2+y^2+2x+y}.$$
        \item \chu{Trường hợp 2.} Nếu $5\mid (3y-1),$ ta có 
        $$5\mid 2(3y-1)=(y-2)+5y$$ nên $y\equiv 2\pmod{5}.$ Do $5\mid (x+y)$ nên $x\equiv -2\pmod{5}.$ Từ đó
        $$2x^2+y^2+2x+y\equiv 2\cdot(-2)^2+2^2+2\cdot(-2)+2=10\equiv 0\pmod{5}.$$
    \end{itemize}
    \item Nếu $x-2y+1$ chia hết cho $5,$ ta có $x\equiv 2y-1\pmod{5}.$ Kết hợp với giả thiết, ta được
    $$x^2-2xy-y\equiv (2y-1)^2-2(2y-1)y-y=1-3y\equiv 2y+1\pmod{5}.$$
    Từ đó, ta có $5\mid 3(2y+1)=5y+(y+3)$ hay $y\equiv 2\pmod{5}$ và $y\equiv 3\pmod{5}.$ Như vậy
        $$2x^2+y^2+2x+y\equiv 2\cdot3^2+2^2+2\cdot3+2=30\equiv 0\pmod{5}.$$    
\end{enumerate}
Bài toán được chứng minh trong mọi trường hợp.}
\end{gbtt}

\begin{gbtt}
Cho hai số nguyên $m,n.$ Chứng minh rằng nếu $5(m+n)^2+mn$ chia hết cho $441$ thì $mn$ chia hết cho $441.$
\nguon{Chuyên Toán Trung học thực hành Đại học Sư phạm thành phố Hồ Chí Minh 2013}
\loigiai{
Từ giả thiết, ta có $4\tron{5(m+n)^2+mn}$ chia hết cho $441,$ và vì thế
\begin{align*}
    4\tron{5(m+n)^2+mn}
    &=20(m+n)^2+4mn \\&=20(m+n)^{2}+\left[(m+n)^2-(m-n)^2\right]
    \\&=21(m+n)^2-(m-n)^2.
\end{align*}
chia hết cho $21,$ thế nên $(m-n)^2$ cũng là bội của $21.$ Ta lần lượt suy ra
\begin{align*}
    21\mid (m-n)^2
    &\Rightarrow \heva{3\mid (m-n)^2 \\ 7\mid (m-n)^2}
    \\&\Rightarrow \heva{3\mid (m-n) \\ 7\mid (m-n)}
    \\&\Rightarrow \heva{9&\mid (m-n)^2 \\ 49&\mid (m-n)^2} 
    \\&\Rightarrow [9,49]\mid (m-n)^2
    \\&\Rightarrow 441\mid (m-n)^2.
\end{align*}
Từ lập luận này kết hợp với đẳng thức
$$4\tron{5(m+n)^2+mn}=21(m+n)^2-(m-n)^2$$
ta chỉ ra $21(m+n)^2$ chia hết cho $441.$ Bằng lập luận tương tự, ta có $m+n$ chia hết cho cả $3$ và $7.$ Như vậy
\begin{align*}
\heva{&3\mid (m+n) \\ &7\mid (m+n) \\ &3\mid (m-n) \\ &7\mid (m-n)}
&\Rightarrow \heva{&[3,7]\mid (m+n) \\ &[3,7]\mid (m-n)}
\\&\Rightarrow \heva{&21\mid (m+n) \\ &21\mid (m-n)}
\\&\Rightarrow \heva{&21\mid \bigg((m+n)+(m-n)\bigg)\\&21\mid \bigg((m+n)-(m-n)\bigg)}
\\&\Rightarrow \heva{&21\mid 2m\\&21\mid 2n}
\\&\Rightarrow \heva{&21\mid m\\&21\mid n}
\\&\Rightarrow 441\mid mn.
\end{align*}
Bài toán đã cho được chứng minh.}
\end{gbtt}

\begin{gbtt}
Cho $n$ là số nguyên dương tùy ý. Với mỗi số nguyên dương $k,$ đặt $${{S}_{k}}={{1}^{k}}+{{2}^{k}}+\cdots+{{n}^{k}}.$$
Chứng minh rằng ${{S}_{2019}}$ chia hết cho ${{S}_{1}}$.
\nguon{Chuyên Toán Thanh Hóa 2019}
\loigiai{
Ta tính được $S_1=\dfrac{n(n+1)}{2}.$ Ta sẽ chứng minh kết quả tổng quát
$$n(n+1)\mid 2S_k,\text{ với }k\text{ là số nguyên dương lẻ}.$$
Trước hết, từ nhận xét
 $2{{S}_{k}}=2\left( {{1}^{k}}+{{2}^{k}}+\cdots+{{n}^{k}} \right)$
ta có $2S_k$ chia hết cho $(n+1).$ Tiếp theo, từ nhận xét $$2{{S}_{k}}=2{{n}^{k}}+\left[ {{1}^{k}}+{{\left( n-1 \right)}^{k}} \right]+\cdots+\left[ {{\left( n-1 \right)}^{k}}+{{1}^{k}} \right],$$
ta có $2S_k$ chia hết cho $n.$ Như vậy, cả kết quả tổng quát và bài toán đã cho được chứng minh.}
\end{gbtt}

\section{Đồng dư thức với số mũ lớn}
\subsection*{Ví dụ minh họa}
\begin{bx}
Chứng minh rằng với mọi số tự nhiên $n,$ ta luôn có \[5^{n+2}+26\cdot5^{n}+8^{2 n+1}\text{ chia hết cho } 59.\]
\loigiai{
Xét trong hệ đồng dư modulo $59,$ ta có
$$5^{n+2}+26\cdot5^{n}+8^{2n+1}=51\cdot 5^n+8\cdot 64^n\equiv 51\cdot 5^n+8\cdot 5^n=59\cdot 5^n\equiv 0\pmod{59}.$$
Bài toán được chứng minh.}
\end{bx}

\begin{bx}
Xác định tất cả các số tự nhiên $n$ sao cho $2^n-3$ chia hết cho $7.$
\loigiai{
Trong bài toán này, ta sẽ xét các số dư của $n$ khi chia cho $3.$ Cụ thể
\begin{enumerate}
    \item Nếu $n$ chia hết cho $3,$ ta đặt $n=3k,$ với $k$ là số tự nhiên. Ta có.
    $$2^{3k}-1=8^k-1\equiv 1-1\equiv 0\pmod{7}.$$
    \item Nếu $n$ chia cho $3$ dư $1,$ ta đặt $n=3k+1,$ với $k$ là số tự nhiên. Ta có.
    $$2^{3k+1}-1=2\cdot8^k-1\equiv 2-1\equiv 1\pmod{7}.$$
    \item Nếu $n$ chia cho $3$ dư $2,$ ta đặt $n=3k+2,$ với $k$ là số nguyên dương. Ta có.
    $$2^{3k+2}-1=4\cdot8^k-1\equiv 4-1\equiv 3\pmod{7}.$$
\end{enumerate}
Theo trên, ta kết luận rằng tất cả các số tự nhiên $n$ chia cho  $3$ dư $2$ đều thỏa yêu cầu bài toán.}
    \begin{luuy}
\chu{Nhận xét.} Cơ sở của việc chọn modulo $3$ để xét cho $2^n$ nhằm tìm ra dạng của nó là nhờ vào hai định lí sau
    \begin{enumerate}
        \item Cho số nguyên dương $a$ và số nguyên tố $p,$ khi đó nếu $(a,p)=1$ thì $$a^{p-1}\equiv 1\pmod{p}.$$
        \item Cho các số nguyên dương $a,b$ nguyên tố cùng nhau. Khi đó, tồn tại số nguyên dương $n$ sao cho $a^n\equiv1\pmod{b}.$
    \end{enumerate}
    Số nguyên dương $n$ nhỏ nhất ở trong định lí thứ hai chính là modulo ta cần xét. Đặc biết, trong định lí thứ hai, nếu $b=p$ là một số nguyên tố, ta còn có thể chỉ ra $n$ là ước của $p-1.$
    \end{luuy}
\end{bx}

\begin{bx}
Tìm số dư trong phép chia $5^{182}$ cho $7.$
\end{bx}
\nx Dựa vào việc tìm được số nguyên dương $n$ nhỏ nhất thỏa mãn $5^n\equiv 1\pmod{7}$ là $n=6,$ ta sẽ xuất phát bài toán này từ đồng dư thức
\[5^6\equiv 1\pmod{7}.\]
\loigiai{
Ta nhận thấy rằng
$5^{6}=15625=7\cdot2232+1 \equiv 1\pmod{7}.$\\
Căn cứ vào nhận xét trên, ta chỉ ra
$$5^{182}=\left(5^6\right)^{30}\cdot25\equiv 1\cdot 25\equiv 4\pmod{7}.$$
Như vậy, số dư của $5^{182}$ khi chia cho $7$ là $4.$}

\begin{bx}
Tìm chữ số tận cùng của $ 7^{969}.$
\loigiai{
Xét trong hệ đồng dư modulo $10,$ ta có
    \begin{align*} 
        7^2 \equiv -1 \pmod{10} &\Rightarrow 7^4 \equiv 1 \pmod{10}
        \\&\Rightarrow 7^{968} \equiv 1 \pmod{10}\\&
        \Rightarrow 7^{969} \equiv 7 \pmod{10}.
    \end{align*}
    Vậy $7^{969}$ có tận cùng là $7.$}
\end{bx}

\subsection*{Bài tập tự luyện}

\begin{btt}
Chứng minh rằng với mọi số tự nhiên $n,$ ta luôn có
\[12^n+4^n-7^n-9^n\text{ chia hết cho } 15.\]
\end{btt}

\begin{btt}
Chứng minh rằng với mọi số tự nhiên $n,$ ta luôn có
\[2005^n+60^n-1897^n-168^n \text{ chia hết cho } 2004.\]
\nguon{Chuyên Toán Lai Châu 2021}
\end{btt}

\begin{btt}
Chứng minh rằng với mọi số tự nhiên $n,$ ta luôn có \[5^{2 n-1} \cdot 2^{n+1}+3^{n+1} \cdot 2^{2 n-1} \text{ chia hết cho } 38.\]
\end{btt}

\begin{btt}
Với mỗi số nguyên dương $n$, ta đặt $$a_{n}=2^{2 n+1}-2^{n+1}+1, \quad b_{n}=2^{2 n+1}+2^{n+1}+1.$$ 
Chứng minh rằng với mọi $n$ thì $a_{n} b_{n}$ chia hết cho $5$ và $a_{n}+b_{n}$ không chia hết cho $5.$
\nguon{Chuyên Toán Phổ thông Năng khiếu 2003}
\end{btt}

\begin{btt}
Chứng minh rằng $4^n-2019n-1$ chia hết cho $9$ với mọi số tự nhiên $n.$
\nguon{Chuyên Toán Quảng Ninh 2019}
\end{btt}

\begin{btt}
Tìm số nguyên dương $n$ nhỏ nhất sao cho
\begin{multicols}{2}
\begin{enumerate}[a,]
    \item $5^n$ chia $7$ dư $1.$
    \item $11^n$ chia $13$ dư $1.$
    \item $2021^n$ chia $31$ dư $1.$
    \item $14^n$ chia $11$ dư $5.$
    \item $1048^n$ chia $23$ dư $4.$
    \item $227^n+194^n$ chia $11$ dư $8.$
\end{enumerate}
\end{multicols}
\end{btt}

\begin{btt}
Tìm số dư trong các phép chia sau
\begin{multicols}{2}
    \begin{enumerate}[a,]
        \item $3^{123}$ khi chia cho $7$.
        \item $7^{2021}$ khi chia cho $11$.
        \item $8^{227}$ khi chia cho $31$.
        \item $227^{111}$ khi chia cho $8$.
    \end{enumerate}
\end{multicols}
\end{btt}

\begin{btt}\
\begin{enumerate}[a,] 
    \item Tìm tất cả các số nguyên dương $n$ sao cho $2^{n}n+3^{n}$ chia hết cho $5.$
    \item Tìm tất cả các số nguyên dương $n$ sao cho $2^{n}n+3^{n}$ chia hết cho $25.$
\end{enumerate} 
\nguon{Chuyên Toán Phổ thông Năng khiếu 1997}
\end{btt}

\begin{btt}
Tìm số dư trong các phép chia sau
\begin{multicols}{2}
\begin{enumerate}[a,]
    \item $87^{32^{47}}$ khi chia cho $19$.
    \item $9^{8^7} + 5^{6^7}$ khi chia cho $13$.
\end{enumerate}
\end{multicols}
\end{btt}

\begin{btt}
Chứng minh rằng $2020^{2021^{2022}} + 2036$ chia hết cho $52$.
\end{btt}

\begin{btt}
Chứng minh rằng $92^{183^{139}} + 183^{139^{92}} + 139^{92^{183}}$ chia hết cho $138$.
\end{btt}

\begin{btt}
Chứng minh rằng với mọi số tự nhiên $n,$ ta luôn có
\[2^{3^{4n+1}}+3^{2^{4n+1}}+5\text{ chia hết cho }22.\]
\end{btt}

\begin{btt}
Tìm chữ số tận cùng của $4^{2021} + 7^{2022} + 9^{2023}.$
\end{btt}

\begin{btt}
Tìm hai chữ số tận cùng của các số sau
\begin{multicols}{3}
    \begin{enumerate}[a,]
    \item $6^{2021}$.
    \item $69^{2022}.$
    \item $15^{15^{15^{15}}}$.
\end{enumerate}
\end{multicols}
\end{btt}

\begin{btt}
Tìm ba chữ số tận cùng của $A=3\cdot9\cdot15\ldots2025$.
\end{btt}

\subsection*{Hướng dẫn bài tập tự luyện}

\begin{gbtt}
Chứng minh rằng với mọi số tự nhiên $n,$ ta luôn có
\[12^n+4^n-7^n-9^n\text{ chia hết cho } 15.\]
\loigiai{
Đặt $A=12^n+4^n-7^n-9^n.$ Ta sẽ chứng minh $A$ chia hết cho $3$ và $5.$ Thậy vậy
\begin{itemize}
    \item $A=(3\cdot4)^n+(3+1)^n-(3\cdot2+1)^n-(3\cdot3)^n\equiv 0+1-1-0\equiv 0\pmod{3}.$
    \item $A=(5\cdot2+2)^n+4^n-(5+2)^n-(5+4)^n\equiv 2^n+4^n-2^n-4^n\equiv 0\pmod{5}.$    
\end{itemize}
Dựa vào các nhận xét trên, ta suy ra $A$ chia hết cho $[3,5]=15.$ Như vậy, bài toán được chứng minh.
}
\end{gbtt}

\begin{gbtt}
Chứng minh rằng với mọi số tự nhiên $n,$ ta luôn có
\[2005^n+60^n-1897^n-168^n \text{ chia hết cho } 2004.\]
\nguon{Chuyên Toán Lai Châu 2021}
\loigiai{
Đặt $A=2005^n+60^n-1897^n-168^n.$ Ta sẽ chứng minh $A$ chia hết cho $3,4$ và $167.$ Thật vậy
\begin{itemize}
    \item $A=(3\cdot668+1)^n+(3\cdot20)^n-(3\cdot632+1)^n-(3\cdot56)^n\equiv 1+0-1-0\equiv0\pmod{3}.$
    \item $A=(4\cdot501+1)^n+(4\cdot15)^n-(4\cdot474+1)^n-(4\cdot42)^n\equiv 1+0-1-0\equiv0\pmod{4}.$
    \item $A=(12\cdot167+1)^n+60^n-(11\cdot167+60)^n-(167+1)^n\equiv0\pmod{167}.$
\end{itemize}
Dựa vào các nhận xét trên, ta suy ra $A$ chia hết cho $[3,4,167]=2004.$ Bài toán được chứng minh.}
\end{gbtt} 

\begin{gbtt}
Chứng minh rằng với mọi số tự nhiên $n,$ ta luôn có \[5^{2 n-1} \cdot 2^{n+1}+3^{n+1} \cdot 2^{2 n-1} \text{ chia hết cho } 38.\]
\loigiai{
Đặt $A=5^{2 n-1} \cdot 2^{n+1}+3^{n+1} \cdot 2^{2 n-1}.$ Ta sẽ chứng minh $A$ chia hết cho $2$ và $19.$ Thật vậy
\begin{itemize}
    \item $A=2^{n+1}\tron{5^{2 n-1}+3^{n+1} \cdot 2^{n-2}}\equiv0\pmod{2}.$
    \item $A=2^{n+1}\tron{25^{n-2}\cdot5^3+3^{n+1} \cdot 2^{n-2}}=2^{n+1}\tron{6^{n-2}\cdot125+6^{n-2}\cdot3^3}=2^{n+1}\cdot6^{n-2}\cdot152\equiv0\pmod{19}.$
\end{itemize}
Dựa vào các nhận xét trên, ta suy ra $A$ chia hết cho $[2,19]=38.$ Bài toán được chứng minh.
}
\end{gbtt}

\begin{gbtt}
Với mỗi số nguyên dương $n$, ta đặt $$a_{n}=2^{2 n+1}-2^{n+1}+1, \quad b_{n}=2^{2 n+1}+2^{n+1}+1.$$ 
Chứng minh rằng với mọi $n$ thì $a_{n} b_{n}$ chia hết cho $5$ và $a_{n}+b_{n}$ không chia hết cho $5.$
\nguon{Chuyên Toán Phổ thông Năng khiếu 2003}
\loigiai{
\begin{enumerate}[a,]
    \item Biến đổi $a_nb_n$, ta thu được
    $$\tron{2^{2 n+1}-2^{n+1}+1}\tron{2^{2 n+1}+2^{n+1}+1}=\tron{2^{2 n+1}+1}^2-\tron{2^{n+1}}^2=2^{4n+2}+1.$$
    Xét hệ đồng dư modulo $5$, ta có 
    $$a_nb_n=2^{4n+2}+1\equiv16^n\cdot4+1\equiv4+1\equiv0\pmod{5}.$$
    \item Biến đổi $a_n+b_n$, ta thu được
    $$\tron{2^{2 n+1}-2^{n+1}+1}+\tron{2^{2 n+1}+2^{n+1}+1}= 2^{2n+2}+2= 4^{n+1}+2.$$
    Ta luôn có $4^a\equiv-1\pmod{5}$ và $4^a\equiv1\pmod{5}$ với $a$ là số tự nhiên. Do đó, ta thu được
    $$a_n+b_n=4^{n+1}+2\equiv3\pmod{5}.$$
    $$a_n+b_n=4^{n+1}+2\equiv1\pmod{5}.$$
\end{enumerate}
Như vậy, bài toán được chứng minh.}
\end{gbtt}

\begin{gbtt}
Chứng minh rằng $4^n-2019n-1$ chia hết cho $9$ với mọi số tự nhiên $n.$
\nguon{Chuyên Toán Quảng Ninh 2019}
\loigiai{
Xét trong modulo $9$, ta luôn có
$$4^n-2019n-1\equiv4^n-3n-1\pmod{9}.$$
Từ đây, ta xét các số dư của $n$ khi chia cho $3$. Ta có
\begin{enumerate}
    \item Nếu $n$ chia hết cho $3$, ta đặt $n=3k$ với $k$ là số tự nhiên. Ta nhận được
    $$4^{3k}-9k-1=64^k-9k-1\equiv1-0-1\equiv0\pmod{9}.$$
    \item Nếu $n$ chia cho $3$ dư $1$, ta đặt $n=3k+1$ với $k$ là số tự nhiên. Ta nhận được
    $$4^{3k+1}-3\tron{3k+1}-1=64^k\cdot4-9k-4\equiv4-0-4\equiv0\pmod{9}.$$
     \item Nếu $n$ chia cho $3$ dư $2$, ta đặt $n=3k+2$ với $k$ là số tự nhiên. Ta nhận được
    $$4^{3k+2}-3\tron{3k+2}-1=64^k\cdot4^2-9k-7\equiv16-0-7\equiv0\pmod{9}.$$
\end{enumerate}
Như vậy, $4^n-2019n-1$ chia hết cho $9$ với mọi số tự nhiên $n.$}
\end{gbtt}

\begin{gbtt}\label{cap.so.na}
Tìm số nguyên dương $n$ nhỏ nhất sao cho
\begin{multicols}{2}
\begin{enumerate}[a,]
    \item $5^n$ chia $7$ dư $1.$
    \item $11^n$ chia $13$ dư $1.$
    \item $2021^n$ chia $31$ dư $1.$
    \item $14^n$ chia $11$ dư $5.$
    \item $1048^n$ chia $23$ dư $4.$
    \item $227^n+194^n$ chia $11$ dư $8.$
\end{enumerate}
\end{multicols}
\loigiai{
\begin{enumerate}[a,]
    \item Từ nhận xét ở ví dụ 2, số nguyên dương $n$ cần tìm phải là ước của $7-1=6.$\\
    Theo đó, ta cần phải xét các trường hợp sau đây.
    \begin{itemize}
        \item Nếu $n=1,$ ta có
        $5^n=5^1 \equiv 5 \pmod{7},$ không thỏa.
        \item Nếu $n=2,$ ta có
        $5^n=5^2 \equiv 4 \pmod{7},$ không thỏa.
        \item Nếu $n=3,$ ta có
        $5^n=5^3 \equiv 6 \pmod{7},$ không thỏa.
    \end{itemize}
    Vậy $n=6$ là số nhỏ nhất để $5^n$ chia $7$ dư $1.$
    \item Từ nhận xét ở ví dụ 2, số nguyên dương $n$ cần tìm phải là ước của $13-1=12.$ \\Theo đó, ta cần phải xét các trường hợp sau đây.
    \begin{itemize}
        \item Nếu $n=1,$ ta có
        $11^n=11^1 \equiv 11 \pmod{13},$ không thỏa.
        \item Nếu $n=2,$ ta có
        $11^n=11^2 \equiv 4 \pmod{13},$ không thỏa.
        \item Nếu $n=3,$ ta có
        $11^n=11^3 \equiv 4\cdot11 \equiv 5 \pmod{13},$ không thỏa.
        \item Nếu $n=4,$ ta có
        $11^n=11^4 \equiv 5\cdot11 \cdot3 \pmod{13},$ không thỏa.
        \item Nếu $n=6,$ ta có
        $11^n=11^6 \equiv 3\cdot11 \equiv 6 \pmod{13},$ thỏa yêu cầu.
    \end{itemize}
    Vậy $n=6$ là số nhỏ nhất để $11^n$ chia $13$ dư $1.$
    \item Từ nhận xét ở ví dụ 2, số nguyên dương $n$ cần tìm phải là ước của $31-1=30.$ \\Theo đó, ta cần phải xét các trường hợp sau đây.
    \begin{itemize}
        \item Nếu $n=1,$ ta có
        $2021^n=2021^1 \equiv 6 \pmod{31},$ không thỏa.
        \item Nếu $n=2,$ ta có
        $2021^n=2021^2 \equiv 6^2 \equiv 5 \pmod{31},$ không thỏa.
        \item Nếu $n=3,$ ta có
        $2021^n=2021^3 \equiv 6^3 \equiv 5\cdot6 \equiv 30 \pmod{31},$ không thỏa.
        \item Nếu $n=5,$ ta có
        $2021^n=2021^5 \equiv 6^5 \equiv 5\cdot30 \equiv 26 \pmod{31},$ không thỏa.
        \item Nếu $n=6,$ ta có
        $2021^n=2021^6 \equiv 6^6 \equiv 26\cdot6 \equiv 1 \pmod{31},$ thỏa yêu cầu.
    \end{itemize}
    Vậy $n=6$ là số nhỏ nhất để $2021^n$ chia $31$ dư $1.$
    \item Ta biết được rằng $11^5\equiv 1\pmod{11}.$\\ Như vậy, nếu gọi $r$ là số dư của phép chia $n$ cho $5$ và đặt $n=5k+r,$ ta có
    $$14^{5k+r}=14^r\cdot\tron{14^5}^r\equiv 14^r\pmod{11}.$$
    Như vậy, số nguyên dương $n$ nhỏ nhất cần tìm chính là một trong những số dư của phép chia $n$ cho $5.$
    \begin{itemize}
        \item Nếu $n=1,$ ta có
        $14^n=14^1 \equiv 3 \pmod{11},$ không thỏa.
        \item Nếu $n=2,$ ta có
        $14^n=14^2 \equiv 3^2 \equiv 9 \pmod{11},$ không thỏa.
        \item Nếu $n=3,$ ta nhận được
        $14^n=14^3  \equiv 9\cdot3 \equiv 5 \pmod{17},$ thỏa yêu cầu. 
    \end{itemize}
    Vậy $n=3$ là số nhỏ nhất để $14^n$ chia $11$ dư $5.$
    \item Tương tự ý trên, số nguyên dương $n$ nhỏ nhất cần tìm chính là một trong những số dư của phép chia $n$ cho $11.$
    \begin{itemize} 
        \item Nếu $n=1,$ ta có
        $1048^n=1048^1 \equiv 13 \pmod{23},$ không thỏa.
        \item Nếu $n=2,$ ta có
        $1048^n=1048^2 \equiv 13^2 \equiv 8 \pmod{23},$ không thỏa.
        \item Nếu $n=3,$ ta có
        $1048^n=1048^3  \equiv 8\cdot13 \equiv 12 \pmod{23},$ không thỏa.
         \item Nếu $n=4,$ ta có
        $1048^n=1048^4 \equiv 12\cdot13 \equiv 18 \pmod{23},$ không thỏa.
        \item Nếu $n=5,$ ta có
        $1048^n=1048^5 \equiv 18\cdot13 \equiv 4 \pmod{23},$ thỏa yêu cầu.
    \end{itemize}
    Vậy $n=5$ là số nhỏ nhất để $1048^n$ chia $23$ dư $4.$
    \item Ta nhận thấy rằng
    $$227^n + 194^n=(220+7)^n+(187+7)^n \equiv 7^n + 7^n \equiv 2\cdot7^n \pmod{11}.$$
    Tương tự ý vừa rồi, số nguyên dương $n$ nhỏ nhất cần tìm chính là một trong những số dư của phép chia $n$ cho $10.$
    \begin{itemize} 
        \item Nếu $n=1,$ ta có
        $7^n=7^1 \equiv 7 \pmod{11},$ không thỏa.
        \item Nếu $n=2,$ ta có
        $7^n=7^2 \equiv 5 \pmod{11},$ không thỏa.
        \item Nếu $n=3,$ ta có
        $7^n=7^3  \equiv 2  \pmod{11},$ không thỏa.
        \item Nếu $n=4,$ ta có
        $7^n=7^4 \equiv 2\cdot7 \equiv 3 \pmod{11},$ không thỏa.
        \item Nếu $n=5,$ ta có
        $7^n=7^5 \equiv 3\cdot7 \equiv 10 \pmod{11}.$
        \item Nếu $n=6,$ ta có
        $7^n=7^6  \equiv 4  \pmod{11},$ thỏa yêu cầu.
        \end{itemize}
        Vậy $n=6$ là số nhỏ nhất để $227^n+194^n$ chia $11$ dư $8.$
\end{enumerate}
}
\end{gbtt}

\begin{gbtt}
Tìm số dư trong các phép chia sau
\begin{multicols}{2}
    \begin{enumerate}[a,]
        \item $3^{123}$ khi chia cho $7$.
        \item $7^{2021}$ khi chia cho $11$.
        \item $8^{227}$ khi chia cho $31$.
        \item $227^{111}$ khi chia cho $8$.
    \end{enumerate}
\end{multicols}
\loigiai{
\begin{enumerate}[a,]
    \item Ta có $3^3 \equiv -1 \pmod{7} \Rightarrow \tron{3^3}^{41} \equiv 3^{123} \equiv \tron{-1}^{41} \equiv -1 \equiv 6 \pmod{7}.$
    \item Ta có $7^{10} \equiv 1 \pmod{11}
        \Rightarrow \tron{7^{10}}^{202} \equiv 7^{2020} \equiv 1  \pmod{11}
        \Rightarrow 7^{2021} \equiv 1\cdot 7 \equiv 7 \pmod{11}.$
    \item Ta có
    $8^5 \equiv 4\cdot8 \equiv 32 \equiv 1 \pmod{31}\Rightarrow 8^{225} \equiv 1 \pmod{31}\Rightarrow 8^{227} \equiv 8^2 \equiv2 \pmod{31}.$
    \item Do $227\equiv 3\pmod{8}$ nên $227^{111} \equiv 3^{111} \pmod{8}.$ Ta nhận thấy rằng
    $$3^2 \equiv 1 \pmod{8} \Rightarrow 3^{110} \equiv 1 \pmod{8}\Rightarrow 3^{111} \equiv 3 \pmod{8}.$$
\end{enumerate}}
\end{gbtt}

\begin{gbtt}\
\begin{enumerate}[a,] 
    \item Tìm tất cả các số nguyên dương $n$ sao cho $2^{n}n+3^{n}$ chia hết cho $5.$
    \item Tìm tất cả các số nguyên dương $n$ sao cho $2^{n}n+3^{n}$ chia hết cho $25.$
\end{enumerate} 
\nguon{Chuyên Toán Phổ thông Năng khiếu 1997}
\loigiai{
\begin{enumerate}[a,]
    \item Xét hệ đồng dư modulo $5$, ta có 
    $$2^{n}n+3^{n}\equiv 2^{n}n+\tron{-2}^{n}\pmod{5}.$$
    Ta xét các trường hợp sau đây của $n$.
    \begin{itemize}
        \item \chu{Trường hợp 1.} Với $n$ chẵn, đặt $n=2x$, và ta nhận được 
        $$ 2^{2x}\cdot2x+\tron{-2}^{2x}=2^{2x}\cdot2x+2^{2x}=2^{2x}(2x+1)\equiv0 \pmod{5}.$$
        Vì $(2,5)=1$ nên $2x+1$ chia hết cho $5$. Từ đây, ta suy ra  
        $$2x=5y-1\Rightarrow 2\tron{x-2}=5\tron{y-1}\Rightarrow 5\mid \tron{x-2}\Rightarrow x=5k+2\Rightarrow n=10k+4.$$
       \item \chu{Trường hợp 2.} Với $n$ lẻ, đặt $n=2x+1$, và ta nhận được 
        $$ 2^{2x+1}\cdot(2x+1)+\tron{-2}^{2x+1}=2^{2x+1}\cdot(2x+1)-2^{2x+1}=2^{2x+1}\cdot2x\equiv0 \pmod{5}.$$
        Vì $(2,5)=1$ nên $2x$ chia hết cho $5$. Từ đây, ta suy ra $x=5k,$ và $n=10k+1.$
    \end{itemize}
    Như vậy, các số nguyên dương $n$ thỏa mãn đề bài là $n=10k+1$ hoặc $n=10k+4$ với $k$ là số tự nhiên. 
    \item Áp dụng kết quả ở câu a, ta thu được $n=10k+1$ hoặc $n=10k+4.$ Ta có nhận xét sau
    $$2^{10}\equiv -1\pmod{25},\qquad 3^{10}\equiv-1 \pmod{10}.$$
    Từ đây, ta xét $2$ trường hợp.
    \begin{itemize}
        \item \chu{Trường hợp 1.} Với $n=10k+1$ kết hợp với nhận xét, ta có
        \begin{align*}
            2^{10k+1}\tron{10k+1}+3^{10k+1}&\equiv \tron{-1}^k\tron{20k+2}+\tron{-1}^k3\\&\equiv \tron{-1}^k\tron{20k+5}\\&
            \equiv0\pmod{25}.
        \end{align*}
        Từ đây, ta suy ra $20k+5$ chia hết cho $25.$ Đặt $20k+5=25t$, ta thu được
        $$4k+1=5t\Rightarrow4(k-1)=5(t-1).$$
        Vì $\tron{4,5}=1$, ta lần lượt suy ra $$5\mid \tron{k-1}\Rightarrow k=5k+1 \Rightarrow n=50k+11.$$
         \item \chu{Trường hợp 2.} Với $n=10k+4$ kết hợp với nhận xét, ta có
         \begin{align*}
             2^{10k+4}\tron{10k+4}+3^{10k+4}&\equiv \tron{-1}^k\tron{160k+64}+\tron{-1}^k81\\&\equiv \tron{-1}^k\tron{160k+145}\\&
             \equiv0\pmod{25}.
         \end{align*}
        Từ đây, ta suy ra $160k+145$ chia hết cho $25.$ Đặt $160k+145=25t$, ta thu được
        $$32k-96=5t-125\Rightarrow32(k-3)=5(t-25).$$
        Vì $\tron{32,5}=1$, ta lần lượt suy ra $$5\mid \tron{k-3}\Rightarrow k=5k+3 \Rightarrow n=50k+34.$$
    \end{itemize}
     Như vậy, các số nguyên dương $n$ thỏa mãn là $n=50k+11$ hoặc $n=50k+34$ với $k$ là số tự nhiên. 
\end{enumerate}}
\end{gbtt}

\begin{gbtt}
Tìm số dư trong các phép chia sau
\begin{multicols}{2}
\begin{enumerate}[a,]
    \item $87^{32^{47}}$ khi chia cho $19.$
    \item $9^{8^7} + 5^{6^7}$ khi chia cho $13.$
\end{enumerate}
\end{multicols}
\loigiai{
\begin{enumerate}[a,]
\item Với việc $87\equiv 11\pmod{19},$ ta có
    $$87 \equiv 11 \pmod{19} \Rightarrow 87^{32^{47}} \equiv 11^{32^{47}} \pmod{19}.$$
    Ta xét số dư của $32^{47}$ cho $3$. Thật vậy
    $$32 \equiv -1 \pmod{3} \Rightarrow 32^{47} \equiv -1 \equiv 2 \pmod{3}.$$
    Theo đó, ta có thể đặt $32^{47} = 3k + 2$ với $k$ là số nguyên dương. Phép đặt này cho ta
    $$11^{32^{47}} \equiv 11^{3k+2} \equiv 121\cdot\tron{11^3}^k\equiv 7\pmod{19}.$$
    Kết luận, số dư của $87^{32^{47}}$ khi chia cho $19$ là $7.$
    \item Ta xét lần lượt số dư của các số hạng khi chia cho $13.$\\
     Từ $8^7 \equiv (-1)^7 \equiv 2 \pmod{3},$ ta có thể đặt $8^7=3k+2,$ với $k$ nguyên dương. Phép đặt cho ta
        $$9^{8^7} \equiv 9^{3k+2} \equiv 9^{3k} \cdot 9^2 \equiv 9^2 \equiv 3 \pmod{13}.$$
    Rõ ràng, $6^7$ chia hết cho $4.$ Ta đặt $6^7 = 4l$ với $l$ là số nguyên dương. Phép đặt này cho ta
        $$5^{6^7} \equiv 5^{4l} \equiv 1 \pmod{13}.$$
    Từ các lập luận trên, ta chỉ ra $9^{8^7} + 5^{6^7}\equiv 3+1\equiv 4\pmod{13}.$ Số đã cho chia $13$ dư $4.$
\end{enumerate}}
\end{gbtt}

\begin{gbtt}
Chứng minh rằng $2020^{2021^{2022}} + 2036$ chia hết cho $52$.
\loigiai{Trong bài toán này, do $52=4\cdot13,$ ta sẽ lần lượt đi tìm số dư của $A$ trong phép chia nó cho $4$ và cho $13.$\\
Ta có $A$ chia hết cho $4,$ thật vậy, điều này xảy ra là vì $2020$ và $2036$ đều chia hết cho $4.$ \\
Tiếp theo, ta tìm số dư của $A$ khi chia cho $13.$ Ta có 
        $$2021 \equiv 1 \pmod{4} \Rightarrow 2021^{2022} \equiv 1 \pmod{4}.$$
        Như vậy, ta có thể đặt $2021^{2022} = 4k + 1 $ với $k$ là số nguyên dương. Phép đặt này cho ta
        \begin{align*}
        2020^{2021^{2022}} + 2036 
        &\equiv 5^{2021^{2022}} + 8
        \\&\equiv 5^{2021^{2022}}+8 
        \\&\equiv 5^{4k+1}+8
        \\&\equiv 5^{4k}\cdot5+8 
        \\&\equiv1\cdot5+8 
        \\&\equiv 0 \pmod{13}
    \end{align*}
Số $A$ kể trên chia hết cho cả $4$ và $13,$ lại do $[4,13]=52$ nên bài toán được chứng minh.}    
\end{gbtt}

\begin{gbtt}
Chứng minh rằng $92^{183^{139}} + 183^{139^{92}} + 139^{92^{183}}$ chia hết cho $138$.
\loigiai{
 Trong bài toán này, do $52=2\cdot3\cdot23,$ ta sẽ lần lượt đi tìm số dư của $B$ trong phép chia nó cho $2,3$ và $23.$\\
Đầu tiên, $B$ chia hết cho $2$ vì rõ ràng $B$ là số chẵn.\\
Thứ hai, từ những đồng dư thức
        $$92\equiv -1\pmod{3},\quad 183\equiv 0\pmod{3},\quad 139\equiv 1\pmod{3},$$
        ta chỉ ra được rằng
        $$92^{183^{139}} + 183^{139^{92}} + 139^{92^{183}} \equiv 1^{183^{139}}  +(-1)^{92^{183}}\equiv 0 \pmod{3}.$$
 Cuối cùng, từ những đồng dư thức
        $$92\equiv 0\pmod{23},\quad 183\equiv -1\pmod{23},\quad 139\equiv 1\pmod{23},$$
        ta chỉ ra được rằng
        $$92^{183^{139}} + 183^{139^{92}} + 139^{92^{183}} \equiv (-1)^{139^{92}} + 1^{92^{183}}\equiv 0 \pmod{23}.$$
    Số $B$ kể trên chia hết cho cả $2,3$ và $23,$ lại do $[2,3,23]=138$ nên bài toán được chứng minh.}
\end{gbtt}

\begin{gbtt}
Chứng minh rằng với mọi số tự nhiên $n,$ ta luôn có
\[2^{3^{4n+1}}+3^{2^{4n+1}}+5\text{ chia hết cho }22.\]
\loigiai{
Do $22=2\cdot11$ nên ta sẽ chứng minh lần lượt $2^{3^{4n+1}}+3^{2^{4n+1}}+5$ chia hết cho 
$2$ và $11$.\\
Đầu tiên, ta dễ dàng chứng minh được $2^{3^{4n+1}}+3^{2^{4n+1}}+5$ chia hết cho $2$ vì rõ ràng nó là số chẵn. \\
Tiếp theo, xét trong hệ đồng dư modulo $11$, ta luôn có 
    $$2^{10}\equiv1\pmod{11}, \qquad 3^{5}\equiv 1\pmod{11}.$$
    Mặt khác, ta có các nhận xét sau
    $$\heva{3^{4n+1}\equiv81^n\cdot3\equiv3&\pmod{10}\\ 2^{4n+1}\equiv16^n\cdot2\equiv2&\pmod{5}.}$$
    Từ đây, ta đặt $3^{4n+1}=10x+3$ và $2^{4n+1}=5y+2$. Thay trở lại biểu thức đã cho, ta thu được
    $$2^{3^{4n+1}}+3^{2^{4n+1}}+5=2^{10x+3}+3^{5y+2}=\tron{2^{10}}^k\cdot8+\tron{3^{5}}^k\cdot9+5\equiv22\equiv0\pmod{11}.$$
Như vậy $2^{3^{4n+1}}+3^{2^{4n+1}}+5$ chia hết cho cả $2$ và $11$, lại do $\vuong{2,11}=22$ nên bài toán được chứng minh.
}
\end{gbtt}

\begin{gbtt}
Tìm chữ số tận cùng của $4^{2021} + 7^{2022} + 9^{2023}.$
\loigiai{Trong bài này, ta sẽ tìm chữ số tận cùng của từng số hạng.
    \begin{itemize}
        \item $4^2 \equiv 6 \pmod{10} \Rightarrow (4^2)^{1010} \equiv 6 \pmod{10} \Rightarrow 4^{2021} \equiv 4^1\equiv4\pmod{10}.$
    \item $7^4 \equiv 1 \pmod{10} \Rightarrow (7^4)^{505} \equiv 1 \pmod{10} \Rightarrow 7^{2022} \equiv 9\pmod{10}.$
    \item $9^2 \equiv 1 \pmod{10} \Rightarrow (9^2)^{1011} \equiv 1 \pmod{10} \Rightarrow 9^{2023} \equiv 9\pmod{10}.$
    \end{itemize}
    Dựa vào các lập luận trên, ta chỉ ra $4^{2021} + 7^{2022} + 9^{2023} \equiv 4 + 9 + 9\equiv2\pmod{10}.$ \\
    Như vậy, số đã cho có chữ số tận cùng là $2.$}
\end{gbtt}

\begin{gbtt}
Tìm hai chữ số tận cùng của các số sau
\begin{multicols}{3}
    \begin{enumerate}[a,]
    \item $6^{2021}$.
    \item $69^{2022}.$
    \item $15^{15^{15^{15}}}$.
\end{enumerate}
\end{multicols}
\loigiai{
\begin{enumerate}[a,]
    \item Dễ dàng chứng minh được $6^{2021}$ chia hết cho $4.$\\ 
    Ngoài ra, khi xét trong hệ đồng dư modulo $25,$ ta có
        $$6^5\equiv 1 \pmod{25} \Rightarrow 6^{2020} \equiv 1 \pmod{25} \Rightarrow 6^{2021} \equiv 6 \pmod{25}.$$
    Theo đó, ta có thể đặt $6^{2021} = 25n + 6=4m$ với $m,n$ là các số nguyên dương. Phép đặt này cho ta
    \begin{align*}
        4m = 25n + 6 
        \Rightarrow 4m + 44 = 25n + 50
        \Rightarrow 4\tron{m + 11} = 25\tron{n + 2}.
    \end{align*}
    Vì $\tron{25,4} = 1$ nên $\tron{n + 2}$ chia hết cho $4$, và ta lần lượt suy ra
    $$n\equiv 2\pmod{4}\Rightarrow 25n\equiv 50\pmod{100}\Rightarrow 25n+6\equiv 56\pmod{100}.$$
    Như vậy, số $6^{2021}$ có hai chữ số tận cùng là $56.$
    \item Dễ dàng chứng minh được $69^{2022}$ chia cho $4$ dư $1.$\\
    Ngoài ra, khi xét trong hệ đồng dư modulo $25$, ta nhận thấy rằng
    $$69^{2002} \equiv 14^{2002}=(14^5)^{400}\cdot14^2\equiv 14^2\equiv21\pmod{25}.$$
    Ta có thể đặt $69^{2002} = 25n + 21=4m+1$ với $m,n$ là các số nguyên dương. Ta có
    \begin{align*}
        4m + 1 = 25n + 21\Rightarrow 4m - 20 = 25n\Rightarrow 4(m - 5) = 25n.
    \end{align*}
    Vì  $\tron{4, 25} = 1$ nên $n$ chia hết cho $4$, và ta lần lượt suy ra
    $$n\equiv 0\pmod{4}\Rightarrow 25n\equiv 0\pmod{100}\Rightarrow 25n+21\equiv 21\pmod{100}.$$
    Như vậy, số $69^{2022}$ có hai chữ số tận cùng là $21.$
    \item Khi xét trong hệ đồng dư modulo $4$, ta nhận thấy rằng
    $$15\equiv -1 \pmod{4} \Rightarrow 15^{15^{15^{15}}} \equiv - 1 \pmod{4}. $$
    Ngoài ra, khi xét trong hệ đồng dư modulo $25$, ta nhận thấy rằng
    $$15^2 \equiv 0 \pmod{25} \Rightarrow 15^{15^{15^{15}}} \equiv 0 \pmod{25}.$$
    Như vậy, ta có thể đặt $15^{15^{15^{15}}}  = 25n=4m+1,$ trong đó $m,n$ là các số nguyên dương. Ta có
    \begin{align*}
        4m - 1 = 25n
        \Rightarrow 4m - 76 &= 25n - 7
        \Rightarrow 4\tron{m - 19} = 25\tron{n - 3}.
    \end{align*}
    Vì $\tron{4, 25} = 1$ nên $\tron{n - 3}$ chia hết cho $4$, và ta lần lượt suy ra
    $$n\equiv 3\pmod{4}\Rightarrow 25n\equiv 75\pmod{4}.$$
    Như vậy, số $15^{15^{15^{15}}}$ có hai chữ số tận cùng là $75.$
\end{enumerate}}
\end{gbtt}

\begin{gbtt}
Tìm ba chữ số tận cùng của $A=3\cdot9\cdot15\cdots2025$.
\loigiai{
Trong bài toán này, ta sẽ xét số dư của $A$ khi chia cho $8$ và khi chia cho $125.$
\begin{enumerate}
    \item Trong $A$ chứa các thừa số là $15,45,75.$ Tích ba thừa số là bội của $5$ này chia hết cho $125,$ kéo theo $A$ cũng chia hết cho $125.$
    \item Ta xét số dư khi chia cho $8$ của tích một cặp hai số có dạng $(12n+3,12n+9)$ trong $A.$ Ta có 
    $$\tron{12n + 3}\tron{12n + 9} = 144n^2 + 144n + 27 \equiv 3 \pmod{8}.$$
    Bằng cách này, khi xét $A$ trong hệ đồng dư modulo $8,$ ta được
    $$A\equiv (3\cdot9)(15\cdot21)\ldots(2019\cdot2025)\equiv 3\cdot3\ldots3=3^{169}\pmod{8}.$$
\end{enumerate}
Dựa vào hai lập luận kể trên, ta có thể đặt $A=125m=8n+3,$ trong đó $m,n$ là các số nguyên dương. Phép đặt này cho ta
\begin{align*}
    125m= 8n + 3
   \Rightarrow 125m + 125= 8n + 128
   \Rightarrow 125\tron{m + 1}= 8\tron{n + 16}.
\end{align*}
Vì $\tron{8, 125} = 1$ nên $\tron{m + 1}$ chia hết cho $8$. Ta lần lượt suy ra
$$m\equiv 7\pmod{8}\Rightarrow 125m\equiv 875\pmod{1000}.$$
Như vậy, ba chữ số tận cùng của tích $A=3\cdot9\cdot15\ldots2025$ là $875.$}
\end{gbtt}

\section{Một số bổ đề đồng dư thức với số mũ nhỏ}

%chỗ này để viết lời nói đầu

\subsection*{Mở đầu}

Trước khi nói đến lí thuyết, ta sẽ xem xét một vài ví dụ.

\begin{bx}
Cho số nguyên $n$. Chứng minh rằng số dư của phép chia $n^2$ cho $3$ không thể bằng $2.$
\loigiai{
Mọi số nguyên $n$ bất kì chỉ có thể biểu diễn dưới một trong hai dạng, hoặc là $n=3k,$ hoặc là $n=3k\pm 1.$ Ta xét các trường hợp kể trên.
\begin{enumerate}
    \item Với $n=3k,$ ta có $n^2=(3k)^2=9k^2$ chia hết cho $3.$
    \item Với $n=3k\pm 1,$ ta có $n^2=(3k\pm 1)^2=9k^2\pm 6k+1$ chia cho $3$ dư $1.$
\end{enumerate}
Dựa theo các lập luận trên, ta chỉ ra $n^2\equiv 0,1\pmod{3},$ tức $n^2$ không thể nhận số dư là $2$ khi chia cho $3.$ Bài toán được chứng minh.
}
\begin{luuy}
Ngoài cách làm kể trên, ta có thể thực hiện xét bảng đồng dư theo modulo $3$ cho $n$ và $n^2$ như sau
\begin{center}
    \begin{tabular}{c|c|c}
        $n$ &  $0$ & $\pm 1$\\
        \hline
        $n^2$ & $0$ & $1$
    \end{tabular}
\end{center}
Cơ sở để xây dựng bảng đồng dư trên chính là các lập luận
\begin{enumerate}
    \item $n\equiv 0\pmod{3}\Rightarrow n^2\equiv 0\pmod{3}.$
    \item $n\equiv \pm 1\pmod{3}\Rightarrow  n^2\equiv (\pm 1)^2\equiv 1\pmod{3}.$
\end{enumerate}
\end{luuy}
\end{bx}

\begin{bx}
Cho số nguyên $n.$ Tìm tất cả các số dư có thể của $n^2$ khi đem chia cho $7.$
\loigiai{
Trong bài toán này, ta xét bảng đồng dư theo modulo $7$ như sau
\begin{center}
    \begin{tabular}{c|c|c|c|c}
        $n$ &  $0$ & $\pm 1$ & $\pm 2$ & $\pm 3$\\
        \hline
        $n^2$ & $0$ & $1$ & $4$ & $2$
    \end{tabular}
\end{center}
Như vậy, $n^2\equiv 0,1,2,4\pmod{7}.$ Nói cách khác, các số dư có thể của $n^2$ khi chia cho $7$ là $0,1,2$ và $4.$}
\end{bx}

\begin{bx}
Cho $n$ là một số nguyên. Tìm tất cả các số dư có thể của $n^3$ khi đem chia cho $9.$
\loigiai{
Trong bài toán này, ta xét bảng đồng dư theo modulo $9$ như sau
\begin{center}
    \begin{tabular}{c|c|c|c|c|c|c|c|c|c}
        $n$ & $0$ & $1$ & $2$ & $3$ & $4$ & $5$ & $6$ & $7$ & $8$\\
        \hline
        $n^3$ & $0$ & $1$ & $8$ & $0$ & $1$ & $8$ & $0$ & $1$ & $8$
    \end{tabular}
\end{center}
Như vậy, $n^3\equiv 0,1,8\pmod{9}.$ Nói cách khác, các số dư có thể của $n^3$ khi chia cho $9$ là $0,1,8.$}
\end{bx}

Bằng cách làm tương tự các bài toán vừa rồi, ta thu được một vài kết quả đồng dư sau.

\subsection*{Lí thuyết}

Với mọi số nguyên $n,$ ta có

\begin{multicols}{2}
\begin{enumerate}
    \item $n^2\equiv 0,1\pmod{3}.$
    \item $n^2\equiv 0,1\pmod{4}.$
    \item $n^2\equiv 0,1,4\pmod{5}.$   
    \item $n^2\equiv 0,1,2,4\pmod{7}.$    
    \item $n^2\equiv 0,1,4\pmod{8}.$    
    \item $n^3\equiv 0,1,-1\pmod{7}.$ 
    \item $n^3\equiv 0,1,-1\pmod{9}.$    
    \item $n^4\equiv 0,1\pmod{5}.$    
    \item $n^4\equiv 0,1\pmod{16}.$    
    \item $n^5\equiv 0,1\pmod{11}.$         
\end{enumerate}
\end{multicols}

Lí thuyết trên trải dài trên cả cuốn sách. Các bổ đề về đồng dư này giúp chúng ta thuận lợi hơn trong các bài toán xét số dư. Dưới đây là một vài ví dụ minh họa.

\subsection*{Ví dụ minh họa}

\begin{bx}
Cho $n$ là số nguyên dương nguyên tố cùng nhau với $10.$
\\Chứng minh rằng $n^4-1$ chia hết cho $40.$
\nguon{Chuyên Toán Hà Nội}
\loigiai{
Ta thực hiện các bước làm sau đối với bài toán này.
\begin{enumerate}[\color{tuancolor}\bf\sffamily Bước 1.]
    \item Chứng minh $n^4-1$ chia hết cho $5.$\\
    Ta đã biết $n^2\equiv 0,1,-1\pmod{5},$ nhưng do $(n,5)=1$ nên 
    $$n^2\equiv -1,1\pmod{5}.$$
    Lấy bình phương hai vế, ta thu được $n^4\equiv 1\pmod{5},$ tức là $n^4-1$ chia hết cho $5.$
    \item Chứng minh $n^4-1$ chia hết cho $8.$ \\  
    Ta đã biết $n^2\equiv 0,1,4\pmod{8},$ nhưng do $(n,2)=1$ nên 
    $$n^2\equiv 1\pmod{8}.$$
    Lấy bình phương hai vế, ta thu được $n^4\equiv 1\pmod{8},$ tức là $n^4-1$ chia hết cho $8.$  
\end{enumerate}
Hai nhận xét trên cho ta biết $n$ chia hết cho $[5,8]=40.$ Bài toán được chứng minh.}
\end{bx}

\begin{bx}\label{fermatcuamu.7}
Chứng minh rằng với mọi số nguyên dương $n,$ ta có $n^7-n$ chia hết cho $42.$
\loigiai{
Ta chia bài toán trên thành các bước làm sau.
\begin{enumerate}[\color{tuancolor}\bf\sffamily Bước 1.]
    \item Ta chứng minh $n^7-n$ chia hết cho $6.$ \\
    Ta có $n^7-n=n(n-1)(n+1)\left(n^2-n+1\right)\left(n^2+n+1\right).$\\
    Với nhận xét thu được là $n^7-n$ chia hết cho $n(n-1)(n+1)$ là tích ba số tự nhiên liên tiếp, ta chỉ ra $n^7-n$ chia hết cho $6.$
    \item Ta chứng minh $n^7-n$ chia hết cho $7.$\\
    Ta có $n^7-n=n\left(n^3-1\right)\left(n^3+1\right).$\\
    Ta đã biết, $n^3\equiv -1,0,1\pmod{7}.$ Ta xét các trường hợp trên.
    \begin{itemize}
        \item \chu{Trường hợp 1.} Nếu $n^3\equiv -1\pmod{7},$ ta có $n^3+1$ chia hết cho $7.$
        \item \chu{Trường hợp 2.} Nếu $n^3\equiv 1\pmod{7},$ ta có $n^3-1$ chia hết cho $7.$     
        \item \chu{Trường hợp 3.} Nếu $n^3\equiv 0\pmod{7},$ do $7$ là số nguyên tố nên $n$ chia hết cho $7.$ 
    \end{itemize}
\end{enumerate}
Dựa theo các bước làm trên, ta có $n^7-n$ chia hết cho $[6,7]=42.$ Chứng minh hoàn tất.}
\begin{luuy}
Bài toán trên có thể được giải quyết bằng phương pháp quy nạp ở phần phụ lục, hoặc phương pháp phân tích đa thức thành nhân tử đã học ở mục trước. Tuy nhiên, bổ đề về đồng dư cho ta những cách nhìn mới hơn về bài toán. Cách chứng minh y hệt cho ta một vài kết quả tương tự, chẳng hạn như
\begin{multicols}{2}
\begin{enumerate}
    \item $n^3\equiv n\pmod{6}.$
    \item $n^5\equiv n\pmod{30}.$
\end{enumerate}
\end{multicols}
Tổng quát cho bài toán ở trên chính là định lí $Fermat$ nhỏ:
\begin{quote}
\it
"Với mọi số nguyên dương $a$ và số nguyên tố $p,$ ta có
$a^p\equiv a\pmod{p}$".
\end{quote}
\end{luuy}
\end{bx}

\begin{bx}
Cho các số nguyên $x,y,z$ thỏa mãn 
\[\dfrac{x^2-1}{2}=\dfrac{y^2-1}{3}=z.\]
Chứng minh rằng $z$ chia hết cho $40.$
\nguon{Chuyên Khoa học Tự nhiên 2016}
\loigiai{
Với các số $x,y,z$ thỏa mãn giả thiết, ta có
$$2z+1=x^2,\quad 3z+1=y^2.$$
Trong bài toán này, ta sẽ chứng minh rằng $z$ chia hết cho cả $5$ và $8.$
\begin{enumerate}[\color{tuancolor}\bf\sffamily Bước 1.]
    \item Ta chứng minh $z$ chia hết cho $8.$\\
    Từ giả thiết, ta nhận thấy $x^2-1$ chia hết cho $2,$ vậy nên $x$ là số lẻ. Ta có
    $$x^2\equiv 1\pmod{4}\Rightarrow 2z+1\equiv 1\pmod{4}\Rightarrow z\equiv 0\pmod{2}.$$
    Lập luận trên cho ta $z$ là số chẵn, kéo theo $y$ là số lẻ. Ta đã biết 
    $$x^2\equiv 0,1,4\pmod{8},\quad y^2\equiv 0,1,4\pmod{8}.$$
    Do $x,y$ là hai số lẻ, ta có $x^2-y^2\equiv 1-1\equiv 0\pmod{8},$ và như vậy, $z$ chia hết cho $8.$
    \item Ta chứng minh $z$ chia hết cho $5.$ \\
    Cộng theo vế hai đẳng thức $2z+1=x^2$ và $3z+1=y^2,$ ta nhận thấy
    $$5z+2=x^2+y^2.$$
    Ta suy ra $x^2+y^2\equiv 2\pmod{5}.$ Mặt khác, ta xét bảng đồng dư modulo $5$ sau đây
    \begin{center}
        \begin{tabular}{c|c|c|c|c|c|c|c|c|c}
            $x^2$ & $0$ & $0$ & $0$ & $1$ & $1$ & $1$ & $4$ & $4$ & $4$ \\
            \hline
            $y$ & $0$ & $1$ & $4$ & $0$ & $1$ & $4$& $0$ & $1$ & $4$ \\
            \hline 
            $x^2+y^2$ & $0$ & $1$ & $4$ & $1$ & $2$ & $0$ & $4$ & $0$ & $3$
        \end{tabular}
    \end{center}
    Bảng trên cho ta biết, chỉ trường hợp $x^2\equiv y^2\equiv 1\pmod{5}$ là có thể xảy ra. Như vậy
    $$x^2-y^2\equiv 0\pmod{5}\Rightarrow z\equiv 0\pmod{5}.$$
    Ta thu được $z$ chia hết cho $5$ từ đây.
\end{enumerate}
Hai nhận xét trên cho ta biết $z$ chia hết cho $[5,8]=40.$ Bài toán được chứng minh.}
\end{bx}

\subsection*{Bài tập tự luyện}

\begin{btt}
Tìm tất cả các số tự nhiên $n$ thỏa mãn $A=n\left(n^2+1\right)\left(n^2+4\right)$ chia hết cho $120.$
\end{btt}

\begin{btt}
Cho các số nguyên dương $x,y,z$ thỏa mãn 
\[x^2+y^2=z^2.\]
Chứng minh rằng $xyz$ chia hết cho $60.$
\end{btt}

\begin{btt}
Cho các số nguyên $x,y,z$ thỏa mãn $x^2+y^2+z^2=2xyz.$ \\Chứng minh rằng $xyz$ chia hết cho $24.$
\nguon{Chuyên Toán Vĩnh Phúc 2021}
\end{btt}

\begin{btt}
Cho số nguyên dương $n>2.$ Chứng minh rằng
\begin{enumerate}[a,]
    \item $A=n^3-3n^2+2n$ chia hết cho $6.$
    \item $B=n^{A+1}-1$ chia hết cho $7.$
\end{enumerate}
\nguon{Chuyên Toán Tuyên Quang 2021}
\end{btt}

\begin{btt}
Cho ba số nguyên dương $a,b,c$ thỏa mãn $a^3+b^3+c^3$ chia hết cho $14.$ Chứng minh rằng $abc$ cũng chia hết cho $14.$
\nguon{Chuyên Đại học Sư phạm Hà Nội 2019}
\end{btt}

\begin{btt}
Cho $a,b$ là các số nguyên dương. Ta đặt
$$M=a^2+ab+b^2.$$
Biết rằng chữ số hàng đơn vị của $M$ là $0.$ Tìm chữ số hàng chục của $M$.
\nguon{Chuyên Toán Hồ Chí Minh 2014}
\end{btt}

\begin{btt}
Tìm tất cả các bộ ba số nguyên $(a, b, c)$ sao cho số $$\dfrac{(a-b)(b-c)(c-a)}{2}+2$$ là một lũy thừa đúng của $2018^{2019}.$
\nguon{Junior Balkan Mathematical Olympiad Shortlist 2016}
\end{btt}

\begin{btt}
Cho các số nguyên $a_1,a_2,\ldots,a_n$. Ta đặt 
$$A=a_1+a_2+\ldots+a_n,\qquad B=a_1^3+a_2^3+\ldots+a_n^3.$$ 
Chứng minh rằng $A$ chia hết cho $6$ khi và chỉ khi $B$ chia hết cho $6.$
\end{btt}

\begin{btt}
Chứng minh rằng với mọi số nguyên dương $n$, ta luôn có
$$\left[(27 n+5)^{7}+10\right]^{7}+\left[(10 n+27)^{7}+5\right]^{7}+\left[(5 n+10)^{7}+27\right]^{7}.$$
chia hết cho $42.$
\nguon{Chuyên Khoa học Tự nhiên 2019}
\end{btt}

\subsection*{Hướng dẫn bài tập tự luyện}

\begin{gbtt}
Tìm tất cả các số tự nhiên $n$ thỏa mãn $A=n\left(n^2+1\right)\left(n^2+4\right)$ chia hết cho $120.$
\loigiai{
Ta sẽ tìm điều kiện của $n$ sao cho $A$ chia hết cho $3,5$ và $8.$
\begin{enumerate}[a,]
    \item Xét trong hệ đồng dư modulo $3,$ ta có
    $$A\equiv n\left(n^2+1\right)^2\pmod{3}.$$
    Ta đã biết $n^2+1\equiv 1,2\pmod{3}.$ Kết quả trên cho ta $\left(n^2+1,3\right)=1,$ và vì thế, $n$ chia hết cho $3.$
    \item Xét trong hệ đồng dư modulo $5,$ ta có
    \begin{align*}
    A&=n\left[(n-2)(n+2)+5\right]\left[(n-1)(n+1)+5\right]\\&\equiv n(n-2)(n+2)(n-1)(n+1) \pmod{5}.
    \end{align*}
    Bên trên là tích của $5$ số nguyên liên tiếp, và vì thế, $A$ chia hết cho $5.$ 
    \item Xét trong hệ đồng dư modulo $8,$ ta nhận thấy
    \begin{itemize}
        \item\chu{Trường hợp 1.} Với $n$ lẻ, ta có $n\left(n^2+4\right)$ lẻ, còn $$n^2+1\equiv 2\pmod{4},$$ thế nên $A\equiv 2,6\pmod{8},$ và $A$ không chia hết cho $8.$
        \item\chu{Trường hợp 2.} Với $n$ chẵn, ta có $n\left(n^2+4\right)$ chia hết cho $8$, thế nên $A$ chia hết cho $8$.
    \end{itemize}
\end{enumerate}
Kết hợp các nhận xét bên trên, ta kết luận tất cả các số tự nhiên $n$ cần tìm là các bội tự nhiên của $6.$}
\end{gbtt}

\begin{gbtt}
Cho các số nguyên dương $x,y,z$ thỏa mãn 
\[x^2+y^2=z^2.\]
Chứng minh rằng $xyz$ chia hết cho $60.$
\loigiai{
Trong bài toán này, ta sẽ chứng minh rằng $xyz$ chia hết cho cả $3,4$ và $5$.
\begin{enumerate}[\color{tuancolor}\bf\sffamily Bước 1.]
    \item Ta chứng minh $xyz$ chia hết cho $3.$\\
    Giả sử $x,y,z$ không chia hết cho $3$. Ta luôn có với $a$ là số nguyên dương thì $a^2\equiv0,1\pmod{3}$.\\ Kết hợp nhận xét trên và điều giả sử, ta nhận được
    $$x^2\equiv1\pmod{3},\quad y^2\equiv 1\pmod{3}, \quad z^2\equiv1\pmod{3}.$$
    Thế vào giả thiết, ta thu được
    $$z^2=x^2+y^2\equiv 1+1\equiv2\pmod{3}.$$
    Điều này mâu thuẫn với điều ta đã chứng minh phía trên. \\Do đó, giả sử sai nên trong $x,y,z$ có một số chia hết cho $3$ hay $xyz$ chia hết cho $3$.
    \item Ta chứng minh $xyz$ chia hết cho $4.$\\
    Giả sử $x,y,z$ không chia hết cho $4$. Ta luôn có với $a$ là số nguyên dương thì $a^2\equiv0,1\pmod{4}$.\\
    Kết hợp nhận xét trên và điều giả sử, ta nhận được
    $$x^2\equiv1\pmod{4},\quad y^2\equiv 1\pmod{4}, \quad z^2\equiv1\pmod{4}.$$
    Thế vào giả thiết, ta thu được
    $$z^2=x^2+y^2\equiv 1+1\equiv2\pmod{4}.$$
    Điều này mâu thuẫn với điều ta đã chứng minh phía trên.\\ Do đó, giả sử sai nên trong $x,y,z$ có một số chia hết cho $4$ hay $xyz$ chia hết cho $4$.
    \item Ta chứng minh $xyz$ chia hết cho $5.$\\
    Giả sử $x,y,z$ không chia hết cho $5$. Ta luôn có với $a$ là số nguyên dương thì $a^2\equiv0,1,4\pmod{5}$.\\
    Kết hợp nhận xét trên và điều giả sử, ta nhận được
    $$x^2\equiv1,4\pmod{5},\quad y^2\equiv 1,4\pmod{5}, \quad z^2\equiv1,4\pmod{5}.$$
    Ta xét bảng đồng dư modulo $5$ sau đây
    \begin{center}
        \begin{tabular}{c|c|c|c|c}
            $x^2$ & $1$ & $1$ & $4$ & $4$  \\
            \hline
            $y^2$ & $1$ & $4$ & $1$ & $4$\\
            \hline           $z^2$ & $2$ & $0$ & $0$ & $3$\\
        \end{tabular}
    \end{center}
    Điều này mâu thuẫn với điều ta đã chứng minh phía trên. \\Do đó, giả sử sai nên trong $x,y,z$ có một số chia hết cho $5$ hay $xyz$ chia hết cho $5$.
\end{enumerate}
Ba nhận xét trên cho ta biết $xyz$ chia hết cho $\vuong{3,4,5}=60$. Bài toán được chứng minh.}
\end{gbtt}

\begin{gbtt}
Cho các số nguyên $x,y,z$ thỏa mãn $x^2+y^2+z^2=2xyz.$ \\Chứng minh rằng $xyz$ chia hết cho $24.$
\nguon{Chuyên Toán Vĩnh Phúc 2021}
\loigiai{
Ta thực hiện các bước làm sau đối với bài toán này.
\begin{enumerate}[\color{tuancolor}\bf\sffamily Bước 1.]
    \item Ta chứng minh $xyz$ chia hết cho $8.$ \\
    Từ giả thiết, ta suy ra $x^2+y^2+z^2$ là số chẵn, thế nên trong $x,y,z$ có $1$ hoặc $3$ số chẵn. Không mất tính tổng quát, ta xét các trường hợp sau.
    \begin{itemize}
        \item \chu{Trường hợp 1.} Nếu $x$ là số chẵn, $y,z$ là số lẻ, ta có
        \[xyz\equiv 0\pmod{4}.\tag{1}\label{vp1}\]
        Ta cũng biết rằng một số chẵn dạng $a^2$ khi chia cho $8$ được dư là $0$ hoặc $4,$ trong khi số dư của một số lẻ dạng $a^2$ là $1.$ Lập luận này chỉ ra cho ta
        \[x^2+y^2+z^2\equiv 2,6\pmod{8}.\tag{2}\label{vp2}\]
        Đối chiếu (\ref{vp1}) và (\ref{vp2}), ta nhận thấy điều mâu thuẫn. Trường hợp này không tồn tại.
        \item \chu{Trường hợp 2.} Nếu $x,y,z$ đều là số chẵn, hiển nhiên $xyz$ chia hết cho $8.$
    \end{itemize}    
    \item Ta chứng minh $xyz$ chia hết cho $3.$ \\
    Giả sử phản chứng rằng $x,y,z$ không chia hết cho $3,$ lúc này
    $$x^2+y^2+z^2\equiv 1+1+1\equiv 0\pmod{3}.$$
    Suy ra $xyz$ chia hết cho $3,$ tức một trong ba số $x,y,z$ chia hết cho $3,$ mâu thuẫn với giả sử.\\ Giả sử sai, và một trong ba số $x,y,z$ chia hết cho $3.$     
\end{enumerate}
Như vậy, $xyz$ chia hết cho $[3,8]=24.$ Bài toán được chứng minh.}
\end{gbtt}

\begin{gbtt} Cho số nguyên dương $n>2.$ Chứng minh rằng
\begin{enumerate}[a,]
    \item $A=n^3-3n^2+2n$ chia hết cho $6.$
    \item $B=n^{A+1}-1$ chia hết cho $7.$
\end{enumerate}
\nguon{Chuyên Toán Tuyên Quang 2021}
\loigiai{
\begin{enumerate}[a,]
    \item Ta viết lại biểu thức $A$ như sau: $$A=n\left(n^2-3n+2\right)=n(n-1)(n-2).$$ 
    Ta có $A$ là tích ba số nguyên dương liên tiếp, và vì thế, $A$ chia hết cho $6.$
    \item Theo như kết quả trên, ta có thể đặt $A=6m,$ với $m$ là số nguyên dương. Ta xét các trường hợp sau.
    \begin{itemize}
        \item \chu{Trường hợp 1.} Nếu $n$ chia hết cho $7,$ hiển nhiên $B$ cũng chia hết cho $7.$
        \item \chu{Trường hợp 2.} Nếu $n$ không chia hết cho $7,$ ta suy ra $n^3\equiv 1,-1\pmod{7},$ và khi ấy
        \begin{align*}
            n^6\equiv 1\pmod{7}
            &\Rightarrow n^{6m}\equiv 1\pmod{7}\\&\Rightarrow n^{6m+1}\equiv n\pmod{7}\\&
            \Rightarrow 7\mid\left(n^{A+1}-n\right).
        \end{align*}
    \end{itemize}  
    Như vậy, bài toán được chứng minh trong mọi trường hợp.
\end{enumerate}}
\end{gbtt}

\begin{gbtt}
Cho ba số nguyên dương $a,b,c$ thỏa mãn $a^3+b^3+c^3$ chia hết cho $14.$ Chứng minh rằng $abc$ cũng chia hết cho $14.$
\nguon{Chuyên Đại học Sư phạm Hà Nội 2019}
\loigiai{
Ta chia bài toán thành các bước làm sau đây.
\begin{enumerate}[\color{tuancolor}\bf\sffamily Bước 1.]
    \item Ta chứng minh $abc$ chẵn. \\
    Nếu cả $a,b,c$ cùng lẻ thì $a^3+b^3+c^3$ lẻ và không thể chia hết cho $14,$ mâu thuẫn.\\ Do vậy một trong ba số $a,b,c$ chẵn, và tích $abc$ cũng vậy.
    \item Ta chứng minh $abc$ chia hết cho $7$. \\
    Phản chứng, giả sử $abc$ không chia hết cho $7.$ Lúc này
    $$a^3,\ b^3,\ c^3\equiv -1,1\pmod{7}.$$
    Ta suy ra $a^3+b^3+c^3\equiv -3,-1,1,3\pmod{7},$ trái giả thiết. Phản chứng là sai, và $7\mid abc.$
\end{enumerate}
Như vậy, $abc$ chia hết cho $[2,7]=14.$ Bài toán được chứng minh.}
\end{gbtt}

\begin{gbtt}
Cho $a,b$ là các số nguyên dương. Ta đặt
$$M=a^2+ab+b^2.$$
Biết rằng chữ số hàng đơn vị của $M$ là $0.$ Tìm chữ số hàng chục của $M$.
\nguon{Chuyên Toán Hồ Chí Minh 2014}
\loigiai{
Ta dự đoán rằng $M$ chia hết cho $100.$ Ta tiến hành bài toán theo các bước làm sau.
\begin{enumerate}[\color{tuancolor}\bf\sffamily Bước 1.]
    \item Chứng minh $M$ chia hết cho $4.$ \\
    Trong bước này, trước hết ta sẽ lập bảng đồng thư theo modulo $2$ của $M.$
    \begin{center}
        \begin{tabular}{c|c|c|c|c}
           $a$  & lẻ & lẻ & chẵn & chẵn \\
           \hline
            $b$ & lẻ & chẵn & lẻ & chẵn \\
            \hline
            $a^2+ab+b^2$ & lẻ & lẻ & lẻ & chẵn
        \end{tabular}
    \end{center}
    Do $M$ có tận cùng là $0$ nên $M$ chẵn. Dựa vào bảng, ta chỉ ra cả $a$ và $b$ đều chẵn. Vì thế
    $$4\mid a^2,\quad 4\mid b^2,\quad 4\mid ab.$$
    Ta suy ra $M$ chia hết cho $4$ từ đây.
    \item Chứng minh $M$ chia hết cho $25.$\\
    Do $M$ có tận cùng là $0$ nên 
    $$4M=4\tron{a^2+ab+b^2}=(2a+b)^2+3b^2$$
    chia hết cho $5.$ Từ đây, ta lập được bảng đồng dư theo modulo $5$
    \begin{center}
        \begin{tabular}{c|c|c|c}
           $b^2$  &  $0$ & $1$ & $4$\\
           \hline
            $3b^2$ &  $0$ & $3$ & $2$\\
            \hline
            $(2a+b)^2$ &  $0$ & $2$ & $3$\\
        \end{tabular}
    \end{center}
    Do một số chính phương không thể chia $5$ dư $2$ hoặc $3$ nên từ bảng trên, ta có
    $$(2a+b)^2\equiv 0\pmod{5}.$$
    Ta suy ra $2a+b$ chia hết cho $5.$ Kết hợp với $5\mid M,$ ta được $5\mid 3b^2$ hay $5\mid b,$ kéo theo $5\mid a.$ Vậy
    $$25\mid a^2,\quad 25\mid ab,\quad 25\mid b^2.$$
    Ta suy ra $M$ chia hết cho $25$ từ đây.
\end{enumerate}
Các kết quả trên cho ta biết $M$ chia hết cho $[4,25]=100.$ Chữ số hàng chục của $M$ là $0.$}
\end{gbtt}

\begin{gbtt}
Tìm tất cả các bộ ba số nguyên $(a, b, c)$ sao cho số $$\dfrac{(a-b)(b-c)(c-a)}{2}+2$$ là một lũy thừa đúng của $2018^{2019}.$
\nguon{Junior Balkan Mathematical Olympiad Shortlist 2016}
\loigiai{
Già sử $a, b, c$ là các số nguyên và $n$ là một số nguyên dương sao cho
$$
(a-b)(b-c)(c-a)+4=2\cdot 2018^{2019 n}.
$$
Đặt $a-b=-x,b-c=-y.$ Ta có
\[x y(x+y)+4=2\cdot2018^{2019 n}.\tag{1}\label{2018^2019}\]
Với $n \geq 1,$ vế phải của (\ref{2018^2019}) chia hết cho $7,$ vì thế
\[x y(x+y)+4 \equiv 0\pmod{7}.\tag{2}\label{2018^2019.1}\]
Dựa theo hằng đẳng thức $(x+y)^3-x^3-y^3=3xy(x+y),$ ta chỉ ra được rằng
$$(x+y)^{3}-x^{3}-y^{3}\equiv 2\pmod{7}.$$
Với mọi số nguyên dương $k,$ ta luôn có $k^3\equiv -1,0,1\pmod{7},$ vì thế nếu $$(x+y)^{3}-x^{3}-y^{3} \equiv 2\pmod{7}$$ thì trong ba số $x+y,\ x,\ y,$ nhất thiết phải có một số chia hết cho $7,$ và lúc này
$$xy(x+y)\equiv 0\pmod{7}.$$
Lập luận trên mâu thuẫn với (\ref{2018^2019.1}). Giả sử $n\ge 1$ là sai, và ta chỉ phải xét trường hợp $n=0.$ Khi đó
$$x y(x+y)+4=2 \Leftrightarrow x y(x+y)=-2.$$
Tới đây, ta lập bảng giá trị sau đây
\begin{center}
    \begin{tabular}{c|c|c|c|c}
        $xy$ & $2$ & $1$ & $-1$ & $-2$\\
        \hline
        $x+y$ & $1$ & $2$ & $-2$ & $-1$\\
        \hline
        $(x,y)$ & $\notin\mathbb{Z}^2$ & $(1,1)$ & $\notin\mathbb{Z}^2$ & $(2,-1)$ hoặc $(1,-2)$
    \end{tabular}
\end{center}
Như vậy, tất cả các bộ ba số thỏa mãn yêu cầu bài toán là $(a, b, c)=(k+2, k+1, k)$ cùng các hoán vị, trong đó $k$ là một số nguyên bất kì.}
\end{gbtt}

\begin{gbtt}
Cho các số nguyên $a_1,a_2,\ldots,a_n$. Ta đặt 
$$A=a_1+a_2+\ldots+a_n,\qquad B=a_1^3+a_2^3+\ldots+a_n^3.$$ 
Chứng minh rằng $A$ chia hết cho $6$ khi và chỉ khi $B$ chia hết cho $6.$

\loigiai{Với mọi số nguyên $x$, ta có $x^3-x=(x-1)x(x+1)$ chia hết cho $3!=6$ do đây là tích $3$ số nguyên liên tiếp. Nhận xét này giúp ta suy ra
\begin{align*}
A-B
=\left(a_1^3-a_1\right)+\left(a_2^3-a_2\right)+\ldots+\left(a_n^3-a_n\right)   
\end{align*}
chia hết cho $6,$ và bài toán được chứng minh.}
\end{gbtt}

\begin{gbtt}
Chứng minh rằng với mọi số nguyên dương $n$, ta luôn có
$$\left[(27 n+5)^{7}+10\right]^{7}+\left[(10 n+27)^{7}+5\right]^{7}+\left[(5 n+10)^{7}+27\right]^{7}.$$
chia hết cho $42.$
\nguon{Chuyên Khoa học Tự nhiên 2019}
\loigiai{
Ta đặt $A=\left[(27 n+5)^{7}+10\right]^{7}+\left[(10 n+27)^{7}+5\right]^{7}+\left[(5 n+10)^{7}+27\right]^{7}.$ \\Theo như kết quả ở \chu{ví dụ \ref{fermatcuamu.7}}, ta nhận thấy rằng
\begin{align*}
A
&\equiv
(27n+5+10)^7+(10n+27+5)^7+(5n+10+27)^7
\\&\equiv 
27n+5+10+10n+27+5+5n+10+27
\\&\equiv 42n+42\\&\equiv 0\pmod{42}.
\end{align*}
Như vậy, bài toán đã cho được chứng minh.}
\end{gbtt}
 %ước, bội, chia hết và phép toán đồng dư
\section{Bất đẳng thức trong chia hết}

\subsection*{Lí thuyết}

Trong mục này, chúng ta sẽ làm quen với các kĩ thuật đánh giá
\begin{quote}
    \it
    Nếu $a$ chia hết cho $b$ thì hoặc $a=0,$ hoặc $|a|\ge |b|.$
\end{quote}
Một số kĩ thuật đánh giá bất đẳng thức khác trong chia hết cũng được đưa vào viết ở trong tiểu mục này. Xin mời bạn đọc tìm hiểu qua một vài ví dụ.

\subsection*{Ví dụ minh họa}

\begin{bx}
Tìm các số nguyên dương $x,y$ thỏa mãn $x+1$  chia hết cho $y$ và $y+1$ chia hết cho $x.$
\loigiai{
Ta đã biết, với các số nguyên dương $a,b,$ nếu $a$ chia hết cho $b$ thì $a\ge b.$ \\
Như vậy, kết hợp hai giả thiết đã cho, ta có
$$x+1\ge y\ge x-1.$$
Do $y$ là số nguyên dương, ta suy ra $y\in \{x-1;x;x+1\}.$ 
\begin{enumerate}
    \item Với $y=x-1,$ ta có $x-1$ là ước của $x+1=(x-1)+2,$ tức $x-1$ là ước của $2.$ Ta được các cặp
    $$(x,y)=(2,1),\quad (x,y)=(3,2).$$
    \item  Với $y=x,$ ta có $x$ là ước của $x+1$ tức $x=1,$ kéo theo $(x,y)=(1,1).$
    \item Với $y=x+1,$ ta có $x$ là ước của $x+2,$ kéo theo $(x,y)=(1,2)$ hoặc $(x,y)=(2,3)$.
\end{enumerate}
Kết luận, có tất cả $5$ cặp $(x,y)$ thỏa mãn đề bài, bao gồm 
$$(1,1),(1,2),(2,1),(2,3),(3,2).$$}
\end{bx}

\begin{bx}
Tìm tất cả các số nguyên dương $a,b$ sao cho $\dfrac{a^2b+b}{ab^2+9}$ là số nguyên.
\nguon{Rioplantense Mathematical Olympiad 1998}
\loigiai{
Với các số $a,b$ thỏa yêu cầu, ta có 
$$ab^2+9\mid b\left(a^2b+b\right)-a\left(ab^2+9\right)=b^2-9a.$$
Tới đây, ta xét các trường hợp sau
\begin{enumerate}
    \item Với $b=1,$ ta có $1-9a$ chia hết cho $a+9.$ Ta tìm được $a=32,a=73.$
    \item Với $b=2,$ ta có $4-9a$ chia hết cho $4a+9.$ Ta tìm được $a=22.$
    \item Với $b\ge 3,$ ta nhận xét
    $$-9-ab^2<b^2-ab^2\le b^2-9a< b^2<ab^2+9.$$
    Nhận xét trên kết hợp với việc $ab^2+9$ là ước của $b^2-9a$ giúp ta suy ra $b^2=9a.$ Ta có
    $$\dfrac{a^2b+b}{ab^2+9}=\dfrac{\left(\dfrac{b^2}{9}\right)^2b+b}{\left(\dfrac{b^2}{9}\right)b^2+9}=\dfrac{b^5+81b}{9b^4+729}=\dfrac{b}{9}.$$
    Số bên trên là số nguyên. Bộ $(a,b)$ trong trường hợp này có dạng $\left(9k^2,9k\right),$ với $k$ nguyên dương.
\end{enumerate}
Kết luận, các bộ $(a,b)$ thỏa yêu cầu là $$(22,2),\quad (32,1),\quad (73,1)$$ và dạng tổng quát $\left(9k^2,9k\right),$ với $k$ là số nguyên dương.}
\begin{luuy}
\nx Nhờ vào biến đổi $b\left(a^2b+b\right)-a\left(ab^2+9\right)=b^2-9a$ ở phía trên, bậc của số bị chia được hạ từ \chu{bậc ba} xuống \chu{bậc hai}, trong khi đó bậc của số chia vẫn là bậc ba. Sự chênh lệch bậc này dẫn ta nghĩ đến việc so sánh số bị chia và số chia, để thông qua đó ta có thể áp dụng bổ đề
\begin{quote}
\it
    "Nếu $a$ chia hết cho $b$ thì $|a|\ge |b|$ hoặc $a=0.$"
\end{quote}
\end{luuy}
\end{bx}

%nguyệt anh
\begin{bx}
Tìm tất cả các số nguyên dương $k$ sao cho tồn tại số nguyên dương $n$ thỏa mãn $2^n+11$ chia hết cho $2^k-1.$
\nguon{Hanoi Open Mathematics Competitions 2018}
\loigiai{
Giả sử tồn tại số nguyên dương $k$ thỏa mãn đề bài. Ta xét các trường hợp sau.
\begin{enumerate}
    \item Với $k\ge5,$ ta có $2^{k-1}+11\le2^k-1$. Đặt $n=ka+b$ trong đó $a,b$ là số tự nhiên và $0\le b\le k-1.$ Phép đặt này cho ta
    $$2^n+11\equiv\tron{2^k}^a\cdot2^b+11\equiv2^b+11\equiv0\pmod{2^k-1}.$$
    Kết hợp với nhận xét dưới đây
    $$0<2^b+11\le 2^{k-1}+11\le 2^k-1,$$
    ta suy ra $2^b+11=2^k-1.$ Biến đổi tương đương cho ta
    $$2^b\tron{2^{k-b}-1}=10.$$
    Ta dễ dàng nhận thấy $b=1,$ kéo theo $2^{k-1}-1=5.$ Không có $k$ thỏa mãn trong trường hợp này.
    \item Với $k=4,$ ta thấy $n=2$ là một giá trị của $n$ thỏa mãn.
    \item Với $k=3,$ ta có $2^k-1=7.$ Đặt $n=3k+a$ trong đó $a$ là số tự nhiên và $0\le a\le 2.$ Ta có
    $$2^n+11\equiv \tron{2^3}\cdot2^a+4\equiv2^a+4\equiv 5,6,1\pmod{7}.$$
    Do đó, không tồn tại số tự nhiên $n$ thỏa mãn $2^n+11$ chia hết cho $2^3-1.$
    \item Với $k=2,$ ta thấy $n=1$ là một giá trị của $n$ thỏa mãn.
    \item Với $k=1,$ ta thấy $n=1$ là một giá trị của $n$ thỏa mãn.    
\end{enumerate}
Như vậy, các số nguyên dương $k$ thỏa mãn đề bài là $1,2,4.$}
\begin{luuy}
Ý tưởng được sử dụng trong bài trên là so sánh các biểu thức có chứa số mũ. Cụ thể, biểu thức chứa lũy thừa của số mũ cao cộng hoặc trừ đi một số sẽ hầu như lớn hơn biểu thức chứa lũy thừa của số mũ thấp cộng hoặc trừ đi một số. Để tìm hiểu thêm các ý tưởng so sánh liên quan tới mũ khác, bạn đọc có thể tham khảo phần bài tập tự luyện.
\end{luuy}
\end{bx}

\subsection*{Bài tập tự luyện}

\begin{btt}
Tìm tất cả các các cặp số nguyên dương $(x,y)$ thỏa mãn $3x-1$  chia hết cho $y$ và $3y-1$ chia hết cho $x.$
\end{btt}

\begin{btt}
Tìm các cặp số nguyên dương $(x,y)$ sao cho $2xy-1$ chia hết cho $(x-1)(y-1).$
\end{btt}

\begin{btt}
Tìm ba số nguyên dương thỏa mãn tổng hai số bất kì chia hết cho số còn lại.
\end{btt}

\begin{btt}
Cho các số nguyên dương $m,n$ thỏa mãn $5m+n$ chia hết cho $m+5n.$ Chứng minh rằng $m$ chia hết cho $n.$
\nguon{Chuyên Toán thành phố Hồ Chí Minh 2017}
\end{btt}

\begin{btt}
Tìm tất cả các cặp số nguyên dương $(a,b)$ sao cho $a^2-2$ chia hết cho $ab+2.$
\end{btt}

\begin{btt}
Tìm các số nguyên dương $a,b$ sao cho $a^2+b$ chia hết cho $b^2-a$ và $b^2+a$ chia hết cho $a^2-b.$
\nguon{Asian Pacific Mathematical Olympiad 2002}
\end{btt}

\begin{btt}
Cho ba số nguyên dương $a, p, q$ thỏa mãn các điều kiện
\begin{multicols}{2}
\begin{enumerate}[i,]
        \item $ap+1$ chia hết cho $q.$
        \item $aq+1$ chia hết cho $p.$
\end{enumerate}
\end{multicols}
Chứng minh rằng  $a>\dfrac{pq}{2\left( p+q \right)}.$
\nguon{Chuyên Đại học Sư phạm Hà Nội 2009 $-$ 2010}
\end{btt}

Tìm tất cả các số nguyên $m, n$ trong đó $m \geqslant n \geqslant 0$ sao cho $9n\left( {{m^2} + mn + {n^2}} \right) + 16$ chia hết cho ${\left( {m + 2n} \right)^3}$.
\nguon{Chuyên Toán Hải Phòng 2016 $-$ 2017}

\begin{btt}
Tìm tất cả các cặp số nguyên dương $(a,b)$ thỏa mãn $ab$ là ước của $a+b^2+1.$
\nguon{Tạp chí Pi, tháng 1 năm 2018}	
\end{btt}

\begin{btt}
Tìm tất cả các số nguyên dương $a,b$ sao cho $a+b^2$ chia hết cho $a^2b-1.$
\end{btt}

\begin{btt}
Tìm tất cả các số nguyên dương $x,y$ sao cho $\dfrac{x^2y+x+y}{xy^2+y+7}$ là số nguyên.
\nguon{International Mathematical Olympiad 1998}
\end{btt}

\begin{btt}
Tìm tất cả các số nguyên dương $1<a< b< c$ thỏa mãn $(a-1)(b-1)(c-1)$ là một ước của $abc-1.$
\end{btt}

\begin{btt}
Tìm tất cả các số nguyên dương $n$ sao cho $8^n+n$ chia hết cho $2^n+n$.
\nguon{Japanese Mathematical Olympiad Finals 2009}
\end{btt}

\begin{btt}
Tìm tất cả các số nguyên dương $m,n$ thoả mãn $m^2+2m$ chia hết cho $n^m-m.$
\nguon{Pan African 2012}
\end{btt}

\begin{btt}
Tìm tất cả các số nguyên dương $x$ sao cho tồn tại số nguyên dương $n$ thỏa mãn \[\tron{x^n+2^n+1} \mid \tron{x^{n+1}+2^{n+1}+1}. \]
\nguon{Romanian Team Selection Test 1998}
\end{btt}

\begin{btt}
Tìm tất cả các số nguyên dương $n$ thỏa mãn $n\cdot 2^{n+1}+1$ là số chính phương.
\nguon{Junior Balkan Mathematical Olympiad 2010}
\end{btt}

\begin{btt}
Tìm tất cả các số nguyên dương $n$ sao cho tồn tại hoán vị $\left(a_1, a_2, \ldots, a_{n}\right)$ của bộ số $(1,2,3, \ldots, n)$ thỏa mãn tổng $a_{1}+a_{2}+\ldots+a_{k}$ chia hết cho $k$ với mọi $k=1,2,3, \ldots, n .$
\nguon{Trường đông Toán học Nam Trung Bộ 2016}
\end{btt}

\begin{btt}
Tìm các số nguyên dương $n$ sao cho với mọi số nguyên $a$ lẻ, nếu $a^2\leq n$ thì $a$ là ước của $n$.
\end{btt}

\begin{btt}
Cho số nguyên dương $n>1$ thỏa mãn với mọi ước nguyên dương $d$ của $n,$ $d+1$ là ước nguyên dương của $n+1.$ Chứng minh rằng $n$ là số nguyên tố.
\nguon{Chọn đội tuyển chuyên Khoa học Tự nhiên 2015 $-$ 2016}
\end{btt}

\begin{btt}
Tìm tất cả các số tự nhiên $n>1$ thỏa mãn tính chất:
\begin{it}
Với hai ước nhỏ hơn $n$ của $n$ là $k$ và $l$, ít nhất một trong hai số $2 k-l$ và $2l-k$ cũng là một ước (không nhất thiết phải dương) của $n.$
\end{it}
\nguon{Benelux Mathematical Olympiad 2014}
\end{btt}

\subsection*{Hướng dẫn bài tập tự luyện}

\begin{gbtt}
Tìm tất cả các các cặp số nguyên dương $(x,y)$ thỏa mãn $3x-1$  chia hết cho $y$ và $3y-1$ chia hết cho $x.$
\loigiai{
Do $x,y$ có vai trò như nhau nên không mất tính tổng quát, ta giả sử $x\ge y.$ Giả sử này cho ta
$$3y-1\le 3x.$$
Kết hợp với giả thiết $3y-1$ chia hết cho $x,$ ta xét các trường hợp sau đây.
\begin{enumerate}
    \item Với $3y-1=x,$ ta có $y$ là ước của
    $$3x-1=3\tron{3y-1}-1=9y-4.$$
    Ta có $y$ là ước của $4,$ và ta chỉ ra $(x,y)=(2,1),(x,y)=(5,2)$ và $(x,y)=(11,4)$ từ đây.
    \item Với $3y-1=2x,$ ta có $y$ là ước của  
    $$3x-1=\dfrac{3(3y-1)}{2}-1=\dfrac{9y-5}{2}.$$
    Ta có $y$ là ước của $5,$ và ta chỉ ra $(x,y)=(1,1)$ và $(x,y)=(7,5)$ từ đây.
\end{enumerate}
Kết luận, có tất cả $9$ cặp $(x,y)$ thỏa mãn đề bài, bao gồm
\[(1,1),(1,2),(2,1),(2,5),(5,2),(5,7),(7,5),(4,11),(11,4).\]}
\end{gbtt}

\begin{gbtt}
Tìm các cặp số nguyên dương $(x,y)$ sao cho $2xy-1$ chia hết cho $(x-1)(y-1).$
\loigiai{
Từ giả thiết, ta suy ra $2xy-1$ chia hết cho cả $x-1$ và $y-1.$ Ta chia bài toán thành các trường hợp sau.
\begin{enumerate}
    \item Với $x=2,$ ta có $y=1$ hoặc $y=2.$ Thử trực tiếp, chỉ có cặp $(x,y)=(2,2)$ thỏa mãn.
    \item Với $x\ge 3,$ không mất tổng quát, ta giả sử $y\le x.$ Ta có nhận xét
    $$2y-1\le 2x-1<3(x-1).$$
    Nhận xét trên kết hợp với chú ý $2y-1$ chia hết cho $x-1$ cho ta biết 
    $$2y-1=x-1\text{ hoặc }2y-1=2x-2.$$ 
    Trường hợp $2y-1=2x-2$ rõ ràng không xảy ra do tính chẵn lẻ ở hai bên, vậy nên chỉ có $2y=x.$\\ Kết hợp với $2xy-1$ chia hết cho $(x-1)(y-1),$ ta có $2y+1$ chia hết cho $y-1.$ Ta dễ dàng tìm ra $$(x,y)=(4,2),\quad (x,y)=(8,4)$$
\end{enumerate}
Kết luận, có tất cả $5$ bộ $(x,y,z)$ thỏa yêu cầu, bao gồm
$$(2,2,7),(2,4,5),(4,2,5),(4,8,3),(8,4,3).$$}
\end{gbtt}

\begin{gbtt}
Tìm ba số nguyên dương thỏa mãn tổng hai số bất kì chia hết cho số còn lại.
\loigiai{
Gọi ba số đã cho là $a,b,c.$ Ta giả sử $a\ge b\ge c.$ Kết hợp với giả thiết $b+c$ chia hết cho $a,$ ta có
$$a\le b+c\le a+a=2a.$$
Như vậy $b+c=a$ hoặc $b+c=2a.$ Ta xét các trường hợp kể trên.
\begin{enumerate}
    \item Nếu $b+c=2a,$ dấu bằng trong đánh giá vừa rồi phải xảy ra, tức là $a=b=c.$
    \item Nếu $b+c=a,$ ta lần lượt suy ra
    $$b\mid (c+a)\Rightarrow b\mid (b+2c)\Rightarrow b\mid 2c.$$
    Do giả sử $b\ge c$ nên $b=c$ hoặc $b=2c.$ Thử trực tiếp, các trường hợp này đều thỏa.
\end{enumerate}
Như vậy, các bộ $(a,b,c)$ cần tìm sẽ có dạng $(k,k,k)$ và hoán vị của $(3k,2k,k),$ trong đó $k$ là một số nguyên dương tùy ý.}
\end{gbtt}

\begin{gbtt}
Cho các số nguyên dương $m,n$ thỏa mãn $5m+n$ chia hết cho $m+5n.$ Chứng minh rằng $m$ chia hết cho $n.$
\nguon{Chuyên Toán thành phố Hồ Chí Minh 2017}
\loigiai{
Do $5m+n<5(m+5n)$ nên $\dfrac{5m+n}{m+5n}\in\{1;2;3;4\}.$ Ta xét các trường hợp kể trên.
\begin{enumerate}
    \item Nếu $5m+n=m+5n,$ ta có $m=n.$
    \item Nếu $5m+n=2(m+5n),$ ta có $m=3n.$    
    \item Nếu $5m+n=3(m+5n),$ ta có $m=7n.$     
    \item Nếu $5m+n=4(m+5n),$ ta có $m=19n.$          
\end{enumerate}
Trong mọi trường hợp, $m$ đều chia hết cho $n.$ Bài toán được chứng minh.}
\end{gbtt}

\begin{gbtt}
Tìm tất cả các cặp số nguyên dương $(a,b)$ sao cho $a^2-2$ chia hết cho $ab+2.$
\loigiai{
Từ giả thiết, ta có $ab+2$ là ước của
$$b\tron{a^2-2}=a^2b-2b=a(ab+2)-2(a+b),$$
thế nên $ab+2$ là ước của $2(a+b).$ Ta suy ra
$$ab+2\le 2(a+b)\Rightarrow (a-2)(b-2)\le 2.$$
Tới đây, ta xét các trường hợp sau.
\begin{enumerate}
    \item Nếu một trong các trường hợp $a=1,\ a=2,\ b=1,\ b=2$ xảy ra, ta thử trực tiếp và không tìm ra được kết quả thỏa mãn.
    \item Nếu $a\ge 3$ và $b\ge 3,$ từ $(a-2)(b-2)\le 2$ ta nhận được
    $$(a,b)\in \{(3,3);(3,4);(4,3)\}.$$
    Thử trực tiếp, ta thấy chỉ có $(a,b)=(4,3)$ thỏa mãn, và đây là đáp số bài toán.
\end{enumerate}
Bài toán được giải quyết.}
\end{gbtt}

\begin{gbtt}
Tìm tất cả các số nguyên dương $a,b$ sao cho $a^2+b$ chia hết cho $b^2-a$ và $b^2+a$ chia hết cho $a^2-b.$
\nguon{Asian Pacific Mathematical Olympiad 2002}
\loigiai{
Không mất tính tổng quát, ta giả sử $a\ge b.$ Do $b^2+a$ chia hết cho $a^2-b$ nên là
$$0\le \tron{b^2+a}-\tron{a^2-b}=(a+b)(b-a-1).$$
Đánh giá trên cho ta $b-a\ge 1,$ nhưng do $a\ge b$ nên chỉ có hai trường hợp sau xảy ra.
\begin{enumerate}
    \item Với $b-a=0,$ ta có $a^2+a$ chia hết cho $a^2-a.$ Ta tìm ra $a=2,a=3$ từ đây.
    \item Với $b-a=1,$ ta có $a^2+3a+1$ chia hết cho $a^2-a-1.$ Ta tìm ra $a=1,a=2$ từ đây.
\end{enumerate}
Kết luận, có tất cả $6$ cặp $(a,b)$ thỏa yêu cầu, bao gồm
$$(1,2),(2,1),(2,3),(3,2),(2,2),(3,3).$$}
\end{gbtt}

\begin{gbtt}
Cho ba số nguyên dương $a, p, q$ thỏa mãn các điều kiện
\begin{multicols}{2}
\begin{enumerate}[i,]
        \item $ap+1$ chia hết cho $q.$
        \item $aq+1$ chia hết cho $p.$
\end{enumerate}
\end{multicols}
Chứng minh rằng  $a>\dfrac{pq}{2\left( p+q \right)}.$
\nguon{Chuyên Đại học Sư phạm Hà Nội 2009 $-$ 2010}
\loigiai{
Từ giả thiết ta có $pq\mid \left( ap+1 \right)\left( aq+1 \right).$ Ta suy ra
\begin{align*}
    pq\mid \left( {{a}^{2}}pq+ap+aq+1 \right)
    &\Rightarrow pq\mid (ap+aq+1)
    \\&\Rightarrow pq\le ap+aq+1
    \\&\Rightarrow pq< 2(ap+aq)
    \\&\Rightarrow a>\dfrac{pq}{2(p+q)}.
\end{align*}
Bất đẳng thức đã cho được chứng minh.}
\end{gbtt}

\begin{gbtt}
Tìm tất cả các số nguyên $m, n$ trong đó $m \geqslant n \geqslant 0$ sao cho $9n\left( {{m^2} + mn + {n^2}} \right) + 16$ chia hết cho ${\left( {m + 2n} \right)^3}$.
\nguon{Chuyên Toán Hải Phòng 2016 $-$ 2017}
\loigiai{
Đặt $A=9n\left( {{m^2} + mn + {n^2}} \right) + 16 - {\left( {m + 2n} \right)^3}.$ Ta có
\[A = {n^3} - 3{n^2}m + 3n{m^2} - {m^3} + 16 = {\left( {n - m} \right)^3} + 16.\]
Đặt $B=(m+2n)^3.$ Ta có
\[B = {\left( {m - n + 3n} \right)^3} = {\left( {m - n} \right)^3} + 9{\left( {m - n} \right)^2}n + 27\left( {m - n} \right){n^2} + 27{n^3}.\]
Do giả thiết $A$ là bội của $B$ nên $\left| A \right| \geqslant \left| B \right|$. Ta xét các trường hợp sau.
\begin{enumerate}
    \item Nếu $m-n\ge 3$ thì ta có $|A|=(m-n)^3-16.$ Mặt khác
    \[{\left( {m - n} \right)^3} - 16 \le {\left( {m - n} \right)^3} + 9{\left( {m - n} \right)^2}n + 27\left( {m - n} \right){n^2} + 27{n^3}.\]
    Ta chỉ ra $|A|\le |B|,$ mâu thuẫn.
    \item Nếu $m-n=2,$ ta có $|A|=8$ chia hết cho 
    $$B=(m+2n)^3=(n+2+2n)^3=(3n+2)^3.$$
    Ta tìm ra $n=0$ từ đây, kéo theo $m=2.$
    \item Nếu $m-n=1,$ ta có $|A|=15$ chia hết cho 
    $$B=(m+2n)^3=(n+1+2n)^3=(3n+1)^3.$$
    Ta tìm ra $n=0$ từ đây, kéo theo $m=1.$ \item Nếu $m=n,$ ta có $|A|=16$ chia hết cho  
    $$B=(m+2n)^3=(3n)^3=27n^3.$$
    Ta có $16$ chia hết cho $27$ từ đây, vô lí.
\end{enumerate}
Như vậy, có hai cặp $(m,n)$ thỏa mãn yêu cầu là $(2,0)$ và $(1,0).$}
\end{gbtt}

\begin{gbtt}
Tìm tất cả các cặp số nguyên dương $(a,b)$ thỏa mãn $ab$ là ước của $a+b^2+1.$
\nguon{Tạp chí Pi, tháng 1 năm 2018}	
\loigiai{Giả sử $(a,b)$ là cặp số nguyên dương thỏa yêu cầu. Ta dễ dàng suy ra $a\mid \tron{b^2+1}$ và $b\mid (a+1)$. Đặt
\[a+1=bk.\tag{1}\label{pip1331}\]
Do $a\mid \tron{b^2+1}$ nên $(bk-1)\mid (b^2+1)=((b^2+1)k-(bk-1)b),$ hay là
\[(bk-1)\mid (b+k).\tag{2}\label{pip1332}\]
Do $b+k>0$ nên $b+k\geq bk-1,$ dẫn tới $bk-b-k+1\leq 2$ hay 
\[(b-1)(k-1)\leq 2.\tag{3}\label{pip1333}\]
Vì $b,k$ là các số nguyên dương nên $b-1$ và $k-1$ là các số tự nhiên. Do đó, từ (\ref{pip1333}) ta suy ra 
$$(b-1)(k-1) \in \lbrace 0;1;2\rbrace.$$
Ta sẽ xét các trường hợp kể trên.
\begin{enumerate}
	\item Với $(b-1)(k-1)=0,$ ta có $b=1$ hoặc $k=1$. 
	\begin{itemize}
	    \item \chu{Trường hợp 1.} Với $k=1,$ thế vào (\ref{pip1332}) ta có 
	    $$(b-1)\mid(b+1).$$
	    Ta tìm được $b=2$ hoặc $b=3$ từ đây, lại thế hết vào (\ref{pip1331}) thì ta có $(a,b)=(1,2),(2,3).$
	    \item \chu{Trường hợp 2.} Với $b=1,$ thế vào giả thiết ta có $a$ là ước của $a+2,$ suy ra $a=1$ hoặc $a=2.$	
    \end{itemize}
	\item Với $(b-1)(k-1)=1,$ ta có $b-1=k-1=1,$ hay là $b=k=2.$ Thế trở lại (\ref{pip1331}), ta tìm ra $a=3.$
	\item Với $(b-1)(k-1)=2,$ ta có $b-1=1$ và $k-1=2$ hoặc $b-1=2$ và $k-1=1,$ tức $$(b,k)\in \{(3,2);(2,3)\}.$$ 
	Lần lượt thế các giá trị của $b$ và $k$ trở lại (\ref{pip1331}), ta được $a=5.$
\end{enumerate}
Tất cả các cặp số $(a,b)$ thu được từ mỗi trường hợp trên là
$$ (1,2),\ (2,3),\ (1,1),\ (2,1),\ (3,2),\ (5,2),\ (5,3).$$
Bằng cách kiểm tra trực tiếp, chúng đều thỏa tính chất đã cho. Bài toán được giải quyết.}
\end{gbtt}

\begin{gbtt}
Tìm tất cả các số nguyên dương $a,b$ sao cho $a+b^2$ chia hết cho $a^2b-1.$
\loigiai{
Giả sử $a+{{b}^{2}}$ chia hết cho ${{a}^{2}}b-1$, khi đó tồn tại số nguyên dương $k$ sao cho \[a+{{b}^{2}}=k\left( {{a}^{2}}b-1 \right)\Leftrightarrow a+k=b\left( k{{a}^{2}}-b \right).\]
Đặt $m=k{{a}^{2}}-b$ với $m$ là một số nguyên. Ta lại có $mb=a+k,$ kéo theo $$mb-m-b=a+k-ka^2.$$ Biến đổi tương đương cho ta
\[mb-m-b+1=a+k-k{{a}^{2}}+1\Leftrightarrow \left( m-1 \right)\left( b-1 \right)=\left( a+1 \right)\left( k+1-ka \right).\tag{*}\label{chanchiahet11}\]
Do $a,b,k$ là các số nguyên dương nên ta suy ra được $m\ge 1$.
Điều này dẫn đến \[\left( b-1 \right)\left( m-1 \right)\ge 0.\] Từ đây, ta suy ra  $\left( a+1 \right)\left( k+1-ka \right)\ge 0.$
Vì $a$ là số nguyên dương nên ta có \[k+1-ka\ge 0\] hay $k\left( a-1 \right)\le 1.$
Lại có $k$ cũng là số nguyên dương nên từ $k\left( a-1 \right)\le 1$ ta được \[k\left( a-1 \right)=0\quad \text{hoặc}\quad k\left( a-1 \right)=1.\]  
Tới đây, ta xét các trường hợp sau.
\begin{enumerate}
    \item Nếu $k(a-1)=0$ thì $a=1.$ Thế vào (\ref{chanchiahet11}) ta được
    \[\left( b-1 \right)\left( m-1 \right)=2.\]
    Ta tìm được $b=2$ và $b=3$ từ đây. Thử lại, ta thấy thỏa mãn.
    \item Nếu $k\left( a-1 \right)=1$ thì $k=1$ và $a=2.$ Thế vào (\ref{chanchiahet11}) ta được
    \[\left( b-1 \right)\left( m-1 \right)=0.\]
    Nếu $b=1,$ thử lại ta thấy thỏa mãn. Nếu $m=1,$ kết hợp với hệ thức $mb=a+k$ ta có $b=3$.\\ Trường hợp này cho ta $(a,b)=(2,1),(2,3)$.
\end{enumerate}
Vậy có tất cả $4$ cặp số nguyên dương $(a,b)$ thỏa mãn yêu cầu bài toán, bao gồm
$$(1,2),\quad (1,3),\quad (2,1),\quad (2,3).$$}
\end{gbtt}

\begin{gbtt}
Tìm tất cả các số nguyên dương $x,y$ sao cho $\dfrac{x^2y+x+y}{xy^2+y+7}$ là số nguyên.
\nguon{International Mathematical Olympiad 1998}
\loigiai{
Với các số $x,y$ thỏa yêu cầu, ta có $xy^2+y+7$ là ước của
$$y\left(x^2y+x+y\right)-x\left(xy^2+y+7\right)=y^2-7x.$$
Tới đây, ta xét các trường hợp sau
\begin{enumerate}
    \item Với $y=1,$ ta có $1-7x$ chia hết cho $x+8.$ Ta tìm được các cặp $(x,y)=(11,1),(49,1).$
    \item Với $y=2,$ ta có $4-7x$ chia hết cho $6x+7.$ Ta không tìm ra bộ $(x,y,z)$ nguyên dương nào.
    \item Với $y\ge 3,$ ta nhận xét
    $$-xy^2-y-7<-xy^2<-7x<y^2-7x<y^2<xy^2+y+7.$$
    Nhận xét trên kết hợp với việc $xy^2+y+7$ là ước của $y^2-7x$ giúp ta suy ra $y^2=7x.$ Ta có
    $$\dfrac{x^2y+x+y}{xy^2+y+7}=\dfrac{    \left(\dfrac{y^2}{7}\right)^2y+    \dfrac{y^2}{7}+y}{    \left(\dfrac{y^2}{7}\right)y^2+y+7}=\dfrac{y}{7}.$$
    Số bên trên là số nguyên. Bắt buộc, $(x,y)=\left(7k^2,7k\right),$ với $k$ nguyên dương.
\end{enumerate}
Đối chiếu và thử lại, tất cả các bộ $(x,y)$ thỏa yêu cầu là $(11,1),(49,1)$ và dạng tổng quát $\left(7k^2,7k\right),$ với $k$ là số nguyên dương.}
\end{gbtt}

\begin{gbtt}
Tìm tất cả các số nguyên dương $1<a< b< c$ thỏa mãn $(a-1)(b-1)(c-1)$ là một ước của $abc-1.$
\loigiai{
Ta đặt $x=a-1,y=b-1,z=c-1$. Giả thiết về chia hết cho ta $xyz\mid \left[ (x+1)(y+1)(z+1)-1 \right],$ hay là 
\[xyz\mid (xy+yz+zx+x+y+z).\]
Theo đó, tồn tại số nguyên dương $k$ sao cho $kxyz=xy+yz+zx+x+y+z.$ Sự tồn tại ấy kéo theo  $$k=\dfrac{1}{x}+\dfrac{1}{y}+\dfrac{1}{z}+\dfrac{1}{xy}+\dfrac{1}{yz}+\dfrac{1}{zx}.$$
Ngoài ra, điều kiện phép đặt $1\le x<y<z$ còn cho ta
\[k\le 1+\dfrac{1}{2}+\dfrac{1}{3}+\dfrac{1}{1\cdot 2}+\dfrac{1}{2\cdot 3}+\dfrac{1}{3\cdot 1}<3.\] 
Do vậy, $k=1$ hoặc $k=2$. Hơn thế nữa, nếu như $x\ge 3,y\ge 4, z\ge 5,$ ta có \[k\le\dfrac{1}{3}+\dfrac{1}{4}+\dfrac{1}{5}+\dfrac{1}{3\cdot 4}+\dfrac{1}{4\cdot 5}+\dfrac{1}{5\cdot 3}<1.\] 
Đây là điều không thể xảy ra. Sự không tồn tại này chứng tỏ $x\in\{1;2\}.$ Ta xét các trường hợp sau.
\begin{enumerate}
    \item Với $x=1,k=1,$ ta có $1+2(y+z)+yz=yz.$ Phương trình này vô nghiệm nguyên dương.
    \item Với $x=1,k=2$, ta có 
    $$1+2(y+z)=yz\Leftrightarrow yz-2y-2z-1=0\Leftrightarrow (y-2)(z-2)=5.$$
    Ta tìm được $y=3,z=7$ từ đây, thế nên $a=2,b=4,c=8.$
    \item Với $x=2,k=1$, ta có 
    $$2+3(y+z)=yz\Leftrightarrow yz-3y-3z-2=0\Leftrightarrow (y-3)(z-3)=11.$$ 
    Ta tìm được $y=4,z=15$ từ đây, thế nên $a=3,b=5,c=16.$
    \item Với $x=2,k=2.$ ta có
    $$2=\dfrac{1}{2}+\dfrac{1}{y}+\dfrac{1}{z}+\dfrac{1}{2y}+\dfrac{1}{yz}+\dfrac{1}{2z}\le \dfrac{1}{2}+\dfrac{1}{3}+\dfrac{1}{4}+\dfrac{1}{6}+\dfrac{1}{12}+\dfrac{1}{8}<2.$$
    Đây là điều không thể xảy ra.
\end{enumerate} 
Kết luận, có tất cả hai bộ $(a,b,c)$ thỏa mãn yêu cầu đề bài là $(2,4,8)$ và $(3,5,16).$}
\end{gbtt}

\begin{gbtt}
Tìm tất cả các số nguyên dương $n$ sao cho $8^n+n$ chia hết cho $2^n+n$.
\nguon{Japanese Mathematical Olympiad Finals 2009}
\loigiai{
Với mọi số nguyên dương $n,$ ta luôn luôn có $\tron{2^n+n}\mid \tron{8^n+n^3}.$ \\Kết hợp giả thiết, ta suy ra $n^3-n$ chia hết cho $2^n+n.$ Ta sẽ chứng minh rằng
\[2^n>n^3,\text{ với mọi }n\ge 10.\]
Bất đẳng thức trên đúng với $n=10,$ đồng thời
$$2^{n+1}=2\cdot2^n>2n^3>(n+1)^3.$$
Bất đẳng thức trên đã được chứng minh bằng quy nạp.\\ Quay lại bài toán, ta xét các trường hợp sau đây.
\begin{enumerate}
    \item Nếu $n\ge 10,$ từ $n^3-n$ chia hết cho $2^n+n,$ ta suy ra
    $$n^3-n\ge 2^n+n>n^3+n.$$
    Đánh giá trên không thể xảy ra do $n\ge 1.$
    \item Nếu $n\le 9,$ kiểm tra trực tiếp, các giá trị thỏa mãn của $n$ là $n=1,n=2,n=4$ và $n=6.$    
\end{enumerate}}
\end{gbtt}

\begin{gbtt}
Tìm tất cả các số nguyên dương $m,n$ thoả mãn $m^2+2m$ chia hết cho $n^m-m.$
\nguon{Pan African 2012}
\loigiai{
Đầu tiên, ta sẽ chứng minh bằng quy nạp rằng nếu $m\geq 6$ thì $$2^m>m^2+3m.$$
Bất đẳng thức trên đúng với $n=6,$ đồng thời
$$2^{m+1}=2\cdot 2^m=2\left(m^2+3m\right)>(m+1)^2+3(m+1).$$
Theo nguyên lí quy nạp, bất đẳng thức trên được chứng minh. \\Quay trở lại bài toán. Với $n=1$, kiểm tra trực tiếp, ta tìm ta $m=2$ hoặc $m=4.$\\
Với $n\ge 2,$ ta có
$$n^m-m\geq 2^m-m> m^2+2m.$$
Đánh giá này này mâu thuẫn với $\tron{n^m-m}\mid \tron{m^2+2m},$ thế nên $m\le 5.$\\ Thử trực tiếp, ta chỉ ra có tất cả $6$ cặp $(m,n)$ thỏa yêu cầu, bao gồm $$(2,1),\ (4,1),\ (1,2),\ (1,4),\ (3,2),\ (4,2).$$}
\end{gbtt}

\begin{gbtt}
Tìm tất cả các số nguyên dương $x$ sao cho tồn tại số nguyên dương $n$ thỏa mãn
\[\tron{x^n+2^n+1} \mid \tron{x^{n+1}+2^{n+1}+1}. \]
\nguon{Romanian Team Selection Test 1998}
\loigiai{
Ta có nhận xét
$x^{n+1}+2^{n+1}+1=x(x^n+2^n+1)-2^nx+1-x+2^{n+1}.$
Kết hợp với giả thiết, ta suy ra
$$\left(x^n+2^n+1\right)\mid \left(2^{n+1}-2^nx+1-x\right).$$
Với $x=1,x=2$ ta dễ dàng chỉ ra được điều mâu thuẫn. Ta chỉ xét $x\geq 3$. Khi đó không khó để thấy $$2^{n+1}-2^nx+1-x< 0,$$ 
và ta nghĩ đến đánh giá
$$2^nx-2^{n+1}+x-1=\left|2^{n+1}-2^nx+1-x \right|\geq x^n+2^n+1.$$
Rút gọn, ta được $\left(2^n+1\right)x\geq x^n+2^{n+1}+2^n+2.$ \\
Với $n\geq 3$, ta chứng minh bằng quy nạp được là $$x^n\ge x\left(2^n+1\right),$$ thế nên
$x^n+2^{n+1}+2^n+2> x(2^n+1),$
mâu thuẫn. Do vậy, ta phải có $n\leq 2$.
\begin{enumerate}
    \item Với $n=2,$ từ $\left(2^n+1\right)x\geq x^n+2^{n+1}+2^n+2$ ta có $5x\geq x^2+14$. Không có số $x\in\mathbb{Z}$ nào như vậy.
    \item Với $n=1,$ thay vào giả thiết ban đầu, ta được
    $$(x+3)\mid \left(x^2+5\right)=\big[(x+3)(x-3)+14\big].$$ Ta suy ra $x+3$ là ước của $14,$ vì thế $x=1$ hoặc $x=14.$
\end{enumerate}
Kết luận, tất cả các giá trị của $x$ thỏa yêu cầu là $x=1$ và $x=14.$}
\end{gbtt}

\begin{gbtt}
Tìm tất cả các số nguyên dương $n$ thỏa mãn $n\cdot 2^{n+1}+1$ là số chính phương.
\nguon{Junior Balkan Mathematical Olympiad 2010}
\loigiai{
Giả sử tồn tại số nguyên dương $n$ thỏa yêu cầu bài toán. \\
Do $n\cdot 2^{n+1}+1$ là số lẻ, ta đặt $n\cdot2^{n+1} + 1 = \tron{2k+1}^2.$ Phép đặt này cho ta
\[n\cdot2^{n-1}=k\tron{k+1}.\tag{1}\label{junibmo2010}\]
Trong hai số tự nhiên liên tiếp $k$ và $k+1,$ có một số là lẻ. Số này nguyên tố cùng nhau với $2^{n-1},$ và bắt buộc số còn lại chia hết cho $2^{n-1}.$ Lập luận này cho ta biết được rằng
\[\hoac{&2^{n-1}\mid k \\ &2^{n-1}\mid (k+1)}
\Rightarrow 
\hoac{&2^{n-1}\le k \\ &2^{n-1}\le k+1}
\Rightarrow 
2^{n-1}\le k+1\Rightarrow k\ge 2^{n-1}-1.\tag{2}\label{junibmo2010.2}\]
Kết hợp (\ref{junibmo2010}) và (\ref{junibmo2010.2}), ta được
$n\cdot 2^{n-1}\ge 2^{n-1}(2^{n-1}-1),$ hay là $$n\ge 2^{n-1}-1.$$
Dựa vào chứng minh bằng quy nạp, ta dễ thấy bất đẳng thức trên đổi chiều với $n\ge4,$ và bắt buộc $n\le 3.$ Ta xét các trường hợp dưới đây.
\begin{enumerate}
    \item Với $n=1,$ ta có $n\cdot 2^{n+1}+1=5$ không là số chính phương.
    \item Với $n=2,$ ta có $n\cdot 2^{n+1}+1=17$ không là số chính phương.
    \item Với $n=3,$ ta có $n\cdot 2^{n+1}+1=49=7^2.$ 
\end{enumerate}
Như vậy, $n=3$ là giá trị duy nhất của $n$ thỏa yêu cầu bài toán.}
\end{gbtt}

\begin{gbtt}
Tìm tất cả các số nguyên dương $n$ sao cho tồn tại hoán vị $\left(a_1, a_2, \ldots, a_{n}\right)$ của bộ số $(1,2,3, \ldots, n)$ thỏa mãn tổng $a_{1}+a_{2}+\ldots+a_{k}$ chia hết cho $k$ với mọi $k=1,2,3, \ldots, n .$
\nguon{Trường đông Toán học Nam Trung Bộ 2016}
\loigiai{
Kiểm tra trực tiếp với $n=1,2,3,$ ta thấy $n=1,n=3$ thỏa mãn. \\Ta sẽ chứng minh rằng với $n\ge 4$ không còn số nào thỏa mãn nữa. 
\begin{enumerate}[a,]
    \item Ta sẽ tính trực tiếp $a_n.$ Cho $k=n,$ ta nhận thấy $n$ là ước của
    $$a_1+a_2+\ldots+a_n=\dfrac{n(n+1)}{2},$$
    thế nên $n$ lẻ và $n\ge 5.$ Tiếp theo, cho $k=n-1,$ ta nhận thấy $n-1$ là ước của
    $$a_1+a_2+\ldots+a_{n-1}=\dfrac{n(n+1)}{2}-a_n=\frac{(n+1)(n-1)}{2}+\frac{n+1}{2}-a_n,$$ 
    thế nên $\dfrac{n+1}{2}-a_n$ chia hết cho $n-1.$ Lập luận này kết hợp với các đánh giá
    \begin{align*}
        &\dfrac{n+1}{2}-a_n\ge \dfrac{n+1}{2}-n=-\dfrac{n-1}{2}>-n+1, \\
        &\dfrac{n+1}{2}-a_n\le \dfrac{n+1}{2}-1=\dfrac{n-1}{2}<n-1.
    \end{align*}
    cho ta biết $\dfrac{n+1}{2}-a_n=0,$ hay là $a_n=\dfrac{n+1}{2}.$
    \item Ta sẽ tính trực tiếp $a_{n-1}.$ Cho $k=n-2,$ ta nhận thấy $n-2$ là ước của    
    $$a_1+a_2+\ldots+a_{n-2}=\dfrac{n^2-1}{2}-a_{n-1}=\frac{(n-2)(n+1)}{2}+\frac{n+1}{2}-a_{n-1},$$  
    thế nên $\dfrac{n+1}{2}-a_{n-1}$ chia hết cho $n-2.$ Lập luận này kết hợp với các đánh giá
    \begin{align*}
        &\dfrac{n+1}{2}-a_{n-1}\ge \dfrac{n+1}{2}-n+1=-\dfrac{n-3}{2}>-n-2, \\
        &\dfrac{n+1}{2}-a_{n-1}\le \dfrac{n+1}{2}-1=\dfrac{n-1}{2}<n-2.
    \end{align*}
    cho ta biết $\dfrac{n+1}{2}-a_{n-1}=0,$ hay là $a_{n-1}=\dfrac{n+1}{2}.$
\end{enumerate}
Dựa vào các tính toán bên trên, ta chỉ ra $a_n=a_{n-1},$ mâu thuẫn với việc các số trong hoán vị là phân biệt. Do đó, trường hợp $n\ge 4$ không xảy ra. Tất cả các số tự nhiên $n$ cần tìm là $n=1$ và $n=3.$}
\end{gbtt}

\begin{gbtt}
Tìm tất cả các số nguyên dương $n$ sao cho với mọi số nguyên $a$ lẻ, nếu $a^2\leq n$ thì $a$ là ước của $n$.
\loigiai{Ta thấy $n=1$ là một kết quả. Giả sử tồn tại số nguyên dương $n\ge 1$ thỏa mãn. Xét số nguyên lẻ $m$ lớn nhất sao cho $m^2<n.$ Ta có $n\le\tron{m+2}^2.$  Ta xét các trường hợp sau đây.
\begin{enumerate}
    \item Với $m\ge7,$ theo giả thiết, ta có $m,m-2,m-4$ là ước của $n.$\\ Vì ba số $m,m-2,m-4$ đôi một nguyên tố cùng nhau nên ta suy ra
    $$m\tron{m-2}\tron{m-4}\mid n.$$
    Từ những nhận xét trên, ta thu được
    $$m\tron{m-2}\tron{m-4}< n \le \tron{m+2}^2.$$
    Biến đổi bất phương trình $m\tron{m-2}\tron{m-4}<\tron{m+2}^2$ cho ta
    $$m^2\tron{7-m}+4\tron{1-m}=\tron{m+2}^2-m\tron{m-2}\tron{m-4}\le 0.$$
    Điều này không thể xảy ra.
    \item Với $1\le m\le 5,$ ta có $n<49.$ Ta lập bảng giá trị sau đây.
    \begin{center}
        \begin{tabular}{c|c|c|c}
            Điều kiện của $n$ & $2\le n\le 8$ & $9\le n\le 24$ & $25\le n \le 48$  \\
            \hline
            Tính chia hết &  & $3\mid n$ & $[3,5]\mid n$ \\     
            \hline
            $n$ & $2,3,4,5,6,7,8$  & $12,15,18,21,24$ & $30,45$
        \end{tabular}
    \end{center}    
\end{enumerate}
Như vậy, các số nguyên dương $n$ thỏa mãn yêu cầu là $1,2,3,4,5,6,7,8,12,15,18,21,24,30,45.$}
\begin{luuy}
Một câu hỏi đặt ra là tại sao tác giả lại tìm được mốc $m\ge 7.$ \\Hãy để ý bất đẳng thức sau trong trường hợp đầu tiên
$$m(m-2)(m-4)<(m+2)^2.$$
Bất đẳng thức kể trên đổi dấu khi đi qua giá trị $m$ xấp xỉ $6,478.$ Tác giả chọn mốc đánh giá của $m$ là $7$ để số lượng các trường hợp còn lại cần xét trở nên ít hơn.
\end{luuy}
\end{gbtt}

\begin{gbtt}
Cho số nguyên dương $n>1$ thỏa mãn với mọi ước nguyên dương $d$ của $n,$ $d+1$ là ước nguyên dương của $n+1.$ Chứng minh rằng $n$ là số nguyên tố.
\nguon{Chọn đội tuyển chuyên Khoa học Tự nhiên 2015 $-$ 2016}
\loigiai{
Giả sử phản chứng rằng $n$ là hợp số. Ta đặt $n=ab,$ trong đó  $b\ge a\ge2.$ Do $b+1$ là ước của $$ab+1=a(b+1)+1-a$$ nên $b+1$ cũng là ước của $a-1.$ Ta suy ra
$$a+1\le b+1\le a-1.$$
Đây là điều không thể xảy ra. Giả sử phản chứng là sai. Bài toán được chứng minh.}
\begin{luuy}
Phản chứng, áp dụng phép chia đa thức đã học, và sử dụng bất đẳng thức trong chia hết là tất cả các phương pháp sử dụng trong bài toán trên.
\end{luuy}
\end{gbtt}

\begin{gbtt}
Tìm tất cả các số tự nhiên $n>1$ thỏa mãn tính chất:
\begin{it}
Với hai ước nhỏ hơn $n$ của $n$ là $k$ và $l$, ít nhất một trong hai số $2 k-l$ và $2l-k$ cũng là một ước (không nhất thiết phải dương) của $n.$
\end{it}
\nguon{Benelux Mathematical Olympiad 2014}
\loigiai{
Ta dễ thấy tất cả các số $n$ nguyên tố đều thỏa yêu cầu. \\
Nếu $n$ là hợp số, ta gọi $p$ là ước nguyên tố nhỏ nhất của $n,$ đồng thời đặt $n=pm.$ Khi cho $(k,l)=(1,m),$ ta nhận thấy $n$ chia hết cho $2m-1$ hoặc $m-2.$ Ước lớn nhất của $n$ là $m,$ lại do
    $$m<2m-1<2m\le n$$
    nên $2m-1$ không là ước của $n,$ và bắt buộc $m-2$ là ước của $n.$ Ta có
    $$(m-2)\mid mp=(m-2)p+2p.$$
    Dựa vào nhận xét trên, ta chỉ ra $m-2$ là ước của $2p,$ vì thế $m-2$ chỉ có thể nhận một trong các giá trị $1,2,p,2p.$ Ta xét các trường hợp kể trên.
    \begin{enumerate}
        \item Với $m-2=1$ hay $m=3,$ do $p\le m$ nên $p\in\{2;3\}.$ \\
        Từ đó, ta chỉ ra $n=6$ hoặc $n=9.$ Thử trực tiếp, cả hai số này thỏa yêu cầu.
        \item Với $m-2=2$ hay $m=4,$ do $p\le m$ nên $p\in\{2;3\}.$\\
        Từ đó, ta chỉ ra $n=8$ hoặc $n=12.$ Thử trực tiếp, không có số nào thỏa yêu cầu.
        \item  Với $m-2=p$ hay $m=p+2,$ ta có $n=p(p+2).$ 
        Ngoài ra, ta còn có thể giả sử $p>2$ (do trường hợp $m-2=2$ đã được giải quyết). Khi cho $(k,l)=(1,p),$ ta chỉ ra $p-2$ hoặc $2p-1$ là ước của $n.$
        \begin{itemize}
            \item Nếu $n=p(p+2)$ chia hết cho $p-2,$ ta tìm ra $p=3,$ và khi ấy $n=15.$
            \item Nếu $n=p(p+2)$ chia hết cho $2p-1,$ ta tìm ra $p=5,$ và khi ấy $n=35.$
        \end{itemize}
        Thử với từng trường hợp, ta thấy chỉ có $n=15$ thỏa yêu cầu.
        \item Với $m-2=2p$ hay $m=2p+2,$ vậy nên $n$ là số chẵn và $p=2.$ Ta tìm ra $n=12.$ \\Thử lại, ta thấy $n=12$ không thỏa yêu cầu.
    \end{enumerate}
Kết luận, tất cả các số nguyên tố $n$ và $n=6,n=9,n=15$ là các số $n$ ta cần tìm.}
\end{gbtt}

\section{Tính nguyên tố cùng nhau}

\subsection*{Lí thuyết}

Lí thuyết quan trọng nhất được sử dụng trong mục này là tính chất: 
\begin{it}
"Với mọi số nguyên $a,b,c,$ nếu $ab$ chia hết cho $c$ và $(a,c)=1$ thì $b$ chia hết cho $c$".
\end{it}


\subsection*{Ví dụ minh họa}

\begin{bx}
Tìm tất cả số tự nhiên $n$ sao cho $(2n+1)^{3}+1$ chia hết cho $2^{2021}.$
\nguon{Chuyên Toán Phổ thông Năng khiếu 2021}
\loigiai{
Từ giả thiết, ta có $2^{2020}$ là ước của $(n+1)\tron{4n^2+2n+1}.$ Ta nhận thấy $4n^2+2n+1$ là số lẻ, do đó $$\left(4n^2+2n+1,2^{2021}\right)=1.$$ Như vậy, phép chia hết trong giả thiết tương đương với
    $$2^{2021}\mid (n+1).$$
Kết quả, tất cả các số tự nhiên $n$ cần tìm có dạng $2^{2020}k-1,$ ở đây $k$ nguyên dương.}
\end{bx}

\begin{bx}
Tìm tất cả các số hữu tỉ dương $a,b$ sao cho $$a+b\:\text{ và }\:\dfrac{1}{a}+\dfrac{1}{b}$$ đồng thời là các số nguyên dương.
\nguon{Tạp chí Toán Tuổi thơ}
\loigiai{
Ta đặt $a=\dfrac{x}{m},\ b=\dfrac{y}{n},$ trong đó $(x,m)=(y,n)=1.$ Phép đặt này cho ta số sau đây nguyên
$$a+b=\dfrac{x}{m}+\dfrac{y}{n}=\dfrac{xn+ym}{mn}.$$
Từ đây và điều kiện $(x,m)=(y,n)=1,$ ta lần lượt suy ra
$$mn\mid\tron{xn+ym}\Rightarrow \heva{m\mid xn \\ n\mid ym}\Rightarrow \heva{m\mid n \\ n\mid m}\Rightarrow m=n.$$
Bằng cách làm tương tự, ta chỉ ra được $x=y.$ Kết hợp với $m=n,$ ta có $a=b.$ Thay trở lại giả thiết thì
$$2a\in \mathbb{Z}^+,\quad \dfrac{2}{a}\in\mathbb{Z}^+.$$
Đặt $2a=c,$ ta có $\dfrac{2}{a}=\dfrac{4}{2a}=\dfrac{4}{c}$ là số nguyên dương, kéo theo $c\in \{1;2;4\}.$ Dựa trên kết quả này, ta tìm ra $$(a,b)=\tron{\dfrac{1}{2},\dfrac{1}{2}},\quad (a,b)=(1,1),\quad (a,b)=(2,2)$$ là tất cả các cặp số hữu tỉ thỏa yêu cầu.}
\end{bx}


\subsection*{Bài tập tự luyện}


\begin{btt}
Cho hai số nguyên dương $x,y$ lớn hơn $1.$ Chứng minh rằng rằng nếu $x^{3}-y^{3}$ chia hết cho $x+y$ thì $x+y$ không là số nguyên tố.
\nguon{Chuyên Toán Phổ thông Năng khiếu 2017}
\end{btt}

\begin{btt} \label{bdscp1}
Cho $x, y$ là số nguyên dương sao cho $x^{2}+y^{2}-x$ chia hết cho $x y$. \\Chứng minh $x$ là số chính phương.
\nguon{Polish Mathematical Olympiad 2000}
\end{btt}

\begin{btt}
Tìm tất cả các số nguyên $m,n$ lớn hơn $1$ thỏa mãn $mn-1$ là ước của $n^3-1.$
\nguon{International Mathematical Olympiad 1998}
\end{btt}

\begin{btt}
Tìm tất cả các số nguyên dương $x,y$ thỏa mãn $xy>1$ và $x^3+x$ chia hết cho $xy-1.$
\end{btt}


\begin{btt}
Tìm tất cả các số nguyên dương $(x,y)$ thỏa mãn $x^2y+x$ chia hết cho $xy^2+7.$
\nguon{Korea Junior Mathematical Olympiad 2014}
\end{btt}

\begin{btt}
Tìm tất cả các số nguyên dương $a,b$ thỏa mãn $a+b^3$ và $a^3+b$ cùng chia hết cho $a^2+b^2.$
\end{btt}

\begin{btt}
Cho $a,b$ là các số nguyên tố cùng nhau. Chứng minh rằng $\dfrac{2 a\left(a^{2}+b^{2}\right)}{a^{2}-b^{2}}$ không phải là số nguyên.
\nguon{Thailand Mathematical Olympiad 2019}
\end{btt}

\begin{btt}
Tìm tất cả các số nguyên tố $p>2$ sao cho $\dfrac{(p+2)^{p+2}(p-2)^{p-2} - 1}{(p+2)^{p-2}(p-2)^{p+2}- 1}$ là một số nguyên.
\nguon{Tạp chí Pi, tháng 9 năm 2017}
\end{btt}

\begin{btt}
Cho các số nguyên dương $x,y$ khác $-1$ thỏa mãn
$$\dfrac{x^4-1}{y+1}+\dfrac{y^4-1}{x+1}$$
là số nguyên. Chứng minh rằng $x^4y^{44}-1$ cho hết cho $y+1.$
\nguon{Vietnam Mathematical Olympiad 2007}
\end{btt}

\begin{btt}
Tìm tất cả các bộ $ 3 $ số hữu tỉ dương $ a, b, c $ sao cho $a+\dfrac{1}{b},\ b + \dfrac{1}{c}$ và $ c + \dfrac{1}{a} $ đều là các số nguyên.	
\nguon{Tạp chí Pi tháng 1 năm 2017}
\end{btt}

\begin{btt}
Tìm bộ các số nguyên dương $x,y,z$ thỏa mãn điều kiện $xy+1,\ yz+1$ và $zx+1$ lần lượt chia hết cho $z,x,y.$
\end{btt}

\begin{btt}
Tìm tất cả các bộ ba số nguyên dương $x,y,z$ đôi một nguyên tố cùng nhau thỏa mãn đồng thời các điều kiện
\[\heva{(x+1)(y+1)&\equiv 1\pmod{z}\\
    (y+1)(z+1)&\equiv 1\pmod{x}\\    (z+1)(x+1)&\equiv 1\pmod{y}.}\]
\nguon{Chuyên Đại học Sư phạm Hà Nội 2007 $-$ 2008}
\end{btt}
\subsection*{Hướng dẫn bài tập tự luyện}
\begin{gbtt}
Cho hai số nguyên dương $x,y$ lớn hơn $1.$ Chứng minh rằng rằng nếu $x^{3}-y^{3}$ chia hết cho $x+y$ thì $x+y$ không là số nguyên tố.
\nguon{Chuyên Toán Phổ thông Năng khiếu 2017}
\loigiai{
Ta sẽ chứng minh bài toán bằng phản chứng. Giả sử $x+y$ là số nguyên tố.\\
Do $x+y$ là ước của cả $x^3+y^3$ và $x^3-y^3$ nên nó cũng là ước của
\[\tron{x^3+y^3}+\tron{x^3-y^3}=2x^3.\]
Vì $x+y$ là số nguyên tố lớn hơn $2$ nên $(x+y)\mid x^3.$ Tiếp tục sử dụng điều kiện $x+y$ nguyên tố, ta có $(x+y)\mid x$. Đây là điều không thể xảy ra do $x<x+y.$ Giả sử phản chứng là sai. Chứng minh hoàn tất.}
\end{gbtt}

\begin{gbtt} \label{bdscp1}
Cho $x, y$ là số nguyên dương sao cho $x^{2}+y^{2}-x$ chia hết cho $x y$. \\Chứng minh $x$ là số chính phương.
\nguon{Polish Mathematical Olympiad 2000}
\loigiai{
Gọi ước chung lớn nhất của $x$ và $y$ là $d.$ Ta đặt $x=dm,y=dn,$ với $(m,n)=1.$ Phép đặt này cho ta

\[\heva{x^2+y^2-x&=d^2m^2+d^2n^2-dm  \\ xy&=d^2mn.}\]
Kết hợp với giả thiết, ta được 
\[d^2mn\mid \left(d^2m^2+d^2n^2-dm\right) \Rightarrow dmn\mid \left(dm^2+dn^2-m\right).\tag{*}\label{pmo2000}\]
Kết hợp (\ref{pmo2000}) với việc xét tính chia hết cho $m$ và $d$ ở cả số bị chia và số chia, ta lần lượt suy ra
$$\heva{&m\mid dn^2 \\ &d\mid m}\Rightarrow \heva{&m\mid d \\ &d\mid m}\Rightarrow d=m.$$
Ta có $x=dm=d^2$ là số chính phương. Bài toán được chứng minh.}
\begin{luuy}
Phép gọi ước chung trong bài toán trên giúp chúng ta tận dụng tính chia hết dựa trên các điều kiện về tính nguyên tố cùng nhau.
\end{luuy}
\end{gbtt}

\begin{gbtt}
Tìm tất cả các số nguyên $m,n$ lớn hơn $1$ thỏa mãn $mn-1$ là ước của $n^3-1.$
\nguon{International Mathematical Olympiad 1998}
\loigiai{
Với các số nguyên $m,n$ thỏa yêu cầu, ta $mn-1$ là ước của
$$n^3-1-\tron{mn-1}=n\tron{m^2-n}.$$
Do $(n,mn-1)=1$ nên $m^2-n$ chia hết cho $mn-1.$ Theo đó, $mn-1$ cũng là ước cửa
$$n\tron{m^2-n}=m^2n-n^2=mn(m-1)+m-n^2.$$
Ta suy ra $n^2-m$ cũng chia hết cho $mn-1.$ Vai trò tương đương của $m$ và $n$ được chứng tỏ. Không mất tính tổng quát, ta giả sử $m\ge n.$ Ta có
$$1-mn<1-m<n^2-m\le mn-m<mn-1.$$
Phép so sánh trên kết hợp với lập luận $n^2-m$ chia hết cho $mn-1$ cho ta $m=n^2.$ Thử trực tiếp, ta thấy thỏa mãn. Như vậy tất cả các cặp $(m,n)$ thỏa yêu cầu là $\tron{k,k^2}$ và $\tron{k^2,k},$ trong đó $k$ là số nguyên dương lớn hơn $1.$}
\end{gbtt}

\begin{gbtt}
Tìm tất cả các số nguyên dương $x,y$ thỏa mãn $xy>1$ và $x^3+x$ chia hết cho $xy-1.$
\loigiai{
Với các số nguyên dương $x,y$ thỏa yêu cầu, ta lần lượt suy ra
$$(xy-1)\mid x\tron{x^2+1}
\Rightarrow (xy-1)\mid\tron{x^2+1}
\Rightarrow (xy-1)\mid\tron{x^2y+y}
\Rightarrow (xy-1)\mid\tron{x+y}.$$
Phép chia hết kể trên cho ta $xy-1\le x+y.$ Biến đổi bất đẳng thức này, ta có
$$xy-x-y-1\le 0\Leftrightarrow (x-1)(y-1)\le 2.$$
Theo như đánh giá ấy, một trong hai số $x,y$ phải bằng $1,$ hoặc $$(x,y)\in \{(2,2);(2,3);(3,2)\}.$$ Ta xét các trường hợp kể trên.
\begin{center}
    \begin{tabular}{c|c|c}
        $\quad (x,y)\quad$ & $\quad$ Trạng thái chia hết  $\quad$  & $\quad$ Kiểm tra $\quad$ \\
    \hline
        $(1,y)$ & $\tron{y-1}\mid 2$ & $(x,y)=(1,2),(1,3)$\\
    \hline
        $(x,1)$ & $(x-1)\mid\tron{x^3+x}$ & $(x,y)=(2,1),(3,1)$\\  
    \hline
        $(2,2)$ & $3\mid 10$ & Loại\\
    \hline
        $(2,3)$ & $5\mid 10$ & Chọn\\     
    \hline        
        $(3,2)$ & $5\mid 30$ & Chọn\\
    \end{tabular}
\end{center} 
Như vậy, có $6$ cặp $(x,y)$ thỏa yêu cầu, đó là
$(1,2),(1,3),(2,1),(3,1),(2,3),(3,2).$
}
\end{gbtt}

\begin{gbtt}
Tìm tất cả các số nguyên dương $(x,y)$ thỏa mãn $x^2y+x$ chia hết cho $xy^2+7.$
\nguon{Korea Junior Mathematical Olympiad 2014}
\loigiai{Giả sử tồn tại $x,y$ nguyên dương thỏa mãn đề bài. Với giả sử như vậy, ta chỉ ra $xy^2+7$ là ước của
$$x^2y+x=x\tron{xy+1}.$$
Ta nhận thấy $\tron{xy^2+7,x} \in \left\{1,7\right\}$. Ta xét các trường hợp sau.
\begin{enumerate}
    \item Với $\tron{x^2y+7,x}=1,$ ta có
    $\tron{xy^2+7}\mid(xy+1).$\\
    Do $xy+1>xy^2+7$ với mọi $x,y$ nguyên dương, bất đẳng thức trên không thể xảy ra.
    \item Với $\tron{x^2y+7,x}=7,$ ta có $x$ chia hết cho $7.$ Đặt $x=7z.$ Ta có
    $$\tron{7zy^2+7}\mid\vuong{(7z)^2y+7z}
    \Rightarrow \tron{zy^2+1}\mid z\tron{7zy+1}.$$
    Do $\tron{zy^2+1,z}=1,$ ta có $7zy+1$ chia hết cho $zy^2+1.$ Lúc này
    $$zy^2+1\le 7yz+1\Rightarrow zy^2\le 7yz\Rightarrow y\le 7.$$
    Tới đây, ta lập được bảng như sau
    \begin{center}
        \begin{tabular}{c|c|c|c}
            $\quad y\quad$ & $\quad$ Trạng thái chia hết  $\quad$  & $\quad z\quad $ & $\quad x\quad $\\
        \hline
            $1$ & $\tron{7z+1}\mid\tron{7z^2+1}$ & $1,2,5$ & $7,14,35$\\
        \hline
            $2$ & $\tron{28z+1}\mid\tron{7z^2+1}$ & $1$ & $7$\\  
        \hline
            $3$ & $\tron{63z+1}\mid\tron{7z^2+1}$ & $\not\in\mathbb{N}^*$ & $\not\in\mathbb{N}^*$ \\  
        \hline
            $4$ & $\tron{112z+1}\mid\tron{7z^2+1}$ & $\not\in\mathbb{N}^*$ & $\not\in\mathbb{N}^*$ \\          
        \hline        
            $5$ & $\tron{175z+1}\mid\tron{7z^2+1}$ & $\not\in\mathbb{N}^*$ & $\not\in\mathbb{N}^*$ \\   
        \hline        
            $6$ & $\tron{252z+1}\mid\tron{7z^2+1}$ & $\not\in\mathbb{N}^*$ & $\not\in\mathbb{N}^*$ \\   
        \hline        
            $7$ & $\tron{343z+1}\mid\tron{7z^2+1}$ & $z$ & $7z$
        \end{tabular}
    \end{center}    
\end{enumerate}
Kết luận, tất cả các cặp $(x,y)$ thỏa yêu cầu là $$(7,1),\ (7,2),\ (14,1),\ (35,1)$$ và dạng tổng quát $(7z,7),$ trong đó $z$ là số nguyên dương tùy ý.}
\end{gbtt}
\begin{gbtt}
Tìm tất cả các số nguyên dương $a,b$ thỏa mãn $a+b^3$ và $a^3+b$ cùng chia hết cho $a^2+b^2.$
\loigiai{
Ta đặt $d=(a,b),$ như vậy tồn tại các số nguyên dương $m,n$ sao cho $(m,n)=1,a=dm,b=dn.$ Khi đó, $$a^2+b^2=(dm)^2+(dn)^2=d^2\tron{m^2+n^2}$$ là ước của
$a+b^3=dm+(dn)^3=d\tron{m+d^2n}.$\\
Ta suy ra $d\tron{m^2+n^2}$ là ước của $m+d^2n,$ thế nên $n$ chia hết cho $d.$ Tương tự, $m$ chia hết cho $d,$ nhưng do $(m,n)=1$ nên $d=1.$ Cũng từ giả thiết $a^3+b$ và $a+b^3$ đều là bội của $a^2+b^2,$ ta có
$$\left(a^3+b\right)-\left(a+b^3\right)=\tron{a-b}\tron{a^2+ab+b^2-1}$$
chia hết cho $a^2+b^2.$ Tiếp theo, đặt $d'=\tron{a-b,a^2+b^2},$ và ta có
\begin{align*}
    \heva{&d'\mid (a-b) \\ &d'\mid\tron{a^2+b^2}}
    \Rightarrow \heva{&a\equiv b\pmod{d'}\\ &d'\mid\tron{a^2+b^2}}
    \Rightarrow \heva{&d\mid 2a^2 \\ &d\mid 2b^2}
    \Rightarrow d\mid 2\tron{a,b}^2\Rightarrow d\in \{1;2\}.
\end{align*}
Tới đây, ta xét các trường hợp sau.
\begin{enumerate}
    \item Nếu $a=b,$ ta có $a+b^3=a+a^3$ chia hết cho $a^2+b^2=2a^2.$ Ta dễ dàng tìm ra $a=b=1$ từ đây.
    \item Nếu $\tron{a-b,a^2+b^2}=1,$ ta có
    $\tron{a^2+b^2}\mid \tron{a^2+ab+b^2-1}.$ Bằng nhận xét
    $$a^2+b^2\le a^2+ab+b^2-1\le a^2+\dfrac{a^2+b^2}{2}+b^2-1=\dfrac{3}{2}\tron{a^2+b^2}-1<2\tron{a^2+b^2},$$
    ta chỉ ra $a^2+ab+b^2-1=a^2+b^2,$ và như vậy $a=b=1,$ mâu thuẫn điều kiện $\tron{a-b,a^2+b^2}=1.$
    \item Nếu $\tron{a-b,a^2+b^2}=2,$ ta có
    $\dfrac{a^2+b^2}{2}\mid \tron{a^2+ab+b^2-1}.$ Bằng nhận xét tương tự là
    $$2\tron{\dfrac{a^2+b^2}{2}}\le a^2+ab+b^2-1<3\tron{\dfrac{a^2+b^2}{2}},$$
    ta chỉ ra $a^2+ab+b^2-1=2\tron{\dfrac{a^2+b^2}{2}}$ hay $a=b=1,$ thỏa mãn.
\end{enumerate}
Kết luận, cặp $(a,b)=(1,1)$ là cặp số duy nhất thỏa mãn đề bài.}
\end{gbtt}

\begin{gbtt}
Cho $a,b$ là các số nguyên tố cùng nhau. Chứng minh rằng $\dfrac{2 a\left(a^{2}+b^{2}\right)}{a^{2}-b^{2}}$ không phải là số nguyên.
\nguon{Thailand Mathematical Olympiad 2019}
%mà khoan, em thiếu gỉa sử phản chứng
\loigiai{Ta giả sử phản chứng rằng $\dfrac{2 a\left(a^{2}+b^{2}\right)}{a^{2}-b^{2}}$ là số nguyên. Đặt $\tron{a,a^2-b^2}=d_1$, phép đặt này cho ta
$$\heva{d_1&\mid a\\d_1&\mid \tron{a^2-b^2}}\Rightarrow 
\heva{d_1&\mid a^2 \\ d_1&\mid \tron{a^2-b^2}}\Rightarrow d_1\mid a^2-\tron{a^2-b^2}\Rightarrow d_1\mid b^2.$$
Từ đây, ta suy ra $\tron{a,b^2}=d_1$. Kết hợp giả thiết $(a,b)=1,$ ta lại có $\tron{a,b}=1$. Bắt buộc, $d_1$ phải bằng $1$.\\
Ta tiếp tục đặt $\tron{a^2+b^2, a^2-b^2}=d_2$, và ta sẽ có
$$\heva{d_2\mid \tron{a^2+b^2}\\d_2\mid \tron{a^2-b^2}}\Rightarrow \heva{d_2\mid \tron{a^2+b^2}+\tron{a^2-b^2}\\d_2\mid \tron{a^2+b^2}-\tron{a^2-b^2} }\Rightarrow\heva{d_2&\mid2a^2\\d_2&\mid 2b^2.}$$
Do đó, $\tron{2a^2,2b^2}=d_2$. Bằng lập luận tương tự, ta thu được $d\in \left\{1,2\right\}$. Ta xét các trường hợp sau.
\begin{enumerate}
    \item Với $d_2=1$, ta có $\dfrac{2 a\left(a^{2}+b^{2}\right)}{a^{2}-b^{2}}$ là số nguyên chỉ khi $a^2-b^2$ là ước của $2.$\\
    Ta dễ dàng nhận thấy không có số nguyên $a,b$ thỏa mãn trường hợp này.
    \item  Với $d_2=2$, ta đặt $a^2=2x$ và $b^2=2y$ trong đó $\tron{x,y}=1$. Phép đặt kể trên cho ta biết $$\dfrac{2 a\left(a^{2}+b^{2}\right)}{a^{2}-b^{2}}=\dfrac{2\tron{x+y}}{x-y}\in \mathbb{Z}.$$ 
    Do $\tron{x+y,x-y}=1$ nên $x-y$ là ước của $2.$ Ta xét trường hợp sau.
    \begin{itemize}
        \item Với $x-y=1$, ta có $a^2-b^2=2$. Ta nhận thấy không có $a,b$ nguyên thỏa mãn khả năng này.
        \item Với $x-y=2$, ta có $a^2-b^2=4$. Ta suy ra $a=2$ và $b=0$, mâu thuẫn với điều kiện bài toán.
    \end{itemize}
\end{enumerate}
%phần này kết luận chưa tốt, phải là "Vậy ... không phải là số nguyên với mọi cặp (a,b) nguyên tố cùng nhau". Nêú không muốn viết dài, em có thể viết là
Như vậy, giả sử phản chứng là sai. Bài toán được chứng minh.
}
\end{gbtt}

\begin{gbtt}
Tìm tất cả các số nguyên tố $p>2$ sao cho $$\dfrac{(p+2)^{p+2}(p-2)^{p-2} - 1}{(p+2)^{p-2}(p-2)^{p+2}- 1} $$ là một số nguyên. 
\nguon{Tạp chí Pi, tháng 9 năm 2017}
\loigiai
{Ta đặt $T=(p+2)^{p+2}(p-2)^{p-2} - 1,M=(p+2)^{p-2}(p-2)^{p+2}- 1.$ \\
Với giả sử $T$ chia hết cho $M$, ta nhận thấy $M$ cũng là ước của
$$T-M= (p+2)^{p-2}(p-2)^{p-2} \left((p+2)^4 - (p-2)^4 \right).$$ 
Với điều hiển nhiên là $\tron{M, (p+2)^{p-2}(p-2)^{p-2}}=1,$ từ trên ta chỉ ra
$$(p+2)^4 - (p-2)^4$$
chia hết cho $M.$ Với hai số nguyên dương $A,B$ nếu $A$ chia hết cho $B$ thì $A\ge B.$ Kết quả này cho ta
 $$M \leq (p+2)^4 - (p-2)^4 \leq (p+2)^4 - 1\Rightarrow (p+2)^{p-6} \cdot (p-2)^{p+2} \leq 1.$$
Do vậy, $p<7,$ vì với $p\leq 7$, hiển nhiên ta có $(p+2)^{p-6}(p-2)^{p+2} > 1.$\\ Thử trực tiếp $p=2,3,5,$ ta kết luận chỉ có $p=3$ là số nguyên tố thỏa yêu cầu.}
\begin{luuy}
\nx 	
Trong lời giải trên, tính nguyên tố của $p$ chỉ được sử dụng khi kiểm tra các giá trị $p<7$. Vì vậy, khi thay đổi yêu cầu đã cho "tìm tất cả các số nguyên tố $p>2$" bởi yêu cầu "tìm tất cả các số nguyên $p>2$", bài toán vẫn có hướng giải quyết tương tự.
\end{luuy}
\end{gbtt}

\begin{gbtt}
Cho các số nguyên dương $x,y$ khác $-1$ thỏa mãn
$$\dfrac{x^4-1}{y+1}+\dfrac{y^4-1}{x+1}$$
là số nguyên. Chứng minh rằng $x^4y^{44}-1$ cho hết cho $y+1.$
\nguon{Vietnam Mathematical Olympiad 2007}
\loigiai{
Đặt $\dfrac{a}{m}=\dfrac{x^4-1}{y+1},$ $\dfrac{b}{n}=\dfrac{y^4-1}{x+1}$ với $a,b,m,n$ là các số nguyên dương thỏa mãn
$$\tron{a,m}=1=\tron{b,n}=1.$$
Ta có $\dfrac{x^4-1}{y+1}+\dfrac{y^4-1}{x+1}=\dfrac{a}{m}+\dfrac{b}{n}=\dfrac{an+bm}{mn}.$ Nhờ điều kiện $\tron{a,m}=\tron{b,n}=1,$ ta suy ra 
$$mn\mid (an+bm)\Rightarrow\heva{m&\mid an\\n&\mid bm}\Rightarrow\heva{m&\mid n\\n&\mid m}\Rightarrow n=m.$$
Ngoài ra, khi lấy tích hai số hạng trong tổng đã cho, ta được
$$\dfrac{ab}{mn}=\tron{\dfrac{x^4-1}{y+1}}\tron{\dfrac{y^4-1}{x+1}}=\tron{x-1}\tron{x^2+1}\tron{y-1}\tron{y^2+1}\in \mathbb{Z}.$$
Nhận xét trên kết hợp với $m=n$ cho ta $ab$ chia hết cho $m^2,$ nhưng vì $(a,m)=1$ nên $m=n=1.$ \\Ta có $x^4-1$ chia hết cho $y+1$ và như vậy
$$x^4y^{44}-1\equiv y^{44}-1\equiv(-1)^{44}-1\equiv 0\pmod{y+1}.$$
Đồng dư thức trên chứng tỏ $x^4y^{44}-1$ chia hết cho $y+1.$ Bài toán được chứng minh.}
\end{gbtt}

\begin{gbtt}
Tìm tất cả các bộ $ 3 $ số hữu tỉ dương $ a, b, c $ sao cho 
$$a+\dfrac{1}{b},\ b + \dfrac{1}{c}\text{ và } c + \dfrac{1}{a} $$ đều là các số nguyên.	
\nguon{Tạp chí Pi tháng 1 năm 2017}
\loigiai
{
Đặt $ a = \dfrac{m}{x},\ b = \dfrac{n}{y}, \ c = \dfrac{p}{z} $, với $m,n,n,y,z $ là các số nguyên dương và 
$$(m,x)=(n,y)=(p,z)=1. $$
Tính toán trực tiếp, ta được
$a + \dfrac{1}{b} = \dfrac{m}{x} + \dfrac{y}{n} = \dfrac{mn + xy}{nx}.$\\
Do giả thiết $a+\dfrac{1}{b}$ là số nguyên và điều kiện $(m,x)=(n,y)=(p,z)=1,$ ta có
$$nx\mid  (mn+xy) 
\Rightarrow \heva{&n\mid xy \\ &x\mid mn}
\Rightarrow \heva{&n\mid x \\ &x\mid n}\Rightarrow n=x.$$
Hoàn toàn tương tự, ta chỉ ra $n=x,p=y,m=z.$
Ta viết lại ràng buộc của bài toán thành
$$y\mid (x+z),\quad z\mid (y+x),\quad x\mid (z+y).$$
Không mất  tổng quát, giả sử $ x \geq y \geq t $. Khi đó,  ta có $ 0 < \dfrac{z + y}{x } \leq 2$. Ta xét các trường hợp dưới đây.
\begin{enumerate}
	\item Với $y+z=2x,$ hiển nhiên $x=y=z$ lúc này. Bộ số thu được ở đây là $(a,b,c)=(1,1,1).$
	\item Với $y+z=x,$ từ $y\mid x+z=2z-y,$ ta suy ra $y\mid 2z,$ lại do $y\ge z$ nên $y=z$ hoặc $y=2z.$
	\begin{itemize}
		\item\chu{Trường hợp 1.} Với $y=z,$ ta có $x=2y.$ Bộ số thu được ở đây là $(a,b,c)=\left(\dfrac{1}{2},2,1\right).$
		\item\chu{Trường hợp 2.} Với $y=2z,$ ta có $x=3y.$ Bộ số thu được ở đây là $(a,b,c)=\left(\dfrac{3}{2},2,\dfrac{1}{3}\right).$
	\end{itemize}	
\end{enumerate}	
Chú ý rằng, nếu bộ ba số $ (a_0, b_0, c_0) $ thỏa mãn điều kiện đề bài thì mỗi bộ ba số nhận được từ nó nhờ phép hoán vị vòng quanh cũng là một bộ ba số thỏa mãn điều kiện đề bài. Vì vậy, từ các kết quả thu được ở trên, ta thấy có tất cả $ 7 $ bộ ba số hữu tỉ dương thỏa mãn điều kiện đề bài, đó là $$ (1,1,1), \left ( \dfrac{3}{2}, 2, \dfrac{1}{3} \right ), \left ( 2, \dfrac{1}{3}, \dfrac{3}{2} \right ), \left (\dfrac{1}{3}, \dfrac{3}{2},2 \right ), \left ( 2,1, \dfrac{1}{2} \right ), \left ( 1,\dfrac{1}{2},2 \right ) \text{ và } \left ( \dfrac{1}{2}, 2, 1 \right ). $$}
\end{gbtt}

\begin{gbtt}
Tìm bộ các số nguyên dương $x,y,z$ thỏa mãn điều kiện $xy+1,\ yz+1$ và $zx+1$ lần lượt chia hết cho $z,x,y.$
\loigiai{
Giả sử $x$ và $z$ tồn tại một ước chung là $p,$ khi đó $xy+1$ chia hết cho $p$ và $xy$ chia hết cho $p.$ Hai khẳng định này mâu thuẫn với nhau, chứng tỏ $(x,z)=1.$ Bằng lập luận tương tự, ta chỉ ra
$$(x,y)=(y,z)=(z,x)=1.$$
Tiếp theo, ta suy ra được các điều sau đây từ giả thiết
\begin{align*}
    z&\mid (xy+1)+z(x+y)=xy+yz+zx+1,\\
    y&\mid (zx+1)+y(z+x)=xy+yz+zx+1,\\
    x&\mid (yz+1)+y(z+x)=xy+yz+zx+1.
\end{align*}
Kết hợp nhận xét trên với chứng minh $x,y,z$ đôi một nguyên tố cùng nhau, ta chỉ ra
$$xyz\mid (xy+yz+zx+1).$$
Lập luận trên hướng ta tới việc xét thương của $xy+yz+zx+1$ và $xyz.$ Ta có
$$\dfrac{xy+yz+zx+1}{xyz}=\dfrac{1}{x}+\dfrac{1}{y}+\dfrac{1}{z}+\dfrac{1}{xyz}\le 4.$$
Không mất tổng quát, ta giả sử $x\ge y\ge z.$ Do $\dfrac{xy+yz+zx+1}{xyz}$ là số nguyên, ta xét các trường hợp sau.
\begin{enumerate}
    \item Nếu $xy+yz+zx+1=4xyz,$ dấu bằng ở đánh giá bên trên phải xảy ra, tức $x=y=z=1.$
    \item Nếu $xy+yz+zx+1=3xyz,$ ta chứng minh $z=1.$ Thật vậy, nếu $x\ge y\ge z\ge 2,$ ta có
    $$\dfrac{1}{x}+\dfrac{1}{y}+\dfrac{1}{z}+\dfrac{1}{xyz}\le \dfrac{1}{2}+\dfrac{1}{2}+\dfrac{1}{2}+\dfrac{1}{2\cdot2\cdot2}=\dfrac{13}{8}<3,$$
    một điều mâu thuẫn. Mâu thuẫn này chứng tỏ $z=1.$ Thế ngược lại $z=1,$ ta được
    $$xy+x+y+1=3xy\Leftrightarrow 2xy-x-y-1=0\Leftrightarrow (2x-1)(2y-1)=3.$$
    Giải phương trình ước số trên, ta suy ra $(x,y)=(2,1).$
    \item Nếu $xy+yz+zx+1=2xyz,$ lập luận tương tự, ta tìm ra $z=1.$
    Thế ngược lại $z=1,$ ta có
    $$xy+x+y+1=2xy\Leftrightarrow xy-x-y-1=0\Leftrightarrow (x-1)(y-1)=2.$$
    Giải phương trình ước số trên, ta thu được $(x,y)=(3,2).$    
    \item Nếu $xy+yz+zx+1=xyz,$ lập luận tương tự, ta tìm ra $z\le 3.$
    \begin{itemize}
        \item \chu{Trường hợp 1.} Nếu như $z=3,$ ta có
        \begin{align*}
            xy+3x+3y+1=3xy&\Leftrightarrow 2xy-3x-3y-1=0\\&\Leftrightarrow (2x-3)(2y-3)=11.
        \end{align*}
        Giải phương trình ước số trên, ta thu được $(x,y)=(7,2),$ mâu thuẫn với giả sử $y\ge z.$
        \item \chu{Trường hợp 2.} Nếu như $z=2,$ ta có
        \begin{align*}
            xy+2x+2y+1=2xy&\Leftrightarrow xy-2x-2y-1=0\\&\Leftrightarrow (x-2)(y-2)=5.
        \end{align*}
        Giải phương trình ước số trên, ta thu được $(x,y)=(7,3).$    
        \item \chu{Trường hợp 3.} Nếu như $z=1,$ ta có
        $$xy+x+y+1=xy\Leftrightarrow x+y+1=0.$$
        Phương trình trên không có nghiệm nguyên dương.    
    \end{itemize}
\end{enumerate}
Tổng kết lại, có tất cả $16$ bộ $(x,y,z)$ thỏa yêu cầu bài toán, bao gồm $$(1,1,1),\ (1,1,2),\ (1,2,3),\ (2,3,7)$$ và tất cả các hoán vị của chúng.}
\begin{luuy}
Ở bước chỉ ra
\begin{align*}
    z&\mid (xy+1)+z(x+y)=xy+yz+zx+1,\\
    y&\mid (zx+1)+y(z+x)=xy+yz+zx+1,\\
    x&\mid (yz+1)+y(z+x)=xy+yz+zx+1,
\end{align*}
ta đã tạo ra một \chu{số bị chia chung}, trong khi số chia có thể được tăng lên nhờ vào tính nguyên tố cùng nhau. Bằng cách này, ta có thể so sánh số bị chia và số chia, để rồi cho ra được các kết quả của $x,y,z.$
\end{luuy}
\end{gbtt}

\begin{light}
Một bài toán khác liên quan tới các tổng $xy+1,yz+1$ và $zx+1$ lần đầu xuất hiện trên tạp chí \chu{Mathematics Magazine} số 71, được xuất bản vào tháng 2 năm 1988 và được giới thiệu bởi tác giả \chu{Kiran S. Kedlaya}.
\begin{center}
    \includegraphics[scale=0.55]{magazine.png}
\end{center}
Lời giải của tác giả cho bài toán này vô cùng tinh tế và thú vị, các bạn có thể tham khảo ở đây:
\url{https://www.jstor.org/stable/2691347}
\end{light}

\begin{gbtt}
Tìm tất cả các bộ ba số nguyên dương $x,y,z$ đôi một nguyên tố cùng nhau thỏa mãn đồng thời các điều kiện
\[\heva{(x+1)(y+1)&\equiv 1\pmod{z}\\
    (y+1)(z+1)&\equiv 1\pmod{x}\\    (z+1)(x+1)&\equiv 1\pmod{y}.}\]
\nguon{Chuyên Đại học Sư phạm Hà Nội 2007 $-$ 2008}
\loigiai{
Từ giả thiết, ta suy ra được
$$\heva{(x+1)(y+1)(z+1)\equiv z+1&\equiv1\pmod{z}\\
(x+1)(y+1)(z+1)\equiv x+1&\equiv1\pmod{x}\\
(x+1)(y+1)(z+1)\equiv y+1&\equiv1\pmod{y}.}$$
Từ đây, ta có $(x+1)(y+1)(z+1)-1$ chia hết cho $\vuong{x,y,z}$. Kết hợp với giả thiết $x,y,z$ đôi một nguyên tố cùng nhau, ta thu được $xyz\mid(x+1)(y+1)(z+1)-1.$ Biến đổi ta được
$$xyz\mid \tron{xyz+xy+yz+xz+x+y+z}\Rightarrow xyz\mid \tron{xy+yz+xz+x+y+z}.$$
Lập luận trên hướng ta đến việc xét thương của $xy+yz+xz+x+y+z$ và $xyz$. Ta có
$$\dfrac{xy+yz+xz+x+y+z}{xyz}=\dfrac{1}{z}+\dfrac{1}{x}+\dfrac{1}{y}+\dfrac{1}{yz}+\dfrac{1}{xz}+\dfrac{1}{xy}\le 6.$$
Ta giả sử $x\ge y\ge z.$ Do $\dfrac{xy+yz+zx+x+y+z}{xyz}$ là số nguyên, ta xét các trường hợp sau.
\begin{enumerate}
    \item Với $xy+yz+zx+x+y+z=6xyz,$ dấu bằng ở đánh giá xảy ra, tức $x=y=z=1.$
    \item Với $xy+yz+zx+x+y+z=5xyz,$ nếu như $x\ge y\ge z\ge2,$ ta có
    $$\dfrac{1}{z}+\dfrac{1}{x}+\dfrac{1}{y}+\dfrac{1}{yz}+\dfrac{1}{xz}+\dfrac{1}{xy}\le \dfrac{1}{2}+\dfrac{1}{2}+\dfrac{1}{2}+\dfrac{1}{2\cdot2}+\dfrac{1}{2\cdot2}+\dfrac{1}{2\cdot2}=\dfrac{9}{4}<5.$$
    một điều mâu thuẫn. Mâu thuẫn này chứng tỏ $z=1.$ Thế ngược lại $z=1,$ ta được
    $$xy+y+x+x+y+1=5xy\Leftrightarrow 4xy-2x-2y=1.$$
    Điều này không thể xảy ra vì $4xy-2x-2y$ chia hết cho $2$ còn $1$ không chia hết cho $2$.
    \item Với $xy+yz+zx+x+y+z=4xyz,$ lập luận tương tự, ta suy ra $z=1.$ Thế ngược lại $z=1,$ ta có
    $$xy+y+x+x+y+1=4xy\Leftrightarrow 3xy-2x-2y-1=0\Leftrightarrow(3x-2)(3y-2)=7.$$
    Giải phương trình ước số trên, ta nhận được $(x,y)=(3,1).$
    \item Với $xy+yz+zx+x+y+z=3xyz,$ lập luận tương tự, ta suy ra $z=1.$ Thế ngược lại $z=1,$ ta có
    $$xy+y+x+x+y+1=4xy\Leftrightarrow 2xy-2x-2y=1.$$
    Điều này không thể xảy ra vì $2xy-2x-2y$ chia hết cho $2$ còn $1$ không chia hết cho $2$.
    \item Với $xy+yz+zx+x+y+z=4xyz,$ cách chặn tương tự cho ta $z\le 2.$
    \begin{itemize}
        \item\chu{Trường hợp 1.} Nếu như $z=1,$ ta có
        $$xy+y+x+x+y+1=2xy\Leftrightarrow xy-2x-2y-1=0\Leftrightarrow(x-2)(y-2)=5.$$
        Giải phương trình ước số trên, ta suy ra $(x,y)=(7,3).$
        \item\chu{Trường hợp 2.} Nếu như $z=2,$ ta có
         $$xy+2y+2x+x+y+2=4xy\Leftrightarrow 3xy-3x-3y=2.$$
        Điều này không thể xảy ra vì $3xy-3x-3y$ chia hết cho $3$ còn $1$ không chia hết cho $2$.
    \end{itemize}
    \item Với $xy+yz+zx+x+y+z=xyz,$ cách chặn tương tự cho ta $z\le3.$ \\Thử trực tiếp các trường hợp của $z$, ta tìm được $(x,y,z)=(14,4,2).$
\end{enumerate}
Như vậy, các bộ $(x,y,z)$ thỏa mãn đề bài là $(1,1,1),(3,1,1), (7,3,1),(14,4,2)$ và hoán vị.}
\end{gbtt}

\section{Phép đặt ước chung đôi một cho ba biến số}
    Trong một số bài toán cần xét tới tính chia hết của ba biến số, chẳng hạn như $a,b,c,$ rất nhiều bạn loay hoay tìm hướng giải quyết, nhưng rồi chẳng thể nhìn thấy mấu chốt của vấn đề. Nhằm giúp các bạn tháo gỡ khúc mắc này, tác giả xin phép đưa ra một cách đặt ẩn phụ như bên dưới. Mục này trong sách cũng chính là một phần bài viết của tác giả \text{\it Nguyễn Nhất Huy} được gửi lên tập san \text{\it Gặp gỡ toán học}.
\subsection*{Bài toán mở đầu} 
\begin{light}
\chu{Bài toán.} Cho ba số nguyên dương $a,b,c$ thỏa mãn $(a,b,c)=1.$ Chứng minh rằng tồn tại các số nguyên dương $x,y,z,m,n,p$ sao cho $a=myz,b=nxz,c=pxy,$ đồng thời
\begin{align*}
    (m,n)&=(n,p)=(p,m)=(x,y)=(y,z)\\&=(z,x)=(m,x)=(n,y)=(p,z)=1.
\end{align*}
\end{light}
\chu{Chứng minh.}
Đầu tiên, ta đặt
\[(a,b)=z,(b,c)=x,(c,a)=y.\tag{*}\label{dat3bien.1}\]
Ta sẽ chứng minh $(x,y)=1.$ Thật vậy, nếu $x$ và $y$ có ước chung, ta sẽ có
$$2\le ((b,c),(c,a))=(a,b,c)=1,$$
một điều mâu thuẫn. Mâu thuẫn này kết hợp với suy luận tương tự cho ta
$$(x,y)=(y,z)=(z,x)=1.$$
Ngoài ra, cách đặt ở (\ref{dat3bien.1}) còn cho ta $a$ chia hết cho $y$ và $z,$ nhưng vì $(y,z)=1$ nên $a$ chia hết cho $yz.$ \\Tới đây, sự tồn tại đã cho được chứng tỏ. \hfill $\square$
\begin{luuy}
\nx Ngoài những kết quả về ước trong bài toán, phép đặt trên còn cho ta những kết quả về bội chung nhỏ nhất, đó là
\begin{multicols}{2}
\begin{itemize}
    \item $[a,b]=mnxyz,$
    \item $[b,c]=npxyz,$
    \item $[c,a]=pmxyz,$   
    \item $[a,b,c]=mnpxyz.$    
\end{itemize}
\end{multicols}
\end{luuy}

\subsection*{Ví dụ minh họa}
\begin{bx}
Chứng minh rằng với mọi số nguyên dương $m,n,p,$ ta luôn có
\[\left( m,\left[ n,p \right] \right)=\left[ \left( m,n \right),\left( m,p \right) \right].\]
\loigiai{
Ta đặt $d=(m,n,p),$ khi đó tồn tại các số nguyên dương $M,N,P$ sao cho $(M,N,P)=1,$ đồng thời
$$m=dM,\quad n=dN,\quad p=dP.$$
Tiếp theo, ta đặt $M=abx,N=bcy,P=caz,$ ở đây
$$(a,b)=(b,c)=(c,a)=(x,y)=(y,z)=(z,x)=(a,y)=(b,z)=(c,x)=1.$$
Tính nguyên tố cùng nhau kể trên giúp ta nhận thấy
\begin{align*}
    VT&=\left( dabx,\left[ dbcy,dcaz \right] \right)=\left( dabx,dabcz \right)=dab,
    \\VP&=\left[ \left( dabx,dbcy \right),\left( dabx,dcaz \right) \right]=\left[ db,da \right]=dab.   
\end{align*}
Vế trái bằng vế phải. Đẳng thức được chứng minh.}
\end{bx}

\begin{bx}
Tìm tất cả các bộ số nguyên dương $a,b,c$ nguyên tố cùng nhau thỏa mãn $$\dfrac{a}{b}+\dfrac{b}{c}+\dfrac{c}{a} \text{ và } \dfrac{a}{c}+\dfrac{c}{b}+\dfrac{b}{a}$$ là hai số nguyên dương.
\loigiai{
Ta đặt $a=mnx,b=npy,c=pmz,$ trong đó
$$(m,n)=(n,p)=(p,m)=(x,y)=(y,z)=(z,x)=(m,y)=(n,z)=(p,x)=1.$$
Phép đặt này cho ta số sau đây nguyên dương
$$\dfrac{a}{b}+\dfrac{b}{c}+\dfrac{c}{a}=\dfrac{mnx}{npy}+\dfrac{npy}{pmz}+\dfrac{pmz}{mnx}=\dfrac{nz(mx)^2+px(ny)^2+my(pz)^2}{mnpxyz}.$$
Xét tính chia hết cho $mz$ ở cả tử và mẫu, ta chỉ ra
$mz\mid px(ny)^2.$ 
\begin{itemize}
    \item[i,] Điều kiện phép đặt $(m,y)=(m,n)=(m,p)$ cho ta $m\mid x.$
    \item[ii,] Điều kiện phép đặt $(z,x)=(z,y)=(z,n)$ cho ta $z\mid p.$
\end{itemize}
Một cách tương tự, ta chỉ ra được các phép chia hết là
$$m\mid x,\quad z\mid p,\quad n\mid y,\quad x\mid m,\quad p\mid z,\quad y\mid n.$$
Các nhận xét trên cho ta $x=m,y=n,z=p.$ \\
Bằng cách làm tương tự đối với điều kiện $\dfrac{a}{c}+\dfrac{c}{b}+\dfrac{b}{a}$ nguyên dương, ta chỉ ra thêm
$$x=n,\quad y=p,\quad z=n.$$
Tổng kết lại, ta có
$x=y=z=m=n=p.$ Theo đó, ta suy ra $a=b=c.$ \\Tuy nhiên, do điều kiện $(a,b,c)=1,$ bộ số duy nhất thỏa mãn đề bài chỉ có thể là $(a,b,c)=(1,1,1).$
}
\begin{luuy}
Bài toán trên vẫn có thể được tiến hành bằng cách làm tương tự trong trường hợp ba số $a,b,c$ không có ràng buộc phải nguyên tố cùng nhau.
\end{luuy}
\end{bx}


\subsection*{Bài tập tự luyện}
\begin{btt}
Chứng minh rằng với mọi số nguyên dương $m,n,p,$ ta luôn có
\[\left[ m,n,p \right]=\displaystyle\dfrac{mnp\left( m,n,p \right)}{\left( m,n \right)\left( n,p \right)\left( p,m \right)}.\]
\end{btt}

\begin{btt}
Cho các số nguyên dương $a,b,c$ thỏa mãn $\dfrac{1}{a}+\dfrac{1}{b}=\dfrac{1}{c}.$ \\ Chứng minh rằng $a^2+b^2+c^2$ là số chính phương.
\end{btt}

\begin{btt}
Cho $a,b,c$ là các số nguyên dương thỏa mãn $(a,b,c)=1$ và $a^3b^3+b^3c^3+c^3a^3$ chia hết cho $a^2b^2c^2$. Chứng minh rằng $abc$ là số chính phương.
\end{btt}

\begin{btt}
Cho $a, b, c$ là độ dài ba cạnh của một tam giác, $(a, b, c)=1$ và
$$\dfrac{a^{2}+b^{2}-c^{2}}{a+b-c}, \quad \dfrac{b^{2}+c^{2}-a^{2}}{b+c-a}, \quad \dfrac{c^{2}+a^{2}-b^{2}}{c+a-b}$$
đều là các số nguyên. Chứng minh rằng một trong hai số sau đây là số chính phương
$$(a+b-c)(b+c-a)(c+a-b),\quad 2(a+b-c)(b+c-a)(c+a-b).$$ 
\nguon{Czech and Slovak Olympiad 2018}
\end{btt}

\begin{btt}
Tìm các số nguyên dương $a,b,c$ sao cho $a^3+b^3+c^3$ chia hết cho $a^2b,b^2c$ và $c^2a.$
\nguon{Bulgaria Mathematical Olympiad 2001}
\end{btt}

\begin{btt}
Cho ba số nguyên dương $a,b,c$ đôi một phân biệt thỏa mãn $(a,b,c) = 1$ và 
$$a \mid (b - c)^2, \quad b \mid (c-  a)^2, \quad c \mid (a - b)^2.$$ Chứng minh rằng $a, b$  $c$ không là độ dài ba cạnh của một tam giác.
\nguon{Baltic Way 2015}
\end{btt}

\begin{btt}
Chứng minh rằng không tồn tại ba số nguyên dương $a,b,c$ nguyên tố cùng nhau nào thỏa mãn đồng thời các điều kiện 
\[2(a,b)+[a,b]=a^2,\quad 2(b,c)+[b,c]=b^2,\quad 2(c,a)+[c,a]=c^2.\]
\end{btt}

\begin{btt}
Tìm tất cả các bộ ba số tự nhiên $(m,n,p)$ thỏa mãn đồng thời các điều kiện
\[m + n = {\left( {m,n} \right)^2},\quad n + p= {\left( {n,p} \right)^2},\quad p+m = {\left( {p,m} \right)^2}.\]
\end{btt}

\begin{btt}
Tìm tất cả các số nguyên dương $a,b,c$ thỏa mãn \[[a,b,c]=\dfrac{ab+bc+ca}{4}.\]
\nguon{Junior Japanese Mathematical Olympiad 2019}
\end{btt}

\subsection*{Hướng dẫn bài tập tự luyện}
\begin{gbtt}
Chứng minh rằng với mọi số nguyên dương $m,n,p,$ ta luôn có
\[\left[ m,n,p \right]=\displaystyle\dfrac{mnp\left( m,n,p \right)}{\left( m,n \right)\left( n,p \right)\left( p,m \right)}.\]
\loigiai{
Ta đặt $d=(m,n,p),$ khi đó tồn tại các số nguyên dương $M,N,P$ sao cho $(M,N,P)=1,$ đồng thời
$$m=dM,\quad n=dN,\quad p=dP.$$
Tiếp theo, ta đặt $M=abx,N=bcy,P=caz,$ ở đây
$$(a,b)=(b,c)=(c,a)=(x,y)=(y,z)=(z,x)=(a,y)=(b,z)=(c,x)=1.$$
Bằng cách đặt này, đẳng thức đã cho trở thành
$$[abx,bcy,caz]=\dfrac{(abx)(bcy)(caz)(abx,bcy,caz)}{(abx,bcy)(bcy,caz)(caz,abx)}.$$
Đẳng thức trên là đúng do ta nhận xét được
\begin{itemize}
    \item[i,] $[abx,bcy,caz]=[abcxy,caz]=abcxyz.$
    \item[ii,] $(abx,bcy,caz)=(b,caz)=1.$
    \item[iii,] $(abx,bcy)=b,\quad (bcy,caz)=c,\quad (caz,abx)=a.$
\end{itemize}
Bài toán được chứng minh.}
\end{gbtt}

\begin{gbtt}
Cho các số nguyên dương $a,b,c$ thỏa mãn $\dfrac{1}{a}+\dfrac{1}{b}=\dfrac{1}{c}.$ \\ Chứng minh rằng $a^2+b^2+c^2$ là số chính phương.
\loigiai{Ta chỉ cần xét bài toán này trong trường hợp $(a,b,c)=1.$ \\
Theo như bổ đề, ta có thể đặt $a=mnx,b=npy,c=pmz,$ trong đó
$$(m,n)=(n,p)=(p,m)=(x,y)=(y,z)=(z,x)=(m,y)=(n,z)=(p,x)=1.$$
Đẳng thức đã cho trở thành $\dfrac{1}{{mnx}} + \dfrac{1}{{npy}} = \dfrac{1}{{pmz}}.$ Biến đổi tương đương, ta có
$$\dfrac{1}{{mnx}} + \dfrac{1}{{npy}} = \dfrac{1}{{pmz}} \Leftrightarrow \dfrac{{pzy + mzx}}{{nxy}} = 1 \Leftrightarrow  pzy+mzx=nxy.$$
Xét tính chia hết cho $y$ ở cả hai vế, ta chỉ ra $mzx$ chia hết cho $y,$ nhưng do điều kiện phép đặt
$$(m,y)=(y,z)=(x,y)=1$$ 
nên $y=1.$ Một cách tương tự, ta chứng minh được $x=y=1.$ \\
Thế $x=y=1$ vào $pzy+mzx=nxy,$ ta có
$$z(m+p)=n.$$
Do $(n,z)=1$ và $n$ chia hết cho $z,$ ta nhận được $z=1,$ đồng thời $n=m+p.$ Lúc ấy
$$a^2+b^2+c^2=\left(m^2+pm\right)^2+\left(p^2+mp\right)^2+(pm)^2=\tron{m^2+mp+p^2}^2.$$
Ta nhận được $a^2+b^2+c^2$ là số chính phương. Chứng minh hoàn tất.}
\end{gbtt}

\begin{gbtt} 
Cho $a,b,c$ là các số nguyên dương thỏa mãn $(a,b,c)=1$ và $a^3b^3+b^3c^3+c^3a^3$ chia hết cho $a^2b^2c^2$. Chứng minh rằng $abc$ là số chính phương.
\nguon{Diễn đàn Art Of Problem Solving}
\loigiai{ Vì $(a,b,c)=1$ nên ta có thể đặt $a=myz,b=nzx,c=pxy,$ ở đây
$$(m,n)=(n,p)=(p,m)=(x,y)=(y,z)=(z,x)=(m,x)=(n,y)=(p,z)=1.$$
Kết hợp với giả thiết $a^2b^2c^2 \mid \tron{a^3b^3+b^3c^3+c^3a^3}$ ta có 
$$m^2n^2p^2xyz \mid\left(m^3n^3z^3+n^3p^3x^3+m^3p^3y^3\right).$$
Xét tính chia hết cho $m^2$ ở cả tử và mẫu, ta chỉ ra được
$$m^2\mid n^3p^3x^3.$$
Nhờ vào điều kiện phép đặt $(m,n)=(m,p)=(m,x)=1,$ ta có $m=1.$ Hoàn toàn tương tự, ta chứng minh được $n=p=1.$ Với các kết quả thu được vừa rồi, ta nhận thấy 
$$abc=mnpx^2y^2z^2=x^2y^2z^2.$$
Số bên trên là số chính phương, và bài toán được chứng minh.}
\end{gbtt}

\begin{gbtt}
Cho $a, b, c$ là độ dài ba cạnh của một tam giác, $(a, b, c)=1$ và
$$\dfrac{a^{2}+b^{2}-c^{2}}{a+b-c}, \quad \dfrac{b^{2}+c^{2}-a^{2}}{b+c-a}, \quad \dfrac{c^{2}+a^{2}-b^{2}}{c+a-b}$$
đều là các số nguyên. Chứng minh rằng một trong hai số sau đây là số chính phương
$$(a+b-c)(b+c-a)(c+a-b),\quad 2(a+b-c)(b+c-a)(c+a-b).$$ 
\nguon{Czech and Slovak Olympiad 2018}
\loigiai{ Ta đặt $x=a+b-c, y=b+c-a, z=c+a-b.$ Phép đặt này cho ta  
$$a=\dfrac{1}{2}(z+x),\quad b=\dfrac{1}{2}(x+y),\quad c=\dfrac{1}{2}(y+z).$$
Bằng khai triển trực tiếp, ta chỉ ra
$$\dfrac{a^2+b^2-c^2}{a+b-c}=\dfrac{x^2+xy+zx-yz}{2x}.$$
Số kể trên là số nguyên, chứng tỏ $yz$ chia hết cho $2x.$ Lập luận tương tự, ta chỉ ra
$$2x\mid yz,\quad 2y\mid zx,\quad 2z\mid xy.$$
Mặt khác, giả thiết $(a,b,c)=1$ cho ta $(x,y,z)\in \{1;2\}.$ Ta xét các trường hợp kể trên.
\begin{enumerate}
    \item Nếu $({x}, {y}, {z})=1,$ ta đặt $x=mnX,b=npY,c=pmZ,$ trong đó
\begin{align*}
    (m,n)&=(n,p)=(p,m)=\left(X,Y\right)=(Y,Z)\\&=(Z,X)=(m,Y)=(n,Z)=(p,X)=1.
\end{align*}
Từ việc $yz$ chia hết cho $2x,$ ta lần lượt suy ra
$$2mnX\mid (npY)(pmZ)\Rightarrow 2X\mid p^2YZ.$$
Nhờ điều kiện $(X,Y)=(X,Z)=(X,p)=1,$ nhận xét trên cho ta $1$ chia hết cho $2X.$ \\
Điều này không thể xảy ra.
    \item Nếu $({x}, {y}, {z})=2,$ ta đặt $x=2mnX,b=2npY,c=2pmZ,$ trong đó
\begin{align*}
    (m,n)&=(n,p)=(p,m)=\left(X,Y\right)=(Y,Z)\\&=(Z,X)=(m,Y)=(n,Z)=(p,X)=1.
\end{align*}
Lập luận tương tự trường hợp trên, ta chỉ ra $X,Y,Z$ đều là ước của $2.$ Không mất tổng quát, giả sử $$X\ge Y\ge Z.$$
Tới đây, ta xét các trường hợp sau.
\begin{itemize}
    \item \chu{Trường hợp 1.} Với $X=Y=Z=2,$ ta có $(X,Y,Z)=2>1,$ mâu thuẫn.
    \item \chu{Trường hợp 2.} Với $X=Y=2$ và $Z=1,$ ta có $$xyz=4mn \cdot 4np \cdot 2pm=32(mnp)^2$$ là hai lần một số chính phương.
    \item \chu{Trường hợp 3.} Với $X=2$ và $Y=Z=1,$ ta có $$xyz=4mn \cdot 2np \cdot 2pm=16(mnp)^2$$ là một số chính phương.
    \item \chu{Trường hợp 4.} Với $X=Y=Z=1,$ ta có $$xyz=2mn \cdot 2np \cdot 2pm=8(mnp)^2$$ là hai lần một số chính phương.    
\end{itemize}
\end{enumerate}
Bài toán được chứng minh.}
\end{gbtt}

\begin{gbtt}
Tìm các số nguyên dương $a,b,c$ sao cho $a^3+b^3+c^3$ chia hết cho $a^2b,b^2c$ và $c^2a.$
\nguon{Bulgaria Mathematical Olympiad 2001}
\loigiai{
Do cả $a,b,c$ đều dương, ta có thể gọi $d$ là ước chung lớn nhất của $a,b,c.$\\ Phép gọi này cho ta biết, tồn tại các số nguyên dương $A,B,C$ sao cho
$$(A,B,C)=1,a=dA,b=dB,c=dC.$$
Ta viết lại giả thiết thành
$$d^3A^3+d^3B^3+d^3C^3\text{ chia hết cho }d^3A^2B,d^3B^2C,d^3C^2A,$$
hay là
$A^3+B^3+C^3\text{ chia hết cho }A^2B,B^2C,C^2A.$ \\Do $(A,B,C)=1,$ nên ta có thể đặt $A=mnx,B=npy,C=pmz,$ với các số $m,n,p,x,y,z$ thỏa mãn
$$(m,n)=(n,p)=(p,m)=(x,y)=(y,z)=(z,x)=(m,y)=(n,z)=(p,x)=1.$$
Đầu tiên, ta viết lại điều kiện $A^2B\mid \left(A^3+B^3+C^3\right)$ về thành
\[(mnx)^2npy\mid \left(m^3n^3x^3+n^3p^3y^3+p^3m^3z^3\right).\tag{1}\label{bai3bien1}\]
Do cả $(mnx)^2npy,m^3n^3x^3,n^3p^3y^3$ đều chia hết cho $n^3$ nên $p^3m^3z^3$ cũng chia hết cho $n^3.$ Tuy nhiên, vì $$(m,n)=(p,n)=(z,n)=1,$$ ta bắt buộc phải có $n=1.$ Lập luận tương tự, ta chỉ ra $m=n=p=1.$ Chính vì thế, (\ref{bai3bien1}) cho ta
\[x^2y\mid \left(x^3+y^3+z^3\right).\tag{2}\label{bai3bien2}\]
Thiết lập các đánh giá tương tự, ta được
\[y^2z\mid \left(x^3+y^3+z^3\right),\tag{3}\label{bai3bien3}\]
\[z^2x\mid \left(x^3+y^3+z^3\right).\tag{4}\label{bai3bien4}\]
Bội chung nhỏ nhất của $x^2y,y^2z$ và $z^2x$ bằng $x^2y^2z^2,$ thế nên (\ref{bai3bien2}),(\ref{bai3bien3}) và (\ref{bai3bien4}) cho ta
$$x^2y^2z^2\mid \left(x^3+y^3+z^3\right).$$
Không mất tổng quát, ta giả sử $x=\max\{x;y;z\}.$ Giả sử này cho ta
$$x^2y^2z^2\le x^3+y^3+z^3\le 3x^3.$$
Ta suy ra $y^2z^2\le 3x$ từ đây. Từ (\ref{bai3bien2}), ta suy ra thêm $z^3+y^3$ chia hết cho $x^2,$ và như vậy
$$z^3+y^3\ge x^2\ge \dfrac{y^4z^4}{3}.$$
Với đánh giá $z^3+y^3\ge \dfrac{y^4z^4}{3}$ bên trên, chia cả hai vế cho $y^3z^3,$ ta được
$$yz\le \dfrac{3}{y^3}+\dfrac{3}{x^3}\le 3+3=6.$$
Lần lượt kiểm tra trực tiếp các trường hợp
$$xy=6,\quad xy=5,\quad xy=4,\quad xy=3,\quad xy=2,\quad xy=1,$$
ta chỉ ra có vô hạn bộ $(a,b,c)=(kx,ky,kz)$ như sau

    $$(k,k,k),\: (k,2k,3k),\: (k,3k,2k),$$
    $$(2k,k,3k),\: (2k,3k,k),\: (3k,k,2k),\: (3k,2k,k),$$

ở đây $k$ là số nguyên dương bất kì.}
\end{gbtt}

\begin{gbtt}
Cho ba số nguyên dương $a,b,c$ đôi một phân biệt thỏa mãn $(a,b,c) = 1$ và 
$$a \mid (b - c)^2, \quad b \mid (c-  a)^2, \quad c \mid (a - b)^2.$$ Chứng minh rằng $a, b$  $c$ không là độ dài ba cạnh của một tam giác.
\nguon{Baltic Way 2015}
\loigiai{
Ta đặt $a=mnx,b=npy,c=pmz,$ trong đó
$$(m,n)=(n,p)=(p,m)=(x,y)=(y,z)=(z,x)=(m,y)=(n,z)=(p,x)=1.$$
Phép đặt này cho ta 
$$mnx \mid (npy - pmz)^2, \quad npy \mid (pmz-  mnx)^2, \quad pmz \mid (mnx - npy)^2.$$
Dựa vào $mnx \mid (npy - pmz)^2,$ ta nhận thấy $(npy)^2$ chia hết cho $m.$ Kết hợp với điều kiện $$(m,y)=(m,p)=(m,n)=1,$$ ta chỉ ra $m=1.$ Tương tự, ta có $m=n=p=1,$ vậy nên hệ trên trở thành
$$x \mid (y - z)^2, \quad y \mid (z-  x)^2, \quad z \mid (x - y)^2.$$ 
Không khó để chỉ ra $$x^2+y^2+z^2-2xy-2yz-2zx=(y-z)^2+x(x-2y-2z)$$ chia hết cho $x.$ Bằng việc thiết lập các đẳng thức tương tự, ta thu được
\begin{align*}
    x \mid \left(x^2+y^2+z^2-2xy-2yz-2zx\right), \\
    y \mid \left(x^2+y^2+z^2-2xy-2yz-2zx\right), \\
    z \mid \left(x^2+y^2+z^2-2xy-2yz-2zx\right).
\end{align*}
Do $(x,y)=(y,z)=(z,x)=1,$ từ các nhận xét trên ta có
$$xyz\mid \left(x^2+y^2+z^2-2xy-2yz-2zx\right).$$
Ta giả sử phản chứng rằng $a,b,c$ là độ dài ba cạnh tam giác. Lúc này
\begin{align*}
    \heva{a>|b-c| \\ b>|c-a| \\ c>|a-b|}
    \Rightarrow
    \heva{x>|y-z| \\ y>|z-x| \\ z>|x-y|}
    \Rightarrow
    \heva{x^2&>(y-z)^2 \\ y^2&>(z-x)^2 \\ z&>(x-y)^2.}
\end{align*}
Lấy tổng theo vế rồi rút gọn, ta được
$$x^2+y^2+z^2<2xy+2yz+2zx.$$
Hướng đi tiếp theo của chúng ta là so sánh $2xy+2yz+2zx-x^2-y^2-z^2$ với $xyz.$\\ Ta xét các trường hợp dưới đây, với giả sử không mất tổng quát rằng $x\ge y\ge z.$
\begin{enumerate}
    \item Nếu $x\ge y\ge z\ge 3,$ ta có
$$0<2xy+2yz+2zx-x^2-y^2-z^2\le xy+yz+zx\le \dfrac{xyz}{3}+\dfrac{xyz}{3}+\dfrac{xyz}{3}=xyz.$$
Dấu bằng ở đánh giá trên phải xảy ra, tức là $x=y=z,$ mâu thuẫn với việc $x,y,z$ đôi một phân biệt.
    \item Nếu $z=2,$ ta có $2>|x-y|=x-y,$ và do $x\ne y$ nên $x=y+1,$ nhưng lúc này $(x-y)^2$ không chia hết cho $z,$ mâu thuẫn.
    \item Nếu $z=1,$ ta có $1>|x-y|,$ kéo theo $x\ne y,$ mâu thuẫn.
\end{enumerate}
Mâu thuẫn chỉ ra ở các trường hợp chứng tỏ giả sử phản chứng là sai. Bài toán được chứng minh.
}
\end{gbtt}

\begin{gbtt}
Chứng minh rằng không tồn tại ba số nguyên dương $a,b,c$ nguyên tố cùng nhau nào thỏa mãn đồng thời các điều kiện 
\[2(a,b)+[a,b]=a^2,\quad 2(b,c)+[b,c]=b^2,\quad 2(c,a)+[c,a]=c^2.\]

\loigiai{
Giả sử tồn tại các số nguyên dương $a,b,c$ thỏa mãn. Ta đặt $a=mnx,b=npy,c=pmz,$ trong đó
$$(m,n)=(n,p)=(p,m)=(x,y)=(y,z)=(z,x)=(m,y)=(n,z)=(p,x)=1.$$
Các phương trình đã cho được viết lại thành
$$\heva{
    2n+mnpxy&=m^2n^2x^2 \\
    2p+mnpyz&=n^2p^2y^2 \\
    2m+mnpzx&=p^2m^2z^2}
\Leftrightarrow 
\heva{
    2+mpxy&=m^2nx^2  \\
    2+nmyz&=n^2py^2  \\
    2+pnzx&=p^2mz^2.}$$
Lần lượt xét tính chia hết cho $mx,ny,pz$ ở cả $3$ đẳng thức, ta chỉ ra $2$ chia hết cho cả $mx,ny,pz.$ \\Ta xét các trường hợp sau.
\begin{enumerate}
    \item Nếu $mx=1,$ xuất phát từ $2+mpxy=m^2nx^2,$ ta có
    $$2+py=n.$$
    Trong trường hợp này, ta chỉ ra $n=2+py\ge 3,$ vô lí do $n$ là ước của $2.$
    \item Nếu $mx=2,$ xuất phát từ $2+mpxy=m^2nx^2,$ ta có   
    $$2+2py=4n\Leftrightarrow 1+py=2n.$$
    Rõ ràng, từ điều trên ta suy ra cả $p$ và $y$ là số lẻ. Kết hợp với dữ kiện $p,y$ đều là ước của $2,$ ta nhận được $p=y=1,$ và được thêm $n=1.$ Tuy nhiên, khi thế tất cả vào $2+nmyz=n^2py^2,$ ta nhận thấy không thỏa mãn.
\end{enumerate}
Như vậy, giả sử ban đầu là sai. Bài toán được chứng minh.}
\end{gbtt}

\begin{gbtt}
Tìm tất cả các bộ ba số tự nhiên $(m,n,p)$ thỏa mãn đồng thời các điều kiện
\[m + n = {\left( {m,n} \right)^2},\quad n + p= {\left( {n,p} \right)^2},\quad p+m = {\left( {p,m} \right)^2}.\]
\loigiai{
Nếu một trong ba số $m,n,p$ bằng $0,$ không mất tổng quát, ta giả sử $m=0.$ Trong trường hợp này, ta có
$$0+n=(0,n)^2\Rightarrow n=n^2\Rightarrow
\left[\begin{aligned}
  &n=0\\
  &n=1.
\end{aligned}\right.$$
Một cách tương tự, ta chỉ ra $p=0$ hoặc $p=1.$ Khi kiểm tra trực tiếp, ta thấy có các bộ $$(0,0,0),\quad (0,1,0),\quad (0,0,1)$$ thỏa mãn. Nếu cả $m,n,p$ đều dương, ta gọi $d$ là ước chung lớn nhất của $m,n,p.$ Phép gọi này cho ta biết, tồn tại các số nguyên dương $M,N,P$ sao cho
$$(M,N,P)=1,\ m=dM,\ n=dN,\ p=dP.$$
Lần lượt, ta thu được ba phương trình
$$dM+dN=(dM,dN)^2,\quad dN+dP=(dN,dP)^2,\quad dP+dM=(dP,dM)^2.$$
Hệ bên trên tương đương với
$$M+N=d(M,N)^2,\quad N+P=d(N,P)^2,\quad P+M=d(P,M)^2,$$
Ta có $d$ là ước của $M+N,N+P,P+M,$ thế nên
$$d\mid (M+N)+(N+P)-(P+M)=2N.$$
Một cách tương tự, ta chỉ ra $d$ là ước của $2M,2N,2P,$ lại do điều kiện $(M,N,P)=1$ nên $d\mid 2,$ tức $d=1$ hoặc $d=2.$ Ta xét hai trường hợp kể trên. 
\begin{enumerate}
    \item Nếu $d=1,$ do $(M,N,P)=(m,n,p)=1$ nên ta có thể đặt $$m=axy,\ n=byz,\ p=czx,$$ ở đây
$(a,b)=(b,c)=(c,a)=(x,y)=(y,z)=(z,x)=(a,z)=(b,x)=(c,y)=1.$\\
Phương trình đầu tiên trở thành
$axy+byz=y^2,$
hay là
\[ax+bz=y.\tag{1}\label{3bien1}\]
Một cách tương tự, ta thu được hệ gồm (\ref{3bien1}) và hai phương trình
\begin{align}
by+cx&=z,\tag{2}\label{3bien2}\\
cz+ay&=x.\tag{3}\label{3bien3}  
\end{align}
Không mất tổng quát, ta giả sử $x=\max \{x;y;z\}.$ Giả sử này kết hợp với (\ref{3bien3}) cho ta
$$x\ge cx+ax=x(c+a).$$
Đây là điều mâu thuẫn.
    \item Nếu $d=2,$ tương tự như trường hợp trước, ta có thể đặt $m=2axy,n=2byz,p=2czx,$ ở đây
$$(a,b)=(b,c)=(c,a)=(x,y)=(y,z)=(z,x)=(a,z)=(b,x)=(c,y)=1.$$
Đồng thời, hệ phương trình ta thu được gồm $3$ phương trình sau
\begin{align}
ax+bz&=2y,\tag{4}\label{3bien4}\\
by+cx&=2z,\tag{5}\label{3bien5}\\
cz+ay&=2x.\tag{6}\label{3bien6}
\end{align}
Không mất tổng quát, ta giả sử $x=\max \{x;y;z\}.$ Giả sử này kết hợp với (\ref{3bien6}) cho ta
$$2x=cz+ay\ge cx+ax=x(c+a).$$
Ta thu được $2\ge c+a.$ Do cả $c$ và $a$ là số nguyên dương nên $c=a=1.$ \\
Ngoài ra, kết quả $c=a=1$ này cho ta biết, dấu bằng ở đánh giá $$cz+ay\ge cz+ax$$ phải xảy ra, tức là $x=y=z.$ Tuy nhiên, điều kiện $$(x,y)=(y,z)=(z,x)=1$$ bắt buộc $x,y,z$ đều phải bằng $1.$ Thế $x=y=z=1,a=c=1$ vào (\ref{3bien6}), ta tìm ra $b=1.$\\ Đối chiếu lại với các phép đặt, ta có $m=n=p=2.$
\end{enumerate}
Kết luận, có $5$ bộ $(x,y,z)$ thỏa mãn đề bài, đó là $(0,0,0),(2,2,2)$ và các hoán vị của bộ $(0,0,1).$}
\end{gbtt}

\begin{gbtt}
Tìm tất cả các số nguyên dương $a,b,c$ thỏa mãn \[[a,b,c]=\dfrac{ab+bc+ca}{4}.\]
\nguon{Junior Japanese Mathematical Olympiad 2019}
\loigiai{
Giả sử tồn tại các số nguyên dương $a,b,c$ thỏa mãn.
\\Ta đặt $a=dmnx,b=dnpy,c=dpmz,$ ở đây $d=(a,b,c)$ và
$$(m,n)=(n,p)=(p,m)=(x,y)=(y,z)=(z,x)=(m,y)=(n,z)=(p,x)=1.$$
Đẳng thức đã cho được viết lại thành
$$4dmnpxyz=d^2n^2mpxy+d^2p^2nmyz+d^2m^2pnzx.$$
Chia cả hai vế cho $dmnp,$ ta được
$$4xyz=d\left(xyn+yzp+zxm\right).$$
Ta sẽ chứng minh $(xyn+yzm+zxp,xyz)=1.$ Thật vậy, 
\begin{itemize}
    \item[i,] $(xyn+yzp+zxm,x)=(yzp,x)=1$ do $(x,y)=(x,z)=(x,p)=1.$
    \item[ii,] $(xyn+yzp+zxm,y)=(zxm,y)=1$ do $(y,z)=(y,x)=(y,m)=1.$    
    \item[iii,] $(xyn+yzp+zxm,z)=(xyn,z)=1$ do $(z,x)=(z,y)=(z,n)=1.$
\end{itemize}
Các chứng minh vừa rồi dẫn đến $4$ chia hết cho $d\left(xyn+yzp+zxm\right).$ Với việc
$$xyn+yzp+zxm\ge 1\cdot1\cdot1+1\cdot1\cdot1+1\cdot1\cdot1=3,$$
ta chỉ ra $d=1$ và $xyn+yzp+zxm=4.$ Trong ba số $xyn,yzp,zxm$ lúc này, phải có hai số bằng $1$ và một số bằng $2.$ Các trường hợp trên cho ta ba bộ $(a,b,c)$ thỏa mãn đề bài là $(1,2,2),(2,1,2)$ và $(2,2,1).$}
\end{gbtt}

\section{Bài toán về các ước của một số nguyên dương}

\subsection*{Lí thuyết}

Trước hết, tác giả xin phép nhắc lại một vài lí thuyết quan trọng được sử dụng trong phần này.
\begin{enumerate}
    \item Số ước dương của một số nguyên dương $A$ gồm $n$ ước nguyên tố và có phân tích tiêu chuẩn $A=p_1^{k_1}p_2^{k_2}\ldots p_n^{k_n}$ là $\left(k_1+1\right)\left(k_2+1\right)\ldots\left(k_n+1\right).$
    \item Nếu số nguyên dương $A$ có các ước $1<d_1<d_2<\ldots<d_n$ thì 
    $$A=d_1d_n=d_2d_{n-1}=\ldots=d_kd_{n+1-k}.$$
    \item Ước dương nhỏ thứ nhất của một số lớn hơn $1$ luôn là $1$, trong khi ước dương tiếp theo là số nguyên tố.
\end{enumerate}
\subsection*{Ví dụ minh họa}
\begin{bx}
Tìm tất cả số nguyên $n$ có đúng $16$ ước nguyên dương $$1=d_1<d_2<d_3<\ldots<d_{16}=n$$ thỏa mãn $d_6=18$ và $d_9-d_8=7.$
\loigiai{
Từ giả thiết $d_6=18,$ ta chỉ ra
$$d_1=1,\: d_2=2,\: d_3=3,\: d_4=6,\: d_5=9.$$
Theo đó, $n$ nhận $2$ và $3$ là ước nguyên tố, ngoài ra số mũ của $3$ phải lớn hơn $1.$
Với điều kiện $k_1,k_2,\ldots,k_n$ đều là các số nguyên dương, ta đặt
	$n=2^{k_1}3^{k_2}p_3^{k_3}p_4^{k_4}\ldots p_m^{k_m}.$
Tổng số ước của $A$ là $16,$ vậy nên
	$$\left(k_1+1\right)\left(k_2+1\right)\ldots\left(k_m+1\right)=16.$$
Ta nhận thấy $k_2+1\ge 3,$ vậy nên ta có $k_1+1\ge 2,k_2+1\ge 4,$ kéo theo $m\le 3.$ \\
Tới đây, ta chia bài toán làm hai trường hợp.
\begin{enumerate}
    \item Nếu $m=2,k_1=3,k_2=3,$ ta tìm ra $n=216.$ Lúc này, $d_6=8$, mâu thuẫn.
    \item Nếu $m=3,k_1=1,k_2=3,k_3=1,$ để đơn giản, ta đặt $p_3=p,$ khi đó $n=2\cdot 3^3\cdot p,$ và rõ ràng $p>18.$
    \begin{itemize}
        \item \chu{Trường hợp 1. }Nếu $18<p<27,$ ta có 
        $$d_7=p,\quad d_8=27,\quad d_9=2p.$$
        Ta nhận được $2p-27=7,$ thế nên $p=17,$ mâu thuẫn với $18<p<27.$
        \item \chu{Trường hợp 2. }Nếu $p>27,$ ta có
        $$d_7=27,\quad d_8=p,\quad d_9=54.$$
        Ta nhận được $54-p=7,$ thế nên $p=47.$ 
    \end{itemize}
\end{enumerate}
Tổng kết lại và thử trực tiếp, ta có số $n=2\cdot 3^3\cdot 47=2538$ thỏa yêu cầu bài toán.}
\end{bx}

\subsection*{Bài tập tự luyện}

\begin{btt}
Tìm tất cả các số nguyên dương $n$ thỏa mãn $$n=d^2_1+d^2_2+d^2_3+d^2_4,$$ trong đó $d_1<d_2<d_3<d_4$ là $4$ ước số dương nhỏ nhất của $n.$
\nguon{Iran Mathematical Olympiad 1999}
\end{btt}

\begin{btt}
Tìm tất cả các số nguyên dương $n$ thỏa mãn $$n=d_2d_3+d_3d_5+d_5d_2,$$ trong đó $d_1<d_2<d_3<d_4<d_5$ là $5$ ước số dương nhỏ nhất của $n.$
\end{btt}

\begin{btt}
Tìm tất cả số nguyên $n$ có các ước nguyên dương $$1=d_1<d_2<d_3<\ldots<d_{k}=n$$ thỏa mãn $n$ chia hết cho $2019$ và $n=d_{19}d_{20}.$
\nguon{Belarusian Mathematical Olympiad 2019}
\end{btt}

\begin{btt}
Tìm tất cả số nguyên $n$ có các ước nguyên dương $$1=d_1<d_2<d_3<\ldots<d_{k}=n$$ thỏa mãn $n=d_{13}+d_{14}+d_{15}$ và $\left(d_5+1\right)^3=d_{15}+1.$ 
\end{btt}

\begin{btt}
Tìm tất cả số nguyên $n$ có các ước nguyên dương $$1=d_1<d_2<d_3<\ldots<d_{k}=n$$ thỏa mãn $d_5-d_3=50$ và $11d_5+8d_7=3n.$
\nguon{Czech and Slovak Mathematical Olympiad 2014}
\end{btt}

\begin{btt}
Tìm tất cả các số nguyên dương $n$ có $12$ ước dương $1=d_{1}<d_{2}<\cdots<d_{12}=n$ thỏa mãn \[d_{d_4-1}=\left(d_1+d_2+d_4\right)d_8.\]
\end{btt}

\begin{btt}
Tìm tất cả các số nguyên dương $n>1$ thỏa mãn nếu $n$ có $k$ ước nguyên dương $1=d_1<d_2<\ldots <d_k=n$ thì $d_1+d_2,d_1+d_2+d_3,\ldots ,d_1+d_2+\ldots+d_{k-1}$ cũng là các ước của $n$. 
\end{btt}

\begin{btt}
Cho số nguyên dương $n$ có tất cả $k$ ước số dương là $d_{1}< d_{2}< \ldots<d_{k}$. Giả sử
$$d_{1}+d_{2}+\ldots+d_{k}+k=2 n+1.$$ Chứng minh rằng $2m$ là số chính phương.
\end{btt}

\begin{btt}
Số nguyên dương $n$ được gọi là số \chu{điều hòa} nếu như tổng bình phương các ước dương của nó (kể cả $1$ và $n$) đúng bằng ${{\left( n+3 \right)}^{2}}$.
\begin{enumerate}[a,]
    \item Chứng minh rằng số $287$ là một số \chu{điều hòa}.
    \item Chứng minh rằng số $n=p^3$ (với $p$ là một số nguyên tố) không thể là số \chu{điều hòa}.
    \item Chứng minh rằng nếu số $n=pq$ (với $p$ và $q$ là các số nguyên tố khác nhau) là số \chu{điều hòa} thì $n+2$ là một số chính phương.
\end{enumerate}
\nguon{Chuyên Toán Phổ thông Năng khiếu 2013}
\end{btt}

\begin{btt}
Tìm tất cả các số tự nhiên $N$ biết rằng tổng tất cả các ước số của $N$ bằng $2N$ và tích tất cả các ước số của $N$ bằng $N^2$.
\nguon{Tạp chí Toán học và Tuổi trẻ số 512, tháng 2 năm 2020}
\end{btt}

\subsection*{Hướng dẫn bài tập tự luyện}

\begin{gbtt}
Tìm tất cả các số nguyên dương $n$ thỏa mãn $$n=d^2_1+d^2_2+d^2_3+d^2_4,$$ trong đó $d_1<d_2<d_3<d_4$ là $4$ ước số dương nhỏ nhất của $n.$
\nguon{Iran Mathematical Olympiad 1999}
\loigiai{Hiển nhiên, ta có $d_1=1.$ Với số nguyên dương $n$ thỏa yêu cầu, ta chứng minh $d_2=2.$ Trong trường hợp $n$ là một số lẻ thì $d_{1}, d_{2}, d_{3}, d_{4}$ cũng là 4 số lẻ, do đó
$$d_{1}^{2} \equiv d_{2}^{2} \equiv d_{3}^{2} \equiv d_{4}^{2} \equiv 1 \pmod{4}.$$
Kết hợp với giả thiết $n=d^2_1+d^2_2+d^2_3+d^2_4,$ ta chỉ ra
 $$n=d_{1}^{2}+d_{2}^{2}+d_{3}^{2}+d_{4}^{2} \equiv 0 \pmod{4}.$$  Điều này trái với điều kiện $n$ là số lẻ. Mâu thuẫn thu được chứng tỏ $d_2=2.$  \\
Với $n$ chia hết cho $4,$ do $d_{1}=1$ và $d_{2}=2$ nên
$$n \equiv 1+0+d_{3}^{2}+d_{4}^{2} \equiv 0 \pmod{4}.$$
Ta suy ra $d_3^2+d_4^2\equiv 3\pmod{4}$ từ đây, nhưng điều này không thể xảy ra vì
$$d_3^2+d_4^2\equiv 0,1,2\pmod{4}.$$
Do đó $n$ không thể chia hết cho $4,$ và kéo theo $n \equiv 2 \pmod{4}$. Ta xét các trường hợp sau.
\begin{enumerate}
    \item Nếu $d_3= p$ và $d_4=q$ với $p, q$ là các số nguyên tố lẻ, ta có 
    $$n=1+p^2+q^2\equiv 1+1+1\equiv 3 \pmod{4},$$ 
    mâu thuẫn với $n \equiv 2 \pmod{4}.$ Trường hợp này không xảy ra.
    \item Nếu $d_2=p,d_3=2p$ với $p$ là số nguyên tố, ta có
    $$n=d_{1}^{2}+d_{2}^{2}+d_{3}^{2}+d_{4}^{2} \equiv 5\left(1+p^{2}\right).$$
    Đánh giá trên chứng tỏ $p=5,$ và vậy thì $n=130.$
\end{enumerate}
Kiểm tra trực tiếp số $n=130,$ ta thấy đây là giá trị $n$ duy nhất thỏa mãn đề bài. Bài toán được giải quyết.}
\end{gbtt}

\begin{gbtt}
Tìm tất cả các số nguyên dương $n$ thỏa mãn $$n=d_2d_3+d_3d_5+d_5d_2,$$ trong đó $d_1<d_2<d_3<d_4<d_5$ là $5$ ước số dương nhỏ nhất của $n.$
\loigiai{Đầu tiên, ta có $d_2=p,$ với $p$ là số nguyên tố, đồng thời $d_3=p^2$ hoặc $d_3=q,$ với $q$ là số nguyên tố.
\begin{enumerate}
    \item Trong trường hợp $d_3=p^2,$ ta sẽ có
    $$n=d_2d_3+d_3d_5+d_5d_2=p^3+d_5\left(p^2+p\right).$$
    Với việc $n$ và $p^3$ cùng chia hết cho $p^2,$ ta nhận thấy $d_5\left(p^2+p\right)$ chia hết cho $p^2,$ hay là $p\mid d_5.$ \\
    Tuy nhiên, nếu $d_5$ có ước nguyên tố $r>p,$ ta nhận thấy 
    $$r\mid n,\quad r\nmid p^3,\quad r\mid d_5\left(p^2+p\right).$$
    Các nhận xét này mâu thuẫn với việc $n=p^3+d_5\left(p^2+p\right),$ thế nên $d_5$ chỉ có duy nhất một ước nguyên tố là $p.$ Nhờ vào giả thiết $d_5$ là ước thứ $5$ của $n,$ ta suy ra $d_5=p^3$ hoặc $d_5=p^4.$ 
    \begin{itemize}
        \item\chu{Trường hợp 1.} Nếu $d_5=p^3,$ ta có $n=p^3\left(p^2+p+1\right)$ và $d_4=p^2+p+1$ nguyên tố.\\ Kiểm tra trực tiếp, tất cả các số có dạng trên đều thỏa mãn. 
        \item\chu{Trường hợp 2.} Nếu $d_5=p^4,$ ta có $n=p^3+p^5+p^6$ và $d_4=p^3$ nguyên tố. \\Tuy nhiên, $n$ không chia hết cho $p^4$ trong khả năng này, một điều mâu thuẫn. 
    \end{itemize}
    \item Trong trường hợp $d_3=q,$ ta sẽ có
    $$n=d_2d_3+d_3d_5+d_5d_2=pq+d_5p+d_5q.$$
    Do cả $n,pq$ và $d_5p$ chia hết cho $p$ nên $d_5q$ chia hết cho $p,$ và vì $(p,q)=1$ nên $p\mid d_5.$ \\Một cách tương tự, ta chỉ ra $pq\mid d_5,$ và bắt buộc $d_5=p^2q.$ Ta suy ra
    $$n=pq+p^2q\cdot p+p^2q\cdot q=pq(p^2+pq+1).$$
    Do $n$ chia hết cho $p^2q,$ ta cần phải có $p^2+pq+1$ chia hết cho $p.$ Điều này là không thể.
\end{enumerate}
Tóm lại, các số $n$ cần tìm có dạng $n=p^3\left(p^2+p+1\right),$ trong đó $p$ và $p^2+p+1$ là số nguyên tố.}
\end{gbtt}

\begin{gbtt}
Tìm tất cả số nguyên $n$ có các ước nguyên dương $$1=d_1<d_2<d_3<\ldots<d_{k}=n$$ thỏa mãn $n$ chia hết cho $2019$ và $n=d_{19}d_{20}.$
\nguon{Belarusian Mathematical Olympiad 2019}
\loigiai{ 
Vì $n=d_{19}d_{20}$ và $1=d_1<d_2<d_3<\ldots<d_{k}=n$ nên $n$ có $19\cdot2=38$ ước nguyên dương. Đặt $$n=p_1^{\alpha_1}p_2^{\alpha_2}\cdots p_m^{\alpha_m},$$ trong đó $p_i$ là các số nguyên tố phân biệt, $\alpha_i$ là số tự nhiên và $i=\overline{1,2,\cdots,m}.$ Phép đặt này cho ta
$$\tron{\alpha_1+1}\tron{\alpha_2+1}\cdots\tron{\alpha_m+1}=38=2\cdot19.$$
Nhờ nhận xét trên, ta thu được $m\le 2,$ và dẫn đến $n=p_1^{\alpha_1}p_2^{\alpha_2}$ hoặc $n=p_1^{\alpha_1}.$ Với giả thiết $n$ chia hết cho $2019,$ ta suy ra số ước nguyên tố của $n$ phải lớn hơn $1,$ và chỉ tồn tại trường hợp $$n=p_1^{\alpha_1}p_2^{\alpha_2}.$$ 
Hơn nữa, từ
$\tron{\alpha_1+1}\tron{\alpha_2+1}=2\cdot19,$ ta chỉ ra $\tron{\alpha_1,\alpha_2}=\tron{1,18}$ hoặc $\tron{\alpha_1,\alpha_2}=\tron{18,1}.$\\
Như vậy, tất cả các số nguyên $n$ thỏa mãn là $3\cdot673^{18}$ và $673\cdot3^{18}$}
\end{gbtt}

\begin{gbtt}
Tìm tất cả số nguyên $n$ có các ước nguyên dương $$1=d_1<d_2<d_3<\ldots<d_{k}=n$$ thỏa mãn $n=d_{13}+d_{14}+d_{15}$ và $\left(d_5+1\right)^3=d_{15}+1.$ 
\loigiai{
Từ dữ kiện $n=d_{13}+d_{14}+d_{15}$, ta suy ra $d_{13}+d_{14}$ chia hết cho $d_{15}.$ Tuy nhiên, do
$$d_{13}+d_{14}< 2d_{15}$$
nên bắt buộc $d_{13}+d_{14}=d_{15}.$ Ta cũng đã biết $d_{13}d_4=d_{14}d_3=n,$ thế nên là
$$n=d_{13}+d_{14}+d_{15}=2d_{13}+2d_{14}=\dfrac{2n}{d_4}+\dfrac{2n}{d_3}.$$
Ta suy ra
$d_4d_3=2d_4+2d_3.$
Chuyển về dạng phương trình ước số tương đương, ta được
$$\left(d_4-2\right)\left(d_3-2\right)=4.$$
Do $d_4-2>d_3-2>0$ nên bắt buộc $d_4=6,d_3=3.$\\ Nhận xét này cho ta biết $n$ nhận $2$ và $3$ là ước nguyên tố.  Ta xét các trường hợp sau đây
\begin{enumerate}
    \item Nếu $d_5=p$ là số nguyên tố (và chắc chắn lớn hơn $6$), ta có
    $$(p+1)^3=d_{15}+1\Rightarrow p^3=\dfrac{n}{2}-3p^2-3p.$$
    Xét tính chia hết cho $3$ ở cả hai vế, ta chỉ ra $p^3$ chia hết cho $3,$ tức $p=3,$ mâu thuẫn với $p>6.$
    \item Nếu $d_5$ không là số nguyên tố, các ước nguyên tố của $d_5$ chỉ có thể là $2$ và $3,$ đồng thời số mũ của chúng không vượt quá $2.$ Ta xét tới các khả năng nhỏ sau đây.
    \begin{itemize}
        \item \chu{Trường hợp 1.} Nếu $d_5=2^2,$ ta có $d_5<4\le d_4,$ mâu thuẫn.
        \item \chu{Trường hợp 2.} Nếu $d_5=3^2,$ ta tìm được $d_{15}=999,$ và khi ấy $n=1998.$
        \item \chu{Trường hợp 3.} Nếu $d_5=2^a\cdot3^b,$ với ít nhất một trong hai số $a,b$ lớn hơn $1,$ ta chỉ ra tồn tại ít nhất $5$ ước của $n$ nhỏ hơn $d_5,$ mâu thuẫn. 
    \end{itemize}
\end{enumerate}
Thử trực tiếp $n=1998,$ ta thấy thỏa yêu cầu, và đây là số nguyên dương ta cần tìm.}
\end{gbtt}


\begin{gbtt}
Tìm tất cả số nguyên $n$ có các ước nguyên dương $$1=d_1<d_2<d_3<\ldots<d_{k}=n$$ thỏa mãn $d_5-d_3=50$ và $11d_5+8d_7=3n.$
\nguon{Czech and Slovak Mathematical Olympiad 2014}
\loigiai{Giả sử tồn tại số nguyên $n$ thỏa mãn.
Vì $d_5,d_7$ đều là ước của $n$ nên từ giả thiết $3n=11d_5+8d_7,$ ta có
$$d_5\mid 8d_7,\qquad d_7\mid11d_5.$$
Đặt $8d_7=ad_5$ và $11d_5=bd_7,$ điều này dẫn đến $88d_7=abd_7$ kéo theo $b\mid 88.$ Vì $11d_5=bd_7$ và $d_5<d_7$ nên $11>b.$ Từ những nhận xét trên, ta suy ra $b\in\left\{1,2,4,8\right\}.$ Ta xét các trường hợp sau.
\begin{enumerate}
    \item Với $b=8,$ ta có $11d_5=8d_7$ dẫn đến $8\mid d_5$ hay $8\mid n.$ Vì $d_5-d_3=50$ và $d_5$ là số chẵn nên $d_3$ chẵn. Do $n$ chia hết cho $8$ nên $d_3=4.$ Từ đây, ta suy ra $d_5=54,$ và $n$ chia hết cho $3,$ nhưng lúc này $d_3=3,$ mâu thuẫn.
    \item Với $b=4,$ ta có $11d_5=4d_7$ dẫn đến $4\mid d_5$ hay $4\mid n.$ Chứng minh tương tự trường hợp trên, ta nhận thấy không có $n$ thỏa mãn.
    \item Với $b=2,$ tương tự trường hợp trên, ta có $2\mid d_5,n,d_3.$ Kết hợp với $11\mid d_7,$ ta suy ra $11\mid n.$ \\Do đó, $d_3$ là số chẵn thỏa mãn $2<d_3\le 11.$ Ta xét các trường hợp sau.
    \begin{itemize}
        \item\chu{Trường hợp 1.} Với $d_3=4,$ ta có $d_5=50+4=54,$ nhưng lúc này $d_3=3,$ mâu thuẫn.
        \item\chu{Trường hợp 2.} Với $d_3=6,$ ta có $3\mid n$ kéo theo $d_3=3,$ mâu thuẫn.
        \item\chu{Trường hợp 3.} Với $d_3=8,$  ta có $4\mid n$ kéo theo $d_3=4,$ mâu thuẫn.
        \item\chu{Trường hợp 4.} Với $d_3=10,$ ta có $5\mid n$ kéo theo $d_3=5,$ mâu thuẫn.
    \end{itemize}
    \item Với $b=1,$ ta có $11d_5=d_7,$ và khi ấy
    $$11d_5+88d_7=3n,$$
    hay $33d_5=n.$ Từ đây, ta suy ra $n=33d_3+1650,$ và $d_3$ là ước của $1650.$ Bằng phản chứng, ta dễ dàng chứng minh được $d_3$ chỉ có một ước nguyên tố. Do $d_3$ là ước của $1650$ nên
    $d_3\in\left\{3,5,11\right\}.$
    \begin{itemize}
        \item \chu{Trường hợp 1.} Nếu $d_3=3,$ ta có $n=33\cdot3+1650=1749.$ Thử trực tiếp, ta thấy $$d_5-d_3=33-3=30,$$ mâu thuẫn với giả thiết $d_5-d_3=50.$
        \item \chu{Trường hợp 2.} Nếu $d_3=5,$ ta có $n=33\cdot5+1650=1815.$ Thử trực tiếp, ta thấy $$d_5-d_3=15-5=10,$$ mâu thuẫn với giả thiết $d_5-d_3=50.$      
        \item \chu{Trường hợp 3.} Nếu $d_3=11,$ ta có $n=33\cdot11+1650=2013.$ Thử trực tiếp, ta thấy thỏa. 
    \end{itemize}
\end{enumerate}
Như vậy, có duy nhất một số nguyên $n$ thỏa mãn là $2013.$}
\end{gbtt}

\begin{gbtt}
Tìm tất cả số nguyên dương $n$ có $12$ ước dương $1=d_{1}<d_{2}<\cdots<d_{12}=n$ thỏa mãn \[d_{d_4-1}=\left(d_1+d_2+d_4\right)d_8.\]
\loigiai{
Do $d_8$ là một ước của $n$ và $d_8$ chia hết cho $d_{d_4-1}$ nên tồn tại $1 \leqslant i \leqslant 12$ sao cho 
$$d_i = d_1 + d_2 + d_4.$$ 
Vì $d_i>d_4$ nên ta có $i \geqslant 5$. Ngoài ra, do $d_id_8=d_{d_4- 1}\leqslant n$ nên $i\le 5.$ Như vậy $i = 5$ và 
$$d_1+d_2+d_4=d_5.$$ 
Từ đây và $d_{d_4 - 1}=d_5d_8=n=d_{12},$ ta suy ra $d_4 = 13$ còn ${d_5} = 14 + {d_2}$. Tất nhiên ${d_2}$ là ước số nguyên tố nhỏ nhất của $n$ và vì ${d_4} = 13$, chúng ta chỉ có thể có ${d_2} \in \left\{ {2;3;5;7;11} \right\}$. Lại vì $n$ có 12 ước số, nó có nhiều nhất $3$ thừa số nguyên tố.
\begin{enumerate}
    \item Nếu $d_2=2$ thì $d_5=16,$ kéo theo $d_3=4,d_4=8,$ nhưng khi ấy $d_5\ne d_1+d_2+d_4,$ mâu thuẫn.
    \item Nếu $d_2=3$ thì $d_5=17.$ Do đây là hai số nguyên tố nên trong $d_3,d_4$ chỉ có một số nguyên tố lớn hơn $3$ và nhỏ hơn $17.$ Với việc $n$ có $12$ ước nguyên dương và $3$ ước nguyên tố, phải có một ước nguyên tố của $n$ mang mũ $2.$ Từ các lập luận kể trên, ta chỉ ra
    $$\tron{d_3,d_4}\in\left\{\tron{5,9};\tron{7,9};\tron{9,11};\tron{9,13}\right\}.$$
    Thử trực tiếp, ta tìm ra $n=9\cdot13\cdot17=1989$ khi $d_3=9,d_4=13.$
    \item Nếu $d_2\in\{5;7;11\}$ do $d_2<d_3<d_4<d_5<d^2_2$ nên $d_3,d_4$ là số nguyên tố, nhưng khi ấy $n$ có ít nhất $4$ ước nguyên tố, mâu thuẫn.
\end{enumerate}
Như vậy, $n=1989$ là số nguyên dương duy nhất thỏa yêu cầu.}
\end{gbtt}

\begin{gbtt}
Tìm tất cả các số nguyên dương $n>1$ thỏa mãn nếu $n$ có $k$ ước nguyên dương $$1=d_1<d_2<\ldots <d_k=n$$ thì $d_1+d_2,d_1+d_2+d_3,\ldots ,d_1+d_2+\ldots+d_{k-1}$ cũng là các ước của $n.$ 
\loigiai{
Với việc $d_1,d_2,\ldots,d_k$ là các số nguyên dương, ta nhận xét
$$d_2<d_1+d_2<d_1+d_2+d_3<\ldots <d_1+d_2+\ldots+d_{k-1}\le n.$$
Do $d_1+d_2,d_1+d_2+d_3,\ldots ,d_1+d_2+\ldots+d_{k-1}$ cũng là các ước của $n$ nên ta chỉ ra được rằng 
\begin{align*}
    d_3&=d_1+d_2,\\
    d_4&=d_1+d_2+d_3,\\
    &\ldots\\
    d_k&=d_1+d_2+d_3\ldots+d_{k-1}.
\end{align*}
Rõ ràng $k\ge 3.$ Các nhận xét trên kéo theo
\begin{align*}
    d_3&=d_1+d_2,\\
    d_4&=d_1+d_2+d_3=d_3+d_3=2d_3,\\
    d_5&=d_1+d_2+d_3+d_4=d_4+d_4=2d_4=4d_3,\\   
    d_6&=d_1+d_2+d_3+d_4+d_5=d_5+d_5=2d_5=8d_3,\\       
    &\ldots\\
    d_k&=d_1+d_2+d_3\ldots+d_{k-1}=d_{k-1}+d_{k-1}=2d_{k-1}=2^{k-3}d_3.
\end{align*}
Do $d_4=2d_3$ và $n$ chia hết cho $d_4$ nên $d_2=2.$ Như thế thì $d_3=3$ và $n=d_k=3\cdot2^{k-3}.$\\ Đây cũng là dạng tổng quát của các số nguyên $n$ thỏa yêu cầu.}
\end{gbtt}
 
\begin{gbtt}
Cho số nguyên dương $n$ có tất cả $k$ ước số dương là $d_{1}< d_{2}< \ldots<d_{k}$. Giả sử
$$d_{1}+d_{2}+\ldots+d_{k}+k=2 n+1.$$ Chứng minh rằng $2m$ là số chính phương.
\loigiai{Gọi $l_{1}, l_{2}, \ldots, l_{s}$ là các ước lẻ của $n$ và $2^m$ là lũy thừa lớn nhất của $2$ trong khai triển $n.$ Các ước của $n$ là 
$$l_{1}, l_{2}, \ldots, l_{s}, 2 l_{1}, 2 l_{2}, \ldots, 2 l_{s}, \ldots, 2^{m} l_{1}, 2^{m} l_{2}, \ldots, 2^{m} l_{s}.$$
Đẳng thức ở đề bài trở thành
$$l_{1}+l_{2}+\ldots+l_{s}+2 l_{1}+2 l_{2}+\ldots+2 l_{s}+\ldots+2^{m} l_{1}+2^{m} l_{2}+\ldots+2^{m} l_{s}+(m+1) s=2 n+1.$$
Đưa các nhân tử $l_{1}+l_{2}+\ldots+l_{s}$ về một nhóm, đẳng thức trên tương đương
\[\left(l_{1}+l_{2}+\ldots+l_{s}\right)\left(2^{m+1}-1\right)+(m+1) s=2 n+1 .\label{n2lascp}\tag{*}\]
Tới đây, ta xét các trường hợp sau đây.
\begin{enumerate}
    \item Nếu $s$ chẵn thì vế trái của (\ref{n2lascp}) chẵn, còn vế phải của (\ref{n2lascp}) lẻ, mâu thuẫn.
    \item Nếu $s$ lẻ và $m$ chẵn thì vể trái của vế trái của (\ref{n2lascp}) cũng chẵn, còn vế phải của (\ref{n2lascp}) lẻ, mâu thuẫn.
    \item Nếu $s$ lẻ và $m$ chẵn, $\dfrac{n}{2^m}$ có số ước lẻ. Số $\dfrac{n}{2^{m}}=p_{1}^{k_{1}} p_{2}^{k_{2}} \cdots p_{m}^{k_{m}}$ có số ước là 
    $$\left(k_{1}+1\right)\left(k_{2}+1\right) \ldots\left(k_{m}+1\right).$$
    Theo đó, các $k_i$ là số chẵn, và $2^{m+1}\cdot\dfrac{n}{2^m}=2n$ là số chính phương.
\end{enumerate}
Như vậy, bài toán đã cho được chứng minh.}
\end{gbtt}

\begin{gbtt}
Số nguyên dương $n$ được gọi là số \chu{điều hòa} nếu như tổng bình phương các ước dương của nó (kể cả $1$ và $n$) đúng bằng ${{\left( n+3 \right)}^{2}}$.
\begin{enumerate}[a,]
    \item Chứng minh rằng số $287$ là một số \chu{điều hòa}.
    \item Chứng minh rằng số $n=p^3$ (với $p$ là một số nguyên tố) không thể là số \chu{điều hòa}.
    \item Chứng minh rằng nếu số $n=pq$ (với $p$ và $q$ là các số nguyên tố khác nhau) là số \chu{điều hòa} thì $n+2$ là một số chính phương.
\end{enumerate}
\nguon{Chuyên Toán Phổ thông Năng khiếu 2013}
\loigiai{
\begin{enumerate}[a,]
    \item  Dễ thấy $287=1\cdot 7\cdot 41.$ Ta có 
    $${{\left( 287+3 \right)}^{2}}={{290}^{2}}= 84100={{1}^{2}}+{{7}^{2}}+{{41}^{2}}+{{287}^{2}}.$$ 
    Từ đây, ta suy ra $287$ là số điều hòa.
    \item Giả sử $n=p^3$ là số điều hòa. Các ước dương của $n$ lúc này là $1,p,p^2,p^3.$ Giả thiết cho ta
    $${{\left( {{p}^{3}}+3 \right)}^{2}}={{1}^{2}}+{{p}^{2}}+{{p}^{4}}+{{p}^{6}}.$$ Biến đổi tương đương, ta được
    $${{p}^{6}}+6{{p}^{3}}+9={{1}^{2}}+{{p}^{2}}+{{p}^{4}}+{{p}^{6}}\Leftrightarrow p\left( {{p}^{3}}-6{{p}^{2}}+p \right)=8.$$
    Do đó ta được $p\mid 8$ mà $p$ là số nguyên tố nên $p=2$. Khi đó $p\left( {{p}^{3}}-6{{p}^{2}}+p \right)=28\ne 8.$\\ 
    Vì vậy, giả sử sai hay  $n={{p}^{3}}$ không thể là số điều hòa.
    \item Ta có $n=pq$ là số điều hòa với $p$ và $q$ là các nguyên tố khác nhau. Ta nhận được
    $$1+p^2+q^2+p^2q^2=(pq+3)^2\Leftrightarrow p^2-6pq+q^2=8\Leftrightarrow (p-q)^2=4pq+8.$$
    Ta có $4\mid{{\left( p-q \right)}^{2}}$ nên $2\mid (p-q)$. Như vậy
    $$n+2=pq+2=\dfrac{(p-q)^2}{4}=\tron{\dfrac{p-q}{2}}^2$$
    là một số chính phương. Bài toán được chứng minh.
\end{enumerate}}
\end{gbtt}

\begin{gbtt}
Tìm tất cả các số tự nhiên $N$ biết rằng tổng tất cả các ước số của $N$ bằng $2N$ và tích tất cả các ước số của $N$ bằng $N^2$.
\nguon{Tạp chí Toán học và Tuổi trẻ số 512, tháng 2 năm 2020}
\loigiai{Ta không xét số $N=0$ vì số $0$ có vô hạn ước số. Số $N$ khác $1$ vì tổng tất cả các ước số của $N=1$ bằng $1$ khác $2N=2$. Ta xét các trường hợp sau đối với $N>1.$
\begin{enumerate}
\item  Nếu $N$ có nhiều hơn bốn ước số, gọi $a$ là ước số nhỏ nhất của $N~(a\neq 1)$ và $c$ là ước số lớn nhất
		 của $N~(c\neq N)$ thì tồn tại ít nhất một ước số $b$ thỏa mãn 
		 $$1<a<b<c<N.$$ 
		 Ngoài ra ta còn có $N=ac.$ Lúc đó tích tất cả các ước số của $N$ bằng $$a^2c^2\stackrel{\text{phép đặt}}{=}N^2\stackrel{\text{giả thiết}}{=}1 \cdot a \cdot b\cdots c \cdot N\geq abcac=a^2c^2b,$$ suy ra $b=1$ trái giả thiết $b>a>1$.
        \item Nếu $N$ có đúng hai ước số, tích tất cả các ước số của $N$ bằng $N^2,$ suy ra $N=1,$ trái giả thiết $N>1$.
		\item Nếu $N$ có đúng ba ước số, xét phân tích tiêu chuẩn của $N$
		$$N=p_1^{k_1}p_2^{k_2}\ldots p_n^{k_n}.$$
		Do số ước nguyên dương của $N$ bằng $3$ nên 
		$$\tron{k_1+1}\tron{k_2+1}\cdots\tron{k_n+1}=3.$$
		Trong các thừa số ở vế trái, buộc phải có $n-1$ thừa số bằng $1$ và thừa số còn lại bằng $3.$ Như vậy $N=a^2,$ với $a$ nguyên dương. Tích tất cả các ước số của $N$ bằng $N^2$ nên
		$$a^3=1\cdot a\cdot N=N^2=a^4,$$
		suy ra $a=1$, trái giả thiết $a>1$.
		\item Nếu $N$ có đúng bốn ước số, ta lập luận tương tự trường hợp trước rồi xét các trường hợp nhỏ hơn.
		\begin{itemize}
		    \item \chu{Trường hợp 1.} Nếu $N=pq$ với $p,q$ là các số nguyên tố, tổng các ước của $N$ bằng $2N$ nên
		    $$2pq=pq+p+q+1.$$
		    Chuyển vế, ta được $(p-1)(q-1)=2.$ Ta tìm ra $(p,q)=(2,3),(3,2),$ kéo theo $N=pq=6.$
		    \item \chu{Trường hợp 2.} Nếu $N=k^3$ với $k$ là số nguyên tố, tổng các ước của $N$ bằng $2N$ nên
		    $$2k^3=1+k+k^2+k^3.$$
		    Ta không tìm được $k$ nguyên từ đây.
		\end{itemize}
\end{enumerate}
Kết luận, chỉ có một giá trị của $N$ thoả mãn là $N=6.$}
\end{gbtt}

\section{Sự tồn tại trong các bài toán chia hết, ước, bội}

Mục này chủ yếu đưa ra một vài bài toán về sự tồn tại trong số học và có sử dụng phép chia hết. 

\subsection*{Bài tập tự luyện}

\begin{btt} \
\begin{enumerate}[a,]
    \item Tìm tất cả các số tự nhiên có thể viết thành tổng của hai số nguyên lớn hơn $1$ và nguyên tố cùng nhau.
    \item Tìm tất cả các số tự nhiên có thể viết thành tổng của ba số nguyên lớn hơn $1$ và đôi một nguyên tố cùng nhau.
\end{enumerate}
\end{btt}

\begin{btt}
Chứng minh rằng mọi số nguyên dương $n$ đều có thể biểu diễn dưới dạng
\[n=\dfrac{ab+bc+ca}{a+b+c+\min\{a;b;c\}},\]
trong đó $a,b,c$ là các số nguyên dương.
\nguon{Titu Andreescu}
\end{btt}

\begin{btt}
Kí hiệu $(a,b)$ là ước chung lớn nhất của hai số nguyên $a,b.$ Chứng minh rằng mọi số nguyên dương $n$ đều có thể biểu diễn dưới dạng
\[n=(a,b)\left(c^2-ab\right)+(b,c)\left(a^2-bc\right)+(c,a)\left(b^2-ca\right).\]
\nguon{Kazakhstan Mathematical Olympiad 2013, Grade 9}
\end{btt}

\begin{btt}
Chứng minh rằng có vô hạn số nguyên dương $n$ sao cho $n!$ chia hết cho $n^3-1.$
	\nguon{Tạp chí Pi, tháng 3 năm 2017}
\end{btt}

\begin{btt}
Tìm tất cả các cặp số nguyên $(a, b)$ thoả mãn tính chất: 
\begin{it}
Tồn tại số nguyên $d \geq 2$ sao cho $a^{n}+b^{n}+1$ chia hết cho $d$ với mọi số nguyên dương $n.$
\end{it}
\end{btt}

\begin{btt}
Một số tự nhiên $n$ được gọi là \chu{đẹp}, nếu như $n$ có thể được biểu diễn dưới dạng $a+b+c,$ trong đó $b$ vừa chia hết cho $a$ vừa là ước của $c,$ đồng thời $a<b<c.$
\begin{enumerate}[a,]
    \item Chứng minh rằng tập các số \chu{đẹp} là vô hạn.
    \item Tính tổng tất cả các số không \chu{đẹp}.
\end{enumerate}
\nguon{Indian National Mathematical Olympiad 2011}
\end{btt}

\begin{btt}
Một số nguyên dương $n$ được gọi là \chu{đẹp} nếu như nó có thể biểu diễn dưới dạng
$$n=\dfrac{\left(x^{2}+y\right)\left(x+y^{2}\right)}{(x-y)^{2}} \text{ với } x,y \text{ là các số nguyên dương thỏa mãn }x>y.$$ 
\begin{enumerate}[a,]
    \item Chứng minh rằng có vô số số \chu{đẹp} chẵn và vô số số \chu{đẹp} lẻ.
    \item Tìm số \chu{đẹp} nhỏ nhất.
\end{enumerate}
\nguon{Olympic Toán học Nam Trung Bộ 2020}
\end{btt}

\begin{btt}
Một số nguyên dương $n$ được gọi là \chu{đẹp} nếu tồn tại các số nguyên dương $a,b,c,d$ thỏa mãn đồng thời các điều kiện
$$n\le a<b<c<d\le n+49,\quad ad=bc.$$
Hãy tìm số \chu{đẹp} lớn nhất.
\nguon{Trường thu Trung du Bắc Bộ 2019}
\end{btt}

\begin{btt}
Cho số nguyên dương $n$. Chứng minh rằng tồn tại các số tự nhiên $a$, $b$ sao cho $$0<b\leq\sqrt{n}+1,\quad \sqrt{n}\leq\dfrac{a}{b}\leq\sqrt{n+1}.$$
	\nguon{Tạp chí Pi tháng 2 năm 2017, IMO 2001}
\end{btt}

\begin{btt}
Cho $n$ là số nguyên dương. Chứng minh rằng tất cả các số lớn hơn $\dfrac{n^4}{16}$ có thể viết thành tối đa một cách dưới tích của hai ước nguyên dương của số đó mà có hiệu không vượt quá $n.$
\nguon{Saint Petersburg Mathematical Olympiad 1989}
\end{btt}

\subsection*{Hướng dẫn bài tập tự luyện}
\begin{gbtt} \
\begin{enumerate}[a,]
    \item Tìm tất cả các số tự nhiên có thể viết thành tổng của hai số nguyên lớn hơn $1$ và nguyên tố cùng nhau.
    \item Tìm tất cả các số tự nhiên có thể viết thành tổng của ba số nguyên lớn hơn $1$ và đôi một nguyên tố cùng nhau.
\end{enumerate}
\loigiai{
\begin{enumerate}[a,]
    \item Ta giả sử tồn tại số tự nhiên $n$ thỏa mãn yêu cầu bài toán. Rõ ràng $n\ge 5.$\\Ta sẽ xét trường hợp dựa theo các số dư của $n$ khi chia cho $4.$
    \begin{itemize}
        \item \chu{Trường hợp 1.} Với $n$ lẻ, ta đặt $n=2k+1.$ Dễ thấy $k\ge 2.$ \\
        Trong trường hợp này, ta lựa chọn cách biểu diễn
      $$n = \tron{k+1} + k.$$ Do $\tron{k+1, k} = 1,$ tất các số tự nhiên trong trường hợp này đều thỏa yêu cầu.
         \item\chu{Trường hợp 2.} Với $n$ chia cho $4$ dư $2,$ ta đặt $n = 4k + 2.$ Dễ thấy $k\ge 1.$\\
        Trong trường hợp này, ta lựa chọn cách biểu diễn
         $$n= \tron{2k+3} + \tron{2k-1}.$$
         Do $2k+1$ và $2k-1$ là hai số nguyên dương lẻ có khoảng cách bằng $4,$ chúng nguyên tố cùng nhau. Như vậy, trừ khi $n=6,$ các số $n$ còn lại trong trường hợp này đều thỏa yêu cầu.
         \item \chu{Trường hợp 3.} Với $n$ chia hết cho $4,$ ta đặt $n = 4k.$ Dễ thấy $k\ge2.$\\
        Trong trường hợp này, ta lựa chọn cách biểu diễn
          $$n= \tron{2k+1} + \tron{2k-1}.$$
          Do $2k+3$ và $2k-1$ là hai số nguyên dương lẻ liên tiếp, chúng nguyên tố cùng nhau. Tất cả các số $n$ trong trường hợp này đều thỏa yêu cầu.
        \end{itemize}
        Kết luận, tập các số tự nhiên $n$ thỏa mãn đề bài là $S=\left\{{n\in \mathbb{N}^*| x \ne 1,2,3,4,6}\right\}.$
    \item  Ta giả sử tồn tại số tự nhiên $n$ thỏa mãn yêu cầu bài toán. Rõ ràng $n\ge 9.$ Ta sẽ xét trường hợp sau.
    \begin{itemize}
        \item \chu{Trường hợp 1.} Nếu $n= 6x+2y,$ với $x,y$ là các số nguyên dương, ta biểu diễn
        $$n= \tron{6x+2y-8} + 3 + 5.$$
        Vì $6x+2y-8$ là số chẵn nên $3, 5, \tron{6x+2y-8}$ đôi một nguyên tố cùng nhau. \\Các số tự nhiên $n$ trong trường hợp này đều thỏa mãn yêu cầu.
        \item \chu{Trường hợp 2.} Nếu $n=6x+3,$ với $x$ là số nguyên dương, ta biểu diễn
        $$n= \tron{2x+3}+ \tron{2x+1} +\tron{2x-1}.$$
        Do $2x+3, \ 2x+1, \ 2x-1$ là ba số lẻ liên tiếp, chúng đôi một nguyên tố cùng nhau.
        \\Các số tự nhiên $n$ trong trường hợp này đều thỏa mãn yêu cầu.        
        \item \chu{Trường hợp 3.} Nếu $n=6x+1,$ với $x$ là số nguyên dương, ta biểu diễn
        \begin{align*}
            n &= \tron{2x+3} +\tron{2x+1}+ \tron{2x-3} \\
            &= \tron{2x+7} + \tron{2x-1} + \tron{2x-5}\\ 
            &= \tron{2x+5} + \tron{2x-1} + \tron{2x-3}.
        \end{align*}
        Trong cách biểu diễn trên, ít nhất một bộ ba số hạng ở vế phải nguyên tố cùng nhau đôi một.
        \\Các số tự nhiên $n$ trong trường hợp này đều thỏa mãn yêu cầu.               
        \item \chu{Trường hợp 4.} Nếu $n=6x+5,$ trong đó $x$ là các số nguyên dương, bằng cách làm hoàn toàn tương tự \chu{trường hợp 3}, ta chỉ ra tất cả các số $n$ trong trường hợp này cũng thỏa yêu cầu.
    \end{itemize}
    Kết luận, tập các số tự nhiên $n$ thỏa mãn đề bài là $S=\left\{{n\in \mathbb{N}^*| n \ge 9}\right\}.$
\end{enumerate}
}
\end{gbtt}

\begin{gbtt}
Chứng minh rằng mọi số nguyên dương $n$ đều có thể biểu diễn dưới dạng
\[n=\dfrac{ab+bc+ca}{a+b+c+\min\{a;b;c\}},\]
trong đó $a,b,c$ là các số nguyên dương.
\nguon{Titu Andreescu}
\loigiai{
Với mỗi số nguyên dương $n,$ ta chọn $a=n,b=n,c=2n.$ Ta nhận thấy rằng
$$\dfrac{ab+bc+ca}{a+b+c+\min\{a;b;c\}}=\dfrac{n^2+2n^2+2n^2}{n+n+2n+n}=n.$$
Bài toán được chứng minh.}
\begin{luuy}
\nx Chiến thuật chọn bộ số trong bài toán trên chính là chọn $a,b,c$ tỉ lệ thuận với $n.$ Cụ thể, khi chọn $a=xn,b=yn,c=zn,$ trong đó $x\le y\le z,$ ta có
$$n=\dfrac{n^2xy+n^2yz+n^2zx}{2nx+ny+nz}\Rightarrow 2x+y+z=xy+yz+zx.$$
Để tối ưu hóa quá trình tìm ra $y$ và $z$, ta sẽ chọn $x=1.$ Cách chọn này cho ta
$$y+z+2=y+z+yz\Rightarrow yz=2.$$
Ta thu được $y=1,z=2,$ và đây là cơ sở cho phép chọn ở phần lời giải.
\end{luuy}
\end{gbtt}

\begin{gbtt}
Kí hiệu $(a,b)$ là ước chung lớn nhất của hai số nguyên $a,b.$ Chứng minh rằng mọi số nguyên dương $n$ đều có thể biểu diễn dưới dạng
\[n=(a,b)\left(c^2-ab\right)+(b,c)\left(a^2-bc\right)+(c,a)\left(b^2-ca\right).\]
\nguon{Kazakhstan Mathematical Olympiad 2013, Grade 9}
\loigiai{
Ứng với mỗi số nguyên dương $n,$ bộ $(a,b,c)=\left(1,n\left(n^2+n-1\right),n(n+1)\right)$ thỏa yêu cầu bài toán, và đây là bộ ta cần chọn.}
\begin{luuy}
Do quan sát được rằng vế phải xuất hiện các đại lượng ước chung là $(a,b),(b,c),(c,a),$ ta nghĩ đến việc chọn $a=1$ nằm đưa $(a,b)$ và $(c,a)$ về $1.$ Công việc còn lại chỉ là chọn $b,c$ sao cho
$$n=c^2-b+(b,c)\left(1-bc\right)+b^2-c.$$
Nhằm triệt tiêu $n$ ở cả hai vế, ta sẽ chọn $(b,c)$ sao cho $(b,c)=n.$ Khi đó
$$c^2-b-nbc+b^2-c=0\Leftrightarrow n=\dfrac{b^2+c^2-b-c}{bc}.$$
Việc tìm ra cách chọn $b,c$ tốt nhất, mời các độc giả tự suy nghĩ.
\end{luuy}
\end{gbtt}


\begin{gbtt}
	Chứng minh rằng có vô hạn số nguyên dương $n$ sao cho $n!$ chia hết cho $n^3-1.$
	\nguon{Tạp chí Pi, tháng 3 năm 2017}
	\loigiai
	{Ta có $n^3-1=(n-1)(n^2+n+1).$
		Xét $n=a^4,$ với $a$ là số nguyên lớn hơn $1.$
		Khi đó 
		{\allowdisplaybreaks
	\begin{align*}
		n^3-1&=\left(a^4-1\right)\left(a^8+a^4+1\right)\\
		&=\left(a^4-1\right)\left[(a^8+2a^4+1)-a^4\right]\\
		&=\left(a^4-1\right)\left(a^4+a^2+1\right)\left(a^4-a^2+1\right)\\
		&=
		\left(a^4-1\right)\left(a^2+a+1\right)\left(a^2-a+1\right)\left(a^4-a^2+1\right).
		\end{align*}}Dễ thấy, với $a>1,$ các số $a^4-1$, $a^2+a+1$, $a^2-a+1$ và $a^4-a^2+1$ đôi một khác nhau và đều nhỏ hơn $a^4=n.$
		Thế nên nếu $n$ có dạng $a^4,$ thì $n!$ chia hết cho $n^3-1.$ \\
		Kết luận, có vô số số nguyên dương $n$ sao cho $n!$ chia hết cho $n^3-1.$}
\end{gbtt}

\begin{gbtt}
Tìm tất cả các cặp số nguyên $(a, b)$ thoả mãn tính chất: 
\begin{it}
Tồn tại số nguyên $d \geq 2$ sao cho $a^{n}+b^{n}+1$ chia hết cho $d$ với mọi số nguyên dương $n.$
\end{it}
\loigiai{ 
Giả sử tồn tại cặp số nguyên $(a,b)$ thỏa mãn. Xét ước nguyên tố lớn nhất $p \mid d$. Nếu $p=2$ thì một trong hai số $a,b$ lẻ. Nếu $p\ge 3,$ ta sẽ chứng minh rằng $p=3$. Thật vậy ta có
$$
\begin{aligned}
a+b \equiv-1 &\pmod{p},\\
a^{2}+b^{2} \equiv-1 &\pmod{p}, \\
a^{3}+b^{3} \equiv-1 &\pmod{p}.
\end{aligned}
$$
Ta thấy rằng $a b=\dfrac{1}{2}\left((a+b)^{2}-\left(a^{2}+b^{2}\right)\right) \equiv \dfrac{1}{2}(1-(-1))=1\pmod{p}.$ Khi đó
$$
\begin{aligned}
-1 & \equiv a^{3}+b^{3} \\
&=(a+b)\left(a^{2}+b^{2}-a b\right) \\
& \equiv(-1) \cdot(-1-1) \\
&=2 \pmod{p}.
\end{aligned}
$$
Do đó $p=3$. Từ $a^{2}+b^{2} \equiv 2\pmod{3}$ và $a+b \equiv 2\pmod{3}$ dễ dàng để thu được $$a\equiv b\pmod{3}.$$
Vậy tất cả cặp $(a,b)$ thỏa mãn là các cặp $(a,b)$ cùng lẻ và các cặp $(a,b)$ cùng chia $3$ dư $1.$}
\end{gbtt}

\begin{gbtt}
Một số tự nhiên $n$ được gọi là \chu{đẹp}, nếu như $n$ có thể được biểu diễn dưới dạng $a+b+c,$ trong đó $b$ vừa chia hết cho $a$ vừa là ước của $c,$ đồng thời $a<b<c.$
\begin{enumerate}[a,]
    \item Chứng minh rằng tập các số \chu{đẹp} là vô hạn.
    \item Tính tổng tất cả các số không \chu{đẹp}.
\end{enumerate}
\nguon{Indian National Mathematical Olympiad 2011}
\loigiai{
\begin{enumerate}
    \item Ta nhận thấy các số có dạng $7m$ là số \chu{đẹp}, với $m$ là số tự nhiên bất kì, bởi vì
    $$7m=4m+2m+m,\quad m\mid 2m,\quad 2m\mid 4m.$$
    Do tập các số tự nhiên $m$ là vô hạn, tập các số \chu{đẹp} cũng là vô hạn.
    \item Theo như chứng minh ở câu đầu tiên, nếu $n$ là số \chu{đẹp}, $kn$ cũng là số \chu{đẹp} với $k$ là số tự nhiên nào đó. Với số nguyên tố $p>5,$ ta nhận thấy rằng
    $$p=(p-3)+2+1.$$
    Vì lẽ đó, $kp$ là số \chu{đẹp} với mọi số tự nhiên $k,$ và lập luận này chứng tỏ nếu $n$ không là số \chu{đẹp}, $n$ không thể có ước nguyên tố $p>5.$ Nói cách khác,
    $$n=2^x3^y5^z,\quad x,y,z\in \mathbb{N}.$$
    Ta quan sát được
    $$2^{4}=16=12+3+1,\quad3^{2}=9=6+2+1,\quad 5^{2}=25=22+2+1.$$
    Quan sát trên cho ta biết $2^4,3^2$ và $5^2$ không phải là các số \chu{đẹp}, thế nên $$x \le 3,\quad y\le 1,\quad z\le 1.$$
    Với việc chặn $x,y,z$ như vừa rồi, $n$ chỉ có thể nhận các giá trị là
    $$1,2,3,4,5,6,8,10,12,15,20,24,30,40,60,120.$$
    Trong các số ấy, có những số \chu{đẹp} là
    \begin{multicols}{3}
    \begin{itemize}
        \item $120=112+7+1,$
        \item $60=48+8+4,$
        \item $40=36+3+1,$
        \item $30=18+9+3,$
        \item $20=12+6+2,$
        \item $15=12+2+1,$
        \item $10=6+3+1,$
    \end{itemize}
    \end{multicols}
    còn các số $1,2,3,4,5,6,8,12,24$ thì không. Như vậy, tổng cần tìm là $$1+2+3+4+5+6+8+12+24=65.$$
\end{enumerate}
}
\end{gbtt}

\begin{gbtt}
Một số nguyên dương $n$ được gọi là \chu{đẹp} nếu như nó có thể biểu diễn dưới dạng
$$n=\dfrac{\left(x^{2}+y\right)\left(x+y^{2}\right)}{(x-y)^{2}} \text{ với } x,y \text{ là các số nguyên dương thỏa mãn }x>y.$$ 
\begin{enumerate}[a,]
    \item Chứng minh rằng có vô số số \chu{đẹp} chẵn và vô số số \chu{đẹp} lẻ.
    \item Tìm số \chu{đẹp} nhỏ nhất.
\end{enumerate}
\nguon{Olympic Toán học Nam Trung Bộ 2020}
\loigiai{
\begin{enumerate}[a,]
    \item Với một số \chu{đẹp} $n$ có biểu diễn như đề bài, cho $x=y+1$, ta có
$$n=\frac{\left(x^{2}+y\right)\left(x+y^{2}\right)}{(x-y)^{2}}=\left(x^{2}+x+1\right)\left(x+(x+1)^{2}\right)=\left(x^{2}+x+1\right)\left(x^{2}+3 x+1\right).$$
Rõ ràng $x^{2}+x=x(x+1)$ và $x^{2}+3 x=x(x+3)=x(x+1)+2x$ đều là các số chẵn nên biểu thức trên là số lẻ, tức là tồn tại vô số số \chu{đẹp} lẻ. Lại cho $x=2 y+1$, ta được
$$n=\frac{\left(x^{2}+y\right)\left(x+y^{2}\right)}{(x-y)^{2}}=\frac{\left((2 y+1)^{2}+y\right)\left(2 y+1+y^{2}\right)}{(2 y+1-y)^{2}}=4 y^{2}+5 y+1.$$
Với mọi $y$ lẻ, ta có $4 y^{2}+5 y+1$ chẵn, suy ra tồn tại vô số số \chu{đẹp} chẵn. Bài toán được chứng minh.
\item Ta giả sử $n$ là số \chu{đẹp}, khi đó vì
\[\begin{aligned}
  (x - y)&\mid\left[ {\underbrace {\left( {{x^2} + y} \right) - \left( {x + {y^2}} \right)}_{ = (x - y)(x + y - 1)}} \right] \hfill, \quad
  {(x - y)^2}&\mid\left( {{x^2} + y} \right)\left( {x + {y^2}} \right) \hfill \\ 
\end{aligned} \]
nên mỗi số $x^{2}+y, x+y^{2}$ đều chia hết cho $x-y$.
Từ đây, ta có thể đặt
$$x^{2}+y=u(x-y),\quad x+y^{2}=v(x-y),$$
ở đây $u,v$ là các số nguyên dương. Rõ ràng $u>v\ge 2,$ và ta có
$$n=\frac{x^{2}+y}{x-y} \cdot \frac{x+y^{2}}{x-y}=uv,\quad x+y-1=u-v.$$
Do $x>y,$ ta có $u-v=x+y+1\ge 2y\ge 2,$ kéo theo $v\ge 2,u\ge 4,n\ge 8.$ Ta xét các trường hợp sau đây.
\begin{itemize}
    \item \chu{Trường hợp 1. }Nếu $n=8,$ ta có
    $$\heva{u&=4 \\ v&=2}
    \Rightarrow x+y-1=2
    \Rightarrow x+y=3
    \Rightarrow \heva{x&=2 \\ y&=1.}$$
    Thử lại, ta thấy $n=15,$ mâu thuẫn với $n=8.$
    \item \chu{Trường hợp 2. }Nếu $n=9,$ ta có $uv=9,$ mâu thuẫn với $1<v<u.$
    \item \chu{Trường hợp 3. }Nếu $n=10,$ ta nhận thấy cặp $(x,y)=(3,1)$ thỏa yêu cầu.
\end{itemize}
Kết luận, số \chu{đẹp} bé nhất là $10.$
\end{enumerate}}
\begin{luuy}
Đây là bài toán khá nhẹ nhàng về tính chia hết và sự tồn tại trong số học. Rõ ràng chỉ cần xét các trường hợp đặc biệt trong quan hệ giữa $x, y$ là có thể làm cho biểu thức $n$ đã cho đơn giản hơn rất nhiều. Ở ý sau, cần chú ý rằng hiệu của 2 nhân tử trên tử số là
$$\left(x^{2}+y\right)-\left(x+y^{2}\right)=(x-y)(x+y-1),$$ chia hết cho $x-y$
thì lập luận cũng dễ dàng hơn rất nhiều. Ta có thể xét
\begin{align*}
    &x=(k+1) d, y=k d \Rightarrow x^{2}+y=(k+1)^{2} d^{2}+k d, x+y^{2}=(k+1) d+k^{2} d^{2},\\
    &\left(x^{2}+y\right)\left(x+y^{2}\right)=d^{2}\left[(k+1)^{2} d+k\right]\left[(k+1)+k^{2} d\right] \text { và }(x-y)^{2}=d^{2}
\end{align*}
Suy ra $n=\left[(k+1)^{2} d+k\right]\left[(k+1)+k^{2} d\right]$ với $k, d$ là các số nguyên. Từ đây, ta có thể có nhiều câu hỏi khác liên quan đến $n$.
\nguon{Trích "Tổng hợp đề thi và lời giải trường đông ba miền 2015"}
\end{luuy}
\end{gbtt}

\begin{gbtt} \label{duyhungdz}
Một số nguyên dương $n$ được gọi là \chu{đẹp} nếu tồn tại các số nguyên dương $a,b,c,d$ thỏa mãn đồng thời các điều kiện
$$n\le a<b<c<d\le n+49,\quad ad=bc.$$
Hãy tìm số \chu{đẹp} lớn nhất.
\nguon{Trường thu Trung du Bắc Bộ 2019}
\loigiai{
Trước hết, ta sẽ chứng minh bổ đề sau đây.
\begin{light}
    Cho bốn số nguyên dương $a,b,c,d$ thỏa mãn $ad=bc.$ Khi đó, tồn tại các số nguyên dương $x,y,z,t$ thỏa mãn
    $$(x,y)=(z,t)=1,a=xt,b=yt,c=xz,d=yz.$$
\end{light}
\chu{Chứng minh.}\\
Ta đặt $x=(a,c).$ Lúc này, tồn tại các số nguyên dương $t,z$ thỏa mãn $$(t,z)=1,a=xt,c=xz.$$ Kết hợp với $ad=bc,$ phép đặt này cho ta
    $xt\cdot d=xz\cdot b,$
    hay là 
    $bx=dz.$ \\
    Ta nhận thấy $t\mid dz,$ nhưng do $(t,z)=1$ nên $t\mid d.$ Tiếp tục đặt $d=yt,$ ta được $b=yz.$ Bằng các cách đặt như vậy, ta chỉ ra được sự tồn tại của các số nguyên dương $x,y,z,t$ sao cho $$a=xt,b=yt,c=xz,d=yz.$$
    Bổ đề được chứng minh.\\
\chu{Quay lại bài toán.}\\
Giả sử tồn tại số $n$ \chu{đẹp}. Theo như bổ đề vừa phát biểu, tồn tại các số nguyên dương $x,y,z,t$ thỏa mãn
    $$(x,y)=(z,t)=1,\quad n\le xt<yt<xz<yz\le n+49.$$
    Với số $n$ \chu{đẹp} như vậy, rõ ràng, $t<z$ và $x<y,$ và từ đây ta suy ra
    $$n\le xt\le (y-1)(z-1)=yz-(y+z)+1\le \left(\sqrt{yz}-1\right)^2\le\left(\sqrt{n+49}-1\right)^2.$$
    Theo như tính chất bắc cầu, ta có
    $$n\le \left(\sqrt{n+49}-1\right)^2\Rightarrow n\le \sqrt{n+49}-1\Rightarrow \sqrt{n+49}+\sqrt{n}\le 49\Rightarrow n\le 576.$$
    Nói riêng, với $n=576,$ tất cả các dấu bằng trong đánh giá phía trên phải xảy ra, thế nên $$xt=576,yz=625,y=z=x+1=t+1.$$
    Ta tìm ra $x=t=24,y=z=25,$ nhưng lúc này $b=c=600,$ mâu thuẫn. \\
    Do đó, $n\le 575.$ Số $n=575$ rõ ràng là một số \chu{đẹp}, do
    $$575= 23\cdot 25<23\cdot26<24\cdot 25<24\cdot26=575+49.$$
    Kết luận, số \chu{đẹp} lớn nhất là $575.$}
\end{gbtt}

\begin{gbtt}
	Cho số nguyên dương $n$. Chứng minh rằng tồn tại các số tự nhiên $a$, $b$ sao cho $$0<b\leq\sqrt{n}+1,\quad \sqrt{n}\leq\dfrac{a}{b}\leq\sqrt{n+1}.$$
	\nguon{Tạp chí Pi tháng 2 năm 2017, IMO 2001}
	\loigiai{Gọi $m$ là số nguyên dương sao cho $m^{2}\leq n< (m+1)^{2}$. Dễ thấy, có duy nhất số $m$ như vậy. \\
	Đặt $s=n-m^{2}$, ta có $s$ là số tự nhiên và $0\leq s \leq 2m$. Ta xét hai trường hợp sau.
\begin{enumerate}
	\item Nếu $s$ là số chẵn, ta có
	$$n=m^{2}+s\leq m^{2}+s+\dfrac{s^{2}}{(2m)^{2}}\leq m^{2}+s+1= n+1.$$
	Suy ra $\sqrt{n}\leq m+\dfrac{s}{2m}\leq \sqrt{n+1}$. \\
	Do đó các số tự nhiên $a=m^{2}+\dfrac{s}{2}$ và $b=m$ thỏa mãn yêu cầu đề bài.
	\item Nếu $s$ là số lẻ, ta có
	{\allowdisplaybreaks
		\begin{align*}
		n&=m^{2}+s\\&=(m+1)^{2}-(2m+1-s) \\
		&\leq (m+1)^{2}-(2m+1-s)+\left(\dfrac{2m+1-s}{2(m+1)}\right)^{2} \\
		&\leq (m+1)^{2}-(2m+1-s)+1\\&
		= n+1.
		\end{align*}}Suy ra $\sqrt{n}< m+1-\dfrac{2m+s-1}{2(m+1)}\leq\sqrt{n+1}$.
\end{enumerate}
Do đó, các số tự nhiên $a=(m+1)^{2}-m+\dfrac{s-1}{2}$ và $b=m+1$ thỏa mãn yêu cầu đề bài.}
\end{gbtt}

\begin{gbtt}
Cho $n$ là số nguyên dương. Chứng minh rằng tất cả các số lớn hơn $\dfrac{n^4}{16}$ có thể viết thành tối đa một cách dưới tích của hai ước nguyên dương của số đó mà có hiệu không vượt quá $n.$
\nguon{Saint Petersburg Mathematical Olympiad 1989}
\loigiai{
Ta giả sử phản chứng rằng, tồn tại hai cách viết một số lớn hơn $\dfrac{n^4}{16}$ nào thành tích của hai ước nguyên dương của số đó, sao cho hiệu giữa chúng không vượt quá $n.$ Đối với giả sử phản chứng này, tồn tại các số nguyên dương $a,b,c,d$ sao cho 
$$ad=bc>\dfrac{n^4}{16},\ a>b\ge c>d,\ n\ge a-d.$$
Theo như kết quả của \chu{bài \ref{duyhungdz}}, tồn tại các số nguyên dương $x,y,z,t$ sao cho
$$a=xt,b=yt,c=xz,d=yz.$$
Do $a>b\ge c>d,$ ta suy ra $x>y$ và $t>z.$ Áp dụng bất đẳng thức $AM-GM$ với chú ý bên trên, ta có
$$n\ge a-d=xt-yz\ge (y+1)(z+1)-yz=y+z+1\ge 2\sqrt{yz}+1=2\sqrt{d}+1.$$
Theo như tính chất bắc cầu, ta suy ra $n\ge 2\sqrt{d}+1$ hay $\left(\dfrac{n-1}{2}\right)^2\ge d.$ Vì thế
$$a\le d+n\le n+\left(\dfrac{n-1}{2}\right)^2=\left(\dfrac{n+1}{2}\right)^2.$$
Lập luận trên cho ta nhận xét được
$$ad\le \left(\dfrac{n+1}{2}\right)^2\left(\dfrac{n-1}{2}\right)^2=\dfrac{n^4-2n^2+1}{16}<\dfrac{n^4}{16}.$$
Điều bên trên mâu thuẫn với giả sử. Giả sử phản chứng là sai. Bài toán được chứng minh.}
\end{gbtt}
 %cắt từ input 1 ra
\chapter{Số nguyên tố, hợp số}

Trong cấp học trung học cơ sở, các bạn học sinh sẽ được tiếp xúc với nhiều khái niệm mới lạ, bao gồm số nguyên tố (tiếng anh được gọi là \chu{prime number}). Theo nhiều tài liệu ghi chép lại, số nguyên tố lần đầu tiên xuất hiện trong các công trình toán học Hi Lạp cổ đại. Nhà toán học Hi Lạp lỗi lạc $Euclid$ đã đưa ra những định lí cơ bản về số học, trong đó bao gồm cả chứng minh sự tồn tại của vô hạn các số nguyên tố. Riêng đối với cuốn sách này, chương III tập trung nghiên cứu vào các vấn đề liên quan đến số nguyên tố, cụ thể là các tính chất từ cơ bản đến nâng cao của chúng. Chương được chia ra làm $5$ phần
\begin{itemize}
    \item\chu{Phần 1.} Về số dư của số nguyên tố trong phép chia cho một số nguyên dương.
    \item\chu{Phần 2.} Phân tích tiêu chuẩn của một số nguyên tố.
    \item\chu{Phần 3.} Ứng dụng của đồng dư thức.
    \item\chu{Phần 4.} Định lí $Fermat$ và ứng dụng.
    \item\chu{Phần 5.} Số nguyên tố và tính nguyên tố cùng nhau.
\end{itemize}


\section{Các định nghĩa, tính chất cơ bản}
\subsection{Các định nghĩa}
\begin{light}
\chu{Định nghĩa 1.} Số nguyên tố là số tự nhiên lớn hơn $1,$ chỉ có hai ước dương là $1$ và chính nó.
\end{light} 

Chẳng hạn
\begin{enumerate}
    \item Số $2$ được coi là số nguyên tố bởi vì nó có đúng hai ước số tự nhiên lớn hơn $1,$ đó là $1$ và $2.$ Đây cũng là số nguyên tố nhỏ nhất và là số nguyên tố chẵn duy nhất.
    \item Số $27$ không được coi là số nguyên tố bởi vì nó có bốn ước tự nhiên là $1,3,9$ và $27.$  
\end{enumerate}
\begin{light}
\chu{Định nghĩa 2.} Hợp số là số tự nhiên lớn hơn 1 và có nhiều hơn hai ước dương.
\end{light}
Chẳng hạn
\begin{enumerate}
    \item Số $6$ là hợp số vì nó có $4$ ước dương là $1,2,3$ và $6$.
    \item Số $13$ không là hợp số vì nó chỉ có hai ước dương là $1$ và $13$. Số $13$ là số nguyên tố.
\end{enumerate}
Chú ý rằng, hai số 0 và 1 không là số nguyên tố, cũng không là hợp số.
\subsection{Một số tính chất cơ bản}
\begin{enumerate}
    \item Mỗi số tự nhiên lớn hơn 1 đều có duy nhất một cách phân tích ra thừa số nguyên tố. Chẳng hạn, ta có cách viết các số $11,100,2408$ thành tích các thừa số nguyên tố\[11=11,\quad 100=2^5\cdot 5^2,\quad 2408=2^3\cdot 7\cdot 43.\]Các cách viết trên là duy nhất!
    \item Tập hợp các số nguyên tố là vô hạn.
    \item Với $a,b$ là các số nguyên và tích $ab$ chia hết cho số nguyên tố $p$, khi đó ít nhất một trong hai số $a,b$ chia hết cho $p.$
\end{enumerate}

\section{Về số dư của số nguyên tố trong phép chia cho một số nguyên dương}
\setcounter{bx}{0}
Tiểu mục này chủ yếu nghiên cứu về một vài số nguyên tố đầu tiên của bảng, và số dư của nó khi đem chia cho một vài số khác.
\subsection{Ví dụ minh họa}

\begin{bx}
Tìm tất cả các số tự nhiên $n\ge 1$ sao cho $n+1,\:n+3$ và $n+11$ đều là số nguyên tố.
\loigiai{
Ba số nguyên $n+1,n+3,n+11$ khác số dư nhau khi chia cho $3,$ bởi vì
$$n+11\equiv n+2\pmod{3},\quad n+3\equiv n\pmod{3},\quad n+1\equiv n+1\pmod{3}.$$
Theo đó, trong chúng có một số chia hết bằng $3.$ Các số $n+3$ và $n+11$ đều lớn hơn $3,$ vì vậy chỉ tồn tại trường hợp $n+1=3.$ Đáp số bài toán là $n=2.$}
\end{bx}

\begin{bx}
Số dư của một số nguyên tố trong phép chia cho $30$ không phải là một số nguyên tố. Tìm số dư đó.
\loigiai{
Trong phép chia đã cho, ta lần lượt gọi số bị chia là $p,$ thương là $q,$ số dư là $r\le 30.$ Ta có
$p=30q+r.$
Do $r$ không là số nguyên tố, ta xét các trường hợp.
\begin{enumerate}
    \item Nếu $r$ là hợp số nhỏ hơn $30,$ ước nguyên tố của $r$ chỉ có thể là $2,3,5.$ Lúc này số $p$ sẽ là hợp số vì nó nhận ít nhất một trong ba số $2,3,5$ làm ước và lớn hơn chính ước ấy. Điều này mâu thuẫn với giả thiết.
    \item Nếu $r=1,$ ta sẽ chỉ cần chỉ ra một số nguyên tố $p$ thỏa mãn, chẳng hạn như $p=31.$
\end{enumerate}
Nói tóm lại, số dư cần tìm là $r=1.$}
\end{bx}

\begin{bx} \label{snt01} \
\begin{enumerate}[a,] 
    \item Chứng minh rằng một số nguyên tố $p\ge 3$ chỉ có thể có dạng $4k+1$ hoặc $4k+3,$ với $k$ là số nguyên dương.
    \item Chứng minh rằng một số nguyên tố $p\ge 5$ chỉ có thể có dạng $6k+1$ hoặc $6k+5,$ với $k$ là số nguyên dương.
\end{enumerate}
\loigiai{
\begin{enumerate}[a,] 
    \item Với $k$ nguyên dương, ta sẽ chứng minh một số nguyên tố $p$ không thể có các dạng còn lại. Thật vậy
    \begin{itemize}
        \item Nếu $p=4k,$ ta suy ra $p$ là hợp số vì nó có ba ước $1,2,4k.$
        \item Nếu $p=4k+2=2(2k+1),$ ta suy ra $p$ là hợp số vì nó có ba ước $1,2,2(2k+1).$      
    \end{itemize}
    \item Với $k$ nguyên dương, ta sẽ chứng minh một số nguyên tố $p$ không thể có các dạng còn lại. Thật vậy
    \begin{itemize}
        \item Nếu $p=6k,$ ta suy ra $p$ là hợp số vì nó có ba ước  $1,2,3.$
        \item Nếu $p=6k+2=2(3k+1),$ ta suy ra $p$ là hợp số vì nó có ba ước $1,2,2(3k+1).$   
        \item Nếu $p=6k+3=3(2k+1),$ ta suy ra $p$ là hợp số vì nó có ba ước $1,3,3(2k+1).$
        \item Nếu $p=6k+4=2(3k+2),$ ta suy ra $p$ là hợp số vì nó có ba ước $1,2,2(3k+2).$        
    \end{itemize}
    Hoàn tất chứng minh.
\end{enumerate}
}
\end{bx}

\begin{bx}
Tìm bốn số tự nhiên $x_{1}<x_{2}<x_{3}<x_{4}$ sao cho tất cả $6
$ hiệu giữa chúng đều là số nguyên tố.
\loigiai{Từ giả thiết, ta có thể đặt
\begin{align*}
    p_1=x_4-x_3,\qquad p_2=x_3-x_2,\qquad p_3=x_2-x_1,\\
    p_4=x_4-x_2,\qquad p_5=x_3-x_1,\qquad p_6=x_4-x_1.
\end{align*}
Ta đã biết, số nguyên tố chẵn duy nhất là số $2.$ Căn cứ vào lập luận này, ta xét các trường hợp dưới đây.
\begin{enumerate}
    \item Nếu $p_1,p_2,p_3$ cùng tính chẵn lẻ, ta có $p_4=p_1+p_2$ và $p_5=p_2+p_3$ cùng là số chẵn và lớn hơn $2$ nên không thể là số nguyên tố, mâu thuẫn.
    \item Nếu trong $p_1,p_2,p_3$ có một số bằng $2$ và hai số lẻ, ta có $p_6=p_1+p_2+p_3$ là số chẵn và lớn hơn $2$ nên không thể là số nguyên tố, mâu thuẫn.
    \item Nếu trong $p_1,p_2,p_3$ có một số bằng $2$ và hai số lẻ, ta chia bài toán thành các trường hợp sau.
    \begin{itemize}
        \item \chu{Trường hợp 1. }Nếu ${p}_{1}$ là số lẻ và  ${p}_{2}={p}_{3}=2,$ ${p}_{5}=4$ là hợp số, mâu thuẫn.
        \item \chu{Trường hợp 2. }Nếu ${p}_{3}$ là số lẻ và ${p}_{1}={p}_{2}=2,$  ${p}_{4}=4$ là hợp số, mâu thuẫn.
        \item \chu{Trường hợp 3. }Nếu ${p}_{2}$ là số lẻ và ${p}_{1}={p}_{3}=2$, ta có ${p}_{4}={p}_{5}={p}_{2}+2 $ và ${p}_{6}={p}_{2}+4$ đều là các số nguyên tố. Lần lượt xét các số dư của $p_2$ khi chia cho $3,$ ta chỉ ra $p_2=3,p_5=5,p_6=7.$
    \end{itemize}
\end{enumerate}
Tổng kết lại, tất cả các bộ số thỏa yêu cầu đều có dạng
$$\left(x_1,x_2,x_3,x_4\right)=\left(a,a+2,a+5,a+7\right),$$
trong đó $a$ là một số tự nhiên tùy ý.}
\end{bx}

\begin{bx} \label{snt02} 
Một cặp hai số nguyên tố liên tiếp cách nhau 2 đơn vị còn được gọi là cặp số nguyên tố sinh đôi $-$ \chu{twin primes}. Chứng minh rằng một cặp số nguyên tố lớn hơn $5$ sinh đôi bất kì đều có dạng $(6k-1,6k+1),$ trong đó $k$ là số nguyên dương.
\loigiai{
Ta giả sử $(p,p+2)$ là một cặp nguyên tố sinh đôi thỏa mãn $p>5.$ Ta xét các trường hợp sau.
\begin{enumerate}
    \item Nếu $p$ có dạng $6k+1$ với $k$ nguyên dương, ta có $p+2=6k+3=3(2k+1)$ là hợp số, mâu thuẫn.
    \item Nếu $p$ có dạng $6k-1$ với $k$ nguyên dương, ta có $p+2=6k+1.$
\end{enumerate}
Bài toán được chứng minh.}
\end{bx}

\begin{bx}
Cho số nguyên dương $p.$ Chứng minh rằng nếu $p$ và $8p^2+1$ là số nguyên tố thì $8p^2-1$ là hợp số.
\loigiai{
Rõ ràng $8p^2-1\ge 7.$ Ta giả sử phản chứng rằng $8p^2-1$ là số nguyên tố. Lúc này, do $8p^2-1$ và $8p^2+1$ là hai số nguyên tố sinh đôi lớn hơn $5$ nên theo \chu{ví dụ \ref{snt02}}, tồn tại số nguyên dương $k$ sao cho
$$8p^2-1=6k-1,\qquad 8p^2+1=6k+1.$$
Ta nhận được $8p^2=6k.$ Ta lần lượt suy ra
$$3\mid 4p^2\Rightarrow 3\mid p^2\Rightarrow 3\mid p\Rightarrow p=3.$$
Tuy nhiên, với $p=3,$ ta có $8p^2+1=25$ là hợp số, mâu thuẫn với giả sử.\\
Giả sử đã cho là sai. Bài toán được chứng minh.}
\end{bx}

\subsection{Bài tập tự luyện}

\begin{btt}
Tìm số nguyên tố $n$ sao cho $n-2,n+6,n+12$ và $n+14$  đều là số nguyên tố.
\end{btt}

\begin{btt}
Tìm số nguyên dương $n$ sao cho dãy 
$$n+1, \:n+2,\:n+3,\ldots,\:n+10$$ chứa nhiều số nguyên tố nhất có thể.
\end{btt}

\begin{btt}
Chứng minh rằng trong $10$ số lẻ liên tiếp lớn hơn $5$, tồn tại ít nhất $4$ hợp số. Đồng thời, hãy chỉ ra một dãy thỏa mãn điều kiện trên.
\end{btt}

\begin{btt}
Tìm tất cả các cặp số nguyên tố $(p,q)$ cho $p^2-2q^2=1.$
\end{btt}

\begin{btt}
Tìm tất cả các số nguyên tố $p$ sao cho $10p^3-7$ và $10p^3+7$ đều là số nguyên tố.
\end{btt}

\begin{btt}
Tìm tất cả các số nguyên dương $n$ sao cho $2^n-1$ và $2^n+1$ đều là số nguyên tố. 
\end{btt}

\begin{btt}
Số hoàn hảo $-$ \chu{perfect number} $-$ là một số nguyên dương có các ước nguyên dương (không tính nó) bằng chính số đó. Tìm tất cả các số hoàn hảo $n$ sao cho $n+1$ và $n-1$ là hai số nguyên tố sinh đôi.
\nguon{Đề thi thử chuyên Khoa học Tự nhiên 2017}
\end{btt}

\begin{btt}
Cho hai dãy các số nguyên tố $5<p_{1}<p_{2}<p_{3}<p_{4}$ và $5<q_{1}<q_{2}<q_{3}<q_{4}$ thỏa mãn $p_{4}-p_{1}=8$ và $q_{4}-q_{1}=8.$ Chứng minh rằng $p_1-q_1$ chia hết cho $30.$
\nguon{Indian National Mathematical Olympiad 2011}
\end{btt}

\begin{btt}
Cho $m,p,r$ là các số nguyên tố thỏa mãn $mp+1=r.$ Chứng minh rằng $m^2+r$ hoặc $p^2+r$ là số chính phương.
\nguon{Chuyên Toán Kiên Giang 2021}
\end{btt}

\begin{btt}
Tìm tất cả bộ ba số nguyên tố $(p,q,r)$ thỏa mãn $pq=r+1$ và $2\left(p^2+q^2\right)=r^2+1.$
\nguon{Chuyên Toán Quảng Nam 2021}
\end{btt}

\begin{btt}
Tìm các số nguyên tố $p,q$ đồng thời thỏa mãn hai điều kiện
    \begin{enumerate}[i,]
        \item $p^2q+p$ chia hết cho $p^2+q.$
        \item $pq^2+q$ chia hết cho $q^2-p.$
    \end{enumerate}
\nguon{Chuyên Toán Phú Thọ 2021}
\end{btt}

\begin{btt}
Tìm tất cả các số nguyên tố $p,q$ thỏa mãn $7p+q$ và $pq+11$ cũng là số nguyên tố.
\end{btt}

\begin{btt}
Tìm các số nguyên tố $x,y,z$ thỏa mãn $x^2+3xy+y^2=5^z$ 
\end{btt}

\begin{btt}
Tìm tất cả các bộ ba số nguyên tố \(\left ( p,q,r \right )\) thỏa mãn $p<q<r$ và
\[\dfrac{p^2+2q}{q+r},\quad \dfrac{q^2+9r}{r+p},\quad \dfrac{r^2+3p}{p+q}\]
đều là các số nguyên.
\nguon{Junior Balkan Mathematical Olympiad Shortlist 2020}
\end{btt}

\begin{btt}
Tìm tất cả các cặp số nguyên dương $ (a, b) $ thỏa mãn đúng 3 trong 4 điều kiện dưới đây
\begin{multicols}{2}
\begin{enumerate}
	\item[i,] $a=5b+9 $.
	\item[ii,] $a+6 $ chia hết cho $ b $.
	\item[iii,] $a+2017b $ chia hết cho $ 5 $.
	\item[iv,] $a+7b $ là số nguyên tố.
\end{enumerate}
\end{multicols}
\nguon{Tạp chí Pi, tháng 4 năm 2017}
\end{btt}

\subsection{Hướng dẫn bài tập tự luyện}

\begin{gbtt}
Tìm số nguyên tố $n$ sao cho $n-2,n+6,n+12$ và $n+14$  đều là số nguyên tố.
\loigiai{
Năm số nguyên tố $n-2,n,n+6, n+12$ và $n+14$ khi chia cho $5$ sẽ có $5$ số dư khác nhau, chứng tỏ trong đó có số $5.$ 
\begin{enumerate}
    \item Nếu $n=5,$ tất cả các số đã cho đều là số nguyên tố.
    \item Nếu $n-2=5$, ta có $n=7$, lúc này $n+14=21$ không phải là số nguyên tố.
\end{enumerate}
Đáp số bài toán là $n=5.$}
\end{gbtt}

\begin{gbtt}
Tìm số nguyên dương $n$ sao cho dãy 
$$n+1, \:n+2,\:n+3,\ldots,\:n+10$$ chứa nhiều số nguyên tố nhất có thể.
\loigiai{
Ta thấy $n+1, \:n+2,\: \ldots,n+10$ là $10$ số tự nhiên liên tiếp. Ta xét các trường hợp sau.
\begin{enumerate}
    \item Với $n=0$, dãy đã cho có $4$ số nguyên tố là $2,3,5,7.$
    \item Với ${n}=1$, dãy đã cho có $5$ số nguyên tố là $2,3,5,7,11.$
    \item Với $n>1$, dãy sẽ gồm $5$ số chẵn lớn hơn $2,$ và ít nhất một trong $5$ số lẻ còn lại chia hết cho $3$ và lớn hơn $3.$ Như vậy, trường hợp này cho ta không quá $4$ số nguyên tố trong dãy.
\end{enumerate}
Đáp số bài toán là $n=1.$}
\end{gbtt}

\begin{gbtt}
Chứng minh rằng trong $10$ số lẻ liên tiếp lớn hơn $5$, tồn tại ít nhất $4$ hợp số. Đồng thời, hãy chỉ ra một dãy thỏa mãn điều kiện trên.
\loigiai{
Trong $10$ số lẻ liên tiếp đã cho, tồn tại 
\begin{enumerate}
    \item[i,] Ít nhất ba số là bội của $3,$ và lớn hơn $3.$
    \item[ii,] Đúng hai số là bội của $5,$ và lớn hơn $5.$
    \item[iii,] Tối đa ba số vừa là bội của $3,$ vừa là bội của $15.$
\end{enumerate}
Như vậy, số lượng hợp số ít nhất trong dãy $10$ số lẻ liên tiếp lớn hơn $5$ là $3+2-1=4.$ \\Chẳng hạn, dãy $7,9,11,13,15,17,19,21,23,25$ thỏa điều kiện.}
\end{gbtt}

\begin{gbtt}
Tìm tất cả các cặp số nguyên tố $(p,q)$ cho $p^2-2q^2=1.$
\loigiai{
Rõ ràng $p$ là số nguyên tố lẻ. Đặt $p=2k+1$ với $k$ nguyên dương, ta có
$$(2 {k}+1)^{2}=2 {q}^{2}+1 \Leftrightarrow 4 {k}^{2}+4 {k}+1=2 {q}^{2}+1 \Leftrightarrow 2 {k}({k}+1)={q}^{2}.$$
Do đó $q^2$ là số chẵn nên $q$ cũng chẵn. Kết hợp với việc $q$ là số nguyên tố, ta có $q=2.$\\ Thay trở lại, ta tìm ra $(p,q)=(3,2)$ là cặp số nguyên tố thỏa mãn.}
\end{gbtt} 

\begin{gbtt}
Tìm tất cả các số nguyên tố $p$ sao cho $10p^3-7$ và $10p^3+7$ đều là số nguyên tố.
\loigiai{
Nếu $10p^3-7=6k+1,$ ta có $10p^3+7=6k+15$ là hợp số lớn hơn $3,$ trái điều kiện. Do vậy, ta có 
$$10p^3-7=6k+5.$$ 
Suy ra $10p^3=6k+12=3(2k+6).$
Do $(10,3)=1$ nên $p^3$ chia hết cho $3.$ Chỉ có $p=3$ thỏa mãn điều kiện này. Thử lại, ta kết luận $p=3$ là số nguyên tố duy nhất thỏa yêu cầu.}
\end{gbtt}

\begin{gbtt}
Tìm tất cả các số nguyên dương $n$ sao cho $2^n-1$ và $2^n+1$ đều là số nguyên tố.
\loigiai{
Thử trực tiếp với $n=1,2,$ ta thấy $n=2$ thỏa mãn. \\
Với $n\ge 3,$ ta có $2^n-1$ và $2^n+1$ là hai số nguyên tố sinh đôi lớn hơn $5$.\\
Áp dụng kết quả của \chu{ví dụ \ref{snt02}}, tồn tại số nguyên dương $k$ sao cho 
$$2^n-1=6k-1,\qquad 2^n+1=6k+1.$$
Ta nhận được $2^n=6k,$ điều này vô lí do $2^n$ không chia hết cho $3.$ Đáp số của bài toán là $n=2.$}
\end{gbtt}

\begin{gbtt}
Số hoàn hảo $-$ \chu{perfect number} $-$ là một số nguyên dương có các ước nguyên dương (không tính nó) bằng chính số đó. Tìm tất cả các số hoàn hảo $n$ sao cho $n+1$ và $n-1$ là hai số nguyên tố sinh đôi.
\nguon{Đề thi thử chuyên Khoa học Tự nhiên 2017}
\loigiai{
Theo như chứng minh ở \chu{ví dụ \ref{snt02}}, ta suy ra $n$ chia hết cho $6.$ Trong trường hợp $n\ge 37,$ tập $$A=\Bigg\{1;2;3;6;\dfrac{n}{6};\dfrac{n}{3};\dfrac{n}{2};n\Bigg\}.$$ là tập con của tập ước của $n.$ Tổng các phần tử của $A$ bằng
$$1+2+3+6+\dfrac{n}{6}+\dfrac{n}{3}+\dfrac{n}{2}+n=2n+10.$$
Tổng này lớn hơn $2n,$ và các số $n\ge 37$ không thỏa mãn. Do đó, $n$ là tất cả bội dương của $6$ nhỏ hơn $36.$ Thử trực tiếp với $n=6,12,18,24,30,$ ta chỉ ra $n=6$ là đáp số bài toán.}
\end{gbtt}

\begin{gbtt}
Cho hai dãy các số nguyên tố $5<p_{1}<p_{2}<p_{3}<p_{4}$ và $5<q_{1}<q_{2}<q_{3}<q_{4}$ thỏa mãn $p_{4}-p_{1}=8$ và $q_{4}-q_{1}=8.$ Chứng minh rằng $p_1-q_1$ chia hết cho $30.$
\loigiai{
Do $p_1>5$ nên $p_1$ có dạng $6k+1$ hoặc $6k+5.$ Nếu $p_1=6k+1$ thì $p_4=6k+9$ là hợp số, trái giả thiết. Như vậy, $p_1$ có dạng $6k+5.$ Ngoài ra, bằng việc xét số dư của $p_1$ khi chia cho $10,$ ta dễ dàng chỉ ra $p_1$ nhận một trong các dạng $10l+1,10l+3,10l+7,10l+9.$
\begin{enumerate}
    \item Nếu $p_1=10l+1,$ ta có thể thấy bộ $\tron{p_1,p_2,p_3,p_4}=\tron{11,13,17,19}$ là một cấu hình thỏa mãn.
    \item Nếu $p_1=10l+3$ thì $p_4=10l+11,$ đồng thời $$10l+3<p_2<p_3<10l+11.$$ Do $p_2$ và $p_3$ là hai số lẻ khác $10l+5$ nên $p_2=10l+7,p_3=10l+9.$ Ta có $p_2,p_3,p_4$ là ba số lẻ liên tiếp lớn hơn $3$ nên trong chúng có một hợp số chia hết cho $3,$ trái giả thiết.
    \item Nếu $p_1=10l+7$ thì $p_4=10l+15$ là hợp số chia hết cho $5,$ trái giả thiết.
    \item Nếu $p_1=10l+9$ thì $p_4=10l+17,$ đồng thời $$10l+9<p_2<p_3<10l+17.$$ Do $p_2$ và $p_3$ là hai số lẻ khác $10l+15$ nên $p_2=10l+11,p_3=10l+13.$ Ta có $p_1,p_2,p_3$ là ba số lẻ liên tiếp lớn hon $3$ nên trong chúng có một hợp số chia hết cho $3,$ trái giả thiết.    
\end{enumerate}
Nói chung, $p_1$ chia $6$ dư $5$ và chia $10$ dư $1.$ Chứng minh tương tự, $q_1$ cũng có cùng số dư khi chia cho $6$ và $10.$ Như vậy $p_1-q_1$ chia hết cho $[6,10]=30.$ Bài toán được chứng minh.}
\end{gbtt}

\begin{gbtt}
Cho $m,p,r$ là các số nguyên tố thỏa mãn $mp+1=r.$ Chứng minh rằng $m^2+r$ hoặc $p^2+r$ là số chính phương.
\nguon{Chuyên Toán Kiên Giang 2021}
\loigiai{
Không mất tính tổng quát, ta giả sử $m\ge p.$ Ta xét các trường hợp sau.
\begin{enumerate}
    \item Nếu $m,p,r$ cùng lẻ, ta có $mp+1$ chẵn, nhưng $r$ lẻ, vô lí.
    \item Nếu $r=2,$ ta có $mp+1=2,$ hay là $mp=1,$ vô lí.
    \item Nếu $p=2,$ ta có $r=2m+1.$ Ta nhận thấy rằng
        $m^2+r=(m+1)^2$
    là số chính phương.
\end{enumerate}
Bài toán được chứng minh.}
\end{gbtt}

\begin{gbtt}
Tìm tất cả bộ ba số nguyên tố $(p,q,r)$ thỏa mãn $pq=r+1$ và $2\left(p^2+q^2\right)=r^2+1.$
\nguon{Chuyên Toán Quảng Nam 2021}
\loigiai{Ta chia bài toán thành các trường hợp sau.
\begin{enumerate}
    \item Với $p,q,r$ là ba số nguyên tố lẻ, ta có $pq$ lẻ, còn $r+1$ chẵn. Điều này vô lí.
    \item Với $r=2,$ ta có $pq=3.$ Điều này vô lí.
    \item Với $q=2,$ ta thu được hệ phương trình
    $$\heva{&2p=r+1 \\ &2\left(p^2+4\right)=r^2+1}
    \Leftrightarrow \heva{&r=2p-1 \\ &2\left(p^2+4\right)=(2p-1)^2+1}
    \Leftrightarrow \heva{&r=2p-1 \\ &(p-3)(p+2)=0.}$$
    Do $p$ nguyên tố, ta được $p=3$ và $r=2.$
    \item Với $p=2,$ làm tương tự trường hợp trên, ta được $q=3$ và $r=2.$
\end{enumerate}
Kết quả, có $2$ bộ $(p,q,r)$ thỏa mãn đề bài là $(2,3,2)$ và $(3,2,2).$}
\end{gbtt}

\begin{gbtt}
Tìm các số nguyên tố $p,q$ đồng thời thỏa mãn hai điều kiện
    \begin{enumerate}[i,]
        \item $p^2q+p$ chia hết cho $p^2+q.$
        \item $pq^2+q$ chia hết cho $q^2-p.$
    \end{enumerate}
\nguon{Chuyên Toán Phú Thọ 2021}
\loigiai{
Giả sử tồn tại các số nguyên tố $p,q$ thỏa mãn đề bài. Điều kiện i cho ta
    $$\left(p^{2}+q\right)\mid\left(q(p^2+q)+p-q^{2}\right) \Rightarrow \left(p^{2}+q\right)\mid\left(p-q^{2}\right).$$
    Đồng thời, điều kiện ii cho ta
    $$\left(q^{2}-p\right)\mid\left(p\left(q^2-p\right)+q+p^{2}\right)\Rightarrow \left(q^{2}-p\right)\mid\left(q+p^{2}\right).$$
    Dựa vào hai nhận xét trên, ta có $\left|p^{2}+q\right|=\left|q^{2}-p\right|.$ Ta xét các trường hợp sau.
\begin{enumerate}
    \item Với $q^2\ge p,$ ta lần lượt suy ra
        $$p^2+q=q^2-p\Rightarrow (p+q)(p-q+1)=0\Rightarrow q=p+1.$$
        Lúc này $p,q$ là hai số nguyên tố liên tiếp, và bắt buộc $p=2,q=3.$
    \item Với $p>q^2,$ ta lần lượt suy ra
        $$p^2+q=p-q^2\Rightarrow p(p-1)+q(q+1)=0,$$
        điều này là không thể xảy ra do $p^2>p$ và $q^2>-q.$
\end{enumerate}
Kết luận $(p,q)=(2,3)$ là cặp số nguyên tố duy nhất thỏa mãn.}    
\end{gbtt}

\begin{gbtt}
Tìm tất cả các số nguyên tố $p,q$ thỏa mãn $7p+q$ và $pq+11$ cũng là số nguyên tố.
\loigiai{
Giả sử tồn tại các số nguyên tố $p,q$ thoả yêu cầu. Trong bài toán này, ta xét các trường hợp sau.
\begin{enumerate}
    \item Nếu $p,q$ cùng lẻ thì $7p+q$ là số chẵn lơn hơn $2,$ trái giả sử.
    \item Nếu $p=2,$ ta có $q+14$ và $2q+11$ đều là số nguyên tố.
    \begin{itemize}
        \item Nếu $q=3$ thì hai số trên nguyên tố.
        \item Nếu $q$ chia $3$ dư $1,$ ta có $q+14$ chia hết cho $3$ và lớn hơn $3,$ do vậy không là số nguyên tố.
        \item Nếu $q$ chia $3$ dư $2,$ ta có $2q+11$ chia hết cho $3$ và lớn hơn $3,$ do vậy không là số nguyên tố.   
    \end{itemize}
    \item Nếu $q=2,$ bằng cách xét tương tự trường hợp vừa rồi, ta tìm được $p=3.$
\end{enumerate}
Kết luận, có hai cặp $(p,q)$ thoả yêu cầu là $(2,3)$ và $(3,2).$}
\begin{luuy}
Ngoài cách chia trường hợp như trên, ta còn có thể tiến hành bài toán theo một cách chia khác
\begin{enumerate}
    \item $p\equiv q\pmod{3}\Rightarrow pq+11\equiv 0\pmod{3}.$
    \item $3\mid (p+q)\Rightarrow 7p+q\equiv 0\pmod{3}.$
    \item $p=3\Rightarrow q=2.$
    \item $p=2\Rightarrow q=3.$
\end{enumerate}
\end{luuy}
\end{gbtt}

\begin{gbtt}
Tìm các số nguyên tố $x,y,z$ thỏa mãn $x^2+3xy+y^2=5^z$ 
\loigiai{
Phương trình đã cho tương đương với
$$x^2-2xy+y^2+5xy=5^z \Leftrightarrow (x-y)^2+5xy=5^z.$$
Với giả sử tồn tại các số nguyên tố $x,y,z$ thỏa yêu cầu, ta có
$$5\mid(x-y)^2\Rightarrow 5\mid (x-y)\Rightarrow 25\mid (x-y)^2\Rightarrow 25\mid 5xy\Rightarrow 5\mid xy\Rightarrow\hoac{5\mid x\\ 5\mid y}\Rightarrow \hoac{x&=5\\ y&=5.}$$
Kkhông mất tổng quát, ta giả sử $y=5.$ Từ $x-y$ chia hết cho $5$ và $y=5,$ ta lại suy ra được $x$ chia hết cho $5,$ và bắt buộc $x=5.$ Bằng cách thế trở lại, ta kết luận $(x,y,z)=(5,5,3)$ là bộ số nguyên tố duy nhất thỏa yêu cầu.}
\end{gbtt}

\begin{gbtt}
Tìm tất cả các bộ ba số nguyên tố \(\left ( p,q,r \right )\) thỏa mãn $p<q<r$ và
\[\dfrac{p^2+2q}{q+r},\quad \dfrac{q^2+9r}{r+p},\quad \dfrac{r^2+3p}{p+q}\]
đều là các số nguyên.
\nguon{Junior Balkan Mathematical Olympiad Shortlist 2020}
\loigiai{
Nếu $p,q,r$ cùng lẻ, ta có số lẻ $p^2+2q$ chia hết cho số chẵn $q+r,$ mâu thuẫn. Do vậy, một trong ba số $p,q,r$ phải bằng $2.$ Với việc $p<q<r,$ ta có $p=2.$ Ta viết lại hệ điều kiện đã cho thành
    $$\tron{q+r}\mid\tron{4+2q},\quad \tron{r+2}\mid\tron{q^2+9r},\quad \tron{2+q}\mid\tron{r^2+6}.$$
    Từ $\tron{q+r}\mid \tron{4+2q}$ và $2(q+r)\le 4+2q,$ ta nhận thấy hoặc $2(q+r)=4+2q,$ hoặc $q+r=4+2q.$
\begin{enumerate}
    \item Nếu $2q+2r=4+2q$ thì $r=2.$ Kết hợp với $\tron{2+q}\mid\tron{r^2+6}$ thì $q=3.$ Thử với điều kiện còn lại là $\tron{r+2}\mid\tron{q^2+9r},$ ta thấy không thỏa.
    \item Nếu $q+r=4+2q$ thì $r=q+4.$ Thế trở lại điều kiện thứ ba, ta có $$\tron{2+q}\mid\tron{\tron{q+4}^2+6},$$ suy ra $\tron{2+q}\mid 10.$ Trường hợp này cũng cho ta $q=3$ và $r=7.$ Thử lại, ta thấy thỏa mãn.    
\end{enumerate}
Như vậy, $(p,q,r)=(2,3,7)$ là bộ số duy nhất thỏa yêu cầu.}
\end{gbtt}

\begin{gbtt}
Tìm tất cả các cặp số nguyên dương $ (a, b) $ thỏa mãn đúng 3 trong 4 điều kiện dưới đây
\begin{multicols}{2}
\begin{enumerate}
	\item[i,] $a=5b+9 $.
	\item[ii,] $a+6 $ chia hết cho $ b $.
	\item[iii,] $a+2017b $ chia hết cho $ 5 $.
	\item[iv,] $a+7b $ là số nguyên tố.
\end{enumerate}
\end{multicols}
\nguon{Tạp chí Pi, tháng 4 năm 2017}
	\loigiai{
		Giả sử $ (a, b) $ là cặp số nguyên dương thỏa mãn đề bài.
		Ta có các nhận xét sau.
\begin{enumerate}[\color{tuancolor}\bfseries\sffamily Nhận xét 1.]
	\item $ (a, b) $ không thể thỏa mãn đồng thời các điều kiện \chu{i} và \chu{iv}.\\
		Thật vậy, giả sử ngược lại, $(a, b)$ thỏa mãn đồng thời \chu{i}  và \chu{iv}. Khi đó, phải có 
		$$ a+7b=(5b+9)+7b=12b+9 $$
		là số nguyên tố, vô lí vì $ 12b+9 $ lớn hơn $ 3 $ và chia hết cho $ 3 $.
	\item $ (a, b) $ không thể thỏa mãn đồng thời các điều kiện \chu{iii} và \chu{iv}.\\
		Thật vậy, giả sử ngược lại, $ (a, b) $ thỏa mãn đồng thời \chu{iii} và \chu{iv}. Khi đó, phải có
		$$ a+2017b=(a+7b)+2010b$$
		chia hết cho $ 5 $, kéo theo $ a+7b $ chia hết cho $ 5 $, mâu thuẫn với việc $ a+7b $ là số nguyên tố.
	\end{enumerate}
		Từ hai nhận xét trên, hiển nhiên $ (a, b) $ không thể thỏa mãn \chu{iv}, vì nếu ngược lại thì $ (a, b) $ chỉ có thể thỏa mãn tối đa hai điều kiện \chu{ii} và \chu{iv}, trái với yêu cầu của đề bài.
		Như vậy, $ (a, b) $ thỏa mãn đồng thời \chu{i}, \chu{ii} và \chu{iii}.
		Từ \chu{i}  và \chu{ii}, ta suy ra
		$$ a+6=(5b+9)+6=5b+15 $$
		chia hết cho $ b $, thế nên $ 15 $ chia hết cho $ b $. Ta nhận được $b=1,b=3,b=5$ hoặc $b=15.$ \\
		Mặt khác, từ \chu{i} và \chu{iii}, suy ra $$ a+2017b=(5b+9)+2017b=5(404b+1)+2b+4 $$
		chia hết cho $ 5 $, dẫn tới $ 2b+4 $ chia hết cho $ 5 $. Thử trực tiếp các trường hợp $b=1,b=3,b=5,b=15,$ ta được $ b=3 $, và đồng thời, ta tìm ra $a=24$. Kết luận $(a,b)=(24,3)$ là cặp duy nhất thỏa yêu cầu.}
\end{gbtt}

\section{Phân tích tiêu chuẩn của một số nguyên tố}

\subsection{Lí thuyết}

Trong mục này, chúng ta sẽ làm quen với một số bài tập có sử dụng một bổ đề khá quen thuộc. Bổ đề được phát biểu như sau
\begin{quote}
    \it "Với số nguyên tố $p$ và các số nguyên dương $a,b$ thỏa mãn $ab=p,$ số nhỏ hơn trong $a$ và $b$ phải bằng $1.$"
\end{quote}
Theo đó, với một biểu thức cho trước có giá trị nguyên tố, ta có thể tìm điều kiện cho các nhân tử của chúng dựa vào bổ đề trên. Sau đây là các bài toán trong mục.

\subsection{Ví dụ minh họa}

\begin{bx}
Tìm tất cả các số nguyên $n$ sao cho $n^2-9n+20$ là một số nguyên tố.
\loigiai{
Ta nhận thấy rằng
$n^2-9n+20=|n-4||n-5|.$
Theo yêu cầu bài toán, do $n^2-9n+20$ là số nguyên tố nên hoặc $|n-4|=1,$ hoặc $|n-5|=1.$
\begin{enumerate}
    \item Nếu $|n-4|=1,$ ta tìm ra $n=3$ hoặc $n=5.$ Thử lại, ta thấy chỉ có $n=3$ thỏa mãn.
    \item Nếu $|n-5|=1,$ ta tìm ra $n=4$ hoặc $n=6.$ Thử lại, ta thấy chỉ có $n=6$ thỏa mãn.
\end{enumerate}
Như vậy, có tất cả hai giá trị của $n$ thỏa yêu cầu, đó là $n=3$ và $n=6.$}
\end{bx}

\begin{bx}
Tìm tất cả các số nguyên dương $n$ để $n^5+n^4+1$ là số nguyên tố.
\nguon{Chuyên Toán Sóc Trăng 2021}
\loigiai{
Ta nhận thấy rằng
\begin{align*}
    n^5+n^4+1
    &=\left(n^5-n^2\right)+\left(n^4-n\right)+\left(n^2+n+1\right)
    \\&=n^2(n-1)\left(n^2+n+1\right)+n(n-1)\left(n^2+n+1\right)+\left(n^2+n+1\right)
    \\&=\left(n^2+n+1\right)\left(n^2(n-1)+n(n-1)+1\right)\\&
=\left(n^2+n+1\right)\left(n^3-n+1\right).
\end{align*}
Theo yêu cầu bài toán, một trong hai số $n^2+n+1$ và $n^3-n+1$ phải bằng $1.$ Tuy nhiên, do $$n^2+n+1\ge 1+1+1=3$$ nên $n^3-n+1=1,$ và ta tìm ra $n=1.$ Thử trực tiếp, ta thấy $n=1$ thỏa mãn đề bài.}
\end{bx}

\begin{bx}\hfill
\begin{enumerate}[a,]
    \item Cho số nguyên dương $n.$ Chứng minh nếu $2^{n}-1$ là số nguyên tố thì $n$ cũng là số nguyên tố.
    \item Liệu rằng với số nguyên tố $p$ bất kì, $2^p-1$ có luôn là số nguyên tố không? Giải thích tại sao.
\end{enumerate}
\loigiai{
\begin{enumerate}[a,]
    \item Giả sử phản chứng rằng $n$ là hợp số. Ta đặt $n=mk,$ với $m,k$ là các số tự nhiên lớn hơn $1.$ Ta có
    $$2^n-1=\left(2^k\right)^m-1=\left(2^k-1\right)\left[\left(2^k\right)^{m-1}+\left(2^k\right)^{m-2}+\ldots+2^k+1\right].$$
    Vì $m,k$ là các số tự nhiên lớn hơn $1$ nên
    $$2^{k}-1>1,\quad \left(2^{k}\right)^{m-1}+\left(2^{k}\right)^{m-2}+\ldots+2^{k}+1>1.$$ 
    $2^n-1$ được phân tích thành hai thừa số lớn hơn $1,$ chứng tỏ đây là hợp số, mâu thuẫn với giả thiết. Như vậy, giả sử là sai, và ta có điều phải chứng minh.
    \item Câu trả lời là phủ định. Thật vậy, với số nguyên tố $p=11,$ ta có
    \[2^p-1=2^{11}-1=2047=23\cdot89.\]
\end{enumerate}
} 
\end{bx}
\subsection{Bài tập tự luyện}
\begin{btt}
Tìm tất cả các số tự nhiên $n$ sao cho $\dfrac{n^3 - 1}{9}$ là số nguyên tố.
\end{btt}
\begin{btt}
Tìm tất cả các số nguyên dương $m$ sao cho $$1+3^{20\left(m^2+m+1\right)}+9^{14\left(m^2+m+1\right)}$$ là số nguyên tố.
\nguon{Đề thi chọn đội tuyển Phổ thông Năng khiếu 2012 $-$ 2013.}
\end{btt}

\begin{btt}
Tìm các số nguyên dương $x$ và $y$ sao cho $x^4+4y^4$ là số nguyên tố.
\end{btt}

\begin{btt}
Tìm tất cả các số nguyên dương $x,y$ và số nguyên tố $p$ thỏa mãn \[\dfrac{1}{x}+\dfrac{1}{y}=\dfrac{1}{p}.\]
\end{btt}

\begin{btt}
Cho các số nguyên dương $a,b,c$ đôi một phân biệt thỏa mãn điều kiện $\dfrac{a}{c}=\dfrac{a^2+b^2}{c^2+b^2}$. Chứng minh $a^2+b^2+c^2$ không phải là số nguyên tố.
\end{btt} 

\begin{btt}
Tìm tất cả các số nguyên dương ${n}$ sao cho $\left[\dfrac{{n}^{3}+8 {n}^{2}+1}{3 {n}}\right]$ là một số nguyên tố, trong đó $[A]$ được kí hiệu là số nguyên lớn nhất không vượt quá $A.$ 
\end{btt}

\begin{btt}
Chứng minh rằng một số nguyên tố tùy ý có dạng $2^{2^n}+1$ (với $n$ nguyên dương) không thể biểu diễn dưới dạng hiệu các lũy thừa bậc năm của hai số tự nhiên.
\end{btt}

\begin{btt} \
\begin{enumerate}[a,]
    \item Cho số nguyên dương $a>1$ và số nguyên dương $n.$ Chứng minh rằng nếu $a^n+1$ là số nguyên tố thì $n$ là lũy thừa số mũ tự nhiên của $2.$
    \item Tìm tất cả các số nguyên dương $n$ sao cho cả hai số $n^n+1$ và $(2n)^{2n}+1$ là số nguyên tố.
\end{enumerate}
\end{btt}

\begin{btt}
Tìm tất cả các số nguyên dương \(n\) thỏa mãn \(4k^2+n\) là số nguyên tố, với mọi số nguyên \(k\) không âm nhỏ hơn \(n\).
\nguon{Tạp chí Kvant}
\end{btt}

\subsection{Hướng dẫn bài tập tự luyện}

\begin{gbtt}
Tìm tất cả các số tự nhiên $n$ sao cho $\dfrac{n^3 - 1}{9}$ là số nguyên tố.
\loigiai{
Xét các số dư của $n$ khi chia cho $3,$ ta được $n$ chia $3$ dư là $1.$ Đặt $n=3k+1,$ ta có
\begin{align*}
   \dfrac{n^3 - 1}{9}= \dfrac{(3k+1)^3 - 1}{9}&= \dfrac{27k^3 + 27k^2 + 9k}{9} \\&= 3k^3 + 3k^2 + k \\&= k(3k^2 + 3k + 1).
\end{align*}
Dựa vào so sánh $1\le k<3k^2+3k+1,$ ta suy ra $k=1.$ Lúc này $n = 4$ và $\dfrac{n^3 - 1}{9} = \dfrac{64-1}{9} = 7$ là số nguyên tố. Như vậy, $n = 4$ là giá trị duy nhất cần tìm.}
\end{gbtt}

\begin{gbtt}
Tìm tất cả các số nguyên dương $m$ sao cho $$1+3^{20\left(m^2+m+1\right)}+9^{14\left(m^2+m+1\right)}$$ là số nguyên tố.
\nguon{Đề thi chọn đội tuyển Phổ thông Năng khiếu 2012 $-$ 2013.}
\loigiai{
Đặt $n=3^{4\left(m^2+m+1\right)}.$ Ta có $1+3^{20\left(m^2+m+1\right)}+9^{14\left(m^2+m+1\right)}=n^7+n^5+1.$ Ta nhân thấy rằng
\begin{align*}
    n^7+n^5+1
    &=\left(n^7-n\right)+\left(n^5-n^2\right)+\left(n^2+n+1\right)
    \\&=n(n+1)\tron{n^2-n+1}(n-1)\tron{n^2+n+1}+n^2(n-1)\tron{n^2+n+1}+\left(n^2+n+1\right)
    \\&=\left(n^2+n+1\right)\tron{n(n+1)\tron{n^2-n+1}(n-1)+n^2(n-1)+1}
    \\&=\left(n^2+n+1\right)\left(n^5-n^4+n^3-n+1\right).
\end{align*}
Theo yêu cầu bài toán, một trong hai số $n^2+n+1$ và $n^5-n^4+n^3-n+1$ phải bằng $1.$ Ta xét hiệu
$$n^5-n^4+n^3-n+1-\tron{n^2+n+1}=n^5-n^4+n^3-n^2-n=n^4\tron{n-1}+n\tron{n^2-n-1}.$$
Hiệu kể trên lớn hơn $0$ nếu như $n\ge 2.$ Theo đó, trong trường hợp $n\ge 2,$ ta có
$$n^2+n+1=1\Leftrightarrow n(n+1)=0.$$
Trong trường hợp này, ta không tìm được $n$ nguyên dương. Ngược lại, nếu $n=1,$ ta có
$$3^{4\left(m^2+m+1\right)}=1.$$ 
Ta không tìm được $m$ nguyên dương từ đây.  }
\end{gbtt}

\begin{gbtt}
Tìm các số nguyên dương $x$ và $y$ sao cho $x^4+4y^4$ là số nguyên tố.
\loigiai {
Phân tích thành nhân tử, ta có $$x^4+4y^4 = \left(x^2+2y^2\right)^2 - (2xy)^2 = \left(x^2+2xy+2y^2\right) \left(x^2-2xy+2y^2\right).$$
Dựa vào nhận xét $0<x^2-2xy+2y^2<x^2+2xy+2y^2,$ từ việc $x^4+4y^4$ là số nguyên tố, ta suy ra $$x^2 - 2xy + 2y^2 = 1\Leftrightarrow (x-y)^2+y^2=1.$$
Do $x$ và $y$ là hai số nguyên dương nên $x=y=1$. Thử trực tiếp, ta được $x^4 + 4y^4 = 5$ là số nguyên tố.}
\end{gbtt}

\begin{gbtt}
Tìm tất cả các số nguyên dương $x,y$ và số nguyên tố $p$ thỏa mãn \[\dfrac{1}{x}+\dfrac{1}{y}=\dfrac{1}{p}.\]
\loigiai{
Với các số $x,y,p$ thỏa mãn đề bài, ta có
\begin{align*}
   \dfrac{1}{x}+\dfrac{1}{y}=\dfrac{1}{p}&\Rightarrow xy-px-py=0\\&\Rightarrow xy-px-py+p^2=p^2\\&
   \Rightarrow (x-p)(y-p)=p^2. 
\end{align*}
Không mất tính tổng quát, ta giả sử $x\le y,$ thế thì $x-p\le y-p.$ Ngoài ra, cả hai số $x$ và $y$ đều lớn hơn $p,$ nên là $0\le x-p\le y-p.$ Ta xét hai trường hợp sau đây.
\begin{enumerate}
    \item Nếu $x-p=y-p=p,$ ta tìm được $x=y=2p.$
    \item Nếu $x-p=1$ và $y-p=p^2,$ ta tìm được $x=p+1,y=p^2+p.$
\end{enumerate}
Kết luận, tất cả các bộ $(p,x,y)$ thỏa yêu cầu bài toán là
$$\left(p,2p,2p\right),\quad \left(p,p+1,p^2+p\right),\quad \left(p,p^2+p,p+1\right),$$
trong đó, $p$ là một số nguyên tố tùy ý.}
\end{gbtt}


\begin{gbtt}
Cho các số nguyên dương $a,b,c$ đôi một phân biệt thỏa mãn điều kiện $\dfrac{a}{c}=\dfrac{a^2+b^2}{c^2+b^2}$. Chứng minh $a^2+b^2+c^2$ không phải là số nguyên tố.
\loigiai{
Với các số $a,b,c$ thỏa mãn yêu cầu, ta có
$$\dfrac{{a}}{{c}}=\dfrac{{a}^{2}+{b}^{2}}{{c}^{2}+{b}^{2}} \Leftrightarrow {a}\left({c}^{2}+{b}^{2}\right)={c}\left({a}^{2}+{b}^{2}\right) \Leftrightarrow({a}-{c})\left({b}^{2}-{ac}\right)=0.$$
Do ${a} \neq {c}$ nên ta được ${b}^{2}-{ac}=0$, hay là ${b}^{2}={ac}$, và như vậy thì
\begin{align*}
    a^{2}+b^{2}+c^{2}&=a^{2}+a c+c^{2}\\&=a^{2}+2 a c+c^{2}-a c\\&=(a+c)^{2}-b^{2}\\&=(a-b+c)(a+b+c).
\end{align*}
Ta giả sử rằng $a^2+b^2+c^2$ là số nguyên tố. Theo đó, do $0<a-b+c<a+b+c,$ ta cần phải có $a-b+c= 1,$ thế nên
    \begin{align*}
        & \quad \:\:\: 4a^2+4b^2+4c^2-4a-4b-4c=0
        \\&\Leftrightarrow \left(4a^2-4a+1\right)+\left(4b^2-4b+1\right)+\left(4c^2-4c+1\right)=3
        \\&\Leftrightarrow (2a-1)^2+(2b-1)^2+(2c-1)^2=3.
    \end{align*}
Do $a,b,c$ là các số nguyên dương, ta suy ra $2a-1=2b-1=2c-1=1,$ nên là $a=c=1,$ trái giả thiết $a\ne c.$ Các mâu thuẫn chỉ ra chứng tỏ giả sử là sai. Bài toán được chứng minh.}
\end{gbtt} 

\begin{gbtt}
Tìm tất cả các số nguyên dương ${n}$ sao cho $\left[\dfrac{{n}^{3}+8 {n}^{2}+1}{3 {n}}\right]$ là một số nguyên tố, trong đó $[A]$ được kí hiệu là số nguyên lớn nhất không vượt quá $A.$ 
\loigiai{
Đặt $A=\dfrac{n^{3}+8 n^{2}+1}{3 n}=\dfrac{n^{2}}{3}+\dfrac{8 n}{3}+\dfrac{1}{3 n} .$ Ta xét các trường hợp sau.
\begin{enumerate}
    \item Nếu ${n}=3 {k}$ với ${k}$ là một số nguyên dương, ta có
    $$\left[A\right]=\left[3k^2+8k+\dfrac{1}{9k}\right]=3k^2+8k=k(3k+8).$$
    Do $1\le k<3k+8$ nên $[A]$ là số nguyên tố chỉ khi $k=1.$ Từ đây, ta tìm ra $n=3.$
    \item Nếu ${n}=3 {k}+1$ với ${k}$ là một số tự nhiên, ta có
    $$[A]=\left[3 k^{2}+10 k+3+\dfrac{1}{9 k+3}\right]=3k^2+10k+3=(k+3)(3k+1).$$
    Do $k+3\ge 2$ nên $[A]$ là số nguyên tố chỉ khi $3k+1=1.$ Từ đây, ta tìm ra $n=1.$
    \item Nếu ${n}=3 {k}+2$ với ${k}$ là một số nguyên dương, ta có
    $$[{A}]=\left[3 {k}^{2}+12 {k}+6+\dfrac{1}{9 {k}+3}+\dfrac{2}{3}\right]=3 {k}^{2}+12 {k}+6=3\left({k}^{2}+4 {k}+2\right).$$
    Cả hai số $3$ và $k^2+4k+2$ đều lớn hơn $1.$ Trong trường hợp này, $[A]$ là hợp số.
\end{enumerate}
Tổng kết lại, $n=1$ và $n=3$ là hai giá trị của $n$ thỏa yêu cầu bài toán.}
\end{gbtt}

\begin{gbtt}
Chứng minh rằng một số nguyên tố tùy ý có dạng $2^{2^n}+1$ (với $n$ nguyên dương) không thể biểu diễn dưới dạng hiệu các lũy thừa bậc năm của hai số tự nhiên.
\loigiai{
Ta giả sử phản chứng rằng tồn tại hai số tự nhiên $a,b$ thỏa mãn
$$2^{2^n}+1=b^5-a^5=(b-a)\tron{a^4+a^3b+a^2b^2+ab^3+b^4}.$$
Do $b-a\le b\le b^4\le a^4+a^3b+a^2b^2+ab^3+b^4$ và $2^{2^n}+1$ là số nguyên tố nên là
$$b-a=1,\quad a^4+a^3b+a^2b^2+ab^3+b^4=2^{2^n}+1.$$
Thế $b=a+1$ vào đẳng thức còn lại, ta được
\begin{align*}
    2^{2^n}+1
    &=a^4+a^3(a+1)+a^2(a+1)^2+a(a+1)^3+(a+1)^4
    \\&=5\tron{a^4+2a^3+2a^2+1}+1.
\end{align*}
Ta suy ra $2^{2^n}=5\tron{a^4+2a^3+2a^2+1},$ nhưng điều này vô lí vì $2^{2^n}$ không chia hết cho $5.$ \\
Giả sử là sai, và bài toán được chứng minh.}
\end{gbtt}


\begin{gbtt} \
\begin{enumerate}[a,]
    \item Cho số nguyên dương $a>1$ và số nguyên dương $n.$ Chứng minh rằng nếu $a^n+1$ là số nguyên tố thì $n$ là lũy thừa số mũ tự nhiên của $2.$
    \item Tìm tất cả các số nguyên dương $n$ sao cho cả hai số $n^n+1$ và $(2n)^{2n}+1$ là số nguyên tố.
\end{enumerate}
\loigiai{
\begin{enumerate}[a,]
    \item Giả sử phản chứng rằng $n$ không là lũy thừa số mũ nguyên dương của $2.$ Ta đặt
    $$n=2^kl,\text{trong đó }k\text{ là số tự nhiên, }l\text{ là số nguyên dương lẻ}.$$
    Phép đặt này cho ta
    $$a^n+1=a^{2^kl}+1=\tron{a^{2^k}+1}\tron{a^{2^k(l-1)}-a^{2^k(l-2)}+\ldots-a^{2^k}+1}.$$
    Số kể trên là hợp số, bởi vì $a^n+1>a^{2^k}+1>1.$ Giả sử phản chứng là sai. Chứng minh hoàn tất.
    \item Theo như câu a, $n$ là lũy thừa số mũ tự nhiên của $2.$ Đặt $n=2^k,$ trong đó $k$ là số tự nhiên. Khi đó
    \begin{align*}
        n^n+1&=\tron{2^k}^{2^k}+1=2^{k\cdot2^k}+1,\\
        (2n)^{2n}+1&=\tron{2^{k+1}}^{2^{k+1}}+1=2^{(k+1)2^{k+1}}+1.
    \end{align*}
    Nếu như $k=0,$ ta tìm ra $n=1.$ Ngược lại, nếu $k\ge 1,$ áp dụng kết quả câu a một lần nữa, $k\cdot2^k$ và $(k+1)2^{k+1}$ cũng phải là lũy thừa số mũ tự nhiên của $2,$ thế nên $k$ và $k+1$ có tính chất y hệt. Ta đặt
    $$k=2^a,\: k+1=2^b.$$
    Lấy hiệu theo vế hai phép đặt trên, ta có
    $$1=2^b-2^a=2^a\tron{2^{b-a}-1}.$$
    Bắt buộc, ta phải có $2^a=2^{b-a}-1=1.$ Ta tìm ra $a=0,b=1,$ và thế thì $k=1,n=2.$\\
    Kết luận, $n=1$ và $n=2$ là tất cả các số tự nhiên thỏa yêu cầu.
\end{enumerate}}
\end{gbtt}

\begin{gbtt}
Tìm tất cả các số nguyên dương \(n\) thỏa mãn \(4k^2+n\) là số nguyên tố, với mọi số nguyên \(k\) không âm nhỏ hơn \(n\).
\nguon{Tạp chí Kvant}
\loigiai{
Ta giả sử tồn tại số nguyên tố $n$ thỏa yêu cầu bài toán. \\
Cho $k=0,$ ta chỉ ra $n$ là số nguyên tố. Tới đây, ta xét các trường hợp sau.
\begin{enumerate}
    \item Nếu $n=2,$ thử trực tiếp, ta thấy thỏa mãn.
    \item Nếu $n=4m+1,$ cho $k=\dfrac{n-1}{4},$ ta được
    $$4k^2+n=\dfrac{(n-1)^2}{4}+n=\dfrac{(n+1)^2}{4}.$$
    Do $\dfrac{n+1}{2}$ là số tự nhiên, ta có $\left(\dfrac{n+1}{2}\right)^2$ là hợp số, mâu thuẫn.
    \item Nếu $n=4m-1,$ ta gọi $d$ là một ước lẻ nào đó của $m.$ Cho $k=\dfrac{d+1}{2},$ ta có
    $$4k^2+n=(d+1)^2+4m-1=d^2+2d+4m.$$
    Số kể trên chia hết cho $d,$ thế nên nó là số nguyên tố chỉ khi $d=1.$ Ước lẻ duy nhất của $m$ là $1,$ chứng tỏ $m$ là một lũy thừa của $2.$ Tới đây, ta xét tiếp các khả năng sau.
    \begin{itemize}
        \item \chu{Trường hợp 1. }Nếu $m=2^{2a}$ với $a$ là số tự nhiên, ta có
        $$n=4\cdot2^{2a}-1=2^{2a+2}-1=\left(2^{a+1}-1\right)\left(2^{a+1}+1\right).$$
        Do $n$ là số nguyên tố, bắt buộc $2^{a+1}-1=1.$ Ta tìm ra $a=0,m=1,$ và $n=3$ từ đây.
        \item \chu{Trường hợp 2. }Nếu $m=2^{4a+1}$ với $a$ là số tự nhiên, ta có $n=2^{4a+3}-1.$\\     
        Ta nhận thấy với $a=0,$ ta có $n=7$ thỏa mãn. Nếu $a\ge 0,$ cho $k=2^{2a-1},$ ta được
        $$4k^2+n=2^{4a}+2^{4a+3}-1=9\cdot2^{4a}-1=\left(3\cdot2^{2a}-1\right)\left(3\cdot2^{2a}+1\right).$$
        Do $4k^2+n$ là số nguyên tố, bắt buộc $3\cdot 2^{2a}-1=1.$ Ta không tìm được $a$ từ đây.
        \item \chu{Trường hợp 3. }Nếu $m=2^{4a+3}$ với $a$ là số tự nhiên, ta có $n=2^{4a+5}-1.$\\
        Cho $k=2^{2a},$ ta được
        $$4k^2+n=4\cdot2^{4a}+2^{4a+5}-1=36\cdot2^{4a}-1=\left(6\cdot2^{2a}-1\right)\left(6\cdot2^{2a}+1\right).$$  Do $4k^2+n$ là số nguyên tố, bắt buộc $6\cdot 2^{2a}-1=1.$ Ta không tìm được $a$ từ đây.
    \end{itemize}
\end{enumerate}
Kết luận, các số nguyên dương $n$ thỏa yêu cầu bài toán là $n=2,n=3$ và $n=7.$}
\end{gbtt}

\section{Ứng dụng của đồng dư thức}

\subsection{Lí thuyết}

Trong mục này, chúng ta sẽ tìm hiểu về các ứng dụng về đồng dư thức đối với các bài toán chứa yếu tố số nguyên tố. Tác giả nhắc lại các kiến thức đã học ở \chu{chương I}.

\begin{light}
Với mọi số nguyên dương $n,$ ta luôn có
\begin{multicols}{2}
\begin{enumerate}
    \item $n^2\equiv 0,1\pmod{3}.$
    \item $n^2\equiv 0,1\pmod{4}.$
    \item $n^2\equiv 0,1,4\pmod{5}.$   
    \item $n^2\equiv 0,1,2,4\pmod{7}.$    
    \item $n^2\equiv 0,1,4\pmod{8}.$    
    \item $n^3\equiv 0,1,-1\pmod{7}.$ 
    \item $n^3\equiv 0,1,-1\pmod{9}.$    
    \item $n^4\equiv 0,1\pmod{5}.$    
    \item $n^4\equiv 0,1\pmod{16}.$    
    \item $n^5\equiv 0,1\pmod{11}.$    
\end{enumerate}
\end{multicols}    
\end{light}

\subsection{Ví dụ minh họa}

\begin{bx}
Cho số nguyên tố $p>5.$ Chứng minh rằng
\begin{multicols}{2}
\begin{enumerate}[a,]
    \item $p^2-1$ chia hết cho $24.$
    \item $p^4-1$ chia hết cho $240.$
\end{enumerate}
\end{multicols}
\loigiai{
\begin{enumerate}[a,]
    \item Một số chính phương chỉ có thể đồng dư $0$ hoặc $1$ theo modulo $3,$ vậy nên
    $$p^2\equiv 0,1 \pmod{3}.$$
    Tuy nhiên, do $p$ là số nguyên tố lớn hơn $3$ nên $3\nmid p^2.$ Ta suy ra
    \[p^2\equiv 1 \pmod{3}\Rightarrow 3\mid \left(p^2-1\right).\tag{1}\label{p211}\]
    Hơn nữa, một số chính phương chỉ có thể đồng dư $0,1,4$ theo modulo $8,$ vậy nên
    $$p^2\equiv 0,1,4 \pmod{8}.$$
    Tuy nhiên, do $p$ là số nguyên tố lẻ nên $4\nmid p^2.$ Ta suy ra
    \[p^2\equiv 1 \pmod{8}\Rightarrow 8\mid \left(p^2-1\right).\tag{2}\label{p212}\]    
    Do $(3,8)=1$ nên kết hợp $(\ref{p211})$ và $(\ref{p212}),$ ta được $24\mid \left(p^2-1\right).$ Bài toán được chứng minh.
    \item Một lũy thừa mũ $4$ chỉ có thể đồng dư với $0$ hoặc $1$ theo modulo $3,5,16.$\\
    Bằng cách lập luận tương tự câu $a,$ ta chỉ ra $p^4-1$ chia hết cho $3\cdot 5\cdot 16=240.$
\end{enumerate}}
\begin{luuy}
Từ nay về sau, các kết quả tương tự như trong bài toán trên sẽ được dùng trực tiếp mà không thông qua chứng minh. Chẳng hạn
\begin{enumerate}
    \item Nếu $p$ là số nguyên tố và $p>3$ thì $p^2\equiv 1\pmod{3}.$
    \item Nếu $p$ là số nguyên tố và $p>5$ thì $p^2\equiv 1\pmod{5}$ và $p^2\equiv 4\pmod{5}.$ 
\end{enumerate}
\end{luuy}
\end{bx}

\begin{bx}
Tìm các số nguyên tố $p,q,r$ liên tiếp sao cho $p^2+q^2+r^2$ cũng là số nguyên tố.
\loigiai{Không mất tính tổng quát, ta giả sử $p> q> r.$ Trong bài toán này, ta xét các trường hợp sau.
\begin{enumerate}
    \item Với $r>3,$ cả $p,q,r$ đều không chia hết cho $3,$ thế nên
    $$p^2+q^2+r^2\equiv 1+1+1\equiv 3 \pmod{3}.$$
    Do giả thiết $p^2+q^2+r^2$ là số nguyên tố, ta suy ra $p^2+q^2+r^2=3,$ vô lí.
    \item Với $r=3,$ ta có $q=5$ và $p=7.$ Ta được $p^2+q^2+r^2=83$ là số nguyên tố.
    \item Với $r=2,$ ta có $q=3$ và $p=5.$ Ta được $p^2+q^2+r^2=38$ không là số nguyên tố.
\end{enumerate}
Như vậy, có $6$ bộ $(p,q,r)$ thỏa mãn đề bài là $(3,5,7)$ và các hoán vị.}
\end{bx}

\begin{bx}
Tìm tất cả các số nguyên tố ${p}$ sao cho $2^p+p^2$ cũng là số nguyên tố.
\loigiai{Trong bài toán này, ta xét các trường hợp sau.
\begin{enumerate}
    \item Nếu $p=2$, ta có $2^p+p^2=2^{2}+2^{2}=8$ là hợp số.
    \item Nếu $p=3$, ta có $2^{{p}}+{p}^{2}=2^{3}+3^{2}=17$ là số nguyên tố.
    \item Nếu $p>3$, ta nhận thấy $p$ không chia hết cho $3$ và $2.$ Ta đặt $p=2k+1,$ khi đó
    $$2^p+p^2=2^{2k+1}+p^2=2\cdot 4^k+p^2\equiv 2+1\equiv 0\pmod{3}.$$
    Lập luận trên cho ta biết, $2^p+p^2$ là hợp số, mâu thuẫn.
\end{enumerate}
Kết luận, $p=3$ là số nguyên tố duy nhất thỏa yêu cầu bài toán.}
\end{bx} 

\begin{bx} \label{chedejbmo1}
Chứng minh rằng với mọi số nguyên tố $p$ thì $7^p-4^p$ không thể là lũy thừa với số mũ lớn hơn $1$ của một số nguyên dương.
\nguon{Junior Balkan Mathematical Olympiad Shortlist}
\loigiai{
Ta giả sử rằng $7^p-4^p$ là một lũy thừa với số mũ lớn hơn $1.$ Do $7^p-4^p$ chia hết cho $3,$ ta có thể đặt $$7^p-4^p=(3m)^x,$$ 
trong đó $x$ là số tự nhiên lớn hơn $1$ và $m$ nguyên dương. Ta xét các trường hợp sau.
\begin{enumerate}
    \item Nếu $p=3,$ ta có $7^p-4^p=7^3-4^3=279$ không là lũy thừa của $3.$
    \item Nếu $p$ có dạng $3k+1,$ ta nhận thấy
        \begin{align*}
        7^p-4^p=7^{3k+1}-4^{3k+1}&=7\cdot 343^k-4\cdot 64^k\equiv 7-4\equiv 3\pmod{9}.
        \end{align*}
        Ta suy ra $(3m)^x\equiv 3\pmod{9}$ từ đây, mâu thuẫn.
    \item Nếu $p$ có dạng $3k+2,$ ta nhận thấy
        \begin{align*}
        7^p-4^p=7^{3k+2}-4^{3k+2}&=49\cdot 343^k-16\cdot 64^k\equiv 49-16\equiv 6\pmod{9}.
        \end{align*}
        Ta suy ra $(3m)^x\equiv 6\pmod{9}$ từ đây, mâu thuẫn.
        \end{enumerate}
Giả sử phản chứng là sai. Bài toán được chứng minh.}
\end{bx}

\begin{bx}
Xét dãy số nguyên tố $p_{1}, p_{2}, p_{3}, \ldots, p_{n}$ thỏa mãn $p_{1}=2$ và với mọi $n \geq 1, p_{n+1}$ là ước nguyên tố lớn nhất của $p_{1} p_{2} \cdots p_{n}+1$ với mọi $n\ge 2.$ Chứng minh rằng $p_n\ne 5$ với mọi số nguyên dương $n.$
\nguon{Regional Mathematical Olympiad 2014}
\loigiai{
Thử trực tiếp, ta tính toán được $p_{1}=2, p_{2}=3, p_{3}=7.$  \\Ta sẽ đi chứng minh $p_n\ne 2$ và $p_n\ne 3,$ với mọi $n\ge 3.$ Thật vậy, do $p_n$ là ước của 
    $$p_{1} p_{2} p_3p_4\cdots p_{n}+1=2\cdot3\cdot p_3p_4\cdots p_n+1$$
    nên $(p_n,2)=(p_n,3)=1,$ và lại do $p_n$ là số nguyên tố nên chúng khác $2$ và $3.$\\ Tiếp theo, ta giả sử tồn tại số nguyên tố $p_n=5.$ Khi phân tích số
    $$p_{1} p_{2} p_3p_4\cdots p_{n-1}+1$$ ra thừa số nguyên tố, ta không nhận được các thừa số $2$ và $3,$ đồng thời $5$ là thừa số nguyên tố lớn nhất của phân tích này. Vậy nên, $p_1p_2p_3p_4\cdots p_{n-1}+1$ phải là lũy thừa của $5.$ Ta đặt
    $$p_1p_2p_3\cdots p_{n-1}+1=5^r.$$
    Tích $p_1p_2p_3\cdots p_{n-1}$ gồm một thừa số $2$ và các thừa số còn lại lẻ nên tích này chia $4$ dư $2.$ Chính vì lẽ đó
    $$5^r\equiv 2+1\equiv 3\pmod{4}.$$
    Đây là một điều không thể này xảy ra, bởi vì $5^r\equiv 1\pmod{4}.$ 
\\
Như vậy, giả sử phản chứng là sai. Bài toán được chứng minh.}
\end{bx}

\subsection{Bài tập tự luyện}

\begin{btt}
Tìm tất cả các số nguyên tố $p$ sao cho cả $4p^2+1$ và $6p^2+1$ cũng là các số nguyên tố.
\end{btt}

\begin{btt}
Tìm các số nguyên tố $p,q$ thỏa mãn $p+q, p+q^2, p+q^3,p+q^4$ đều là số nguyên tố.
\end{btt}

\begin{btt}
Tìm số nguyên dương $n$ nhỏ nhất thỏa mãn $n$ là ước của mọi số nguyên dương $p^6-1$ với $p$ là số nguyên tố lớn hơn $7.$
\nguon{Junior Balkan Mathematical Olympiad 2016}
\end{btt}

\begin{btt}
Tìm số nguyên tố $p$ sao cho $p^2+59$ có đúng $6$ ước số nguyên dương.
\nguon{Chuyên Toán Thái Nguyên 2019}
\end{btt}

\begin{btt}
Tìm số nguyên tố $p$ sao cho $p^4+29$ có đúng $8$ ước số nguyên dương.
\end{btt}

\begin{btt}
Tìm các số nguyên tố $p,q,r,s$ phân biệt sao cho \[p^3+q^3+r^3+s^3=1709.\]  
\end{btt}

\begin{btt}
Tìm tất cả các số nguyên tố $p_1,p_2,\ldots,p_8$ thỏa mãn điều kiện
\[p^2_1+p^2_2+\ldots +p^2_7=p^2_8.\]
\end{btt}

\begin{btt}
Tìm tất cả các bộ $7$ số nguyên tố sao cho tích của chúng bằng tổng các lũy thừa bậc sáu của chúng.

\end{btt}

\begin{btt}
Tìm tất cả các cặp số nguyên tố $(p,q)$ sao cho $p^2+15pq+q^2$ là
\begin{enumerate}[a,]
	\item Một lũy thừa số mũ nguyên dương của $17.$
	\item Bình phương một số tự nhiên.
\end{enumerate}
\nguon{Tạp chí Pi tháng 10 năm 2017}
\end{btt}

\begin{btt}
Tìm số nguyên tố $p$ nhỏ nhất sao cho $p$ viết được thành $10$ tổng có dạng
$$p=x_1^2+y_1^2=x_{2}^{2}+2 y_{2}^{2}=x_{3}^{2}+3 y_{3}^{2}=\ldots=x_{10}^{2}+10 y_{10}^{2}.$$ 
Trong đó, $x_1,x_2, \ldots , x_{10}$ và $y_1, y_2, \ldots y_{10}$ là các số nguyên dương. 
\nguon{Tạp chí toán học và Tuổi trẻ số 330}
\end{btt}

\begin{btt}
Tìm các số nguyên tố $p,q,r$ thỏa mãn $p^q+q^p=r.$ 
\end{btt}

\begin{btt}
Cho $p>3$ là số nguyên tố, chứng minh rằng với mọi số tự nhiên $n$ thì ba số $p+2,2^{n}+p$ và $2^{n}+p+2$ không thể đều là số nguyên tố.
\nguon{Chọn học sinh giỏi quốc gia Quảng Trị 2017 $-$ 2018}
\end{btt}

\begin{btt}
Tìm tất cả các số tự nhiên $n$ để $A=2^{2^{2n+1}}+3$ là số nguyên tố.
\end{btt}

\begin{btt}
Tìm tất cả các số tự nhiên $m,n$ thỏa mãn $P=3^{3m^2+6n-61}+4$ là số nguyên tố.
\nguon{Chọn học sinh giỏi thành phố Hà Tĩnh 2016}
\end{btt}

\begin{btt}
Tồn tại hay không số tự nhiên \(n\) thỏa mãn \(8^n+47\) là số nguyên tố?
\nguon{Junior Balkan Mathematical Olympiad Shorlist 2020}
\end{btt}

\begin{btt}
Tìm tất cả các số nguyên tố $p$ sao cho $5^{p}+12^{p}$ là lũy thừa với số mũ lớn hơn $1$ của một số nguyên dương.
\end{btt}

\begin{btt}
Xét dãy số nguyên tố $p_1,p_2,p_3,\ldots, p_n$ thỏa mãn $p_1=5$ và với mọi $n \geq 1, p_{n+1}$ là ước nguyên tố lớn nhất của $p_{1} p_2\cdots p_n+1$ với mọi $n\ge 2.$ Chứng minh rằng $p_n\ne 7$ với mọi số nguyên dương $n.$

\end{btt}

\begin{btt}
Tìm tất cả các số nguyên dương \(n\) sao cho tồn tại số nguyên tố \(p\) thỏa mãn  $p^n-(p-1)^n$ là một lũy thừa của \(3\).
\nguon{Junior Balkan Mathematical Olympiad Shortlist 2017}
\end{btt}

\begin{btt}
Tìm tất cả các số chính phương $n$ sao cho với mọi ước nguyên dương $a\ge 15$ của $n$ thì $a+15$ là lũy thừa của một số nguyên tố.
\nguon{Junior Balkan Mathematical Olympiad Shortlists 2019}
\end{btt}

\subsection{Hướng dẫn bài tập tự luyện}

\begin{gbtt}
Tìm tất cả các số nguyên tố $p$ sao cho cả $4p^2+1$ và $6p^2+1$ cũng là các số nguyên tố.
\loigiai{Ta đã biết rằng nếu $p$ là số nguyên tố lớn hơn $5$ thì $p^2\equiv \pm 1 \pmod{5}.$
\begin{enumerate}
    \item Với $p^2\equiv 1\pmod{5},$ ta có $4p^2+1$ chia hết cho $5$ và lớn hơn $5,$ thế nên đây là hợp số, mâu thuẫn.
    \item Với $p^2\equiv -1\pmod{5},$ ta có $6p^2+1$ chia hết cho $5$ và lớn hơn $5,$ thế nên đây là hợp số, mâu thuẫn.  
    \item Với $p=5,$ thử trực tiếp, ta thấy thỏa mãn.
\end{enumerate}
Như vậy, số nguyên tố cần tìm là $p=5.$}
\end{gbtt} 

\begin{gbtt}
Tìm các số nguyên tố $p,q$ thỏa mãn
$p+q,p+q^2, p+q^3,p+q^{4}$ đều là số nguyên tố.
\loigiai{
Nếu $p$ và $q$ cùng lẻ, ta có $p+q$ là hợp số chẵn lớn hơn $2,$ vô lí. Do đó, một trong hai số $p,q$ phải bằng $2.$ Ta xét các trường hợp sau đây.
\begin{enumerate}
    \item Nếu $p=2$ thì từ $q^2+2$ là số nguyên tố, ta bắt buộc phải có 
    $$q^2+2\equiv 1,2\pmod{3}\Rightarrow q^2\equiv -1,0\pmod{3}.$$
    Do $q^2$ không thể đồng dư $-1$ theo modulo $3$ nên bắt buộc $q^2$ chia hết cho $3,$ hay là $q=3.$ \\Thử lại, ta thấy thỏa mãn.
    \item Nếu $q=2$ thì $p+2,p+4,p+8,p+16$ là số nguyên tố. Lần lượt xét $$p=3k,\quad p=3k+1,\quad p=3k+2,$$ ta chỉ ra chỉ có $p=3k$ thỏa mãn, nhưng do $p$ nguyên tố nên $p=3.$ Thử lại, ta thấy thỏa mãn.
\end{enumerate}
Kết luận, có hai cặp số $(p,q)$ thỏa yêu cầu là $(2,3)$ và $(3,2).$}
\end{gbtt}

\begin{gbtt}
Tìm số nguyên dương $n$ nhỏ nhất thỏa mãn $n$ là ước của mọi số nguyên dương $p^6-1$ với $p$ là số nguyên tố lớn hơn $7.$
\nguon{Junior Balkan Mathematical Olympiad 2016}
\loigiai{
Bằng kiểm tra trực tiếp, ta chỉ ra
$\left(11^6-1,13^6-1\right)=504.$ Lập luận trên cho ta biết $n$ là ước của $504.$ Ta sẽ chứng minh rằng $n=504,$ thông qua việc chứng minh $p^6-1$ chia hết cho $504$ với mọi số nguyên tố $p>7.$ Xét phân tích $$p^6-1=\left(p^3+1\right)\left(p^3-1\right).$$
Do $p>7$ nên $p$ không chia hết cho $7$ và $9.$ Kết hợp với lí thuyết đã học, ta có.
$$\heva{&p\equiv -1,1 \pmod{7} \\ &p\equiv -1,1\pmod{9}.}$$
Ta được $p^6-1$ là bội của $7$ và $9.$ Ta đã biết, $p^6-1$ chia hết cho $p^2-1$ và $p^2-1$ chia hết cho $24,$ vậy nên $p^6-1$ chia hết cho bội chung nhỏ nhất của $7,9$ và $24,$ tức là $p^6-1$ chia hết cho $504.$ \\
Kết quả của bài toán là $n=504.$}
\end{gbtt}

\begin{gbtt}
Tìm số nguyên tố $p$ sao cho $p^2+59$ có đúng $6$ ước số nguyên dương.
\nguon{Chuyên Toán Thái Nguyên 2019}
\loigiai{
Trong bài toán này, ta xét ba trường hợp sau đây.
\begin{enumerate}
    \item Với $p=2$, ta thấy $p^{2}+11=15$ có đúng $4$ ước số nguyên dương.
    \item Với $p=3$, ta thấy $p^{2}+11=20$ có đúng $6$ ước số.
    \item Với $p>3$, dựa vào biến đổi $p^2+59=p^2-1+60,$ ta chỉ ra $p^2+59$ chia hết cho $12.$ \\
    Theo đó, số này có $7$ ước là $1,2,3,4,6,12$ và chính nó, mâu thuẫn.
\end{enumerate}
Tổng kết lại, $p=3$ là giá trị duy nhất thỏa yêu cầu.}
\end{gbtt}

\begin{gbtt}
Tìm số nguyên tố $p$ sao cho $p^4+29$ có đúng $8$ ước số nguyên dương.
\loigiai{
Trong bài toán này, ta xét các trường hợp sau đây.
\begin{enumerate}
    \item Với $p\le 5,$ thử trực tiếp, ta thấy có $p=3$ và $p=5$ thỏa mãn.
    \item Với $p>5,$ theo như lí thuyết đã học, ta nhận xét được
    $$p^4+29\equiv 1^2+29\equiv 0\pmod{30}.$$
    Số $p^4+29$ lúc này lớn hơn $30$ và chia hết cho $30,$ thế nên số ước của nó phải lớn hơn số ước của $30$ (là $8$ ước). Trường hợp này không xảy ra.
\end{enumerate}
Tổng kết lại, $p=3$ và $p=5$ là các số nguyên tố cần tìm.}
\end{gbtt}

\begin{gbtt}
Tìm các số nguyên tố $p,q,r,s$ phân biệt sao cho \[p^3+q^3+r^3+s^3=1709.\]  
\loigiai{
Không mất tính tổng quát, ta giả sử $p>q>r>s.$ Ta sẽ lần lượt đi tìm $s,r,q,p.$
\begin{enumerate}[\color{tuancolor}\bf\sffamily Bước 1.]
    \item Ta chứng minh $s=2.$ \\Nếu $s>2,$ cả $p,q,r,s$ đều là số lẻ, thế nên
    $$1709=p^3+q^3+r^3+s^3 \equiv 1+1+1+1 \equiv 0 \pmod{2}.$$
    Điều này là vô lí. Mâu thuẫn này chứng tỏ $s=2.$
    \item Ta chứng minh $r=3.$ \\Thay $s=2$ vào phương trình ban đầu, ta được
    \[ p^3+q^3+r^3=1701.\tag{*}\label{1701snt}\]
    Nếu $r>3,$ cả $p,q,r$ đều không chia hết cho $3.$ Theo kiến thức đã biết, lúc này $p^3,q^3,r^3$ sẽ chỉ có thể đồng dư với $1$ hoặc $-1$ theo modulo $9.$ Như vậy
    $$1701=p^3+q^3+r^3 \equiv \pm 1 \pm 1 \pm 1  \equiv -3,-1,1,3 \pmod{9}.$$
    Do $1709$ chia hết cho $9,$ điều trên là vô lí. Mâu thuẫn này chứng tỏ $r=3.$  
    \item Ta chứng minh $q=7.$ \\Thay $r=3$ vào phương trình (\ref{1701snt}), ta được
    $$p^3+q^3=1674.$$
    Với $q=5,$ ta không tìm được $p.$ Còn với $q>7,$  cả $p$ và $q$ đều không chia hết cho $7.$ Theo kiến thức đã biết, lúc này $p^3,q^3$ sẽ chỉ có thể đồng dư với $1$ hoặc $-1$ theo modulo $7.$ Như vậy
    $$1674=p^3+q^3 \equiv \pm 1 \pm 1 \pm  \equiv -2,0,2 \pmod{7}.$$ 
    Do $1674$ khi chia cho $7$ được dư là $1,$ điều trên là vô lí. Mâu thuẫn này chứng tỏ $q=7.$ \\
    Tiếp tục thay ngược lại, ta tìm được $p=11.$    
\end{enumerate}
Như vậy, các bộ số $(p,q,r,s)$ thỏa mãn đề bài là $(2,3,7,11)$ kèm theo các hoán vị.}
\end{gbtt}

\begin{gbtt}
Tìm tất cả các số nguyên tố $p_1,p_2,\ldots,p_8$ thỏa mãn điều kiện
\[p^2_1+p^2_2+\ldots +p^2_7=p^2_8.\]
\loigiai{Ta đã biết, với $p$ là số nguyên tố lẻ, $p^2\equiv 1\pmod{8}.$ Từ giả thiết, ta suy ra $p_8>2,$ do đó nó là số nguyên tố lẻ. Ta gọi $a$ là số các số lẻ ở vế trái. Vế trái lúc này có đúng $7-a$ số $2,$ thế nên
$$a+4(7-a)\equiv 1 \pmod{8}.$$
Một cách tương đương, ta có $27-3a$ chia hết cho $8,$ nhưng vì $0\le a\le 7$ nên $a=1.$\\
Vế trái có đúng một số lẻ, còn $6$ số bằng $2.$ Giả sử $p_7$ lẻ, lúc này
\begin{align*}
    6\cdot 2^2+p_7^2&=p_8^2\\ \left(p_8-p_7\right)\left(p_8+p_7\right)&=24.
\end{align*}
Do $p_8-p_7+p_8+p_7$ là số chẵn, $p_8-p_7$ và $p_8+p_7$ cùng tính chẵn lẻ.
\begin{enumerate}
    \item Với $p_8-p_7=2$ và $p_8+p_7=12,$ ta được $p_8=7$ và $p_7=5.$
    \item Với $p_8-p_7=4$ và $p_8+p_7=6,$ ta được $p_8=5$ và $p_7=1,$ mâu thuẫn.    
\end{enumerate}
Tổng kết lại, tất cả các bộ nguyên tố $\left(p_1,p_2,\ldots,p_8\right)$ cần tìm là hoán vị $7$ phần tử đầu của bộ $$(2,2,2,2,2,2,5,7).$$}
\end{gbtt}

\begin{gbtt}
Tìm tất cả các bộ $7$ số nguyên tố sao cho tích của chúng bằng tổng các lũy thừa bậc sáu của chúng.

\loigiai{
Ta đã biết, với $p$ là một số nguyên tố khác $7,$ ta có $p^3\equiv -1,1\pmod{7}$, thế nên $p^6\equiv 1\pmod{7}.$\\
Gọi các số nguyên tố thoả yêu cầu là $p_1,p_2,\ldots,p_7.$ Ta có
$$p_1^6+p_2^6+\ldots+p_7^6=p_1p_2\cdots p_7.$$
Gọi $a$ số các số $7$ ở vế trái. Ta xét các trường hợp sau.
\begin{enumerate}
    \item Nếu $a=0$ thì $VT\equiv 1+1+\ldots+1\equiv 7\pmod{7},$ còn vế phải không chia hết cho $7,$ mâu thuẫn.
    \item Nếu $a=7$ thì $p_1=p_2=\ldots=p_6=7.$ Thử lại, ta thấy thỏa mãn.
    \item Nếu $1\le a\le 6$ thì vế phải chia hết cho $7$ và
    $$VT\equiv 7-a\pmod{7}.$$
    Bắt buộc, $7-a\equiv 7\pmod{7},$ vô lí do $1\le a\le 6.$
\end{enumerate}
Kết luận, bộ bảy số nguyên tố duy nhất thỏa mãn là bộ bảy số $7.$}
\end{gbtt}

\begin{gbtt}
Tìm tất cả các cặp số nguyên tố $(p,q)$ sao cho $p^2+15pq+q^2$ là
\begin{enumerate}[a,]
	\item Một lũy thừa số mũ nguyên dương của $17.$
	\item Bình phương một số tự nhiên.
\end{enumerate}
\nguon{Tạp chí Pi tháng 10 năm 2017}
\loigiai
{\begin{enumerate}[a,]
	\item Giả sử tồn tại hai số nguyên tố $p,q$ và số nguyên dương $z\ge 2$ thỏa mãn
        $$(p-q)^2+17pq=17^z.$$
    Từ giả sử trên, ta có
    $$17\mid(p-q)\Rightarrow 17^2\mid (p-q)^2\Rightarrow 17^2\mid 17pq \Rightarrow 17\mid pq\Rightarrow\hoac{17\mid p\\ 17\mid q}\Rightarrow \hoac{p&=17\\ q&=17.}$$
    Không mất tổng quát, ta giả sử $q=17.$ Từ $p-q$ chia hết cho $17$ và $q=17,$ ta lại suy ra được $p$ chia hết cho $17,$ và bắt buộc $p=17.$ Bằng cách thế trở lại, ta kết luận $(p,q)=(17,17)$ là cặp số nguyên tố duy nhất thỏa yêu cầu.
	\item Giả sử $(p,q)$ là cặp số nguyên tố sao cho tồn tại số nguyên dương $r$ thỏa mãn
		$$p^2+15pq+q^2=r^2$$
	là một số chính phương. Trong trường hợp $p,q$ đều khác $3,$ ta có $p^2\equiv q^2 \equiv 1 \pmod{3},$ và vì thế
		\[p^2+15pq+q^2\equiv p^2+q^2\equiv 2\pmod{3}.\]
	Không có bình phương số tự nhiên nào đồng dư $2$ theo modulo $3$, vậy nên trường hợp giả định trên không thể xảy ra, tức là một trong hai số $p,q$ phải bằng $3.$ Do vai trò của $p$ và $q$ tương đương nhau nên không mất tổng quát, ta giả sử $p=3$. Thế trở lại, ta được
		\begin{align*} 
		q^2+45q+9=r^2&\Leftrightarrow 4q^2+180q+36=4r^2\\
			& \Leftrightarrow(2q+45)^2-1989=4r^2\\
			&\Leftrightarrow (2q-2r+45)(2q+2r+45)=1989.
		\end{align*}
		Với việc $1\le 2q-2r+45<2q+2r+45,$ ta xét bảng giá trị dưới đây
		\begin{center}
		\begin{tabular}{c|c|c|c|c|c|c}
			 $2q-2r+45$ & $1$ & $3$ & $9$ & $13$ & $17$ & $39$  \\
			     \hline
			 $2q+2r+45$ & $1989$ & $663$ & $221$ & $153$ & $117$ & $51$ \\
			     \hline
			 $q$ & $475$ & $144$ & $35$ & $19$ & $11$ & $0$
		\end{tabular}
		\end{center}
	    Tổng kết lại, tất cả các cặp số nguyên tố thỏa mãn yêu cầu của đề bài là \[(3,11),(11,3),(3,19),(19,3).\]
\end{enumerate}}
\end{gbtt}

\begin{gbtt}
Tìm số nguyên tố $p$ nhỏ nhất sao cho $p$ viết được thành $10$ tổng có dạng
$$p=x_1^2+y_1^2=x_{2}^{2}+2 y_{2}^{2}=x_{3}^{2}+3 y_{3}^{2}=\ldots=x_{10}^{2}+10 y_{10}^{2}.$$ 
Trong đó, $x_1,x_2, \ldots , x_{10}$ và $y_1, y_2, \ldots y_{10}$ là các số nguyên dương. 
\nguon{Tạp chí toán học và Tuổi trẻ số 330}
\loigiai{
Rõ ràng, ta phải có $p>10.$ Ta sẽ xét số dư của $p$ khi chia cho một vài số nguyên tố.
\begin{enumerate} 
    \item[i,] Từ ${p}={x}_{3}^{2}+3 {y}_{3}^{2}$, ta suy ra $p\equiv x^2_3\equiv 0,1 \pmod{3},$ nhưng do $p$ nguyên tố nên $p\equiv 1\pmod{3}.$
    \item[ii,] Từ $p=x_5^2+5y_5^2,$ bằng lập luận tương tự, ta suy ra $p\equiv 1,4 \pmod{5}.$    
    \item[iii,] Từ ${p}={x}_{7}^{2}+7 {y}_{7}^{2}$, bằng lập luận tương tự, ta suy ra $p\equiv 1,2,4 \pmod{7}.$    
    \item[iv,] Từ ${p}={x}_{8}^{2}+8 {y}_{8}^{2}$, bằng lập luận tương tự, ta suy ra $p\equiv 1 \pmod{8}.$
\end{enumerate}
Tới đây, ta chia bài toán thành các trường hợp sau.
\begin{enumerate}
    \item Nếu $p-1$ chia hết cho cả $3,5$ và $8,$ các số nguyên tố $p$ sẽ có dạng $120k+1.$ Ta sẽ lần lượt thử từng trường hợp của $k$ thỏa mãn $p$ là số nguyên tố.
    \begin{itemize}
        \item Với $k=2,$ ta có $p=241$ nguyên tố. Lúc này $p$ chia $7$ dư $3,$ mâu thuẫn.
        \item Với $k=5,$ ta có $p=601$ nguyên tố. Lúc này $p$ chia $7$ dư $6,$ mâu thuẫn.                
        \item Với $k=10,$ ta có $p=1201$ nguyên tố. Nhờ biểu diễn
        \begin{align*}
            1201
            &=25^2+24^2
            =7^2+2\cdot 24^2
            =1^2+3\cdot 20^2
            =25^2+4\cdot 12^2
            =34^2+5\cdot 3^2
            \\&=5^2+6\cdot 14^2
            =33^2+7\cdot 4^2
            =7^2+8\cdot 12^2
            =25^2+9\cdot 8^2
            =29^2+10\cdot 6^2.
        \end{align*}
        ta chỉ ra $p=1201$ là số nhỏ nhất thỏa trường hợp này.
        \end{itemize}    
    \item Nếu $p-1$ chia hết cho cả $3$ và $8$ còn $p-4$ chia hết cho $5,$ do $p-49$ chia hết cho $120$ nên các số nguyên tố $p$ sẽ có dạng $120k+49.$ Bằng cách thử tương tự, ta nhận thấy $p=1009$ là số nguyên tố nhỏ nhất thỏa trường hợp này. Cụ thể:
        \begin{align*}
            1009
            &=15^2+28^2
            =19^2+2\cdot 18^2
            =31^2+3\cdot 4^2
            =15^2+4\cdot 14^2
            =17^2+5\cdot 12^2
            \\&=25^2+6\cdot 8^2
            =1^2+7\cdot 12^2
            =19^2+8\cdot 9^2
            =28^2+9\cdot 5^2
            =3^2+10\cdot 10^2.
        \end{align*}
\end{enumerate}
Do $1009<1201,$ ta kết luận $p=1009$ là số nguyên tố nhỏ nhất thỏa yêu cầu.}
\begin{luuy}
\nx Bài toán gốc của bài toán này là đề thi chọn đội tuyển toán quốc gia của Ba Lan và được đăng trên \chu{Tạp chí Toán học và Tuổi trẻ số 330}. Đây là một thách thức không hề nhỏ với bạn đọc và không có bất kì đóng góp lời giải nào gửi về báo năm đấy.
\end{luuy}
\end{gbtt}

\begin{gbtt}
Tìm các số nguyên tố $p,q,r$ thỏa mãn $p^q+q^p=r.$ 
\loigiai{
Giả sử tồn tại các số $p,q,r$ thỏa yêu cầu. Trong bài toán này, ta xét các trường hợp sau đây.
\begin{enumerate}
    \item Nếu $p,q$ cùng lẻ thì $p^q+q^p$ là hợp số chẵn lớn hơn $2,$ mâu thuẫn.
    \item Nếu $q=2$ thì $p^2+2^p$ là số nguyên tố. Dễ thấy $p=3$ thỏa mãn, còn $p=2$ thì không. \\
    Với $p\ge 5,$ do $p$ lẻ và $p$ không chia hết cho $3$ nên là
    $$p^2+2^p\equiv 1+2\equiv 3\pmod{3}.$$
    Do $p^2+2^p>3$ nên lúc này $p^2+2^p$ là hợp số, mâu thuẫn.
    \item Nếu $p=2,$ ta dễ dàng tìm ra $q=3$ và $r=17.$
\end{enumerate}
Kết luận, các bộ $(p,q,r)$ thỏa yêu cầu là $(2,3,17)$ và $(3,2,17).$}
\end{gbtt}

\begin{gbtt} Cho $p>3$ là số nguyên tố, chứng minh rằng với mọi số tự nhiên $n$ thì ba số $p+2,2^{n}+p$ và $2^{n}+p+2$ không thể đều là số nguyên tố.
\nguon{Chọn học sinh giỏi quốc gia Quảng Trị 2017 $-$ 2018}
\loigiai{
Giả sử tồn tại số nguyên dương $n$ và số nguyên tố $p$ thỏa mãn. Ta xét các trường hợp sau.
    \begin{itemize}
        \item\chu{Trường hợp 1.} Nếu $n$ là số chẵn, ta có $2^n\equiv 1\pmod{3},$ vì thế
        $$2^n+p\equiv p+1\pmod{3},\quad 2^n+p+2\equiv p\pmod{3}.$$
        \item\chu{Trường hợp 2.} Nếu $n$ là số chẵn, ta có $2^n\equiv 2\pmod{3},$ vì thế        
        $$2^n+p\equiv p+2\pmod{3},\quad 2^n+p+2\equiv p+1\pmod{3}.$$        
    \end{itemize}
    Trong mọi trường hợp, các số $p+2,2^n+p$ và $2^n+p+2$ không có cùng số dư khi chia cho $3.$ Cả $3$ số này đều lớn hơn $3$ nên nó không là hợp số. Giả sử sai, và ta thu được điều phải chứng minh.}
\end{gbtt}

\begin{gbtt}
Tìm tất cả các số tự nhiên $n$ để $A=2^{2^{2n+1}}+3$ là số nguyên tố.
\loigiai{
Với $n=0$ thì $A=2^2+3=7$ là số nguyên tố. Với $n\ge 1$, ta có 
\[2^{2^{2 n+1}}=2^{2.2^{2 n}}=\left(2^{2^{2 n}}\right)^{2}=\left(2^{4^{n}}\right)^{2}.\]
Vì $4^n$ chia $3$ dư $1$ nên ta có thể đặt $4^n=3k+1,$ với $n$ nguyên dương. Khi đó $A=4\cdot 8^{2k}+3.$ Ta nhận xét 
\[8 \equiv 1
\pmod{7}\Rightarrow8^{2 k} \equiv 1\pmod{7} \Rightarrow 4 \cdot 8^{2 k} \equiv 4\pmod{7}.\]
Lập luận trên cho ta biết $A$ chia hết cho $7$, lại do $A>7$ nên $A$ là hợp số, mâu thuẫn. \\
Tóm lại, $n=0$ là giá trị duy nhất thỏa yêu cầu.}
\end{gbtt}

\begin{gbtt}
Tìm tất cả các số tự nhiên $m,n$ thỏa mãn $P=3^{3m^2+6n-61}+4$ là số nguyên tố.
\nguon{Chọn học sinh giỏi thành phố Hà Tĩnh 2016}
\loigiai{
Rõ ràng $3m^2+6n-61$ cũng là số tự nhiên. Ngoài ra, do $3m^2+6n-61\equiv 2\pmod{3}$ nên ta có thể đặt $$3m^2+6n-61=3k+2,$$ với $k$ là số tự nhiên. Phép đặt này cho ta
	$$P=3^{3k+2}+4=9 \cdot 27^k+4\equiv 9+4\equiv 0\pmod{13}.$$
    Dựa vào nhận xét trên, ta suy ra $P$ chia hết cho $13,$ thế nên $P$ là số nguyên tố chỉ khi $P=13.$ Điều này chứng tỏ $3m^2+6n-61=2,$ hay $m^2+2n=21.$ Do đó
    $$m^2\le 21\Rightarrow m\le \sqrt{21}\Rightarrow m\le 4.$$
Kiểm tra trực tiếp với $m=1,2,3,4,$ ta chỉ ra có $2$ cặp số $(m,n)$ thỏa yêu cầu là $(1,10)$ và $(3,6)$.}
\end{gbtt}

\begin{gbtt}
Tồn tại hay không số tự nhiên \(n\) thỏa mãn \(8^n+47\) là số nguyên tố?
\nguon{Junior Balkan Mathematical Olympiad Shorlist 2020}
\loigiai{
    Câu trả lời phủ định. Thật vậy, ta xem xét các trường hợp sau đây.
\begin{enumerate}
    \item Nếu \(n\) là số chẵn, ta đặt \(n=2k\), trong đó \(k\) là một số nguyên dương nào đó. Lúc này
    \[8^n+47=64^{k}+47 \equiv 1+2 \equiv 0 \pmod{3},\]
    lại do \(8^n+47>3\) nên nó không thể là số nguyên tố.
    \item Nếu \(n\equiv 1\pmod{4}\), ta đặt \(n=4l+1\), trong đó \(l\) là một số tự nhiên nào đó. Lúc này
    \[8^n+47=8 \cdot\left(8^{k}\right)^{4}+47 \equiv 3+2 \equiv 0\pmod{5},\]
    lại do \(8^n+47>5\) nên nó không thể là số nguyên tố.
    \item Nếu \(n\equiv 3\pmod{4}\), ta đặt \(n=4m+1\), trong đó \(m\) là một số tự nhiên nào đó. Lúc này
    \[m=8\left(64^{2 k+1}+1\right) \equiv 8\left((-1)^{2 k+1}+1\right) \equiv 0\pmod{13},\]
    lại do \(8^n+47>13\) nên nó không thể là số nguyên tố.
\end{enumerate}}
\end{gbtt}

\begin{gbtt} \label{chedejbmo2}
Tìm tất cả các số nguyên tố $p$ sao cho $5^{p}+12^{p}$ là lũy thừa với số mũ lớn hơn $1$ của một số nguyên dương.
\loigiai{
Ta xét các trường hợp sau đây.
\begin{enumerate}
    \item  Nếu $p=2,$ ta có $5^p+12^p=169=13^2$ thỏa yêu cầu bài toán.
    \item Nếu $p\ge 3,$ ta nhận thấy $p$ là số lẻ, thế nên
    $$5^p+12^p=17\left(5^{p-1}-5^{p-2}\cdot12+5^{p-3}\cdot12^2-\ldots-5\cdot12^{p-2}+12^{p-1}\right).$$
    Cũng do $p$ là số lẻ, ta có các nhận xét dưới đây
    \begin{align*}
        5^{p-1}\equiv (-12)^{p-1}\equiv 12^{p-1}&\pmod{17},\\
        -5^{p-2}\cdot12\equiv -(-12)^{p-2}\cdot12\equiv 12^{p-1}&\pmod{17},  \\
        \ldots \\
        -5\cdot12^{p-2}\equiv -(-12)\cdot12^{p-2}\equiv 12^{p-1}&\pmod{17}.       
    \end{align*}
    Lấy tổng theo vế, ta chỉ ra
    \[5^{p-1}-5^{p-2}\cdot12+\ldots-5\cdot12^{p-2}+12^{p-1}\equiv p\cdot12^{p-1}\pmod{17}.\tag{1}\label{3p4p.1.1}\]
    Mặt khác, do $5^p+12^p$ chia hết cho $17$ và là một lũy thừa số mũ nguyên dương lớn hơn $1$ của một số nguyên dương,  ta cần phải có $5^p+12^p$ chia hết cho $17^2,$ kéo theo
    \[5^{p-1}-5^{p-2}\cdot12+\ldots-5\cdot12^{p-2}+12^{p-1}\equiv p\cdot12^{p-1}\pmod{17}.\tag{2}\label{3p4p.1.2}\] 
    Đối chiếu (\ref{3p4p.1.1}) và (\ref{3p4p.1.2}), ta chỉ ra $p\cdot12^{p-1}$ chia hết cho $17,$ thế nên $p=17.$ 
    \\Tuy nhiên, khi thử trực tiếp, số $p=17$ này không thỏa mãn.
\end{enumerate}
Kết luận, $p=2$ là số nguyên tố duy nhất thỏa yêu cầu.}
\end{gbtt}

\begin{gbtt}
Xét dãy số nguyên tố $p_1,p_2,p_3,\ldots, p_n$ thỏa mãn $p_1=5$ và với mọi $n \geq 1, p_{n+1}$ là ước nguyên tố lớn nhất của $p_{1} p_2\cdots p_n+1$ với mọi $n\ge 2.$ Chứng minh rằng $p_n\ne 7$ với mọi số nguyên dương $n.$

\loigiai{
Thử trực tiếp, ta tính toán được $p_1=5, \:p_2=3, p_3=2,\:p_4=31.$  Ta sẽ đi chứng minh $p_n\ne 2,3,5,$ với mọi $n\ge 4.$ Thật vậy, do $p_n$ là ước của 
    $$p_{1} p_{2} p_3p_4\cdots p_{n}+1=5\cdot3\cdot2\cdot p_4p_5\cdots p_n+1$$
nên $(p_n,2)=(p_n,3)=(p_n,5)=1,$ và lại do $p_n$ là số nguyên tố nên chúng khác $2,3$ và $5.$\\ Tiếp theo, ta giả sử tồn tại số nguyên tố $p_n=7.$ Khi phân tích số
    $$p_{1} p_{2} p_3p_4\cdots p_{n-1}+1$$ ra thừa số nguyên tố, ta không nhận được các thừa số $2,3$ và $5,$ đồng thời $7$ là thừa số nguyên tố lớn nhất của phân tích này. Vậy nên, $p_1p_2p_3p_4\cdots p_{n-1}+1$ phải là lũy thừa của $7.$ Ta đặt
    $$p_1p_2p_3\cdots p_{n-1}+1=7^r.$$
    Tích $p_1p_2p_3\cdots p_{n-1}$ gồm một thừa số $5$ nên $7^r-1$ chia hết cho $5.$ Xét các trường hợp về số dư của $r$ khi chia cho $4,$ ta chỉ ra $r\equiv 0\pmod{4}.$ Nhưng lúc này $7^r-1$ chia hết cho
    $$7^4-1=2400=2^5\cdot3\cdot5^2,$$
    mâu thuẫn với việc tích $p_1p_2p_3p_4\cdots p_{n-1}$ chỉ chứa một thừa số $2,$ còn các thừa số kia lẻ. Như vậy, giả sử phản chứng là sai. Bài toán được chứng minh.}
\end{gbtt}

\begin{gbtt}
Tìm tất cả các số nguyên dương \(n\) sao cho tồn tại số nguyên tố \(p\) thỏa mãn  $p^n-(p-1)^n$ là một lũy thừa của \(3\).
\nguon{Junior Balkan Mathematical Olympiad Shortlist 2017}
    \loigiai{
    Giả sử tồn tại số nguyên dương \(n\) thỏa mãn 
    \[p^{n}-(p-1)^{n}=3^{a},\tag{*}\label{jbmonuane.1}\]
    với \(p\) là một số nguyên tố nào đó và \(a\) là một số nguyên dương.
    \begin{enumerate}
        \item Với \(p=2\), thế vào (\ref{jbmonuane.1}) ta được $2^{n}-1=3^{a}.$
        Lấy modulo $3$ hai vế, ta có 
        $$(-1)^{n}-1 \equiv 0\pmod{3}.$$
        Bắt buộc, \(n\) phải là số chẵn. Ta đặt \(n=2s\). Lúc này, vì $2^n-1=3^a$ nên
        $$\left(2^{s}-1\right)\left(2^{s}+1\right)=3^{a}.$$
        Từ đây ta suy ra $2^{s}-1$ và $2^{s}+1$ đều là lũy thừa của \(3\). Hai số này không cùng chia hết cho $3$ nên một trong hai số ấy phải bằng $1,$ và ta thu được $s=1,n=2.$
        \item Với \(p=3\), thế vào (\ref{jbmonuane.1}) ta được
        $3^n-2^n=3^a.$
        Lấy đồng dư modulo $3$ hai vế, ta chỉ ra $2^n$ chia hết cho $3.$ Điều này là không thể.
        \item Với $p \geq 5$, ta có $p$ không chia hết cho $3.$ Kết hợp với (\ref{jbmonuane.1}), cả $p-1$ cũng không chia hết cho $3.$ Hai nhân xét trên và việc lấy đồng dư modulo $3$ hai vế của (\ref{jbmonuane.1}) cho ta
        $$2^n-1\equiv 0\pmod{3}.$$
        Ta lại tìm ra $n$ chẵn. Đặt \(n=2k\), trong đó \(k\) là một số nguyên dương. Thế vào (\ref{jbmonuane.1}), ta lần lượt suy ra
        $$p^{2 k}-(p-1)^{2 k}=3^{a} \Rightarrow\left(p^{k}-(p-1)^{k}\right)\left(p^{k}+(p-1)^{k}\right)=3^{a}.$$
        Nếu ta đặt $d=\left(p^{k}-(p-1)^{k}, p^{k}+(p-1)^{k}\right)$ thì $d\mid 2p^{k}.$ Tuy nhiên do cả hai số đều là lũy thừa của \(3\) nên ta phải có \(d=1\), kéo theo $$p^{k}-(p-1)^{k}=1, p^{k}+(p-1)^{k}=3^{a}.$$
        Tới đây, ta xét các trường hợp sau
        \begin{itemize}
            \item \chu{Trường hợp 1.} Nếu \(k=1\) thì \(n=2\), và ta có thể chọn \(p=5\).
            \item \chu{Trường hợp 2.} Nếu $k \geq 2$, ta đánh giá
            $$1=p^{k}-(p-1)^{k} \geq p^{2}-(p-1)^{2}.$$
            Đánh giá trên là đúng, do $$p^{2}\left(p^{k-2}-1\right) \geq(p-1)^{2}\left((p-1)^{k-2}-1\right).$$
            %alo rep ib nhóm anh ơiii để t gửi
            Vậy nên, $1\geq p^{2}-(p-1)^{2}=2 p-1 \geq 9.$ Điều này là không thể.
        \end{itemize}
     \end{enumerate}
Tổng kết lại, $n=2$ là số tự nhiên duy nhất thỏa yêu cầu.}
\end{gbtt}

\begin{gbtt}
Tìm tất cả các số chính phương $n$ sao cho với mọi ước nguyên dương $a\ge 15$ của $n$ thì $a+15$ là lũy thừa của một số nguyên tố.
\nguon{Junior Balkan Mathematical Olympiad Shortlists 2019}
\end{gbtt}
\nx Ta đã biết, tổng của một số lẻ và $15$ là một số chẵn. Trong khi đó, nếu lũy thừa của một số nguyên tố mang giá trị là chẵn thì bắt buộc số nguyên tố đó phải bằng $2.$ Dựa theo lập luận này, ta sẽ tìm được hướng đi tốt nhất cho bài toán.
\loigiai{ 
Trong bài toán này, ta xét các trường hợp sau đây.
 \begin{enumerate}
     \item Nếu \(n\) là một lũy thừa của \(2\), ta tìm được $n\in\{1;4;16;64\}.$ Thật vậy, ta sẽ đi chứng minh $n<2^7.$ Trong trường hợp $n\ge 2^7,$ $n$ sẽ nhận $2^7$ làm ước. Tuy nhiên
     $$2^7+15=143=11\cdot13.$$
     không là lũy thừa số nguyên tố nào, trái giả thiết.
     \item Nếu $n$ nhận $3$ làm ước nguyên tố và $n$ không có ước nguyên tố lớn hơn $3,$ ta có thể biểu diễn $n=4^r9^s.$ Dựa trên các tính toán sau
     $$2^7+15=143=11\cdot13,\quad 3^3+1=28=2\cdot14,$$
     ta chỉ ra $r\le 3$ và $s=1.$ Thử trực tiếp, chỉ có $n=9$ thỏa mãn.
     \item Nếu \(n\) có ước nguyên tố lẻ $p>3,$ do $n$ chính phương nên $p^2$ cũng là ước của $n$.\\
     Mặt khác, từ $p^2\ge 25>15,$ ta suy ra tồn tại số nguyên tố $q$ sao cho
     \[p^{2}+15=q^{m},\]
     Với việc $p$ là số nguyên tố lẻ, ta nhận định $q^m$ là số chẵn. Theo đó, $q=2,$ và
     \[p^{2}+15=2^{m}.\tag{*}\label{jbmo.ne.ae}\]
     Lấy modulo $3$ hai vế (\ref{jbmo.ne.ae}), ta có
     $1\equiv (-1)^m\pmod{3}.$
     Bắt buộc, $m$ phải là số chẵn. Ta đặt \(m=2k\), trong đó $k$ là một số nguyên dương. Thế vào (\ref{jbmo.ne.ae}), ta chỉ ra
     \[\left(2^{k}-p\right)\left(2^{k}+p\right)=15,\]
     Do $\left(2^{k}+p\right)-\left(2^{k}-p\right)=2 p \geqslant 10$, ta phải có \(2^{k}-p=1\) \text {và} \(2^{k}+p=15\), kéo theo \(p=7\) và \(k=3\). Từ đây, ta có thể biểu diễn $n$ dưới dạng $$n=4^{x} \cdot 9^{y} \cdot 49^{z},$$ trong đó \(x,y,z\) là các số nguyên không âm nào đó.
    Dựa trên các tính toán
     $$2^7+1=143=11\cdot13,\quad 3^3+1=28=2^2\cdot7,\quad 7^3+1=2^3\cdot43.$$
     ta chỉ ra $x\le 3,y\le 1$ và $z\le 1.$ Thử trực tiếp, ta tìm ra $n=49$ và $n=196.$
 \end{enumerate}
Tổng kết lại, có tất cả $7$ số nguyên dương $n$ thỏa yêu cầu, đó là
$1,4,9,16,49,64,196.$}

\section{Định lí Fermat và ứng dụng}

\subsection{Lí thuyết}

Trước khi đến với các dạng bài tập trong phần này, chúng ta cần làm quen với một định lí khá nổi tiếng, đó là định lí $Fermat$ nhỏ
\begin{light}
Với mọi số nguyên tố $p$ và số nguyên $a$ bất kì, ta có
\[a^p\equiv a\pmod{p}.\]
Ngoài ra, trong trường hợp $(a,p)=1,$ ta còn có
\[a^{p-1}\equiv 1\pmod{p}.\]
\end{light}
\chu{Chứng minh.} Ta sẽ chứng minh định lí trên bằng phương pháp quy nạp.\\
Thật vậy, bài toán đúng với $a=1.$ Giả sử bài toán đúng với $a=1,2,\ldots,n.$ Theo giả thiết quy nạp, ta có
$$n^p\equiv n\pmod{p}.$$
Mặt khác, ta nhận thấy rằng
$$\vuong{(n+1)^p-(n+1)}-\tron{n^p-n}=a_1n+a_2n^2+\ldots+a_{p-1}n^{p-1}.$$
Theo như kiến thức đã biết về khai triển nhị thức $Newton,$ các số $a_1,a_2,\ldots,a_{p-1}$ chia hết cho $p.$ Như vậy
$$(n+1)^p-(n+1)\equiv n^p-n\equiv0\pmod{p}.$$
Theo nguyên lí quy nạp, bài toán được chứng minh.

\subsection{Ví dụ minh họa}

\begin{bx}
Cho $p$ là một số nguyên tố và hai số nguyên dương $x,y$. Chứng minh rằng 
\[x y^{p}-y x^{p}\text{ chia hết cho }p.\]
\loigiai{
Theo giả thiết, ta có $p$ là số nguyên tố. Áp dụng định lí \textit{Fermat} nhỏ, ta có
$$\heva{x^p\equiv x\pmod{p}\\
y^p\equiv y\pmod{p}
}\Rightarrow xy^p-x^py\equiv xy-xy\equiv0\pmod{p}.$$
Bài toán đã được chứng minh.}
\end{bx}

\begin{bx} \label{bodeveus1}
Cho $p$ là số nguyên tố có dạng $4 k+3.$ Chứng minh rằng \[p \mid \left(x^2+y^2\right)\Leftrightarrow \heva{&p\mid x\\&p\mid y.}\]
\loigiai{
Ở đây, tác giả xin phép chỉ trình bày chiều thuận của phần chứng minh.\\
Ta giả sử phản chứng rằng $x$ không chia hết cho $p.$ Nhờ vào giả thiết chiều thuận là $x^2+y^2$ chia hết cho $p,$ ta suy ra $y$ cũng không chia hết cho $p.$ Mặt khác, ta có
\begin{align*}
  x^2\equiv -y^2\pmod{p}
&\Rightarrow \left(x^2\right)^{2k+1}\equiv \left(-y^2\right)^{2k+1} \pmod{p}
\\&\Rightarrow x^{4k+2}\equiv -y^{4k+2} \pmod{p}.  
\end{align*}
Áp dụng định lí $Fermat$ nhỏ, ta chỉ ra rằng
$$x^{4k+2}\equiv y^{4k+2}\equiv 1\pmod{p}.$$
Hai nhận xét trên cho ta $1\equiv -1\pmod{p},$ tức là $p=2,$ mâu thuẫn. \\
Giả sử phản chứng là sai. Khẳng định chiều thuận được chứng minh.}
\end{bx}

\begin{bx}\label{bodeveus2}
Cho $p$ là số nguyên tố có dạng $3k+2.$ Chứng minh rằng \[p \mid \left(x^2+xy+y^2\right)\Leftrightarrow \heva{&p\mid x\\&p\mid y.}\]
\loigiai{
Ở đây, tác giả xin phép chỉ trình bày chiều thuận của phần chứng minh.\\
Ta giả sử phản chứng rằng $x$ không chia hết cho $p.$ Nhờ vào giả thiết chiều thuận là $x^2+xy+y^2$ chia hết cho $p,$ ta suy ra $y$ cũng không chia hết cho $p.$ Mặt khác, ta có
\begin{align*}
  x^2+xy+y^2\equiv 0\pmod{p}&\Rightarrow x^3\equiv y^3\pmod{p}
\\&\Rightarrow x^{3k}\equiv y^{3k} \pmod{p}
\\&\Rightarrow yx^{3k+1}\equiv xy^{3k+1} \pmod{p}.  
\end{align*}
Áp dụng định lí $Fermat$ nhỏ, ta chỉ ra rằng
$$x^{3k+1}\equiv y^{3k+1}\equiv 1\pmod{p}.$$
Hai nhận xét trên cho ta $x\equiv y\pmod{p}.$\\ Tiếp tục kết hợp với giả thiết chiều thuận $x^2+xy+y^2$ chia hết cho $p,$ ta có
$$0\equiv x^2+xy+y^2\equiv 3x^2 \pmod{p}.$$
Ta được $3x^2$ chia hết cho $p,$ nhưng do $x$ không chia hết cho $p$ nên bắt buộc $p=3,$ mâu thuẫn. \\
Giả sử phản chứng là sai. Khẳng định chiều thuận được chứng minh.}
\begin{luuy}
\nx Các bổ đề trên cũng được sử dụng để giải quyết vấn đề quen thuộc, đó là
\begin{enumerate}
    \item \chu{Euler problem.} \\
    Phương trình $4 x y-x-y=z^{2}$ không có nghiệm nguyên dương.
    \item \chu{Lebesgue problem.} \\ 
    Phương trình $x^{2}-y^{3}=7$ không có nghiệm nguyên dương.
\end{enumerate}
\end{luuy}
\end{bx}
\subsection{Bài tập tự luyện}

\begin{btt}
Cho $p, q$ là hai số nguyên tố phân biệt. Chứng minh rằng 
\[p^{q-1}+q^{p-1}-1\text{ chia hết cho }pq.\]
\end{btt}

\begin{btt}
Cho $p$ là số nguyên tố khác $2$ và $a, b$ là hai số tự nhiên lẻ sao cho $a+b$ chia hết cho $p$ và $a-b$ chia hết cho $p-1$. Chứng minh rằng $a^{b}+b^{a}$ chia hết cho $2p.$
\end{btt}

\begin{btt}
Tìm các số nguyên dương $n$ sao cho $a^{25}-a$ chia hết cho $n$ với mọi số nguyên $a.$
\nguon{Bulgarian Mathematical Olympiad 1995}
\end{btt}

\begin{btt}
Chứng minh rằng với mọi số nguyên tố $p>7,$ ta có
\[3^{p}-2^{p}-1\text{ chia hết cho }42p.\]
\end{btt}

\begin{btt}
Tìm tất cả các số nguyên tố \(p\) thỏa mãn
\[(x+y)^{19}-x^{19}-y^{19}\text{ chia hết cho }p,\]
với mọi số nguyên dương $x,y.$
\nguon{Junior Balkan Mathematical Olympiad 2020}
\end{btt}

\begin{btt}
Tìm tất cả các số nguyên tố \(p\) sao cho tồn tại các số nguyên dương \(x,y,z\) thỏa mãn
    $$x^{p}+y^{p}+z^{p}-x-y-z$$
là tích của ba số nguyên tố phân biệt.
\nguon{Junior Balkan Mathematical Olympiad Shortlist 2019}
\end{btt}

\begin{btt}
Chứng minh rằng với mọi số nguyên tố $p$ thì $p^3+\dfrac{p-1}{2}$ không phải là tích của hai số tự nhiên liên tiếp.
\nguon{Chọn học sinh giỏi Hà Tĩnh 2014}
\end{btt}

\begin{btt}
Với $p$ là số nguyên tố lẻ, đặt $A=23 p+3^{p} - 4.$ Chứng minh rằng
\begin{enumerate}[a,]
    \item $A$ không phải là bình phương bất kì số tự nhiên nào.
    \item $A$ không phải là tích của bất kì hai số nguyên dương liên tiếp nào.
\end{enumerate}
\end{btt}

\begin{btt}
Cho \(a,b,c\) là các số nguyên dương, gọi \(p\) là số nguyên tố thỏa mãn đồng thời
\begin{multicols}{3}
\begin{enumerate}[i,]
    \item \(p\mid \left(a^2+ab+b^2\right)\).
    \item \(p\mid \left(a^5+b^5+c^5\right)\).
    \item \(p\nmid \left(a+b+c\right)\).
\end{enumerate}
\end{multicols}
Chứng minh rằng \(p\) là một số nguyên tố có dạng \(6k+1\), trong đó \(k\) là một số nguyên dương.
\nguon{Chọn đội dự tuyển toán 10 Phổ thông Năng khiếu 2015}
\end{btt}

\subsection{Hướng dẫn tập tự luyện}

\begin{gbtt}
Cho $p, q$ là hai số nguyên tố phân biệt. Chứng minh rằng 
\[p^{q-1}+q^{p-1}-1\text{ chia hết cho }pq.\]
\loigiai{
Từ giả thiết, ta nhận thấy $(p,q)=1.$ Áp dụng định lí $Fermat$ nhỏ, ta có
$$
\heva{&p\mid \tron{q^{p-1}-1} \\ &q\mid \tron{p^{q-1}-1}}
\Rightarrow
\heva{&p\mid \tron{p^{q-1}+q^{p-1}-1} \\ &q\mid \tron{q^{p-1}+p^{q-1}-1}}
\Rightarrow
pq\mid\tron{p^{q-1}+q^{p-1}-1}.
$$
Bài toán đã cho được chứng minh.}
\end{gbtt}

\begin{gbtt}
Cho $p$ là số nguyên tố khác $2$ và $a, b$ là hai số tự nhiên lẻ sao cho $a+b$ chia hết cho $p$ và $a-b$ chia hết cho $p-1$. Chứng minh rằng $a^{b}+b^{a}$ chia hết cho $2p.$
\loigiai{
Áp dụng định lí \textit{Fermat} nhỏ, ta có
$p\mid b\tron{b^{p-1}-1}.$ Ngoài ra, do $p-1$ là ước của $a-b$ nên 
$$p\mid b\tron{b^{p-1}-1}\mid b\tron{b^{a-b}-1}\mid b^b\tron{b^{a-b}-1}.$$
Kết hợp với đồng dư thức thu được từ giả thiết là $a\equiv-b\pmod{p},$ ta suy ra
$$a^{b}+b^{a}\equiv -b^b+b^a\equiv b^b\tron{b^{a-b}-1}\equiv0\pmod{p}.$$
Bài toán được chứng minh.
}
\end{gbtt}

\begin{gbtt}
Tìm các số nguyên dương $n$ sao cho $a^{25}-a$ chia hết cho $n$ với mọi số nguyên $a.$
\nguon{Bulgarian Mathematical Olympiad 1995}
\loigiai{
Vì $n$ là ước của $a^{25}-a$ với mọi số nguyên $a$ nên $n$ là ước chung của $2^{25}-2$ và $3^{25}-3.$ Ta có $$\tron{2^{25}-2,3^{25}-3}=2\cdot3\cdot5\cdot7\cdot13.$$
Ta sẽ chứng minh $a^{25}-a$ chia hết cho $2\cdot3\cdot5\cdot7\cdot13$ với mọi số tự nhiên $a$. Vì $2,3,5,7,13$ là các số nguyên tố nên hướng đi của ta là áp dụng định lí \textit{Fermat} nhỏ.
\begin{itemize} 
    \item Xét modulo $2$, ta thu được 
    $2\mid a(a-1)\mid a\tron{a^{24}-1}.$
    \item Xét modulo $3$, ta thu được     $3\mid a\tron{a^2-1}\mid a\tron{a^{24}-1}.$
    \item Xét modulo $5$, ta thu được $5\mid a\tron{a^4-1}\mid a\tron{a^{24}-1}.$
      \item Xét modulo $7$, ta thu được $7\mid a\tron{a^6-1}\mid a\tron{a^{24}-1}.$
      \item Xét modulo $13$, ta thu được $13\mid a\tron{a^{12}-1}\mid a\tron{a^{24}-1}.$
\end{itemize}
Vậy các số tự nhiên $n$ cần tìm là ước nguyên dương của $2\cdot3\cdot5\cdot7\cdot13=2730.$ 
}
\end{gbtt}

\begin{gbtt}
Chứng minh rằng với mọi số nguyên tố $p>7,$ ta có
\[3^{p}-2^{p}-1\text{ chia hết cho }42p.\]
\loigiai{
Ta đặt $A=3^{p}-2^{p}-1.$ Ta chia bài toán thành các bước làm sau.
\begin{enumerate}[\color{tuancolor}\bf\sffamily Bước 1.]
    \item {Chứng minh $A$ chia hết cho $p.$}\\
    Áp dụng định lí \textit{Fermat} nhỏ, ta có $3^p\equiv3\pmod{p}$ và $2^p\equiv2\pmod{p}$. Ta suy ra
    $$3^p-2^p-1\equiv 3-2-1\equiv 0 \pmod{p}.$$
    \item {Chứng minh $A$ chia hết cho $2.$} Điều này là hiển nhiên, do $A$ chẵn.
    \item {Chứng minh $A$ chia hết cho $3.$}\\
    Vì $p>7$, do đó $p$ là số nguyên tố lẻ. Đặt $p=2k+1$ với $k$ là số nguyên dương. Phép đặt này cho ta
    $$2^p\equiv 2^{2k+1}\equiv4^k\cdot2\equiv2\pmod{3}\Rightarrow 3^{p}-2^{p}-1\equiv0-2-1\equiv0\pmod{3}.$$
    \item {Chứng minh $A$ chia hết cho $7.$}\\
    Áp dụng định lí \textit{Fermat} nhỏ, ta có 
    $3^6\equiv 1\pmod{7}$ và $2^6\equiv1\pmod{7}$.\\
    Vì $p$ là số nguyên tố lớn hơn $7$ nên $p$ có dạng $6k+1$ hoặc $6k-1.$
    \begin{itemize}
        \item\chu{Trường hợp 1.} Với $p=6k+1$, ta có
        $$3^{p}-2^{p}-1\equiv3^{6k+1}-2^{6k+1}-1\equiv 3-2-1\equiv0\pmod{7}.$$
        \item\chu{Trường hợp 2.} Với $p=6k+5$, ta có
         $$3^{p}-2^{p}-1\equiv3^{6k+5}-2^{6k+5}-1\equiv 3^5-2^5-1\equiv210\equiv0\pmod{7}.$$
    \end{itemize}
\end{enumerate}
Với việc các số $2,3,7,p$ có tích bằng $42p$ và đôi một nguyên tố cùng nhau, bài toán được chứng minh.}
\end{gbtt}

\begin{gbtt}
Tìm tất cả các số nguyên tố \(p\) thỏa mãn
\[(x+y)^{19}-x^{19}-y^{19}\text{ chia hết cho }p,\]
với mọi số nguyên dương $x,y.$
\nguon{Junior Balkan Mathematical Olympiad 2020}
\loigiai{
Với $x=y=2$, ta có
\begin{align*}
    (2+2)^{19}-2^{19}-2^{19}&= 2^{38}-2^{20}\\&=2^{20}\tron{2^{18}-1}\\&=2^{20}\tron{2^9+1}\tron{2^3+1}\tron{2^3-1}\\&
    =2^{20}\cdot513\cdot9\cdot7.
\end{align*}
Vì $p$ là ước của $(x+y)^{19}-x^{19}-y^{19}$ với mọi số nguyên dương $x,y$ nên  ta có
$$p\mid 2^{20}\cdot513\cdot9\cdot7.$$
Ta sẽ chứng minh $(x+y)^{19}-x^{19}-y^{19}$ chia hết cho $2,3,7,19$ với mọi $x,y$ nguyên dương.\\
Ta dễ dàng chứng minh được nếu $x$ hoặc $y$ chia hết cho $p$ thì $(x+y)^{19}-x^{19}-y^{19}$ chia hết cho $p$. \\Ta xét trường hợp $\tron{x,p}=1$ và $\tron{y,p}=1.$ 
\begin{enumerate}
    \item Với $x+y$ chia hết cho $p$, áp dụng định lí \textit{Fermat} nhỏ cho modulo $3$, ta nhận được $x^2\equiv1\pmod{3}$ và $y^2\equiv 1\pmod{2}$. Từ đây, ta có
     $$(x+y)^{19}-x^{19}-y^{19}\equiv 0^{19}-\tron{x^2}^9 x-\tron{y^2}^9 y\equiv0-x-y\equiv0\pmod{3}.$$
    Chứng minh tương tự với modulo $2,7,19$, ta thu được $(x+y)^{19}-x^{19}-y^{19}$ chia hết cho $2,7,19.$
    \item Với $\tron{x+y,p}=1$ , áp dụng định lí \textit{Fermat} nhỏ cho modulo $3$, ta có 
    $$\tron{x+y}^2\equiv1\pmod{3}, \quad x^2\equiv 1\pmod{3}, \quad y^2\equiv1\pmod{3}.$$
    Các đồng dư thức trên cho ta
    \begin{align*}
        (x+y)^{19}-x^{19}-y^{19}
        &\equiv \tron{(x+y)^2}^{9}\tron{x+y}-\tron{x^2}^9x-\tron{y^2}^9 y
        \\&\equiv x+y-x-y\\&
        \equiv0\pmod{3}.
    \end{align*}
    Chứng minh tương tự với modulo $2,7,19$, ta nhận được $(x+y)^{19}-x^{19}-y^{19}$ chia hết cho $2,7,19.$
\end{enumerate}
Như vậy, tất cả các số nguyên tố $p$ cần tìm là $2,3,7,19.$}
\end{gbtt}

\begin{gbtt}
Tìm tất cả các số nguyên tố \(p\) sao cho tồn tại các số nguyên dương \(x,y,z\) thỏa mãn
    $$x^{p}+y^{p}+z^{p}-x-y-z$$
là tích của ba số nguyên tố phân biệt.
\nguon{Junior Balkan Mathematical Olympiad Shortlist 2019}
    \loigiai{
    Trước tiên, ta đặt $A=x^{p}+y^{p}+z^{p}-x-y-z$. Trong bài toán này, ta xét các trường hợp sau.
    \begin{enumerate}
        \item  Nếu \(p=2\), ta chọn $x=y=4$ và $z=3,$ khi đó $A=30=2\cdot 3\cdot 5$ là tích ba số nguyên tố phân biệt.
        \item Nếu \(p=3\), ta chọn $x=3,y=2$ và $z=1$, khi đó $A=30=2 \cdot 3 \cdot 5$ là tích ba số nguyên tố phân biệt.
        \item Nếu \(p=5\), ta chọn $x=2,y=1$ và $z=1$, khi đó $A=30=2 \cdot 3 \cdot 5$ là tích ba số nguyên tố phân biệt.
        \item Nếu $p \geqslant 7$, xét \(A\) trong modulo \(2\) và \(3\) thì ta thấy \(A\) chia hết cho cả \(2\) và \(3\). Hơn nữa, từ định lí nhỏ $Fermat$ ta có đồng dư thức sau
        \[x^{p}+y^{p}+z^{p}-x-y-z\equiv x+y+z-x-y-z\equiv 0\pmod{p}.\]
        Với giả sử $A$ là tích ba số nguyên tố phân biệt, ba số đó bắt buộc là $2,3$ và $p.$ Như vậy
        \[x^{p}+y^{p}+z^{p}-x-y-z=6p.\]
        Rõ ràng, ta không thể chọn $x=y=z=1.$ Theo đó, trong ba số \(x,y,z\), có ít nhất một số lớn hơn hoặc bằng \(2\), giả sử là $x.$ Giả sử này cho ta
        \[6 p \geqslant x^{p}-x=x\left(x^{p-1}-1\right) \geqslant 2\left(2^{p-1}-1\right)=2^{p}-2.\]
        Dễ kiểm tra bằng quy nạp rằng $2^{n}-2>6 n$ với mọi số tự nhiên $n \geqslant 6$. Điều vô lí này đã khẳng định rằng khi $p \geqslant 7$ thì không tồn tại \(x,y,z\) thỏa mãn $A$ là tích của ba số nguyên tố liên tiếp.  \end{enumerate}
Kết luận, có tất cả ba số nguyên tố thỏa yêu cầu, đó là $p=2,p=3$ và $p=5.$}
\end{gbtt}

\begin{gbtt}
Chứng minh rằng với mọi số nguyên tố $p$ thì $p^3+\dfrac{p-1}{2}$ không phải là tích của hai số tự nhiên liên tiếp.
\nguon{Chọn học sinh giỏi Hà Tĩnh 2014}
\loigiai{
Trong bài toán này, ta xét các trường hợp dưới đây.
\begin{enumerate}
	 \item Với $p=2,$ ta có $p^3+\dfrac{p-1}{2}=\dfrac{17}{2}$ không là số nguyên.
	 \item Với $p=4k+1$, ta có $p^3+\dfrac{p-1}{2}=(4k+1)^3+2k$ là số lẻ, nên $p^3+\dfrac{p-1}{2}$ không thể là tích của hai số tự nhiên liên tiếp.
	 \item Với $p=4k+3$, ta giả sử tồn tại số nguyên dương $x$ thỏa mãn $p^3+\dfrac{p-1}{2}=x(x+1).$ Khi đó  $$2p(2p^2+1)=(2x+1)^2+1.$$ Ta suy ra $(2x+1)^2+1$ chia hết cho $p,$ vô lí do $p=4k+3.$
\end{enumerate}
Nhờ các mâu thuẫn chỉ ra bên trên, bài toán được chứng minh.}
\end{gbtt}

\begin{gbtt}
Với $p$ là số nguyên tố lẻ, đặt $A=23 p+3^{p} - 4.$ Chứng minh rằng
\begin{enumerate}[a,]
    \item $A$ không phải là bình phương bất kì số tự nhiên nào.
    \item $A$ không phải là tích của bất kì hai số nguyên dương liên tiếp nào.
\end{enumerate}
\loigiai{
\begin{enumerate}[a,]
    \item  Giả sử tồn tại số tự nhiên $x$ sao cho $23 p+3^{p}-4=x^{2}$. Theo định lí $Fermat$ nhỏ, ta có
    $$x^2+1=23p+3^p-3\equiv 0\pmod{p}.$$
    Ta suy ra $x^2+1$ chia hết cho $p,$ và $p$ phải có dạng $4 k+1$. Điều này kéo theo
    $$23 p+3^{p}-4 x^2\equiv-p+(-1)^{p} \equiv 2 \pmod{4}.$$
    Do $x^2\equiv 0,1\pmod{4},$ đây là một điều mâu thuẫn. Kết luận, $A$ không phải là bình phương bất kì số tự nhiên nào.
    \item Giả sử tồn tại số nguyên dương $x$ sao cho $23p+3^p-4=x(x+1).$ Ta xét các trường hợp sau.
    \begin{itemize}
        \item\chu{Trường hợp 1.} Nếu $p=3,$ ta có $A=82,$ không là tích của hai số nguyên dương liên tiếp nào.
        \item\chu{Trường hợp 2.} Nếu $p \equiv 1\pmod{3},$ ta có
        $$x^2+x+1\equiv 23p\equiv 23\equiv2\pmod{3}.$$
       Đây là điều vô lí, do $x^2+x+1\equiv 0,1\pmod{3},$ với mọi số nguyên $x.$
        \item\chu{Trường hợp 3.} Nếu $p \equiv 2\pmod{3},$ áp dụng định lí $Fermat$ nhỏ, ta có
        $$x^2+x+1=23p+3^p-3\equiv 0\pmod{p}.$$
        Ta nhận được $x^2+x+1$ chia hết cho $p.$\\
        Theo như kiến thức đã học, ta có $1$ chia hết cho $p,$ mâu thuẫn.
    \end{itemize}
Kết luận, $A$ không phải là tích bất kì hai số nguyên dương liên tiếp nào.
\end{enumerate}}
\end{gbtt}

\begin{gbtt}
Cho \(a,b,c\) là các số nguyên dương, gọi \(p\) là số nguyên tố thỏa mãn đồng thời
\begin{multicols}{3}
\begin{enumerate}[i,]
    \item \(p\mid \left(a^2+ab+b^2\right)\).
    \item \(p\mid \left(a^5+b^5+c^5\right)\).
    \item \(p\nmid \left(a+b+c\right)\).
\end{enumerate}
\end{multicols}
Chứng minh rằng \(p\) là một số nguyên tố có dạng \(6k+1\), trong đó \(k\) là một số nguyên dương.
\nguon{Chọn đội dự tuyển toán 10 Phổ thông Năng khiếu 2015}
\loigiai{
Từ i, ta suy ra \(p\mid \left ( a-b \right )\left (a^2+ab+b^2 \right )=a^3-b^3\), do đó ta thu được
\[a^3\equiv b^3\pmod{p}.\]
Đồng thời, nếu $a$ và $b$ cùng chia hết cho $p$ thì $c$ chia hết cho $p,$ mâu thuẫn với iii. Mâu thuẫn này chứng tỏ $a$ và $b$ không đồng thời chia hết cho $p.$ Ta xét các trường hợp sau đây.
\begin{enumerate}
    \item Nếu \(p=2\) thì từ i, dễ thấy \(a,b\) phải cùng chẵn. Kết hợp với ii, ta được $c$ cũng là số chẵn, nhưng điều này dẫn đến $a+b+c$ chia hết cho $p=2,$ mâu thuẫn với iii.
    \item Nếu \(p=3\), áp dụng định lí $Fermat$ nhỏ, ta suy ra
    \[a\equiv a^{3}\equiv b^{3}\equiv b\pmod{3}.\]
    Kết hợp với ii, ta được \(3\mid\tron{2a^5+c^5}\), nhưng do
    \[3\mid a^5-a=\left ( a^3-a \right )\left ( a^2+1 \right )\]
    nên \(0\equiv 2a^5+c^5\equiv 2a+c\equiv a+b+c\pmod{3}\), mâu thuẫn với iii.
    \item Nếu \(p>3\) và $p\equiv 2\pmod{3},$ từ i và kiến thức đã học, cả $a$ và $b$ cùng chia hết cho $p,$ mâu thuẫn.
    \item Nếu \(p>3\) và $p\equiv 1\pmod{3},$ ta suy ra $p-1$ chia hết cho cả $3$ và $2,$ thế nên $p=6k+1$ với $k$ là một số nguyên dương nào đó. Trường hợp này là có thể xảy ra, vì chẳng hạn với $a=2,b=4,c=8,$ ta tìm được $p=7.$ 
\end{enumerate}
Tổng kết lại, bài toán được chứng minh.}
\end{gbtt}
 %số nguyên tố
\section{Số nguyên tố và tính nguyên tố cùng nhau}

\begin{it}
Tính chia hết của một tích cho một số nguyên tố là một tính chất đặc biệt của số nguyên tố. Trong các bài toán ấy, chúng ta sẽ đưa những đẳng thức hoặc phương trình đã cho về dạng phân tích $AB=pC$ với $p$ là số nguyên tố, rồi sau đó xét tới các trường hợp $A$ chia hết cho $p$ và $B$ chia hết cho $p.$ Dưới đây là một vài ví dụ minh họa.
\end{it}

\subsection{Một vài ví dụ mở đầu}

\begin{bx}
Cho $a$, $b$ là các số tự nhiên lớn hơn $2$ và $p$ là số tự nhiên thỏa mãn $\dfrac{1}{p} = \dfrac{1}{a^2} + \dfrac{1}{b^2}$. Chứng minh $p$ là  hợp số.
\nguon{Chọn học sinh giỏi lớp 9 Hà Nội 2011}
\loigiai{
Giả sử phản chứng rằng $p$ là số nguyên tố.\\
Với các số tự nhiên $a,b,p$ thỏa mãn đề bài, ta có $a^2b^2=p\left(a^2+b^2\right)$, thế nên
\[p \mid a^2b^2 \Rightarrow \hoac{& p \mid a^2\\& p \mid b^2 }\Rightarrow \hoac{& p \mid a\\& p \mid b}\Rightarrow p^2\mid a^2b^2=p\left(a^2+b^2\right)\Rightarrow p\mid \left(a^2+b^2\right).\label{homnayvuiquahihi}\]
Trong các suy luận kể trên, ta cũng chứng minh được $a$ hoặc $b$ chia hết cho $p.$ Kết hợp với việc $a^2+b^2$ chia hết cho $p,$ ta suy ra cả $a$ và $b$ chia hết cho $p,$ thế nên 
 $$\dfrac{1}{p}
 \le \dfrac{1}{a^2}+\dfrac{1}{b^2}\leqslant \dfrac{2}{p^2}\Rightarrow \dfrac{1}{p}\leqslant \dfrac{2}{p^2}\Rightarrow p\leqslant 2.$$
Ngoài ra, do $a,b$ đều là các số tự nhiên lơn hơn $2$ nên
$$\dfrac{1}{p}=\dfrac{1}{a^2}+\dfrac{1}{b^2}<\dfrac{1}{4}+\dfrac{1}{4}=\dfrac{1}{2}.$$
Hai lập luận trên mâu thuẫn nhau. Giả sử sai nên bắt buộc $p$ là hợp số.}
\end{bx}

\begin{bx}
Cho các số nguyên dương $a, b, c, d$ thỏa mãn $a^2+b^2+ab=c^2+d^2+cd.$ Chứng
minh rằng $a+b+c+d$ là hợp số.
\loigiai{Biến đổi tương đương giả thiết, ta có
\begin{align*}
    a^{2}+b^{2}+a b=c^{2}+d^{2}+c d 
    &\Leftrightarrow(a+b)^{2}-a b=(c+d)^{2}-c d
    \\&\Leftrightarrow(a+b)^{2}-(c+d)^{2}=a b-c d \\&\Leftrightarrow(a+b+c+d)(a+b-c-d)=a b-c d.
\end{align*}
Phản chứng, ta giả sử ${a}+{b}+{c}+{d}$ là số nguyên tố. Đặt ${a}+{b}+{c}+{d}={p}$, ta nhận thấy rằng $p$ là ước của
$$ab-cd+cp=ab-cd+ca+cb+c^2+cd=(a+c)(b+c).$$
Tuy nhiên, do $0<{c}+{a}, {c}+{b}<{p}$ nên $({c}+{a}, {p})=({b}+{c}, {p})=1$. Lập luận này chứng tỏ $p$ không là ước của $({a}+{c})({b}+{c})$, mâu thuẫn. Giả sử phản chứng là sai. Bài toán được chứng minh.}
\end{bx} 

\begin{bx}
Cho $\overline{a b c}$ là số nguyên tố. Chứng minh rằng $b^{2}-4 a c$ không phải là số chính phương.
\nguon{Titu Andreescu}
\loigiai{
Giả sử phản chứng rằng, tồn tại số nguyên dương $k$ sao cho $b^2-4ac=k^2.$ Giả sử này cho ta
\begin{align*}
    4a\cdot \overline{a b c}&=400 a^{2}+40 a b+4 a c\\&=400 a^{2}+40 a b+b^{2}-k^{2}\\&=(20 a+b+k)(20 a+b-k)
\end{align*}
Do $\overline{a b c}$ là số nguyên tố, một trong hai số $20a+b+k$ và $20a+b-k$ chia hết cho $\overline{abc}.$\\
Tuy nhiên, điều này không xảy ra vì
$$0<20a+b-k<20a+b+k<20a+b+b<100a+10b+c<\overline{abc}.$$
Giả sử phản chứng là sai. Bài toán được chứng minh.}
\end{bx}

\begin{bx} \label{tieuzuongcuoc}
Cho số tự nhiên $n\ge 2$ và số nguyên tố $p$ thỏa mãn $p-1$ chia hết cho $n$ và $n^3-1$ chia hết cho $p.$ Chứng minh rằng $n+p$ là số chính phương.
\nguon{Chuyên Tin Thanh Hóa 2021}
\loigiai{
Để giải bài toán này, ta xét các trường hợp sau.
\begin{enumerate}
    \item  Nếu $n-1$ chia hết cho $p,$ ta đặt $n=lp+1,$ với $l$ là số tự nhiên. Kết hợp giả thiết $n\mid (p-1),$ ta có
        $$(lp+1)\mid (p-1)\Rightarrow p-1\ge lp+1\Rightarrow (l-1)p\le -2\Rightarrow l=0,p=2.$$
        Với $l=0,p=2,$ ta tìm ra $n=1,$ trái giả thiết $n\ge 2.$
    \item Nếu $n^2+n+1$ chia hết cho $p,$ do giả thiết $n\mid (p-1),$ ta có thể đặt $p=kn+1,$  thế thì
        $$n^2+n+1=n^2+n-kn+kn+1=n(n-k+1)+kn+1$$
        là bội của $kn+1.$ Do $(n,kn+1)=1$ nên $(kn+1)\mid(n-k+1).$ Bằng dãy đánh giá
        $$-kn-1\le-k-1<n-k+1\le n\le kn<kn+1,$$
        ta chỉ ra $k=n+1,$ tức là $p=n^2+n+1.$
        Như vậy, $n+p=(n+1)^2$ là số chính phương.
\end{enumerate}
Chứng minh hoàn tất.}
\end{bx}

\subsubsection*{Bài tập tự luyện}

\begin{btt}
Cho $p$ là số nguyên tố và các số nguyên dương $a,b$ thỏa mãn
$$\dfrac{a}{b}=\dfrac{1}{1}+\dfrac{1}{2}+\ldots+\dfrac{1}{p-1}.$$
Chứng minh rằng $a$ chia hết cho $p.$
\nguon{Mở rộng của định lí Wolstenholme}
\end{btt}

\begin{btt}
Chứng minh rằng không thể biểu diễn bất kì một số nguyên tố nào thành tổng bình phương của hai số tự nhiên theo các cách khác nhau.
\end{btt}

\begin{btt}
Cho số tự nhiên $n\ge 2$ và số nguyên tố $p.$ Chứng minh rằng nếu $p-1$ chia hết cho $n$ và $n^6-1$ chia hết cho $p$ thì ít nhất một trong hai số $p-n$ và $p+n$ là số chính phương.
\end{btt}

\subsubsection*{Hướng dẫn tập tự luyện}

\begin{gbtt}
Cho $p$ là số nguyên tố lẻ và các số nguyên dương $a,b$ thỏa mãn
$$\dfrac{a}{b}=\dfrac{1}{1}+\dfrac{1}{2}+\cdots+\dfrac{1}{p-1}.$$
Chứng minh rằng $a$ chia hết cho $p.$
\nguon{Mở rộng của định lí Wolstenholme}
\loigiai{
Từ đẳng thức đã cho, ta có
\begin{align*}
    \dfrac{a}{b}
    &=\left(\dfrac{1}{1}+\dfrac{1}{p-1}\right)+\left(\dfrac{1}{2}+\dfrac{1}{p-2}\right)+\left(\dfrac{1}{3}+\dfrac{1}{p-3}\right)+\cdots+\left(\dfrac{1}{\dfrac{p-1}{2}}+\dfrac{1}{\dfrac{p+1}{2}}\right)
    \\&=p\left(\dfrac{1}{1\cdot (p-1)}+\dfrac{1}{2\cdot(p-2)}+\cdots+\dfrac{1}{\dfrac{p-1}{2} \cdot\dfrac{p+1}{2}}\right).
\end{align*}
Như vậy, tồn tại số nguyên dương $c$ thỏa mãn $$\dfrac{a}{b}=\dfrac{pc}{1\cdot2\cdot3 \cdots (p-1)}.$$ 
Đẳng thức kể trên tương đương với
$$\tron{1\cdot 2\cdot 3\cdots (p-1)}a=pcb.$$
Do $p$ là số nguyên tố nên $p$ không là ước của $1\cdot 2\cdot 3\cdots (p-1),$ điều này chứng tỏ $a$ chia hết cho $p.$ \\
Bài toán được chứng minh.}
\end{gbtt}

\begin{gbtt}
Chứng minh rằng không thể biểu diễn bất kì một số nguyên tố lẻ nào thành tổng bình phương của hai số tự nhiên theo các cách khác nhau.
\loigiai{
Giả sử phản chứng rằng tồn tại các số nguyên dương $a,b,c,d$ sao cho
$$p=a^2+b^2=c^2+d^2,$$
trong đó $a<c<d<b.$ Từ phép đặt kể trên, ta có
\begin{align*}
    p^2=\tron{a^2+b^2}\tron{c^2+d^2}
    &=(ac+bd)^2+(ad-bc)^2
    \\&=(ad+bc)^2+(ac-bd)^2.
    \tag{*}\label{huykhai.stole}
\end{align*}
Ngoài ra, ta còn có
$$(ac+bd)(ad+bc)=\tron{a^2+b^2}cd+\tron{c^2+d^2}ab=p(ab+cd).$$
Như vậy, một trong hai số $ac+bd$ hoặc $ad+bc$ chia hết cho $p.$ Không mất tổng quát, giả sử $ac+bd$ chia hết cho $p.$ Kết hợp với (\ref{huykhai.stole}), ta được $ad=bc.$ Điều này vô lí do $a<c <d<b.$ Giả sử sai khi đó bài toán được chứng minh.}
\end{gbtt}

\begin{gbtt}\label{bodejbmo}
Cho số tự nhiên $n\ge 2$ và số nguyên tố $p.$ Chứng minh rằng nếu $p-1$ chia hết cho $n$ và $n^6-1$ chia hết cho $p$ thì ít nhất một trong hai số $p-n$ và $p+n$ là số chính phương.
\loigiai{
Từ giả thiết $p\mid \tron{n^6-1}$, ta chỉ ra $p$ là ước của một trong bốn số $n-1,\ n+1,\ n^2+n+1,\ n^2-n+1.$ Ta xét các trường hợp sau.
\begin{enumerate}
    \item  Nếu $n-1$ chia hết cho $p,$ ta đặt $n=lp+1,$ với $l$ là số tự nhiên. Kết hợp giả thiết $n\mid (p-1),$ ta có
        $$(lp+1)\mid (p-1)\Rightarrow p-1\ge lp+1\Rightarrow (l-1)p\le -2\Rightarrow l=0,\ p=2.$$
        Với $l=0,p=2,$ ta tìm ra $n=1,$ trái giả thiết $n\ge 2.$
    \item  Nếu $n+1$ chia hết cho $p,$ ta đặt $n=ap-1,$ với $a$ là số tự nhiên. Kết hợp giả thiết $n\mid (p-1),$ ta có
        $$(ap-1)\mid (p-1)\Rightarrow p-1\ge ap-1\Rightarrow (a-1)p\le0.$$
    Ta suy ra $a=1$ và $n=p-1,$ khi đó $p-1=n$ là số chính phương.   
    \item Nếu $n^2+n+1$ chia hết cho $p,$ do giả thiết $n\mid (p-1),$ ta có thể đặt $p=kn+1,$  thế thì
        $$n^2+n+1=n^2+n-kn+kn+1=n(n-k+1)+kn+1$$
        là bội của $kn+1.$ Do $(n,kn+1)=1$ nên $(kn+1)\mid(n-k+1).$ Bằng dãy đánh giá
        $$-kn-1\le-k-1<n-k+1\le n\le kn<kn+1,$$
        ta chỉ ra $k=n+1,$ tức là $p=n^2+n+1.$
        Như vậy, $n+p=(n+1)^2$ là số chính phương.
    \item Nếu $n^2-n+1$ chia hết cho $p,$ do giả thiết $n\mid (p-1),$ ta có thể đặt $p=kn+1,$ thế thì
        $$n^2-n+1=n^2-n-kn+kn+1=n(n-k-1)+kn+1$$
        là bội của $kn+1.$ Do $(n,kn+1)=1$ nên $(kn+1)\mid(n-k-1).$ Bằng dãy đánh giá
        $$-kn-1\le-k-1<n-k-1< n\le kn<kn+1,$$
        ta chỉ ra $k=n-1,$ tức là $p=n^2-n+1.$
        Như vậy, $p-n=(n-1)^2$ là số chính phương.    
\end{enumerate}
Tóm lại, bài toán được chứng minh trong mọi trường hợp.}
\end{gbtt}


\subsection{Về một bổ đề với hợp số}

Dưới đây là một bổ đề rất đẹp về hợp số và có nhiều ứng dụng
\begin{light}
Cho $a,b,c,d$ là $4$ số nguyên dương thỏa mãn $ab=cd$. Khi đó
$a+b+c+d$ là một hợp số.
\end{light} 
\cm{
\begin{enumerate}[\bfseries \sffamily \color{tuancolor} Cách 1.]
    \item Giả sử $a+b+c+d=p$ là một số nguyên tố. Khi đó, từ giả thiết, ta có
$$cp=a c+b c+c^{2}+c d=a c+b c+c^{2}+a b=(a+c)(b+c).$$
Ta suy ra $p$ là ước của một trong hai số $a+c$ và $b+c$. Tuy nhiên, điều này không thể xảy ra, do 
$$0<a+c<p,\quad 0<b+c<p.$$
Giả sử phản chứng là sai nên bài toán được chứng minh. 
    \item Ta đặt $x=(a,c).$ Lúc này, tồn tại các số nguyên dương $t,z$ thỏa mãn $$(t,z)=1,a=xt,c=xz.$$ Kết hợp với $ab=cd,$ phép đặt này cho ta
    $xt.b=xz.d,$
    hay là 
    $bx=dz.$ \\
    Ta nhận thấy $t\mid dz,$ nhưng do $(t,z)=1$ nên $t\mid d.$ Tiếp tục đặt $d=yt,$ ta được $b=yz.$ Bằng các cách đặt như vậy, ta chỉ ra được sự tồn tại của các số nguyên dương $x,y,z,t$ sao cho $$a=xt,b=yz,c=xz,d=yt.$$
    Theo đó $a+b+c+d=xy+yz+xz+yt=(x+t)(y+z)$ là hợp số. Chứng minh hoàn tất.
\end{enumerate}}
\begin{light}
Bạn đọc có thể tham khảo một vài ứng dụng của bổ đề, nằm ở các bài toán phần tự luyện.
\end{light}

\subsubsection*{Bài tập tự luyện}

\begin{btt}
Cho $a,b,c,d$ là $4$ số nguyên dương thỏa mãn $ab=cd.$ Chứng minh rằng
$$a^n+b^n+c^n+d^n$$ là hợp số.
\end{btt}

\begin{btt}
Cho $6$ số nguyên dương $a,b,c,d,e,f$ thỏa mãn $abc=def.$ 
Chứng minh rằng $$a\tron{b^2+c^2}+d\tron{e^2+f^2}$$ là hợp số.
\nguon{Tạp chí Toán học và Tuổi trẻ số 346}
\end{btt}

\begin{btt}
Cho $a,b,c$ là các số nguyên dương. Chứng minh rằng $a+b+2\sqrt{ab+c^{2}}$ không phải là một số nguyên tố.
\nguon{Chuyên Toán Hà Nội 2017}
\end{btt}

\begin{btt}
Cho các số nguyên dương $a,b,c$ thỏa mãn $a^2 - bc$ là số chính phương. Chứng minh rằng $2a + b + c$ là một hợp số.
\nguon{Eye Level Math Olympiad 2019}
\end{btt}

\begin{btt}
Tìm các số tự nhiên $a,b$ thỏa mãn $a^3+3=b^2$ và $a^2+2(a+b)$ là một số nguyên tố.
\nguon{Chọn đội tuyển Khoa học Tự nhiên 2019}
\end{btt}

\subsubsection*{Hướng dẫn bài tập tự luyện}

\begin{gbtt}
Cho $a,b,c,d$ là $4$ số nguyên dương thỏa mãn $ab=cd.$ Chứng minh rằng
$$a^n+b^n+c^n+d^n$$ là hợp số.
\loigiai{
Theo như kết quả đã biết, ta nhận thấy tồn tại các số nguyên dương $x,y,z,t$ sao cho
$$a=xt,\ b=yz,\ c=xz,\ d=yt.$$
Sự tồn tại này cho ta biết
$$a^n+b^n+c^n+d^n=x^nt^n+y^nz^n+x^nz^n+y^nt^n=\left(x^n+t^n\right)\left(y^n+z^n\right).$$
Do $x^n+t^n\ge x+y\ge 2$ và $y^n+z^n\ge y+z\ge 2$ nên số bên trên là hợp số. Bài toán được chứng minh.}
\end{gbtt} 

\begin{gbtt}
Cho $6$ số nguyên dương $a,b,c,d,e,f$ thỏa mãn $abc=def.$ 
Chứng minh rằng $$a\tron{b^2+c^2}+d\tron{e^2+f^2}$$ là hợp số.
\nguon{Tạp chí Toán học và Tuổi trẻ số 346}
\loigiai{Từ giả thiết, ta có $\left(a b^{2}\right)\left(a c^{2}\right)=\left(d e^{2}\right)\left(d f^{2}\right).$ Theo như kết quả đã biết, ta suy ra $$ab^2+ac^2+de^2+df^2=a\left(b^2+c^2\right)+d\left(e^2+f^2\right)$$ là hợp số. Bài toán được chứng minh.}
\end{gbtt}

\begin{gbtt}
Cho $a,b,c$ là các số nguyên dương. Chứng minh rằng $a+b+2\sqrt{ab+c^{2}}$ không phải là một số nguyên tố.
\nguon{Chuyên Toán Hà Nội 2017}
\loigiai{Đặt $a b+c^{2}=d^{2}$ với $d$ là số nguyên dương. Phép đặt này cho ta
$$ab=(d-c)(d+c).$$
Theo kết quả đã biết thì $a+b+d-c+d+c=a+b+2d$ là hợp số. Bài toán được chứng minh.}
\end{gbtt}

\begin{gbtt}
Cho các số nguyên dương $a,b,c$ thỏa mãn $a^2 - bc$ là số chính phương. Chứng minh rằng $2a + b + c$ là một hợp số.
\nguon{Eye Level Math Olympiad 2019}
\loigiai{
Từ giả thiết, ta có thể đặt $a^2-bc=x^2,$ trong đó $x$ là một số tự nhiên. Ta có
$$bc=(a-x)(a+x).$$
Do cả $b,c,x-a,x+a$ đều nguyên dương nên theo bổ đề đã học, số $$b+c+(a-x)+(a+x)=b+c+2a$$ là hợp số. Bài toán đã cho được chứng minh.}
\end{gbtt}

\begin{gbtt}
Tìm các số tự nhiên $a,b$ thỏa mãn $a^3+3=b^2$ và $a^2+2(a+b)$ là một số nguyên tố.
\nguon{Chọn đội tuyển Khoa học Tự nhiên 2019}
\loigiai{
Từ điều kiện $a^3+3=b^2,$ ta có
$$a^3-1=b^2-4\Leftrightarrow (a-1)\left(a^2+a+1\right)=(b-2)(b+2).$$
Tới đây, ta xét các trường hợp sau.
\begin{enumerate}
    \item Nếu $a\ge 2,$ ta có $b-2>0,$ và lúc này
    $$a-1+a^2+a+1+b-2+b+2=a^2+2a+2b$$
    là hợp số, mâu thuẫn.
    \item Nếu $a=1,$ ta có $b=2.$ Lúc này, $a^2+2(a+b)=7$ là số nguyên tố.
    \item Nếu $a=0,$ ta không tìm được $b$ tự nhiên.
\end{enumerate}
Kết luận, $(a,b)=(1,2)$ là cặp số duy nhất thỏa yêu cầu.}
\end{gbtt}

\subsection{Đồng dư thức với modulo nguyên tố}

\subsubsection*{Ví dụ minh họa}

\begin{bx}
Tìm tất cả các số nguyên tố $p$ có dạng $p=a^2+b^2+c^2$ với $a,b,c$ nguyên dương thỏa mãn $a^4+b^4+c^4$ chia hết cho $p.$
\nguon{Chuyên Đại học Sư phạm Hà Nội 2012}
\loigiai{
Không mất tính tổng quát, ta giả sử $c=\max\{a;b;c\}.$ Với các số $a,b,c,p$ thỏa mãn yêu cầu, ta có
\begin{align*}
    a^4+b^4+c^4\equiv 0\pmod{p}
    &\Rightarrow a^4+b^4+\tron{-a^2-b^2}^2\equiv 0\pmod{p}
    \\&\Rightarrow 2a^4+2b^4+2a^2b^2\equiv 0\pmod{p}
    \\&\Rightarrow 2\tron{a^2-ab+b^2}\tron{a^2+ab+b^2}\equiv 0\pmod{p}
    \\&\Rightarrow p\mid 2\tron{a^2-ab+b^2}\tron{a^2+ab+b^2}
\end{align*}
Phần còn lại của bài toán, ta chia làm ba trường hợp sau.
\begin{enumerate}
    \item Nếu $2$ chia hết cho $p,$ ta có $p\le 2,$ vô lí.
    \item Nếu $a^2-ab+b^2$ chia hết cho $p,$ ta có
    $$a^2-ab+b^2\ge a^2+b^2+c^2>a^2-ab+b^2,$$
    đây là điều không thể xảy ra.
    \item Nếu $a^2+ab+b^2$ chia hết cho $p,$ do $c=\max\{a;b;c\},$ ta có 
    $$a^2+ab+b^2\ge a^2+b^2+c^2\ge a^2+b^2+ab.$$
    Dấu bằng trong đánh giá trên phải xảy ra, tức là $a=b=c.$\\
    Lúc này $p=3a^2$ là số nguyên tố nên $a=1,p=3.$
\end{enumerate}
Như vậy, $p=3$ là số nguyên tố duy nhất thỏa yêu cầu bài toán.}
\end{bx}

\begin{bx}
Cho $p$ là một số nguyên tố lẻ, còn $a,b$ và $c$ là các số nguyên dương phân biệt thỏa mãn
$$ab+1\equiv bc+1\equiv ca+1\equiv 0\pmod{p}.$$
Chứng minh rằng với mọi bộ số $(a,b,c,p)$ như vậy, ta luôn có
\[p+2 \leq \frac{a+b+c}{3}.\]
\nguon{Dutch Mathematical Olympiad 2014}
\loigiai{Xét hiệu theo vế các đồng dư thức trong giả thiết, ta thu được
\[\heva{ab+1-ca-1\equiv0\pmod{p}\\ca+1-bc-1\equiv0\pmod{p}}
\Rightarrow
\heva{a\tron{b-c}\equiv 0\pmod{p}\\ c\tron{a-b}\equiv0\pmod{p}.}\tag{1}\label{dutch1}\]
Ngoài ra, từ giả thiết, ta còn nhận thấy rằng
\[\tron{b,p}=\tron{c,p}=1.\tag{2}\label{dutch2}\]
Kết hợp (\ref{dutch1}) và (\ref{dutch2}), ta thu được 
$p\mid \tron{b-c}, \: p\mid \tron{a-b},$ và do $a,b,c$ phân biệt nên
$$p\le |b-c|,\qquad p\le |a-b|.$$
Không mất tính tổng quát, ta giả sử $a\ge b\ge c.$ Giả sử bên trên giúp ta chỉ ra
$$p\le b-c, \qquad p\le a-b.$$
Vì $b\ge 1$ nên $b-c\ge p+1$ và $a-b\ge p+1$. Các đánh giá theo hiệu ấy cho ta biết
$$b\ge c+p+1\ge p+2,\qquad a\ge b+p+1\ge p+3,$$
vậy nên $a+b+c\ge 3p+6.$
Đẳng thức xảy ra khi và chỉ khi $(a,b,c)=(1,p+1,2p+1)$ và các hoán vị. Bài toán được chứng minh.}
\end{bx}

\subsubsection*{Bài tập tự luyện}

\begin{btt}
Cho các số tự nhiên $m, n$ thỏa mãn $m+n+1$ là một số nguyên tố và là ước của $2\left(m^2+n^2\right)-1$. Chứng minh rằng $m=n.$
\nguon{Switzerland Final Round 2010}
\end{btt}

\begin{btt}
Tìm các số nguyên dương ${m}$ và ${n}$ sao cho ${p}={m}^{2}+{n}^{2}$ là số nguyên tố và
${m}^{3}+{n}^{3}-4$ chia hết cho ${p}$
\nguon{Olympic 30/4 khối 10 năm 2013}
\end{btt}

\begin{btt}
Tìm tất cả các số nguyên dương $s\ge 4$ sao cho tồn tại các số nguyên dương $a,b,c,d$ thỏa mãn $s=a+b+c+d$ và $abc+bcd+cda+bad$ chia hết cho $s.$
\end{btt}

\begin{btt}
Cho số nguyên tố $p>7$ và các số nguyên dương $a,b,c,d$ phân biệt, nhỏ hơn $p-1$ thỏa mãn $a+d-b-c,ab-c-d,cd-a-b$ đều chia hết cho $p.$ Tìm số dư của $ac+bd$ khi chia cho $p.$
\nguon{Trường thu Trung du Bắc Bộ 2018}
\end{btt}

\begin{btt}
Cho số nguyên tố $p$ và ba  nguyên dương $x, y, z$ mãn $x<y<z<p$ Chứng minh rằng nếu $x^{3} \equiv y^{3} \equiv z^{3} \pmod{p}$ thì $x^2+y^2+z^2$ chia hết cho $x+y+z.$
\nguon{Duyên hải Bắc Bộ 2016}
\end{btt}

\begin{btt}
Cho các số nguyên dương $a,b,c,d,e$ phân biệt và số nguyên tố $p$ thỏa mãn các số
$$abc+1,bcd+1,cde+1,dea+1,eab+1$$
đều chia hết cho $p.$ Chứng minh rằng
\[a+b+c+d+e\ge 10p+5.\]
\end{btt}

\begin{btt}
Tìm tất cả các số nguyên dương $m,n$ và số nguyên tố $p$ thỏa mãn $$\left(m^3+n\right)\left(m+n^3\right)=p^3.$$
\nguon{Turkish Team Selection Test 2017}
\end{btt}

\subsubsection*{Hướng dẫn bài tập tự luyện}

\begin{gbtt}
Cho các số tự nhiên $m, n$ thỏa mãn $m+n+1$ là một số nguyên tố và là ước của $2\left(m^2+n^2\right)-1$. Chứng minh rằng $m=n.$
\nguon{Switzerland Final Round 2010}
\loigiai{
Với các số nguyên dương $m,n$ thỏa yêu cầu, ta có
\begin{align*}
    2m^2+2n^2-1\equiv 0\pmod{m+n+1}
    &\Rightarrow 2m^2+2\tron{-m-1}^2-1
    \equiv 0\pmod{m+n+1}
    \\&\Rightarrow (2m+1)^2\equiv 0\pmod{m+n+1}   
    \\&\Rightarrow (m+n+1)\mid\tron{2m+1}^2.
\end{align*}
Do $m+n+1$ là số nguyên tố nên $m+n+1$ cũng là ước của $2m+1.$ Dựa vào so sánh
$$2m+1<2(m+n)<2(m+n+1),$$
ta chỉ ra $2m+1=m+n+1,$ tức là $m=n.$ Bài toán được chứng minh.}
\end{gbtt}

\begin{gbtt}
Tìm các số nguyên dương ${m}$ và ${n}$ sao cho ${p}={m}^{2}+{n}^{2}$ là số nguyên tố và
${m}^{3}+{n}^{3}-4$ chia hết cho ${p}$
\nguon{Olympic 30/4 khối 10 năm 2013}
\loigiai{
Đặt $m+n=S,mn=P,m^2+n^2=p.$ Với các số nguyên dương $m,n$ thỏa yêu cầu, ta có
\begin{align*}
    \tron{S^2-2P}\mid\tron{S^3-3SP-4}
    &\Rightarrow S^3\equiv 3SP+4\pmod{S^2-2P}
    \\&\Rightarrow 2S^3\equiv 6SP+8\pmod{S^2-2P}
    \\&\Rightarrow 2S^3\equiv 3S^3+8\pmod{S^2-2P} 
    \\&\Rightarrow S^3+8\equiv0\pmod{S^2-2P}      
    \\&\Rightarrow (S+2)\tron{S^2-2S+4}\equiv0\pmod{S^2-2P}        
    \\&\Rightarrow \tron{S^2-2P}\mid(S+2)\tron{S^2-2S+4}.
    \\&\Rightarrow p\mid ({m}+{n}+2)\left[{m}^{2}+{n}^{2}+2 {mn}-2({m}+{n})+4\right]    
\end{align*}
Phần còn lại của bài toán, ta chia làm hai trường hợp sau.
\begin{enumerate}
    \item Nếu $m+n+2$ chia hết cho $p$, ta có 
    $$m+n+2\ge m^2+n^2\Rightarrow m(m-1)+n(n-1)\le 2.$$
    Với cú ý rằng ${m}$ và ${n}$ là các số nguyên dương, ta có
$${m}({m}-1)+{n}({n}-1) \leq 2 \Rightarrow \hoac{
m(m-1)+n(n-1)=2  \\
m(m-1)+n(n-1)=1  \\
m(m-1)+n(n-1)=0 }
\Rightarrow \hoac{
{m}=1,\: {n}=2 \\
{m}=2,\: {n}=1 \\
{m}=1,\: {n}=1.}$$
    \item Nếu $p \mid \left[{m}^{2}+{n}^{2}+2 {mn}-2({m}+{n})+4\right],$ ta có 
    $$\left({m}^{2}+{n}^{2}\right) \mid [2mn-2(m+n)+4].$$
    Tới đây, ta nhận thấy rằng
    $2mn-2m-2n+4\ge (m+n)^2-2mn,$
    và thế thì
    $$-2m-2n+4\ge (m+n)^2-4mn=(m-n)^2\ge 0.$$
    Bằng cách chặn $m$ và $n$ như trên, ta chỉ ra
    $$\heva{({m}+{n})^{2}=4 {mn} \\ 2({m}+{n})=4} \Rightarrow \heva{{m}=1 \\ {n}=1.}$$
\end{enumerate}
Kiểm tra trực tiếp, ta thấy tất cả các cặp số $(m,n)$ cần tìm là $(1,1),(1,2)$ và $(2,1).$}
\end{gbtt}

\begin{gbtt}
Tìm tất cả các số nguyên dương $s\ge 4$ sao cho tồn tại các số nguyên dương $a,b,c,d$ thỏa mãn $s=a+b+c+d$ và $abc+bcd+cda+bad$ chia hết cho $s.$
\loigiai{Trước hết, tất cả các hợp số đều thỏa yêu cầu. Thật vậy, nếu $s$ là hợp số, ta viết $s=xy$ và chọn
$$a=1,\quad b=x-1,\quad c=y-1,\quad d=(x-1)(y-1).$$ Khi đó,  $abc+bcd+cda+bad$ chia hết cho $s$ vì
$$a b c+a b d+a c d+b c d=x y(x-1)(y-1).$$
Bây giờ, ta sẽ chứng minh tất cả các hợp số đều không thỏa yêu cầu.\\ Giả sử $s$ là hợp số, thế thì do $d\equiv -a-b-c\pmod{s}$ nên là
\begin{align*}
0 & \equiv a b c-(a+b+c)(a b+b c+ca) \pmod{s} \\
& \equiv-\left(a^{2} b+a b^{2}+c^{2}a+ca^{2}+b^{2} c+b c^{2}+2 a b c\right) \pmod{s} \\
& \equiv-(a+b)(b+c)(c+a) \pmod{s}.
\end{align*}
Do $s$ nguyên tố nên một trong $a+b,\ b+c$ và $c+a$ chia hết cho $s,$ nhưng điều này vô lí do cả $3$ số này đều nhỏ hơn $s.$ Như vậy, tất cả các số $s$ thỏa yêu cầu là hợp số.}
\end{gbtt}

\begin{gbtt}
Cho số nguyên tố $p>7$ và các số nguyên dương $a,b,c,d$ phân biệt, nhỏ hơn $p-1$ thỏa mãn $a+d-b-c,\ ab-c-d,\ cd-a-b$ đều chia hết cho $p.$ Tìm số dư của $ac+bd$ khi chia cho $p.$
\nguon{Trường thu Trung du Bắc Bộ 2018}
\loigiai{
Từ giả thiết, ta chỉ ra các đồng dư thức dưới đây
\begin{align}
    a+d\equiv b+c&\pmod{p},\tag{1}\label{bai1.hung.1}\\
ab\equiv c+d&\pmod{p},\tag{2}\label{bai1.hung.2}\\
cd\equiv a+b&\pmod{p}.\tag{3}\label{bai1.hung.3}
\end{align}
Lấy hiệu theo vế của (\ref{bai1.hung.2}) và (\ref{bai1.hung.3}), ta được
\[ab-cd\equiv c+d-a-b\pmod{p}.\]
Chuyển vế rồi cộng thêm $1,$ đồng dư thức kể trên tương đương với
\[(a+1)(b+1)\equiv (c+1)(d+1)\pmod{p}.\tag{4}\label{bai1.hung.4}\]
Ta đặt $x=a+1,y=b+1,z=c+1,t=d+1.$ Phép đặt này kết hợp (\ref{bai1.hung.1}) và (\ref{bai1.hung.4}) cho ta
$$x-y\equiv z-t \pmod{p},\quad xy\equiv zt\pmod{p}.$$
Áp dụng phép thế đồng dư $t\equiv z+y-x\pmod{p}$ vào $xy\equiv zt\pmod{p},$ ta được
$$xy\equiv z(z+y-x)\pmod{p}\Rightarrow (x-z)(y+z)\equiv 0\pmod{p}.$$
Tới đây, ta xét các trường hợp sau.
\begin{enumerate}
    \item Nếu $y-z$ chia hết cho $p,$ ta có
    $p\le y-z=b-c<b<p-1,$ vô lí.
    \item Nếu $y+z$ chia hết cho $p,$ ta có $p\le y+z< 2p,$ và vì thế 
    $$p=y+z=b+c+2.$$
    Tương tự, ta cũng chỉ ra $p=a+d+2.$ Kết hợp $p=b+c+2=a+d+2$ với (\ref{bai1.hung.2}), ta được
    $$ab\equiv (p-b-2)+(p-a-2)\pmod{p}\Rightarrow ab+a+b+4\equiv 0\pmod{p}.$$
    Như vậy
    \begin{align*}
        ac+bd&=a(p-b-2)+b(p-a-2)\\&\equiv -a(b+2)-b(a+2)\\&\equiv-2ab-2a-2b\\&\equiv 8 \pmod{p}.
    \end{align*}
\end{enumerate}
Kết luận, số dư của phép chia $ac+bd$ cho $p$ là $8.$}
\end{gbtt}

\begin{gbtt}
Cho số nguyên tố $p$ và ba  nguyên dương $x, y, z$ mãn $x<y<z<p$. Chứng minh rằng nếu $x^{3} \equiv y^{3} \equiv z^{3} \pmod{p}$ thì $x^2+y^2+z^2$ chia hết cho $x+y+z.$
\nguon{Duyên hải Bắc Bộ 2016}
\loigiai{Từ giả thiết $x^{3} \equiv y^{3} \equiv z^{3} \pmod{p},$  ta có 
\[\heva{&p\mid \left(y^3-x^3\right)\\ &p\mid \left(z^3-y^3\right)}
\Rightarrow
\heva{&p\mid \left(y-x\right)\left(x^2+xy+y^2\right)\\ &p\mid (z-y)\left(y^2+yz+z^2\right).}\]
Dựa theo so sánh $0<y-x<y<p$ và $0<z-y<z<p,$ ta suy ra $(y-x,p)=(z-y,p)=1.$ Từ đây,
\[
\begin{aligned}
\heva{&p\mid \left(x^2+xy+y^2\right)\\ &p\mid \left(y^2+yz+z^2\right)}
&\Rightarrow
p\mid \left(x^2+xy+y^2\right)- \left(y^2+yz+z^2\right)
\\&\Rightarrow
p\mid (x-z)(x+y+z). \end{aligned}
\label{dhbb15.2}\]
Dựa theo so sánh $0<z-x<z<p,$ ta suy ra $(z-x,p)=1.$ Ta lại suy ra $p\mid (x+y+z).$ Tuy nhiên, do
$$0<x+y+z<p+p+p<3p$$
nên $x+y+z=p$ hoặc $x+y+z=2p.$
Ngoài ra, việc kết hợp $x+y+z$ chia hết cho $p$ và $x^2+xy+y^2$ chia hết cho $p$ còn cho ta
\[\begin{aligned}
\heva{&(x+y)^2\equiv xy \pmod{p} \\ &x+y\equiv -z\pmod{p}}
&\Rightarrow 
\heva{&z^2\equiv xy \pmod{p} \\ &x^2+xy+y^2\equiv 0\pmod{p}}
\\&\Rightarrow
p\mid \left(x^2+y^2+z^2\right).
\end{aligned}\]
Ta xét hai trường hợp sau đây
\begin{enumerate}
    \item Nếu ${x}+{y}+{z}={p},$ ta có ngay $x^2+y^2+z^2$ chia hết cho $x+y+z.$
    \item Nếu ${x}+{y}+{z}=2p,$ ta có $x^2+y^2+z^2$ chia hết cho $\dfrac{x+y+z}{2}.$ \\
    Ta sẽ chứng minh $p\ne 2.$ Thật vậy, nếu $p=2,$ ta có $x<y<z<2,$ mâu thuẫn điều kiện $x,y,z$ nguyên dương. Như vậy, $p$ phải là số nguyên tố lẻ và $x^2+y^2+z^2$ chia hết cho $x+y+z.$
\end{enumerate}
Tổng kết lại, bài toán được chứng minh.}
\end{gbtt}

\begin{gbtt}
Cho các số nguyên dương $a,b,c,d,e$ phân biệt và số nguyên tố $p$ thỏa mãn các số
$$abc+1,bcd+1,cde+1,dea+1,eab+1$$
đều chia hết cho $p.$ Chứng minh rằng
\[a+b+c+d+e\ge 10p+5.\]
\loigiai{
Do cả $abc+1$ và $bcd+1$ cùng chia hết cho $p$ nên
$$p\mid\tron{abc+1-bcd-1}=bc\tron{a-d}.$$
Nếu $b$ hoặc $c$ chia hết cho $p$ thì $1$ chia hết cho $p.$ Điều này không thể xảy ra. Do đó, $p$ phải là ước của $a-d.$ Chứng minh hoàn toàn tương tự, ta có
$$p\mid\tron{b-e},\quad p\mid\tron{c-a},\quad p\mid \tron{d-b}.$$
Nói cách khác, các số $a,b,c,d,e$ có cùng số dư khi chia cho $q.$ Đặt $$a=q_1p+r,\quad b=q_2p+r,\ldots,\quad e=q_5p+r,$$ trong đó $q_1,q_2,\quad\ldots,q_5$ là số tự nhiên đôi một khác nhau và $r\in \left\{1;2;\ldots;p-1\right\}.$ Từ đây, ta suy ra
$$a+b+c+d+e=q_1+q_2+\ldots+q_5+5r\ge 0p+p+2p+3p+4p+5=10p+5.$$
Dấu bằng xảy ra chẳng hạn tại
$$(a,b,c,d,e)=(p+1,2p+1,3p+1,4p+1,5p+1).$$
Bất đẳng thức đã cho được chứng minh.}
\end{gbtt}

\begin{gbtt}
Tìm tất cả các số nguyên dương $m,n$ và số nguyên tố $p$ thỏa mãn $$\left(m^3+n\right)\left(m+n^3\right)=p^3.$$
\nguon{Turkish Team Selection Test 2017}
\loigiai{
Không mất tính tổng quát, ta giả sử $m\ge n.$ Giả sử này cho ta
$2\le m+n^3<m^3+n.$\\
Cả $m+n^3$ và $m^3+n$ đều là lũy thừa cơ số $p,$ vậy nên
\begin{align}
m+n^3&=p,\tag{1}\label{turkey17.1}\\
m^3+n&=p^2.\tag{2}\label{turkey17.2}
\end{align}

Ta dễ thấy $p>m\ge n$ và $p>n^3.$ Ngoài ra, từ (\ref{turkey17.1}), ta có $m\equiv -n^3\pmod{p}.$ Kết hợp với (\ref{turkey17.2}), ta chỉ ra
$$n\equiv -m^3\equiv -\left(-n^3\right)^3=n^9\pmod{p}.$$
Thực hiện chuyển vế rồi phân tích nhân tử, ta thu được
$$p\mid \left(-n^9+n\right)=-n(n-1)(n+1)\left(n^2+1\right)\left(n^4+1\right).$$
Nếu $n=1,$ thử lại, ta chỉ ra $(m,n,p)=(2,1,3).$ Nếu $n\ge 2,$ ta xét các trường hợp đưới dây.
\begin{enumerate}
    \item Với $p\mid n,$ ta có $p\le n<p,$ mâu thuẫn.
    \item Với $p\mid (n-1),$ ta có $p\le n-1<p-1,$ mâu thuẫn.
    \item Với $p\mid (n+1),$ ta có $n+1\ge p>n^3>n+1,$ mâu thuẫn.
    \item Với $p\mid \left(n^2+1\right),$ ta có $n+1\ge p>n^3>n^2+1,$ mâu thuẫn.
    \item Với $p\mid \left(n^4+1\right),$ kết hợp với (\ref{turkey17.1}), ta có
    $$p\mid n\left(m+n^3\right)-\left(n^4+1\right)=mn.$$
    Bắt buộc, $mn\ge p.$ Tuy nhiên, lấy tích theo vế của (\ref{turkey17.1}) và (\ref{turkey17.2}), ta nhận thấy rằng
    $$p^3=\left(m+n^3\right)\left(m^3+n\right)>n^3m^3.$$
    Lấy căn bậc ba, ta suy ra $p>mn,$ mâu thuẫn.
\end{enumerate}
Tổng kết lại, có hai bộ $(m,n,p)$ thỏa yêu cầu bài toán, đó là $(1,2,3)$ và $(2,1,3).$}
\begin{luuy}
Trong bài toán trên, ta đã sử dụng phép thế đồng dư từ $m\equiv -n^3\pmod{p}$ vào $m^3\equiv -n\pmod{p}.$ Nhờ phép thế này, ta chỉ ra được 
$$p\mid \left(n^9-n\right).$$
Hướng đi phân tích nhân tử $n^9-n$ ở phía sau giúp ta xét các trường hợp về bội của $p.$
\end{luuy}
\end{gbtt}

\subsection{Ứng dụng trong tìm số nguyên tố thỏa mãn phương trình cho trước}

\subsubsection*{Ví dụ minh họa}

\begin{bx}
Tìm tất cả các số nguyên tố $p$ và số tự nhiên $n$ thỏa mãn
\[n^3-3n+3=(p-1)^2.\]
\loigiai{
Giả sử tồn tại số nguyên tố $p$ và số tự nhiên $n$ thỏa mãn yêu cầu. Giả sử này cho ta
$$n^3-3n+2=p^2-2p\Rightarrow (n-1)^2(n+2)=p(p-2).$$
Do $p$ là số nguyên tố, một trong hai số $n-1$ và $n+2$ chia hết cho $p,$ và như vậy
$$\hoac{&n-1\ge p \\ &n+2\ge p}\Rightarrow n\ge p-2.$$
Nhận xét trên kết hợp với đẳng thức $(n-1)^2(n+2)=p(p-2)$ chỉ ra
$$p(p-2)=(n-1)^2(n+2)\ge (p-3)^2p.$$
Rút gọn, ta được $(p-3)^2\le p-2,$ và bất đẳng thức này đổi dấu khi $p\ge 3.$ Như vậy, ta có $p=2,$ và kéo theo $n=1.$ Kết luận, $(n,p)=(1,2)$ là cặp số duy nhất thỏa mãn yêu cầu.}
\end{bx}

\begin{bx}
Tìm tất cả các số nguyên tố $p,q,r$ thỏa mãn
\[p(p+1)+q(q+1)=r(r+1).\]
\nguon{Kazakhstan Mathematical Olympiad 2007}
\loigiai{
Ta giả sử tồn tại các số $p,q,r$ thỏa yêu cầu, trong đó $p\ge q.$ Ta dễ dàng nhận được $r>p,$ đồng thời
\begin{align*}
    r(r+1)&\le 2p(p+1)<2p(2p+1)\Rightarrow r<2p.
\end{align*}
Ngoài ra, phương trình đã cho tương đương
\[(r-q)(r+q+1)=p(p+1).\tag{*}\] 
Do $p$ nguyên tố nên đến đây, ta xét hai trường hợp.
\begin{enumerate}
    \item Nếu $p$ là ước của $r-q,$ do $0<r-q<r<2p$ nên ta lần lượt suy ra
    $$r-q=p\Rightarrow r+q+1=p+1\Rightarrow r+q=p.$$
    Điều này mâu thuẫn với việc $p<r.$
    \item Nếu $p$ là ước của $r+q+1,$ do $p<r+q+1<3p+1$ nên $r+q+1=2p$ hoặc $r+q+1=3p.$
    \begin{itemize}
        \item \chu{Trường hợp 1.} Nếu $r+q+1=3p,$ dấu bằng trong đánh giá đầu tiên ở phần lời giải phải xảy ra, tức là $r=2p-1$ và $p=q.$ Thế ngược lại (*), ta có
        $$2p(p+1)=(2p-1)2p.$$
        Ta tìm được $p=2$ từ đây. Trường hợp này cho ta bộ $(p,q,r)=(2,2,3).$
        \item \chu{Trường hợp 2.} Nếu $r+q+1=2p,$ thế trở lại (*) ta được
        $$2(r-q)=p+1\Rightarrow r=\dfrac{p+2q+1}{2}.$$
        Với việc $r=2p-q-1$ và $r=\dfrac{p+2q+1}{2},$ ta nhận thấy rằng $3p=4q+3.$ Lúc này, $4q$ chia hết cho $3,$ lại vì $q$ nguyên tố nên $q=3.$ Kiểm tra lại, ta tìm được $q=5,$ nhưng khi ấy $r=6$
    \end{itemize}
\end{enumerate}
Như vậy $(p,q,r)=(2,2,3)$ là bộ số duy nhất thỏa mãn đề bài.}
\begin{luuy}
\begin{enumerate}
    \item Cách chuyển vế để biến đổi tương đương phương trình chính là mấu chốt của bài toán. Tất cả các bài tập dạng này đều được giải quyết theo cách chuyển phương trình về dạng một vế là tích một vài thừa số, vế còn lại lộ ra ước nguyên tố; rồi sau đó xét các trường hợp riêng lẻ.
    \item Bài toán này hoàn toàn có thể giải với $r$ không nhất thiết là số nguyên tố.
\end{enumerate}
\end{luuy}
\end{bx}

\subsubsection*{Bài tập tự luyện}

\begin{btt}
Tìm tất cả các số nguyên tố $p$ và số tự nhiên $n$ thỏa mãn
\[n^4=2p^2+3p-4.\]
\end{btt}

\begin{btt}
Cho số nguyên tố $p$. Tìm hai số nguyên không âm phân biệt $a,b$ thỏa mãn \[a^4-b^4=p\left(a^3-b^3\right).\]
\nguon{Junior Balkan Mathematical Olympiad Shortlist 2019}
\end{btt}

\begin{btt}
Tìm tất cả các số nguyên tố $p,q$ thỏa mãn $$p\left(p^3+1\right)=q\left(q+2p-1\right).$$
\end{btt}

\begin{btt}
Tìm tất cả các số nguyên tố $p,q$ thỏa mãn $$p^3-q^5=(p+q)^2.$$
\nguon{Saudi Arabia JBMO Training Test 2017}
\end{btt}

\begin{btt}
Tìm tất cả các số nguyên tố $p$ và số nguyên dương $x, y$ thoả mãn
\[\heva{p-1=2x(x+2) \\ p^2-1=2y(y+2).}\]
\nguon{Chuyên Toán Hà Nội 2015}
\end{btt}

\begin{btt}
Cho các số nguyên dương lẻ $a, b, c$. Biết rằng $a-2$ không là số số chính phương, đồng thời
$$a^{2}+a+3=3\left(b^{2}+b+3\right)\left(c^{2}+c+3\right).$$
Chứng minh rằng trong hai số $b^{2}+b+3$ và $c^{2}+c+3$, có ít nhất một số là hợp số.
\nguon{Baltic Way Mathematical Olympiad 2020}
\end{btt}

\begin{btt}
Cho số nguyên tố $p$ và số nguyên dương $n.$ Chứng minh rằng không tồn tại các số nguyên dương $x,y$ thỏa mãn
\[\dfrac{x^2+x}{y^2+y}=p^{2n}.\]
\nguon{Cao Đình Huy}
\end{btt}

\subsubsection*{Hướng dẫn bài tập tự luyện}

\begin{gbtt}
Tìm tất cả các số nguyên tố $p$ và số tự nhiên $n$ thỏa mãn
\[n^4=2p^2+3p-4.\]
\loigiai{
Giả sử tồn tại các số nguyên $p,n$ thỏa yêu cầu. Ta viết lại
\[\tron{n^2-2n+2}\tron{n^2+2n+2}=p(2p+3)\tag{*}\label{sinnoinguytanh}.\]
Do $p$ là số nguyên tố nên một trong hai số $n^2-2n+2$ và $n^2+2n+2$ chia hết cho $p.$
\begin{enumerate}
    \item Nếu $n^2-2n+2$ chia hết cho $p,$ ta lại tiếp tục xét các trường hợp nhỏ hơn.
    \begin{itemize}
        \item \chu{Trường hợp 1.} Nếu $n^2-2n+2\ge 2p,$ hiển nhiên $n^2+2n+2>2p.$ Kết hợp với (\ref{sinnoinguytanh}), ta sẽ có
        $$p(2p+3)>2p\cdot2p=p(4p).$$
        Ta suy ra $2p+3>4p$ hay $2p<3$ từ đây. Không có số nguyên tố nào như vậy.
        \item \chu{Trường hợp 2.} Nếu $n^2-2n+2=p,$ thế trở lại (\ref{sinnoinguytanh}) ta có
        $$\tron{n^2-2n+2}\tron{n^2+2n+2}=\tron{n^2-2n+2}\tron{2n^2-4n+7}.$$
        Ta tìm được $n=5$ và $n=1$ từ đây. Chỉ trường hợp $n=5$ cho ta $p$ nguyên tố, cụ thể là $p=17.$
    \end{itemize}        
    \item Nếu $n^2+2n+2$ chia hết cho $p,$ ta lại tiếp tục xét các trường hợp nhỏ hơn.
    \begin{itemize}
        \item \chu{Trường hợp 1.} Nếu $n^2+2n+2\ge 2p,$ kết hợp với (\ref{sinnoinguytanh}), ta sẽ có
        $$2\tron{n^2-2n+2}<2p+3.$$
        Đánh giá bất đẳng thức trên cho ta
        $$2\tron{n^2-2n+2}<2p+3\le n^2+2n+2+3.$$
        Chỉ có $n=1,2,3,4,5,6$ thỏa mãn bất đẳng thức trên. Không trường hợp nào cho đáp số.
        \item \chu{Trường hợp 2.} Nếu $n^2+2n+2=p,$ thế trở lại (\ref{sinnoinguytanh}) ta có
        $$\tron{n^2-2n+2}\tron{n^2+2n+2}=\tron{n^2+2n+2}\tron{2n^2+4n+7}.$$
        Ta tìm được $n=-5$ và $n=-11$ từ đây, không thỏa $n$ tự nhiên.     
    \end{itemize}
\end{enumerate}
Kết luận, cặp số $(p,n)$ thỏa yêu cầu là $(17,5).$}
\end{gbtt}

\begin{gbtt}
Cho số nguyên tố $p$. Tìm hai số nguyên không âm phân biệt $a,b$ thỏa mãn \[a^4-b^4=p\left(a^3-b^3\right).\]
\nguon{Junior Balkan Mathematical Olympiad Shortlist 2019}
\loigiai{
Đặt $(a,b)=d,$ khi đó tồn tại các số nguyên dương $x,y$ sao cho $(x,y)=1,a=dx,b=dy.$ Ta có
$$d^4\left(x^4-y^4\right)=pd^3\left(x^3-y^3\right).$$
Do $a\ne b$ nên $x\ne y.$ Thực hiện chia cả hai vế phương trình cho $x-y,$ ta được
\[d(x+y)\left(x^2+y^2\right)=p\left(x^2+xy+y^2\right).\tag{*}\label{huyngu}\]
Ta sẽ đi chứng minh các nhận xét sau đây.
\begin{enumerate}
    \item[i,] $\left(x+y,x^2+xy+y^2\right)=1$. Thật vậy, đặt $m=\left(x+y,x^2+xy+y^2\right)=1,$ ta có
    \begin{align*}
    \heva{&m\mid (x+y) \\ &d\mid (x+y)^2-\left(x^2+xy+y^2\right)}
    &\Rightarrow 
    \heva{&m\mid (x+y) \\ &d\mid xy}
    \\&\Rightarrow
    \heva{&m\mid (x+y)x-xy \\ &d\mid (x+y)y-xy}  
    \\&\Rightarrow
    \heva{&m\mid x^2 \\ &d\mid y^2}\\&\Rightarrow m=1.   
    \end{align*}
    \item[ii,] $\left(x^2+y^2,x^2+xy+y^2\right)=1$. Thật vậy, đặt $n=\left(x^2+y^2,x^2+xy+y^2\right),$ ta có
    \begin{align*}
    \heva{&n\mid \left(x^2+y^2\right) \\ &n\mid \left(x^2+xy+y^2\right)}
    &\Rightarrow 
    \heva{&n\mid \left(x^2+y^2\right) \\ &n\mid xy}
    \\&\Rightarrow
    \heva{&n\mid (x-y)^2 \\ &n\mid (x+y)^2}  
   \\& \Rightarrow
    n\mid (x+y,x-y)^2\\&\Rightarrow n\in\{1;2\}.   
    \end{align*}    
    Tuy nhiên, nếu như $n=2$ thì $x^2+y^2$ và $x^2+xy+y^2$ cùng chẵn nên $x,y$ cũng cùng chẵn, mâu thuẫn với $(x,y)=1.$ Do vậy $n=1.$
\end{enumerate}
Dựa vào các nhận xét trên và (\ref{huyngu}), ta chỉ ra $d$ chia hết cho $x^2+xy+y^2.$ Đặt $d=z\left(x^2+xy+y^2\right),$ trong đó $z$ là một số nguyên dương. Thực hiện vào (\ref{huyngu}), ta có
$$z(x+y)\left(x^2+y^2\right)=p.$$
Theo đó, trong hai số $x+y$ và $x^2+xy+y^2,$ phải có ít nhất một số bằng $1.$ Lập luận này cho ta $(x,y)=(1,0)$ hoặc $(x,y)=(0,1),$ và ngoài ra $z=p.$ Các cặp $(a,b)$ thỏa yêu cầu là $(0,p)$ và $(p,0).$}
\end{gbtt}

\begin{gbtt}
Tìm tất cả các số nguyên tố $p,q$ thỏa mãn $p\left(p^3+1\right)=q\left(q+2p-1\right).$
\loigiai{
Xét tính chia hết cho $q$ ở hai vế, ta có $q(q-1)$ chia hết cho $p.$ Xét tính chia hết cho $p$ ở hai vế, ta có $p\tron{p+1}\tron{p^2-p+1}$ chia hết cho $q.$ Ta xét các trường hợp sau.
\begin{enumerate}
    \item Với $q\mid p,$ ta suy ra $p=q.$ Thế trở lại đề bài, ta thu được
    $$p\tron{p^3+1}=p\tron{3p-1}.$$
    Giải phương trình trên, ta thấy không có số nguyên tố $p$ thỏa mãn.
    \item Với $q\mid \tron{p+1}$ và $p\ne q,$ từ $q(q-1)$ chia hết cho $p$ ta suy ra $p\mid (q-1).$ Ta có
    $$\heva{q\mid\tron{p+1}\\p\mid (q-1)}\Rightarrow \heva{p+1\ge q\\q-1\ge p}\Rightarrow p+1\ge q\ge p+1\Rightarrow q=p+1.$$
    Thế trở lại phương trình ban đầu, ta tìm ra $(p,q)=(2,3).$
    \item Với $q\mid \tron{p^2-p+1}$ và $p\ne q,$ ta cũng suy ra $p\mid (q-1).$ 
    Theo như \chu{bài \ref{bodejbmo}}, ta chỉ ra $q=p^2-p+1.$ Thế trở lại phương trình ban đầu, ta thấy thỏa mãn.
\end{enumerate}
Như vậy, các cặp số $(p,q)$ thỏa yêu cầu là $\tron{p,p^2-p+1},$ trong đó $p^2-p+1$ là một số nguyên tố.}
\end{gbtt}

\begin{gbtt}
Tìm tất cả các số nguyên tố $p,q$ thỏa mãn $p^3-q^5=(p+q)^2.$
\nguon{Saudi Arabia JBMO Training Test 2017}
\loigiai{
Giả sử tồn tại các số nguyên tố $p,q$ thỏa mãn đề bài. Ta xét hai trường hợp sau đây
\begin{enumerate}
    \item Với $p=q,$ ta có
    $$p^3-p^5=4p^2\Rightarrow p^5-p^3+4p^2=0\Rightarrow p^3\left(p^2-1\right)+4p^2=0.$$
    Vế trái lớn hơn $0,$ mâu thuẫn.
    \item Với $p\ne q,$ ta biến đổi
        \begin{align*}
        p^3-q^5=p^2+2pq+q^2&\Rightarrow p^3-p^2=q^5+q^2+2pq\\&\Rightarrow p^2(p-1)=q\left(q^4+q+2p\right).
        \end{align*}
    Tới đây, ta thực hiện xét tính chia hết cho $q$ và $p$ ở cả hai vế. Cụ thể
    \begin{itemize}
        \item Vế phải chia hết cho $q$ và $(p,q)=1,$ chứng tỏ $p-1$ chia hết cho $q.$
        \item Vế trái chia hết cho $p$ và $(p,q)=1,$ chứng tỏ $p$ là ước của 
        $$ q^4+q+2p=q(q+1)\left(q^2-q+1\right)+2p.$$
        Một cách tương đương, $q(q+1)\left(q^2-q+1\right)$ chia hết cho $p.$
    \end{itemize}
    Ta sẽ tiếp tục chia bài toán thành các trường hợp nhỏ hơn dựa theo nhận xét thứ hai.
    \begin{itemize}
        \item \chu{Trường hợp 1. }Nếu $q$ chia hết cho $p,p-1$ chia hết cho $q,$ ta có $p-1\ge q\ge p,$ mâu thuẫn.
        \item \chu{Trường hợp 2. } Nếu $q+1$ chia hết cho $p$ và $p-1$ chia hết cho $q,$ ta có $$p\ge q+1\ge p.$$ 
        Ta suy ra $p=q+1.$ Hai số $p$ và $q$ lúc này khác tính chẵn lẻ, nên bắt buộc $q=2$ và $p=3.$ Thử lại, ta thấy không thỏa $p^3-q^5=(p+q)^2.$
        \item \chu{Trường hợp 3. }Nếu $q^2-q+1$ chia hết cho $p$ và $p-1$ chia hết cho $q,$ áp dụng phần nhận xét ở sau \chu{bài \ref{tieuzuongcuoc}}, ta chỉ ra $p=q^2-q+1.$ Thế ngược lại phương trình ban đầu, ta có
        $$\left(q^2-q+1\right)^3-q^5=\left(q^2+1\right)^2\Rightarrow q(q-3)\left(q^2+1\right)\left(q^2-q+1\right)=0.$$
        Do $q$ là số nguyên tố, ta có $q=3,$ và thế thì $p=q^2-q+1=7.$
    \end{itemize}
\end{enumerate}
Kết luận, $(p,q)=(7,3)$ là cặp số nguyên tố duy nhất thỏa yêu cầu.}
\end{gbtt}

\begin{gbtt}
Tìm tất cả các số nguyên tố $p$ và số nguyên dương $x, y$ thoả mãn
\[\heva{p-1=2x(x+2) \\ p^2-1=2y(y+2).}\]
\nguon{Chuyên Toán Hà Nội 2015}
\loigiai{
Lấy hiệu theo vế, ta được
$$p(p-1)=2y^2+4y-2x^2-4x=2(y-x)(y+x+2).$$
Ngoài ra, ta còn thu được so sánh $x<y<p$ từ giả thiết. Thật vậy, nếu $y\ge p,$ ta có
$$p^2-1=2y(y+2)\ge 2p(p+2),$$
một điều mâu thuẫn. Dựa vào các chứng minh trên, ta chia bài toán làm các trường hợp sau.
\begin{enumerate}
    \item Nếu ${p} \mid 2,$ ta có $p=2,$ nhưng khi đó $2x^2+4x=1,$ mâu thuẫn.
    \item Nếu $p \mid y-x$ ta có $p\le y-x<y<p$, mâu thuẫn.
    \item Nếu $p \mid {x}+{y}+2$, ta có 
    $$p\le (x-y)+2y+2\le -1+2(p-1)+2=2p-1<2p,$$ 
    thế nên $x+y+2=p.$ Ta thu được hệ sau
    \begin{align*}
    \heva{
    p&=x+y+2\\
    p-1&=2y-2x\\
    p-1&=2x^2+4x
    }
    &\Rightarrow
    \heva{
    x+y+1&=2y-2x\\    
    p&=x+y+2\\
    p-1&=2x^2+4x
    }
    \\&\Rightarrow
    \heva{
    y&=3x+1\\
    p&=4x+3\\
    4x+2&=2x^2+4x
    }
    \\&\Rightarrow 
    \heva{
    x&=1\\
    y&=4\\
    p&=7.
    }    
    \end{align*}
\end{enumerate}
Như vậy, $(x,y,p)=(1,4,7)$ là bộ số duy nhất thỏa yêu cầu.}
\end{gbtt} 

\begin{gbtt}
Cho các số nguyên dương lẻ $a, b, c$. Biết rằng $a-2$ không là số số chính phương, đồng thời
$$a^{2}+a+3=3\left(b^{2}+b+3\right)\left(c^{2}+c+3\right).$$
Chứng minh rằng trong hai số $b^{2}+b+3$ và $c^{2}+c+3$, có ít nhất một số là hợp số.
\nguon{Baltic Way Mathematical Olympiad 2020}
\loigiai{
Ta giả sử phản chứng rằng $b^2+b+3$ và $c^2+c+3$ là hai số nguyên tố. \\
Đồng thời, không mất tính tổng quát ta giả sử $b \geq c$. 
\begin{enumerate}
    \item Với $b=1,b=3,b=5,b=7$ hoặc $b=9$, ta nhận thấy chỉ trường hợp $b=1$ và $b=7$ cho ta $$b^2+b+3$$ là số nguyên tố.
    \begin{itemize}
        \item Nếu như $b=1,$ ta có $c\le b=1$ nên $c=1,$ và khi thay trực tiếp, ta không tìm được $a.$
        \item Nếu như $b=7,$ ta có $c\le 7$ nên $c=1$ hoặc $c=7.$ Thử trực tiếp, ta cũng không tìm được $a.$
    \end{itemize}
    \item Với $b\ge 11,$ ta sẽ so sánh $a,b$ và $c.$ Thật vậy
    $$a^2+a+3=3\left(b^2+b+3\right)\left(c^2+c+3\right)>4\left(b^2+b+3\right)>(2b)^2+2b+3.$$
    Do đó, $a>2b,$ và đồng thời, vì $b\ge 11$ nên là
    $$a^2+a+3=3\left(b^2+b+3\right)\left(c^2+c+3\right)<3\left(b^2+b+3\right)^2<4b^4+2b^2+3.$$
    Hai so sánh trên cho ta $3b^2>a>2b\ge 2c$. Ngoài ra, khi trừ hai vế đẳng thức ban đầu đi $b^2+b+3,$ ta được đẳng thức tương đương là
    \[(a-b)\left(a+b+1\right)=\left(b^2+b+3\right)\left(3c^2+3c+8\right).\tag{*}\label{xbinhcong1luon}\]
    Tương tự như bài trước, ta chia phần còn lại của bài này ra làm hai trường hợp.
    \begin{itemize}
    \item \chu{Trường hợp 1. }Nếu $a-b$ chia hết cho $b^2+b+3$, ta có 
    $$b^2+b+3\le a-b<2b^2-b-1<2b^2+2b+6.$$
    Bắt buộc, ta phải có $a-b=b^2+b+3,$ và khi đó $a-2=(b+1)^2$ là số chính phương, mâu thuẫn.
    \item \chu{Trường hợp 2. }Nếu $a+b+1$ chia hết cho $b^2+b+3$, ta có $$b^2+b+3\le a+b+1<2b^2+b+1<2b^2+2b+6.$$
    Bắt buộc, ta phải có $a+b+1=b^2+b+3,$ và khi đó $a-2=(b-1)^2$ là số chính phương, mâu thuẫn.
\end{itemize}    
\end{enumerate}
Mâu thuẫn xảy ra trong tất cả trường hợp trên. Giả sử phản chứng là sai. Bài toán được chứng minh.}
\end{gbtt}

\begin{gbtt}
Cho số nguyên tố $p$ và số nguyên dương $n.$ Chứng minh rằng không tồn tại các số nguyên dương $x,y$ thỏa mãn
\[\dfrac{x^2+x}{y^2+y}=p^{2n}.\]
\nguon{Cao Đình Huy}
\loigiai{
Ta sẽ chứng minh bài toán này bằng phản chứng. Ta giả sử tồn tại các số nguyên dương $x,y,$ thỏa mãn yêu cầu bài toán. Theo đó
\[x^2+x=\tron{y^2+y}p^{2n} \Rightarrow x\tron{x+1}= y\tron{y+1}p^{2n}.\tag{1}\label{snt1}\]
Dựa vào (\ref{snt1}), ta chỉ ra $p^{2n}$ là ước của $x\tron{x+1}.$ Do $(x,x+1)=1$ nên $p^{2n}$ là ước của $x$ hoặc $x+1.$ Ta xét các trường hợp kể trên.
\begin{enumerate}
    \item Nếu $p^{2n}\mid x,$ ta đặt $x=p^{2n}k$ trong đó $k$ là số nguyên dương. Thế vào (\ref{snt1}), ta được
    \[p^{2n}k\tron{p^{2n}k+1}=y\tron{y+1}p^{2n}\Rightarrow p^{2n}k^2+k=y^2+y\Rightarrow \tron{p^nk-y}\tron{p^nk+y}=y-k.\tag{2}\label{snt2}\]
    Ta tiếp tục chia bài toán thành các trường hợp nhỏ hơn.
    \begin{itemize}
        \item \chu{Trường hợp 1.} Với $y=k$, từ (\ref{snt2}) ta có 
        $$\tron{p^nk-y}\tron{p^nk+y}=0\Rightarrow p^nk=y \Rightarrow p^n=1\Rightarrow n=0.$$ 
        Điều này mâu thuẫn với giả thiết.
        \item \chu{Trường hợp 2.} Với $y> k$, ta có $$\tron{p^nk-y}\tron{p^nk+y}> 0\Rightarrow\tron{p^nk-y}>0\Rightarrow\tron{p^nk-y}\geq1.$$
        Kết hợp với (\ref{snt2}), ta suy ra
        $$y-k \ge p^nk+y\Rightarrow0> k\tron{p^n+1}\Rightarrow0\ge p^n+1.$$
        Điều này mâu thuẫn với giả thiết.
        \item \chu{Trường hợp 3.} Với $y< k$, ta có $y<p^nk,$ thế nên $$\tron{p^nk-y}\tron{p^nk+y}>0>y-k.$$
        Đánh giá trên mâu thuẫn với (\ref{snt2}).
    \end{itemize}
    \item Nếu  $p^{2n}\mid x+1,$ ta đặt $x+1=p^{2n}$ với $k$ là số nguyên dương. Thế vào (\ref{snt1}), ta được 
    \[\tron{p^{2 n}k-1}p^{2n}k=p^{2n}y\tron{y+1}\Rightarrow p^{2n}k^2-k=y^2+y=\tron{p^nk-y}\tron{p^nk+y}=y+k.\tag{3}\label{snt3}\]
    Ta tiếp tục chia bài toán thành các trường hợp nhỏ hơn.    
    \begin{itemize}
        \item \chu{Trường hợp 1.} Với $p^nk\le y,$ từ (\ref{snt3}), ta có
        $$y+k=\tron{p^nk-y}\tron{p^nk+y}\le0.$$
        Điều này mâu thuẫn với giả thiết.
        \item \chu{Trường hợp 2.} Với $p^nk> y,$ từ (\ref{snt3}), ta có
        $$p^nk+y\le k+y\Rightarrow p^n\le 1.$$
        Điều này mâu thuẫn với giả thiết.
    \end{itemize}
\end{enumerate}
Mâu thuẫn chỉ ra trong tất cả các trường hợp chứng tỏ giả sử phản chứng là sai. Chứng minh hoàn tất.}
\end{gbtt}

\subsection{Tính chia hết cho lũy thừa một số nguyên tố}

\subsubsection*{Ví dụ minh họa}

\begin{bx}
Tìm các số nguyên dương $x,y$ và số nguyên tố $p$ thỏa $2^xp^2+27=y^3.$
\loigiai{
Phương trình đã cho tương đương với
$$2^xp^2=(y-3)\left(y^2+3y+9\right).$$
Với mục tiêu là xác định dạng của $y-3$ và $y^2+3y+9$, ta có các đánh giá sau
\begin{enumerate}
    \item[i,] $y^2+3y+9$ là số lẻ, thế nên $\left(2^x, y^2+3y+9\right)=1.$
    \item[ii,] Với $p\ne 3$, ta có $\left(y-3,y^2+3y+9\right)=1.$ Còn với $p=3,$ ta có $y^3$ chia hết cho $9$ nhưng không chia hết cho $27,$ mâu thuẫn.
    \item[iii,] $1<y-3<y^2+3y+9.$ 
\end{enumerate}
Ba đánh giá trên cho ta $y-3=2^x$ và  $y^2+3y+9=p^2.$
Kết hợp hai nhận xét vừa rồi, ta có
\begin{align*}
    \left(2^x+3\right)^2+3\cdot\left(2^x+3\right)+9=p^2
    &\Rightarrow 2^{2x}+15\cdot 2^x+27=p^2.
\end{align*}
Nếu $x\ge 2,$ do $2^{2x}$ và $2^x$ đều chia hết cho $4,$ ta nhận thấy
$$p^2=2^{2x}+15\cdot 2^x+27\equiv 3\pmod{4}.$$
Không có bình phương số nào đồng dư $3$ theo modulo $4,$ thế nên bắt buộc $x=1.$\\
Thử trực tiếp, ta nhận được $(p,x,y)=(7,1,5)$ là bộ số duy nhất thỏa mãn đề bài.}
\end{bx}

\begin{bx}
Tìm tất cả các số nguyên tố $p$ và số nguyên dương $n$ thỏa mãn
\[n^8-p^5=n^2+p^2.\]
\nguon{Zhautykov}
\loigiai{
Giả sử tồn tại số nguyên tố $p$ và số tự nhiên $n$ thỏa mãn yêu cầu. Giả sử này cho ta
\[n^2(n-1)(n+1)\tron{n^2-n+1}\tron{n^2+n+1}=p^2\tron{p^3+1}.\tag{*}\label{8522}\]
Do $p$ là số nguyên tố, ta sẽ xét $3$ trường hợp sau đây.
\begin{enumerate}
    \item Một trong ba số $n,n-1,n+1$ chia hết cho $p.$\\
    Trong trường hợp này, ta có $n\ge p-1.$ Kết hợp với (\ref{8522}), ta được
    $$p^2\tron{p^3+1}\ge \tron{p-1}^2\tron{p-2}p\tron{p^2-3p+3}\tron{p^2-p+1}.$$
    Một cách tương đương, ta có
    $$p(p+1)\ge (p-1)^2\tron{p-2}\tron{p^2-3p+3}.$$
    Bất đẳng thức trên đổi chiều với $p\ge 4,$ thế nên ta có $p\le 3.$ \\
    Thử trực tiếp với $p=2,3$ và thay trở lại (\ref{8522}), ta thu được $n=2$ tại $p=3.$
    \item Cả hai số $n^2-n+1$ và $n^2+n+1$ chia hết cho $p.$ Trong trường hợp này, ta có
    $$
    \heva{&p\mid\tron{n^2-n+1}\\&p\mid\tron{n^2+n+1}}
    \Rightarrow
    \heva{&p\mid2\tron{n^2+1}\\&p\mid2n}
    \Rightarrow
    p\mid 2\tron{n^2+1,n}
    \Rightarrow
    p\mid 2\Rightarrow p=2.
    $$
    Thế $p=2$ vào (\ref{8522}), ta không tìm được $n$ nguyên dương.
    \item Một trong hai số $n^2-n+1$ và $n^2+n+1$ chia hết cho $p^2.$ Trong trường hợp này, ta có
    $$
    \hoac{&p^2\le n^2-n+1 \\ &p^2\le n^2+n+1}
    \Rightarrow
    p^2\le n^2+n+1
    \Rightarrow
    p^2<(n+1)^2
    \Rightarrow
    p<n+1.
    $$
    Ta nhận thấy rằng $n>p-1.$ Xử lí tương tự \chu{trường hợp 1}, ta không tìm được $p$ nguyên tố và $n$ nguyên dương tương ứng.
\end{enumerate}
Kết luận, $(n,p)=(2,3)$ là cặp số duy nhất thỏa yêu cầu bài toán.}
\end{bx}

\begin{bx}
Tìm tất cả các số nguyên tố $p$ và số tự nhiên $x,y$ thỏa mãn $p^x=y^4+4.$
\nguon{Indian National Mathematical Olympiad 2008}
\loigiai{
Với các số $p,x,y$ thỏa yêu cầu, ta có
$$p^x=\left(y^2-2y+2\right)\left(y^2+2y+2\right).$$
Ta suy ra cả $y^2-2y+2$ và $y^2+2y+2$ đều là lũy thừa của $p.$ Đặt $$y^2-2y+2=p^m,\quad y^2+2y+2=p^n,$$ với $m,n$ là các số tự nhiên. Ta có nhận xét
\begin{align*}
    y^2+2y+2>y^2-2y+2&\Rightarrow n>m\\&\Rightarrow p^n>p^m\\&\Rightarrow p^m\mid p^n\\&\Rightarrow \left(y^2-2y+2\right)\mid \left(y^2+2y+2\right).
\end{align*}
Với việc quy phép chia hết về một biến, ta lại tiếp tục suy ra
\begin{align*}
    \left(y^2-2y+2\right)\mid \left(y^2+2y+2\right)
    &\Rightarrow
    \left(y^2-2y+2\right)\mid 4y
    \\&\Rightarrow
    4y\ge y^2-2y+2
    \\&\Rightarrow 
    y^2-6y+2\le 0
    \\&\Rightarrow (y-3)^2\le 7
    \\&\Rightarrow y\le 5.
\end{align*}
Theo đó, $y$ chỉ có thể nhận tối đa $6$ giá trị là $0,1,2,3,4,5.$ Ta lập bảng sau đây.
\begin{center}
    \begin{tabular}{c|c|c|c|c|c|c}
       $y$ & $0$ & $1$ & $2$ & $3$ & $4$ & $5$  \\
       \hline
       $p^x=y^4+4$  & $4$ & $5$ & $20$ & $85$ & $260$ & $629$   \\
       \hline 
       $(p,x)$ & $(2,2)$ & $(5,1)$ & $\not\in\mathbb{N}^2$ & $\not\in\mathbb{N}^2$ & $\not\in\mathbb{N}^2$ & $\not\in\mathbb{N}^2$
    \end{tabular}
\end{center}
Kết luận, có hai bộ $(p,x,y)$ thỏa yêu cầu, đó là $(2,2,0)$ và $(5,1,0).$}
\end{bx}

\begin{bx}
Tìm các cặp số nguyên dương $a,b$ phân biệt thỏa mãn $b^2+a$ vừa là lũy thừa một số nguyên tố, vừa là ước của $a^2+b.$
\nguon{Saint Patersburg 2001}
\loigiai{
Giả sử tồn tại cặp số nguyên dương $(a,b)$ thỏa mãn đề bài. Ta đặt $b^2+a=p^n.$ Theo giả thiết, ta có 
$$p^n=\tron{b^2+a}\mid \tron{a^2+b}.$$
Giả thiết kể trên cho ta biết $a\equiv -b^2\pmod{p^n},$ và thế thì
$$0\equiv a^2+b\equiv b^4+b\equiv b\tron{b+1}\tron{b^2-b+1}\pmod{p^n}.$$
Với việc $\tron{b,b+1}=\tron{b,b^2-b+1}=1,$ ta suy ra hoặc $b,$ hoặc $\tron{b+1}\tron{b^2-b+1}$ chia hết cho $p^n.$ Song, nếu $b$ chia hết cho $p^n$ thì $b\ge p^n=b^2+a,$ mâu thuẫn. Còn nếu  $\tron{b+1}\tron{b^2-b+1}$ chia hết cho $p^n$ thì vì cả hai số $b+1$ và $b^2-b+1$ đều nhỏ hơn $p^n=b^2+a$ nên ta có
$$\heva{&p\mid (b+1) \\ &p\mid\tron{b^2-b+1}}
\Rightarrow 
\heva{&p\mid (b+1) \\ &p\mid\tron{(b+1)(b-2)+3}}
\Rightarrow p=3.$$
Ta có $b+1$ chia hết cho $3.$ Đặt $b=3k-1.$ Phép đặt này cho ta
$$b^2-b+1=(3k-1)^2-(3k-1)+1=9k^2-9k+3.$$
Điều này chứng tỏ $b^2-b+1$ chia hết cho $3$ nhưng không chia hết cho $9.$ Do tích $(b+1)\tron{b^2-b+1}$ chia hết cho $3^n$ nên $b+1$ chia hết cho $3^{n-1}.$ Ta suy ra
$$b+1\ge 3^{n-1}\Rightarrow 3(b+1)\ge 3^n=b^2+a\ge b^2+1\Rightarrow (b-1)(b-2)\le 0.$$
Giải bất phương trình trên, ta được $b=2$ hoặc $b=1.$ Ta xét các trường hợp kể trên.
\begin{enumerate}
    \item Với $b=1,$ từ $a^2+b$ chia hết cho $b^2+a$ ta có $a+1$ là ước của $a^2+1=(a-1)(a+1)+2.$ Ta tìm được $a=1$ từ đây.
    \item Với $b=2,$ từ $a^2+b$ chia hết cho $b^2+a$ ta có $a+4$ là ước của $a^2+2=(a-4)(a+4)+18.$ Ta tìm được $a=14,a=5$ và $a=2$ từ đây.
\end{enumerate}
Thử lại từng trường hợp, ta kết luận rằng chỉ có cặp $(a,b)=(5,2)$ thỏa yêu cầu bài toán.}
\end{bx}

\subsubsection*{Bài tập tự luyện}

\begin{btt}
Tìm số nguyên dương $m$ số nguyên tố $p, q$ sao cho $2^{m}p^{2}+1=q^5.$
\nguon{Finland Finish National High School Mathematics Competition 2013}
\end{btt}

\begin{btt}
Tồn tại không các số nguyên tố $p,q$ thỏa mãn $p^2\left(p^3-1\right)=q(q+1)?$
\nguon{Junior Balkan Mathematical Olympiad Shortlist 2012}
\end{btt}

\begin{btt}
Tìm các số nguyên tố $p,q$ thỏa mãn $q^3+1$ chia hết cho $p^2$ và $p^6-1$ chia hết cho $q^2.$
\nguon{Bulgarian National Mathematical Olympiad 2014}
\end{btt}

\begin{btt}
Tìm các số tự nhiên $n$ sao cho $n^5-4n^4+16n^2-9n-7$ có đúng một ước nguyên tố.
\end{btt}

\begin{btt}
Tìm tất cả các bộ ba $\left( p,y,n \right)$ nguyên dương thoả mãn $n+1$ không chia hết cho số nguyên tố $p,$ đồng thời $p^n+1=y^{n+1}.$
\end{btt}

\begin{btt}
Tìm tất cả các số nguyên tố $p,q,r$ và số tự nhiên $n$ thỏa mãn
$$p^2=q^2+r^n.$$
\nguon{Olympic Toán học Bắc Trung Bộ 2020}
\end{btt}

\begin{btt}
Cho số nguyên tố $p$ và số nguyên dương $n$ thỏa mãn $p^2$ là ước của 
$$\tron{1^2+1}\tron{2^2+1}\ldots\tron{n^2+1}.$$
Chứng minh rằng $p<2n.$
\nguon{China Western Mathematical Olympiad 2017}
\end{btt}

\begin{btt}
Tìm các số nguyên dương $a, b$ sao cho $\dfrac{a^2(b-a)}{b+a}$ là bình phương một số nguyên tố.
\end{btt}

\begin{btt}
Tìm các số nguyên dương $x,y$ thỏa mãn $\dfrac{x y^{3}}{x+y}$ là lập phương một số nguyên tố.
\nguon{Thai Mathematical Olympiad 2013}
\end{btt}

\begin{btt}
Tìm tất cả các số nguyên tố $p$ và các số nguyên dương $x,y,n$ thỏa mãn
\[p^n=x^3+y^3.\]
\nguon{Hungary, 2020}
\end{btt}

\begin{btt}
Tìm các cặp số nguyên dương $a,b$ thỏa mãn $a^3+b$ vừa là lũy thừa một số nguyên tố lẻ, vừa là ước của $a+b^3.$
\end{btt}

\begin{btt}
Cho hai số nguyên dương $a,b$ phân biệt và lớn hơn $1$ thỏa mãn $b^2+a-1$ là ước của $a^2+b-1.$ Chứng minh rằng $b^2+a-1$ không thể có nhiều hơn một ước nguyên tố.
\nguon{Saint Patersburg 2001}
\end{btt}

\subsubsection*{Hướng dẫn bài tập tự luyện}

\begin{gbtt}
Tìm số nguyên dương $m$ số nguyên tố $p, q$ sao cho $2^{m}p^{2}+1=q^5.$
\nguon{Finland Finish National High School Mathematics Competition 2013}
\loigiai{Phương trình đã cho tương đương
$$2^{m}p^{2}=(q-1)\left(q^{4}+q^{3}+q^{2}+q+1\right).$$
Với mục tiêu là xác định dạng của $q-1$ và $q^{4}+q^{3}+q^{2}+q+1,$ ta có các đánh giá sau
\begin{enumerate}
    \item[i,] $q$ là số lẻ, thế nên $\left(2^{m}, q^{4}+q^{3}+q^{2}+q+1\right)=1.$
    \item[ii,] Với $q\ne 5$, ta có $\left(q-1,q^{4}+q^{3}+q^{2}+q+1\right)=1.$ Còn với $q=5,$ ta không tìm được $m$ và $p.$
    \item[iii,] $1<q-1<q^4+q^3+q^2+q+1.$ 
\end{enumerate}
Ba đánh giá trên cho ta
$q-1=2^m$ và $q^4+q^3+q^2+q+1=p^2.$
Kết hợp hai nhận xét vừa rồi, ta có
\begin{align*}
    q^4+q^3+q^2+q=p^2-1
    &\Rightarrow \left(q^2+q\right)\left(q^2+1\right)=(p-1)(p+1)
    \\&\Rightarrow \left(\left(2^m+1\right)^2+\left(2^m+1\right)\right)\left(\left(2^m+1\right)^2+1\right)=(p-1)(p+1)
    \\&\Rightarrow \left(2^{2 m}+3\cdot 2^{m}+2\right)\left(2^{2 m}+2^{m+1}+2\right)=(p-1)(p+1).
\end{align*}
Nếu $m \geq 3$, do $2^m$ chia hết cho $8$ nên ta nhận thấy
$$2^{2m}+3\cdot 2^m+2\equiv 2\pmod{8},\quad2^{2 m}+2^{m+1}+2\equiv 2\pmod{8},$$
và khi đó $(p-1)(p+1)$ chia cho $8$ dư $4,$ mâu thuẫn với việc tích hai số chẵn liên tiếp $(p-1)(p+1)$ bắt buộc phải chia hết cho $8.$ Trường hợp trên không xảy ra, chứng tỏ $m=1$ hoặc $m=2.$ Kiểm tra trực tiếp, ta nhận thấy $(p,q,m)=(11,3,1)$ là bộ số duy nhất thỏa yêu cầu.}
\end{gbtt}

\begin{gbtt}
Tồn tại không các số nguyên tố $p,q$ thỏa mãn $p^2\left(p^3-1\right)=q(q+1)?$
\nguon{Junior Balkan Mathematical Olympiad Shortlist 2012}
\loigiai{
Câu trả lời là phủ định. Thật vậy, ta giả sử phản chứng rằng, tồn tại các số nguyên tố $p,q$ thỏa mãn
\[p^2\left(p^3-1\right)=q(q+1).\tag{*}\label{jbmosl2.012}\]
Nếu như $p=q,$ ta có
$$p^2\left(p^3-1\right)=p(p+1)\Rightarrow p^5-2p^2-p=0\Rightarrow p\left(p^4-2p-1\right)=0.$$
Do $p^4-2p-1=p\left(p^3-2\right)\ge p\left(2^3-2\right)-1=6p-1>0$ nên trường hợp $p=q$ không xảy ra, do vậy $p\ne q.$  Tới đây, ta thực hiện xét tính chia hết cho $q$ và $p$ ở cả hai vế. Cụ thể
\begin{itemize}
    \item[i,] Vế phải của (\ref{jbmosl2.012}) chia hết cho $q$ và $(p,q)=1,$ chứng tỏ $p^3-1=(p-1)\left(p^2+p+1\right)$ chia hết cho $q.$
    \item[ii,] Vế trái của (\ref{jbmosl2.012}) chia hết cho $p^2$ và $\left(p^2,q\right)=1,$ chứng tỏ $q+1$ chia hết cho $p^2.$
\end{itemize}
Ta sẽ tiếp tục chia bài toán thành các trường hợp nhỏ hơn dựa theo nhận xét thứ nhất.
\begin{enumerate}
    \item Nếu $p-1$ chia hết cho $q$ và $q+1$ chia hết cho $p$ (chính xác là $p^2$), ta có $p-1\ge q\ge p-1,$ thế nên $q=p-1.$ Thế trở lại (\ref{jbmosl2.012}), ta có
    $$p^2\left(p^3-1\right)=p(p-1)\Rightarrow p\left(p^2+p+1\right)=1.$$
    Ta không chỉ ra được $p$ nguyên tố từ đây.
    \item Nếu $p^2+p+1$ chia hết cho $q$ và $q+1$ chia hết cho $p^2,$ ta đặt $q+1=kp^2$ với $k$ là số nguyên dương. Từ $p^2+p+1$ chia hết cho $q$ và phép đặt, ta có
    $$p^2+p+1\ge q\ge kp^2-1.$$
    Bằng phản chứng, ta chỉ ra $k\ge 2$ không thỏa, thế nên bắt buộc $k=1.$ Với $k=1,$ ta có 
    $$q=p^2-q=(p-1)(p+1).$$
    Do $q$ là số nguyên tố và $1\le p-1\le p+1$ nên $p-1=1$ và $p+1=q,$ hay là $p=2,q=3.$ \\
    Thế trở lại (\ref{jbmosl2.012}), ta thấy không thỏa.
\end{enumerate}
Như vậy, giả sử phản chứng là sai. Bài toán được chứng minh.}
\end{gbtt}

\begin{gbtt}
Tìm các số nguyên tố $p,q$ thỏa mãn $q^3+1$ chia hết cho $p^2$ và $p^6-1$ chia hết cho $q^2.$
\nguon{Bulgarian National Mathematical Olympiad 2014}
\loigiai{
\begin{enumerate}
    \item Với $p,q\le 3,$ ta tìm ra các bộ $(p,q)=(2,3)$ và $(p,q)=(3,2)$ thoả yêu cầu.
    \item Với $p,q \geq 5$, ta có các nhận xét
    \begin{enumerate}[i,]
        \item[i,] $q^2\mid p^6-1=\tron{p-1}\tron{p+1}\tron{ p^2-p+1}\tron{p^2+p+1}.$
        \item[ii,] Các số $p-1,p+1,p^2-p+1,p^2+p+1$ đôi một nguyên tố cùng nhau.
    \end{enumerate}
    Hai nhận xét trên cho ta biết $q^2$ là ước của một trong các số $p-1, p+1, p^2-p+1, p^2+p+1$.\\
    Ta xét các trường hợp kể trên.
    \begin{itemize}
        \item \chu{Trường hợp 1.} $q^2$ là ước của $p-1$ hoặc $p+1.$ Trong trường hợp này, ta có
        $$\heva{&\hoac{&q^2\mid (p-1) \\ &q^2\mid (p+1)}\\&p^2\mid\tron{q^3+1}}\Rightarrow \heva{&q^2\le p+1 \\ &p^2\le q^3+1}\Rightarrow \tron{q^2-1}^2\le q^3+1\Rightarrow p^2(p-2)(p+1)\le 0.$$
        Điều này không thể xảy ra.
        \item \chu{Trường hợp 2.} $q^2$ là ước của  $p^2-p+1$ hoặc $p^2+p+1.$ Rõ ràng $q^2\le p^2+p+1.$ Ngoài ra, dựa vào giả thiết $q^3+1$ chia hết cho $p^2,$ ta tiếp tục chia trường hợp này thành các khả năng nhỏ hơn.
        \begin{itemize}
            \item \chu{Khả năng 1.} Nếu $q+1$ chia hết cho $p,$ ta đặt $q+1=kp.$ Phép đặt này cho ta
            $$(kp)^2\le p^2-p+1\Rightarrow \tron{k^2-1}p^2+p\le 1\Rightarrow k<0,$$
            mâu thuẫn với việc $p,q$ dương.
            \item \chu{Khả năng 2.} Nếu $q^2+q+1$ chia hết cho $p^2,$ ta đặt $q^2+q+1=lp^2.$ Phép đặt này cho ta $(q+1)^2>lp^2,$ thế nên $q+1>p\sqrt{l}.$ Kết hợp với $q^2\le p^2+p+1,$ ta có      
            $$p^2+p+1\ge q^2>\tron{p\sqrt{l}-1}^2.$$
            Chỉ có $l=1$ hoặc $l=2$ thỏa mãn đánh giá trên. \\
            Bạn đọc tự tìm cách thế ngược lại để thấy được điều mâu thuẫn.
        \end{itemize}
    \end{itemize}
Kết luận, các cặp số nguyên tố thỏa yêu cầu là $(p,q)=(2,3)$ và $(p,q)=(3,2).$
\end{enumerate}}
\end{gbtt}

\begin{gbtt}
Tìm các số tự nhiên $n$ sao cho $n^5-4n^4+16n^2-9n-7$ có đúng một ước nguyên tố.
\loigiai{
Ta nhận thấy $n=0$ và $n=1$ thỏa mãn. Với $n\ge 2,$ do $n^5-4n^4+16n^2-9n-7>0$ nên ta có thể đặt
$$n^5-4n^4+16n^2-9n-7=p^x,$$
trong đó $p$ nguyên tố và $x$ nguyên dương. Với các số $p,x,y$ như vậy, ta có
$$p^x=\left(n^2-5n+7\right)\left(n^3+n^2-2n-1\right).$$
Ta suy ra cả $n^2-5n+7$ và $n^3+n^2-2n-1$ đều là lũy thừa của $p.$ Đặt $$n^2-5n+7=p^a,\quad n^3+n^2-2n-1=p^b,$$ với $a,b$ là các số tự nhiên. Ta dễ dàng chứng minh được $b>a$ và $n^3+n^2-2n-1$ chia hết cho $n^2-5n+7.$ Ta lần lượt suy ra
\begin{align*}
    \left(n^2-5n+7\right)\mid \left(n^3+n^2-2n-1\right)
    &\Rightarrow
    \left(n^2-5n+7\right)\mid \bigg((n+6)\tron{n^2-5n+7}+21n-43\bigg)
    \\&\Rightarrow 
    n^2-5n+7\le 21n-43
    \\&\Rightarrow (n-13)^2\le 119   \\& \Rightarrow 13-\sqrt{119}\le n\le 13+\sqrt{119}.
\end{align*}
Theo đó, $n$ nhận các giá trị từ $3$ cho đến $23.$ \\
Thử từng trường hợp, ta kết luận $n=0,\ n=1,\ n=2,\ n=3$ là tất cả các giá trị của $n$ thỏa mãn đề bài.}
\end{gbtt}

\begin{gbtt}
Tìm tất cả các bộ ba $\left( p,y,n \right)$ nguyên dương thoả mãn $n+1$ không chia hết cho số nguyên tố $p,$ đồng thời $p^n+1=y^{n+1}.$
\loigiai{
Phương trình đã cho tương đương với
$$(y-1)\left({y^n} + {y^{n - 1}} +\ldots+ y + 1\right)=p^n.$$
Giả sử $p$ là ước nguyên tố chung của $y-1$ và ${y^n} + {y^{n - 1}} +\ldots+ y + 1.$ Do $y\equiv 1\pmod{p},$ ta có
$${y^n} + {y^{n - 1}} +\ldots+ y + 1\equiv 1+1+\ldots+1\equiv n+1\pmod{p}.$$
Nhờ vào giả thiết $n+1$ không chia hết cho $p,$ ta chỉ ra giả sử trên là sai. Theo đó, $y-1$ không thể là lũy thừa cơ số $p,$ và bắt buộc $y=2.$ Thế trở lại $y=2,$ ta có
$$p^n+1=2^{n+1}.$$
Tới đây, ta chia bài toán thành trường hợp sau.
\begin{enumerate}
    \item Với $p=2,$ kiểm tra trưc tiếp, ta không tìm ra được $n$ nguyên dương.
    \item Với $p=3,$ kiểm tra trực tiếp, ta tìm ra $n=1.$ 
    \item Với $p\ge 5,$ ta có
    $2\cdot2^n=2^{n+1}=p^n+1\ge 5^n+1,$ đây là điều vô lí.
\end{enumerate}
Tổng kết lại, $(p,y,n)=(3,2,1)$ là bộ số duy nhất thỏa yêu cầu.}
\end{gbtt}

\begin{gbtt}
Tìm tất cả các số nguyên tố $p,q,r$ và số tự nhiên $n$ thỏa mãn
$$p^2=q^2+r^n.$$
\nguon{Olympic Toán học Bắc Trung Bộ 2020}
\loigiai{
Đầu tiên, nếu cả ba số $p,q,r$ đều lẻ, hai vế phương trình khác tính chẵn lẻ, mâu thuẫn. Do vậy, trong các số $p,q,r$ phải có một số bằng $2.$ Giả sử tồn tại các số $p,q,r,n$ thỏa yêu cầu.
\begin{enumerate}
    \item Với $p=2,$ ta có $q^2+r^n=4.$ Ta không tìm được $q$ và $r$ từ đây, do $$q^2+r^n\ge 2^2+2=6>4.$$
    \item Với $q=2,$ ta có
    $r^n=(p-q)(p+q)=(p-2)(p+2).$
    Ta suy ra cả $p-2$ và $p+2$ đều là lũy thừa của $r.$ Nếu hai số này có ước chung là $r,$ ta nhận thấy
    $$r\mid (p+2)-(p-2)=4,$$
    kéo theo $r=2,$ và lúc này $p$ là số nguyên tố chẵn, vô lí. Ta suy ra $p-2=1,$ và dễ dàng tìm được $p=3,r=7,n=1$ trong trường hợp này.
    \item Với $r=2,$ ta có $(p-q)(p+q)=r^n=2^n.$ Ta nhận thấy cả $p-q$ và $p+q$ đều là lũy thừa của $2.$ Ta đặt $p-q=2^x,p+q=2^y,$ với $x,y$ là các số tự nhiên. Phép đặt này cho ta
    $$
    \heva{&p-q=2^x \\ &p+q=2^y}
    \Rightarrow \heva{&p=\dfrac{2^x+2^y}{2} \\ &q=\dfrac{2^y-2^x}{2}}
    \Rightarrow \heva{&p=\dfrac{2^x\left(2^{y-x}+1\right)}{2} \\ &q=\dfrac{2^x\left(2^{y-x}-1\right)}{2}.}   
    $$
    Do $p,q$ là các số nguyên tố lẻ, ta bắt buộc phải có $x=1,$ vậy nên
    $$p=2^{y-x}+1,\qquad q=2^{y-x}-1.$$
    Trong ba số tự nhiên liên tiếp $p,\dfrac{p+q}{2}$ và $q,$ số $\dfrac{p+q}{2}$ không chia hết cho $3,$ thế nên hoặc $p=3,$ hoặc $q=3.$
    \begin{itemize}
        \item\chu{Trường hợp 1.} Nếu $q=3,$ ta có $p=q+2=5,$ và ta còn tìm được thêm $n=4.$
        \item\chu{Trường hợp 2.} Nếu $p=3,$ ta có $q=p-2=1,$ mâu thuẫn.     
    \end{itemize}
\end{enumerate}
Kết luận, các bộ $(n,p,q,r)$ thỏa yêu cầu bao gồm $(1,3,2,7)$ và $(4,5,3,2).$}
\end{gbtt}

\begin{gbtt}
Cho số nguyên tố $p$ và số nguyên dương $n$ thỏa mãn $p^2$ là ước của 
$$\tron{1^2+1}\tron{2^2+1}\ldots\tron{n^2+1}.$$
Chứng minh rằng $p<2n.$
\nguon{China Western Mathematical Olympiad 2017}
\loigiai{
Ta giả sử phản chứng rằng $p\ge 2n.$
Đầu tiên, với mọi số nguyên dương $a\in \vuong{1;n},$ ta có $a^2+1$ không chia hết cho $p^2.$ Điều này xảy ra là do
$$a^2+1\le n^2+1<4n^2\le p^2.$$
Tích của $n$ số $1^2+1,2^2+1,\ldots,n^2+1$ chia hết cho $p^2,$ trong đó không có số nào chia hết cho $p^2.$ Vậy nên, tồn tại hai số $a,b\in\vuong{1;n}$ thỏa mãn $a^2+1$ và $b^2+1$ cùng chia hết cho $p$. Xét hiệu của $a^2+1$ và $ b^2+1$, ta có
$$p\mid \tron{a^2+1-b^2-1}=(a-b)(a+b).$$
Do $p$ là số nguyên tố, $p$ bắt buộc là ước của $1$ trong $2$ số $a-b$ và $a+b.$  Như vậy
$$\hoac{&p\le a-b \\ &p\le a+b}\Rightarrow p\le a+b \le n-1+n= 2n-1.$$
Điều này mâu thuẫn với giả sử $p\ge 2n$. Giả sử phản chứng sai. Bài toán được chứng minh.}
\end{gbtt}

\begin{gbtt}
Tìm các số nguyên dương $a, b$ sao cho $\dfrac{a^2(b-a)}{b+a}$ là bình phương một số nguyên tố.
\loigiai{
Từ giả thiết, ta có thể đặt $a^2(b-a)=p^2(b+a)$ với $p$ là số nguyên tố. Đặt $(a,b)=d,$ khi đó tồn tại các số nguyên dương $x,y$ sao cho $(x,y)=1,a=dx,b=dy.$ Phép đặt này cho ta
\[d^2x^2(y-x)=p^2(x+y).\tag{*}\label{bainayhay}\]
Nhờ vào việc chứng minh được $\left(x+y,x^2\right)=1$ kết hợp với $p^2(x+y)$ chia hết cho $x^2,$ ta suy ra $p^2$ chia hết cho $x^2.$ Tới đây, ta chia bài toán làm các trường hợp sau.
\begin{enumerate}
    \item Với $x=p,$ thế trở lại (\ref{bainayhay}) rồi rút gọn ta có
    \[d^2(y-p)=p+y.\tag{**}\label{bainayrachay}\]
    Dựa vào (\ref{bainayrachay}), ta suy ra $p+y$ chia hết cho $y-p.$ Hơn thế nữa, do $(p+y,p-y)$ nhận một trong các giá trị $1,2,p,2p$ nên $y-p$ cũng nhận một trong các giá trị ấy. Ta xét các trường hợp kể trên.
    \begin{itemize}
        \item\chu{Trường hợp 1.} Với $y-p=1$ hay $y=p+1,$ thế vào (\ref{bainayrachay}), ta được
        $$d^2=2p+1\Rightarrow 2p=(d-1)(d+1).$$
        Do $d-1$ và $d+1$ cùng tính chẵn lẻ, $2p$ chia hết cho $4,$ hoặc là số lẻ, vô lí.
        \item\chu{Trường hợp 2.} Với $y-p=2$ hay $y=p+2,$ thế vào (\ref{bainayrachay}), ta được
        $$2d^2=2p+2\Rightarrow p=(d-1)(d+1).$$
        Do $0<d-1<d+1,$ ta bắt buộc có $d-1=1,d+1=p.$ Theo đó, $d=2$ và $p=3.$ Thay trở lại vào (\ref{bainayrachay}), ta được $y=5,$ còn $x=p=3.$ Cùng với đó, $a=dx=6$ và $b=dy=10.$
        \item\chu{Trường hợp 3.} Với $y-p=p$ hay $y=2p,$ thế vào (\ref{bainayrachay}), ta được
        $$d^2\cdot p=3p\Rightarrow d^2=3,$$
        mâu thuẫn với điều kiện $d$ nguyên dương.
        \item \chu{Trường hợp 4.} Với $y-p=2p$ hay $y=3p,$ thế vào (\ref{bainayrachay}), ta được
        $$d^2\cdot 2p=4p\Rightarrow d^2=2,$$
        mâu thuẫn với điều kiện $d$ nguyên dương.
    \end{itemize}
    \item Với $x=1,$ thế trở lại (\ref{bainayhay}) ta có
    $$d^2(y-1)^2=p^2(y+1)(y-1).$$
    Theo đó, $(y+1)(y-1)$ là số chính phương. Đặt $(y-1)(y+1)=z^2$ với $z$ nguyên dương, thế thì
    $$y^2-1=z^2\Rightarrow (y-z)(y+z)=1\Rightarrow y=1,z=0.$$
    Tuy nhiên, với $y=1,$ rõ ràng $d^2(y-1)=0<p^2(y+1),$ mâu thuẫn.
\end{enumerate}
Tổng kết lại, $(a,b)=(6,10)$ là cặp số nguyên dương duy nhất thỏa yêu cầu.}
\end{gbtt}

\begin{gbtt}
Tìm các số nguyên dương $x,y$ thỏa mãn $\dfrac{x y^{3}}{x+y}$ là lập phương một số nguyên tố.
\nguon{Thai Mathematical Olympiad 2013}
\loigiai{
Giả sử tồn tại cặp $(x,y)$ thỏa yêu cầu bài toán. Từ giả thiết, ta có thể đặt $$xy^3=p^3\tron{x+y},$$ 
trong đó $p$ là số nguyên tố. Đặt $\tron{x,y}=d$, khi đó tồn tại số nguyên dương $u,v$ sao cho $$\tron{u,v}=1, x=du, y=dv.$$ Phép đặt này cho ta
\[du\cdot d^3v^3=p^3d\tron{u+v}\Rightarrow d^3uv^3=p^3\tron{u+v}.\tag{*}\label{thai.13}\]
Nhờ vào việc chứng minh được $\tron{v^3,u+v}=1$ kết hợp với $p^3\tron{u+v}$ chia hết cho $v^3$, ta suy ra $p^3$ chia hết cho $v^3$. Tới đây, ta chia bài toán làm các trường hợp sau.
\begin{enumerate}
    \item Nếu $v=p$, thế vào (\ref{thai.13}), ta có $d^3u=u+p$.\\
    Đẳng thức trên cho ta biết $u\mid (u+p),$ kéo theo $u\mid p.$ Do đó, $u\in\left\{1,p\right\}.$ 
    \begin{itemize}
        \item \chu{Trường hợp 1.} Nếu $u=1,$ kết hợp với $d^3u=u+p$, ta nhận thấy
        $$d^3=1+p\Rightarrow p= \tron{d-1}\tron{d^2+d+1}.$$
        Do hai số $d-1<d^2+d+1$ có tích bằng một số nguyên tố, ta có $d-1=1,$ hay là $d=2.$ Dựa vào kết quả này, ta lần lượt chỉ ra $p=7,v=7,u=1,x=14,y=2.$
        \item \chu{Trường hợp 2.} Nếu $u=p$, kết hợp với $d^3u=u+p$, ta nhận thấy
        $$d^3p=2p\Rightarrow d^3=2.$$
        Không có số nguyên $d$ thỏa mãn trong trường hợp này.
    \end{itemize}
    \item Nếu $v=1$, thế vào (\ref{thai.13}), ta có 
    $$d^3u=p^3\tron{u+1}\Rightarrow d^3u^3=p^3\tron{u+1}u^2.$$
    Ta chỉ ra $\tron{u+1}u^2$ phải là lập phương của một số.
    Ta đặt $\tron{u+1}u^2=w^3$ với $w$ nguyên dương, phép đặt này cho ta $u^3+u^2=w^3$. Lại có
    $$u^3\le u^3+u^2=w^3<\tron{u+1}^3=u^3+3u^2+3u+1.$$
    Do đó, $w^3=u^3$ khi và chỉ khi $u^3= u^3+u^2$, thế nên $u=0,$ mâu thuẫn.
\end{enumerate}
Như vậy, $\tron{x,y}=\tron{14,2}$ là cặp số nguyên dương duy nhất thỏa yêu cầu.}
\begin{luuy}
Suy luận "$\tron{u+1}u^2=w^3\Rightarrow u=0$" trong bài toán trên là ứng dụng của \chu{phương pháp kẹp lũy thừa}. Bạn đọc có thể nghiên cứu kĩ hơn phương pháp này ở \chu{chương III}.
\end{luuy}
\end{gbtt}

\begin{gbtt}
Tìm tất cả các số nguyên tố $p$ và các số nguyên dương $x,y,n$ thỏa mãn
\[p^n=x^3+y^3.\]
\nguon{Hungary, 2020}
\loigiai{
Trong bài toán này, ta xét các trường hợp sau đây.
\begin{enumerate}
    \item Với ${p}=2$, ta chỉ ra ${n}=1$ và ${x}={y}=1$ thỏa mãn đẳng thức.
    \item Với ${p}=3$, ta chỉ ra ${n}=2,x=1$ và $y=2$ thỏa mãn đẳng thức.
    \item Với ${p}>3$, giả sử tồn tại bộ $(x,y,n)$ thỏa đẳng thức. Trong số các bộ ấy, ta xét bộ $(x,y,n)$ có giá trị của $n$ nhỏ nhất. Đầu tiên, ta nhận thấy rằng
    $$(x+y)\left(x^2-xy+y^2\right)=p^n.$$
    Do $p>3$ nên $(x,y)\ne (1,1),$ kéo theo $2< x+y\le x^2-xy+y^2.$ Bằng lập luận này, khi đặt $$x^2-xy+y^2=p^k,\:x+y=p^l,$$ ta suy ra $k>l\ge 2.$ Ta có
    \begin{align*}
    \heva{&p\mid (x+y) \\ &p\mid \left(x^2-xy+y^2\right)}
    &\Rightarrow 
    \heva{&p\mid (x+y) \\ &p\mid \left(x^2+2xy+y^2-3xy\right)}
    \\&\Rightarrow 
    \heva{&p\mid (x+y) \\ &p\mid 3xy}
   \\& \Rightarrow 
    \heva{&p\mid (x+y) \\ &p\mid xy}
    \\&\Rightarrow 
    \heva{&p\mid x \\ &p\mid y.}
    \end{align*}
    Bằng lập luận trên và biến đổi 
    $${p}^{{n}-3}=\left(\dfrac{{x}}{{p}}\right)^{3}+\left(\dfrac{{y}}{{p}}\right)^{3},$$
    ta chỉ ra còn có bộ $\left(\dfrac{x}{p},\dfrac{y}{p},n-3\right)$ ngoài bộ $(x,y,n)$ thỏa mãn, mâu thuẫn với tính nhỏ nhất của $n.$
\end{enumerate}
Tổng kết lại, tất các số nguyên tố $p$ thỏa mãn yêu cầu bài toán là ${p}=2$ và ${p}=3.$}
\begin{luuy}
Ngoài cách gọi ra bộ có $n$ nhỏ nhất, bạn đọc có thể tiến hành bài toán theo cách bình thường theo các bước:
\begin{enumerate}
    \item Chứng minh cả $x$ và $y$ đều chia hết cho $p.$
    \item Áp dụng phép lùi vô hạn, chỉ ra tất cả các bộ
    $\left(\dfrac{x}{p^k},\dfrac{y}{p^k},n-3k\right)$
    đều thỏa mãn đẳng thức, nhưng do số mũ của $p$ trong phân tích của $x$ là hữu hạn nên điều này vô lí.
\end{enumerate}
\end{luuy}
\end{gbtt} 

\begin{gbtt}
Tìm các cặp số nguyên dương $a,b$ thỏa mãn $a^3+b$ vừa là lũy thừa một số nguyên tố lẻ, vừa là ước của $a+b^3.$
\loigiai{
Ta giả sử $a^3+b$ là lũy thừa một số nguyên tố. Ta đặt $a^3+b=p^n.$ Theo giả thiết, ta có
$$p^n=\tron{a^3+b}\mid \tron{b^3+a}.$$
Giả thiết kể trên cho ta biết $b\equiv \tron{-a}^3 \pmod{p^n},$ và thế thì
$$0\equiv b^3+a\equiv\tron{-a}^3+a\equiv a\tron{1-a^4}\tron{1+a^4}\pmod{p^n}.$$
Do $p>2,$ các số $a,\ 1+a^4$ và $1-a^4$ đôi một nguyên tố cùng nhau. Điều này dẫn đến chỉ duy nhất một trong $3$ số ấy chia hết cho $p^n.$ Lại có $a<a^3+b=p^n$ nên $p^n\mid \tron{1+a^4}$ hoặc $p^n\mid \tron{a^4-1}.$ Ta xét các trường hợp sau.
    \begin{enumerate}
        \item Nếu $p^n\mid \tron{a^4-1},$ ta có $p^n\mid\tron{a-1}\tron{a+1}\tron{a^2+1}.$ Vì $3$ số này đôi một nguyên tố cùng nhau nên chỉ có $1$ số chia hết cho $p^n.$ Thế nhưng, do
        $$0\le a-1<a+1<a^2+1<a^3+b=p^n$$
        nên bắt buộc $a=1,$ và $b=p^n-1.$ Thử lại thấy thỏa mãn.
        \item Nếu $p^n\mid\tron{a^4+1},$ ta có nhận xét
        $$p^n\mid a\tron{a^3+b}-\tron{a^4+1}=ab-1.$$
        Nhận xét này giúp ta chỉ ra $ab-1\ge p^n=a^3+1,$ nhưng mâu thuẫn.
    \end{enumerate}
Như vậy, các cặp $(a,b)$ thỏa mãn có dạng $\tron{1,p^n-1},$ trong đó $p$ nguyên tố và $n$ nguyên dương tùy ý.}
\end{gbtt}

\begin{gbtt}
Cho hai số nguyên dương $a,b$ phân biệt và lớn hơn $1$ thỏa mãn $b^2+a-1$ là ước của $a^2+b-1.$ Chứng minh rằng $b^2+a-1$ không thể có nhiều hơn một ước nguyên tố.
\nguon{Saint Patersburg 2001}
\loigiai{
Ta giả sử $b^2+a-1$ là lũy thừa một số nguyên tố. Ta đặt $b^2+a-1=p^n.$ Theo giả thiết, ta có 
$$p^n=\tron{b^2+a-1}\mid \tron{a^2+b-1}.$$
Giả thiết kể trên cho ta biết $a\equiv 1-b^2\pmod{p^n},$ và thế thì
$$0\equiv a^2+b-1\equiv \tron{1-b^2}^2+b-1= b\tron{b+1}\tron{b^2+b-1}\pmod{p^n}.$$
Do ba số $b,b+1$ và $b^2+b-1$ đôi một nguyên tố cùng nhau nên chỉ một trong chúng được phép chia hết cho $p^n=b^2+a-1.$ Nhờ vào nhận xét
$$b-1<b<b^2+a-1,$$
ta thấy chỉ trường hợp $b^2+a-1$ là ước của $b^2+b-1$ xảy ra, thế nên $b\ge a.$\\ Mặt khác, vì $b^2+a-1$ là ước của $a^2+b-1$ nên
$$0\le \tron{a^2+b-1}-\tron{b^2+a-1}=(a-b)(a+b-1).$$
Ta suy ra $a\ge b,$ kết hợp với $b\ge a$ thì $a=b.$ Điều này mâu thuẫn với giả thiết $a,b$ phân biệt. \\
Giả sử phản chứng ban đầu là sai. Bài toán được chứng minh.}
\end{gbtt}
 %tính nguyên tố cùng nhau
\chapter{Số chính phương, số lập phương, căn thức}

Cũng như số nguyên tố, số chính phương là một “loại số” mới được giới thiệu lần đầu ở bậc trung học cơ sở. Số chính phương (tiếng Anh là \chu{square number}) còn được gọi là số hình vuông bởi nó là diện tích của một hình vuông có cạnh là số tự nhiên. Có rất nhiều tính chất thú vị xoay quanh số chính phương như chúng là những số tự nhiên duy nhất có một lượng lẻ ước nguyên dương, hay tổng của dãy những số lẻ đầu tiên là một số chính phương, tổng của lập phương các số tự nhiên đầu tiên cũng là số chính phương,...\\ \\
Trong khi đó, căn thức trong một vài trường hợp, là phép toán ngược của phép lũy thừa. Mối liên hệ mật thiết giữa các căn bậc và lũy thừa, chẳng hạn như căn bậc hai và số chính phương, đã kiến tạo nên những tính chất số học hay và đẹp.
\\ \\
Ở chương III của quyển sách này, tác giả muốn tập trung nghiên cứu các bài toán xoay quanh hai "loại số" này và các lũy thừa có số mũ cao hơn, cùng với đó tìm hiểu thêm tính số học của căn thức. Chương III được chia làm 5 phần, tương ứng với 5 dạng bài tập khác nhau.
\begin{itemize}
    \item\chu{Phần 1.} Biến đổi đại số.
    \item\chu{Phần 2.} Ứng dụng của đồng dư thức.
    \item\chu{Phần 3.} Phương pháp kẹp lũy thừa.
    \item\chu{Phần 4.} Ước chung lớn nhất và tính chất lũy thừa.
    \item\chu{Phần 5.} Căn thức trong số học.
\end{itemize}

\section*{Các định nghĩa}
\begin{dx}
Số chính phương là số viết được thành bình phương một số tự nhiên.
\end{dx}
Một vài số chính phương đầu dãy có thể kể đến như
\begin{multicols}{4}
\begin{itemize}
    \item $0=0^2$ 
    \item $1=1^2$ 
    \item $4=2^2$ 
    \item $9=3^2$ 
    \item $16=4^2$
    \item $25=5^2$
    \item $36=6^2$
    \item $49=7^2$
    \item $64=8^2$
    \item $81=9^2$
    \item $100=10^2$
    \item $121=11^2$
\end{itemize}
\end{multicols}

\begin{dx}
Số lập phương là số viết được thành lập phương một số nguyên.
\end{dx}
Một vài số lập phương không âm đầu dãy có thể kể đến như
\begin{multicols}{4}
\begin{itemize}
    \item $0=0^3$ 
    \item $1=1^3$ 
    \item $8=2^3$ 
    \item $27=3^3$ 
    \item $64=4^3$
    \item $125=5^3$
    \item $216=6^3$
    \item $343=7^3$
    \item $512=8^3$
    \item $729=9^3$
    \item $1000=10^3$
    \item $1331=11^3$
\end{itemize}
\end{multicols}

\begin{dx}
Cho số nguyên dương $n.$ Căn bậc $n$ của một số $x$ là một số $r$ sao cho $r^n=x.$
\end{dx}

Một số có thể có $0,1$ hoặc $2$ căn bậc $n.$ Chẳng hạn
\begin{enumerate}
    \item Số $5$ có hai căn bậc hai là $\sqrt{5}$ và $-\sqrt{5}.$
    \item Số $-12$ không có căn bậc hai nào.  
    \item Số $27$ có một căn bậc ba là $3.$
    \item Với mọi số nguyên dương $n,$ số $0$ có một căn bậc $n$ là chính nó.
\end{enumerate}

\section{Biến đổi đại số}

Trong mục này, ta sử dụng các cách biến đổi đại số đã học nhằm đưa biểu thức về dạng chính phương hoặc lập phương.

\subsection*{Bài tập tự luyện}

\begin{btt}
Cho $a,b$ và $c$ là các số nguyên thỏa mãn $ab + bc + ca = 1$. \\
Chứng minh rằng $\tron{1+a^2}\tron{1+b^2}\tron{1+c^2}$ là số chính phương.
\end{btt}

\begin{btt}
Chứng minh rằng tổng của tích của bốn số tự nhiên liên tiếp và $1$ luôn là số chính phương.
\nguon{Vũ Hữu Bình}
\end{btt}

\begin{btt} \label{scp1.111}
Chứng minh rằng ${A}=1^{3}+2^{3}+3^{3}+\cdots+2016^{3}$ là số chính phương.
\end{btt}

\begin{btt}
Chứng minh rằng $N=\underbrace{11\ldots1}_{1995} \underbrace{00\ldots0}_{1994}5+1$ là một số chính phương.
\nguon{Vietnamese National Mathematical Olympiad 1995, Group A}	
\end{btt}

\begin{btt}
Chứng minh rằng $\underbrace {11\ldots11}_{2021}\underbrace {22\ldots22}_{2022}5$
là một số chính phương.
\end{btt}

\begin{btt} 
Chứng minh rằng với mỗi số nguyên $n \geq 6$ thì $$a_{{n}}=1+\dfrac{2\cdot6\cdot10 \cdots(4 n-2)}{(n+5)(n+6) \cdots(2 n)}$$ là một
số chính phương.
\nguon{Chuyên Đại học Sư phạm Hà Nội 2014}
\end{btt}

\begin{btt}
Cho 2 số nguyên ${a}, {b}$ thỏa mãn ${a}^{2}+{b}^{2}+1=2({ab}+{a}+{b})$. Chứng minh $a$ và $b$ là
hai số chính phương liên tiếp.
\nguon{Chuyên Đại học Sư phạm Hà Nội 2016}
\end{btt}

\begin{btt}
Tìm các số nguyên $n$ sao cho $A=n^{4}+n^{3}+n^{2}$ là số chính phương.
\end{btt}

\begin{btt}
Tìm tất cả các số nguyên $n$ sao cho $$A=n^6-3n^5-3n^4+20n^3-48n^2+96n-80$$ là số lập phương.
\end{btt}

\begin{btt}
Chứng minh rằng không tồn tại số nguyên dương $n$ sao cho tích
$$\left(1^4+1^2+1\right)\left(2^4+2^2+1\right)\cdots\left(n^4+n^2+1\right)$$
là một số chính phương.
\nguon{Tuymada 2019}
\end{btt}

\begin{btt}
Tìm tất cả các số tự nhiên $n$ thỏa mãn $2 n+1,3 n+1$ là các số chính phương và $2n+9$ là số nguyên tố. 
\nguon{Chuyên Toán Hà Nội 2017}
\end{btt}

\begin{btt}
Cho 2 số hữu tỉ $a, b$ thoả mãn điều kiện $a^{3}+4 a^{2} b=4 a^{2}+b^{4}$. Chúng minh rằng $\sqrt{a}-1$ là bình phương của 1 số hữu tỉ.
\nguon{Polish Mathematical Olympiad 2010}
\end{btt}

\begin{btt}
Chứng minh rằng nếu $n$ là tổng của ba số chính phương thì $3n$ được viết dưới dạng tổng của bốn số chính phương.
\nguon{Titu Andresscu}
\end{btt}

\begin{btt}
Cho các số nguyên $a$, $b$, $c$, $d$ thỏa mãn $a + b = c + d$. Chứng minh rằng $a^2 + b^2 + c^2 + d^2$ là tổng của ba số chính phương.
\end{btt}

\begin{btt}
 Cho các số nguyên ${a}, {b}$ và số nguyên tố ${p}$ thỏa mãn $\dfrac{{a}^{2}+{b}^{2}}{{p}} \in \mathbb{Z}$. Cho biết ${p}$ là
tổng của hai số chính phương. Chứng minh rằng $\dfrac{{a}^{2}+{b}^{2}}{{p}}$ cũng là tổng của hai số chính phương.
\nguon{Chọn học sinh giỏi tỉnh Thanh Hóa}
\end{btt}


\subsection*{Hướng dẫn bài tập tự luyện}

\begin{gbtt}
Cho $a,b$ và $c$ là các số nguyên thỏa mãn $ab + bc + ca = 1$. \\
Chứng minh rằng $(1+a^2)(1+b^2)(1+c^2)$ là số chính phương.
\loigiai{Từ giả thiết $ab+bc+ca=1,$ ta có
\begin{align*}
1+a^{2}&=ab+bc+ac+a^2=(a+b)(a+c),\\
1+b^{2}&=ab+bc+ac+b^2=(a+b)(b+c),\\
1+c^{2}&=ab+bc+ac+c^2=(c+b)(a+c).
\end{align*}
Lấy tích theo vế, ta có
$(1+a^2)(1+b^2)(1+c^2)=\left[(a+b)(a+c)(b+c)\right]^2$ là số chính phương. \\Bài toán được chứng minh.}
\end{gbtt}

\begin{gbtt}
Chứng minh rằng tổng của tích của bốn số tự nhiên liên tiếp và $1$ luôn là số chính phương.
\nguon{Vũ Hữu Bình}
\loigiai{
Dựa vào nhận xét 
\begin{align*}
    n(n+1)(n+2)(n+3)+1
    &=\left(n^2+3n\right)\left(n^2+3n+2\right)+1
    \\&=\left(n^2+3n+1\right)^2-1+1
    \\&=\left(n^2+3n+1\right)^2,
\end{align*}
bài toán đã cho được chứng minh.
}
\end{gbtt}

\begin{gbtt} \label{scp1.111}
Chứng minh rằng ${A}=1^{3}+2^{3}+3^{3}+\cdots+2016^{3}$ là số chính phương.
\loigiai{
Ta nhận thấy rằng
\begin{align*}
    A&=\left(\dfrac{1}{2}\right)^{2}\cdot 4\cdot 1+\left(\dfrac{2}{2}\right)^{2}\cdot 4 \cdot 2+\left(\dfrac{3}{2}\right)^{2}\cdot 4\cdot 3+\left(\dfrac{4}{2}\right)^{2}\cdot 4\cdot 4+\cdots+\left(\dfrac{2016}{2}\right)^{2}\cdot 4\cdot 2016
    \\&=\left(\dfrac{1}{2}\right)^{2}\left(2^{2}-0^{2}\right)+\left(\dfrac{2}{2}\right)^{2}\left(3^{2}-1^{2}\right)+\left(\dfrac{3}{2}\right)^{2}\left(4^{2}-2^{2}\right)+\cdots+\left(\dfrac{2016}{2}\right)^{2}\left(2017^{2}-2015^{2}\right)
    \\&=\left(\dfrac{1\cdot 2}{2}\right)^{2}-\left(\dfrac{0\cdot 1}{2}\right)^{2}+\left(\dfrac{2\cdot 3}{2}\right)^{2}-\left(\dfrac{1\cdot 2}{2}\right)^{2}+\cdots+\left(\dfrac{2016\cdot 2017}{2}\right)^{2}-\left(\dfrac{2015\cdot 2016}{2}\right)^{2}
    \\&=\left(\dfrac{2016\cdot 2017}{2}\right)^{2}
    \\&=(1008\cdot 2017)^{2}.
\end{align*}
Như vậy, ${A}$ là số chính phương. Bài toán được chứng minh.}
\end{gbtt}
\begin{gbtt}
Chứng minh rằng $N=\underbrace{11\ldots1}_{1995} \underbrace{00\ldots0}_{1994}5+1$ là một số chính phương.
\nguon{Vietnamese National Mathematical Olympiad 1995, Group A}	
\loigiai{
Biến đổi biểu thức đã cho, ta có
	\begin{align*}
	\underbrace{11\ldots1}_{1995} \underbrace{00\ldots0}_{1994}5+1&=\dfrac{{10}^{1995}-1}{9}\cdot\left({10}^{1995}+5\right)+1
	\\&=\dfrac{{\left({10}^{1995}\right)}^{2}+4\cdot{10}^{1995}-5}{9}+1\\
	&=\dfrac{{\left({10}^{1995}\right)}^{2}+4\cdot{10}^{1995}+4}{9}
	\\&={\left(\dfrac{{10}^{1995}+2}{3}\right)}^{2}.
	\end{align*} 
	Do $10^{1995}+2$ chia hết cho $3,$ ta suy ra $N$ là số chính phương. Bài toán được chứng minh.}	
\end{gbtt}

\begin{gbtt}
Chứng minh rằng $\underbrace {11\cdots11}_{2021}\underbrace {22\cdots22}_{2022}5$
là một số chính phương.
\loigiai{
Biến đổi biểu thức đã cho, ta có
	\begin{align*}
	\underbrace {11\cdots11}_{2021}\underbrace {22\cdots22}_{2022}5
	&=10^{2023}\cdot\underbrace{11\cdots11}_{2021}+10\cdot\underbrace{22\cdots22}_{2022}+5
	\\&=\dfrac{10^{2023}\tron{10^{2021}-1}}{9}+\dfrac{20\tron{10^{2022}-1}}{9}+5
	\\&=\dfrac{10^{4024}-10^{2023}+20\cdot 10^{2022}-20+45}{9}
	\\&=\dfrac{10^{4024}+10\cdot 10^{2022}+25}{9}
	\\&=\tron{\dfrac{10^{2012}+5}{3}}^2.
	\end{align*} 
	Do $10^{2012}+5$ chia hết cho $3,$ ta suy ra $N$ là số chính phương. Bài toán được chứng minh.}
\end{gbtt}

\begin{gbtt} Chứng minh rằng với mỗi số nguyên $n \geq 6$ thì $$a_{{n}}=1+\dfrac{2\cdot6 \cdot10 \cdots(4 n-2)}{(n+5)(n+6) \cdots(2 n)}$$ là một
số chính phương.
\nguon{Chuyên Đại học Sư phạm Hà Nội 2014}
\loigiai{
Với mọi $n\ge 6,$ ta nhận thấy rằng
$$
\begin{aligned}
a_{n} &=1+\dfrac{2^{n} \cdot[1\cdot3 \cdot5 \cdots(2 n-1)] \cdot(n+4) !}{(2 n) !}
\\&=1+\dfrac{2^{n}(n+4) !}{2 \cdot 4 \cdot 6 \cdots 2 n} \\
&=1+\dfrac{2^{n} \cdot 1 \cdot 2 \cdot 3 \cdots n(n+1)(n+2)(n+3)(n+4)}{2^{n} \cdot 1 \cdot 2 \cdot 3 \cdot 4 \cdots n} \\
&=1+(n+1)(n+2)(n+3)(n+4) \\
&=1+\left(n^2+5n+4\right)\left(n^2+5n+6\right) \\
&=\left(n^{2}+5 n+5\right)^{2}-1+1
\\&=\left(n^{2}+5 n+5\right)^{2}.
\end{aligned}
$$
Như vậy, $a_n$ là số chính phương với mọi $n\ge 6.$ Bài toán được chứng minh.}
\end{gbtt}

\begin{gbtt}
Cho 2 số nguyên ${a}, {b}$ thỏa mãn ${a}^{2}+{b}^{2}+1=2({ab}+{a}+{b})$. Chứng minh $a$ và $b$ là
hai số chính phương liên tiếp.

\nguon{Chuyên Đại học Sư phạm Hà Nội 2016}

\loigiai{
Từ giả thiết, ta có \begin{align*}
    a^{2}+b^{2}+1=2(ab+a+b) 
    &\Leftrightarrow a^2+b^2+1-2ab+2a-2 b=4 a\\
    &\Leftrightarrow(a-b+1)^2=4a.
\end{align*}
Ta được $a$ là số chính phương. Đặt ${a}={x}^{2},$ ở đây $x$ là số tự nhiên. Phép đặt này cho ta
$$
\left(x^{2}-b+1\right)^{2}=4 x^{2} \Rightarrow x^2-b+1=2 x \Rightarrow b=(x-1)^{2}.
$$
Vậy $a$ và $b$ là hai số chính phương liên tiếp. Bài toán được chứng minh.}
\end{gbtt}

\begin{gbtt}
Tìm các số nguyên $n$ sao cho $A=n^{4}+n^{3}+n^{2}$ là số chính phương.
\loigiai{Ta có ${A}={n}^{4}+{n}^{3}+{n}^{2}={n}^{2}\left({n}^{2}+{n}+1\right).$ Ta xét các trường hợp sau.
\begin{enumerate}
    \item Nếu $A=0,$ ta thu được $n=0.$ 
    \item Nếu ${A} \neq 0$ thì ${n}^{2}+{n}+1$ là số chính phương. Đặt $n^{2}+n+1=k^{2},$ với $k$ là số tự nhiên. Ta có
\begin{align*}
4\left(n^{2}+n+1\right)=4 k^{2} &\Leftrightarrow(2 n+1)^{2}+3=(2 k)^{2} \\&\Leftrightarrow(2k-2 n-1)(2 k+2 n+1)=3. \end{align*}
Tới đây, ta lập được bảng giá trị sau.
\begin{center}
    \begin{tabular}{c|c|c}
       $2k-2n-1$  & $2k+2n+1$ & $n$ \\
       \hline
       $-3$  & $-1$ & $0$ \\
       \hline       
       $-1$  & $-3$ & $-1$ \\
        \hline
       $1$  & $3$ & $0$ \\
       \hline       
       $3$  & $1$ & $-1$        
    \end{tabular}
\end{center}
Giải phương trình ước số trên, ta tìm được $n=-1$ và $n=0,$ nhưng  $A\ne 0$ nên ta chỉ chọn $n=-1.$ 
\end{enumerate}
Tóm lại, tất cả các giá trị thỏa mãn đề bài của $n$ là $n=-1$ và $n=0.$}
\end{gbtt}

\begin{gbtt}
Tìm tất cả các số nguyên $n$ sao cho $$A=n^6-3n^5-3n^4+20n^3-48n^2+96n-80$$ là số lập phương.
\loigiai{
Ta phân tích được $A=(n-2)^3\left(n^3+3n^2+3n+10\right).$ Với $A=0,$ ta có $n=2,$ còn với $A \neq 0$ thì $$n^3+3n^2+3n+10$$ phải là số lập phương. Đặt $n^3+3n^2+3n+10=m^3.$ Ta có
$$
(n+1)^3+9=m^3\Leftrightarrow (m-n-1)\left[m^2+m(n+1)+(n+1)^2\right]=9.
$$
Do $m^2+m(n+1)+(n+1)^2\ge 0,$ ta xét các trường hợp sau
\begin{enumerate}
    \item Với $m-n-1=1$ và $m^2+m(n+1)+(n+1)^2=9,$ ta có $m=n+2.$ Như vậy
     $$(n+2)^2+(n+2)(n+1)+(n+1)^2=9\Leftrightarrow 3n^2+9n-2=0.$$
     Phương trình trên không có nghiệm nguyên. Trường hợp này không xảy ra.
    \item Với $m-n-1=3$ và $m^2+m(n+1)+(n+1)^2=3,$ ta có $m=n+4.$ Như vậy
    \begin{align*}
        (n+4)^2+(n+4)(n+1)+(n+1)^2=9\Leftrightarrow 3(n+1)(n+4)=0\Leftrightarrow\hoac
          {&n=-1  \\
          &n=-4.}
    \end{align*}
    \item Với $m-n-1=9$ và $m^2+m(n+1)+(n+1)^2=1,$ ta có $m=n+10.$ Như vậy 
     $$(n+10)^2+(n+10)(n+1)+(n+1)^2=9\Leftrightarrow 3n^2+33n+102=0.$$
     Phương trình trên không có nghiệm nguyên. Trường hợp này không xảy ra.     
\end{enumerate}
Tóm lại, tất cả các giá trị thỏa mãn đề bài của $n$ là $n=-1$ và $n=-4.$}
\end{gbtt}

\begin{gbtt}
Chứng minh rằng không tồn tại số nguyên dương $n$ sao cho tích
$$\left(1^4+1^2+1\right)\left(2^4+2^2+1\right)\cdots\left(n^4+n^2+1\right)$$
là một số chính phương.
\nguon{Tuymada 2019}
\loigiai{Khai triển
$k^4+k^2+1=\left(k^2+k+1\right)\left(k^2-k+1\right),\forall k\in \mathbb{N^*}$
cho ta nhận xét
\begin{align*}
    &{\quad}\left(1^4+1^2+1\right)\left(2^4+2^2+1\right)\cdots\left(n^4+n^2+1\right)=3^2\cdot7^2\cdots\left(n^2-n+1\right)^2\left(n^2+n+1\right).
\end{align*}
Từ đây, ta suy ra $n^2+n+1$ là số chính phương. Đặt $n^2+n+1=m^2.$ Phép đặt này cho ta
\begin{align*}
    4n^2+4n+4=4m^2
    &\Rightarrow (2m+1)^2+3=(2m)^2
    \\&\Rightarrow (2m-2n-1)(2m+2n+1)=3
    \\&\Rightarrow \heva{&2m-2n-1=1 \\ &2m+2n+1=3}
    \\&\Rightarrow \heva{&m=1 \\ &n=0.}
\end{align*}
Do giả thiết $n$ nguyên dương, $n=0$ không thỏa mãn. Bài toán được chứng minh.}
\end{gbtt}

\begin{gbtt}
Tìm tất cả các số các số tự nhiên $n$ thỏa mãn $2 n+1,\ 3 n+1$ là các số chính phương và $2n+9$ là số
nguyên tố. 
\nguon{Chuyên Toán Hà Nội 2017}
\loigiai{
Từ giả thiết, ta có thể đặt $2n+1=x^2,3n+1=y^2,$ ở đây $x,y$ là các số nguyên dương. \\
Ta sẽ chọn các tham số $a,b$ sao cho
$$ax^2+by^2=a(2n+1)+b(3n+1)=2n+9.$$
Giải hệ $\heva{&2a+3b=2 \\ &a+b=9},$ ta được $a=25,b=-16.$ Bây giờ, ta xét phân tích
$$2n+9=25x^2-16y^2=(5x-4y)(5x+4y).$$
Do $2n+9$ là số nguyên tố nên trong $5x-4y,\ 5x+4y$ phải có một số bằng $1,$ nhưng vì $$0<5x-4y<5x+4y$$ nên $5x-4y=1.$ Kết hợp với phép đặt $2n+1=x^2,3n+1=y^2,$ ta thu được hệ nghiệm nguyên dương
\begin{align*}
\heva{&5x-4y=1 \\ &3x^2-2y^2-1=0}
&\Leftrightarrow \heva{&y=\dfrac{5x-1}{4} \\ & 3x^2-2\left(\dfrac{5x-1}{4}\right)^2-1=0}    
\\&\Leftrightarrow \heva{&y=\dfrac{5x-1}{4} \\ & (x-1)(x-9)=0} \\&
\Leftrightarrow
\left[\begin{aligned}
     &\heva{x&=1 \\ y&=1}  \\
     &\heva{x&=9 \\ y&=11.} 
\end{aligned}\right.
\end{align*}
Tới đây, ta xét các trường  hợp sau.
\begin{enumerate}
    \item Với $x=1,$ ta được $n=0.$ Lúc này $2n+9=9$ không là số nguyên tố.
    \item Với $x=9,$ ta được $n=40.$ Lúc này $2n+9=89$ là số nguyên tố. 
\end{enumerate}
Như vậy, giá trị duy nhất của $n$ thỏa mãn là $n=40.$}
\end{gbtt}

\begin{gbtt}
Cho 2 số hữu tỉ $a, b$ thoả mãn điều kiện $a^{3}+4 a^{2} b=4 a^{2}+b^{4}$. Chúng minh rằng $\sqrt{a}-1$ là bình phương của 1 số hữu tỉ.
\nguon{Polish Mathematical Olympiad 2010}
\loigiai{
Ta viết lại giả thiết bài toán thành 
\begin{align*}
    a^{3}+4 a^{2} b+4 a b^{2}=4 a^{2}+4 a b^{2}+b^{4} &\Leftrightarrow a(a+2 b)^{2}=\left(2 a+b^{2}\right)^{2}\\&
    \Leftrightarrow a=\dfrac{\left(2 a+b^{2}\right)^{2}}{(a+2 b)^{2}}\\&
    \Leftrightarrow \sqrt{a}=\dfrac{2 a+b^{2}}{a+2 b}.
\end{align*}
Do đó ta có $\sqrt{a} \in \mathbb{Q} .$
Mặt khác ta lại có 
\[\sqrt{a}-1=\dfrac{2 a+b^{2}}{a+2 b}-1=\dfrac{a-2 b+b^{2}}{a+2 b}=\dfrac{\left(a-2 b+b^{2}\right)(a+2 b)}{(a+2 b)^{2}}.\tag{*}\]
Vì vậy ta cần chỉ ra $\left(a-2 b+b^{2}\right)(a+2 b)$ là bình phương của một số hữu tỉ, hay 
$$\tron{a-2 b+b^{2}}(a+2 b)a$$
là bình phương một số hữu tỉ. Thật vậy ta có $$\left(a-2 b+b^{2}\right)(a+2 b) a=a^{2} b^{2}+a^{3}+2 b^{3} a-4 a b^{2}.$$ 
Thay $a^{3}=4 a^{2}+b^{4}-4 a^{2} b$ vào ta có
$$\left(a-2 b+b^{2}\right)(a+2 b) a=a^{2} b^{2}+4 a^{2}+b^{4}-4 a^{2} b+2 b^{3} a-4 a b^{2}=\left(a b+b^{2}-2 a\right)^{2}.$$
Vậy $\left(a-2 b+b^{2}\right)(a+2 b)a$ là bình phương của 1 số hữu tỉ, lại do $a$ là bình phương 1 số hữu tỉ nên 
$$\left(a-2 b+b^{2}\right)(a+2 b)$$
cũng là bình phương một số hữu tỉ. Lúc này từ (*) ta suy ra điều phải chứng minh.}
\end{gbtt}

\begin{gbtt}
Chứng minh rằng nếu $n$ là tổng của ba số chính phương thì $3n$ được viết dưới dạng tổng của bốn số chính phương.
\nguon{Titu Andresscu}
\loigiai{
Giả sử $n = a^2 + b^2 + c^2$. Ta có
\begin{align*}
   3n & = 3a^2 + 3b^2 + 3c^2 \\
      & = \left( a^2 + b^2 + c^2 + 2ab + 2bc + 2ca \right) + \left( a^2 - 2ab + b^2 \right) + \left( b^2 - 2bc + c^2 \right) + \left( c^2 - 2ca + a^2 \right) \\
      & = (a + b + c)^2 + (a - b)^2 + (b - c)^2 + (c - a)^2.
\end{align*}
Bài toán được chứng minh. }
\end{gbtt}

\begin{gbtt}
 Cho các số nguyên $a$, $b$, $c$, $d$ thỏa mãn $a + b = c + d$. Chứng minh rằng $a^2 + b^2 + c^2 + d^2$ là tổng của ba số chính phương.
 \loigiai{
Từ giả thiết ta có
\begin{align*}
   a^2 + b^2 + c^2 + d^2 & = a^2 + b^2 + c^2 + d^2 + 2a(a + b - c - d) \\
   & = 3a^2 + b^2 + c^2 + d^2 + 2ab - 2ac - 2ad \\
   	& = \left( a^2 + 2ab + b^2 \right) + \left( a^2 - 2ac + c^2 \right) + \left( a^2 - 2ad + d^2 \right) \\
   & = (a + b)^2 + (a - c)^2 + (a - d)^2.
  \end{align*}
Bài toán được chứng minh.}
\end{gbtt}

\begin{gbtt}
 Cho các số nguyên ${a}, {b}$ và số nguyên tố ${p}$ thỏa mãn $\dfrac{{a}^{2}+{b}^{2}}{{p}} \in \mathbb{Z}$. Cho biết ${p}$ là
tổng của hai số chính phương. Chứng minh rằng $\dfrac{{a}^{2}+{b}^{2}}{{p}}$ cũng là tổng của hai số chính phương.
\nguon{Chọn học sinh giỏi tỉnh Thanh Hóa}
\loigiai{
Từ giả thiết, ta có thể đặt ${p}={c}^{2}+{d}^{2}$ với ${c}, {d}$ là các số nguyên. Phép đặt trên cho ta $$\dfrac{a^{2}+b^{2}}{p}=\dfrac{\left(a^{2}+b^{2}\right)\left(c^{2}+d^{2}\right)}{p^{2}}=\left(\dfrac{a d+b c}{p}\right)^{2}+\left(\dfrac{a c-b d}{p}\right)^{2}.$$
Một cách tương tự, ta cũng có $$\dfrac{a^{2}+b^{2}}{p}=\dfrac{\left(a^{2}+b^{2}\right)\left(c^{2}+d^{2}\right)}{p^{2}}=\left(\dfrac{a d-b c}{p}\right)^{2}+\left(\dfrac{a c+b d}{p}\right)^{2}.$$
Hơn thế nữa, ta xét tích dưới đây
$$({ac}+{bd})({ac}-{bd})={a}^{2} {c}^{2}-{b}^{2} {d}^{2}={a}^{2}\left({c}^{2}+{d}^{2}\right)-{d}^{2}\left({a}^{2}+{b}^{2}\right).$$
Do ${p}={c}^{2}+{d}^{2}$ và $p\mid \left({{a}^{2}+{b}^{2}}\right)$ nên $p\mid (a c+b d)(a c-b d).$ Như vậy, hoặc $p$ là ước của $ac+bd,$ hoặc $p$ là ước của $ac-bd.$ Đối chiếu từng trường hợp với một trong hai biến đổi trên, ta thu được điều phải chứng minh.}
\begin{luuy}
\nx \\
Biến đổi ở câu trên đều có liên quan đến định thức $Brahmagupta - Fibonacci$
\[\left(a^2+b^2\right)\left(x^2+y^2\right)=\left(ax+by\right)^2+\left(ay-bx\right)^2.\]
\end{luuy}
\end{gbtt}

\section{Ứng dụng của đồng dư thức}
\subsection*{Lí thuyết}
Trong mục này, chúng ta sẽ tìm hiểu về các ứng dụng về đồng dư thức đối với các bài toán chứa yếu tố số chính phương, lập phương. Tác giả sẽ phát biểu lại lại các kiến thức đã học ở \chu{chương I} dưới dạng khác.

\begin{enumerate}
    \item Một số chính phương bất kì chỉ có thể
    \begin{multicols}{2}
            \begin{itemize}
            \item Đồng dư với $0$ hoặc $1$ theo modulo $3.$
            \item Đồng dư với $0$ hoặc $1$ theo modulo $4.$
            \item Đồng dư với $0,1$ hoặc $4$ theo modulo $5.$
            \item Đồng dư với $0,1,2$ hoặc $4$ theo modulo $7.$
            \item Đồng dư với $0,1$ hoặc $4$ theo modulo $8.$
            \item Đồng dư với $0,1,3,4,5,9$ theo modulo $11.$            
        \end{itemize}
    \end{multicols}
    \item Một số lập phương bất kì chỉ có thể
        \begin{itemize}
            \item Đồng dư với $-1,0$ hoặc $1$ theo modulo $7.$
            \item Đồng dư với $-1,0$ hoặc $1$ theo modulo $9.$
        \end{itemize} 
    \item Một lũy thừa bậc bốn bất kì chỉ có thể 
        \begin{itemize}
            \item Đồng dư với $0$ hoặc $1$ theo modulo $5.$
            \item Đồng dư với $0$ hoặc $1$ theo modulo $16.$
        \end{itemize}
\end{enumerate}

\subsection*{Ví dụ minh họa}
\begin{bx}
Chứng minh các số sau đây không thể là số chính phương
\begin{multicols}{2}
\begin{enumerate}[a,]
    \item $A=2020^{2021}+2021^{2020}.$
    \item $B=3^{57}+7^{53}.$
\end{enumerate}
\end{multicols}
\loigiai{
\begin{enumerate}[a,]
    \item Xét trong hệ đồng dư modulo $3,$ ta có
    \begin{align*}
        2020^{2021}+2021^{2020}
        &=(3\cdot673+1)^{2021}+(3\cdot674-1)^{2020}
        \equiv 1^{2021}+(-1)^{2020}
        \equiv 2\pmod{3}.  
    \end{align*}
    Không có số chính phương nào đồng dư $2$ theo modulo $3,$ thế nên $A$ không là số chính phương.
    \item Xét trong hệ đồng dư modulo $7,$ ta có
    \begin{align*}
        3^{57}+7^{53}
        &=27\cdot\left(3^6\right)^9+7^{53}
        \equiv 27\cdot(7\cdot104+1)^9
        \equiv 27\equiv 6\pmod{7}.    
    \end{align*}
    Không có số chính phương nào đồng dư $6$ theo modulo $7,$ thế nên $B$ không là số chính phương.    
    \end{enumerate}}
\end{bx}

\begin{bx}
Chứng minh rằng không tồn tại số tự nhiên $a$ sao cho $a^2+a=2010^{2009}.$
\nguon{Phổ thông Năng khiếu}
\loigiai{Giả sử rằng tồn tại số tự nhiên $a$ thỏa mãn. Giả sử này cho ta
$$4a^2+4a=4\cdot2010^{2009}\Leftrightarrow \tron{2a+1}^2= 4\cdot2010^{2009}+1.$$
Xét trong hệ đồng dư modulo $7$, ta có
$$4\cdot2010^{2009}+1\equiv4\cdot1^{2009}+1\equiv4+1\equiv5\pmod{7}\Rightarrow\tron{2a+1}^2\equiv5\pmod{7}.$$
Không có số chính phương nào đồng dư $5$ theo modulo $7.$ Giả sử sai. Bài toán được chứng minh.}
\end{bx}

\begin{bx}
Tìm tất cả các số nguyên tố $p,q$ thỏa mãn $p^2-q^2-1$ là số chính phương.
\nguon{Adrian Andreescu}
\loigiai{Giả sử tồn tại các số nguyên tố $p,q$ thỏa mãn đề bài. Ta xét các trường hợp sau.
\begin{enumerate}
    \item Với $p>q\ge 3,$ do $p$ và $q$ cùng lẻ, nên $p^2\equiv q^2\equiv 1\pmod{4}.$ Ta có
    $$p^2-q^2-1\equiv 1-1-1=-1\equiv 3 \pmod{4}.$$ 
    Trong trường hợp này, $p^2-q^2-1$ không chính phương.
    \item Với $q=2,$ ta đặt $k^2=p^2-q^2-1.$ Ta lần lượt suy ra
    $$k^2=p^2-5\Rightarrow 5=(p-k)(p+k)\Rightarrow \left\{\begin{matrix} p+k=5\\ p-k=1 \end{matrix}\right. \Rightarrow p=3\Rightarrow q=2.$$
\end{enumerate}
    Kiểm tra trực tiếp, ta thấy $(p,q)=(3,2)$ là cặp số duy nhất thỏa mãn đề bài.}
\end{bx}

\begin{bx}
Tìm số nguyên dương $n>2$ nhỏ nhất thỏa mãn tính chất:
\begin{it}
Tồn tại số $n$ số chính phương liên tiếp có tổng cũng là một số chính phương.
\end{it}
\nguon{Allessandro Ventullo}
\loigiai{Chú ý rằng $18^{2}+19^{2}+\cdots+28^{2}=77^{2}$, vì thế, ta sẽ đi chứng minh $n\ge 11.$ Ta xét các trường hợp sau.
\begin{enumerate}
    \item[i,] $(k-1)^{2}+k^{2}+(k+1)^{2}=3 k^{2}+2\equiv 2 \pmod{3},$
    \item[ii,] $(k-1)^{2}+k^{2}+(k+1)^{2}+(k+2)^{2}=4k^{2}+4 k+6 \equiv 2 \pmod{4},$ 
    \item[iii,] $(k-2)^{2}+(k-1)^{2}+\cdots+(k+2)^{2}+(k+3)^{2}=6k(k+1)+19 \equiv 3 \pmod{4},$     
    \item[iv,] $(k-3)^{2}+(k-2)^{2}+\cdots+(k+4)^{2}=8k(k+1)+44 \equiv 12 \pmod{16}.$
\end{enumerate}
Các đồng dư thức trên chứng tỏ $n$ khác $3,4,6,8.$ Mặt khác
\begin{enumerate}
    \item[v,] $(k-2)^{2}+(k-1)^{2}+\cdots+(k+2)^{2}=5\left(k^{2}+2\right)$ chia hết cho $5$ nhưng không chia hết cho $25,$
    \item[vi,] $(k-3)^{2}+(k-2)^{2}+\cdots+(k+3)^{2}=7\left(k^{2}+4\right)$ chia hết cho $7$ nhưng không chia hết cho $49,$
    \item[vii,] $(k-4)^{2}+(k-3)^{2}+\cdots+(k+4)^{2}=3\left(3 k^{2}+20\right)$ chia hết cho $3$ nhưng không chia hết cho $9,$
    \item[viii,] $(k-4)^{2}+(k-3)^{2}+\cdots+(k+5)^{2}=5\left(2 k^{2}+2 k+17\right)$ chia hết cho $5$ nhưng không cho $25.$
\end{enumerate}
Các lập luận trên chứng tỏ $n$ khác $5,7,9,10.$ Như vậy $n=11$ là đáp số bài toán.}
\end{bx}

\begin{bx}
Tìm các số nguyên dương $n$ thỏa mãn $2^n+12^n+2011^n$ là số chính phương.
\nguon{Adrian Andreescu, Turkey Mathematical Olympiad 2006}
\loigiai{Bằng kiểm tra trực tiếp, ta chỉ ra $n=1$ là đáp số. Ta sẽ chứng minh mọi $n\ge 2$ không thỏa mãn đề bài. 
\begin{enumerate}
    \item Với $n$ chẵn, ta đặt $n=2k,$ trong đó $k$ nguyên dương. Ta có
    \begin{align*}
        2^n+12^n+2011^n&=4^k+12^{n}+\left(3\cdot670+1\right)^{n} \\&\equiv 1+0+1 
        \\&=2 \pmod{3}
    \end{align*}
    nên $2^n+12^n+2011^n$ không chính phương.
    \item Với $n$ lẻ, ta đặt $n=2k+1,$ trong đó $k$ nguyên dương. Ta có
    \begin{align*}
        2^n+12^n+2011^n&=4\cdot 2^{n-2}+12^{n}+(4\cdot 503-1)^{2k+1} \\&\equiv (-1)^{2k+1}\\&\equiv 3 \pmod{4}
    \end{align*}
    nên $2^n+12^n+2011^n$ không chính phương.    
\end{enumerate}
Như vậy, $n=1$ là đáp số bài toán.}
\end{bx}

\subsection*{Bài tập tự luyện}

\begin{btt}
Cho số tự nhiên $n$ thỏa mãn $n(n+1)+6$ không chia hết cho $3$.\\ Chứng minh rằng $2n^{2} +n+8$ không phải là số chính phương.
\nguon{Chuyên Toán Hà Nội 2013}
\end{btt}

\begin{btt}
Tồn tại không số nguyên dương $n$ để $n^5-n+2$ là số chính phương?
\end{btt}

\begin{btt}
Cho số tự nhiên $N=1\cdot3\cdot5\cdot7\cdots2107$. Chứng minh rằng trong $3$ số nguyên liên tiếp $2N-1,  2N$ và $2N+1,$ không có số nào là số chính phương.
\end{btt}

\begin{btt}
Tìm tất cả số tự nhiên $n$ sao cho $12\cdot n!+11^n+2$ là số chính phương.
\end{btt}

\begin{btt}
Tìm số tự nhiên $n \ge 1$ sao cho tổng $1!+2!+3!+\cdots+n!$ là một số chính phương.
\end{btt}

\begin{btt}
Cho số nguyên không âm $A.$ Hãy xác định $A$ biết rằng trong $3$ mệnh đề $\mathcal{P}, \mathcal{Q}, \mathcal{R}$ dưới đây có 2 mệnh đề đúng và 1 mệnh đề sai
\begin{itemize}
    \item $\mathcal{P}:$ "$A+51$ là số chính phương".
    \item $\mathcal{Q}:$ "Chữ số tận cùng của $A$ là $1$".
    \item $\mathcal{R}:$ "$A-38$ là số chính phương".
\end{itemize}
\nguon{Chuyên Toán Phổ thông Năng khiếu 2000}
\end{btt}

\begin{btt}
Giả sử tồn tại hai số nguyên $x,y$ sao cho $x^2-4y$ và $x^2+4y$ đều là số chính phương. Chứng minh rằng $y$ chia hết cho $6.$
\end{btt}

\begin{btt}
Tìm số nguyên tố $p$ để $2p^4-p^2+16$ là số chính phương
\nguon{Leningrad Mathematical Olympiad 1980}
\end{btt}

\begin{btt}
Chứng minh rằng với mọi số nguyên tố $p>3,$ ta có thể biểu diễn $\dfrac{p^6-7}{3}+2p^2$ thành tổng của hai số lập phương.
\nguon{Titu Andreescu}
\end{btt}

\begin{btt}
Tồn tại không các số nguyên tố $p,q,r$ thỏa mãn $\left(p^2-7\right)\left(q^2-7\right)\left(r^2-7\right)$
là số chính phương?
\nguon{Titu Andreescu}
\end{btt}

\begin{btt}
Tìm các số nguyên tố $p,q,r$ thỏa mãn $p+q+r$ không chia hết cho $3,$ đồng thời cả $p+q+r$ và $pq+qr+rp+3$ đều là số chính phương.
\nguon{Turkey Junior Balkan Mathematical Olympiad 2013}
\end{btt}

\begin{btt}
Tồn tại hay không các số nguyên $a,b$ sao cho $$a^5b+3\text{ và }ab^5+3$$
là các số lập phương?
\nguon{USA Junior Mathematical Olympiad 2013}
\end{btt}

\begin{btt}
Tồn tại hay không các số nguyên $a,b,c$ sao cho $$ab^2c+2,\: bc^2a+2\text{ và }ca^2b+2$$
là các số chính phương?
\nguon{China Western Mathematical Olympiad 2013}
\end{btt}

\begin{btt}
Cho tập các số nguyên $S=\{n;n+1;\cdots;n+5\}.$ Chứng minh rằng không thể chia $S$ thành hai tập $A$ và $B$ giao nhau khác rỗng, sao cho tích các phần tử trong $A$ bằng tích các phần tử trong $B.$
\nguon{Cao Đình Huy}
\end{btt}

\begin{btt}
Tìm tất cả các số nguyên $a,b$ sao cho $a^4+b^4+(a+b)^4$ là số chính phương.
\nguon{Vũ Hữu Bình}
\end{btt}

\begin{btt}
Cho $a,b$ là các số nguyên dương. Đặt $A=\tron{a+b}^2-2a^2$ và $B=\tron{a+b}^2-2b^2.$ Chứng minh rằng $A$ và $B$ không thể đồng thời là số chính phương.
\nguon{Chuyên Đại học Sư phạm Hà Nội}
\end{btt}

\begin{btt}
Cho $d$ là một số nguyên dương khác $2,5,13.$ Chứng minh rằng ta có thể tìm được hai số nguyên phân biệt $a,b$ trong tập $\{2;5;13;d\}$ sao cho $ab-1$ không phải là số chính phương.
\end{btt}

\begin{btt}
Với mọi số nguyên dương $n,$ chứng minh rằng $$2020^{4n}+2021^{4n}+2022^{4n}+2023^{4n}$$ không phải là số chính phương.
\end{btt}

\begin{btt}
Tìm tất cả các cặp số tự nhiên $(m,n)$ sao cho $2^m3^n-1$ là số chính phương.
\nguon{Tạp chí Toán học và Tuổi trẻ, tháng 7 năm 2017}
\end{btt}

\begin{btt}
Tìm các số nguyên dương $m,n$ thỏa mãn $A=6^m+2^n+2$ là số chính phương.
\nguon{Croatian Mathematical Olympiad 2009}
\end{btt}

\begin{btt}
Tìm tất cả các số nguyên tố $p$ thỏa mãn $2^p+5\cdot 3^p$ là số chính phương.
\end{btt}

\begin{btt}
Tìm tất cả các số nguyên tố $p$ thỏa mãn $7^p+9^p+259$ là số lập phương.

\end{btt}

\subsection*{Hướng dẫn bài tập tự luyện}

\begin{gbtt}
Cho số tự nhiên $n$ thỏa mãn $n(n+1)+6$ không chia hết cho $3$.\\ Chứng minh rằng $2n^{2} +n+8$ không phải là số chính phương.
\nguon{Chuyên Toán Hà Nội 2013}
\loigiai{
Từ giả thiết $n(n+1)+6$ không chia hết cho $3,$ ta có $n\equiv 1\pmod{3}.$ Như vậy
$$2n^2+n+8\equiv 2+1+8\equiv 2\pmod{3}.$$ 
Không có số chính phương nào chia $3$ dư $2,$ và bài toán được chứng minh.}
\end{gbtt}

\begin{gbtt}
Tồn tại không số nguyên dương $n$ để $n^5-n+2$ là số chính phương?
\loigiai{
Ta nhận thấy
$n^5-n+2=n\left(n^4-1\right)+2.$
Ta đã biết, một lũy thừa mũ $4$ khi chia cho $5$ chỉ có thể dư $0$ hoặc $1.$ Ta xét các trường hợp dưới đây.
\begin{enumerate}
    \item Nếu $n$ chia hết cho $5,$ hiển nhiên $n\left(n^4-1\right)$ chia hết cho $5.$
    \item Nếu $n$ không chia hết cho $5,$ ta có $n^4\equiv 1 \pmod{5},$ và vì thế $5\mid n\left(n^4-1\right).$
\end{enumerate}
Hai lập luận trên giúp ta chỉ ra
$n\left(n^4-1\right)+2\equiv 2\pmod{5}.$
Tuy nhiên, không có số chính phương nào đồng dư $2$ theo modulo $5.$ Câu trả lời là phủ định.}
\end{gbtt}

\begin{gbtt}
Cho số tự nhiên $N=1\cdot3\cdot5\cdot7\cdots2107.$
Chứng minh rằng trong $3$ số nguyên liên tiếp $2N-1,  2N$ và $2N+1,$ không có số nào là số chính phương.
\loigiai{
    Do $N$ là số lẻ, ta có $2N\equiv 2\pmod{4},$ thế nên $2N$ không là số chính phương, và khi đó
    $$2N+1\equiv 3\pmod{4}.$$
    Không có số chính phương nào chia cho $4$ dư $3,$ thế nên $2N+1$ cũng không phải số chính phương. \\
    Hơn thế nữa, rõ ràng $N$ chia hết cho $3,$ và theo đó
    $$2N-1\equiv -1\equiv 2\pmod{3}.$$
    Không có số chính phương nào chia cho $3$ dư $2,$ và $2N-1$ cũng không phải số chính phương.
    \\Bài toán được chứng minh.}
\end{gbtt}

\begin{gbtt}
Tìm tất cả số tự nhiên $n$ sao cho $12\cdot n!+11^n+2$ là số chính phương.
\loigiai{
Với $n\ge 5,$ ta có $12\cdot n!$ chia hết cho $5$. Suy ra \[
12\cdot n!+11^n+2\equiv 0+1+2\equiv 3\pmod{5}.\] 
Điều này chứng tỏ $12\cdot n!+11^n+2$ không là số chính phương với mọi $n\ge 5.$\\
Với $n=0,1,2,3,4,$ ta lập bảng giá trị sau đây.
\begin{center}
    \begin{tabular}{c|c|c|c|c|c}
        $n$ &  $0$ & $1$ & $2$ & $3$ & $4$\\
        \hline
         $12\cdot n!+11^n+2$ & $15$ & $25$ & $147$ & $1405$ & $14931$
    \end{tabular}
\end{center}
Căn cứ vào bảng, ta nhân thấy $n=1$ là số tự nhiên duy nhất thỏa mãn yêu cầu.}
\end{gbtt}

\begin{gbtt}
Tìm số tự nhiên $n \ge 1$ sao cho tổng $1!+2!+3!+\cdots+n!$ là một số chính phương.
\loigiai{
Trong bài toán này, ta xét các trường hợp sau đây.
\begin{enumerate}
	\item Nếu $n=1,$ ta có $1!=1=1^2$ là số chính phương.
	\item Nếu $n=2,$ ta có $1!+2!=3$ không là số chính phương.
	\item Nếu $n=3,$ ta có  $1!+2!+3!=1+1\cdot2+1\cdot2\cdot3=9=3^2$ là số chính phương.
    \item Nếu $n \ge 4,$ do $m!$ chia hết cho $5$ với mọi $n\ge 5$ nên $$1!+2!+3!+\cdots+n!\equiv 1!+2!+3!+4!=33\equiv 3\pmod{5}.$$ 
    Không có số chính phương nào chia cho $5$ được dư là $3.$ Trường hợp này không xảy ra.
\end{enumerate}
Kết luận, $n=1$ và $n=3$ là hai giá trị của $n$ thỏa mãn đề bài.
}
\end{gbtt}

\begin{gbtt}
Cho số nguyên không âm $A.$ Hãy xác định $A$ biết rằng trong $3$ mệnh đề $\mathcal{P}, \mathcal{Q}, \mathcal{R}$ dưới đây có 2 mệnh đề đúng và 1 mệnh đề sai
\begin{itemize}
    \item $\mathcal{P}:$ "$A+51$ là số chính phương".
    \item $\mathcal{Q}:$ "Chữ số tận cùng của $A$ là $1$".
    \item $\mathcal{R}:$ "$A-38$ là số chính phương".
\end{itemize}
\nguon{Chuyên Toán Phổ thông Năng khiếu 2000}
\loigiai{
Giả sử mệnh đề $\mathcal{Q}$ đúng. Xét trong hệ đồng dư modulo $5$, ta thu được
$$A+1 \equiv 2\pmod{5}, \qquad A-38\equiv -7\equiv3\pmod{5}.$$
Không có số chính phương nào đồng dư với $2$ hoặc $3$ theo modulo $5$. Điều này trái với mệnh đề $\mathcal{P}$ và  $\mathcal{R}$. Do đó $2$ cặp mệnh đề $\tron{\mathcal{Q},\mathcal{P}}$ và $\tron{\mathcal{Q},\mathcal{R}}$ không thể cùng đúng. Từ đây, ta suy ra  $\mathcal{P},\mathcal{R}$ là $2$ mệnh đề đúng. Đặt $A+51=x^2$ và $A-38=y^2.$ Trừ theo vế, ta được
$$x^2-y^2=89\Leftrightarrow\tron{x-y}\tron{x+y}=89.$$
Do $0<x-y<x+y$ nên bắt buộc $x-y=1$ và $x+y=89.$ Ta chỉ ra $x=45$ và $A=1974$ từ đây.}
\end{gbtt}

\begin{gbtt}
Giả sử tồn tại hai số nguyên $x,y$ sao cho $x^2-4y$ và $x^2+4y$ đều là số chính phương. Chứng minh rằng $y$ chia hết cho $6.$
\loigiai{
Ta chia bài toán thành các bước làm sau.
\begin{enumerate}[\color{tuancolor}\bf\sffamily Bước 1.]
    \item Ta chứng minh $y$ là số chẵn.\\ Giả sử rằng $y$ là số lẻ. Khi đó do $x^2\equiv 0,1,4\pmod{8}$ nên
    $$x^2+4y\equiv 4,5,0\pmod{8}.$$
    Trường hợp $x^2+4y\equiv 5\pmod{8}$ bị loại. Theo đó, $x$ là số chẵn. Ta đặt $x=2z,$ thế thì
    $$x^2-4y=4\tron{z^2-y},\quad x^2+4y=4\tron{z^2+y}$$
    đều là các số chính phương, và $z^2-y,z^2+y$ cũng như vậy. Ta xét bảng đồng dư modulo $4.$
    \begin{center}
        \begin{tabular}{c|c|c|c|c}
          $\quad z^2\quad $ & $0$ & $1$ & $0$ & $1$ \\
          \hline
          $\quad y\quad $  & $1$ & $1$ & $3$ & $3$ \\
          \hline
          $\quad z^2-y\quad$ & $3$ & $0$ & $1$ & $2$ \\
          \hline
          $\quad z^2+y\quad$ & $1$ & $2$ & $3$ & $0$  \\
        \end{tabular}
    \end{center}
    Ta dễ dàng thấy giả sử phản chứng bị sai từ đây. Ta có $y$ chẵn.
    \item Ta chứng minh $y$ chia hết cho $3.$\\ Ta giả sử phản chứng rằng
    $y\equiv 1,2\pmod{3}.$ Ta lập bảng đồng dư theo modulo $3$
    \begin{center}
        \begin{tabular}{c|c|c|c|c}
          $\quad x^2\quad $ & $0$ & $1$ & $0$ & $1$ \\
          \hline
          $\quad y\quad $  & $1$ & $1$ & $2$ & $2$ \\
          \hline
          $\quad x^2-4y\quad$ & $2$ & $0$ & $1$ & $2$ \\
          \hline
          $\quad x^2+4y\quad$ & $1$ & $2$ & $2$ & $0$  
        \end{tabular}
    \end{center}   
    Ta dễ dàng thấy giả sử phản chứng bị sai từ đây. Ta có $y$ chia hết cho $3.$  
\end{enumerate}
Như vậy, $y$ chia hết cho $[2,3]=6.$ Bài toán được chứng minh.}
\end{gbtt}

\begin{gbtt}
Tìm số nguyên tố $p$ để $2p^4-p^2+16$ là số chính phương
\nguon{Leningrad Mathematical Olympiad 1980}
\loigiai{
Thử trực tiếp với $p=2$ và $p=3,$ ta thấy $p=3$ thỏa mãn. \\
Với $p>3,$ ta có $p^2\equiv 1 \pmod{3}$ do lúc này $p$ không là bội của $3.$ Nhận xét này cho ta
$$2p^{4}-p^{2}+16= 2\left(p^2\right)^2-p^2+16 \equiv 2-1+16=17\equiv 2\pmod{3}.$$ 
Không có số chính phương nào đồng dư $2$ theo modulo $3.$ Trường hợp này không xảy ra. \\
Như vậy, $p=3$ là số nguyên tố cần tìm.}
\end{gbtt}

\begin{gbtt}
Chứng minh rằng với mọi số nguyên tố $p>3,$ ta có thể biểu diễn $\dfrac{p^6-7}{3}+2p^2$ thành tổng của hai số lập phương.
\nguon{Titu Andreescu}
\loigiai{
Theo lí thuyết đã học, ta có $p^2\equiv 0,1 \pmod{3},$ nhưng do $p>3$ nên $p^2\equiv 1 \pmod{3}.$\\
Vì lẽ đó, ta có thể đặt $p^2=3n+1,$ ở đây $n$ là số nguyên dương. Phép đặt này cho ta
\begin{align*}
\dfrac{p^{6}-7}{3}+2 p^{2}&=\dfrac{\left(3n+1\right)^3-7}{3}+2(3n+1) 
\\&=9n^3+9n^2+9n
\\&=(n-1)^3+(2n+1)^3.
\end{align*}
Thông qua biểu diễn trên, bài toán được chứng minh.}
\end{gbtt}

\begin{gbtt}
Tồn tại không các số nguyên tố $p,q,r$ thỏa mãn $\left(p^2-7\right)\left(q^2-7\right)\left(r^2-7\right)$
là một số chính phương?
\nguon{Titu Andreescu}
\loigiai{Ta giả sử phản chứng rằng, tồn tại bộ ba $(p,q,r)$ thỏa mãn đề bài và cả $p\le q\le r.$ Rõ ràng $p^2-7<0$ khi và chỉ khi $p=2,$ và ngược lại, $p^2-7>0$ khi và chỉ khi $p\ge 3.$ Ta xét các trường hợp sau. 
\begin{enumerate}
    \item Với $p=2,$ do $\left(2^{2}-7\right)\left(q^{2}-7\right)\left(r^{2}-7\right)$ nguyên dương nên bắt buộc $q=2,r\ge 3.$ Lúc này 
    $$\left(p^{2}-7\right)\left(q^{2}-7\right)\left(r^{2}-7\right)=9(r^2-7)$$
    là số chính phương, và $r^2-7$ cũng là số chính phương. Ta đặt $k^2=r^2-7,$ khi đó 
    $$(r-k)(r+k)=7\Rightarrow r-k=1, r+k=7.$$
    Ta tìm được $r=4,$ trái với giả thiết $r$ nguyên tố.
    \item Với $p\ge 3,$ xuất phát từ tính chất quen thuộc
    $p^2,q^2,r^2\equiv 1 \pmod{8},$
    ta chỉ ra rằng
    $$p^2-7,q^2-7,r^2-7\equiv 2 \pmod{8},$$   
    Ta đặt $p^2-7=8a+2,q^2-7=8b+2,r^2-7=8c+2.$ Ta có
    $$\tron{p^2-7}\tron{q^2-7}\tron{r^2-7}=8(4a+1)(4b+1)(4c+1).$$
    Số kể trên chia hết cho $8$ nhưng không chia hết cho $16,$ do vậy đây không là số chính phương.
\end{enumerate}
Như vậy, giả sử phản chứng là sai. Bài toán được chứng minh.}
\end{gbtt}

\begin{gbtt}
Tìm các số nguyên tố $p,q,r$ thỏa mãn $p+q+r$ không chia hết cho $3,$ đồng thời cả $p+q+r$ và $pq+qr+rp+3$ đều là số chính phương.
\nguon{Turkey Junior Balkan Mathematical Olympiad 2013}
\loigiai{
Không mất tổng quát, ta giả sử $p\ge q\ge r.$ Dựa vào tính chất của đồng dư thức, ta sẽ lần lượt đi tìm $r,q,p.$ 
\begin{enumerate}[\color{tuancolor}\bf\sffamily Bước 1.]
    \item Ta chứng minh $r=2.$\\ Trong trường hợp $r>2,$ cả $p,q$ và $r$ đều lẻ. Một cách không mất tổng quát, ta xét bảng đồng dư theo modulo $4$ sau.
            \begin{center}
            \begin{tabular}{c|c|c|c|c}
            $p$ & $1$ & $1$ & $1$ & $3$ \\
            \hline
            $q$ & $1$ & $1$ & $3$ & $3$ \\
            \hline
            $r$ & $1$ & $3$ & $3$ & $3$ \\
            \hline
            $pq+qr+rp+3$ & $2$ & $2$ & $2$ & $2$
            \end{tabular}
        \end{center}
    Dựa theo bảng, ta được $pq+qr+rp+3$ chia cho $4$ dư $2,$ do đó $pq+qr+rp+3$ không là số chính phương, mâu thuẫn với giả thiết. Mâu thuẫn này chứng tỏ $r=2.$
    \item Ta chứng minh $q=3.$\\ Với $r=2$ chứng minh được ở trên, ta có $p+q+2$ và $pq+2p+2q+3$ đồng thời là các số chính phương.
    \begin{itemize}
        \item Nếu $q=2,$ ta nhận thấy
        $$pq+2p+2q+3=4p+7\equiv 3\pmod{4}$$
        thế nên $pq+2p+2q+3$ không thể là số chính phương.
        \item Nếu $q>3,$ một cách không mất tổng quát, ta xét bảng đồng dư theo modulo $3$ sau.
            \begin{center}
            \begin{tabular}{c|c|c|c}
            $p$ & $1$ & $1$ & $2$ \\
            \hline
            $q$ & $1$ & $2$ & $2$ \\
            \hline
            $p+q+2$ & $1$ & $2$ & $0$ \\
            \hline
            $pq+2p+2q+3$ & $2$ & $2$ & $0$
            \end{tabular}
        \end{center}
        Dựa theo bảng, ta thấy hai điều kiện $p+q+2$ không chia hết cho $3$ và $pq+2p+2q+3$ chính phương không đồng thời xảy ra, một điều mâu thuẫn. Như vậy $q=3.$
    \end{itemize}
    \item Ta tiếp tục tìm $p$ dựa trên $r=2,q=3.$\\ Với $q=3$ chứng minh được ở trên, ta có $p+5$ và $5p+9$ đồng thời là các số chính phương. Ta đặt $5p+9=a^2,$ với $a$ nguyên dương. Khi đó
    $$5p=a^2-9\Rightarrow 5p=(a-3)(a+3).$$
    Do các ước nguyên dương của $5p$ chỉ có thể là $1,5,p,5p$ nên ta xét các trường hợp sau.
    \begin{itemize}
        \item Nếu $a-3=1,$ ta có $a=4$ nhưng không tìm được $p.$
        \item Nếu $a-3=5,$ ta có $a=8$ và $p=11.$
        \item Nếu $a-3=p$ và $a+3=5,$ ta có $p=-1,$ vô lí.
        \item Nếu $a-3=5p$ và $a+3=1,$ ta có $a=-2,$ vô lí.
    \end{itemize}
\end{enumerate}
Kết quả, tất cả các bộ $(p,q,r)$ cần tìm là $(11,3,2)$ và các hoán vị của nó.}
\end{gbtt}

\begin{gbtt}
Tồn tại hay không các số nguyên $a,b$ sao cho $$a^5b+3\text{ và }ab^5+3$$
là các số lập phương?
\nguon{USA Junior Mathematical Olympiad 2013}
\loigiai{Ta giả sử tồn tại các số tự nhiên $a,b$ thỏa mãn đề bài. Ta có
$$\heva{& a^5b+3 \equiv -1,0,1 \pmod{9} \\& ab^5+3 \equiv -1,0,1 \pmod{9}}\Rightarrow \heva{& a^5b \equiv -2,-3,-4 \pmod{9} \\& ab^5 \equiv -2,-3,-4 \pmod{9}.}$$
Ta xét các trường hợp sau.
\begin{enumerate}
    \item Với $ab^5\equiv -3 \pmod{9},$ ta có $ab^5$ chia hết cho $3.$ Tuy nhiên, nếu $b$ chia hết cho $3$ thì $ab^5$ chia hết cho $3^5,$ vô lí. Mâu thuẫn này chứng tỏ $a$ chia hết cho $3,$ và như vậy
    $$a^5b\equiv 0 \pmod{9},$$
    trái với việc $a^5b \equiv -2,-3,-4 \pmod{9}.$ 
    \item Với $a^5b\equiv -3 \pmod{9},$ ta chỉ ra điều vô lí bằng cách lập luận tương tự trường hợp đầu tiên.
    \item Với $\heva{& a^5b \equiv -2,-4 \pmod{9} \\& ab^5 \equiv -2,-4 \pmod{9}},$ lấy tích theo vế, ta được
    $$a^6b^6\equiv 4,8,16 \pmod{9}. $$
    Vì $a^6b^6$ là số lập phương nên chỉ trường hợp $a^6b^6\equiv 8 \pmod{9}$ là có thể xảy ra, và như vậy thì $a^6b^6$ chia $3$ dư $2,$ vô lí do $a^6b^6$ cũng là số chính phương. 
\end{enumerate}
Các trường hợp trên đều tồn tại mâu thuẫn. Giả sử sai, và chứng minh hoàn tất.}
\end{gbtt}

\begin{gbtt}
Tồn tại hay không các số nguyên $a,b,c$ sao cho $$ab^2c+2,\: bc^2a+2\text{ và }ca^2b+2$$
là các số chính phương?
\nguon{China Western Mathematical Olympiad 2013}
\loigiai{
Ta giả sử tồn tại các số nguyên $a,b,c$ thỏa mãn đề bài. Ta có
$$\heva{ab^2c+2\equiv0,1\pmod{4}\\bc^2a+2\equiv0,1\pmod{4}\\ca^2b+2\equiv0,1\pmod{4}}\Rightarrow \heva{ab^2c&\equiv2,3\pmod{4}\\bc^2a&\equiv2,3\pmod{4}\\ca^2b&\equiv2,3\pmod{4}.}$$
Ta xét các trường hợp sau.
\begin{enumerate}
    \item Nếu $ab^2c\equiv2\pmod{4},$ ta có $ab^2c$ chia hết cho $2.$ Tuy nhiên nếu $b$ chia hết cho $2$ thì $ab^2c$ chia hết cho $2^2$, vô lí. Mâu thuẫn này chứng tỏ $a$ hoặc $c$ chia hết cho $2$. Vì $a,c$ có vai trò ngang nhau, ta giả sử $a$ chia hết cho $2$, và như vậy
    $$ca^2b\equiv 0\pmod{4},$$
    trái với việc $ca^2b\equiv2,3 \pmod{4}.$
    \item Nếu $bc^2a\equiv2\pmod{4},$ ta chỉ ra điều vô lí bằng lập luận tương tự trường hợp đầu tiên.
    \item Nếu $ca^2b\equiv2\pmod{4},$ ta chỉ ra điều vô lí bằng lập luận tương tự trường hợp đầu tiên.
    \item Nếu $ab^2c\equiv bc^2a\equiv ca^2b\equiv3\pmod{4},$ lấy tích theo vế, ta nhận được
    $$a^4b^4c^4\equiv3\pmod{4}.$$
    Vì $a^4b^4c^4$ là số chính phương nên $a^4b^4c^4$ chia $4$ chỉ dư $0$ hoặc $1$, mâu thuẫn với điều trên.
\end{enumerate}
Các trường hợp trên đều tồn tại mâu thuẫn. Giả sử sai, và chứng minh hoàn tất.}
\end{gbtt}

\begin{gbtt}
Cho tập các số nguyên $S=\{n;n+1;\cdots;n+5\}.$ Chứng minh rằng không thể chia $S$ thành hai tập $A$ và $B$ giao nhau khác rỗng, sao cho tích các phần tử trong $A$ bằng tích các phần tử trong $B.$
\nguon{Cao Đình Huy}
\loigiai{
Giả sử phản chứng rằng có thể chia $S$ thành hai tập $A,B$ như đề bài. Theo đó, tích các phần tử trong $S$ (gọi là $P$) phải là một số chính phương. Ta xét các trường hợp sau đây.
\begin{enumerate}
    \item Nếu trong $S$ có số $n+k$ chia hết cho $7,$ ta nhận thấy nó là số duy nhất trong $S$ chia được cho $7.$ Theo đó, trong $A$ và $B,$ có một tập có tích các số chia hết cho $7,$ nhưng tập còn lại thì không, mâu thuẫn.
    \item Nếu trong $S$ không có số nào chia hết cho $7,$ số dư của các số ấy khi chia cho $7$ bắt buộc là $1,2,3,4,5,6$ theo một thứ tự nào đó. Như vậy
    $$P\equiv 1\cdot2\cdot3\cdots6=720\equiv6\pmod{7}.$$
    Không có số chính phương nào đồng dư $6$ theo modulo $7.$ Trường hợp này không xảy ra.
\end{enumerate}
Dựa theo các mâu thuẫn chỉ ra, bài toán đã cho được chứng minh.}
\begin{luuy}
Bạn đọc có thể tự sáng tạo thêm các kết quả thú vị hơn cho bài toán trên bằng cách thay modulo $7$ thành modulo một số nguyên tố có dạng $4k+3.$
\end{luuy}
\end{gbtt}

\begin{gbtt}
Tìm tất cả các số nguyên $a,b$ sao cho $a^4+b^4+(a+b)^4$ là số chính phương.
\nguon{Vũ Hữu Bình}
\loigiai{Ta nhận thấy $a=b=0$ thỏa mãn. Phần còn lại của bài toán, ta xét trường hợp $a,b$ khác $0.$ Đặt $d=(a,b),$ khi đó tồn tại các số nguyên dương $x,y$ sao cho $(x,y)=1$ và $a=dx,b=dy.$ Phép đặt này cho ta
$$a^4+b^4+(a+b)^4=d^4\left[x^4+y^4+(x+y)^4\right].$$
Do $d^4$ là chính phương khác $0$ và giả thiết $a^4+b^4+(a+b)^4$  chính phương, ta suy ra $$x^4+y^4+(x+y)^4$$ cũng là số chính phương. Ta đã biết, một lũy thừa mũ $4$ chỉ có thể đồng dư theo $0$ hoặc $1$ theo modulo $16,$ phụ thuộc vào tính chẵn lẻ của chính lũy thừa đó. Ta xét bảng đồng dư theo modulo $16$ sau
    \begin{center}
        \begin{tabular}{c|c|c|c|c}
            $x^4$ & $0$ & $0$ & $1$ & $1$\\
            \hline
            $y^4$ & $0$ & $1$ & $0$ & $1$\\
            \hline
            $(x+y)^4$ & $0$ & $1$ & $1$ & $0$\\ 
            \hline
            $x^4+y^4+(x+y)^4$ & $0$ & $2$ & $2$ & $2$
        \end{tabular}
    \end{center}
Một số chính phương không thể đồng dư $2$ theo modulo $16,$ thế nên đối chiếu từng cột của bảng, ta chỉ ra $x,y$ cùng chẵn. Tuy nhiên, điều này là không thể do điều kiện phép đặt $(x,y)=1.$ \\
Như vậy, chỉ có cặp $(a,b)=(0,0)$ là thỏa mãn đề bài.}
\end{gbtt}

\begin{gbtt}
Cho $a,b$ là các số nguyên dương. Đặt $A=\tron{a+b}^2-2a^2$ và $B=\tron{a+b}^2-2b^2.$ Chứng minh rằng $A$ và $B$ không thể đồng thời là số chính phương.
\nguon{Chuyên Đại học Sư phạm Hà Nội}
\loigiai{
Giả sử $A$ và $B$ đồng thời là số chính phương. Đặt $d=(a,b),a=dm,b=dn,$ với $(m,n)=1.$ Ta có
$$(m+n)^2-2m^2,\quad (m+n)^2-2n^2$$
đều là các số chính phương. Ta xét các trường hợp sau đây.
\begin{enumerate}
    \item Nếu $m,n$ cùng là số lẻ thì $(m+n)^2$ chia hết cho $4,$ còn $2n^2\equiv 2\pmod{4}.$ Vì thế
    $$(m+n)^2-2n^2\equiv 4-2\equiv 2\pmod{4}.$$
    Không tồn tại số chính phương nào chia $4$ dư $2.$ Trường hợp này không xảy ra.
    \item Nếu $m,n$ khác tính chẵn lẻ, không mất tổng quát, ta giả sử $m$ chẵn và $n$ lẻ. Lúc này
    $$(m+n)^2-2n^2\equiv 1-2\equiv -1\equiv3\pmod{4}.$$
    Không tồn tại số chính phương nào chia $4$ dư $3.$ Trường hợp này cũng không xảy ra.    
\end{enumerate}
Như vậy, giả sử phản chứng ban đầu là sai. Bài toán được chứng minh.}
\end{gbtt}

\begin{gbtt}
Cho $d$ là một số nguyên dương khác $2,5,13.$ Chứng minh rằng ta có thể tìm được hai số nguyên phân biệt $a,b$ trong tập $\{2;5;13;d\}$ sao cho $ab-1$ không phải là số chính phương.
\loigiai{Vì $2\cdot 5-1=3^2,\ 2\cdot 13-1=5^2$ và $5\cdot 13-1=8^2$ nên hai số $a,b$ không thể là hai trong ba số $2,5,13$ trong tập hợp $\{2;5;13;d\}.$ Do đó, trong hai số $a,b$ có một số là $d$ và số kia là một trong ba số $2,5,13.$ Ta giả sử rằng cả $2d-1,5d-1,13d-1$ đều là số chính phương. Khi đó
$$2d-1\equiv 0,1,4,9\pmod{16}.$$
Từ đây ta suy ra $d\equiv 1,5,9,13\pmod{8}.$ Ta xét các trường hợp kể trên.
\begin{enumerate}
    \item Nếu $d\equiv 1\pmod{16}$ thì $13d-1\equiv 12\pmod{16},$ và khi ấy $13d-1$ không là số chính phương.
    \item Nếu $d\equiv 5\pmod{16}$ thì $5d-1\equiv 8\pmod{16},$ và khi ấy $5d-1$ không là số chính phương.
    \item Nếu $d\equiv 9\pmod{16}$ thì $5d-1\equiv12\pmod{16},$ và khi ấy $5d-1$ không là số chính phương.
    \item Nếu $d\equiv 13\pmod{16}$ thì $13d-1\equiv 8\pmod{16},$ và khi ấy $13d-1$ không là số chính phương.    
\end{enumerate}
Như vậy, giả sử phản chứng là sai, và bài toán được chứng minh.}
\end{gbtt}

\begin{gbtt}
Với mọi số nguyên dương $n,$ chứng minh rằng $$2020^{4n}+2021^{4n}+2022^{4n}+2023^{4n}$$ không phải là số chính phương.
\loigiai{
Xét trong hệ đồng dư modulo $4,$ ta có
\begin{align*}
    &2020^{4n}=(4\cdot505)^{4n}\equiv 0\pmod{4},
    \\&2021^{4n}=(4\cdot505+1)^{4n}\equiv 1\pmod{4},
    \\&2022^{4n}=(4\cdot505+2)^{4n}\equiv 2^{4n}=16^n\equiv 0\pmod{4},
    \\&2023^{4n}=(4\cdot506-1)^{4n}\equiv (-1)^{4n}=1\pmod{4}.    
\end{align*}
Các nhận xét trên chỉ ra cho ta
$$2020^{4n}+2021^{4n}+2022^{4n}+2023^{4n}\equiv 0+1+0+1 =2 \pmod{4}.$$
Một số chính phương chỉ có thể đồng dư với $0$ hoặc $1$ theo modulo $4.$ Như vậy $$2020^{4n}+2021^{4n}+2022^{4n}+2023^{4n}$$ không thể là số chính phương. Bài toán được chứng minh.}
\end{gbtt}

\begin{gbtt}
Tìm tất cả các cặp số tự nhiên $(m,n)$ sao cho $2^m3^n-1$ là số chính phương.
\nguon{Tạp chí Toán học và Tuổi trẻ, tháng 7 năm 2017}
\loigiai{
Trong bài toán này, ta xét các trường hợp sau.
\begin{enumerate}
    \item Với $m\ge 2,$ ta có $2^m3^n$ chia hết cho $4,$ và vì thế
    $$2^m3^n-1\equiv -1\equiv  3\pmod{4}.$$ 
    Không có số chính phương nào chia $4$ dư $3.$ Trường hợp này không xảy ra.
    \item Với $m=1,$ ta có $2\cdot 3^n-1$ là số chính phương. Nếu $n\ge 1$ thì
    $$2\cdot 3^n-1\equiv -1\equiv 2\pmod{3}.$$
    Không có số chính phương nào chia $3$ dư $2,$ thế nên $n=0.$ Thử với $m=1,\, n=0,$ ta thấy thoả mãn.
    \item Với $m=0,$ ta có $3^n-1$ là số chính phương. Nếu $n\ge 1$ thì
    $$3^n-1\equiv -1\equiv 2\pmod{3}.$$
    Không có số chính phương nào chia $3$ dư $2,$ thế nên $n=0.$ Thử với $m=n=0,$ ta thấy thoả mãn.    
\end{enumerate}
Như vậy có hai cặp $(m,n)$ thoả mãn đề bài là $(0,0)$ và $(1,0).$}
\end{gbtt}

\begin{gbtt}
Tìm các số nguyên dương $m,n$ thỏa mãn $A=6^m+2^n+2$ là số chính phương.
\nguon{Croatian Mathematical Olympiad 2009}
\loigiai{
Ta có nhận xét $6^{m}+2^{n}+2=2\left(3^{m}\cdot 2^{m-1}+2^{n-1}+1\right).$ Ta nghĩ đến việc xét các trường hợp sau.
\begin{enumerate}
    \item Với $m\ge 2$ và $n\ge 2,$ do cả $2^{m-1}$ và $2^{n-1}$ đều chẵn nên
    $$3^{m}\cdot 2^{m-1}+2^{n-1}+1\equiv 1\pmod{4}.$$
    Điều này giúp ta suy ra $A\equiv 2 \pmod{4}.$ Không có số chính phương nào đồng dư $2$ theo modulo $4,$ thế nên trường hợp này không xảy ra.
    \item Với $m=1,$ ta có $A=4\left(2^{n-2}+2\right)$ là số chính phương.
    \begin{itemize}
        \item Với $n=1,2,3,$ bằng kiểm tra trực tiếp, ta thấy chỉ có $n=3$ thỏa mãn.
        \item Với $n\ge 4,$ ta có $2^{n-2}+2$ là số chính phương chia cho $4$ dư $2.$ Điều này là không thể xảy ra.
    \end{itemize}
    \item Với $n=1,$ ta có $A=6^m+4$ là số chính phương. Ngoài ra, do
    $$6^m+4\equiv (-1)^m+4\equiv 3,5 \pmod{7}.$$
    nên $A$ đồng dư với $3$ hoặc $5$ theo modulo $7.$ \\Tuy nhiên, theo lí thuyết đã học, không có số chính phương nào như vậy.
\end{enumerate}
Vậy cặp số $(m, n)=(1,3)$ là cặp số duy nhất thỏa mãn đề bài.}
\end{gbtt}

\begin{gbtt}
Tìm tất cả các số nguyên tố $p$ thỏa mãn $2^p+5\cdot 3^p$ là số chính phương.
\loigiai{
Nếu $p=2,$ số đã cho chính phương. Nếu $p>2,$ ta đặt $p=2k+1$ và ta có
$$2^p+5\cdot3^p=2\cdot4^k+15\cdot9^k\equiv 3\pmod{4}.$$
Không có số chính phương nào chia $4$ dư $3.$ Kết quả bài toán chỉ có $p=2.$}
\end{gbtt}

\begin{gbtt}
Tìm tất cả các số nguyên tố $p$ thỏa mãn $7^p+9^p+259$ là số lập phương.

\loigiai{Bằng kiểm tra trực tiếp, ta chỉ ra $p=3$ là đáp số. Ta sẽ chứng minh mọi $p\ne 3$ không thỏa mãn đề bài.
\begin{enumerate}
    \item Với $p$ là số nguyên tố dạng $3k+1,$ ta có
    \begin{align*}
        7^p+9^p+259&=7^{3k+1}+9^{3k+1}+259\\&=7.342^k+9^{3k+1}+259\\&\equiv \ 7+0+7 \\&\equiv  5 \pmod{9}
    \end{align*}
    nên $7^p+9^p+259$ không là số lập phương.
    \item Với $p$ là số nguyên tố dạng $3k+2,$ ta có  
    \begin{align*}
        7^p+9^p+259&=7^{3k+2}+9^{3k+2}+259\\&=49.342^k+9^{3k+2}+259\\&\equiv \ 49+0+7 \\&\equiv  2 \pmod{9}
    \end{align*} 
    nên $7^p+9^p+259$ không là số lập phương.
    \end{enumerate}
Như vậy, $p=3$ là đáp số bài toán.}
\end{gbtt}

\section{Phương pháp kẹp lũy thừa}

\subsection*{Lí thuyết}
  Giữa hai lũy thừa số mũ $n$ liên tiếp, không tồn tại một lũy thừa cơ số $n$ nào. Hệ quả, với mọi số nguyên $a$ 
    \begin{enumerate}
        \item Không có số chính phương nào nằm giữa $a^2$ và $\left(a+1\right)^2.$
        \item Số chính phương duy nhất nằm giữa $a^2$ và $\left(a+2\right)^2$ là $\left(a+1\right)^2.$    
        \item Có $k-1$ số chính phương nằm giữa $a^2$ và $\left(a+k\right)^2,$ bao gồm \[\left(a+1\right)^2,\left(a+2\right)^2,\ldots,\left(a+k-1\right)^2.\]  
    \end{enumerate}
    Về các kết quả tương tự với số mũ khác, mời bạn đọc tự nghiên cứu và phát biểu.
\subsection*{Ví dụ minh họa}

\begin{bx}
Tìm tất cả các số nguyên dương $n$ sao cho $n^2+5n-5$ là số chính phương. 
\end{bx}
\nx{Muốn có thể sử dụng phương pháp kẹp lũy thừa, ta chọn hai tham số $a,b$ để cho
$$(n+a)^2\le n^2+5n-5\le(n+b)^2.$$
Do $(n+a)^2=n^2+2an+a^2$ và $(n+b)^2=n^2+2bn+b^2$ nên ta sẽ chọn $a,b$ sao cho khoảng cách giữa các số $2a,5,2b$ không quá lớn. Chẳng hạn, ta chọn $a=2,b=3.$ \\
Tương ứng với kiểu chọn tham số này, ta có hai cách làm như sau cho bài toán.\\}
\loigiai{
\begin{enumerate}[\sffamily\color{tuancolor}\bfseries Cách 1. ]
    \item  Ta xét các hiệu sau đây
\begin{align*}
    n^2+5n-5-(n+2)^2=n-9,\:
    (n+3)^2-\left(n^2+5n-5\right)=4n+4.
\end{align*}
Với $n\ge 10,$ cả hai hiệu trên đều dương, thế nên
$$(n+2)^2<n^2+5n-5<(n+3)^2.$$
Áp dụng phần lí thuyết đã biết, ta suy ra không tồn tại $n\ge 10$ sao cho $n^2+5n-5$ là số chính phương.
Như vậy $n\le 9.$ Thử trực tiếp với $1\le n\le 9,$ ta tìm được $n=1,\ n=2$ và $n=9.$ 
    \item  Ta xét các hiệu sau đây
\begin{align*}
    n^2+5n-5-(n-1)^2&=7n-6\ge 7-6=1> 0,
    \\(n+3)^2-\left(n^2+5n-5\right)&=4n+4\ge 4+4=8>0.
\end{align*}
Các hiệu trên đều dương, chứng tỏ rằng
$$(n-1)^2<n^2+5n-5<(n+3)^2.$$
Do $n^2+5n-5$ là số chính phương nên ta xét các trường hợp sau
\begin{itemize}
    \item\chu{Trường hợp 1.} Với $n^2+5n-5=n^2,$ ta có
    $n^2+5n-5=n^2,$ hay $n=1.$
    \item\chu{Trường hợp 2.} Với $n^2+5n-5=(n+1)^2,$ ta có
    $n^2+5n-5=n^2+2n+1,$ hay $n=2.$   
    \item\chu{Trường hợp 3.} Với $n^2+5n-5=(n+2)^2,$ ta có
    $n^2+5n-5=n^2+4n+4,$ hay $n=9.$
\end{itemize}
Như vậy, có $3$ giá trị của $n$ thỏa mãn đề bài là $n=1,\ n=2$ và $n=9.$
\end{enumerate}}

\begin{bx}
Tìm tất cả các số nguyên dương $n$ sao cho $n^4-9n^3+33n^2-63n+54$ là số chính phương.
\loigiai{Đặt $A=n^4-9n^3+33n^2-63n+54.$ Xét phân tích
$$A=(n-3)^2\left(n^2-3n+6\right).$$
Với $n=3,$ ta có $A=0$ là số chính phương. Còn với $n\ne 3,$ ta có $n^2-3n+6$ là số chính phương. \\
Trong trường hợp này, ta nhận xét được
    \begin{align*}
        n^2-3n+6-(n-2)^2&=n+2>0,
        \\(n+1)^2-\left(n^2-3n+6\right)&=5n-5\ge 5-5\ge 0.
    \end{align*}
    Các nhận xét trên cho ta
    $(n-2)^2<n^2-3n+6\le (n+1)^2.$ \\Do $n^2-3n+6$ là số chính phương, ta xét các trường hợp sau.
\begin{enumerate}
    \item Với $n^2-3n+6=(n-1)^2,$ ta có $n=5.$
    \item Với $n^2-3n+6=n^2,$ ta có $n=2.$ 
    \item Với $n^2-3n+6=(n+1)^2,$ ta có $n=1.$ 
\end{enumerate}
Như vậy, có $4$ giá trị của $n$ thỏa mãn đề bài là $n=1,\ n=2,\ n=3$ và $n=5.$}
\end{bx}

\begin{bx}
Tìm tất cả các số nguyên $n$ sao cho $A=n^3-2n^2+5n+25$ là một số lập phương.
\end{bx}
\nx{Với điều kiện $n$ nguyên, khoảng kẹp của chúng ta cần phải rộng hơn một chút. Tương tự như các bài kẹp số chính phương, ta sẽ chọn các số thực $a,b$ sao cho
$$(n+a)^3\le n^3-2n^2+5n+25\le(n+b)^3.$$
Điều này hướng ta nghĩ đến việc xét các hiệu sau đây
\begin{align*}
    (n+b)^3-\tron{n^3-2n^2+5n+25}&=\tron{3b+2}n^2+\tron{3b^2-5}n+\tron{b^3-25},\\
    \tron{n^3-2n^2+5n+25}-(n+a)^3&=\tron{-3a-2}n^2+\tron{-3a^2+5}n+\tron{-a^3+25}.
\end{align*}
Muốn $\tron{3b+2}n^2+\tron{3b^2-5}n+\tron{b^3-25}$ luôn nhận giá trị không âm với mọi $n$ nguyên, ta cần phải có
$$\heva{&3b+2\ge 0\\&\tron{3b^2-5}^2-4\tron{3b+2}\tron{b^3-25}\le 0.}$$
Do $3b+2\ge 0,$ ta sẽ thử các giá trị $b=0,1,2,\cdots$ rồi kiểm tra xem $$\tron{3b^2-5}^2-4\tron{3b+2}\tron{b^3-25}\le 0$$ kể từ khi $b$ bằng bao nhiêu. Kết quả ở đây là $b=4.$ Một cách tương tự, số $a$ cũng phải thỏa mãn
$$\heva{&3a+2\le 0\\&\tron{3a^2-5}^2-4\tron{3a+2}\tron{a^3-25}\le 0.}$$
Bằng cách thử $a=-1,-2,-3,\cdots,$ ta thấy ngay $a=-1$ thỏa mãn. Dưới đây là lời giải cho bài toán.}
\loigiai{
Ta xét các hiệu sau đây
\begin{align*}
    n^3-2n^2+5n+25-(n-1)^3&=n^2+2n+26\\&=(n+1)^2+25,\\
    (n+4)^3-\tron{n^3-2n^2+5n+25}&=14n^2+43n+39\\&=\dfrac{1}{56}\tron{28n+43}^2+\dfrac{335}{56}.
\end{align*}
Các hiệu trên đều dương, chứng tỏ
$$(n-1)^3<n^3-2n^2+5n+25<(n+4)^3.$$
Do $n^3-2n^2+5n+25$ là số lập phương, ta xét các trường hợp sau.
\begin{enumerate}
    \item Với $n^3-2n^2+5n+25=n^3,$ ta có $(n-5)(2n+5)=0.$ Do $n$ nguyên nên $n=5.$
    \item Với $n^3-2n^2+5n+25=(n+1)^3,$ ta có $(n+2)(5n-12)=0.$ Do $n$ nguyên nên $n=-2.$
    \item Với $n^3-2n^2+5n+25=(n+2)^3,$ ta có $8n^2+7n-17=0.$ Ta không tìm được $n$ nguyên từ đây.
    \item Với $n^3-2n^2+5n+25=(n+3)^3,$ ta có $11n^2+22n+2=0.$ Ta không tìm được $n$ nguyên từ đây.     
\end{enumerate}
Như vậy, $n=-2$ và $n=5$ là tất cả các giá trị của $n$ thỏa mãn đề bài.}


\begin{bx}
Tìm tất cả các số nguyên dương $n$ sao cho $n^4+4n^3-3n^2-n+3$ là số chính phương.
\end{bx}
\nx{Ta không thể phân tích $n^4+4n^3-3n^2-n+3$ thành các đa thức nhân tử hệ số nguyên. Do đó, trong bài toán này, ta nghĩ đến việc chọn các tham số $a,b,c,d$ để chắc chắn rằng
\[\left(n^2+an+b\right)^2\le n^4+4n^3-11n^2-n+11\le\left(n^2+cn+d\right)^2.\tag{*}\]
Đối với những bài toán kẹp bậc bốn như trên, ta nên cố gắng kẹp sao cho \chu{làm mất hệ số bậc ba}. Theo đó, do hệ số bậc ba trong đa thức ở các vế của (*) lần lượt là $2a,4,2c$ nên ta chọn $a=c=2.$ \\
Đặt biểu thức đã cho là $A.$ Tiếp đó, ta xét các hiệu sau
\begin{align*}
    A-\left(n^2+2n+b\right)^2&=(-7-2b)n^2+(-4b-1)n+\left(-b^2+3\right),
    \\\left(n^2+2n+d\right)^2-A&=(7+2d)n^2+(4d+1)n+\left(d^2-3\right).
\end{align*}
Hệ số bậc hai trong các hiệu trên phải dương. Các hệ số bậc hai cũng phải dương để đánh giá được tốt, thế nên để phần kẹp của chúng ta được "sát" nhất, ta nên chọn $b=-4$ và $d=0.$}
\loigiai{Đặt $A=n^4+4n^3-3n^2-n+3.$ Ta nhận xét được rằng là
\begin{align*}
A-\left(n^2+2n-4\right)^2&=n^2+15n-13\ge 1+15-13>0,
\\(n^2+2n)^2-A&=7n^2+n-3\ge 7+1-3>0.
\end{align*}
Các hiệu trên đều dương, chứng tỏ
$$\left(n^2+2n-4\right)^2<A<(n^2+2n)^2.$$
Do $A$ là số chính phương, ta xét các trường hợp sau.
\begin{enumerate}
    \item Với $A=\left(n^2+2n-3\right)^2,$ ta có 
    $$n^4+4n^3-3n^2-n+3=n^4+4n^3-2n^2-12n+9\Leftrightarrow n^2-11n-6=0.$$
    Phương trình trên vô nghiệm nguyên do nó có $\Delta=11^2+4.6=145$ không chính phương.
    \item Với $A=\left(n^2+2n-2\right)^2,$ ta có 
    $$n^4+4n^3-3n^2-n+3=n^4+4n^3-8n+4\Leftrightarrow 3n^2-7n+1=0.$$
    Phương trình trên vô nghiệm nguyên do nó có $\Delta=7^2-4.3=37$ không chính phương.   
    \item Với $A=\left(n^2+2n-1\right)^2,$ ta có 
    \begin{align*}
        n^4+4n^3-3n^2-n+3=n^4+4n^3+2n^2-4n+1
        &\Leftrightarrow 5n^2-3n-2=0.
    \end{align*}
    Do $n$ nguyên, ta chọn $n=1.$    
\end{enumerate}
Như vậy, $n=1$ là giá trị duy nhất của $n$ thỏa mãn đề bài.}

\begin{bx}
Tìm tất cả các số nguyên $n$ sao cho $n^4+n^3+1$ là số chính phương.
\loigiai{Từ giả thiết, ta suy ra $4n^4+4n^3+4$ cũng là số chính phương. Với mọi số nguyên $n,$ ta có
\begin{align*}
    4n^4+4n^3+4-\left(2n^2+n-1\right)^2&=3n^2+2n+3>0,
    \\\left(2n^2+n+3\right)^2-\left(4n^4+4n^3+4\right)&=13n^2+6n+5>0.
\end{align*}
Các đánh giá theo hiệu trên cho ta
$$\left(2n^2+n-1\right)^2<4n^4+4n^3+4<\left(2n^2+n+3\right)^2.$$
Do $4n^4+4n^3+4$ là số chính phương nên ta xét các trường hợp sau.
\begin{enumerate}
    \item Với $4n^4+4n^3+4=\left(2n^2+n\right)^2,$ ta có
    $$4n^4+4n^3+4=4n^4+4n^3+n^2\Leftrightarrow n^2=4\Leftrightarrow n=\pm 2.$$
    \item Với $4n^4+4n^3+4=\left(2n^2+n+1\right)^2,$ ta có
    \begin{align*}
        4n^4+4n^3+4=4n^4+4n^3+5n^2+2n+1&\Leftrightarrow 5n^2+2n-3=0\Leftrightarrow (n+1)(5n-3)=0.
    \end{align*}
    Do $n$ nguyên, ta chọn $n=1.$
    \item Với $4n^4+4n^3+4=\left(2n^2+n+2\right)^2,$ ta có
    $$4n^4+4n^3+4=4n^4+4n^3+9n^2+4n+4\Leftrightarrow 9n^2+4n=0\Leftrightarrow n(9n+4)=0.$$
    Do $n$ nguyên, ta chọn $n=0.$    
\end{enumerate}
Như vậy, có $4$ giá trị của $n$ thỏa mãn đề bài là $n=-2,n=0,n=1$ và $n=2.$}
\end{bx}

\begin{bx}
Tìm các số nguyên $n$ thỏa mãn $9n+16$ và $16n+9$ đều là các số chính phương.
\nguon{Titu Andreescu}
\loigiai{Do $16n+9$ là số chính phương nên $16n+9\ge 0$ hay $n\ge 0.$ Ta có 
$$(9 n+16)(16n+9)=144n^{2}+337n+144$$
cũng là số chính phương. Bởi vì
$$(12 n+12)^{2} \leq 144n^{2}+337n+144<(12 n+15)^{2}.$$
nên $144n^{2}+337n+144$ bằng $(12 n+12)^{2},(12 n+13)^{2},$ hoặc $(12 n+14)^{2}$.\\ Kiểm tra trực tiếp, ta thu được $n=0,\ n=1$ và $n=52$ là các giá trị thỏa mãn đề bài của $n.$}
\end{bx}

%nguyệt anh
\begin{bx}
Tìm tất cả các số nguyên dương $n$ sao cho $n+2$ và $n^2-n-3$ là số lập phương.
\loigiai{
Giả sử rằng tồn tại số nguyên dương $n$ thỏa mãn đề bài. Ta dễ dàng chỉ ra $n\ge 3.$\\ Rõ ràng $(n+2)(n^2-n-3)=n^3+n^2-5n-6$ cũng là số lập phương. Ta có đánh giá
$$(n-2)^3< n^3+n^2-5n-6<(n+1)^3.$$
với mọi $n\ge 3.$ Đánh giá trên cho ta $n^3+n^2-5n-6$ bằng $(n-1)^{3}$ hoặc $n^{3}.$\\ Kiểm tra trực tiếp, ta thu được $n=6$ là giá trị thỏa mãn đề bài của $n.$}
\end{bx}

\begin{bx}
Tìm tất cả các số nguyên dương $x,y$ thỏa mãn
\[x^3+3x^2+2x+9=y^3+2y.\]
\loigiai{
Giả sử tồn tại các số $x,y$ thỏa yêu cầu. Ta xét các hiệu sau đây
\begin{align*}
    \tron{x^3+3x^2+2x+9}-\tron{x^3+2x}&=3x^2+9,\\
    \tron{(x+2)^3+2(x+2)}-\tron{x^3+3x^2+2x+9}&=3x^2+12x+3.
\end{align*}
Các hiệu trên đều dương, chứng tỏ
$$x^3+2x<x^3+3x^2+2x+9<(x+2)^3+2(x+2).$$
Do $x^3+3x^2+2x+9=y^3+2y$ nên \[x^3+2x<y^3+2y<(x+2)^3+2(x+2).\tag{*}\label{kepdongdangne}\]
Ta sẽ chứng minh $y=x+1.$ Thật vậy, cả hai trường hợp $y\ge x+2$ và $y\le x$ đều cho ta mâu thuẫn với (\ref{kepdongdangne}). Với $y=x+1,$ thay trở lại phương trình ban đầu, ta được
\[x^3+3x^2+2x+9=(x+1)^3+2(x+1).\]
Ta tìm ra $x=2$ từ đây, và đáp số bài toán là $(x,y)=(2,3).$}
\begin{luuy}
Trong bài toán trên, ta đã xác định hai số $a,b$ sao cho
$$(x+a)^3+2(x+a)<x^3+3x^2+2x+9<(x+b)^3+2(x+b).$$
Phương pháp kẹp như trên được gọi là phương pháp \chu{kẹp đồng dạng}.
\end{luuy}
\end{bx}

\begin{bx}
Cho $x, y$ là những số nguyên lớn hơn 1 sao cho $4x^2y^2-7x+7y$ là số chính
phương. Chứng minh rằng ${x}={y}$.
\nguon{Chuyên Khoa học Tự nhiên 2014}
\loigiai{
Đặt $A=4 x^{2} y^{2}-7 x+7 y.$ Ta xét các hiệu sau
\begin{align*}
&A-(2 x y-1)^{2}=4 x y-7 x+7 y-1,
\\&(2 x y+1)^{2}-A=4 x y+7 x-7 y+1.
\end{align*}
Ta sẽ chứng minh các hiệu trên đều dương. Từ giả thiết, ta có $x\ge 2$ và $y\ge 2.$ Hai đánh giá này cho ta 
\begin{align*}
4xy-7x+7y-1&=(4y-7)x+6y+(y-1)>0,
\\4xy+7x-7y+1&=(4x-7)y+7x+1>0.
\end{align*}
Như vậy, $(2 {xy}-1)^{2}<{A}<(2 {xy}+1)^{2}.$ Ta có $A=4x^2y^2,$ tức là $x=y.$ Bài toán được chứng minh.}
\end{bx}

\begin{bx}
Tìm các cặp số nguyên dương $(m, n)$ sao cho $m^2+5n$ và $n^2+5m$ đều là số chính phương.
\nguon{Titu Andreescu}
\loigiai{Không mất tính tổng quát, ta giả sử $m\le n.$ Khi đó
$$
n^{2}<n^{2}+5 m\le n^{2}+5 n<(n+3)^{2} .
$$
Đánh giá trên hướng ta đến việc xét các trường hợp sau.
\begin{enumerate}
    \item Nếu $n^2+5m=(n+1)^2$ hay $5m=2n+1,$ ta chứng minh được $n$ chia $5$ dư $2.$ Thật vậy
    $$2n\equiv -1\pmod{5}\Rightarrow 6n\equiv -3\pmod{5}\Rightarrow n\equiv 2\pmod{5}.$$
    Đặt $n=5k+2$ thì $m=2k+1.$ Khi đó $m^{2}+5 n=(2 k+1)^{2}+5(5 k+2)$ là số chính phương. Ta có
    $$(2 k+4)^{2}<(2 k+1)^{2}+5(5 k+2)=4 k^{2}+29 k+11<(2 k+8)^{2}$$
    nên $(2 k+1)^{2}+5(5 k+2)$ nhận một trong các giá trị
    $$(2k+5)^2,\quad (2k+6)^2,\quad (2k+7)^2.$$
    Ta tìm ra $k=5$ và $k=38.$ Trường hợp này cho ta các cặp $(m,n)=(11,27)$ và $(m,n)=(77,192).$
    \item Nếu $n^2+5m=(n+2)^2$ hay $5m=4n+4,$ ta chứng minh được $n$ chia $5$ dư $4.$ Thật vậy
    $$4n\equiv -4\pmod{5}\Rightarrow -n\equiv -4\pmod{5}\Rightarrow n\equiv 4\pmod{5}.$$    
    Đặt $n=5k-1$ thì $m=4k.$ Khi đó $m^{2}+5n=(4k)^{2}+5(5k-1)$ là số chính phương. Ta có
    $$(4k+1)^{2}<(4k)^{2}+5(5k-1)=(4k)^{2}+5(5k-1)<(4k+4)^{2}$$    
    nên $(4k)^{2}+5(5k-1)$ nhận một trong các giá trị
    $$(4k+2)^2,\quad (4k+3)^2.$$    
    Ta tìm ra $k=1$ và $k=14.$ Trường hợp này cho ta các cặp $(m,n)=(4,4)$ và $(m,n)=(56,69).$
\end{enumerate}
Kết luận, có tất cả cả $7$ cặp $(m,n)$ thỏa yêu cầu là
$$(11,27),\ (27,11),\ (77,192),\ (192,77),\ (4,4),\ (56,69),\ (69,56).$$}
\end{bx}

\begin{bx}
Tìm tất cả các bộ số $\left(a, b, c\right)$ là ba cạnh một tam giác thỏa mãn $$a^{2}-3 a+b+c,\quad  b^{2}-3 b+c+a, \quad c^{2}-3 c+a+b$$ đều là các số chính phương. 
\nguon{Titu Andreescu}
\loigiai{Giả sử tồn tại bộ ba $(a,b,c)$ thỏa mãn đề bài. \\
Theo bất đẳng thức tam giác, $b+c>a$, và do $a,b,c$ nguyên nên $b+c \ge a+1.$ Ta có 
$$
a^{2}-3 a+b+c \geq a^{2}-2 a+1=(a-1)^{2}
$$
Không mất tổng quát, ta giả sử $a\ge b\ge c.$ Giả sử này cho ta 
$$(a-1)^{2} \leq a^{2}-3 a+b+c \leq a^{2}-a<a^{2},$$
và như vậy, $a^{2}-3 a+b+c$ chỉ có thể là số chính phương nếu $b+c=a+1$. \\
Tiếp theo, ta cũng có $b^2-3b+c+a=(b-1)^{2}+2(c-1)$ là số chính phương, thế nhưng do
$$(b-1)^2\le(b-1)^{2}+2(c-1)<b^2-1<b^2$$
nên rõ ràng $c=1.$ Kết hợp với đẳng thức $b+c=a+1,$ ta chỉ ra $a=b.$ \\
Cuối cùng, ta có $c^2-3c+a+b=2(a-1)$ là số chính phương, thế nên tồn tại số tự nhiên $m$ sao cho $a=2m^2+1.$ Kiểm tra trực tiếp các bộ số $\left(2 m^{2}+1,2 m^{2}+1,1\right),$ ta thấy chúng đều thỏa mãn đề bài. Đây chính là kết quả bài toán.}
\end{bx}

\begin{bx}
Cho các số nguyên dương $x,y.$ Chứng minh rằng nếu $x^2+2y$ là một số chính phương thì $x^2+y$ biểu diễn được thành tổng của hai số chính phương.
\loigiai{
Do $y$ là số nguyên dương, ta có $x^{2}+2y>x^2$. Theo đó, ta có thể đặt $x^{2}+2 y=(x+t)^{2},$ ở đây $t$ là số nguyên dương. Xét biến đổi sau
$$x^{2}+2 y=(x+t)^{2}\Leftrightarrow x^{2}+2 y=x^{2}+2 x t+t^{2} \Leftrightarrow t^{2}=2(y-x t).$$
Biến đổi trên cho ta $t$ chẵn, đồng thời $y-xt$ cũng là số chẵn. Ta tiếp tục đặt ${t}=2{m},$ với $m$ nguyên dương.\\
Phép đặt này cho ta 
$2y=t^2+2xt=4m^2+4mx,$
và như vậy
$$x^2+y=x^2+\dfrac{4m^2+4mx}{2}=x^2+2mx+2m^2=\left(x+m\right)^2+m^2.$$
Ta biểu diễn được $x^{2}+y$ thành tổng của hai số chính phương. Chứng minh hoàn tất.}
\end{bx}

\begin{bx}
Tìm tất cả các số tự nhiên $n$ sao cho $n^2+7n+4$ là lũy thừa số mũ tự nhiên của $2.$
\loigiai{
Ta đặt $n^2+7n+4=2^m, $ trong đó $m$ là số tự nhiên. Với mọi số tự nhiên $n,$ ta có nhận xét
$$n^2+7n+4\equiv 0,1\pmod{3}.$$
Do $2^m$ không thể chia hết cho $3,$ chỉ trường hợp $2^m$ chia $3$ dư $1$ là thỏa mãn, và ngoài ra, trường hợp này còn cho ta $m$ chẵn. Theo đó, $n^2+7n+4$ là số chính phương. Với mọi số tự nhiên $n$ thì
$$(n+2)^2\le n^2+7n+4<(n+4)^2.$$
Ta suy ra $n^2+7n+4$ hoặc bằng $(n+2)^2,$ hoặc bằng $(n+3)^2.$\\ Các trường hợp này cho ta các kết quả là $n=0$ và $n=5.$}
\end{bx}

\begin{bx}
Tìm tất cả các số nguyên dương $n$ sao cho $3^{2n}+3 n^2+7$ là một số chính phương.
\loigiai{Ta đặt $3^{2n}+3 n^2+7=m^2,$ ở đây $m$ là một số nguyên dương. \\
Do $m^2>3^{2n}$ và $3^{2n}+3 n^2+7=m^2$ là số chính phương, ta suy ra
$$
3^{2n}+3n^2+7 \geq\left(3^n+1\right)^2=3^{2 n}+2\cdot3^n+1.
$$
Đánh giá trên cho ta 
$2\cdot3^{n} \leq 3 n^{2}+6.$
Với $n\ge 3,$ bất đẳng thức trên đảo chiều. Theo đó, ta cần chứng minh bất đẳng thức sau với mọi $n\ge 3$ bằng phương pháp quy nạp.
\[2\cdot3^n>3 n^2+6.\tag{*}\]
Hiển nhiên (*) đúng với $n=3.$ Giả sử (*) đúng với $n=3,4,\ldots,k,$ thế thì
$$2\cdot3^{k+1}=3\cdot 2\cdot 3^k>3\left(3k^2+6\right)>3(k+1)^2+6.$$
Theo nguyên lí quy nạp, (*) được chứng minh với mọi $n\ge 3.$ Điều này đồng nghĩa với việc chỉ tồn các trường hợp $n=1,n=2$ và $n=3.$ Thử trực tiếp, ta thấy $n=2$ là giá trị duy nhất thỏa mãn đề bài.} 
\begin{luuy}
\nx{Thông thường, khi $n$ đủ lớn, hàm số mũ sẽ lớn hơn hàm đa thức. Nguyên lí này chính là cơ sở để ta nghĩ đến việc chứng minh bất đẳng thức (*) như bài trên.}
\end{luuy}
\end{bx}

\begin{bx}
Tìm tất cả các số nguyên dương $n$ sao cho $7^n+8n+67$ là số chính phương.
\loigiai{Nếu $n$ là số lẻ, ta có đánh giá đồng dư sau đây
$$7^n+8n+67\equiv (-1)^n+3\equiv 2\pmod{4}.$$
Không có số chính phương nào đồng dư $2$ theo modulo $4,$ chứng tỏ $n$ chẵn. Ta đặt $n=2m,$ khi đó
$$7^n+8n+67=7^{2m}+16m+67.$$
Do $7^{2m}+16m+67>7^{2m}$ và $7^{2m}+16m+67$ là số chính phương, ta suy ra
$$
7^{2m}+16m+67 \geq\left(7^m+1\right)^2=7^{2 m}+2\cdot 7^m+1.
$$
Đánh giá trên cho ta 
$2\cdot7^m \leq 16m+66,$
hay là
$7^m\le 8m+33.$
Với $m\ge 2,$ bất đẳng thức trên đảo chiều. Theo đó, ta cần chứng minh bất đẳng thức sau với mọi $m\ge 2$ bằng phương pháp quy nạp
\[7^m>8m+8.\tag{*}\]
Hiển nhiên (*) đúng với $m=2.$ Giả sử (*) đúng với $m=2,3,4,\ldots,k,$ thế thì
$$7^{k+1}=7\cdot7^k>7(8k+33)>8(k+1)+33.$$
Theo nguyên lí quy nạp, (*) được chứng minh với mọi $m\ge 2.$ Điều này đồng nghĩa với việc chỉ tồn các trường hợp $n=1$ và $n=2.$ Thử trực tiếp, ta thấy $m=2,$ và $n=2m=4$ là giá trị duy nhất thỏa mãn đề bài.} 
\end{bx}


\subsection*{Bài tập tự luyện}

\begin{btt}
Tìm tất cả các số nguyên $n$ sao cho $8n^3+7n^2+2n+4$ là số lập phương.
\end{btt}

\begin{btt}
Tìm tất cả các số nguyên dương $n$ sao cho $n^4+3n^3+3n^2+7$ là số chính phương.
\end{btt}

\begin{btt}
Tồn tại hay không số nguyên dương $n$ thỏa mãn $3^{6 n-3}+3^{3 n-1}+1$ là số lập phương?
\nguon{Titu Andreescu}
\end{btt}

\begin{btt}
Tìm tất cả các số nguyên dương $n$ sao cho
$$A=4\tron{1+2+\cdots+n}^2+3n^3-10n+1$$
là lũy thừa bậc bốn của một số tự nhiên.
\end{btt}

\begin{btt}
Tìm tất cả các số nguyên dương $n$ sao cho
$$A=24\tron{1^2+2^2+\cdots+n^2}+n+6$$
là một số lập phương.
\end{btt}

\begin{btt}
Tìm tất cả các cặp số nguyên dương $m,n$ thỏa mãn
\[m^6+5n^2=m+n^3.\]
\end{btt}


\begin{btt}
Tìm tất cả các số tự nhiên $n$ sao cho $2n+7$ và $18n+22$ là số chính phương.
\end{btt}

\begin{btt}
Tìm tất cả các số nguyên dương $n$ sao cho $n+1$ và $n^3+n^2-2n+6$ là số chính phương.
\end{btt}

\begin{btt}
Tìm các số nguyên dương $x,y$ sao cho $4x^2+9y+3$ và $4y^2+9x+3$ là số chính phương.
\end{btt}

\begin{btt}
Tìm tất cả các bộ ba số nguyên dương $(a,b,c)$ sao cho $(a+b+c)^2-2a+2b$ là số chính phương.
\nguon{Chuyên Toán Vĩnh Phúc 2021}
\end{btt}

\begin{btt}
Tìm tất cả các số nguyên dương $x,y$ thỏa mãn $x^2y^4-y^3+1$ là số chính phương.
\end{btt}

\begin{btt}
Tìm tất cả các số nguyên dương $a,b$ sao cho $a^3b^3+4a^2-3b$ là số lập phương.
\end{btt}

\begin{btt}
Tìm tất cả các số nguyên dương $a, b, c$ sao cho cả ba số
$$4a^2+5b,\quad 4b^2+5c,\quad 4c^2+5a$$ 
đều là số chính phương.
\nguon{Chuyên Khoa học Tự nhiên 2020}
\end{btt}

\begin{btt}
Tìm tất cả các bộ số tự nhiên \(\left ( a,b,c \right )\) thỏa mãn
\[a^2+2b+c,\quad b^2+2c+a,\quad c^2+2a+b\]
đều là các số chính phương.
\end{btt}

\begin{btt}
Tìm tất cả các số tự nhiên $n$ sao cho $n^4+3n^3+n^2+5$ là lũy thừa cơ số $7$ của một số tự nhiên.
\end{btt}

\begin{btt}
Tìm tất cả các số tự nhiên $n$ sao cho $13^n+7n+13$ là số chính phương.
\end{btt}

\begin{btt}
Tìm tất cả các số tự nhiên $n$ sao cho $4^n+3n+7$ là số lập phương.
\end{btt}

\begin{btt}
Cho $x, y$ là các số nguyên dương. Chứng minh $x^2+y+1$ và $y^2+4x+3$ không đồng thời là số chính phương.
\end{btt}

\begin{btt}
Cho các số nguyên dương $x,y.$ Chứng minh rằng $x^3+y^2+5x+2$ và $y^3+xy+y^2+3$ không cùng là số lập phương.
\end{btt}

\begin{btt}
Tìm tất các các số nguyên tố $p$ sao cho tổng các ước nguyên dương của $p^4$ là số chính phương.
\end{btt}

\begin{btt}
Cho \(n\) là một số nguyên dương. Tìm tất cả các ước nguyên dương \(d\) của \(3n^2\) thỏa mãn \(n^2+d\) là bình phương của một số nguyên.
\end{btt}

\begin{btt}
Tìm tất cả các số nguyên dương $a$ thỏa mãn với mọi số nguyên dương $n,$ ta có $4\left(a^n+1\right)$ là số lập phương.
\nguon{Iran Team Selection Test 2008}
\end{btt}

\begin{btt}
Tìm tất cả các số nguyên dương $n$ sao cho $n^4+8n+11$ có thể viết được thành tích ít nhất hai số nguyên dương liên tiếp.
\nguon{Junior Balkan Mathematical Olympiad 2008}
\end{btt}

\begin{btt}
Với mỗi số thực $a$ ta gọi phần nguyên của $a$ là số nguyên lớn nhất không vượt quá $a$ và ký hiệu là $[{a}]$. Chứng minh rằng vói mọi số nguyên dương ${n}$, biểu thức
$${n}+\left[\sqrt[3]{{n}-\dfrac{1}{27}}+\dfrac{1}{3}\right]^{2}$$ không biểu diễn được dưới dạng lập phương của một số nguyên dương.
\nguon{Chuyên Khoa học Tự nhiên 2011}
\end{btt}

\begin{btt}
Tìm tất cả các số nguyên dương $x,y$ thỏa mãn $4^x+4^y+1$ là một số chính phương.
\nguon{Korean Mathematical Olympiad 2007}
\end{btt}

\begin{btt}
Cho $x$ là một số thực thỏa mãn $4x^5-7$ và $4x^{13}-7$ đều là các số chính phương.
\begin{enumerate}[a,]
    \item Chứng minh rằng $x$ là số nguyên dương.
    \item Tìm tất cả các giá trị có thể của $x.$
\end{enumerate}
\nguon{Trại hè Hùng Vương 2018,
German Mathematical Olympiad 2018}
\end{btt}

\subsection*{Hướng dẫn bài tập tự luyện}

\begin{gbtt}
Tìm tất cả các số nguyên $n$ sao cho $8n^3+7n^2+2n+4$ là số lập phương.
\loigiai{
Ta xét các hiệu sau đây
\begin{align*}
    8n^3+7n^2+2n+4-(2n)^3&=7n^2+2n+4=7\tron{x+\dfrac{1}{7}}^2+\dfrac{27}{7},\\
    (2n+4)^3-\tron{8n^3+7n^2+2n+4}&=41\tron{x+\dfrac{47}{41}}^2+\dfrac{251}{41}.
\end{align*}
Các hiệu trên đều dương, chứng tỏ
$$(2n)^3<8n^3+7n^2+2n+4<(2n+4)^3.$$
Do $8n^3+7n^2+2n+4$ là số lập phương, ta xét các trường hợp sau.
\begin{enumerate}
    \item Với $8n^3+7n^2+2n+4=(2n+1)^3,$ ta có $(5n+4)n=3.$ Ta không tìm được $n$ nguyên từ đây.
    \item Với $8n^3+7n^2+2n+4=(2n+2)^3,$ ta có $17n^2+22n+4=0.$ Ta không tìm được $n$ nguyên từ đây.
    \item Với $8n^3+7n^2+2n+4=(2n+3)^3,$ ta có $(n+1)(29n+23)=0.$ Do $n$ nguyên nên $n=-1.$
\end{enumerate}
Như vậy, $n=-1$ là giá trị duy nhất của $n$ thỏa yêu cầu.}
\end{gbtt}

\begin{gbtt}
Tìm tất cả các số nguyên dương $n$ sao cho $n^4+3n^3+3n^2+7$ là số chính phương.
\loigiai{
Ta xét các hiệu sau đây
\begin{align*}
    4\tron{n^4+3n^3+3n^2+7}-\tron{2n^2+3n}^2&=3n^2+28,\\
    \tron{2n^2+3n+3}^2-4\tron{n^4+3n^3+3n^2+7}&=9n^2+18n-19\\&=8n^2+\tron{n^2-1}+18\tron{n-1}.
\end{align*}
Các hiệu trên đều dương, chứng tỏ
$$\tron{2n^2+3n}^2<4\tron{n^4+3n^3+3n^2+7}<\tron{2n^2+3n+3}^2.$$
Do $4\tron{n^4+n^3+3n^2+7}$ là số chính phương, ta xét các trường hợp sau.
\begin{enumerate}
    \item Với $4\tron{n^4+3n^3+3n^2+7}=(2n^2+3n+1)^2,$ ta có $(n-3)(n+9)=0,$ và $n=3.$
    \item Với $4\tron{n^4+3n^3+3n^2+7}=(2n^2+3n+2)^2,$ ta có $$5n^2+12n-24=0.$$ Ta không tìm được $n$ nguyên dương từ đây.
\end{enumerate}
Như vậy, $n=3$ là giá trị duy nhất của $n$ thỏa yêu cầu.}
\end{gbtt}

\begin{gbtt}
Tồn tại hay không số nguyên dương $n$ thỏa mãn $3^{6n-3}+3^{3n-1}+1$ là số lập phương?
\nguon{Titu Andreescu}
\loigiai{
Ta đặt $3^{n-1}=x,$ khi đó
$3^{6 n-3}+3^{3 n-1}+1=27x^6+9x^3+1.$
Bằng tính toán trực tiếp, ta chỉ ra
$$
\left(3x^2+1\right)^3>27x^6+9x^3+1>\left(3x^2\right)^3.
$$
Theo lí thuyết đã học, $27x^6+9x^3+1$ không là số lập phương. Câu trả lời của bài toán là phủ định.}
\end{gbtt}

\begin{gbtt}
Tìm tất cả các số nguyên dương $n$ sao cho
$$A=4\tron{1+2+\cdots+n}^2+3n^3-10n+1$$
là lũy thừa bậc bốn của một số tự nhiên.

\loigiai{
Trước tiên, ta tính được
\begin{align*}
    4\tron{1+2+\cdots+n}^2+3n^3-10n+1&=4\tron{\dfrac{n(n+1)}{2}}^2+3n^3-10n+1
    \\&=n^2(n+1)^2+3n^3-10n+1
    \\&=n^4+5n^3+n^2-10n+1.
\end{align*}
Do $A$ dương nên $n^4+5n^3+n^2-10n+1>0,$ và ta suy ra $n\ge 2.$ Với mọi $n\ge 2,$ ta có
$$\tron{2n^2+5n-6}^2<4\tron{n^4+5n^3+n^2-10n+1}<\tron{2n^2+5n-4}^2.$$
Do $4\tron{n^4+5n^3+n^2-10n+1}$ là số chính phương nên ta có $$4\tron{n^4+5n^3+n^2-10n+1}=\tron{2n^2+5n-5}^2.$$ Ta tìm được $n=7$ và $n=3$ từ đây. Thử trực tiếp, chỉ trường hợp $n=7$ cho $A$ là lũy thừa mũ bốn của một số nguyên, và đây là đáp số bài toán.}
\end{gbtt}

\begin{gbtt}
Tìm tất cả các số nguyên dương $n$ sao cho
$$A=24\tron{1^2+2^2+\cdots+n^2}+n+6$$
là một số lập phương.

\loigiai{
Trước hết, ta sẽ tìm cách tính tổng $1^2+2^2+\cdots+n^2.$ Ta đã biết
$$k^2=\tron{\dfrac{1}{3}(k+1)^3-\dfrac{1}{2}(k+1)^2+\dfrac{1}{6}(k+1)}-\tron{\dfrac{1}{3}k^3-\dfrac{1}{2}k^2+\dfrac{1}{6}k},$$
với mọi số nguyên dương $k.$ Vì thế, ta tính được
\begin{align*}
    1^2+2^2+\cdots+n^2
    &=\tron{\dfrac{1}{3}(n+1)^3-\dfrac{1}{2}(n+1)^2+\dfrac{1}{6}(n+1)}-\tron{\dfrac{1}{3}-\dfrac{1}{2}+\dfrac{1}{6}}
    \\&=\dfrac{n(n+1)(2n+1)}{6}.
\end{align*}
Như vậy $A=4n(n+1)(2n+1)+n+6=8n^3+12n^2+5n+6.$ Với mọi $n\ge 1,$ ta có nhận xét
$$(2n)^3<8n^3+12n^2+5n+6<(2n+2)^3.$$
Do $8n^3+12n^2+5n+6$ là số lập phương nên $8n^3+12n^2+5n+6=(2n+1)^3,$ hay là $n=5.$}
\end{gbtt}

\begin{gbtt}
Tìm tất cả các cặp số nguyên dương $m,n$ thỏa mãn
\[m^6+5n^2=m+n^3.\]
\loigiai{
Giả sử tồn tại các số $m,n$ thỏa yêu cầu. Ta viết lại phương trình đã cho thành
$$m^6-m=n^3-5n^2.$$
Tiếp đó, ta xét các hiệu sau đây
\begin{align*}
    \tron{m^6-m}-\bigg(\tron{m^2}^3-5m^2\bigg)&=5m^2-m=m\tron{5m-1},\\
    \bigg(\tron{m^2+1}^3-5\tron{m^2+1}\bigg)-\tron{m^6-m}&=3m^4-2m^2+m-4\\&=(m-2)\tron{3m^3+6m^2+10m+21}+38.
\end{align*}
Với $m\ge 2,$ các hiệu trên điều dương, và khi ấy
\[\tron{m^2}^3-5m^2<m^6-m<\tron{m^2+1}^3-5\tron{m^2+1}.\tag{*}\label{kepdongrang}\]
Bây giờ, ta sẽ đi chứng minh hàm số
$f\tron{x}=x^3-5x^2$
đồng biến trên tập các số thực không nhỏ hơn $2.$ Thật vậy, lấy hai số $x_1,\ x_2$ bất kì thỏa mãn $x_1>x_2\ge 2,$ và ta có
$$\dfrac{f\tron{x_1}-f\tron{x_2}}{x_1-x_2}=\dfrac{x_1^3-x_2^3-5x_1+5x_2}{x_1-x_2}=x_1^2+x_1x_2+x_2^2-5\ge  2^2+2\cdot 2+2^2-5>0.$$
Tính đồng biến của hàm $f(x)$ trên tập các số thực không nhỏ hơn $2$ kết hợp với (\ref{kepdongrang}) giúp ta chỉ ra $m<n<m+1,$ mâu thuẫn do $m,n$ đều nguyên. Như vậy $m=1,$ và thay ngược lại ta tìm ra $n=5.$ \\
Cặp số duy nhất thỏa yêu cầu là $(m,n)=(1,5).$}
\end{gbtt}

\begin{gbtt}
Tìm tất cả các số tự nhiên $n$ sao cho $2n+7$ và $18n+22$ là số chính phương.
\loigiai{
Từ giả thiết, ta có số 
$(2n+7)(18n+22)=36n^2+170n+154$
là số chính phương.\\ Rõ ràng $n=1$ không thỏa yêu cầu. Với mọi số nguyên $n\ge 2,$ ta có
\begin{align*}
    36n^2+170n+154-\tron{6n+13}^2&=14n-15>0,\\
    \tron{6n+15}^2-\tron{36n^2+170n+154}&=10n+71>0,    
\end{align*}
thế nên $36n^2+170n+154$ là số chính phương chỉ khi $36n^2+170n+154=\tron{6n+14}^2.$ \\Ta tìm ra $n=21$ từ đây. Thử lại, ta thấy thỏa mãn.}
\end{gbtt}

\begin{gbtt}
Tìm tất cả các số nguyên dương $n$ sao cho $n+1$ và $n^3+n^2-2n+6$ là số chính phương.
\loigiai{
Với số nguyên $n$ thỏa yêu cầu, ta dễ dàng chỉ ra $n\ge 1$ và
$$(n+1)(n^3+n^2-2n+6)=n^4 +2n^3-n^2+4n+6$$ 
cũng là số chính phương. Ta có đánh giá sau đây
$$\tron{n^2+n-1}^{2} \leq n^4 + 2 n^3 - n^2 + 4 n + 6<\tron{n^2+n+2}^{2}.$$
Đánh giá trên cho ta $n^4 + 2 n^3 - n^2 + 4 n + 6$ bằng $\tron{n^2+n}^{2}$ hoặc $\tron{n^2+n+1}^{2}$.\\ Kiểm tra trực tiếp, ta thu được $n=3$ là giá trị thỏa mãn đề bài của $n.$}
\end{gbtt}

\begin{gbtt}
Tìm tất cả các số nguyên dương $x,y$ sao cho $4x^2+9y+3$ và $4y^2+9x+3$ là số chính phương.
\loigiai{
Do vai trò của $x,y$ như nhau nên không mất tổng quát, ta giả sử $x\ge y.$ \\
Với mọi số nguyên dương $x,$ ta dễ dàng có được nhận xét
$$(2x)^2<4x^2+9y+3\le 4x^2+9x+3<(2x+3)^2.$$
Do $4x^2+9y+3$ là số chính phương, ta xét các trường hợp sau.
\begin{enumerate}
    \item Nếu $4x^2+9y+3=(2x+1)^2,$ ta có $9y+2=4x.$ Xét trong hệ đồng dư modulo $9$ thì $x$ chia $9$ dư $5.$ Ta đặt $x=9k+5,$ trong đó $k$ là số tự nhiên, thế thì $y=4k+2.$ Đồng thời, phép đặt này cho ta
    $$4y^2+9x+3=4(4k+2)^2+9(9k+5)+3=64k^2+145k+64.$$
    Với mọi số tự nhiên $k,$ ta luôn có
    $$(8k+8)^2\le 64k^2+145k+64<(8k+10)^2.$$
    Do $64k^2+145k+64$ là số chính phương nên $64k^2+145k+64\in\{(8k+8)^2;(8k+9)^2\}.$
     \begin{itemize}
\item \chu{Trường hợp 1.} Với $64k^2+145k+64=(8k+8)^2,$ ta tìm ra $k=0,$ từ đây $x=5$ và $y=2.$
\item \chu{Trường hợp 2.} Với $64k^2+145k+64=(8k+9)^2,$ ta tìm ra $k=17,$ từ đây $x=158$ và $y=70.$
\end{itemize}
Trường hợp này cho ta $(x,y)=(5,2)$ và $(x,y)=(158,70).$
    \item Nếu $4x^2+9y+3=(2x+2)^2,$ ta có $9y=8x+1.$\\ Tương tự trường hợp trước, ta có thể đặt $x=9k+1,y=8k+1.$ Phép đặt này cho ta
    $$4y^2+9x+3=4(8k+1)^2+9(9k+2)+3=256k^2+145k+25.$$
    Với mọi số tự nhiên $k,$ ta luôn có
    $$(16k+4)^2<256k^2+145k+25\le (16k+5)^2.$$
    Do $256k^2+145k+25$ là số chính phương nên $256k^2+145k+25= (16k+5)^2.$ \\Thử trực tiếp, ta tìm được $k=0$ thỏa mãn trường hợp này, và khi đó $x=1,y=1.$
\end{enumerate}
Kết luận, có năm cặp $(x,y)$ thỏa yêu cầu là $(1,1),\ (2,5),\ (5,2),\ (70,158)$ và $(158,70).$}
\end{gbtt}

\begin{gbtt}
Tìm tất cả các bộ ba số nguyên dương $(a,b,c)$ sao cho $(a+b+c)^2-2a+2b$ là số chính phương.
\nguon{Chuyên Toán Vĩnh Phúc 2021}
\loigiai{
Ta xét các hiệu sau
    \begin{align*}
        (a+b+c+1)^2-\left[(a+b+c)^2-2a+2b\right]&=2a+2b+2c+1+2a-2b\\&=4a+2c+1>0,\\
        \left[(a+b+c)^2-2a+2b\right]-(a+b+c-1)^2&=2a+2b+2c-1-2a+2b\\&=4b+2c-1>0.  
    \end{align*}
    Các đánh giá trên cho ta
    $$(a+b+c-1)^2<(a+b+c)^2-2a+2b<(a+b+c+1)^2.$$
    Như vậy, $(a+b+c)^2-2a+2b$ là số chính phương khi và chỉ khi
    $$(a+b+c)^2-2a+2b=(a+b+c)^2,$$
    tức là $a=b.$ Nói cách khác, tất cả các bộ $(a,b,c)$ thỏa mãn đề bài là $(k,k,t),$ với $k,t$ là các số nguyên dương.}
\end{gbtt}

\begin{gbtt}
Tìm tất cả các số nguyên dương $x,y$ thỏa mãn $x^2y^4-y^3+1$ là số chính phương.
\loigiai{
Từ giả thiết, ta suy ra $4x^4y^4-4x^2y^3+4x^2$ cũng là số chính phương. Ta nhận thấy rằng
\begin{align*}
    4x^4y^4-4x^2y^3+4x^2-\left(2x^2y^2-y-1\right)^2&=4x^2(y^2+1)-(y+1)^2\\
    &\ge 4\left(y^2+1\right)-(y+1)^2 \\
    &\ge 3y^2-2y+3\\&\ge 3\\&>0
    \\
    4x^4y^4-4x^2y^3+4x^2-\left(2x^2y^2-y+1\right)^2&=-4x^2\left(y^2-1\right)-(y-1)^2\\&\le 0.
\end{align*}
Như vậy, $\left(2x^2y^2-y-1\right)^2\le 4x^4y^4-4x^2y^3+4x^2\le\left(2x^2y^2-y+1\right)^2.$ Ta xét hai trường hợp sau đây.
\begin{enumerate}
    \item Với $4x^4y^4-4x^2y^3+4x^2=\left(2x^2y^2-y+1\right)^2,$ ta có $$-4x^2\left(y^2-1\right)-(y-1)^2=0\Rightarrow (y-1)\left(4x^2y+4x^2-y+1\right)=0\Rightarrow y=1.$$
    \item Với $4x^4y^4-4x^2y^3+4x^2=\left(2x^2y^2-y\right)^2,$ ta có $$4x^2=y^2\Rightarrow (2x-y)(2x+y)=0\Rightarrow 2x=y.$$    
\end{enumerate}
Kết luận, các cặp $(x,y)$ thỏa mãn đề bài bao gồm $(x,1)$ và $(x,2x),$ trong đó $x$ là một số nguyên dương tùy ý.}
\end{gbtt}

\begin{gbtt}
Tìm tất cả các số nguyên dương $a,b$ sao cho $a^3b^3+4a^2-3b$ là số lập phương.
\loigiai{
Trong bài toán này, ta xét các trường hợp sau.
\begin{enumerate}
    \item Nếu $a\ge 2$ và $b\ge 2,$ ta xét các hiệu
    \begin{align*}
        (ab+1)^3-\tron{a^3b^3+4a^2-3b}&=a^2\tron{3b^2-4}+3ab+3b+1>0,\\
        \tron{a^3b^3+4a^2-3b}-(ab-1)^3&=3b\tron{a^2b-a-1}+4a^2+1
        \\&>3b(4a-a-1)+4a^2+1
        \\&>0.
    \end{align*}
    Các đánh giá theo hiệu trên chứng tỏ
    $$(ab-1)^3<a^3b^3+4a^2-3b<(ab+1)^3.$$
    Do $a^3b^3+4a^2-3b$ là số lập phương nên $a^3b^3+4a^2-3b=a^3b^3,$ hay $4a^2=3b.$ \\
    Ta có $a$ chia hết cho $3.$ Đặt $a=3k$ với $k$ nguyên dương, ta tìm được $b=12k^2.$
    \item Nếu $a=1,$ ta có $a^3b^3+4a^2-3b=b^3-3b+4$ là số lập phương. Do
    $$(b-1)^3<b^3-3b+4<(b+1)^3$$
    nên $b^3-3b+4=b^3.$ Ta không tìm được $b$ nguyên từ đây.
    \item Nếu $b=1,$ ta có $a^3b^3+4a^2-3b=a^3+4a^2-3$ là số lập phương. Do
    $$a^3<a^3+4a^2-3<(a+2)^3$$
    nên $a^3+4a^2-3=(a+1)^3.$ Ta tìm ra $a=4$ từ đây.
\end{enumerate}
Kết luận, tất cả các cặp $(a,b)$ thỏa mãn đề bài gồm $(4,1)$ và dạng tổng quát
$$\tron{3k,12k^2},\text{ với }k\text{ là số nguyên dương tùy ý}.$$
}
\end{gbtt}


\begin{gbtt}
Tìm tất cả các số nguyên dương $a, b, c$ sao cho cả ba số
$$4a^2+5b,\quad 4b^2+5c,\quad 4c^2+5a$$ 
đều là số chính phương.
\nguon{Chuyên Khoa học Tự nhiên 2020}
\loigiai{
Không mất tính tổng quát, giả sử $a=\max\left\{a,b,c\right\}$. Giả sử kể trên cho ta
$$4a^2<4a^2+5b\le 4a^2+5a<(2a+2)^2.$$ 
Do  $4a^2+5b$ chính phương, bắt buộc $4a^2+5b=(2a+1)^2$ hay $5b=4a+1.$ Xét các số dư của $a$ khi chia cho $5,$ ta chỉ ra $a$ chia $5$ dư $1.$ Đặt $a=5k+1,$ ta có $b=4k+1.$ Từ kết quả này, ta thu được
$$4b^2+5c=4\tron{4k+1}^2+5c.$$
Với $k=0,$ ta tìm ra $a=b=c=1.$ Với $k\ge 1,$ do $c\le 5k+1$ nên là
$$(8k+2)^2<4\tron{4k+1}^2+5c\le 4\tron{4k+1}^2+5(5k+1)\le \tron{8k+4}^2.$$
Dựa theo đánh giá bên trên, ta chỉ ra
$$4\tron{4k+1}^2+5(5k+1)\in\left\{\tron{8k+3}^2;\tron{8k+4}^2\right\}.$$
Hai trường hợp trên lần lượt cho ta $k=0$ và $k=-1,$ mâu thuẫn với điều kiện $k\ge 1.$\\ Như vậy, bộ $3$ số nguyên dương $(a,b,c)$ duy nhất thỏa mãn là $(1,1,1).$}
\end{gbtt}

\begin{gbtt}
Tìm tất cả các bộ số tự nhiên \(\left ( a,b,c \right )\) thỏa mãn
\[a^2+2b+c,\quad b^2+2c+a,\quad c^2+2a+b\]
đều là các số chính phương.
\loigiai{
Không mất tính tổng quát, ta giả sử $a=\max\{a;b;c\}.$ Ta sẽ có
$$a^2<a^2+2b+c<a^2+2a+a<(a+2)^2.$$
Do $a^2+2b+c$ là số chính phương nên $a^2+2b+c=(a+1)^2,$ hay là 
\[a=\dfrac{2b+c-1}{2}.\tag{1}\label{elmo13.1}\]
Tới đây, ta sẽ xét các trường hợp sau.
\begin{enumerate}
    \item Nếu $b\ge c,$ ta có nhận xét
    \begin{align*}
        b^2<b^2+2c+a&=b^2+2c+\dfrac{2b+c-1}{2}\\&\le b^2+2b+\dfrac{2b+b-1}{2}
        \\&\le b^2+\dfrac{7}{2}b-\dfrac{1}{2}
        \\&<(b+2)^2.
    \end{align*}
    Do $b^2+2c+a$ là số chính phương nên $b^2+2c+a=(b+1)^2,$ hay là 
    \[2c=2b-a+1.\tag{2}\label{elmo13.2}\]
    Từ (\ref{elmo13.1}) và (\ref{elmo13.2}), ta có $6b=5a+1,3c=a+2.$ Ta đặt $a=6n+1,b=5n+1,c=2n+1,$ khi đó
    $$c^2+2a+b=(2n+1)^2+2(6n+1)+5n+1=4n^2+21n+4.$$
    Tới đây, áp dụng so sánh $(2n+2)^2\le 4n^2+21n+4<(2n+6)^2$ để tìm được $n=0$ và $n=21.$ Trường hợp này cho ta các cặp $(a,b,c)$ là 
    $(1,1,1)\text{ và }(127,106,43).$
    \item Nếu $c\ge b,$ ta có nhận xét
    \begin{align*}
        c^2<c^2+2a+b&=c^2+(2b+c-1)+b\\&\le c^2+(2c+c-1)+b
        \\&\le c^2+4c-1
        \\&<(c+2)^2.
    \end{align*}
    Do $c^2+2a+b$ là số chính phương nên $c^2+2a+b=(c+1)^2,$ hay là 
    \[2c=2a+b-1.\tag{3}\label{elmo13.3}\]
    Từ (\ref{elmo13.3}), ta suy ra $2c=2a+b-1\ge 2a,$ thế nên $c\ge a.$ Do giả sử của chúng là là $a=\max \{a;b;c\}$ nên bắt buộc dấu bằng $a=b=c=1$ phải xảy ra.
\end{enumerate}
Kết luận, có bốn cặp $(a,b,c)$ thỏa yêu cầu bài toán là
$$(1,1,1),\ (127,106,43),\ (106,43,127),\ (43,127,106).$$}
\end{gbtt}

\begin{gbtt}
Tìm các số tự nhiên $n$ sao cho $n^4+3n^3+n^2+5$ là lũy thừa cơ số $7$ của một số tự nhiên.
\loigiai{
Ta dễ dàng chứng minh được $n$ chẵn. Khi đó
$$n^4+3n^3+n^2+5\equiv 1\pmod{4}.$$
Một lũy thừa của $7$ chia $4$ dư $1$ chỉ khi đây là lũy thừa số mũ chẵn. Như vậy, $n^4+3n^3+n^2+5$ là một số chính phương. Với mọi số nguyên $n\ge 9,$ ta có
$$\tron{2n^2+3n-2}^2<4\tron{n^4+3n^3+n^2+5}<\tron{2n^2+3n-1}^2.$$
Do $n^4+3n^3+n^2+5$ là số chính phương nên trường hợp $n\ge 9$ không xảy ra.
\\Đối với $n\le 8,$ thử trực tiếp, ta có $n=2.$}
\end{gbtt}

\begin{gbtt}
Tìm tất cả các số tự nhiên $n$ sao cho $13^n+7n+13$ là số chính phương.
\loigiai{
Nếu $n$ là số lẻ, ta có đánh giá đồng dư sau đây
$$13^n+7n+13\equiv (-1)^n+6\equiv 5\pmod{7}.$$
Không có số chính phương nào đồng dư $5$ theo modulo $7,$ chứng tỏ $n$ chẵn. Ta đặt $n=2m,$ khi đó
$$13^n+7n+13=13^{2m}+14m+13.$$
Do $13^{2m}+14m+13>13^{2m}$ và $13^{2m}+14m+13$ là số chính phương, ta suy ra
$$13^{2m}+14m+13\ge\left(13^m+1\right)^2=13^{2m}+2\cdot 13^m+1.$$
Đánh giá trên cho ta 
$2\cdot13^m \leq 14m+12,$
hay là
$13^m\le 7m+6.$
Với $m\ge 2,$ bất đẳng thức trên đảo chiều. Theo đó, ta cần chứng minh bất đẳng thức sau với mọi $m\ge 2$ bằng phương pháp quy nạp
\[13^m>7m+6.\tag{*}\]
Hiển nhiên (*) đúng với $m=2.$ Giả sử (*) đúng với $m=2,3,4,\ldots,k,$ thế thì
$$13^{k+1}=13\cdot13^k>13(7k+6)>7(k+1)+6.$$
Theo nguyên lí quy nạp, (*) được chứng minh với mọi $m\ge 2.$ Điều này đồng nghĩa với việc chỉ tồn trường hợp $m=1.$ Thử trực tiếp, ta thấy đây là giá trị duy nhất thỏa mãn đề bài.}
\end{gbtt}

\begin{gbtt}
Tìm tất cả các số tự nhiên $n$ sao cho $4^n+3n+7$ là số lập phương.
\loigiai{Trước hết, ta sẽ chỉ ra $n$ chia hết cho $3.$ Thật vậy
\begin{enumerate}
    \item Nếu $n=3k+1,$ xét trong hệ đồng dư modulo $9$ ta có
    $$4^n+3n+7=4^{3k+1}+3\tron{3k+1}+7=4\cdot64^k+9k+10\equiv 4+0+1\equiv 5\pmod{9}.$$
    Không có số lập phương nào chia $9$ dư $5.$ Trường hợp này không xảy ra.
    \item Nếu $n=3k+2,$ xét trong hệ đồng dư modulo $9$ ta có
    $$4^n+3n+7=4^{3k+2}+3\tron{3k+2}+7=16\cdot64^k+9k+13\equiv 7+0+4\equiv 2\pmod{9}.$$
    Không có số lập phương nào chia $9$ dư $2.$ Trường hợp này không xảy ra.    
\end{enumerate}
Các mâu thuẫn trên chứng tỏ $n$ chia hết cho $3.$ Ta đặt $n=3m,$ khi đó
$$4^n+3n+7=4^{3m}+9m+7.$$
Do $4^{3m}+9m+7>4^{3m}$ và $4^{3m}+9m+13$ là số lập phương, ta suy ra
$$4^{3m}+9m+7\ge\left(4^m+1\right)^3=4^{3m}+3\cdot4^{2m}+3\cdot4^m+1.$$
Đánh giá trên cho ta $3m+2\ge 4^{2m}+4^m.$ Ta sẽ đi chứng minh bất đẳng thức sau với mọi $m\ge 1$
\[4^{2m}>3m+2.\tag{*}\label{mod99}\]
Hiển nhiên (\ref{mod99}) đúng với $m=1.$ Giả sử (\ref{mod99}) đúng với $m=2,3,4,\ldots,k,$ thế thì
$$4^{2(m+1)}=16\cdot4^{2m}>16(3m+2)>3(m+1)+2.$$
Theo nguyên lí quy nạp, (*) được chứng minh với mọi $m\ge 1.$ Điều này đồng nghĩa với việc trường hợp $m\ge 1$ không thỏa. Thử trực tiếp với $m=0,$ ta tìm được đáp số bài toán là $n=0.$}
\end{gbtt}

\begin{gbtt}
Cho $x,y$ là các số nguyên dương. Chứng minh $x^2+y+1$ và $y^2+4x+3$ không đồng thời là số chính phương.
\loigiai{
Ta giả sử phản chứng rằng, $x^2+y+1$ và $y^2+4 x+3$ đều là số chính phương. Ta có nhận xét
$$x^{2}+y+1>x^2.$$
Do $x^{2}+y+1$ là số chính phương, ta suy ra $x^2+y+1\ge (x+1)^2,$ tức là $y\ge 2x.$ Ta lại có nhận xét
$$y^2+4x+3\ge y^2.$$
Do $y^2+4x+3$ là số chính phương, ta suy ra $y^2+4x+3\ge (y+1)^2,$ tức là $2x+1\ge y.$ \\
Sử dụng các kết quả vừa thu được, ta có
$y\in \{2x;2x+1\}.$
Ta sẽ đi xét các trường hợp kể trên.
\begin{enumerate}
    \item Nếu $y=2x,$ ta có $y^2+4x+3=4x^2+4x+3=(2x+1)^2+2$ không là số chính phương do
    $$(2x+1)^2+2\equiv 3\pmod{4}.$$
    \item Nếu $y=2x+1,$ ta có $x^2+y+1=x^2+2x+2=(x+1)^2+1.$ Do
    $$(x+1)^2<(x+1)^2+1<(x+2)^2$$
    nên $(x+1)^2+1$ không là số chính phương, với $x$ nguyên dương.
\end{enumerate}
Các mâu thuẫn trên chứng tỏ giả sử phản chứng là sai. Bài toán được chứng minh.}
\end{gbtt}

\begin{gbtt}
Cho các số nguyên dương $x,y.$ Chứng minh rằng $x^3+y^2+5x+2$ và $y^3+xy+y^2+3$ không cùng là số lập phương.

\loigiai{
Ta giả sử tồn tại các số nguyên dương $x,y$ thỏa mãn $x^3+y^2+5x+2$ và $y^3+xy+y^2+3$ cùng là số lập phương. Do $x^3+y^2+5x+2>x^3$ nên $x^3+y^2+5x+2\ge (x+1)^3,$ hay là
\[y^2\ge 3x^2-2x-1.\tag{1}\label{kep.chau.ba.1}\]
Chứng minh tương tự, ta có $y^3+xy+y^2+3\ge (y+1)^3,$ hay là
\[xy\ge 2y^2+3y-2.\tag{2}\label{kep.chau.ba.2}\]
Tới đây, ta xét các trường hợp sau.
\begin{enumerate}
    \item Với $x=1,$ ta có $y^3+xy+y^2+3=y^3+y^2+y+3$ là số lập phương. Dựa theo nhận xét
    $$y^3<y^3+y^2+y+3<(y+1)^3,$$
    ta chỉ ra mâu thuẫn trong trường hợp này.
    \item Với $x\ge 2,$ ta có $x^2-2x-1>0,$ kết hợp với (\ref{kep.chau.ba.1}) thì
    $$y^2\ge 2x^2+\tron{x^2-2x-1}>2x^2.$$
    Do cả $x$ và $y$ nguyên dương nên $y> x\sqrt{2}.$ Đánh giá này kết hợp với (\ref{kep.chau.ba.2}) cho ta
    $$x\ge \dfrac{2y^2+3y-2}{y}>\dfrac{2y^2}{y}=2y>2x\sqrt{2}.$$
    Ta cũng chỉ ra mâu thuẫn trong trường hợp này.
\end{enumerate}
Như vậy, giả sử phản chứng là sai. Bài toán được chứng minh.}
\end{gbtt}

\begin{gbtt}
Tìm tất các các số nguyên tố $p$ sao cho tổng các ước nguyên dương của $p^4$ là số chính phương.

\loigiai{
Tập các ước của $p^4$ là $ \left\{1; p; p^2; p^3; p^4\right\}$. Tổng các số trong tập này là một số chính phương, vậy nên tồn tại số tự nhiên $n$ sao cho
$4p^4+4p^3+4p^2+4p+4=4n^2.$
Dựa vào đánh giá
$$\left(2p^2+p\right)^2<4p^4+4p^3+4p^2+4p+4\le\left(2p^2+p+2\right)^2,$$
ta chỉ ra $4p^4+4p^3+4p^2+4p+4=\left(2p^2+p+1\right)^2$, hay là $p=3.$ \\Đây là số nguyên tố duy nhất thỏa yêu cầu.}
\end{gbtt}

\begin{gbtt}
Cho \(n\) là một số nguyên dương. Tìm tất cả các ước nguyên dương \(d\) của \(3n^2\) thỏa mãn \(n^2+d\) là bình phương của một số nguyên.
\loigiai{
Nếu \(d\) là ước của \(3n^2\), tồn tại các số nguyên dương \(k\) và \(m\) thỏa mãn $3n^{2}=dk$ và $n^2+d=m^2$.\\
Với phép đặt như vậy, ta lần lượt suy ra
$$n^{2}+\frac{3n^{2}}{k}=m\Rightarrow(mk)^{2}=n^{2}\left(k^{2}+3 k\right).$$
Rõ ràng, $k^{2}+3k$ phải là một số chính phương. Do
$$k^{2}<k^{2}+3 k<(k+2)^{2},$$
nên ta chỉ ra $k^{2}+3 k=(k+1)^{2},$ nghĩa là $k=1.$ \\
Với $k=1,$ ta tìm được $d=3n^2$ thỏa yêu cầu đề bài.}
\end{gbtt}
%anh vẫn đang soát sách à???
\begin{gbtt}
Tìm tất cả các số nguyên dương $a$ thỏa mãn với mọi số nguyên dương $n,$ ta có $4\left(a^n+1\right)$ là số lập phương.
\nguon{Iran Team Selection Test 2008}
\loigiai{Từ giả thiết, ta suy ra cả $4\left(a^3+1\right)$ và $4\left(a^9+1\right)$ đều là số lập phương. Ta có nhận xét
$$4\left(a^9+1\right)=4\left(a^3+1\right)\left(a^6-a^3+1\right).$$
Nhận xét trên kết hợp với việc $a^3+1>0$ cho ta $a^6-a^3+1$ là một số lập phương. Xét hiệu, ta chỉ ra
$$\left(a^3-1\right)^2\le a^6-a^3+1\le a^6.$$
Tóm lại, $a^6-a^3+1$ bằng $\left(a^3-1\right)^2$ hoặc bằng $a^6.$ \\
Ta tính được $a=1$ từ đây, và đó là kết quả bài toán.}
\end{gbtt}

\begin{gbtt}
Tìm tất cả các số nguyên dương $n$ sao cho $n^4+8n+11$ có thể viết được thành tích ít nhất hai số nguyên dương liên tiếp.
\nguon{Junior Balkan Mathematical Olympiad 2008}
\loigiai{Trước hết, ta sẽ lập bảng đồng dư sao theo modulo $3$ của $n$
\begin{center}
    \begin{tabular}{c|c|c|c}
       $n$  &  $0$ & $1$ & $2$\\
       \hline
        $n^4+8n+11$ &  $2$ & $2$ & $1$\\
    \end{tabular}
\end{center}
Căn cứ vào bảng, $n^4+8n+11$ không thể chia hết cho $3,$ và do đó nó chỉ có thể là tích tối đa hai số nguyên dương liên tiếp. Ta đặt $n^4+8n+11=m(m+1),$ trong đó $m$ là số nguyên dương. Ta có
$$4n^4+32n+45=(2m+1)^2.$$
Theo lập luận kể trên, $4n^4+32n+45$ là số chính phương. Thử với $n=1,n=2,$ ta thấy $n=1$ thỏa mãn. Với $n\ge 3,$ ta chứng minh được
$$\tron{2n^2}^2<4n^4+32n+45<\tron{2n^2+4}^2,$$
do vậy $4n^4+32n+45\in\left\{\tron{2n^2+1}^2;\tron{2n^2+2}^2;\tron{2n^2+3}^2\right\}.$\\
Ta không tìm được $n$ nguyên từ đây. Nói tóm lại, đáp số bài toán là $n=1.$}
\end{gbtt}

\begin{gbtt}
Với mỗi số thực $a$ ta gọi phần nguyên của $a$ là số nguyên lớn nhất không vượt quá $a$ và ký hiệu là $[{a}]$. Chứng minh rằng vói mọi số nguyên dương ${n}$, biểu thức
$${n}+\left[\sqrt[3]{{n}-\dfrac{1}{27}}+\dfrac{1}{3}\right]^{2}$$ không biểu diễn được dưới dạng lập phương của một số nguyên dương.
\nguon{Chuyên Khoa học Tự nhiên 2011}
\loigiai{
Từ giả thiết, ta có thể đặt $\left[\sqrt[3]{n-\dfrac{1}{27}}+\dfrac{1}{3}\right]=a.$ Dựa vào tính chất của phần nguyên, ta có
\begin{align*}
a \leq \sqrt[3]{n-\dfrac{1}{27}}+\dfrac{1}{3}<a+1 &\Rightarrow a^{3}-a^{2}+\dfrac{4 a}{3} \leq n<a^{3}+2 a^{2}+\dfrac{7 a}{3}+\dfrac{1}{3}
\\&\Rightarrow a^{3}+\dfrac{4 a}{3} \leq n+a^{2}<a^{3}+3 a^{2}+\dfrac{7 a}{3}+\dfrac{1}{3}.
\end{align*}
Bằng biến đổi đại số trực tiếp, ta chỉ ra 
$$a^3<a^3+\dfrac{4a}{3}<a^3+3a^2+\dfrac{7a}{3}+\dfrac{1}{3}<\left(a+1\right)^3$$ 
với mọi $a$ nguyên dương. Đánh giá này cho ta
$a^3<n+a^2<\left(a+1\right)^3.$\\
Theo lí thuyết đã học, $n+a^2$ không là số lập phương. Bài toán được chứng minh.}
\end{gbtt}

\begin{gbtt}
Tìm tất cả các số nguyên dương $x,y$ thỏa mãn $4^x+4^y+1$ là một số chính phương.
\nguon{Korean Mathematical Olympiad 2007}
\loigiai{
Không mất tính tổng quát, ta giả sử $x\ge y.$ Đầu tiên, ta đặt $4^x+4^y+1=z^2,$ ở đây $z$ là số nguyên dương. Ta dễ dàng chứng minh $z$ lẻ, vậy nên ta xét biến đổi
\begin{align*}
    4^x+4^y=z^2-1
    &\Rightarrow 4^y\left(4^{x-y}+1\right)=(z-1)(z+1) 
    \\&\Rightarrow 4^{y-1}\left(4^{x-y}+1\right)=\left(\dfrac{z-1}{2}\right)\left(\dfrac{z+1}{2}\right).
    \tag{*}
\end{align*}
Do hai số $\dfrac{z-1}{2}$ và $\dfrac{z+1}{2}$ là hai số nguyên liên tiếp, chỉ một trong chúng chia hết cho $4^{y-1}.$ 
\begin{enumerate}
    \item Nếu $\dfrac{z-1}{2}$ chia hết cho $4^{y-1},$ ta suy ra $\dfrac{z-1}{2}\ge 4^{y-1}.$
    Kết hợp với $(*)$, ta được
    \begin{align*}
        4^{y-1}\left(4^{x-y}+1\right)\ge 4^{y-1}\left(4^{y-1}+1\right)
        &\Rightarrow 4^{x-y}+1\ge 4^{y-1}+1
        \\&\Rightarrow x-y\ge y-1
        \\&\Rightarrow x\ge 2y-1.
    \end{align*}
    Nhận xét này cho phép ta chỉ ra
    $$\left(2^x\right)^2<2^{2x}+2^{2y}+1\le 2^{2x}+2^{x+1}+1=\left(2^x+1\right)^2.$$
    Theo như phần lí thuyết đã học ta có $2y=x+1.$ Các bộ $(x,y)$ thỏa mãn trường hợp này có dạng
    $$(x,y)=\left(y,2y-1\right).$$
    \item Nếu $\dfrac{z+1}{2}$ chia hết cho $4^{y-1},$ ta suy ra $\dfrac{z+1}{2}\ge 4^{y-1}.$
    Kết hợp với (*), ta được
    \begin{align*}
        4^{y-1}\left(4^{x-y}+1\right)\ge 4^{y-1}\left(4^{y-1}-1\right)
        &\Rightarrow 4^{x-y}+1\ge 4^{y-1}-1
    \end{align*}    
    Nếu như $x-y<y-1,$ ta sẽ có
    $$4^{x-y}+1\ge 4^{y-1}-1\ge 4^{x-y+1}-1=4\cdot4^{x-y}-1.$$
    Chuyển vế, ta được $4^{x-y}\le \dfrac{2}{3},$ vô lí. Do đó $x-y\ge y-1.$ Tương tự trường hợp trước, ta tìm ra $$(x,y)=(y,2y-1).$$
\end{enumerate}
Tổng kết lại, tất cả các bộ $(x,y)=(y,2y-1)$ đều thỏa yêu cầu bài toán.}
\begin{luuy}
Dạng bài tập bất đẳng thức trong chia hết này đã từng xuất hiện ở \chu{chương I}. Kết hợp thêm công cụ sử dụng \chu{phương pháp kẹp lũy thừa}, bài toán được chứng minh hoàn toàn.
\end{luuy}
\end{gbtt}

\begin{gbtt}
Cho $x$ là một số thực thỏa mãn $4x^5-7$ và $4x^{13}-7$ đều là các số chính phương.
\begin{enumerate}[a,]
    \item Chứng minh rằng $x$ là số nguyên dương.
    \item Tìm tất cả các giá trị có thể của $x.$
\end{enumerate}
\nguon{Trại hè Hùng Vương 2018}
\loigiai{
\begin{enumerate}[a,]
    \item Từ giả thiết, ta có phép đặt sau với $a,b$ là số nguyên dương
    \begin{align*}
        4x^5-7&=a^2\tag{1}\\
        4x^{13}-7&=b^2\tag{2}
    \end{align*}
    Dựa vào phép đặt này, ta có $x^5=\dfrac{a^2+7}{4}$ và $x^{13}=\dfrac{b^2+7}{4}$ đều là số hữu tỉ, thế nên
    $$x=\dfrac{\left(x^5\right)^8}{\left(x^{13}\right)^3}=\dfrac{\left(\dfrac{a^2+7}{4}\right)^8}{\left(\dfrac{b^2+7}{4}\right)^3}.$$
    là số hữu tỉ. Ta tiếp tục đặt $x=\dfrac{p}{q},$ với $(p,q)=1.$ Phép đặt này cho ta
    $$4\left(\dfrac{p}{q}\right)^5=a^2+7.$$
    Ta được $q^5\mid 4p^5,$ nhưng do điều kiện phép đặt là $(p,q)=1$ nên $q^5\mid 4,$ và thế thì $q=1.$ Chứng minh này, hiển nhiên cho ta $x$ nguyên dương.
    \item Rõ ràng, $a$ và $x$ cùng tính chẵn lẻ. Với $a$ là số lẻ, ta có
    $$a^2=4x^5-7\equiv 5\pmod{8}.$$
    Đồng dư thức trên không xảy ra, chứng tỏ $a$ là số chẵn. Lấy tích theo vế của (1) và (2), ta được
    $$\left(4x^5-7\right)\left(4x^{13}-7\right)=(ab)^2.$$
    Ta xét các hiệu
    \begin{align*}
        \left(4x^5-7\right)\left(4x^{13}-7\right)-\left(4x^9-\dfrac{7x^4}{2}-1\right)^2&=8x^9-\dfrac{49x^8}{8}-28x^5-7x^4+48,\\
        \left(4x^9-\dfrac{7x^4}{2}\right)^2-\left(4x^5-7\right)\left(4x^{13}-7\right)&=\dfrac{49x^8}{4}+28x^5-49.
    \end{align*}
    Với mọi số thực $x\ge 4,$ ta có
    \begin{align*}
        8x^9-\dfrac{49x^8}{8}-28x^5-7x^4+48> 8x^9-\dfrac{49x^8 \cdot x}{8 \cdot 4}-\dfrac{28x^5 \cdot x^3}{4^3}-\dfrac{7x^4 \cdot x^4}{4^4}&>0, \\  
        \dfrac{49x^8}{4}+28x^5-49\ge 49 \cdot4^7+28 \cdot4^5-49&>0.
    \end{align*}    
    Theo như phần lí thuyết đã học, ta suy ra $\left(4x^5-7\right)\left(4x^{13}-7\right)$ không thể là số chính phương với $x\ge 4.$ Đối với $x=2,$ thử lại, ta thấy đây là giá trị duy nhất của $x$ thỏa mãn đề bài.
\end{enumerate}}
\end{gbtt} %số chính phương - biến đổi + xét mod + kẹp
\section{Ước chung lớn nhất và tính chất lũy thừa}

\subsection*{Lí thuyết}
\begin{light}
\chu{Bổ đề.} Cho $a,b,c$ là các số nguyên dương. Khi đó
    \begin{enumerate}
        \item  Nếu $a^{2}=bc$ và $(b,c)=1$  thì $b, c$ là các số chính phương.  
        \item  Nếu $a^{3}=bc$ và $(b,c)=1$  thì $b, c$ là các số lập phương.  
        \item  Nếu $a^{2}=bc$ và $(b,c)=d$  thì $b, c$ đều bằng $d$ lần một số chính phương.
    \end{enumerate}
    Tổng quát hơn, với mọi số tự nhiên $a,b,c,n$ khác $0,$ nếu $a^n=bc$ và $(b,c)=1$ thì $b$ và $c$ đều bằng một lũy thừa số mũ $n.$
\end{light}
Dưới đây, tác giả xin phép trình bày phần chứng minh cho bổ đề thứ nhất. Các bổ đề còn lại, ta chứng minh tương tự. \\
\chu{Chứng minh.} Ta xét hai phân tích tiêu chuẩn sau của $b$ và $c$
$$b=p_1^{b_1}p_2^{b_2}\ldots p_m^{b_m},\quad c=q_1^{c_1}q_2^{c_2}\ldots q_m^{c_m}.$$
Rõ ràng các số dạng $p_i$ khác các số dạng $c_j.$ Phân tích tiêu chuẩn của $a^2$ chỉ chứa các thừa số nguyên tố $p_1,p_2,\ldots,p_n,q_1,q_2,\ldots,q_m.$ Điều này chứng tỏ số mũ của các thừa số ấy phải chẵn. Do đó, cả $b$ và $c$ đều là số chính phương.

\subsection*{Ví dụ minh họa}
\begin{bx}
Cho hai số nguyên dương $x,y$ thỏa mãn $2x^2+x=3y^2+y.$ Chứng minh rằng $x-y$ và $2x+2y+1$ đều là số chính phương.
\loigiai{Với các số $x,y$ thỏa mãn giả thiết, ta có
\begin{align*}
    2x^2+x=3y^2+y 
    &\Rightarrow 2x^2-2y^2+x-y=y^2 \\&\Rightarrow 2(x-y)(x+y)+(x-y)=y^2 \\&\Rightarrow (x-y)(2x+2y+1)=y^2.
\end{align*}
Hai số $x-y$ và $2x+2y+1$ không đồng thời bằng $0,$ nên ta có thể đặt $d=(x-y,2x+2y+1).$\\
Do $y^2=(x-y)(2x+2y+1)$ nên $y^2$ chia hết cho $d^2,$ tức $y$ chia hết cho $d.$ Khi đó 
$$\heva{&d\mid (x-y) \\ &d \mid (2x+2y+1) \\ &d\mid y}\Rightarrow \heva{&d\mid x \\ &d \mid (2x+2y+1) \\ &d\mid y} \Rightarrow \heva{&d\mid x \\ &d \mid 1 \\ &d\mid y}\Rightarrow d=1.$$
Ta suy ra $x-y$ và $2x+2y+1$ đều là các số chính phương. Bài toán được chứng minh.}
\begin{luuy}
Hướng đi tách đẳng thức hoặc phương trình đã cho thành \chu{một vế nhân tử} và \chu{một vế chính phương} rồi tiến hành xét ước chung là hướng đi thường thấy trong các bài toán dạng này.
\end{luuy}
\end{bx}

\begin{bx}
Cho các số nguyên dương $a,b,c$ thỏa mãn $\dfrac{1}{a}+\dfrac{1}{b}=\dfrac{1}{c}.$ Chứng minh rằng $a^2+b^2+c^2$ là số chính phương.
\loigiai{Nếu $(a,b,c)=D,$ ta đặt 
$a=Dx,b=Dy,c=Dz.$
Đẳng thức ở giả thiết trở thành $$\dfrac{1}{x}+\dfrac{1}{y}=\dfrac{1}{z}.$$ Trong khi đó, ta chỉ cần đi chứng minh $x^2+y^2+z^2$ là số chính phương. Do đó, ta chỉ cần xét bài toán này trong trường hợp $(a,b,c)=1.$ Rõ ràng $a>c$ và $b>c.$ Đẳng thức đã cho tương đương với
$$(a-c)(b-c)=c^2.$$
Ta đặt $d=(a-c,b-c),$ khi đó
$$\heva {&d\mid (a-c) \\ &d\mid (b-c)  \\ &d\mid c}
\Rightarrow \heva{&d\mid a \\ &d\mid b   \\ &d\mid c }\Rightarrow d\mid (a,b,c)\Rightarrow d=1.$$
Áp dụng phần lí thuyết đã học với chú ý $a-c>0$ và $b-c>0$, ta suy ra $a-c,b-c$ là các số chính phương. Ta đặt $a-c=x^2,b-c=y^2,$ ở đây $x,y$ là các số nguyên dương. Phép đặt này cho ta $$c=xy,\ a=x^2+xy,\ b=y^2+xy,$$ 
vậy nên $a^2+b^2+c^2=\left(x^2+xy+y^2\right)^2.$ Bài toán được chứng minh.}
\end{bx}

\begin{bx}
Tìm các số nguyên dương $n$ thỏa mãn $A=4n^3+2n^2-7n-5$ là một số chính phương.
\nguon{Chuyên Toán Thái Nguyên 2021}
\loigiai{
Giả sử $A$ là số chính phương. Xét phân tích
$$A=(n+1)\left(4n^2-2n-5\right).$$
Ta đặt $d=\left(n+1,4n^2-2n-5\right).$ Phép đặt này cho ta 
$$\heva{&d\mid (n+1) \\ &d \mid \left(4n^2-2n-5\right)}
\Rightarrow \heva{&d\mid (n+1) \\ &d \mid \left[2(n+1)(2n-3)+1\right]} 
\Rightarrow d=1.$$
Tích của hai số dương nguyên tố cùng nhau $n+1$ và $4n^2-2n-5$ là một số chính phương, vậy nên cả $2$ số này chính phương. Tới đây, ta đánh giá
$$(2n-2)^2<4n^2-2n-5<(2n)^2.$$
Do $4n^2-2n-5$ chính phương, đánh giá trên cho ta
$$4n^2-2n-5=(2n-1)^2\Leftrightarrow 4n^2-2n-5=4n^2-4n+1\Leftrightarrow n=3.$$
Thử với $n=3,$ ta được $A$ chính phương. Đây là giá trị duy nhất của $n$ thỏa mãn đề bài.
}
\end{bx} %thainguyen

\begin{bx}
Cho $n$ là số tự nhiên lẻ sao cho $\dfrac{n^2-1}{3}$ là tích của hai số tự nhiên liên tiếp. Chứng minh rằng $n$ là tổng của hai số chính phương liên tiếp.
\loigiai{Từ giả thiết, ta có thể đặt $n=2a+1,$ đồng thời đặt $n^2-1=3b(b+1),$ với $a,b$ là các số nguyên dương. Hai phép đặt này cho ta
\begin{align*}
    (2a+1)^2-1=3b(b+1)
    &\Rightarrow 4a^2+4a=3b(b+1)
    \\&\Rightarrow 16a^2+16a=3\left(4b^2+4b\right)
    \\&\Rightarrow 16a^2+16a+3=3\left(4b^2+4b+1\right)
    \\&\Rightarrow
    (4a+1)(4a+3)=3(2b+1)^2.
    \tag{*}
\end{align*}
Do $3$ là số nguyên tố nên $3\mid (4a+1)$ hoặc $3\mid (4a+3).$ Ta xét các trường hợp kể trên.
\begin{enumerate}
    \item Nếu $4a+1$ chia hết cho $3,$ ta viết lại (*) thành
    $$(2b+1)^2=\left(\dfrac{4a+1}{3}\right)(4a+3).$$
    Do $4a+3$ và $4a+1$ là hai số lẻ liên tiếp nên chúng nguyên tố cùng nhau. Từ đó, ta có $$\left(4a+3,\dfrac{4a+1}{3}\right)=1.$$ Theo lí thuyết đã học, $4a+3$ là số chính phương. Tuy nhiên, do $3\mid (4a+1)$ nên 
    $$a\equiv 2\pmod{3}\Rightarrow 4a+3\equiv 2\pmod{3}.$$
    Không có số chính phương nào đồng dư $2$ theo modulo $3.$ Trường hợp này không xảy ra.
    \item Nếu $4a+3$ chia hết cho $3,$ ta viết lại (*) thành
    $$(2b+1)^2=\left(\dfrac{4a+3}{3}\right)(4a+1).$$
    Bằng lập luận tương tự, ta chỉ ra cả $\dfrac{4a+3}{3}$ và $4a+1$ là các số chính phương. Thậm chí, $4a+1$ còn là số chính phương lẻ. Đặt $4a+1=(2x+1)^2,$ với $x$ nguyên dương. Phép đặt này cho ta
    $$4a=4x^2+4x\Rightarrow a=x^2+x\Rightarrow n=2a+1=2x^2+2x+1=x^2+(x+1)^2$$ 
    $n$ đã được biểu diễn thành tổng của hai số chính phương liên tiếp. Chứng minh hoàn tất.
\end{enumerate}}
\end{bx}

\begin{bx}
Chứng minh rằng không tồn tại các số $a, b$ nguyên dương thỏa mãn đồng thời hai điều kiện 
\begin{enumerate}
    \item[i,] $a b, a+b$ đều là các số chính phương.
    \item[ii,] $16 a-9 b$ là số nguyên tố.
\end{enumerate}
\nguon{Junior Balkan Mathematical Olympiad Shortlists 2014}     
\loigiai{Giả sử tồn tại các số nguyên dương $a,b$ thỏa mãn đề bài. 
Đặt $(a, b)=d,$ khi đó tồn tại các số nguyên dương $x,y$ sao cho $\left(x, y\right)=1$ và $a=dx, b=dy.$
Ta có $d\left(16x-9y\right)=p$ là số nguyên tố. Xét các trường hợp sau.
\begin{enumerate}
    \item Nếu $d=1,$ ta có thể đặt $a=u^2, b=v^2.$ Phép đặt này cho ta
    $$p=16u^2-9v^2=(4 u-3 v)(4 u+3 v).$$ 
    Do $1\le 4u-3v<4u+3v$ nên ta lần lượt suy ra
    $$\heva{4u-3v=1 \\ 4u+3v=p}\Rightarrow \heva{p=8u-1 \\ p=6v+1}\Rightarrow \heva{&p\equiv 7,15,23 &\pmod{24} \\ &p\equiv 1,7,13,19 &\pmod{24}.}$$
    Đối chiếu hai dòng trong hệ, ta có $p$ chia $24$ dư $7.$ Đặt $p=24t+7$, khi đó $$u^2+v^2=(3 t+1)^{2}+(4 t+1)^{2}=25 t^{2}+14 t+2$$ là số chính phương. Tuy nhiên điều này không thể xảy ra vì
    $$
    (5 t+1)^{2}<25 t^{2}+14 t+2<(5 t+2)^{2}.
    $$
    \item Nếu $16x-9y=1,$ ta nhận được các đồng dư thức sau
    $$16x\equiv 1\pmod{9}\Rightarrow 64x \equiv 4\pmod{9}\Rightarrow x\equiv 4\pmod{9}.$$
    Đến đây, ta có thể đặt $x=t+4$ với $t$ nguyên dương. Ta suy ra $y=16 t+7,$ vậy nên
    $$
    a b=d^2xy=d^{2}(9 t+4)(16 t+7)=d^{2}\left(144 t^{2}+127 t+28\right).
    $$
Tuy nhiên, điều này không thể xảy ra là vì
$$
(12 t+5)^{2}<144 t^{2}+127 t+28<(12 t+6)^{2}.
$$
\end{enumerate}
Trong mọi trường hợp, giả sử phản chứng đều sai. Chứng minh hoàn tất.}
\end{bx}

\begin{bx}
Tìm các số nguyên dương $n$ sao cho tồn tại số nguyên dương $x$ thỏa mãn $4x^n+(x+1)^2$ là số chính phương.
\nguon{Tạp chí Kvant}
\loigiai{
Từ giả thiết, ta có thể đặt $4 x^{n}+(x+1)^{2}=y^{2},$ với $y$ nguyên dương. Phép  đặt này cho ta
\[(y-x-1)(y+x+1)=4x^n.\tag{1}\]
Hai số $x-y-1$ và $y+x+1$ có tổng và tích chẵn nên chúng đều là số chẵn. Như vậy, ta nghĩ đến chuyện chia cả hai vế của (1) cho $4$ như sau
$$\left(\dfrac{y-x-1}{2}\right)\left(\dfrac{y+x+1}{2}\right)=x^n.$$
Đặt $d=\left(\dfrac{y-x-1}{2},\dfrac{y+x+1}{2}\right).$ Phép đặt này cho ta
$$
\heva{&d\mid \dfrac{y-x-1}{2} \\ &d\mid \dfrac{y+x+1}{2} \\ &d\mid x^n}
\Rightarrow \heva{&d\mid (x+1) \\ &d\mid x^n}
\Rightarrow d=1.
$$
Theo đó, cả $\dfrac{y-x-1}{2}$ và $\dfrac{y+x+1}{2}$ đều là một lũy thừa mũ $n.$ Ta tiếp tục đặt 
\begin{align*}
    y-x-1=2u^n,
    \quad y+x+1=2v^n.
\end{align*}
Trừ theo vế phương trình trên rồi chia chúng cho $2,$ ta được
\[ x+1=u^n-v^n.\tag{2}\]
Mặt khác, khi đối chiếu với (1), phép đặt này cho ta 
\[2u^n\cdot 2v^n=4x^n\Rightarrow uv=x.\tag{3}\]
Kết hợp (2) và (3), ta có
$uv+1=u^n-v^n.$
Ta nhận thấy rằng, khi $n$ đủ lớn, hiệu $u^n-v^n$ sẽ lớn hơn $uv+1.$ Chính vì lẽ đó, ta xét các trường hợp sau đây.
\begin{enumerate}
    \item Với $n\ge 3,$ ta nhận xét được rằng
    \begin{align*}
        uv+1&=v^{n}-u^{n}
        =(v-u)\left(v^{n-1}+v^{n-2} u+\ldots+u^{n-1}\right) \\&\ge v^{n-1}+v^{n-2} u+\ldots+u^{n-1}
        \\&\ge v^{n-1}+v^{n-2}u+u^{n-1}
        \\&\ge v^2+uv+u^2.
    \end{align*}
    Ta thu được $1\ge u^2+v^2.$ Đây là điều không thể xảy ra.
    \item Với $n=2,$ ta nhận thấy có ít nhất một trường hợp thỏa mãn là $x=2,y=5.$
    \item Với $n=1,$ ta có 
    $uv+1=u-v<u<uv.$ Điều này không thể xảy ra.
\end{enumerate}
Như vậy, $n=2$ là giá trị duy nhất của $n$ thỏa mãn đề bài.}
\end{bx}

\begin{bx}
Tìm tất cả các số nguyên tố $p$ thoả mãn $\dfrac{2^{p-1}-1}{p}$ là số chính phương.
\nguon{Thailand Mathematical Olympiad 2006}
\loigiai{Ta thấy $p=2$ không thoả đề, thế nên $p$ lẻ. Ta đặt $p=2k+1.$ Ta có $$\dfrac{2^{p-1}-1}{p}=\dfrac{\left(2^{k}-1\right)\left(2^{k}+1\right)}{p}.$$
Hai nhân tử $\left(2^{k}-1\right)$ và $\left(2^{k}+1\right)$ trong phân tích trên là hai số lẻ liên tiếp, thế nên chúng có ước chung lớn nhất bằng $1.$ Ta xét các trường hợp sau.
\begin{enumerate}
    \item Nếu $\dfrac{2^k+1}{p}$ và $2^k-1$ là số chính phương, với $k=0,k=1,$ ta thấy $p=3$ thỏa mãn đề bài. Với $k\ge 2,$ ta có 
        $$2^k-1=4.2^{k-2}=1\equiv 3 \pmod{4}.$$
        Do $2^k-1$ là số chính phương, đồng dư thức $2^k-1\equiv 3 \pmod{4}$ không xảy ra. \\
        Trường hợp này không thỏa mãn.
    \item Với $\dfrac{2^{k}-1}{p}$ và $2^{k}+ 1$ là các số chính phương, ta đặt $2^{k}+1=x^{2},$ khi đó 
    $$2^k=(x-1)(x+1)\Rightarrow \heva{x - 1=2^a &\\ x+1=2^b}\Rightarrow 2=2^a\left(2^{b-a}-1\right)\Rightarrow \heva{a&=1 \\b&=2.}$$
    Ta lần lượt tìm được $x=3,k=3,p=7$ từ đây.
\end{enumerate}
Tóm lại $p=3$ và $p=7$ là tất cả các số nguyên tố cần tìm.}
\end{bx}

\subsection*{Bài tập tự luyện}

\begin{btt}
Cho $a,b,c$ là các số nguyên dương đôi một nguyên tố cùng nhau và thỏa mãn
$$a^2+b^2+c^2=(a-b)^2+(b-c)^2+(c-a)^2.$$
Chứng minh rằng $a,b,c$ và $ab+bc+ca$ đều là các số chính phương.
\nguon{Titu Andresscu}
\end{btt}

\begin{btt}
Cho $a$ và $b$ là các số nguyên sao cho tồn tại hai số nguyên liên tiếp $c$ và $d$ thỏa
mãn điều kiện $a-b=a^2c-b^2d.$ Chứng minh rằng $|a-b|$ là một số chính phương.
\end{btt}

\begin{btt}
Cho các số nguyên dương $m,n$ thỏa mãn $(m,n)=1$ và $m-n$ là một số lẻ. Chứng minh rằng $(m+3n)(5m+7n)$ không thể là số chính phương.
\nguon{Thi thử vào chuyên Phổ thông Năng khiếu 2021}
\end{btt}

\begin{btt}
Tìm tất cả các số tự nhiên $n$ để $\left(n^2+1\right)\left(5 n^2+9\right)$ là một số chính phương.
\end{btt}

\begin{btt}
Tìm tất cả các số nguyên $m,n$ thỏa mãn $$m(m+1)(m+2)=n^2.$$
\nguon{Chuyên Hà Tĩnh 2021}
\end{btt}


\begin{btt}
Giả sử $n$ là số tự nhiên thỏa mãn điều kiện $n(n+1)+7$ không chia hết cho $7.$ Chứng minh rằng $4n^3-5n-1$ không là số chính phương.
\end{btt} %thaibinh

\begin{btt}
Tìm tất cả các số nguyên $x,y$ thỏa mãn $$x^4+2x^2=y^3.$$
\nguon{Chuyên Khoa học Tự nhiên 2016}
\end{btt}

\begin{btt}
Tìm số nguyên dương $n$ nhỏ nhất để $\dfrac{4n^2+7n+3}{3}$ là số chính phương.
\end{btt}

\begin{btt}
Tìm tất cả các số nguyên $a\ne 1$ sao cho $\dfrac{a^6-1}{a-1}$ là số chính phương.
\nguon{Olympic Chuyên Khoa học Tự nhiên 2018}
\end{btt}

\begin{btt}
Giả sử tồn tại hai số nguyên dương $x,y$ với $x>1$ và thỏa mãn điều kiện $$2x^{2}-1=y^{15}.$$ Chứng minh rằng $x$ chia hết cho $15.$
\nguon{Chuyên Toán Thanh Hóa 2020}
\end{btt}

\begin{btt}
Chứng minh rằng không tồn tại các số nguyên dương $m,n$ nào thỏa mãn
\[\tron{n+1}^3+\tron{n+2}^3+\cdots+(2n)^3=m^2.\]
\end{btt}

\begin{btt}
Cho số tự nhiên $n$ và số nguyên tố $p$ sao cho $a=\dfrac{2n+2}{p}$ và $b=\dfrac{4n^2+2n+1}{p}$ là các số nguyên. Chứng minh rằng $a$ và $b$ không đồng thời là các số chính phương.
\nguon{Chuyên Toán Phổ thông Năng khiếu 2021}
\end{btt}

\begin{btt}
Cho ba số tự nhiên $a, b, c$ thỏa mãn $a-b$ là số nguyên tố và $3 c^{2}=c(a+b)+a b .$  Chứng minh rằng $8 c+1$ là số chính phương. 
\nguon{Chọn học sinh giỏi chuyên Khoa học Tự nhiên 2018}
\end{btt}

\begin{btt}
Cho số nguyên dương $n$ thỏa mãn $A=2+2 \sqrt{28 n^{2}+1}$ là số nguyên dương. Chứng minh rằng $A$ là số chính phương.
\nguon{Chuyên Bắc Ninh}
\end{btt}

\begin{btt}
Tìm số tự nhiên $n$ sao cho $36^n-6$ là tích của ít nhất hai số tự nhiên liên tiếp.
\nguon{Junior Balkan Mathematical Olympiad 2010}
\end{btt}

\begin{btt} \label{bdscp2}
Cho hai số nguyên dương $x, y$ thỏa mãn $x^{2}-4y+1$ chia hết cho $(x-2 y)(2 y-1)$. Chứng minh rằng $|x-2 y|$ là số chính phương.
\nguon{Korean Mathematical Olympiad 2014}
\end{btt}

\begin{btt}
Cho hai số nguyên dương $m, n$ và số nguyên tố $p$ thỏa mãn
$p=\dfrac{m+n}{2}+3 \sqrt{m n}.$ Chứng minh rằng $p+4 m$ và $p+4 n$ là các số chính phương.
\nguon{Vietnam Mathematical Young Talent Search 2019}
\end{btt}

\begin{btt}
Cho hai số nguyên dương $x,y$ phân biệt và số nguyên tố $p$ thỏa mãn $x+y\ne p.$ Chứng minh rằng $xy(p-x)(p-y)$ không thể là số chính phương.
\nguon{Poland Mathematical Olympiad}
\end{btt}

\begin{btt}
Cho các số nguyên dương $x,y$ thỏa mãn $x>y$ và $$(x-y, xy+1)=(x+y, xy-1)=1.$$
Chứng minh rằng $(x+y)^{2}+(x y-1)^{2}$ không phải là số chính phương.
\nguon{Iran Mathematical Olympiad 2010}
\end{btt}

\begin{btt}
Cho tập hợp $\mathcal{X}$ gồm các số nguyên có dạng $a^2+2b^2$ với $a,b$ là các số nguyên và $b\ne 0.$ Chứng minh rằng nếu $p$ là số nguyên tố và nếu $p^2\in\mathcal{X}$ thì $p\in\mathcal{X}.$
\nguon{Group "Hướng tới Olympic Toán Việt Nam"}
\end{btt}

\begin{btt}
Tìm tất cả các số nguyên $a$ thỏa mãn với mọi số nguyên dương $n,$ ta có $5\left(a^n+4\right)$ là số chính phương.

\end{btt}


\begin{btt}
Tìm các số nguyên tố $p$ sao cho $\dfrac{3^{p-1}-1}{p}$ là số chính phương.
\nguon{Saudi Arabia Junior Balkan Mathematical Olympiad Team Selection Test}
\end{btt}

\begin{btt}
Tìm các số nguyên tố $p$ sao cho $\dfrac{7^{p-1}-1}{p}$ là số chính phương.
\end{btt}

\subsection*{Hướng dẫn bài tập tự luyện}

\begin{gbtt}
Cho $a,b,c$ là các số nguyên dương đôi một nguyên tố cùng nhau và thỏa mãn
$$a^2+b^2+c^2=(a-b)^2+(b-c)^2+(c-a)^2.$$
Chứng minh rằng $a,b,c$ và $ab+bc+ca$ đều là các số chính phương.
\nguon{Titu Andresscu}
\loigiai{Đẳng thức đã cho tương đương với
$$a^2+b^2+c^2=2ab+2bc+2ca\Leftrightarrow \left(a+b-c\right)^2=4ab\Leftrightarrow \left(\dfrac{a+b-c}{2}\right)^2=ab.$$
Ta đặt $d=(a,b).$ Phép đặt này cho ta
$$\heva{&d\mid a  \\ &d\mid b  \\ &d\mid (a+b-c)}\Rightarrow \heva{&d\mid a  \\ &d\mid b  \\ &d\mid c}\Rightarrow  d=1.$$
Ta thu được $a,b$ là các số chính phương. Đặt $a=x^2,b=y^2,$ ở đây $x,y$ nguyên dương. Ta có
$$\left(\dfrac{x^2+y^2-c}{2}\right)^2=x^2y^2\Rightarrow x^2+y^2-c=2xy\Rightarrow c=\left(x-y\right)^2.$$
Như vậy $c$ là số chính phương. Cuối cùng, ta nhận xét $$ab+bc+ca=x^2y^2+x^2(x-y)^2+y^2(x-y)^2=\left(x^2-xy+y^2\right)^2.$$ 
Cả $a,b,c$ và $ab+bc+ca$ đều là số chính phương. Bài toán được chứng minh.}
\end{gbtt}

\begin{gbtt}
Cho $a$ và $b$ là các số nguyên sao cho tồn tại hai số nguyên liên tiếp $c$ và $d$ thỏa mãn điều kiện $a-b=a^2c-b^2d.$ Chứng minh rằng $|a-b|$ là một số chính phương.
\loigiai{
Từ giả thiết, ta có $d=c\pm 1.$ Ta nhận thấy rằng
    \begin{align*}
    a-b=a^{2} c-b^{2} d &\Rightarrow a-b=a^{2} c-b^{2}(c\pm 1) 
    \\&\Rightarrow  a-b=c\left(a^{2}-b^{2}\right)\pm b^{2}
    \\&\Rightarrow  a-b=c(a-b)(a+b)\pm b^{2} 
    \\&\Rightarrow (a-b)(ca+cb+1)=\pm b^{2}
    \\&\Rightarrow |a-b||ca+cb+1|=b^{2}.    
    \end{align*}
    Ta đặt $d=\left(|a-b|,|ca+cb+1|\right).$ Ta sẽ chứng minh $d=1.$ Thật vậy, ta có
    $$\heva{&d\mid(a-b) \\ & d\mid\left(ca+cb+1\right) \\
    &d\mid b}\Rightarrow \heva{&d\mid a \\&d\mid \left(ca+cb+1\right) \\ &d\mid b}\Rightarrow \heva{d\mid a \\d\mid 1 \\ d\mid b}\Rightarrow d=1.$$
    Ta suy ra $|a-b|$ là số chính phương. Bài toán được chứng minh.}
\end{gbtt}

\begin{gbtt}
Cho các số nguyên dương $m,n$ thỏa mãn $(m,n)=1$ và $m-n$ là một số lẻ. Chứng minh rằng $(m+3n)(5m+7n)$ không thể là số chính phương.
\nguon{Thi thử vào chuyên Phổ thông Năng khiếu 2021}
\loigiai{
Giả sử rằng $(m+3n)(5m+7n)$ là số chính phương. Đặt $d=(m+3n,5m+7n).$ Do $d$ lẻ, ta có
$$d=(m+3n,5m+7n-5(m+3n))=(m+3n,-8n)=(m+3n,n)=(m,n)=1.$$
Lúc này, cả $m+3n$ và $5m+7n$ là số chính phương lẻ. Vì thế
$$m+3n\equiv 5m+7n\equiv 1\pmod{8}\Rightarrow 4(m+n)\equiv 0\pmod{8}.$$
Điều này trái giả thiết $m-n$ là số lẻ. Giả sử sai. Chứng minh hoàn tất.}
\end{gbtt}

\begin{gbtt}
Tìm tất cả các số tự nhiên $n$ để $\left(n^2+1\right)\left(5 n^2+9\right)$ là một số chính phương.
\loigiai{
Ta đặt $\left(n^{2}+1,5 n^{2}+9\right)=d.$ Phép đặt này cho ta
$$\heva{&d\mid \left(n^2+1\right) \\ &d\mid \left(5n^2+9\right)}\Rightarrow d\mid \bigg(5\left(n^2+1\right)-\left(5n^2+9\right)\bigg)\Rightarrow d\mid 4\Rightarrow d\in\{1;2;4\}.$$
Tới đây, ta xét các trường hợp sau.
\begin{enumerate}
    \item Với $d=1$ hoặc $d=4,$ ta có $n^{2}+1,5 n^{2}+9$ đều là số chính phương. Bằng cách đánh giá
    $$n^{2}<n^{2}+1 \leq(n+1)^{2},$$ 
    ta chỉ ra $n^{2}+1$ là số chính phương khi và chỉ khi $n^{2}+1=(n+1)^{2},$ hay $n=0$. \\Thay ngược lại, ta thấy $n=0$ thỏa mãn đề bài.
    \item Với $d=2,$  ta có $n^2+1$ và $5n^2+9$ đều bằng $2$ lần một số chính phương. Ta đặt
    $$n^2+1=2x^2,\quad 5n^2+9=2y^2,$$
    ở đây $x,y$ là các số nguyên dương. \\
    Từ $n^2+1=2x^2,$ ta được $n$ lẻ, và như vậy, $n^2\equiv 1 \pmod{8}.$ Đồng dư thức này cho phép ta đánh giá
    $$2y^2=5n^2+9\equiv 5+9 \equiv 6 \pmod{8}\Rightarrow y^2\equiv 3 \pmod{4}.$$
    Không có số chính phương này đồng dư $3$ theo modulo $4.$ Trường hơp này không xảy ra.
\end{enumerate}
Như vậy, $n=0$ là giá trị duy nhất của $n$ thỏa mãn đề bài.}
\end{gbtt}

\begin{gbtt}Tìm tất cả các số nguyên $m,n$ thỏa mãn $$m(m+1)(m+2)=n^2.$$
\nguon{Chuyên Hà Tĩnh 2021}
\loigiai{
Giả sử tồn tại các số nguyên $m,n$ thỏa mãn đề bài. Hai số $m+1$ và $m(m+2)$ không thể đồng thời bằng $0,$ chứng tỏ chúng tồn tại ước chung lớn nhất. Ta đặt $d=\left(n+1,n(n+2)\right).$ Phép đặt này cho ta
$$\heva{&d\mid (m+1) \\ &d\mid m(m+2)}\Rightarrow \heva{&d\mid (m+1) \\ &d\mid \left[(m+1)^2-1\right]}\Rightarrow d\mid 1\Rightarrow d=1.$$
Như vậy, $\left(m+1,m(m+2)\right)=1.$ Lại do $|m+1||m(m+2)|=n^2$ nên cả $|m+1|$ và $|m(m+2)|$ đều là số chính phương. Ta xét các trường hợp sau.
\begin{enumerate}
    \item Với $m=-1,$ ta tìm được $n=0.$
    \item Với $m\ne -1,$ do $m$ là số nguyên nên $m(m+2)\ge 0.$ Lúc này, $|m(m+2)|=m(m+2)$ là số chính phương.
    Tiếp tục đặt $m(m+2)=x^2,$ với $x$ là số nguyên dương, ta được
    \begin{align*}
        m(m+2)=x^2
        &\Rightarrow m^2+2m+1=x^2+1
        \\&\Rightarrow (m+1)^2-x^2=1
        \\&\Rightarrow (m+1-x)(m+1+x)=1.
    \end{align*}
    Đến đây, ta xét các trường hợp sau.
    \begin{itemize}
        \item Với $m+1-x=m+1+x=1,$ ta có $m=0$ và $x=0,$ kéo theo $n=0.$ 
        \item Với $m+1-x=m+1+x=-1,$ ta có $m=-1$ và $x=0,$ kéo theo $n=0.$
    \end{itemize}
\end{enumerate}
Kết quả, có ba cặp số nguyên $(m,n)$ thỏa mãn đề bài là $(-2,0),(-1,0)$ và $(0,0).$}
\begin{luuy}
Sai lầm của một vài bạn gặp phải khi giải bài toán ở trên là quên mất việc xét dấu của $m+1$ và $m(m+2),$ từ đó bỏ sót cặp $(0,0).$
\end{luuy}
\end{gbtt}

\begin{gbtt}
Giả sử $n$ là số tự nhiên thỏa mãn điều kiện $n(n+1)+7$ không chia hết cho $7.$ Chứng minh rằng $4n^3-5n-1$ không là số chính phương.
\nguon{Chuyên Toán Thái Bình 2021}
\loigiai{
Từ giả thiết $n(n+1)+7$ không chia hết cho $7,$ ta suy ra $n+1$ không chia hết cho $7.$
Ta còn có
$$4n^3-5n-1=(n+1)\left(4n^2-4n-1\right).$$
Giả sử $4n^3-5n-1$ là số chính phương. Đặt $d=\left(n+1,4n^2-4n-1\right),$ lúc này
$$\heva{&d\mid (n+1) \\ &d\mid \left(4n^2-4n-1\right)}
\Rightarrow \heva{&d\mid (n+1) \\ &d\mid \left[4n(n+1)-8(n+1)+7\right]}\Rightarrow d\mid 7.$$
Do $n+1$ không là bội của $7$ nên $d=1,$ và $4n^2-4n-1$ chính phương. Đây là điều không thể xảy ra, vì
$$4n^2-4n-1\equiv 3\pmod{4}.$$
Giả sử phản chứng là sai, và ta có điều phải chứng minh.
}
\end{gbtt} %thaibinh

\begin{gbtt}
Tìm tất cả các số nguyên $x,y$ thỏa mãn $$x^4+2x^2=y^3.$$
\nguon{Chuyên Khoa học Tự nhiên 2016}
\loigiai{Phương trình đã cho tương đương với
\[x^4+2x^2+1=y^3+1\Leftrightarrow\left(x^2+1\right)^2=(y+1)\left(y^2-y+1\right).\tag{*}\]
Ta nhận thấy $y+1$ và $y^2-y+1$ không đồng thời bằng $0,$ thế nên ta có thể đặt $$d=\left(y+1,y^2-y+1\right).$$ 
Phép đặt bên trên cho ta
\begin{align*}
\heva{&d\mid (y+1) \\ &d\mid \left(y^2-y+1\right)}
&\Rightarrow \heva{&y\equiv -1 &\pmod{d} \\ &y^2-y+1\equiv 0 &\pmod{d}}
\\&\Rightarrow (-1)^2+1+1\equiv 0\pmod{d}
\\&\Rightarrow d\mid 3.    
\end{align*}
Ta có $d=1$ hoặc $d=3.$ Ta xét các trường hợp kể trên.
\begin{enumerate}
    \item Với $d=1,$ ta chỉ ra $y^2-y+1$ là số chính phương. Ta đặt $y^2-y+1=z^2.$ Phép đặt này cho ta
    \begin{align*}
        4y^2-4y+4=(2z)^2
        &\Rightarrow (2y-1)^2+3=(2z)^2
        \\&\Rightarrow (2z-2y+1)(2z+2y-1)=3.
    \end{align*}
    Giải phương trình ước số trên, ta tìm được $y=0$ và $y=1.$ \\
    Thử trực tiếp, ta thu được duy nhất một nghiệm $(0,0)$ trong trường hợp này.
    \item Với $d=3,$ ta có $3\mid (y+1)$ và $3\mid \left(y^2-y+1\right).$ Suy luận này kết hợp với (*) giúp ta chỉ ra
    $$9\mid \left(x^2+1\right)^2 \Rightarrow 3\mid \left(x^2+1\right).$$
    Không có số chính phương nào chia cho $3$ dư $0$ hoặc $1.$ Trường hợp này không xảy ra.
\end{enumerate}
Như vậy, $(x,y)=(0,0)$ là cặp số duy nhất thỏa mãn đề bài.}
\end{gbtt}

\begin{gbtt}
Tìm số nguyên dương $n$ nhỏ nhất để $\dfrac{4n^2+7n+3}{3}$ là số chính phương.
\loigiai{
Từ giả thiết, ta có thể đặt $3m^2=4n^2+7n+3=(n+1)(4n+3),$ với $m$ nguyên dương. \\
Rõ ràng hoặc $n+1,$ hoặc $4n+3$ là bội của $3$. Ta xét các trường hợp kể trên.
\begin{enumerate}
    \item Nếu $n+1$ chia hết cho $3,$ ta viết lại phép đặt thành
    $$\left(\dfrac{n+1}{3}\right)(4n+3)=m^2.$$
    Đặt $d=\left(\dfrac{n+1}{3},4n+3\right).$ Phép đặt này cho ta
    $$\heva{&d\mid \dfrac{n+1}{3} \\ &d\mid (4n+3)}\Rightarrow \heva{&d\mid (n+1) \\&d\mid (4n+3)}\Rightarrow \heva{&d\mid (4n+4)\\&d\mid (4n+3)}\Rightarrow d\mid 1\Rightarrow d=1. $$
    Theo như bổ đề, $\dfrac{n+1}{3}$ và $4n+3$ đều là số chính phương. Ta đặt $\dfrac{n+1}{3}=u^2,4n+3=v^2,$ với $u,v$ là các số nguyên dương. Phép đặt này cho ta
    $$v^2=4n+3\equiv 3\pmod{4}.$$
    Không có số chính phương nào đồng dư $3$ theo modulo $4.$ Trường hợp này không xảy ra.
    \item Nếu $4n+3$ chia hết cho $3,$ bằng lập luận tương tự như trường hợp trước, ta có thể đặt 
    \begin{align*}
        n+1&=x^2,\tag{1}\\
        4n+3&=3y^2.\tag{2}
    \end{align*}
    Lấy $4\cdot (1)-(2)$ theo vế, ta được
    $$4x^2-3y^2=1\Leftrightarrow 3y^2=(2x-1)(2x+1).$$
    Do $2x-1$ và $2x+1$ là hai số nguyên dương lẻ liên tiếp nên chúng nguyên tố cùng nhau. Từ đó, ta tiếp tục chia trường hợp này thành các trường hợp nhỏ hơn như sau.
    \begin{itemize}
        \item \chu{Trường hợp 1.} Nếu $3\mid (2x-1),$ ta viết lại
        $$y^2=\left(\dfrac{2x-1}{3}\right)(2x+1).$$
        Do $\left(\dfrac{2x-1}{3},2x+1\right)=1$ nên theo bổ đề, cả $\dfrac{2x-1}{3}$ và $2x+1$ đều là số chính phương. \\
        Ta đặt $2x-1=3a^2,2x+1=b^2.$ Phép đặt này cho ta
        $$b^2=3a^2+2\Rightarrow b^2\equiv 2 \pmod{3}.$$
        Không có số chính phương nào đồng dư $2$ theo modulo $3.$ Khả năng này không xảy ra.
        \item \chu{Trường hợp 2.} Nếu $2x+1$ chia hết cho $3,$ bằng cách làm tương tự trường hợp trên, ta đặt
        $$2x-1=c^2,2x+1=3b^2,$$ ở đây $b$ và $c>1$ là các số nguyên dương. Phép đặt này cho ta
        \[ c^2+2=3b^2.\tag{3}\]
        Do $n$ nhỏ nhất, ta có thể thử với từng giá trị nhỏ nhất của $b$ và $c.$ Cụ thể, khi lần lượt thử trực tiếp với $b=1,2,\ldots$ ta tìm ra nghiệm nhỏ nhất của $(3)$ là $(b,c)=(3,5).$ Ta có $n=168$ từ đây.
    \end{itemize}
\end{enumerate}
Như vậy, $n=168$ là số cần tìm.}
\end{gbtt}

\begin{gbtt}
Tìm tất cả các số nguyên $a\ne 1$ sao cho $\dfrac{a^6-1}{a-1}$ là số chính phương.
\nguon{Olympic Chuyên Khoa học Tự nhiên 2018}
\loigiai{
Giả sử tồn tại số nguyên thỏa mãn yêu cầu. Theo đó, $\dfrac{a^6-1}{a-1}=\tron{a^2+a+1}\tron{a^3+1}$ là số chính phương.
Đặt $\tron{a^2-a+1,a^3+1}=d$, phép đặt này cho ta
$$\heva{&d\mid \tron{a^2-a+1}\\ &d\mid \tron{a^3+1}}\Rightarrow \heva{d\mid \tron{a^3-1}\\ d\mid \tron{a^3+1}}\Rightarrow\heva{&d\mid 2a^3\\&d\mid 2\tron{a^3+1}}\Rightarrow d\mid 2.$$
Ta nhận được $d$ là ước của $2.$ Tuy nhiên, do $$a^2-a+1=a(a-1)+1$$
là số lẻ nên $d=1.$ Theo đó, cả $a^2-a+1$ và $a^3+1$ đều là số chính phương. \\
Ta đặt $a^2+a+1=y^2,$ với $y$ là số tự nhiên. Ta nhận được
$$a^2+a+1=y^2\Rightarrow 4a^2+4a+4=4y^2\Rightarrow\tron{2y-2a-1}\tron{2y+2a+1}=3.$$
Tới đây, ta lập bảng giá trị
\begin{center}
    \begin{tabular}{c|c|c|c|c}
      $2y-2a-1$   & $-1$ & $1$ & $-3$ & $3$\\
      \hline
      $2y+2a+1$   & $-3$ & $3$ & $-1$ & $1$\\
      \hline
      $a$         & $-1$ & $0$ & $0$ & $-1$\\
      \hline
      $a^3+1$     & $0$  & $1$ & $1$ & $0$
    \end{tabular}
\end{center}
Kết quả, các số nguyên $a$ thỏa mãn yêu cầu bài toán là $a=0$ và $a=-1.$}
\end{gbtt}

\begin{gbtt}
Giả sử tồn tại hai số nguyên dương $x,y$ với $x>1$ và thỏa mãn điều kiện $$2x^2-1=y^{15}.$$ Chứng minh rằng $x$ chia hết cho $15.$
\nguon{Chuyên Toán Thanh Hóa 2020}
\loigiai{
Muốn chứng minh $x$ chia hết cho $15,$ ta chia bài toán thành các bước làm như sau.
\begin{enumerate}[\color{tuancolor}\bf\sffamily Bước 1.]
    \item Chứng minh $x$ chia hết cho $3.$ \\
    Ta chứng minh được $y$ lẻ. Đẳng thức đã cho tương đương với
    $$x^{2}=\left(\dfrac{y^5+1}{2}\right)\left(y^{10}-y^{5}+1\right).$$
    Ta đặt $d=\left(\dfrac{y^5+1}{2}, y^{10}-y^{5}+1\right)$. Do $y^5\equiv -1\pmod{d},$ ta chỉ ra
    $$y^{10}-y^{5}+1 \equiv 1+1+1\equiv 3\pmod{d}.$$
    Vì $y^{10}-y^5+1\equiv 0\pmod{d}$ nên ta có $0\equiv 3\pmod{d},$ thế thì $d=1$ hoặc $d=3.$
    \begin{itemize}
        \item Với $d=1$, ta suy ra $y^{10}-y^5+1$ là số chính phương. Nhờ vào đánh giá
        $$\left(y^{4}-1\right)^2<y^{10}-y^5+1\le y^{10},$$
        ta chỉ ra được rằng $y^{10}-y^5+1= y^{10},$ hay $y=1.$ Lúc này, ta có $x=1,$ mâu thuẫn.
        \item Với $d=3,$ cả $\dfrac{y^5+1}{2}$ và $y^{10}-y^5+1$ chia hết cho $3$ nên $2x^2$ chia hết cho $3,$ và $x$ cũng vậy.
    \end{itemize}        
    \item Chứng minh $x$ chia hết cho $5.$ \\
    Ta chứng minh được $y$ lẻ. Đẳng thức đã cho tương đương với
    $$x^{2}=\left(\dfrac{y^3+1}{2}\right)\left(y^{12}-y^{9}+y^6-y^3+1\right).$$
    Ta đặt $d'=\left(\dfrac{y^3+1}{2}, y^{12}-y^{9}+y^6-y^3+1\right)$. Do $y^3\equiv -1\pmod{d'},$ ta chỉ ra
    $$y^{12}-y^{9}+y^6-y^3+1 \equiv 1+1+1+1+1\equiv 5\pmod{d'}.$$
    Vì $y^{12}-y^{9}+y^6-y^3+1\equiv 0\pmod{d'}$ nên ta có $0\equiv 5\pmod{d'},$ thế thì $d'=1$ hoặc $d'=5.$
    \begin{itemize}
        \item Với $d'=1$, ta suy ra $y^{12}-y^{9}+y^6-y^3+1$ là số chính phương. Nhờ vào đánh giá
        $$\left(y^{6}-1\right)^2<y^{12}-y^{9}+y^6-y^3+1\le y^{12},$$
        ta chỉ ra được rằng $y^{12}-y^{9}+y^6-y^3+1= y^{12},$ hay $y=1.$ Lúc này, ta có $x=1,$ mâu thuẫn với giả thiết ban đầu là $x>1.$
        \item Với $d=5,$ cả hai số $y^3+1$ và $y^{12}-y^{9}+y^6-y^3+1$ chia hết cho $5$ nên $2x^2$ chia hết cho $5,$ và $x$ cũng chia hết cho $5.$
    \end{itemize}
\end{enumerate}
Cuối cùng, ta suy ra $x$ chia hết cho $[3,5]=15.$ Bài toán được chứng minh.}
\end{gbtt}

\begin{gbtt}
Chứng minh rằng không tồn tại các số nguyên dương $m,n$ nào thỏa mãn
\[\tron{n+1}^3+\tron{n+2}^3+\cdots+(2n)^3=m^2.\]
\loigiai{
Ta đã biết $1^3+2^3+\ldots+n^3=\tron{\dfrac{n(n+1)}{2}}^2.$ Như vậy
\begin{align*}
    \tron{n+1}^3+\tron{n+2}^3+\ldots+(2n)^3
    &=\tron{n(2n+1)}^2-\tron{\dfrac{n(n+1)}{2}}^2
    \\&=\tron{n(2n+1)-\dfrac{n(n+1)}{2}}\tron{n(2n+1)+\dfrac{n(n+1)}{2}}
    \\&=\dfrac{n^2(3n+1)(5n+3)}{4}.
\end{align*}
Ta giả sử tồn tại các số nguyên dương $m,n$ thỏa mãn đẳng thức đã cho. Khi đó
$$n^2(3n+1)(5n+3)=4m^2.$$
Ta có $m$ chia hết cho $n.$ Đặt $m=nk,$ thế thì
$$(3n+1)(5n+3)=4k^2.$$
Bây giờ, ta đặt $d=(3n+1,5n+3).$ Ta sẽ có
\begin{align*}
    \heva{d\mid (3n+1) \\ d\mid (5n+3)}
    \Rightarrow \heva{d\mid (15n+5) \\ d\mid (15n+9)}
    \Rightarrow d\mid (15n+9)-(15n+5)=4
    \Rightarrow d\in\{1;2;4\}.
\end{align*}
Tới đây, ta xét các trường hợp sau.
\begin{enumerate}
    \item Với $d=1$ hoặc $d=4,$ ta có $5n+3$ là số chính phương chia $5$ dư $3,$ vô lí.
    \item Với $d=2,$ ta có $3n+1$ bằng hai lần một số chính phương. Đặt $3n+1=2x^2,$ thế thì
    $$2x^2\equiv 1\pmod{3}\Rightarrow 4x^2\equiv 2\pmod{3}\Rightarrow x^2\equiv 2\pmod{3}.$$
    Không có số chính phương nào chia $3$ dư $2.$ Trường hợp này không xảy ra.
\end{enumerate}
Giả sử phản chứng là sai. Bài toán được chứng minh.}
\end{gbtt}

\begin{gbtt}
Cho số tự nhiên $n$ và số nguyên tố $p$ sao cho $a=\dfrac{2n+2}{p}$ và $b=\dfrac{4n^2+2n+1}{p}$ là các số nguyên. Chứng minh rằng $a$ và $b$ không đồng thời là các số chính phương.
\nguon{Chuyên Toán Phổ thông Năng khiếu 2021}
\loigiai{
Ta nhận thấy $p$ là số lẻ, vậy nên
    $$\heva{&p\mid (2n+2)\\ &p\mid\left(4n^2+2n+1\right)}
    \Rightarrow \heva{&p\mid (n+1)\\ &p\mid\left(4n^2+2n+1\right)}
    \Rightarrow \heva{&p\mid (n+1)\\ &p\mid\left[(4n-2)(n+1)+3\right]}\Rightarrow p=3.$$
    Ta giả sử $a,b$ là số chính phương, khi đó $ab$ cũng là số chính phương. Đặt $ab=m^2,$ ta được
    \begin{align*}
    m^2=\dfrac{2n+2}{3}\cdot \dfrac{4n^2+2n+1}{3}
    &\Leftrightarrow 9m^2=(2n+2)\left(4n^2+2n+1\right)
    \\&\Leftrightarrow 9m^2=(2n+1)^3+1
    \\&\Leftrightarrow (3m-1)(3m+1)=(2n+1)^3. 
    \end{align*}
    Nhận xét $m$ chẵn chỉ ra cho ta $(3m+1,3m-1)=1.$ Theo đó, tồn tại các số nguyên dương $x,y$ sao cho
    $$3m+1=x^3,\quad 3m-1=y^3.$$
    Trừ theo vế hai đẳng thức trên, ta nhận thấy rằng
    $$(x-y)(x^2+xy+y^2)=2\Rightarrow \left[\begin{array}{ll}
         \heva{&x-y=1 \\ &x^2+xy+y^2=2 }  \\
         \heva{&x-y=2 \\ &x^2+xy+y^2=1 } 
    \end{array}\right.\Rightarrow \heva{&x=1 \\ &y=-1.}$$
    Các số $x,y$ không thể âm, thế nên $y=-1$ là vô lí. Giả sử phản chứng là sai. Chứng minh hoàn tất.}
\end{gbtt}

\begin{gbtt}
Cho ba số tự nhiên $a, b, c$ thỏa mãn $a-b$ là số nguyên tố và $3 c^{2}=c(a+b)+a b .$  Chứng minh rằng $8 c+1$ là số chính phương. 
\nguon{Chọn học sinh giỏi chuyên Khoa học Tự nhiên 2018}
\loigiai{
Với các số $a,b,c$ đã cho, ta có
\[4c^{2}=c^{2}+a b+b c+c a=(c+a)(b+c)\tag{*}\]
Ta đặt $(a+c, b+c)=d$, khi đó $d$ là ước của $(a+c)-(b+c)=a-b.$
Do $a-b$ là số nguyên tố nên ta suy $d=1$ hoặc $d=a-b$. Ta xét các trường hợp kể trên.
\begin{enumerate}
    \item Nếu $d=a-b,$ cả $a+c$ và $b+c$ bằng $a-b$ lần một số chính phương. Ta đặt $$c+a=(a-b)x^2,c+b=(a-b)y^2,$$ trong đó $x$ và $y$ là các số tự nhiên. Lấy hiệu theo vế, ta được
    $$a-b=(a-b)\left(x^2-y^2\right)\Rightarrow (x-y)(x+y)=1\Rightarrow \heva{x&=1 \\ y&=0.}$$
    Ta thu được $c+b=0,$ tức $c=b=0.$ Lúc này, $8c+1=1$ là số chính phương. 
    \item Nếu $d=1,$ cả $c+a$ và $b+c$ là số chính phương. Ta đặt $c+a=m^2,c+b=n^2,$ trong đó $m$ và $n$ là các số tự nhiên. Lấy hiệu theo vế, ta được
    $$a-b=m^2-m^2= (m-n)(m+n)\Rightarrow m-n=1\Rightarrow m=n+1.$$
    Kết hợp $m=n+1$ với (*), ta có
    $$4 c^{2}=(c+a)(b+c)=m^{2} n^{2}=n^{2}(n+1)^{2}.$$
    Ta suy ra $2c=n(n+1),$ và như vậy, $8c+1=4 n(n+1)+1=(2n+1)^2$ là số chính phương.
\end{enumerate}
Trong cả hai trường hợp, $8c+1$ đều là số chính phương. Chứng minh hoàn tất.}
\end{gbtt}

\begin{gbtt}
Cho số nguyên dương $n$ thỏa mãn $A=2+2 \sqrt{28 n^{2}+1}$ là số nguyên dương. Chứng minh rằng $A$ là số chính phương.
\nguon{Chuyên Bắc Ninh}
\loigiai{Ta chứng minh được $\sqrt{28n^2+1}$ là số nguyên dương. Ngoài ra, do $28n^2+1$ lẻ nên $\sqrt{28n^2+1}$ cũng là một số lẻ. Ta đặt $28n^2+1=(2m+1)^2,$ ở đây $m$ là số nguyên dương. Phép đặt trên cho ta
\[28n^2+1=4m^2+4m+1\Rightarrow 28n^2=4m^2+4m\Rightarrow 7n^2=m(m+1).\tag{*}\]
Do $7$ là số nguyên tố nên $7\mid m$ hoặc $7\mid (m+1).$ Ta xét các trường hợp kể trên.
\begin{enumerate}
    \item Nếu $m+1$ chia hết cho $7,$ ta viết lại (*) thành
    $$n^2=\left(\dfrac{m+1}{7}\right)m.$$
    Do $\left(m,\dfrac{m+1}{7}\right)=1,$ ta suy ra $m$ là số chính phương. Đây là điều vô lí vì $m\equiv 6\pmod{7}.$
    \item Nếu $m$ chia hết cho $7,$ ta viết lại (*) thành
    $$n^2=\dfrac{m}{7}(m+1),$$
    Do $\left(m+1,\dfrac{m}{7}\right)=1,$ ta suy ra $m+1$ là số chính phương. Đặt $m+1=x^2,$ với $x$ là số nguyên dương. Do $28n^2+1=(2m+1)^2$ nên phép đặt này cho ta
    $$\sqrt{28n^2+1}=2m+1=2x^2-1.$$
    Bằng cách này, ta có $A=2+2\sqrt{28n^2+1}=4x^2$ là số chính phương. 
\end{enumerate}
Chứng minh hoàn tất.}
\end{gbtt}

\begin{gbtt}
Tìm số tự nhiên $n$ sao cho $36^n-6$ là tích của ít nhất hai số tự nhiên liên tiếp.
\nguon{Junior Balkan Mathematical Olympiad 2010}
\loigiai{
Do $36^n-6$ chia cho $4$ dư $2,$ số này không thể là tích của nhiều hơn $4$ số tự nhiên liên tiếp. Dựa vào đây, ta chia bài toán làm các trường hợp như sau.
\begin{enumerate}
    \item Nếu $36^n-6$ là tích hai số tự nhiên liên tiếp, ta đặt $36^n-6=m(m+1),$ trong đó $m$ là số nguyên dương. Phép đặt này cho ta
    \begin{align*}
        36^n=m^2+m+6&\Leftrightarrow 4\cdot 36^n=(2m+1)^2+23\\&\Leftrightarrow \left(2\cdot6^n-2m-1\right)\left(2\cdot6^n+2m+1\right)=23.
    \end{align*}
    Do $0<6^n-2m-1<6^n+2m+1,$ ta nhận được $$2\cdot6^n-2m-1=1,\qquad 2\cdot6^n+2m+1=23.$$
    Lấy tổng theo vế, ta có $4\cdot 6^n=24,$ và do đó $n=1.$
    \item Nếu $36^n-6$ là tích ba số tự nhiên liên tiếp, ta đặt $36^n-6=x(x+1)(x+2),$ trong đó $x$ là số nguyên dương. Phép đặt này cho ta
        $$36^n=x(x+1)(x+2)+6\Leftrightarrow  (6^n)^2=(x+3)\left(x^2+2\right).$$
    Ta tiếp tục đặt $d=\left(x+3,x^2+2\right).$ Lúc này
    $$
    \heva{
    &d\mid (x+3)\\
    &d\mid \left(x^2+2\right)\\
    &d\mid 36^n
    }
    \Rightarrow
    \heva{
    &d\mid (x+3)\\
    &d\mid (x-3)(x+3)+11\\
    &d\mid 36^n
    }
    \Rightarrow
    \heva{
    &d\mid 11\\
    &d\mid 36^n
    }    
    \Rightarrow d=1.
    $$
    Ta có $x^2+2$ là số chính phương. Đây là điều không thể xảy ra, do $x^2+2\equiv 2,3\pmod{4}.$
\end{enumerate}
Tổng kết lại, $n=1$ là số nguyên dương duy nhất thỏa yêu cầu.}
\end{gbtt}

\begin{gbtt} \label{bdscp2}
Cho hai số nguyên dương $x, y$ thỏa mãn $x^{2}-4y+1$ chia hết cho $(x-2 y)(2 y-1)$. Chứng minh rằng $|x-2 y|$ là số chính phương.
\nguon{Korean Mathematical Olympiad 2014}
\loigiai{Từ giả thiết, ta chỉ ra tồn tại số nguyên $z$ sao cho
$$x^2-4y^2+4y^2-4y+1=z(x-2y)(2y-1),$$
$$(2y-1)^2=(x-2y)\left[z(2y-1)-(x+2y)\right].$$
Đặt $d=(x-2y,z(2y-1)-(x+2y)).$ Ta lần lượt suy ra được
$$\heva{&d\mid (x-2y) \\ &d\mid \left[z(2y-1)-(x+2y)\right] \\&d\mid (2y-1)} 
\Rightarrow \heva{&d\mid (x-2y) \\ &d\mid (x+2y) \\ &d\mid (2y-1)}
\Rightarrow \heva{&d\mid 4y \\ &d\mid (2y-1)}
\Rightarrow d=1.$$
Theo đó, $|x-2y|$ là số chính phương. Bài toán được chứng minh.}
\end{gbtt}

\begin{gbtt}
Cho hai số nguyên dương $m, n$ và số nguyên tố $p$ thỏa mãn
$p=\dfrac{m+n}{2}+3 \sqrt{m n}.$ Chứng minh rằng $p+4 m$ và $p+4 n$ là các số chính phương.
\nguon{Vietnam Mathematical Young Talent Search 2019}
\loigiai{Với các số nguyên dương $p,m,n$ đã cho, ta có
$\left(\dfrac{2p-m-n}{6}\right)^2=mn.$
Ta đặt $d=(m,n),$ thế thì
$$\heva{&d\mid m  \\ &d\mid n  \\ &d\mid (2p-m-n)} 
\Rightarrow d\mid 2p\Rightarrow d\in\{1;2;p;2p\}.$$
Đến đây, ta xét các trường hợp sau.
\begin{enumerate}
    \item Với $d=1,$ cả $m$ và $n$ là số chính phương. Đặt $m=x^2$ và $n=y^2,$ với $x,y$ nguyên dương, và như vậy
    $$p=\dfrac{x^2+6xy+y^2}{2}.$$
    Do $d=1$ nên $x,y$ khác tính chẵn lẻ. Chứng minh được $x,y$ cùng lẻ giúp ta chỉ ra $x^2+y^2$ và $6xy$ đều chia $4$ dư $2,$ vậy nên $p$ chia hết cho $2,$ hay là $p=2.$ Điều này không thể xảy ra.
    \item Với $d$ là bội của $p,$ do  $p\mid m$ và $p\mid n$ nên ta có $m\ge p$ và $n\ge p,$ và vì vậy
    $$p=\dfrac{m+n}{2}+3\sqrt{mn}\ge \dfrac{p+p}{2}+3\sqrt{p^2}=4p.$$
    Mâu thuẫn xuất hiện trong đánh giá vừa rồi. Trường hợp này không xảy ra.
    \item Với $d=2,$ ta có thể đặt $m=2x^2$ và $n=2y^2,$ với $x,y$ nguyên dương. Theo đó
    $$p=\dfrac{m+n}{2}+3\sqrt{mn}=\dfrac{2x^2+2y^2}{2}+6xy= x^2+6xy+y^2.$$
    Cộng các vế thêm $4m,$ ta sẽ có
    $$p+4m=x^2+6xy+y^2+8x^2=\left(3x+y\right)^2$$
    là số chính phương. Chứng minh tương tự, $p+4n$ cũng là số chính phương.
\end{enumerate}
 Bài toán kết thúc.}
\end{gbtt}

\begin{gbtt}
Cho hai số nguyên dương $x,y$ phân biệt và số nguyên tố $p$ thỏa mãn $x+y\ne p.$ Chứng minh rằng $xy(p-x)(p-y)$ không thể là số chính phương.
\nguon{Poland Mathematical Olympiad}
\loigiai{
Ta giả sử $xy(p-x)(p-y)$ là bình phương một số tự nhiên. Ta đặt $(x(p-y),y(p-x))=d.$ Khi đó, do tích hai số nguyên tố cùng nhau
$$\left(\dfrac{x(p-y)}{d}\right)\left(\dfrac{y(p-x)}{d}\right)$$ là bình phương một số tự nhiên nên cả $\dfrac{x(p-y)}{d}$ và $\dfrac{y(p-x)}{d}$ đều là các số chính phương. Ta đặt
$$x(p-y)=dz^2,\qquad y(p-x)=dt^2,$$
trong đó, $z$ và $t$ là các số nguyên dương. Lấy hiệu theo vế, ta được
\begin{align*}
    d\tron{z^2-t^2}=x(p-y)-y(p-x)&\Rightarrow d(z-t)(z+t)=xp-xy-yp+xy\\&\Rightarrow d(z-t)(z+t)=p(x-y).
\end{align*}
Do $z\ne t$ nên $p$ là ước của $d(z-t)(z+t).$ Ta xét các trường hợp sau.  
\begin{enumerate}
    \item Nếu $d$ chia hết cho $p,$ ta có $p\mid x(p-y)$ và $p\mid y(p-x).$ Điều này là vô lí do bốn số $x,p-y,y,p-x$ đôi một phân biệt. 
    \item Nếu $z-t$ chia hết cho $p,$ áp dụng bất đẳng thức $AM-GM,$ ta có
    $$0<z-t<z=\sqrt{\dfrac{x(p-y)}{d}}\le \sqrt{x(p-y)}\le \dfrac{x+p-y}{2}<p.$$
    Do $p\mid(z-t),$ nhận xét trên dẫn đến mâu thuẫn.
    \item Nếu $z+t$ chia hết cho $p,$ áp dụng bất đẳng thức $AM-GM,$ ta có
    \begin{align*}
        0<z+t=\sqrt{\dfrac{x(p-y)}{d}}+\sqrt{\dfrac{y(p-x)}{d}}
        &\le \sqrt{x(p-y)}+\sqrt{y(p-x)}
        \\&\le \dfrac{x+p-y}{2}+\dfrac{y+p-x}{2}\\&=p.
    \end{align*}
    Do $p\mid (z+t),$ dấu bằng phải xảy ra, tức là $d=1,x+y=p.$ Điều này mâu thuẫn với giả thiết.
\end{enumerate}
Tóm lại, giả sử ban đầu là sai. Bài toán được chứng minh.}
\end{gbtt}

\begin{gbtt}
Cho các số nguyên dương $x,y$ thỏa mãn $x>y$ và $$(x-y, xy+1)=(x+y, xy-1)=1.$$
Chứng minh rằng $(x+y)^{2}+(x y-1)^{2}$ không phải là số chính phương.
\nguon{Iran Mathematical Olympiad 2010}
\loigiai{Giả sử tồn tại các số nguyên dương $x,y$ thỏa mãn đề bài. \\
Dựa theo định thức $Brahmagupta-Fibonacci,$ ta có phân tích 
\begin{align*}
    \left(x^2+1\right)\left(y^2+1\right)
    =(x-y)^2+(xy+1)^2
    =(x+y)^2+(xy-1)^2.
    \tag{*}
\end{align*}
Nếu $x^2+1$ và $y^2+1$ không nguyên tố cùng nhau, ta giả sử chúng có một ước nguyên tố chung là $p.$
Giả sử này cho ta $p^2\mid\left(x^2+1\right)\left(y^2+1\right),$ và đồng thời
$$\heva{&p\mid \left(x^2+1\right) \\&p\mid \left(y^2+1\right) }\Rightarrow p\mid (x-y)(x+y) \Rightarrow 
\hoac{
     p&\mid (x-y)  \\
     p&\mid (x+y).}$$
Đến đây, ta xét hai trường hợp.
\begin{enumerate}
    \item Với $p\mid (x-y),$ ta có $p^2\mid (x-y)^2.$ Kết hợp với (*), ta thu được $p^2\mid (xy+1)^2,$ tức là $p\mid (xy+1).$ Lúc này $p$ là một ước chung của $x-y$ và $xy+1,$ trái giả thiết hai số này nguyên tố cùng nhau.
    \item Với $p\mid (x+y),$ ta lập luận tương tự trường hợp trên để đi đến kết quả mâu thuẫn với giả thiết.    
\end{enumerate}
Như vậy, $x^2+1$ và $y^2+1$ phải nguyên tố cùng nhau. Theo đó, cả $x^2+1$ và $y^2+1$ đều là số chính phương. Tuy nhiên, thông qua các đánh giá
\begin{align*}
    x^{2}<x^{2}+1<(x+1)^2,
    \qquad y^2<y^2+1<(y+1)^2,
\end{align*}
hai số trên không thể là chính phương, mâu thuẫn với lập luận trước đó. \\
Giả sử phản chứng ban đầu là sai. Bài toán được chứng minh.}
\end{gbtt}

\begin{gbtt}
Cho tập hợp $\mathcal{X}$ gồm các số nguyên có dạng $a^2+2b^2$ với $a,b$ là các số nguyên và $b\ne 0.$ Chứng minh rằng nếu $p$ là số nguyên tố và nếu $p^2\in\mathcal{X}$ thì $p\in\mathcal{X}.$
\nguon{Group "Hướng tới Olympic Toán Việt Nam"}
\loigiai{
Vì $p^2\in\mathcal{X}$ nên tồn tại hai số nguyên $a,b$ thỏa mãn $b\ne 0$ và $p^2=a^2+2b^2.$ Bằng cách thử trực tiếp, ta thấy ngay $p\ne 2$, do đó $p$ là số lẻ, suy ra $a$ cũng là số lẻ. Ta có
$$p^2=a^2+2b^2\Rightarrow 1\equiv 1+2b^2\pmod{8}\Rightarrow 2b^2\equiv 0\pmod{8}.$$
Ta được $b$ là số chẵn. Quay trở lại bài toán, với $p$ là số nguyên tố thỏa mãn $p^2\in\mathcal{X},$ ta có
$$(p-a)(p+a)=2b^2\Leftrightarrow \tron{\dfrac{p-a}{2}}\tron{\dfrac{p+a}{2}}=\dfrac{b^2}{2}.$$
Đặt $d=\tron{\dfrac{p-a}{2},\dfrac{p+a}{2}}.$ Phép đặt này cho ta
\[\heva{&d\mid \dfrac{1}{2}\tron{p-a}\\&d\mid \dfrac{1}{2}\tron{p+a}}\Rightarrow\heva{&d\mid p\\&d\mid a}\Rightarrow d\mid (p,a).\]
Với việc chứng minh được $p>a,$ bắt buộc $(p,a)=1,$ và $d=1.$ Tới đây, ta xét các trường hợp sau.
\begin{enumerate}
    \item Nếu $p-a$ chia hết cho $4,$ ta xét phân tích
    $$\tron{\dfrac{p-a}{4}}\tron{\dfrac{p+a}{2}}=\tron{\dfrac{b}{2}}^2.$$
    Do $\tron{\dfrac{p-a}{4},\dfrac{p+a}{2}}=1$ nên cả hai số $\dfrac{p-a}{4},\dfrac{p+a}{2}$ đều là số chính phương. Ta đặt
    $$p-a=4m^2,\qquad p+a=2n^2.$$
    Lấy tổng theo vế, phép đặt này cho ta
    $$2p=4m^2+2n^2\Rightarrow p=2m^2+n^2\Rightarrow p\in\mathcal{X}.$$
    \item Nếu $p+a$ chia hết cho $4,$ bài toán được tiến hành tương tự.
\end{enumerate}
Như vậy, bài toán đã cho được chứng minh trong mọi trường hợp.}
\end{gbtt}

\begin{gbtt}
Tìm tất cả các số nguyên $a$ thỏa mãn với mọi số nguyên dương $n,$ ta có $5\left(a^n+4\right)$ là số chính phương.

\loigiai{
\begin{enumerate}
    \item Cho $n=1,$ ta chỉ ra $5(a+4)$ là số chính phương. Số này chia hết cho $25,$ kéo theo $$a\equiv 1\pmod{5}.$$
    Ngoài ra, do $5(a+4)$ là số chính phương nên $a\ge -4.$ Thử trực tiếp với $a=-3,-2,-1,0$ ta thấy không thỏa.
    \item Cho $n=4,$ ta chỉ ra $5\left(a^4+4\right)$ là số chính phương, thế nên $a\ne -4,$ đồng thời
    $$\dfrac{a^4+4}{5}=\dfrac{\left(a^2-2a+2\right)\left(a^2+2a+2\right)}{5}$$
    cũng là số chính phương. Ngoài ra, do nhận xét được rằng $a^2+2a+2$ chia hết cho $5$ từ $$a\equiv 1\pmod{5},$$ ta nghĩ đến việc đặt
    $$d=\left(a^2-2a+2,\dfrac{a^2+2a+2}{5}\right).$$
    Phép đặt này cho ta
    $$
    \heva{
    &d\mid \left(a^2-2a+2\right) \\
    &d\mid \left(a^2+2a+2\right)
    }
    \Rightarrow
    \heva{    
    &d\mid 4a \\
    &d\mid 2\left(a^2+2\right)
    }    
    \Rightarrow
    \heva{    
    &d\mid 4a \\
    &d\mid \left(4a^2+8\right)
    }      
    \Rightarrow d\mid8.
    $$
    Ta nhận thấy $d$ là lũy thừa của $2$ (với số mũ không vượt quá $3$). Tuy nhiên, do $$a^2-2a+2=(a-1)^2+1\equiv 1,2\pmod{4}$$
    nên $d=1$ hoặc $d=2,$ và điều này phụ thuộc vào tính chẵn lẻ của $a.$ Kết quả là
    \begin{itemize}
        \item Nếu $a$ lẻ, ta có $d=1,$ và $a^2-2a+2$ là số chính phương. Ta đặt
        $$a^2-2a+2=m^2,$$
        với $m$ là số nguyên dương. Phép đặt này cho ta
        $$(a-1)^2+1=m^2\Rightarrow (a-m-1)(a+m-1)=-1\Rightarrow a=m=1.$$
        \item Nếu $a$ chẵn, ta có $d=2,$ và $\dfrac{a^2+2a+2}{10}$ là số chính phương.
    \end{itemize}
    \item Cho $n=8$ và chỉ xét trường hợp $a$ là số chẵn, ta chỉ ra $\dfrac{a^4+2a^2+2}{10}$ là số chính phương với cách làm tương tự khi cho $n=4.$ Vì lẽ đó
    $$\left(a^4+2a^2+2\right)\left(a^2+2a+2\right)$$
    cũng là số chính phương. Các chứng minh ở phần đầu lời giải cho ta $a\ge 1,$ vì vậy ta có nhận xét
    $$\left(a^3+a^2+\dfrac{3a}{2}\right)^2<\left(a^4+2a^2+2\right)\left(a^2+2a+2\right)<\left(a^3+a^2+\dfrac{3a}{2}+2\right)^2.$$
    Ta suy ra $\left(a^4+2a^2+2\right)\left(a^2+2a+2\right)=\left(a^3+a^2+\dfrac{3a}{2}+1\right)^2.$ Ta không tìm được $a$ nguyên.
\end{enumerate}
Kết luận, $a=1$ là số nguyên duy nhất thỏa yêu cầu bài toán.}
\begin{luuy}
\nx{
\begin{enumerate}
    \item Mấu chốt của bài toán là tìm ra phân tích $a^4+4=\left(a^2-2a+2\right)\left(a^2+2a+2\right).$ Phân tích này mở ra cho ta hướng đi áp dụng phần lí thuyết đã học.
    \item Khi chỉ ra $\dfrac{a^2+2a+2}{10}$ là số chính phương, nhiều bạn không biết hướng giải quyết tiếp theo. Sử dụng phương pháp kẹp số chính phương là cách làm tốt nhất để vượt qua khúc mắc này. Nhằm phục vụ cho việc kẹp, các bước chặn $a$ đã được thể hiện ở phần đầu lời giải.
\end{enumerate}}
\end{luuy}
\end{gbtt}


\begin{gbtt}
Tìm các số nguyên tố $p$ sao cho $\dfrac{3^{p-1}-1}{p}$ là số chính phương.
\nguon{Saudi Arabia Junior Balkan Mathematical Olympiad Team Selection Test}
\loigiai{Ta thấy $p=2$ có thoả đề. Phần còn lại của bài toán, ta chỉ xét $p$ lẻ. Ta đặt $p=2k+1.$ Ta có $$\dfrac{3^{p-1}-1}{p}=\dfrac{\left(3^{k}-1\right)\left(3^{k}+1\right)}{p}.$$ 
Hai nhân tử $3^k-1$ và $3^k+1$ trong phân tích trên là hai số chẵn liên tiếp, thế nên chúng có ước chung lớn nhất bằng $2.$ Ta xét các trường hợp sau.
\begin{enumerate}
    \item Nếu $\dfrac{3^k-1}{2p}$ và $\dfrac{3^k+1}{2}$ là số chính phương, ta đặt $3^{k}+1=2v^{2}.$ Ta lần lượt nhận xét được
    $$3^k+1 \equiv 4\pmod{3}\Rightarrow 2v^2= 4 \pmod{3} \Rightarrow v^2 \equiv 2 \pmod{3}.$$
    Ta thu được mâu thuẫn. Trường hợp này không thỏa mãn.
    \item Nếu $\dfrac{3^k+1}{2p}$ và $\dfrac{3^k-1}{2}$ là số chính phương, ta đặt $3^{k}-1=2 x^{2}, 3^{k}+1=2 p y^{2}$. \begin{itemize}
        \item \chu{Trường hợp 1. }Với $k$ lẻ, ta có 
        $$2py^2\equiv 3^k + 1\equiv 3+1 = 4 \pmod{8} \Rightarrow py^2\equiv 2 \pmod{4}.$$
        Do $p$ lẻ nên ta được $y^2$ chẵn từ đây, tức $y^2$ chia hết cho $4,$ vô lí do $py^2\equiv 2 \pmod{4}.$
        \item \chu{Trường hợp 2. }Với $k$ chẵn, ta đặt $k=2q.$ Phép đặt này cho ta  $$\left(3^{q}-1\right)\left(3^{q}+1\right)=2 x^{2}.$$ 
        Ta dễ dàng chứng minh $\left(3^{q}-1,3^{q}+1\right)=2.$ Bằng lập luận theo modulo $3$ tương tự các trường hợp trước, ta chỉ ra $\dfrac{3^q+1}{2}$ không là số chính phương. Như vậy, ta có $3^{q}+1$ là số chính phương.\\ 
        Ta đặt $3^{q}+1=z^{2},$ khi đó tồn tại hai số tự nhiên $a,b$ sao cho 
        $$3^{q}=(z-1)(z+1)\Rightarrow \heva{z - 1&=3^a &\\ z+1&=3^b}\Rightarrow 2=3^a\left(3^{b-a}-1\right)\Rightarrow \heva{a&=0 \\ b&=2.}$$
    Thế ngược lại, ta lần lượt thu được $q=1,k=2,p=5.$
    \end{itemize}
\end{enumerate}
Tóm lại $p=2$ và $p=5$ là tất cả các số nguyên tố cần tìm.}
\end{gbtt}

\begin{gbtt}
Tìm các số nguyên tố $p$ sao cho $\dfrac{7^{p-1}-1}{p}$ là số chính phương.
\nguon{Định hướng bồi dưỡng học sinh năng khiếu Toán $-$ Tập 3}
\loigiai{
Bằng cách tiến hành tương tự các bài trước, ta chỉ ra $p=2k+1,$ đồng thời xét hai trường hợp như sau.
\begin{enumerate}
    \item Nếu $\dfrac{7^k-1}{2}$ và $\dfrac{7^k+1}{2p}$ là số chính phương, ta đặt $7^k-1=2v^{2}.$ Ta lần lượt suy ra
    \begin{align*}
        7^k-1 \equiv -1 \pmod{7}&\Rightarrow 2v^2 \equiv -1 \pmod{7}\\&\Rightarrow 8v^2 \equiv -4 \pmod{7}\\&\Rightarrow v^2 \equiv 3 \pmod{7}.
    \end{align*}
    Do $v^2$ chia $7$ chỉ có thể dư $0,1,2$ hoặc $4$, đồng dư thức $v^2\equiv 3\pmod{7}$ không xảy ra. Trường hợp này không thỏa mãn.
    \item Nếu $\dfrac{7^k+1}{2}$ và $\dfrac{7^k-1}{2p}$ là số chính phương, ta đặt $7^k+1=2 x^2, 7^k-1=2py^2.$
    \begin{itemize}
        \item \chu{Trường hợp 1. }Với $k=3z+1,$ ta có 
        $$2py^2=7^k-1=7^{3z+1}-1=7\cdot343^z-1\equiv 7-1=6 \pmod{9}.$$
        Ta suy ra $3\mid 2py^2$ từ đây. Rõ ràng, nếu $y$ là bội của $3,$ $2py^2$ sẽ chia hết cho $9,$ mâu thuẫn với $2py^2\equiv 6 \pmod{9}$. Điều này chứng tỏ $p=3.$ Thay ngược lại, ta thấy thỏa mãn. 
        \item \chu{Trường hợp 2. }Với $k=3z+2,$ ta có 
        $$2py^2=7^k-1=7^{3z+2}-1=49.343^z-1\equiv 4-1=3 \pmod{9}.$$
        Lập luận tương tự khả năng trên, ta loại trừ khả năng này.
        \item \chu{Trường hợp 3. }Với $k=3z,$ ta viết
        $$x^2=\dfrac{7^{3z}+1}{2}=\dfrac{7^z+1}{2}\left(7^{2z}-7^z+1\right).$$
        Ta suy ra $7^{2z}-7^z+1$ là số chính phương. Thế nhưng, điều này không xảy ra do
        $$\left(7^z-1\right)^2< 7^{2z}-7^z+1<\left(7^z\right)^2.$$
    \end{itemize}
\end{enumerate}
Tóm lại, $p=3$ là số nguyên tố duy nhất thỏa mãn đề bài.}
\end{gbtt}

\section{Căn thức trong số học}
\subsection*{Ví dụ minh họa}

\begin{bx}
Tìm tất cả các số nguyên dương $n$ thỏa mãn $\sqrt{\dfrac{4 n-2}{n+5}}$ là số hữu tỉ.
\nguon{Chuyên Tin Hà Nội 2015 $-$ 2016}
\loigiai{
Với số tự nhiên $n$ bất kì thỏa mãn đề bài, ta có
$$
\sqrt{\dfrac{4 n-2}{n+5}}=\dfrac{\sqrt{(4 n-2)(n+5)}}{n+5} \in \mathbb{Q}.
$$
Ta được $\sqrt{(4 n-2)(n+5)}$ là số hữu tỉ, và $(4 n-2)(n+5)$ là bình phương một số hữu tỉ. Ta đặt
$$(4n-2)(n+5)=A\text{ và }A=\dfrac{p^2}{q^2},\text{ trong đó }(p,q)=1.$$
Do $A$ là số tự nhiên nên $q^2\mid p^2,$ lại do $(p,q)=1$ nên $q=1.$ Như vậy $A$ là số chính phương. Ta nhận xét
$$(2 n+1)^{2}<(4 n-2)(n+5)<(2 n+5)^{2}.$$
Do $(4 n-2)(n+5)$ chẵn nên $(4 n-2)(n+5)$ nhận một trong các giá trị $(2 n+2)^{2}$ và $(2 n+4)^{2}$.
\begin{enumerate}
    \item Với $(4 n-2)(n+5)=(2 n+2)^{2}$, ta có $10 n=14$ hay $n=\dfrac{7}{5},$ trái điều kiện $n$ nguyên dương.
    \item Với $(4 n-2)(n+5)=(2 n+4)^{2}$, ta có $n=13.$
\end{enumerate}
Như vậy, có duy nhất một số nguyên dương $n$ thỏa mãn là $n=13$.}
\begin{luuy}
Trong bài toán trên, ta đã rút ra được một bổ đề quan trọng
\begin{quote}
    \it Nếu số tự nhiên $A$ là bình phương một số hữu tỉ thì $A$ là số chính phương.
\end{quote}
\end{luuy}
\end{bx}

\begin{bx}
Cho $150$ số thực $x_{1}, x_{2}, \ldots, x_{150},$ trong đó mỗi số trong chúng nhận một trong hai giá trị  $\sqrt{2}+1$ hoặc $\sqrt{2}-1.$ 
Ta xét tổng 
$$S=x_{1} x_{2}+x_{3} x_{4}+x_{5} x_{6}+\ldots+x_{149} x_{150}.$$
Hỏi ta có thể chọn ra $150$ số thực như trên sao cho $S=121$ được không?
\nguon{Argentina Team Selection Tests 2005}
\loigiai{
Mỗi tích có dạng $x_{2i-1}x_{2i}$ như trên nhận một trong ba giá trị, đó là
$$\tron{\sqrt{2}-1}^2=3-2\sqrt{2},\quad\tron{\sqrt{2}+1}^2=3+2\sqrt{2},\quad\tron{\sqrt{2}-1}\tron{\sqrt{2}+1}=1.$$
Ta gọi số tích bằng $3-2\sqrt{2},3+2\sqrt{2}$ và $1$ lần lượt bằng $x,y,z.$ Trong trường hợp $S=121,$ ta có
    $$\tron{3-2\sqrt{2}}x+\tron{3-2\sqrt{2}}y+z=121\Rightarrow 3x+3y+z+\sqrt{2}\tron{y-z}=121.$$
Nếu như $y\ne z,$ vế trái là số vô tỉ, vô lí. Do đó $y=z,$ và $6y+z=121.$ \\
Vậy chỉ cần $x=y=20$ và $z=1$ là $S=121.$ Câu trả lời cho bài toán là khẳng định.}
\end{bx}

\begin{bx}
Tìm tất cả các số thực $a$ sao cho $a+\sqrt{5}$ và $a^{2}+\sqrt{5}$ đều là số hữu tỉ.
\nguon{Vietnam Mathematical Young Talent Search 2019}
\loigiai{Giả sử tồn tại số $a$ thỏa mãn đề bài. Từ giả thiết, ta đặt $a+\sqrt{5}=x,$ ở đây $x$ là số hữu tỉ. Khi đó
$$a^2+\sqrt{5}=\left(x-\sqrt{5}\right)^2+5=x^2+5+(1-2x)\sqrt{5}.$$
Vì $a^2+5$ là số hữu tỉ nên ta được $(1-2x)\sqrt{5}$ cũng là số hữu tỉ. Ta bắt buộc phải có $x=\dfrac{1}{2},$ bởi lẽ nếu $1-2x\ne 0$ thì $(1-2x)\sqrt{5}$ là số vô tỉ. Nhận xét trên cho ta đáp số bài toán là $a=\dfrac{1-2\sqrt{5}}{2}.$}
\end{bx}

\begin{bx} Tìm tất cả các số thực $x$ thỏa mãn trong các số $$x-\sqrt{2},\quad x^2+2\sqrt{2},\quad x+\dfrac{1}{x},\quad x-\dfrac{1}{x}$$ có đúng một số không nguyên. 
\nguon{Chuyên Đại học Sư Phạm Hà Nội 2018}
\loigiai{Không thể xảy ra trường hợp cả $x+\dfrac{1}{x}$ và $x-\dfrac{1}{x}$ đều là số nguyên, bởi vì khi đó
$$x+\dfrac{1}{x}+x-\dfrac{1}{x}=2x$$
là số hữu tỉ, và $x$ hữu tỉ, vô lí. Như vậy, một trong hai số 
$$x+\dfrac{1}{x},\quad x-\dfrac{1}{x}$$ 
phải không nguyên, đồng thời cả hai số $x-\sqrt{2}$ và $x^2+2\sqrt{2}$ đều nguyên. \\
Do $x-\sqrt{2}$ nguyên nên ta có thể đặt $x-\sqrt{2}=a$, với $a$ là số nguyên. Khi đó vì 
$$x^2+2\sqrt{2}=\left(a+\sqrt{2}\right)^2+2\sqrt{2}=a^2+2+2\sqrt{2}(a+1)$$
là số nguyên nên $a=-1.$ Ta cần thử lại. Với $x=-1+\sqrt{2},$ ta có
\begin{align*}
    x+\dfrac{1}{x}&=-1+\sqrt{2}+\dfrac{1}{-1+\sqrt{2}}=2\sqrt{2}\notin\mathbb{Z},\\
	x-\dfrac{1}{x}&=-1+\sqrt{2}-\dfrac{1}{-1+\sqrt{2}}=-2\in\mathbb{Z}.
\end{align*}	
Như vậy, số thực $x$ thỏa yêu cầu bài toán là $x=-1+\sqrt{2}.$}
\end{bx}

\begin{bx} Tìm tất cả các số nguyên dương $n$ thỏa mãn $$\sqrt{n+2}+\sqrt{n+\sqrt{n+2}}$$ là một số nguyên dương. 
\nguon{Chuyên Toán Hà Nội 2019}
\loigiai{Đặt $\sqrt{n+2}+\sqrt{n+\sqrt{n+2}}=a$, ở đây $a$ là số nguyên dương. Ta có các biến đổi 
\begin{align*}
\sqrt{n+\sqrt{n+2}}=a-\sqrt{n+2} &\Rightarrow n+\sqrt{n+2}=a^2+n+2-2a\sqrt{n+2} \\&\Rightarrow (2a+1)\sqrt{n+2}=a^2+2
\\&\Rightarrow \sqrt{n+2}=\dfrac{a^2+2}{2a+1}.    
\end{align*}
Ta được $n+2$ là số chính phương. Tiếp tục đặt $n+2=x^2$ với $x$ nguyên dương, ta sẽ có $$\sqrt{n+2}+\sqrt{n+\sqrt{n+2}}=x+\sqrt{x^2+x-2}$$
là số nguyên dương, thế nên $x^2+x-2$ cũng nguyên dương. Bằng tính toán trực tiếp, ta chỉ ra được $$(x-1)^2<x^2+x-2<(x+1)^2.$$ Như vậy, $x^2+x-2=x^2,$ tức $x=2.$ Ta được $n=2,$ và đây là giá trị duy nhất của $n$ thỏa mãn đề bài.}
\end{bx}

\begin{bx}
Tìm tất cả các số nguyên dương $a,b,c$ thỏa mãn đồng thời hai điều kiện
\begin{enumerate}[i,]
    \item $\dfrac{a+b\sqrt{3}}{b+c\sqrt{3}}$ là số hữu tỉ.
    \item $a^2+b^2+c^2$ là số nguyên tố.
\end{enumerate}
\loigiai{Rõ ràng $b-c\sqrt{3}\ne 0,$ và ta có
\begin{align*}
    \dfrac{a+b\sqrt{3}}{b+c\sqrt{3}}
    &=\dfrac{\left(a+b\sqrt{3}\right)\left(b-c\sqrt{3}\right)}{\left(b-c\sqrt{3}\right)\left(b+c\sqrt{3}\right)}\\&
    =\dfrac{ab-3bc+\left(b^2-ca\right)\sqrt{3}}{b^2-3c^2}
    \\&
    =\dfrac{ab-3bc}{b^2-3c^2}+\dfrac{\left(b^2-ca\right)\sqrt{3}}{b^2-3c^2}.
\end{align*}
Cả  $\dfrac{a+b\sqrt{3}}{b+c\sqrt{3}}$ và $\dfrac{ab+3bc}{b^2-3c^2}$ đều là số hữu tỉ, thế nên $\dfrac{\left(b^2-ca\right)\sqrt{3}}{b^2-3c^2}$ hữu tỉ, tức là $b^2=ca.$ Ta có
$$a^2+b^2+c^2=a^2+2ac+c^2-b^2=(a+c)^2-b^2=(a+c-b)(a+b+c).$$
Do $a^2+b^2+c^2$ là số nguyên tố và $0<a+c-b<a+b+c,$ ta suy ra $a+c-b=1.$ \\
Kết hợp hai hệ thức $b=a+c-1$ và $b^2=ca,$ ta được
\begin{align*}
   (a+c-1)^2=ca
   &\Leftrightarrow a^2+ac+c^2-2a-2c+1=0
   \\&\Leftrightarrow a^2+(c-2)a+c^2-2c+1=0. 
\end{align*}
Coi đây là một phương trình bậc hai ẩn $a$ và tham số $c,$ khi đó
$$\Delta=(c-2)^2-4\left(c^2-2c+1\right)=(c-2)^2-4(c-1)^2=(4-3c)c$$
phải là số chính phương. Với việc $\Delta\ge 0,$ ta có $c=1.$\\ Bằng cách thay ngược lại, ta tìm ra $(a,b,c)=(1,1,1)$ là bộ số duy nhất thỏa mãn đề bài.}
\end{bx}

\begin{bx} 
Cho $a,b$ là hai số hữu tỉ. \\Chứng minh rằng nếu $a\sqrt{2}+b\sqrt{3}$ là số hữu tỉ thì $a=b=0.$
\nguon{Chuyên Đại học Sư phạm Hà Nội 2021}
\loigiai{
Từ giả thiết, ta có thể đặt $a\sqrt{2}+b\sqrt{3}=c,$ với $c$ là số hữu tỉ. Bình phương hai vế, ta được
$$2a^2+3b^2+2ab\sqrt{6}=c^2\Leftrightarrow 2a^2+3b^2-c^2=2ab\sqrt{6}.$$
Trong trường hợp $ab\ne 0,$ chia cả hai vế cho $ab,$ ta suy ra
$$\sqrt{6}=\dfrac{2a^2+3b^2-c^2}{2ab}.$$
Lập luận trên chứng tỏ $\sqrt{6}$ là số hữu tỉ, mâu thuẫn. Như vậy, ta thu được $ab=0.$
\begin{itemize}
    \item Nếu $a=0,$ ta suy ra $b\sqrt{3}$ là số hữu tỉ, thế nên $b=0.$
    \item Nếu $b=0,$ ta suy ra $a\sqrt{2}$ là số hữu tỉ, thế nên $a=0.$
\end{itemize}
Bài toán được chứng minh.}

\begin{it}
Tác giả xin phép tổng quát và mở rộng bổ đề được sử dụng trong bài toán trên.
\end{it}

\begin{light}
Với các số tự nhiên $a,b,$ ta có các khẳng định sau.
\begin{enumerate}
    \item Nếu $\sqrt{a}+\sqrt{b}$ là số hữu tỉ thì $a,b$ là các số chính phương.
    \item Nếu $\sqrt{a}-\sqrt{b}$ là số hữu tỉ thì hoặc $a,b$ là các số chính phương, hoặc $a=b.$
    \item Nếu $\sqrt{a}$ là số hữu tỉ thì $a$ là số chính phương.  
\end{enumerate}    
Tương tự, với các số hữu tỉ không âm $a,b,$ ta có các khẳng định sau.
\begin{enumerate}
    \item Nếu $\sqrt{a}+\sqrt{b}$ là số hữu tỉ thì $a,b$ là bình phương các số hữu tỉ.
    \item Nếu $\sqrt{a}-\sqrt{b}$ là số hữu tỉ thì hoặc $a,b$ là bình phương các số hữu tỉ, hoặc $a=b.$
    \item Nếu $\sqrt{a}$ là số hữu tỉ thì $a$ là bình phương một số hữu tỉ.     \end{enumerate}
\end{light}
\end{bx} %csp

\begin{it}
Đối với một số khẳng định tương tự dạng căn bậc cao hơn và phần chứng minh chúng, mời bạn đọc tự nghiên cứu.
\end{it}

\begin{bx}
Chứng minh không tồn tại số tự nhiên $n$ sao cho $\sqrt{n-1}+\sqrt{n+1}$ là số hữu tỉ.
\nguon{Chuyên Toán Phổ thông Năng khiếu 1996}
\loigiai{
Giả sử tồn tại số tự nhiên $n$ sao cho $\sqrt{n-1}+\sqrt{n+1}$
là số hữu tỉ. Theo như bổ đề đã phát biểu, hai số $n-1$ và $n+1$ đều là số chính phương. Ta đặt $n-1=x^2,$ với $x$ là số tự nhiên. Ta có
$$n+1=x^2+2\equiv 2,3\pmod{4}.$$
Không có số chính phương nào đồng dư $2$ hoặc $3$ theo modulo $4.$\\ Giả sử sai, và bài toán đã cho được chứng minh.}
\end{bx}

\begin{bx} \label{can1}
Chứng minh rằng với mọi số nguyên dương $n,$ ta luôn có 
$$\left(3+\sqrt{7}\right)^n+\left(3-\sqrt{7}\right)^n$$
là một số nguyên dương.
\loigiai{Ta sẽ chứng minh bài toán trên bằng quy nạp. Để đơn giản hóa việc quy nạp, ta đặt $$a=3+\sqrt{7},\quad b=3-\sqrt{7}.$$ Phép đặt này cho ta $a+b=6$ và $ab=-2.$ Đồng thời, ta đặt thêm
$$a_n=a^n+b^n.$$
Với $n=1,n=2$ bài toán được chứng minh do $a_1=6$ và $a_2=32.$ \\
Với $n\ge 3,$ ta sẽ xây dựng một hệ thức giữa $a_{n+2},a_{n+1},a_{n}.$ Thật vậy, ta có
\begin{align*}
    (a+b)\left(a^{n-1}+b^{n-1}\right)
    &=a^n+b^n+ab^{n-1}+ba^{n-1}
    \\&=a^n+b^n+ab\left(a^{n-2}+b^{n-2}\right).
\end{align*}
Ta dễ dàng tính toán được $a+b=6,ab=-2,$ thế nên là
$$6a_{n-1}=a_n-2a_{n-2}\Rightarrow a_n=6a_{n-1}+2a_{n-2}.$$
Hệ thức trên chứng tỏ $a_n$ nguyên dương với mọi $n\ge 2.$ Bài toán được chứng minh.
}
\end{bx}

\begin{bx}\label{can2}
Cho số nguyên $a$ và số nguyên dương $b.$ Chứng minh rằng ứng với mỗi số tự nhiên $n,$ tồn tại các số nguyên $x_n$ và $y_n$ sao cho
\begin{align*}
    &\left(a+\sqrt{b}\right)^n=x_n+y_n\sqrt{b},
    \\&\left(a-\sqrt{b}\right)^n=x_n-y_n\sqrt{b}.
\end{align*}
\loigiai{Ta chứng minh bài toán trên bằng phương pháp quy nạp. \\
Thật vậy, bài toán dễ dàng chứng minh với $n=1,n=2.$\\
Đối với $n\ge 2,$ tương tự \chu{ví dụ \ref{can1}}, ta sẽ xây dựng một hệ thức liên hệ giữa các đại lượng liên quan. Ta có
\begin{align*}
    \left(a+\sqrt{b}\right)^{n+1}
    &=\left(a+\sqrt{b}\right)\left(a+\sqrt{b}\right)^{n}\\&
    =\left(a+\sqrt{b}\right)\left(x_n+y_n\sqrt{b}\right)
    \\&=ax_n+by_n+\left(ay_n+bx_n\right)\sqrt{b}.
\end{align*}
Một cách tương tự, ta cũng chỉ ra
  $$\left(a-\sqrt{b}\right)^{n+1}
  =ax_n+by_n-\left(ay_n+bx_n\right)\sqrt{b}.$$
Ta chọn $x_{n+1}=ax_n+by_n$ và $y_{n+1}=ay_n+bx_n,$ khi đó bài toán được chứng minh theo quy nạp.}
\end{bx}

\subsection*{Bài tập tự luyện}
\begin{btt}
Tìm số nguyên dương $n$ để $\sqrt{\dfrac{n-23}{n+89}}$ là một số hữu tỉ dương.
\nguon{Chuyên Toán Nghệ An 2021}
\end{btt}

\begin{btt}
Tìm tất cả các số thực $x$ sao cho trong bốn số
$$x^2+4\sqrt{3},\quad x^2-\dfrac{4}{x},\quad x^2+\dfrac{4}{x},\quad x^4+56\sqrt{3},$$ 
có đúng một số không phải số nguyên.
\nguon{Tạp chí Pi tháng 7 năm 2017}
\end{btt}

\begin{btt}
Cho số thực $x$ khác $0$ thỏa mãn cả hai số $x+\dfrac{2}{x}$ và $x^3$ đều là số hữu tỉ. Chứng minh rằng $x$ cũng là số hữu tỉ.
\nguon{Chuyên Toán Hà Nội 2021}
\end{btt}

\begin{btt}
Tìm tất cả các số nguyên dương $n$ thỏa mãn đồng thời hai điều kiện
\begin{enumerate}
    \item[i,] $\sqrt{7\sqrt{n+4}-5\sqrt{n}}+\sqrt{7\sqrt{n+4}+5\sqrt{n}}$ là số nguyên dương.
    \item[ii,] $3n-13$ là số nguyên tố.
\end{enumerate}

\end{btt}

\begin{btt}
Tìm tất cả bộ ba số nguyên dương $(x,y,z)$ thỏa mãn
$$\sqrt{\dfrac{2005}{x+y}}+\sqrt{\dfrac{2005}{y+z}}+\sqrt{\dfrac{2005}{z+x}}$$
là một số nguyên dương
\nguon{Bulgarian Mathematical Olympiad 2005}
\end{btt}

\begin{btt}
Tìm tất cả các số tự nhiên $n$ thỏa mãn $\sqrt[3]{7+\sqrt{n}}+\sqrt[3]{7-\sqrt{n}}$
là một số nguyên dương.
\end{btt}

\begin{btt}
Tìm tất cả các số nguyên dương $x,y$ sao cho $$\sqrt{x^3-3x^2y+11x-3y}+\sqrt[3]{8y^3+3y^2+49}$$
là một số hữu tỉ dương.
\end{btt}

\begin{btt}
Tìm tất cả các số nguyên dương $n$ thỏa mãn đồng thời hai điều kiện
\begin{itemize}
    \item[i,] $\sqrt[3]{n-2}+\sqrt[3]{10n-3}$ là một số nguyên dương.
    \item[ii,] $2n+13$ là một số nguyên tố.
\end{itemize}
\end{btt}

\begin{btt}
Chứng minh rằng với mọi số nguyên dương $n$, tồn tại số nguyên dương $m$ sao cho
$$\left(\sqrt{2}-1\right)^n=\sqrt{m+1}-\sqrt{m}.$$
\nguon{Chọn đội tuyển chuyên Nguyễn Du, Đắk Lắk 2020}
\end{btt}

\begin{btt}
Chứng minh rằng tồn tại các số nguyên $a,b,c$ sao cho
$$0<\left|a+b\sqrt{2}+c\sqrt{3}\right|<\dfrac{1}{1000}.$$
\nguon{Chuyên Toán Hà Nội 2016}
\end{btt}

\subsection*{Hướng dẫn bài tập tự luyện}

\begin{gbtt}
Tìm số nguyên dương $n$ để $\sqrt{\dfrac{n-23}{n+89}}$ là một số hữu tỉ dương.
\nguon{Chuyên Toán Nghệ An 2021}
\loigiai{
Dễ thấy $n>23.$ Ta có thể đặt  $\dfrac{n-23}{n+89} = \left(\dfrac{a}{b} \right)^2,$ trong đó $a,b\in \mathbb{N}^*,(a,b)=1,a<b.$\\ Do $\tron{a^2,b^2}=1$ nên tồn tại số nguyên dương $k$ sao cho
$$n-23=a^2k,\quad n+89=b^2k.$$
Trừ tương ứng vế, ta suy ra $(b+a)(b-a)k=112.$ Ta sẽ lập bảng giá trị dựa trên các đánh giá
    \begin{itemize}
        \item[i,] $b+a$ và $b-a$ đều là ước dương của $112.$
        \item[ii,] $b+a$ và $b-a$ cùng tính chẵn lẻ.
        \item[iii,] $b+a>b-a>0.$
    \end{itemize}
Bảng giá trị của chúng ta như sau
\begin{center}
\begin{tabular}{c|c|c|c|c|c}
$b+a$ & $b-a$ & $k$ & $b$ & $a$& $n = a^2k+23$ \\ 
\hline
$28$&$4$ & $1$&$16$ &$12$ & loại vì $(a,b)>1$\\ 
$56$& $2$ &$1$ &$29$ &$27$ & $752$\\
$14$&$8$ & $1$&$11$ &$3$ & $32$\\ 
$14$& $4$ &$2$ &$9$ &$5$ & $73$\\
$28$&$2$ & $2$&$15$ &$13$ & $361$\\ 
$8$&$2$ & $7$ &$5$ &$3$ & $86$\\
$14$&$2$ & $4$&$8$ &$6$ & loại vì $(a,b)>1$\\ 
$4$&$2$ & $14$&$3$ &$1$ & $37$\\ 
$7$&$1$ & $16$&$4$ &$3$ & $167$
\end{tabular}
\end{center}
Kết quả, có $7$ giá trị của $n$ thỏa mãn đề bài, gồm $$n=32,\ n=37,\ n=73,\ n=86,\ n=167,\ n=361,\ n=752.$$}
\end{gbtt}

\begin{gbtt}
Tìm tất cả các số thực $x$ sao cho trong bốn số
$$x^2+4\sqrt{3},\quad x^2-\dfrac{4}{x},\quad x^2+\dfrac{4}{x},\quad x^4+56\sqrt{3},$$ 
có đúng một số không phải số nguyên.
\nguon{Tạp chí Pi tháng 7 năm 2017}
\loigiai{
Không thể xảy ra trường hợp cả $x^2-\dfrac{4}{x}$ và $x^2+\dfrac{4}{x}$ đều là số nguyên, vì khi đó
$$x^2+\dfrac{4}{x}-x^2+\dfrac{1}{x}=\dfrac{8}{x}$$
là số hữu tỉ, thế nên $x$ cũng là số hữu tỉ, vô lí. Như vậy, ít nhất một trong hai số
$$x^2-\dfrac{4}{x},\quad x^2+\dfrac{4}{x}$$
không nguyên, đồng thời
$x^2+4\sqrt{3}$ và $x^4+56\sqrt{3}$ là số nguyên. Đặt $x^2+4\sqrt{3}=a.$ Do
$$x^4+56\sqrt{3}=\left(a-4\sqrt{3}\right)^2+56\sqrt{3}= \left(a^2+48\right)+(56-8a)\sqrt{3}$$
là số nguyên nên $a=7.$ Thay ngược lại, ta có
$$x^2+4\sqrt{3}=7
\Rightarrow x^2=7-4\sqrt{3}
\Rightarrow x^2=\left(2-\sqrt{3}\right)^2
\Rightarrow
\hoac{&x=2-\sqrt{3} \\ &x=\sqrt{3}-2.}
$$
Sau khi kiểm tra, ta kết luận đây là hai giá trị của $x$ thỏa yêu cầu.}
\end{gbtt}

\begin{gbtt}
Cho số thực $x$ khác $0$ thỏa mãn cả hai số $x+\dfrac{2}{x}$ và $x^3$ đều là số hữu tỉ. Chứng minh rằng $x$ cũng là số hữu tỉ.
\nguon{Chuyên Toán Hà Nội 2021}
\loigiai{Từ giả thiết thứ nhất, ta có thế đặt $x+\dfrac{2}{x}=a,$ với $a$ là số hữu tỉ. Phép đặt này cho ta
    $$x^2-ax+2=0\Leftrightarrow \left(x-\dfrac{a}{2}\right)^2=\dfrac{a^2-8}{4}\Leftrightarrow x=\dfrac{a\pm \sqrt{a^2-8}}{2}.$$
    Kết hợp biến đổi vừa rồi giả thiết $x^3$ hữu tỉ, ta suy ra
    $$x^3=\left(\dfrac{a\pm \sqrt{a^2-8}}{2}\right)^3=\dfrac{1}{2}\left[\left(a^3-6a\right)+\left(a^2-2\right)\sqrt{a^2-8}\right]\in \mathbb{Q}.$$
    Nếu như $\sqrt{a^2-8}$ là số vô tỉ, ta bắt buộc phải có $a^2-2=0,$ tức là $a=\pm \sqrt{2}.$ Điều này mâu thuẫn với điều kiện phép đặt là $a$ hữu tỉ. Như vậy, $\sqrt{a^2-8}$ là số hữu tỉ. Dựa vào $x=\dfrac{a\pm \sqrt{a^2-8}}{2},$ ta thu được điều phải chứng minh.}
\end{gbtt}

\begin{gbtt}
Tìm tất cả các số nguyên dương $n$ thỏa mãn đồng thời hai điều kiện
\begin{enumerate}
    \item[i,] $\sqrt{7\sqrt{n+4}-5\sqrt{n}}+\sqrt{7\sqrt{n+4}+5\sqrt{n}}$ là số nguyên dương.
    \item[ii,] $3n-13$ là số nguyên tố.
\end{enumerate}

\loigiai{Với số nguyên dương $n$ thỏa mãn đề bài, ta chứng minh được
$$\left(\sqrt{7\sqrt{n+4}-5\sqrt{n}}+\sqrt{7\sqrt{n+4}+5\sqrt{n}}\right)^2=14\sqrt{n+4}+4\sqrt{6n+49}\in\mathbb{N^*}.$$
Theo như tính chất đã phát biểu, ta thu được $n+4$ và $6n+49$ là số chính phương. \\
Ta đặt $n+4=x^2,6n+49=y^2,$ ở đây $x,y$ là các số nguyên dương. Ta sẽ chọn $a,b$ sao cho
$$ax^2+by^2=a(6n+49)+b(n+4)=3n-13.$$
Giải hệ $\heva{& 6a+b=3 \\& 49a+4b=-13},$ ta được $a=-1,b=9.$ Bây giờ, ta xét phân tích
$$3n-13=9x^2-y^2=\left(3x+y\right)\left(3x-y\right).$$
Do $3n-13$ là số nguyên tố nên trong $3x+y,3x-y$ phải có một số bằng $1,$ nhưng vì $0<3x-y<3x+y$ nên $3x-y=1.$ Kết hợp với phép đặt $n+4=x^2,6n+49=y^2,$ ta thu được hệ nghiệm nguyên dương
$$\heva{& 3x-y=1\\& 6x^2-y^2+25=0 }\Leftrightarrow \heva{& y=3x-1 \\& 6x^2-(3x-1)^2+25=0 } \Leftrightarrow \heva{& x=4 \\& y=13.} $$
Thay ngược lại, ta tìm được $n=12.$ Đây chính là giá trị duy nhất của $n$ thỏa mãn đề bài.}
\end{gbtt}

\begin{gbtt}
Tìm tất cả bộ ba số nguyên dương $(x,y,z)$ thỏa mãn
$$\sqrt{\dfrac{2005}{x+y}}+\sqrt{\dfrac{2005}{y+z}}+\sqrt{\dfrac{2005}{z+x}}$$
là một số nguyên dương
\nguon{Bulgarian Mathematical Olympiad 2005}
\loigiai{Từ giả thiết, ta có thể đặt $A=\dfrac{2005}{x+y},\:B=\dfrac{2005}{y+z},\:C=\dfrac{2005}{z+x},\:D=\sqrt{A}+\sqrt{B}+\sqrt{C},$ ở đây $A,B,C$ là số hữu tỉ, còn $D$ là số nguyên dương. Phép đặt này cho ta
\begin{align*}
    D=\sqrt{A}+\sqrt{B}+\sqrt{C} &\Rightarrow D-\sqrt{A}=\sqrt{B}+\sqrt{C} \\&\Rightarrow D^2+A-2D\sqrt{A}=B+C+2\sqrt{BC} \\&\Rightarrow D^2+A-B-C=2D\sqrt{A}+2\sqrt{BC}.
\end{align*}
Ta suy ra $A$ là bình phương số hữu tỉ. Tiếp tục đặt $\dfrac{2005}{x+y}=\dfrac{m^2}{a^2},$ ở đây $(a,m)=1,$ ta sẽ có
$$2005a^2=m^2\left(x+y\right).$$
Ta có $m^2\mid2005a^2,$ tuy nhiên, do $(a,m)=1$ nên $m^2\mid2005,$ hay là $m=1.$\\
Thay ngược lại, ta được $x+y=2005a^2.$ Chứng minh tương tự, ta suy ra rằng ta có thể đặt $$y+z=2005b^2,\quad z+x=2005c^2.$$
Kết hợp với giả thiết, ta suy ra 
$$\sqrt{\dfrac{2005}{x+y}}+\sqrt{\dfrac{2005}{y+z}}+\sqrt{\dfrac{2005}{z+x}}=\dfrac{1}{a}+\dfrac{1}{b}+\dfrac{1}{c}\in \mathbb{N^*}.$$
Dễ thấy $a+b+c=2(x+y+z)$ là số chẵn, và đồng thời, điều kiện $a,b,c$ nguyên dương cho ta $$\dfrac{1}{a}+\dfrac{1}{b}+\dfrac{1}{c}\le 1+1+1=3.$$ 
Không mất tính tổng quát, ta giả sử $a\le b\le c.$ Ta xét các trường hợp sau.
\begin{enumerate}
    \item Nếu $\dfrac{1}{a}+\dfrac{1}{b}+\dfrac{1}{c}=3,$ ta lập tức thu được $a=b=c=1,$ và thế thì $a+b+c=3$ là số lẻ, vô lí.
    \item Nếu $\dfrac{1}{a}+\dfrac{1}{b}+\dfrac{1}{c}=2,$ ta có đánh giá
    $$2=\dfrac{1}{a}+\dfrac{1}{b}+\dfrac{1}{c}\le \dfrac{1}{a}+\dfrac{1}{a}+\dfrac{1}{a}=\dfrac{3}{a}.$$
    Đánh giá trên cho ta $a\le \dfrac{3}{2},$ tức $a=1.$ Thay ngược lại, ta được
    $$\dfrac{1}{b}+\dfrac{1}{c}=1\Leftrightarrow b+c=bc\Leftrightarrow (b-1)(c-1)=1.$$
    Ta suy ra $b=2,c=2,$ và $a+b+c=5$ là số lẻ. Trường hợp này không xảy ra. 
    \item Nếu $\dfrac{1}{a}+\dfrac{1}{b}+\dfrac{1}{c}=1,$ ta có đánh giá
    $$1=\dfrac{1}{a}+\dfrac{1}{b}+\dfrac{1}{c}\le \dfrac{1}{a}+\dfrac{1}{a}+\dfrac{1}{a}=\dfrac{3}{a}.$$
    Đánh giá trên cho ta $a\le 3,$ tức $a=1,a=2$ hoặc là $a=3.$
    \begin{itemize}
        \item Với $a=1,$ ta có $\dfrac{1}{b}+\dfrac{1}{c}=0.$ Phương trình này vô nghiệm nguyên.
        \item Với $a=2,$ ta có $$\dfrac{1}{b}+\dfrac{1}{c}=\dfrac{1}{2}
        \Leftrightarrow 2b+2c=bc
        \Leftrightarrow (b-2)(c-2)=4
        \Leftrightarrow \hoac
             {&b=3,c=6  \\
             &b=4,c=4.} $$
        Tuy nhiên, do $a+b+c$ chẵn nên ta chỉ có thể chọn $b=c=4.$ Thay ngược lại, ta tìm ra $$(x,y,z)=(2005\cdot 2,2005\cdot 2,2005\cdot 14).$$
        \item Với $a=3,$ do $a\le b\le c$ nên là
        $$1=\dfrac{1}{a}+\dfrac{1}{b}+\dfrac{1}{c}\ge\dfrac{1}{a}+\dfrac{1}{a}+\dfrac{1}{a}=\dfrac{3}{a}=1.$$
        Dấu bằng ở đánh giá trên phải xảy ra, tức $b=c=3.$ Lúc này, $a+b+c=9$ là số lẻ, vô lí.
    \end{itemize}
\end{enumerate}
Tóm lại, các bộ số $(x,y,z)$ cần tìm là $(2005\cdot 2,2005\cdot 2,2005\cdot 14)$ và các hoán vị.}
\end{gbtt}

\begin{gbtt}
Tìm tất cả các số tự nhiên $n$ thỏa mãn $\sqrt[3]{7+\sqrt{n}}+\sqrt[3]{7-\sqrt{n}}$
là một số nguyên dương.
\loigiai{Đặt $A=\sqrt[3]{7+\sqrt{n}}+\sqrt[3]{7-\sqrt{n}}.$ Ta có
\begin{align*}
    A^3
    &=14+3\sqrt[3]{\left(7+\sqrt{n}\right)\left(7-\sqrt{n}\right)}\left(\sqrt[3]{7+\sqrt{n}}+\sqrt[3]{7-\sqrt{n}}\right)
    \\&=14+3A\sqrt[3]{49-n}.
\end{align*}
Do $A>0,$ ta có $\dfrac{A^3-14}{3A}=\sqrt[3]{49-n}.$ Kết hợp với giả thiết $n\ge 0,$ ta lần lượt suy ra
\begin{align*}
    \dfrac{A^3-14}{3A}=\sqrt[3]{49-n}\le \sqrt[3]{49}<4\Rightarrow A^3-12A<14\Rightarrow A\left(A^2-12\right)<14.
\end{align*}
Nếu như $A\ge 4,$ ta có
$A\left(A^2-12\right)\ge 4\left(4^2-12\right)=16>14,$
mâu thuẫn với lập luận $$A\left(A^2-12\right)<14.$$ 
Mâu thuẫn này chứng tỏ $A\in \{1;2;3\}.$ Thử với trường trường hợp, ta thấy chỉ có $n=50$ thỏa mãn.}
\end{gbtt}

\begin{gbtt}
Tìm tất cả các số nguyên dương $x,y$ sao cho $$\sqrt{x^3-3x^2y+11x-3y}+\sqrt[3]{8y^3+3y^2+49}$$
là một số hữu tỉ dương.

\loigiai{Từ giả thiết, ta có thể đặt $A=x^3-3x^2y+11x-3y,B=8y^3+3y^2+49$ và $C=\sqrt{A}+\sqrt[3]{B},$ ở đây $A,B$ là các số nguyên dương, còn $C$ là số hữu tỉ dương. Phép đặt này cho ta
\begin{align*}
    \sqrt{A}+\sqrt[3]{B}=C
    \Rightarrow \sqrt[3]{B}=C-\sqrt{A}
    &\Rightarrow B=\left(C-\sqrt{A}\right)^3
    \\&\Rightarrow B=C^3+3CA^2-\sqrt{A}\left(3C^2+A^2\right)
    \\&\Rightarrow \sqrt{A}\left(3C^2+A^2\right)=C^3+3CA^2-B.
\end{align*}
Do $B>0$ nên $A$ và $C$ không đồng thời bằng $0.$ Ta được
$$\sqrt{A}=\dfrac{C^3+3CA^2-B}{3C^2+A^2}.$$
Theo bổ đề, $A$ là một số  chính phương, và như vậy, $B$ là một số lập phương. Nhờ vào nhận xét
$$8y^3<8y^3+3y^2+49<(2y+3)^3,$$
nên ta suy ra $8y^3+3y^2+49$ bằng $(2y+1)^3$ hoặc $(2y+2)^3.$\\ Thử với từng trường hợp, ta được $y=2$ khi $8y^3+3y^2+49=(2y+1)^3.$ Tiếp theo, ta có 
$$x^3-3x^2y+11x-3y=x^3-6x^2+11x-6=(x-1)(x-2)(x-3)$$
là một số chính phương. Ta đặt $d=\left((x-1)(x-3),(x-2)\right),$ và phép đặt này cho ta
$$\heva{&d\mid (x-2)\\ &d\mid (x-1)(x-3)}\Rightarrow \heva{&d\mid (x-2)\\ &d\mid \left[(x-2)^2-1\right]}\Rightarrow d\mid 1\Rightarrow d=1.$$
Ta nhận được $|(x-1)(x-3)|$ và $|x-2|$ là các số chính phương.\\
Để có thể phá dấu trị tuyệt đối, ta chia bài toán thành các trường hợp sau.
\begin{enumerate}
    \item Với $x=2,$ thử lại ta thấy thỏa mãn.
    \item Với $x\ne 2,$ do $x$ là số nguyên nên $x\ge 3$ hoặc $x\le 1,$ và do đó $|(x-1)(x-3)|=(x-1)(x-3).$ Vì
    $$(x-3)^2\le (x-1)(x-3)\le(x-3)^2,$$
    ta có thể suy ra được $(x-1)(x-3)$ bằng $(x-3)^2,(x-2)^2$ hoặc $(x-1)^2.$ \\
    Chia trường hợp để giải, ta chỉ ra $x=3,x=1$ thỏa mãn.
\end{enumerate}
Như vậy, có tất cả $3$ cặp $(x,y)$ thỏa mãn đề bài, gồm $(1,2),(2,2)$ và $(3,2).$}
\end{gbtt}

\begin{gbtt}
Tìm tất cả các số nguyên dương $n$ thỏa mãn đồng thời hai điều kiện
\begin{itemize}
    \item[i,] $\sqrt[3]{n-2}+\sqrt[3]{10n-3}$ là một số nguyên dương.
    \item[ii,] $2n+13$ là một số nguyên tố.
\end{itemize}
\loigiai{Ta đặt $a^3=n-2,b^3=10n-3.$ Phép đặt này cho ta
$$a^3+b^3=(a+b)\left(a^2-ab+b^2\right)=(a+b)^3-3ab(a+b)\in \mathbb{N^*}.$$
Do $a+b$ và $a^3+b^3$ là các số nguyên dương nên ta suy ra được $ab$ là số nguyên dương từ đây. Mặt khác
$$a^3-b^3=(a-b)\left(a^2+ab+b^2\right)=(a-b)\left[(a+b)^2-ab\right]\in \mathbb{N^*}.$$
Do $a^3-b^3$ và $(a+b)^2-ab$ là các số nguyên dương nên ta suy ra $a-b$ là số nguyên dương từ đây. Cả $a-b,a+b$ là nguyên dương, chứng tỏ $2a$ và $2b$ đều là số nguyên dương. Ta đặt
$$2a=A,\quad 2b=B.$$
Khi đó ta có $a^3=\dfrac{A^3}{8}$ và $b^3=\dfrac{B^3}{8}$ là số nguyên dương, thế nên $A,B$ chẵn, và kéo theo $a$ và $b$ nguyên dương. Ta cũng nhận thấy rằng
\begin{align*}
    2n+13=(10n-3)-8(n-2)
    =b^3-8a^3=(b-2a)\left(b^2+2ab+4a^2\right).
\end{align*}
Nhờ giả thiết $2n+13$ là một số nguyên tố và lập luận $0<b-2a<b<b^2+2ab+4a^2,$ ta có $b-2a=1.$ Từ đây, ta thu được hệ
\begin{align*}
    \heva{&b-2a=1 \\ &b^3-10a^3=17}
    \Leftrightarrow \heva{&b=2a+1 \\ &(2a+1)^3-10a^3=17}
   \Leftrightarrow \heva{&b=2a+1  \\ &(a-1)\left(a^2-5a-8\right)=0.}
\end{align*}
Không có số nguyên $a$ nào thỏa mãn $a^2-5a-8$ nên bắt buộc $a=1,$ và khi đó $b=3.$ \\
Từ đây, ta tính được $n=3.$ Đây là giá trị duy nhất của $n$ thỏa mãn đề bài. }
\end{gbtt}

\begin{gbtt}
Chứng minh rằng với mọi số nguyên dương $n$, tồn tại số nguyên dương $m$ sao cho
$$\left(\sqrt{2}-1\right)^n=\sqrt{m+1}-\sqrt{m}.$$
\nguon{Chọn đội tuyển chuyên Nguyễn Du, Đắk Lắk 2020}
\loigiai{
Với mỗi số tự nhiên $n,$ theo như \chu{ví dụ \ref{can2}}, tồn tại các số nguyên $A_n$ và $B_n$ sao cho
$$\left(-1+\sqrt{2}\right)^n=A_n+B_n\sqrt{2},\quad \left(-1-\sqrt{2}\right)^n=A_n-B_n\sqrt{2}.$$
Lấy tích theo vế, ta được
$(-1)^n=A_n^2-2B_n^2.$
Tới đây, ta xét các trường hợp sau.
\begin{enumerate}
    \item Nếu $n$ chẵn thì từ $\left(-1-\sqrt{2}\right)^n=A_n-B_n\sqrt{2}$ ta có $A_n>0$ và $B_n<0.$ Ngoài ra
    $$A_n^2-2B_n^2=1.$$
    Kết hợp hai dữ kiện vừa rồi, ta có
    $$\left(\sqrt{2}-1\right)^n=A_n+B_n\sqrt{2}=|A_n|-\sqrt{2B_n^2}=\sqrt{A_n^2}-\sqrt{A_n^2-1}.$$
    Số $m$ trong trường hợp này thỏa mãn $m=A_n^2.$
    \item Nếu $n$ lẻ thì từ $\left(-1-\sqrt{2}\right)^n=A_n-B_n\sqrt{2}$ ta có $A_n<0$ và $B_n>0.$ Ngoài ra
    $$A_n^2-2B_n^2=-1.$$
    Kết hợp hai dữ kiện vừa rồi, ta có
    $$\left(\sqrt{2}-1\right)^n=A_n+B_n\sqrt{2}=-|A_n|+\sqrt{2B_n^2}=-\sqrt{A_n^2}+\sqrt{A_n^2+1}.$$
    Số $m$ trong trường hợp này thỏa mãn $m=A_n^2+1.$    
\end{enumerate}
Bài toán đã cho được chứng minh trong mọi trường hợp.}
\end{gbtt}

\begin{gbtt}
Chứng minh rằng tồn tại các số nguyên $a,b,c$ sao cho
$$0<\left|a+b\sqrt{2}+c\sqrt{3}\right|<\dfrac{1}{1000}.$$
\nguon{Chuyên Toán Hà Nội 2016}
\loigiai{
Ta chọn $c=0.$ Lúc này, ta cần chọn $a,b$ sao cho
\[0<\left|a+b\sqrt{2}\right|<\dfrac{1}{1000}.\tag{*}\]
Bằng trực quan, ta thấy số $\left(\sqrt{2}-1\right)^n$ càng gần tới $0$ khi $n$ càng lớn.\\ Do $\sqrt{2}-1<\dfrac{1}{2},$ và $2^{10}>1000,$ ta nghĩ đến việc xét số 
$$\left(\sqrt{2}-1\right)^{10}<\left(\dfrac{1}{2}\right)^{10}=\dfrac{1}{1024}<\dfrac{1}{1000}.$$
Mặt khác, theo như \chu{ví dụ \ref{can2}}, ta chỉ ra tồn tại hai số nguyên $x,y$ sao cho
$$\left(\sqrt{2}-1\right)^{10}=x+y\sqrt{2}.$$
Chọn $a=x,b=y,$ khi đó (*) thỏa mãn. Bài toán được chứng minh.}
\end{gbtt}
 %số chính phương - bổ đề gcd + căn thức
\chapter{Đa thức}

Đa thức là một đối tượng nghiên cứu của phân môn Đại số. Song, ẩn chứa trong đó, đa thức cũng có nhiều tính chất số học thú vị. Đối với chương trình trung học cơ sơ, việc nghiên cứu các tính chất số học trong đa thức đa thức của ta chủ yếu nằm ở đa thức một biến trên trường số nguyên hoặc trường số hữu tỉ. Các lí thuyết  như định lí $Bezout,$ định lí $Viete$ hay tính chất về nghiệm hữu tỉ của đa thức nguyên là những kết quả đẹp và có nhiều ý nghĩa.\\ \\
Để làm rõ những tính chất trên và các bài toán liên quan, chương IV của cuốn sách được chia là 4 phần
\begin{itemize}
    \item\chu{Phần 1.} Tính chia hết của đa thức.
    \item\chu{Phần 2.} Phép đồng nhất hệ số trong đa thức.
    \item\chu{Phần 3.} Nghiệm của đa thức.
    \item\chu{Phần 4.} Đa thức và phương trình bậc hai
\end{itemize}

\section*{Các định nghĩa, phép toán và kí hiệu}
\begin{enumerate}
    \item Định nghĩa đa thức, đơn thức
    \begin{itemize}
        \item Đơn thức là một biểu thức đại số gồm một số, hoặc một biến, hoặc một tích giữa các số và các biến.
        \item Đa thức là một tổng của những đơn thức.
    \end{itemize}
    \item Các thành phần của đa thức
    \begin{itemize}
        \item Bậc của đa thức là bậc của hạng tử có bậc cao nhất trong dạng thu gọn của đa thức đó.
        \item Hệ số bậc $n$ của đa thức là hệ số của đơn thức bậc $n$ trong đa thức ấy.    
    \end{itemize}
    \item Các phép toán thông thường trên đa thức trong chương trình phổ thông, bao gồm
    \begin{itemize}
        \item Phép cộng trừ hai hoặc nhiều đa thức.
        \item Phép nhân hai hoặc nhiều đa thức.
        \item Phép chia đa thức cho đa thức không đồng nhất với $0.$
    \end{itemize}
    \item Nghiệm của đa thức $P(x)$ là một giá trị $a$ nào đó mà tại $x=a,$ đa thức $P(x)$ có giá trị bằng $0.$ 
    \item Một số kí hiệu sử dụng ở trong sách
    \begin{itemize}
        \item $\deg P:$ kí hiệu cho số chỉ bậc của đa thức $P(x).$
        \item $P(x)\equiv a:$ kí hiệu thay cho $P(x)=a,\forall x\in \mathbb{R}.$
    \end{itemize}
\end{enumerate}

\section{Tính chia hết của đa thức}

\subsection*{Lí thuyết}

Cho đa thức $P(x)$ với hệ số nguyên.
\begin{enumerate}
    \item Đa thức $P(x)$ chia hết cho đa thức $Q(x)$ khi và chỉ khi tồn tại đa thức $R(x)$ thỏa mãn
    $$P(x)=Q(x)R(x).$$
    \item Khi nói đa thức $P(x)$ chia cho đa thức $Q(x)$ được thương là $R(x)$ và dư là $S(x),$ ta viết
    $$P(x)=Q(x)R(x)+S(x).$$  
    Ngoài ra, ta còn có $\deg S<\deg Q.$
    \item \chu{Định lý Bezout.} Đa thức $P(x)$ chia cho đa thức $x-a$ được dư là $R$ thì $R=f(a).$
    \item Nếu đa thức $P(x)$ chia hết cho các đa thức $$R_1(x),R_2(x),...,R_n(x)$$ đôi một nguyên tố cùng nhau thì tồn tại đa thức $Q(x)$ sao cho
        $$P(x)=R_1(x)R_2(x)\ldots R_n(x)Q(x).$$
\end{enumerate}

\subsection*{Ví dụ minh họa}
\begin{bx}
Cho đa thức $P(x)$ với hệ số nguyên. Biết rằng $P(1)=5,$ hãy chứng minh $P(12)\ne 40.$
\loigiai{
Ta đặt $P(x)=a_nx^n+a_{n-1}x^{n-1}+\ldots+a_0$, với $a_n \ne 0$. Phép đặt này cho ta
\begin{align*}
    P(12)&=12^na_n+12^{n-1}a_{n-1}+\ldots+a_0,
    \\P(1)&=a_n+a_{n-1}+\ldots+a_0.
\end{align*}
Ta giả sử $P(12)=40.$ Lấy hiệu theo vế, ta được
\begin{align*}
    P(12)-P(1)&=\left(12^na_n+12^{n-1}a_{n-1}+\ldots+a_0\right)-\left(a_n+a_{n-1}+\ldots+a_0\right).\\
    35&=\left(12^n-1\right)a_n+\left(12^{n-1}-1\right)a_{n-1}+\ldots+\left(12-1\right)a_{1}.
\end{align*}
Với mọi số nguyên dương $m,$ ta có $12^m-1$ chia hết cho $11.$ Đối chiếu với đánh giá vừa rồi, ta suy ra $35$ chia hết cho $11,$ một điều vô lí. Như vậy, giả sử là sai, và ta có điều phải chứng minh.}
\begin{luuy}
Qua bài toán trên, ta khám phá được thêm tính chất sau
\begin{quote}
\it
    Với mọi số nguyên $a,b$ khác nhau và mọi đa thức $P(x)$ hệ số nguyên, ta có
    $$(a-b)\mid \left[P(a)-P(b)\right].$$
\end{quote}
Ngoài cách chứng minh trực tiếp bằng xét hiệu, bạn đọc có thể giải quyết bài toán bằng cách sử dụng định lí $Bezout$. Cụ thể, do $a$ là nghiệm của đa thức $P(x)-P(a)$ nên
$$P(x)=(x-a)Q(x)+P(a),$$
trong đó $Q(x)$ là một đa thức hệ số nguyên. Cho $x=b$ ta được
$$P(b)=(b-a)Q(x)+P(a).$$
Ta dễ dàng suy ra $P(a)-P(b)$ chia hết cho $a-b$ từ đây.
\end{luuy}
\end{bx}

\begin{bx}
Cho đa thức $f(x)$ có các hệ số nguyên. Biết rằng $f(1)f(2) = 35$. Chứng minh rằng đa thức $f(x)$ không có nghiệm nguyên.
\loigiai 
{Giả sử đa thức $f(x)$ có nghiệm nguyên $a,$ thế thì $f(x)=(x-a)g(x)$, trong đó $g(x)$ là đa thức có các hệ số nguyên. Kết hợp với giả thiết, ta có
	\begin{align*}
		f(1)&=(1-a)g(1), \\
		f(2)&=(2-a)g(2).
	\end{align*}
Lấy tích theo vế, ta được $35=(1-a)(2-a)g(1)g(2).$ Tuy nhiên, đẳng thức trên không xảy ra vì vế trái là số lẻ, còn vế phải chia hết cho $(a-1)(a-2)$ là một số chẵn. Giả sử phản chứng là sai. Bài toán được chứng minh.}
\end{bx}

\begin{bx}
Cho đa thức $P(x)$ có các hệ số nguyên. Biết rằng $P(x)$ không chia hết cho $3$ với ba giá trị nguyên liên tiếp nào đó của $x,$ chứng minh rằng $P(x)$ không có nghiệm nguyên.
\loigiai {
Giả sử tồn tại số nguyên $a$ sao cho $P(a),P(a+1),P(a+2)$ đều không chia hết cho $3.$ \\
Tiếp theo, giả sử phản chứng rằng đa thức $P(x)$ có nghiệm nguyên $x_0$ thế thì $$P(x)=(x-x_0)g(x_0),$$ trong đó $Q(x)$ là đa thức có các hệ số nguyên. Kết hợp các giả sử, ta có
	\begin{align*}
		P(a)&=(a-x_0)Q(a), \\
		P(a+1)&=(a+1-x_0)Q(a+1), \\
		P(a+2)&=(a+2-x_0)Q(a+2).
	\end{align*}
Lấy tích theo vế, ta được $$P(a)P(a+1)P(a+2)=(a-x_0)(a+1-x_0)(a+2-x_0)Q(a)Q(a+1)Q(a+2).$$ 
Tuy nhiên, đẳng thức trên không xảy ra vì vế trái không chia hết cho $3$, còn vế phải chia hết cho $$(a-x_0)(a+1-x_0)(a+2-x_0)$$ 
là một tích ba số nguyên liên tiếp. Giả sử phản chứng là sai. Bài toán được chứng minh.}
\end{bx}

\begin{bx}
Biết rằng đa thức $P(x)$ nhận giá trị bằng $2$ với $4$ giá trị nguyên khác nhau của $x.$ Chứng minh rằng không tồn tại số nguyên $x_0$ sao cho $P(x_0)=2019.$
\loigiai{Xét đa thức
$Q(x)=P(x)-2.$ Từ giả thiết, ta suy ra $Q(x)$ có $4$ nghiệm nguyên phân biệt. Ta gọi $4$ nghiệm này là $x_1,x_2,x_3,x_4.$ Theo định lí $Bezout$, ta có
$$
Q(x)=\left(x-x_{1}\right)\left(x-x_{2}\right)\left(x-x_{3}\right)\left(x-x_{4}\right) R(x),
$$
ở đây $R(x)$ là một đa thức hệ số nguyên. \\
Bây giờ, giả sử tồn tại số nguyên $x_0$ thỏa mãn $P(x_0)=2019.$ Giả sử này cho ta $Q(x_0)=2017,$ hay là
$$\left(x_0-x_{1}\right)\left(x_0-x_{2}\right)\left(x_0-x_{3}\right)\left(x_0-x_{4}\right) R(x_0)=2017.$$
Do $2017$ là số nguyên tố nên có ít nhất $3$ trong $4$ số $$x_0-x_1,\,\,x_0-x_2,\,\,x_0-x_3,\,\,x_0-x_4$$ có trị tuyệt đối bằng $1.$ Theo nguyên lí $Dirichlet$, có ít nhất hai số bằng nhau trong ba số này, giả sử là $x_0-x_1$ và $x_0-x_2.$ Giả sử này cho ta $x_0-x_1=x_0-x_2,$ hay là 
$$x_1=x_2.$$
Đây là điều mâu thuẫn với giả thiết. Giả sử phản chứng là sai. Bài toán được chứng minh.}
\end{bx}

\begin{bx}
Cho đa thức $P(x)=\dfrac{x^5}{5}+\dfrac{x^3}{3}+\dfrac{7x}{15}.$ Chứng minh rằng đa thức $P(x)$ nhận giá trị nguyên với mọi số nguyên $x.$
\loigiai{
Ta viết lại đa thức $P(x)$ như sau
$P(x)=x+\dfrac{x^5-x}{5}+\dfrac{x^3-x}{3}.$ Theo định lí $Fermat$ nhỏ, ta có $5\mid \left(x^5-x\right)$ và $3\mid \left(x^3-x\right).$ Như vậy $P(x)\in\mathbb{Z}$ với mọi $x\in \mathbb{Z}.$ Đây chính là điều phải chứng minh.}
\begin{luuy}
\begin{enumerate}
    \item Ta nhắc lại định lí $Fermat$ nhỏ kèm hệ quả của nó.
\begin{enumerate}
    \item[i,] Nếu số nguyên dương $a$ không chia hết cho số nguyên tố $p$ thì $a^{p-1}-1$ chia hết cho $p.$
    \item[ii,] Với mọi số nguyên dương $a$ và số nguyên tố $p,$ ta có $a^p-a$ chia hết cho $p.$
\end{enumerate}
    \item Ngoài ra, các bổ đề chia hết áp dụng trong bài trên đã được chứng minh không bằng định lí $Fermat$ ở phần phụ lục và \chu{chương I}.
\end{enumerate}
\end{luuy}
\end{bx}

\begin{bx}
Tìm tất cả các số nguyên $a,b$ sao cho đa thức bậc ba $P(x)=x^3+ax+b$ nhận giá trị chia hết cho $3$ với mọi số nguyên $x.$
\loigiai{
\begin{enumerate}
    \item Giả sử $P(x)$ chia hết cho $3$ với mọi $x$ nguyên. Ta đã biết $x^3-x$ chia hết cho $3,$ vì vậy
    $$P(x)\equiv (a+1)x+b\equiv \heva{
    b, &\text{ nếu } x=3k \\
    a+b+1, &\text{ nếu } x=3k+1 \\
    2a+b+2, &\text{ nếu } x=3k+2
    } \pmod{3}.$$
    Dựa vào nhận xét trên, ta chỉ ra được 
    $$b\equiv 0\pmod{3},\quad a\equiv 2\pmod{3}.$$
    \item Đảo lại, với $b$ và $a-2$ đều chia hết cho $3,$ ta viết lại $P(x)$ dưới dạng
    $$P(x)=x^3-x+(a+1)x+b.$$
    Do cả $x^3-x,a+1$ và $b$ đều chia hết cho $3$ nên điều kiện đủ được chứng minh.
\end{enumerate}
Như vậy, các giá trị của $a,b$ thỏa mãn là $b\equiv 0\pmod{3},\: a\equiv 2\pmod{3}.$ Bài toán được giải quyết.}
\end{bx}

\begin{bx}
Tìm tất cả các số nguyên dương $n$ sao cho đa thức $$P(x)=x^{n+1}+x^n+x^3+1$$ chia hết cho đa thức $x^2+1.$
\loigiai{
Ta sẽ đi tìm số dư của đa thức $P(x)$ khi chia cho $x^2+1.$ Ta đã biết
$$x^4-1=\left(x^2+1\right)\left(x^2-1\right).$$
Vì lẽ đó, ta nghĩ đến việc đặt $n=4m+r,$ với $m,r$ là các số tự nhiên và $0\le r\le 3.$ Phép đặt này cho ta
\begin{align*}
    P(x)
    &=x^{4m+r+1}+x^{4m+r}+x^3+1
    \\&=\left(x^{4m+r+1}-x^{r+1}\right)+\left(x^{4m+r}-x^{r}\right)+\left(x^3+x\right)+x^{r+1}+x^r-x+1.
    \\&=x^{r+1}\left(x^{4m}-1\right)+x^r\left(x^{4m}-1\right)+x\left(x^2+1\right)+x^{r+1}+x^r-x+1.    
\end{align*}
Giả sử tồn tại đa thức $P(x)$ thỏa mãn đề bài. Nhờ vào nhận xét
$$\tron{x^2+1}\mid\tron{x^4-1}\mid\tron{x^{4m}-1}.$$
ta suy ra đa thức $x^{r+1}+x^r-x+1$ chia hết cho đa thức $x^2+1.$
\begin{enumerate}
    \item Với $r=0,$ ta có $x^{r+1}+x^r-x+1=2$ không chia hết cho đa thức $x^2+1.$
    \item Với $r=1,$ ta có $x^{r+1}+x^r-x+1=x^2+1$ chia hết cho đa thức $x^2+1.$
    \item Với $r=2,$ ta có đa thức $$x^{r+1}+x^r-x+1=x^3+x^2-x+1=\left(x^2+1\right)(x+1)-2x$$ không chia hết cho đa thức $x^2+1.$
    \item Với $r=3,$ ta có đa thức $$x^{r+1}+x^r-x+1=x^4+x^3-x+1=\left(x^2+1\right)\left(x^2+x-1\right)-2x+2$$ 
    không chia hết cho đa thức $x^2+1.$
\end{enumerate}
Như vậy, tất cả các số nguyên $n$ cần tìm là $n=4m+1,$ với $m$ là số tự nhiên.} 
\end{bx}

\subsection*{Bài tập tự luyện}

\begin{btt}
Với $b,c$ là các số nguyên và $a$ là số nguyên dương, xét đa thức $P(x)=ax^2+bx+c.$  Biết $P(9)-P(6)=2019,$ chứng minh rằng $P(10)-P(7)$ là một số lẻ.
\nguon{Chuyên Toán Nghệ An 2019}
\end{btt}

\begin{btt}
Cho đa thức $f(x)$ có hệ số nguyên. Chứng minh rằng không tồn tại ba số nguyên phân biệt $a,b,c$ sao cho $f(a)=b,f(b)=c,f(c)=a.$
\end{btt}

\begin{btt}
Cho đa thức $P(x)$ với hệ số nguyên thỏa mãn $P(0)P(1)P(4)P(7)P(8)=19.$ Chứng minh rằng đa thức này không có nghiệm nguyên.
\end{btt}

\begin{btt}
Cho đa thức $P(x)$ có hệ số nguyên. Giả sử tồn tại các số nguyên $a,b,c,d$ thỏa mãn đồng thời các điều kiện
\begin{enumerate}[i,]
    \item $a<b,c<d,a<c.$
    \item$P(a)=P(b)=1,P(c)=P(d)=-1.$
\end{enumerate}
Chứng minh $a,b,c,d$ là bốn số nguyên liên tiếp.

\end{btt}

\begin{btt}
Cho $p_{1}, p_{2}, p_{3}, p_{4}$ là bốn số nguyên tố phân biệt. Chứng minh rằng không tồn tại đa thức $Q(x)$ bậc ba hệ số nguyên thỏa mãn $$\left|Q\left(p_{1}\right)\right|=\left|Q\left(p_{2}\right)\right|=\left|Q\left(p_{3}\right)\right|=\left|Q\left(p_{4}\right)\right|=3.$$
\nguon{Cao Đình Huy}
\end{btt}    

\begin{btt}
Tìm tất cả các cặp số nguyên $(a,b)$ để đa thức hệ số nguyên $P(x)=x^2-ax+b$ thỏa mãn tính chất:
\begin{it}
Tồn tại ba số nguyên đôi một khác nhau $m$, $n$, $p$ thuộc đoạn $[1;9]$ sao cho
\end{it}
	$$|P(m)|=|P(n)|=|P(p)|=7.$$
\nguon{Tạp chí Pi tháng 11 năm 2017}
\end{btt}

\begin{btt}
Cho $n$ số nguyên phân biệt $a_{1}, a_{2}, \ldots, a_{n}$. Chứng minh rằng đa thức
$$P(x)=\left(x-a_{1}\right)\left(x-a_{2}\right) \ldots\left(x-a_{n}\right)-1.$$
không thể phân tích được thành tích của hai đa thức hệ số nguyên khác hằng.
\end{btt}

\begin{btt}
Cho $k\ge 6$ số nguyên dương $x_1<x_2<\ldots<x_k.$ Giả sử tồn tại đa thức $P(x)$ với hệ số nguyên thỏa mãn $P\left(x_1\right),P\left(x_2\right),\ldots,P\left(x_k\right)$ nhận giá trị trong đoạn $[1;k-1].$
\begin{enumerate}[a,]
    \item Chứng minh rằng $P\left(x_1\right)=P\left(x_k\right).$
    \item Chứng minh rằng $P\left(x_1\right)=P\left(x_2\right)=\ldots=P\left(x_k\right).$
\end{enumerate}
\nguon{Moscow Mathematical Olympiad 2008}
\end{btt}

\begin{btt}
Cho $P(x)$ là một đa thức hệ số nguyên có bậc tối đa là mười. Biết rằng tồn tại các số nguyên $x_1,x_2,\ldots,x_{10}$ sao cho
$$P\left(x_1\right)=1,P\left(x_2\right)=2,\ldots,P\left(x_{10}\right)=10.$$
\begin{enumerate}[a,]
    \item Chứng minh rằng $x_1,x_2,\ldots,x_{10}$ là một dãy số nguyên cách đều nhau.
    \item Giả sử $\left|P(10)-P(0)\right|<1000.$ Chứng minh rằng với mọi số nguyên $k,$ luôn tồn tại số tự nhiên $m$ sao cho $P(m)=k.$
\end{enumerate}
\nguon{Titu Andreescu}   
\end{btt}

\begin{btt}
Cho đa thức $P(x)=x^{4}-x^{3}-3 x^{2}-x+1.$ Chứng minh rằng tồn tại vô số số nguyên dương $n$ sao cho $P\left(3^{n}\right)$ là hợp số.
\nguon{Mediterranean Competition 2015}
\end{btt}


\begin{btt}
Tìm tất cả các số thực $a,b,c$ sao cho đa thức bậc hai $P(x)=ax^2+bx+c$ nhận giá trị nguyên với mọi giá trị nguyên của $x.$
\end{btt}

\begin{btt}
Cho đa thức $P(x)$ bậc bốn với hệ số nguyên. Chứng minh rằng $P(x)$ chia hết cho $7$ với mọi số nguyên $x$ khi và chỉ khi tất cả các hệ số của $P(x)$ đều chia hết cho $7.$

\end{btt}

\begin{btt}
Tìm tất cả các số tự nhiên $n$ sao cho đa thức
$$P(x)=x^{3n+7}+2x^{2n}+x^5-x^4+1$$
chia hết cho đa thức $x^5+x^4+x+1.$
\end{btt}

\begin{btt}
Tìm tất cả các số nguyên dương $a,b$ sao cho đa thức $P(x)=x^a+x^b+1$ chia hết cho đa thức $Q(x)=x^2+x+1.$
\end{btt}

\begin{btt}
Với $n$ là một số nguyên dương, xét đa thức $P_n(x)=(x-1)^n+(x+2)^n.$ Xác định $n$ để $P_n(x)$ chia hết cho $2x^2+2x+5.$
\nguon{Chọn đội tuyển chuyên Trần Phú $-$ Hải Phòng 2021}
\end{btt}

\begin{btt}
Cho $P(x)$ và $Q(x)$ là hai đa thức với hệ số nguyên thỏa mãn đa thức $P\tron{x^3}+xQ\tron{x^3}$ chia hết cho đa thức $x^2+x+1.$ Chứng minh rằng $P(2021)-Q(2021)$ chia hết cho $2020.$
\end{btt}

\subsection*{Hướng dẫn bài tập tự luyện}

\begin{gbtt}
Với $b,c$ là các số nguyên và $a$ là số nguyên dương, xét đa thức $P(x)=ax^2+bx+c.$  Biết $P(9)-P(6)=2019,$ chứng minh rằng $P(10)-P(7)$ là một số lẻ.
\nguon{Chuyên Toán Nghệ An 2019}
\loigiai{
Dựa theo tính chất đã biết, ta có
\begin{align*}
    4&=(10-6) \mid \left[P(10)-P(6)\right],
   \\
   2 &= (9-7) \mid
   \left[P(9)-P(7)\right].
\end{align*}
Như vậy, $P(9)-P(6)$ và $P(10)-P(7)$ có cùng tính chẵn lẻ. Do giả thiết $P(9)-P(6)=2019$ lẻ, ta suy ra $P(10)-P(7)$ cũng là số lẻ. Đây chính là điều phải chứng minh.}
\end{gbtt}


\begin{gbtt}
Cho đa thức $f(x)$ có hệ số nguyên. Chứng minh rằng không tồn tại ba số nguyên phân biệt $a,b,c$ sao cho $f(a)=b,f(b)=c,f(c)=a.$
\loigiai{Giả sử tồn tại ba số nguyên $a,b,c$ thỏa mãn đề bài. Dựa theo tính chất đã biết, ta có
\[(a-b)\mid f(a)-f(b)=b-c,\tag{1}\]
\[(b-c)\mid f(b)-f(c)=c-a,\tag{2}\]
\[(c-a)\mid f(c)-f(a)=a-b.\tag{3}\]
Do $a,b,c$ là các số nguyên phân biệt nên
\begin{align*}
    &\text{(1)}\Rightarrow |a-b|\le |b-c|, 
    \\&\text{(2)}\Rightarrow |b-c|\le |c-a|,
    \\&\text{(3)}\Rightarrow |c-a|\le |a-b|.
\end{align*}
Kết hợp các suy luận kể trên, ta được
$$|a-b|\le |b-c|\le |c-a|\le |a-b|.$$
Đẳng thức bắt buộc phải xảy ra, tức là
\[|a-b|=|b-c|=|c-a|.\tag{4}\]
Không mất tổng quát, ta giả sử $a = \max\{ a;b;c\}$. Khi đó ta viết lại (4) như sau
$$a-b=b-c=a-c.$$
Ta suy ra được $b=c,$ mâu thuẫn với giả sử. Giả sử phản chứng là sai. Bài toán được chứng minh.
}
\end{gbtt}

\begin{gbtt}
Cho đa thức $P(x)$ với hệ số nguyên thỏa mãn $P(0)P(1)P(4)P(7)P(8)=19.$ Chứng minh rằng đa thức này không có nghiệm nguyên.
\loigiai{
Giả sử phản chứng $a$ là nghiệm nguyên của $P(x),$ thế thì 
$$P(i)=(i-a)Q(i),$$
trong đó $i=0,1,4,7,8.$ Trong các số $P(0),P(1),P(4),P(7),P(8),$ rõ ràng có $4$ số có trị tuyệt đối bằng $1.$ Ngoài ra, với mọi cách chọn $4$ số từ $5$ số $0,1,4,7,8,$ ta luôn thu được $2$ số khác tính chẵn lẻ trong $4$ số ấy, gọi là $b$ và $c.$ Ta dễ dàng nhận thấy
$$P(b)=(b-a)P(b),\: P(c)=(c-a)P(c).$$
Lấy tích, ta được $P(b)P(c)=(b-a)(c-a)P(c).$ Vế trái có trị tuyệt đối là $1,$ nhưng vế phải chia hết cho $(b-a)(c-a)$ là số chẵn. Giả sử sai. Chứng minh hoàn tất.}
\end{gbtt}

\begin{gbtt}
Cho đa thức $P(x)$ có hệ số nguyên. Giả sử tồn tại các số nguyên $a,b,c,d$ thỏa mãn đồng thời các điều kiện
\begin{enumerate}[i,]
    \item $a<b,c<d,a<c.$
    \item$P(a)=P(b)=1,P(c)=P(d)=-1.$
\end{enumerate}
Chứng minh $a,b,c,d$ là bốn số nguyên liên tiếp.

\loigiai{Xét đa thức $P(x)+1.$ Đa thức này nhận $c$ và $d$ làm nghiệm, thế nên theo định lí $Bezout,$ tồn tại đa thức $Q(x)$ với hệ số nguyên sao cho
\[P(x)+1=(x-c)(x-d)Q(x).\tag{1}\]
Do $P(a)=1$ nên $2=(a-c)(a-d)Q(a).$ Vì $a-c>a-d$ và $a-c<0,$ ta chỉ ra
\[\heva{&a-c=-1 \\ &a-d=-2}\Rightarrow\heva{&a+1=c \\ &a+2=d.}\tag{2}\]
Cũng do $P(b)=1$ nên từ (1) ta có
\[(b-c)(b-d)Q(b)=2.\tag{3}\]
Do $b>a,$ ta xét các trường hợp sau
\begin{enumerate}
    \item Với $b=a+1,$ từ (2) ta có $b=c,$ mâu thuẫn với (3).
    \item Với $b=a+2,$ từ (2) ta có $b=d,$ mâu thuẫn với (3).    
    \item Với $b\ge a+3,$ ta có $b-c>b-d>0.$ Kết hợp với (3), ta được
    \[\heva{&b-c=2 \\ &b-d=1}\Rightarrow\heva{&b=c+2 \\ &b=d+1.}\tag{4}\]
\end{enumerate}
Đối chiếu (2) và (4), ta được
$b=d+1=c+2=a+3.$ Chứng minh hoàn tất.}
\end{gbtt}

\begin{gbtt}
Cho $p_{1}, p_{2}, p_{3}, p_{4}$ là bốn số nguyên tố phân biệt. Chứng minh rằng không tồn tại đa thức $Q(x)$ bậc ba hệ số nguyên thỏa mãn $$\left|Q\left(p_{1}\right)\right|=\left|Q\left(p_{2}\right)\right|=\left|Q\left(p_{3}\right)\right|=\left|Q\left(p_{4}\right)\right|=3.$$
\nguon{Cao Đình Huy}
\loigiai{
Giả sử tồn tại đa thức $Q(x)$ thỏa mãn. Ta xét các trường hợp sau.
\begin{enumerate}
    \item Nếu $Q\left(p_{1}\right)= Q\left(p_{2}\right)= Q\left(p_{3}\right)= Q\left(p_{4}\right),$ không mất tính tổng quát, ta giả sử $$Q\left(p_{1}\right)=Q\left(p_{2}\right)=Q\left(p_{3}\right)=Q\left(p_{4}\right)=3.$$
    Xét đa thức $P(x)=Q(x)-3.$ Ta có $$P\left(p_{1}\right)=P\left(p_{2}\right)=P\left(p_{3}\right)=P\left(p_{4}\right)=0.$$ 
    Đa thức bậc ba $P(x)$ lúc này có $4$ nghiệm là $p_1,\ p_2,\ p_3,\ p_4$, thế nên 
    $$P(x)=\tron{x-p_1}\tron{x-p_2}\tron{x-p_3}\tron{x-p_4}R(x),$$
    trong đó $R(x)$ là đa thức hệ số nguyên khác không. Ta suy ra $\deg P\ge 4$ từ đây, kéo theo $\deg Q\ge 4,$ trái giả thiết $\deg Q=3.$
     \item Nếu trong các số $Q\tron{p_1},\ Q\tron{p_2},\ Q\tron{p_3}$ và $Q\tron{p_4}$ có một số khác các số còn lại, không mất tổng quát, ta giả sử $Q\left(p_{1}\right)=Q\left(p_{2}\right)=Q\left(p_{3}\right)=-3, Q\left(p_{4}\right)=3$ và hệ số bậc ba của $Q(x)$ dương. Gọi $a$ là hệ số bậc ba của $P(x)$ $\big($đồng thời cũng là của $Q(x)\big)$, khi đó ta có
    $$Q(x)=a\left(x-p_{1}\right)\left(x-p_{2}\right)\left(x-p_{3}\right)-3.$$
    Do $Q(p_4)=3$ nên là
    $a\left(p_{4}-p_{1}\right)\left(p_{4}-p_{2}\right)\left(p_{4}-p_{3}\right)=6.$ 
    Lấy trị tuyệt đối hai vế, ta được
      \[\left|a\left(p_{4}-p_{1}\right)\left(p_{4}-p_{2}\right)\left(p_{4}-p_{3}\right)\right|=6.\tag{*}\]
     Tới đây, ta xét các trường hợp sau.
    \begin{itemize}
        \item\chu{Trường hợp 1.} Nếu $p_{4}$ lẻ thì do $p_{1}, p_{2}, p_{3}, p_{4}$ phân biệt nên tồn tại ít nhất hai số nguyên tố lẻ trong ba số $p_{1}, p_{2}, p_{3}$, ta có thể giả sử $p_{2}, p_{3}$ lẻ. Khi đó $$6=a\left(p_{4}-p_{1}\right)\left(p_{4}-p_{2}\right)\left(p_{4}-p_{3}\right)$$
        chia hết cho $\left(p_{4}-p_{2}\right)\left(p_{4}-p_{3}\right),$ vô lí vì $\left(p_{4}-p_{2}\right)\left(p_{4}-p_{3}\right)$ chia hết cho $4.$
        điều này là vô lí.
        \item\chu{Trường hợp 2.} Nếu $p_{4}$ chẵn, $p_{4}=2$. Lúc này, do $p_4$ nhỏ hơn các số nguyên tố còn lại nên
        \begin{align*}
        \left|a\left(p_{4}-p_{1}\right)\left(p_{4}-p_{2}\right)\left(p_{4}-p_{3}\right)\right|
        &=\left|a\right||\left(p_{4}-p_{1}\right)\left(p_{4}-p_{2}\right)\left(p_{4}-p_{3}\right)|       
        \\&=\left|a\right|\left(p_{1}-2\right)\left(p_{2}-2\right)\left(p_{3}-2\right) \\&\geq|a|(3-2)(5-2)(7-2)
        \\&\ge 15|a|\\&\ge 15\\&>6, \text{ mâu thuẫn với (*).}
        \end{align*}
    \end{itemize}
    \item Nếu trong các số $Q\tron{p_1},\ Q\tron{p_2},\ Q\tron{p_3}$ và $Q\tron{p_4}$ có hai số bằng $3$ và hai số bằng $-3,$ không mất tổng quát, ta giả sử $Q\left(p_{1}\right)=Q\left(p_{2}\right)=-3, Q\left(p_{3}\right)=Q\left(p_{4}\right)=3.$ Trong trường hợp này, đa thức $Q(x)+3$ có hai nghiệm là $p_1,p_2.$ Như vậy, tồn tại đa thức $G(x)$ bậc nhất với hệ số nguyên sao cho $$Q(x)=\left(x-p_{1}\right)\left(x-p_{2}\right)G(x)-3.$$
    Bằng cách đặt như vậy, ta có
    $$\left(p_{3}-p_{1}\right)\left(p_{3}-p_{2}\right) G\left(p_{3}\right)=\left(p_{4}-p_{1}\right)\left(p_{4}-p_{2}\right) G\left(p_{4}\right)=6.$$
     Tới đây, ta xét các trường hợp sau.    
    \begin{itemize}
        \item\chu{Trường hợp 1.} Nếu $p_1,p_2,p_3,p_4$ đều lẻ thì 
        $$6=\left(p_{3}-p_{1}\right)\left(p_{3}-p_{2}\right) G\left(p_{3}\right),$$ 
        chia hết cho $\left(p_{3}-p_{1}\right)\left(p_{3}-p_{2}\right)$ vô lí vì $\left(p_{3}-p_{1}\right)\left(p_{3}-p_{2}\right)$ chia hết cho $4.$
        \item\chu{Trường hợp 2.} Nếu $p_1=2$ hoặc $p_2=2,$ không mất tổng quát, ta giả sử $p_1=2.$ Ta có
        $$p_{3}-2 \mid Q\left(p_{3}\right)-Q(2)= 6.$$
        Chú ý rằng $p_3$ lẻ, ta có $p_{3} \in\{3,5\}.$ Một cách tương tư, ta được $p_{4} \in\{3,5\}$. Do $p_{3}, p_{4}$ phân biệt, không mất tính tổng quát, ta giả sử $p_{3}=3, p_{4}=5.$ Giả sử này cho ta
        $$\left(p_{4}-p_{1}\right)\left(p_{4}-p_{2}\right)\left|6 \Rightarrow\left(5-p_{2}\right)\right| 2 \Rightarrow p_{2} \in\{7,3\}.$$
        Lại do $p_2\ne p_3,$ ta suy ra $p_2=7$, khi đó
        $6$ chia hết cho $\left(p_{3}-p_{1}\right)\left(p_{3}-p_{2}\right)=-4,$ vô lí.
        \item\chu{Trường hợp 3.} Nếu $p_3=2$ hoặc $p_4=2,$ lập luận tương tự, ta chỉ ra điều vô lí. 
    \end{itemize}
\end{enumerate}
Các mâu thuẫn chỉ ra chứng tỏ giả sử phản chứng là sai. Chứng minh hoàn tất!}
\end{gbtt}    

\begin{gbtt}
Tìm tất cả các cặp số nguyên $(a,b)$ để đa thức hệ số nguyên $P(x)=x^2-ax+b$ thỏa mãn tính chất: \textit{Tồn tại ba số nguyên đôi một khác nhau $m$, $n$, $p$ thuộc đoạn $[1;9]$ sao cho}
	$$|P(m)|=|P(n)|=|P(p)|=7.$$
\nguon{Tạp chí Pi tháng 11 năm 2017}
\loigiai{
Giả sử $(a,b)$ là cặp số nguyên thỏa mãn yêu cầu đề bài.\\ Khi đó, tồn tại ba số nguyên đôi một khác nhau $m$, $n$, $p\in[1;9]$ sao cho 
\[P(m),\,P(n),\,P(p)\in\{-7;7\}.\]
Dễ dàng thấy rằng các số $P(m),P(n),P(p)$ không nhận cùng một giá trị. Chỉ có hai trường hợp xảy ra là $\tron{P(m),P(n),P(p)}$ là hoán vị của $(7,7,-7)$ hoặc $(7,-7,-7).$ Không mất tổng quát, ta giả sử $m<n<p.$ Tổng cộng có $6$ trường hợp cần xét, bao gồm
\begin{multicols}{2}
\begin{enumerate}
    \item $P(m)=P(n)=7$ và $P(p)=-7.$
    \item $P(n)=P(p)=7$ và $P(m)=-7.$ 
    \item $P(p)=P(m)=7$ và $P(n)=-7.$ 
    \item $P(m)=P(n)=-7$ và $P(p)=7.$
    \item $P(n)=P(p)=-7$ và $P(m)=7.$ 
    \item $P(p)=P(m)=-7$ và $P(n)=7.$
\end{enumerate}
\end{multicols}
Ta sẽ xét đại diện một trường hợp, đó là
$$P(m)=P(n)=-7\text{ và }P(p)=7.$$ 
Lúc này, theo định lí $Bezout,$ ta dễ dàng chỉ ra đa thức $P(x)$ có dạng
$$P(x)=(x-m)(x-n)-7.$$
Do $P(p)=7$ nên $7=(p-m)(p-n)-7,$ hay là
$$(p-n)(p-m)=14.$$
Vì $1\le p-n<p-m\le 8,$ ta suy ra $p-n=2,p-m=7,$ và thế thì
$$(m,n,p)\in\{(1,6,8);(2,7,9)\}.$$
Thay trở lại, trường hợp này cho ta $(a,b)=(7,-1)$ và $(a,b)=(9,7).$ \\
Bạn đọc tự xét các trường hợp khác. Tất cả các cặp $(a,b)$ thỏa yêu cầu bài toán là
$$(11,7),(13,29),(7,-1),(9,7).$$}
\end{gbtt}

\begin{gbtt}
Cho $n$ số nguyên phân biệt $a_{1}, a_{2}, \ldots, a_{n}$. Chứng minh rằng đa thức
$$P(x)=\left(x-a_{1}\right)\left(x-a_{2}\right) \ldots\left(x-a_{n}\right)-1.$$
không thể phân tích được thành tích của hai đa thức hệ số nguyên khác hằng.
\loigiai{
Giả sử tồn tại hai đa thức $Q(x), R(x)$ hệ số nguyên khác hằng sao cho $P(x)=Q(x) R(x).$ \\Lần lượt thay $x=a_1,a_2,\ldots,a_n,$ ta có
$$Q\left(a_{1}\right) R\left(a_{1}\right)=Q\left(a_{2}\right) R\left(a_{2}\right)=\ldots=Q\left(a_{n}\right) R\left(a_{n}\right)=-1.$$
Trong hai số dạng $Q\tron{a_i}$ và $R\tron{a_i},$ bao giờ cũng có một số bằng $1$ và một số bằng $-1.$ Vì lẽ đó 
$$Q\left(a_{1}\right)+R\left(a_{1}\right)=Q\left(a_{2}\right)+R\left(a_{2}\right)=\ldots=Q\left(a_{n}\right)+R\left(a_{n}\right)=0.$$ 
Đa thức $Q(x)+R(x)$ lúc này có $n$ nghiệm nguyên phân biệt. Theo định lí $Bezout,$ ta có thể viết
$$Q(x)+R(x)=\tron{x-a_1}\tron{x-a_2}\ldots\tron{x-a_n}H(x),$$
trong đó $H(x)$ là một đa thức hệ số nguyên khác đa thức không. Bậc của $Q(x)+R(x)$ phải nhỏ hơn $n,$ vì 
$$\deg Q +\deg R=n\text{ và }\deg Q\ge 1,\deg R\ge 1.$$ Trong khi đó, bậc của vế phải lớn hơn $n$ vì nó chia hết cho đa thức $\tron{x-a_1}\tron{x-a_2}\ldots\tron{x-a_n}.$ Sự chênh lệch bậc này dẫn đến mâu thuẫn. Giả sử ban đầu là sai. Bài toán được chứng minh.}
\begin{luuy}
Ta cũng tìm ra được một bổ đề quen thuộc liên hệ giữa bậc và số nghiệm của đa thức trong bài toán trên.
\begin{quote}
\it
    Đa thức bậc $n$ có tối đa $n$ nghiệm thực.
\end{quote}
\end{luuy}
\end{gbtt}

\begin{gbtt}
Cho $k\ge 6$ số nguyên dương $x_1<x_2<\ldots<x_k.$ Giả sử tồn tại đa thức $P(x)$ với hệ số nguyên thỏa mãn $P\left(x_1\right),P\left(x_2\right),\ldots,P\left(x_k\right)$ nhận giá trị trong đoạn $[1;k-1].$
\begin{enumerate}[a,]
    \item Chứng minh rằng $P\left(x_1\right)=P\left(x_k\right).$
    \item Chứng minh rằng $P\left(x_1\right)=P\left(x_2\right)=\ldots=P\left(x_k\right).$
\end{enumerate}
\nguon{Moscow Mathematical Olympiad 2008}
\loigiai{
\begin{enumerate}[a,]
    \item Theo như tính chất đã biết, ta có
    $$\left(x_{k}-x_{1}\right) \mid\left(P\left(x_{k}\right)-P\left(x_{1}\right)\right).$$
    Với các chú ý $2-k\le P\left(x_{k}\right)-P\left(x_{1}\right) \leq k-2$ và $x_k-x_1\ge k-1,$ ta thu được $P\left(x_{k}\right)=P\left(x_{1}\right).$
    \item Ta có thể viết lại đa thức $P(x)$ dưới dạng
    $$P(x)=P\left(x_{1}\right)+\left(x-x_{1}\right)\left(x-x_{k}\right) Q(x),$$
    trong đó $Q(x)$ cũng là một đa thức hệ số nguyên. \\
    Nếu tồn tại số nguyên $i\in \{3;4;\ldots;k-2\}$ để cho $P\left(x_{i}\right) \neq P\left(x_{1}\right),$ ta có $\left|Q\left(x_i\right)\right|\ne 0,$ và thế thì
    $$\left|P\left(x_{i}\right)-P\left(x_{1}\right)\right| \geq\left|\left(x_{i}-x_{1}\right)\left(x_{i}-x_{k}\right)\right| \geq 2(k-2)>k-2,$$
    một điều mâu thuẫn, chứng tỏ $P\left(x_{i}\right)=P\left(x_{1}\right)$ với $i=3, \ldots, k-2.$ Ta tiếp tục viết $P(x)$ dưới dạng
    $$P(x)=P\left(x_{1}\right)+\left(x-x_{1}\right)\left(x-x_{3}\right)  \cdots \left(x-x_{k-2}\right)\left(x-x_{k}\right) R(x),$$
    trong đó $R(x)$ cũng là một đa thức hệ số nguyên. \\
    Tiếp theo, nếu như $P\left(x_{1}\right) \neq P\left(x_{t}\right)$ với $t=2$ hoặc $t=k-1,$ ta có $R\left(x_{t}\right) \neq 0$, và thế thì
    $$\left|P\left(x_{t}\right)-P\left(x_{1}\right)\right| \geq(k-4)!\cdot (k-2)>k-2.$$
    Đây cũng là một điều mâu thuẫn. Ta suy ra $P(x_2)=P(x_1),$ kéo theo điều phải chứng minh.
\end{enumerate}}
\end{gbtt}

\begin{gbtt}
Cho $P(x)$ là một đa thức hệ số nguyên có bậc tối đa là 10. Biết rằng tồn tại các số nguyên $x_1,x_2,\ldots,x_{10}$ sao cho
$$P\left(x_1\right)=1,P\left(x_2\right)=2,\ldots,P\left(x_{10}\right)=10.$$
\begin{enumerate}[a,]
    \item Chứng minh rằng $x_1,x_2,\ldots,x_{10}$ là một dãy số nguyên cách đều nhau.
    \item Giả sử $\left|P(10)-P(0)\right|<1000.$ Chứng minh rằng với mọi số nguyên $k,$ luôn tồn tại số tự nhiên $m$ sao cho $P(m)=k.$
\end{enumerate}
\nguon{Titu Andreescu}   
\loigiai{\hfill
\begin{enumerate}[a,]
    \item Do
    $x_2-x_1$ là ước của $P(x_2)-P(x_1)=1$ nên $x_2-x_1=\pm 1.$ Chứng minh tương tự, dễ thấy $x_3-x_2=\pm 1.$ Tính phân biệt của dãy số cho ta $x_2-x_1=x_3-x_2.$ Nói chung
    $$x_{10}-x_9=x_9-x_8=\ldots=x_3-x_2=x_2-x_1=\pm 1.$$
    \item Không mất tổng quát, ta xét dãy $x_1,x_2,\ldots,x_{10}$ tăng dần. Áp dụng định lí $Bezout$ cho đa thức $$Q(x)=P(x)-1-x+x_1$$ ta được
    $P(x)=c\left(x-x_1\right)\left(x-x_2\right)\ldots\left(x-x_{10}\right)+x+1-x_1.$\\
    Nếu như $c\ne 0,$ ta có
    \begin{align*}
        P(10)-P(0) &=10+c\left[\left(10-x_{1}\right)  \cdots \left(10-x_{10}\right)-\left(0-x_{1}\right)  \cdots \left(0-x_{10}\right)\right] \\
        &=10+(N+20)(N+19)\cdots(N+11)-(N+10)  \cdots (N+1).
    \end{align*}
    Ở trong biến đổi trên, ta đặt $N=x_{1}-1$. Ta cũng chứng minh được
    $$(N+20)(N+19) \cdots (N+11) \text{ và } (N+10)(N+9)\cdots (N+1)$$
    là hai số khác nhau (cụ thể, số thứ nhất lớn hơn khi $N \geq-10,$ còn số thứ hai lớn hơn khi  $N \leq-11$). Cả hai số này đều chia hết cho $10!,$ chính vì vậy
    $$\left|(N+20)(N+19) \cdots(N+11)-(N+10)(N+9) \cdots (N+1)\right| \geq 10 !.$$
    Do đó, $|P(10)-P(0)|>10 !-10>1000$, mâu thuẫn với giả thiết. Ta thu được $c=0.$\\ Bạn đọc tự hoàn tất chứng minh.
\end{enumerate}}
\end{gbtt}

\begin{gbtt}
Cho đa thức $P(x)=x^{4}-x^{3}-3 x^{2}-x+1.$ Chứng minh rằng tồn tại vô số số nguyên dương $n$ sao cho $P\left(3^{n}\right)$ là hợp số.
\nguon{Mediterranean Competition 2015}
\loigiai{
Với $x=3^{2n-1},$ ta có
$P\left(3^{2 n-1}\right)=81^{2 n-1}-27^{2 n-1}-3.9^{2 n-1}-3^{2 n-1}+1 .$
Xét modulo $5$ hai vế, ta được
$$P\left(3^{2 n-1}\right) \equiv 1-2^{2 n-1}-3(-1)^{2 n-1}-3^{2 n-1}+1 \equiv -2^{2 n-1}-3^{2 n-1} \pmod{5}.$$
Do $2^{2 n-1}+3^{2 n-1}$ chia hết cho $5$ nên $P\tron{3^{2n-1}}$ là hợp số. Bài toán được chứng minh.}
\end{gbtt}

\begin{gbtt}
Tìm tất cả các số thực $a,b,c$ sao cho đa thức bậc hai $P(x)=ax^2+bx+c$ nhận giá trị nguyên với mọi giá trị nguyên của $x.$
\loigiai{
\begin{enumerate}
    \item Giả sử $P(x)$ nhận giá trị nguyên với mọi số nguyên $x.$ Do $P(1),P(0)$ và $P(-1)$ nguyên, ta xây dựng được hệ điều kiện dưới đây
    $$\heva{&c\in \mathbb{Z} \\&a+b+c\in \mathbb{Z} \\ &a-b+c\in \mathbb{Z}}\Rightarrow \heva{&c\in \mathbb{Z} \\&a+b\in \mathbb{Z} \\ &a-b\in \mathbb{Z}}\Rightarrow \heva{&c\in \mathbb{Z} \\&2a\in \mathbb{Z} \\ &a+b\in \mathbb{Z}} \Rightarrow \heva{
    c \in \mathbb{Z}\\
    2a \in \mathbb{Z}\\
    2b \in \mathbb{Z}.
    }$$
    \item Đảo lại, với $a,b,c$ như trên, ta viết lại $P(x)$ dưới dạng
    $$P(x)=a\left(x^2-x\right)+(a+b)x+c.$$
    Do cả $x^2-x,a+b$ và $c$ đều chia hết cho $2,$ điều kiện đủ được chứng minh.
\end{enumerate}
 Như vậy, tất cả các giá trị của $a,b,c$ thỏa mãn là $2a,2b$ và $c$ nguyên.}
\end{gbtt}

\begin{gbtt}
Cho đa thức $P(x)$ bậc bốn với hệ số nguyên. Chứng minh rằng $P(x)$ chia hết cho $7$ với mọi số nguyên $x$ khi và chỉ khi tất cả các hệ số của $P(x)$ đều chia hết cho $7.$

\loigiai{
Ta đặt $P(x)=ax^4+bx^3+cx^2+dx+e.$ Lần lượt tính $P(0),\ P(1),\ P(-1),$ ta có các số
$$e,\ a+b+c+d+e,\ a-b+c-d+e$$
đều chia hết cho $7.$ Suy ra $a+c$ và $b+d$ chia hết cho $7.$ Ngoài ra, ta còn có
$$P(2)+P(-2)=8(a+c)+2e+24a$$
cũng chia hết cho $7,$ thế nên $a$ chia hết cho $7.$ Kết hợp với $a+c$ chia hết cho $7,$ ta có $c$ chia hết cho $7.$ Tới đây, xét nốt điều kiện $P(3)$ chia hết cho $7$ rồi sử dụng các kết quả $a,c,e,b+d$ chia hết cho $7$ vừa tìm được để hoàn tất chứng minh.}
\end{gbtt}

\begin{gbtt}
Tìm tất cả các số tự nhiên $n$ sao cho đa thức 
$$P(x)=x^{3n+7}+2x^{2n}+x^5-x^4+1$$
chia hết cho đa thức $x^5+x^4+x+1.$
\loigiai{
Trước hết, ta sẽ tìm dư của $P(x)$ trong phép chia cho đa thức
$$x^8-1=(x-1)(x+1)\tron{x^2+1}\tron{x^4+1}=(x-1)\tron{x^2+1}\tron{x^5+x^4+x+1}.$$
Với việc đặt $n=8q+r,$ trong đó $r$ đóng vai trò như số dư của $n$ khi chia cho $8,$ ta có
$$P(x)=\tron{x^8-1}Q(x)+\tron{x^{3r+7}+2x^{2r}+x^5-x^4+1}.$$
Tiếp theo, ta xét các trường hợp của $r$ để nhận xét khi nào $x^{3r+7}+2x^{2r}+x^5-x^4+1$ chia hết cho $$x^5+x^4+x+1.$$ Đáp số bài toán là tất cả các số tự nhiên $n$ thỏa mãn $n$ chia $8$ dư $6.$}
\end{gbtt}

\begin{gbtt} \label{caohuy123}
Tìm tất cả các số nguyên dương $a,b$ sao cho đa thức $P(x)=x^a+x^b+1$ chia hết cho đa thức $Q(x)=x^2+x+1.$
\loigiai{Ta đã biết
$x^3-1=(x-1)\left(x^2+x+1\right).$ Trong bài toán này, ta sẽ xét số dư khi chia cho $3$ của $a$ và $b.$ \\
Ta đặt $a=3k+r,b=3l+s,$ với $k,l,r,s$ là các số tự nhiên, đồng thời $0\le r\le s\le 2.$ Phép đặt này cho ta
\begin{align*}
    P(x)&=x^{3k+r}+x^{3l+s}+1\\&=x^{3k}x^r+x^{3l}x^s+1\\&=\left(x^{3k}-1\right)x^r+\left(x^{3l}-1\right)x^s+x^r+x^s+1.
\end{align*}
Giả sử tồn tại đa thức $P(x)$ thỏa mãn đề bài. Nhờ vào các nhận xét
$$\tron{x^2+x+1}\mid \tron{x^3-1},\quad \tron{x^3-1}\mid\tron{ x^{3l}-1},\quad \tron{x^3-1}\mid \tron{x^{3k}-1}.$$
ta suy ra $x^r+x^s+1$ chia hết cho $x^2+x+1.$ \\
Với việc $0\le r\le s\le 2,$ ta bắt buộc phải có $r=1,s=2.$ Như vậy, bài toán trên có hai kết quả
\begin{align*}
    &a\equiv 1\pmod{3},\quad b\equiv 2\pmod{3} ;
    \\&a\equiv 2\pmod{3},\quad b\equiv 1 \pmod{3}.   
\end{align*}}
\end{gbtt}

\begin{gbtt}
Với $n$ là một số nguyên dương, xét đa thức $P_n(x)=(x-1)^n+(x+2)^n.$ Xác định $n$ để $P_n(x)$ chia hết cho $2x^2+2x+5.$
\nguon{Chọn đội tuyển chuyên Trần Phú $-$ Hải Phòng 2021}
\loigiai{
Giả sử tồn tại số nguyên dương $n$ thỏa yêu cầu bài toán. Do $P_n(2)=1+4^n$ chia hết cho $17$ nên ta sẽ đi tìm số dư của phép chia $4^n+1$ cho $17.$ Cụ thể
\begin{enumerate}
    \item Nếu $n=4k$ thì $1+4^n=1+4^{4k}=1+256^k\equiv 1+1\equiv 2\pmod{17}.$
    \item Nếu $n=4k+1$ thì $1+4^n=1+4^{4k+1}=1+4\cdot256^k\equiv 1+4\equiv 5\pmod{17}.$    
    \item Nếu $n=4k+2$ thì $1+4^n=1+4^{4k+2}=1+16\cdot256^k\equiv 1+16\equiv 0\pmod{17}.$    
    \item Nếu $n=4k+3$ thì $1+4^n=1+4^{4k+3}=1+64\cdot256^k\equiv 1+64\equiv 14\pmod{17}.$    
\end{enumerate}
Thử lại, ta kết luận các số nguyên dương $n$ thỏa yêu cầu bài toán là $n$ chia $4$ dư $2.$}
\end{gbtt}

\begin{gbtt}
Cho $P(x)$ và $Q(x)$ là hai đa thức với hệ số nguyên thỏa mãn đa thức $P\tron{x^3}+xQ\tron{x^3}$ chia hết cho đa thức $x^2+x+1.$ Chứng minh rằng $P(2021)-Q(2021)$ chia hết cho $2020.$
\loigiai{Giống với các bài trước, ta sẽ dư trong phép chia đa thức $P\tron{x^3}+xQ\tron{x^3}$ cho đa thức $x^2+x+1.$ Ta có
\begin{align*}
    P\tron{x^3}+xQ\tron{x^3}
    =P\tron{x^3}-P(1)+x\left[Q\tron{x^3}-Q(1)\right]+xQ(1)+P(1).
\end{align*}
Tương tự \chu{ví dụ \ref{caohuy123}}, ta suy ra đa thức $xQ(1)+P(1)$ chia hết cho đa thức $x^2+x+1.$ Tuy nhiên, do $$\deg\left(xQ(1)+P(1)\right)\le \deg\left(x^2+x+1\right),$$ 
ta có $Q(1)=P(1)=0.$ Theo định lí $Bezout$, cả $P(x)$ và $Q(x)$ đều chia hết cho đa thức $x-1.$ Ta suy ra
$$\heva{
2020\mid P\left ( 2021 \right )\\ 
2020\mid Q\left ( 2021 \right )}\Rightarrow 2020\mid \left [ P\left ( 2021 \right )-Q\left ( 2021 \right ) \right ].$$
Đây chính là điều phải chứng minh.}
\end{gbtt}

\section{Phép đồng nhất hệ số trong đa thức}

\subsection*{Bài tập tự luyện}

\begin{btt}
Tìm tất cả các đa thức bậc hai $P(x)=a x^{2}+b x+c$ với hệ số nguyên thỏa mãn 
$$\heva{&P(1)<P(2)<P(3) \\ &P^2(1)+P^2(2)+P^2(3)=22.}$$
\nguon{Titu Andreescu}
\end{btt}

\begin{btt}
Xác định tất cả các số nguyên dương $n$ sao cho tồn tại đa thức $P(x)$ hệ số nguyên thỏa mãn
\[\deg P\le 3,\quad P(0)=5,\quad P(n)=11,\quad P(3n)=41.\]
\end{btt}

\begin{btt}
Chứng minh rằng không tồn tại các đa thức $P(x),Q(x)$ có bậc lớn hơn một với hệ số nguyên thỏa mãn $P(x)Q(x)=x^5+2x+1.$
\nguon{India National Olympiad 1999}
\end{btt}

\begin{btt}
Chứng minh rằng đa thức $P(x)=x^4+2x^3+2x^2+2$ không thể phân tích thành tích thành hai đa thức hệ số nguyên và có bậc lớn hơn hoặc bằng một.
\end{btt}

\begin{btt}
Tìm tất cả các đa thức $P(x)$ bậc $n\ge 1$ với hệ số nguyên thỏa mãn đồng thời hai điều kiện
\begin{enumerate}[i,]
    \item Các hệ số của $P(x)$ là hoán vị của bộ $\left(0,1,2,\ldots,n\right).$
    \item Đa thức $P(x)$ có $n$ nghiệm hữu tỉ.
\end{enumerate}
\end{btt}

\subsection*{Hướng dẫn bài tập tự luyện}

\begin{gbtt}
Tìm tất cả các đa thức bậc hai $P(x)=a x^{2}+b x+c$ với hệ số nguyên thỏa mãn 
$$\heva{&P(1)<P(2)<P(3) \\ &P^2(1)+P^2(2)+P^2(3)=22.}$$
\nguon{Titu Andreescu}
\loigiai{
Rõ ràng $P(1), P(2), P(3)$ là các số nguyên. Do tổng bình phương của chúng là $22=4+9+9,$ và điều kiện $P(1)<P(2)<P(3),$ ta suy ra 
$$P(1)=-3, P(2) \in\{-2;2\}, P(3)=3.$$
Lập luận trên hướng ta đến việc giải hai hệ phương trình
$$
\heva{a+b+c &=-3 \\
4 a+2 b+c &=-2 \\
9 a+3 b+c &=3},\quad 
\heva{a+b+c &=-3 \\
4 a+2 b+c &=2 \\
9 a+3 b+c &=3}.
$$
Hai hệ này lần lượt cho ta $(a,b,c)=(2,-5,0)$ và $(a,b,c)=(-2,11,-12).$ \\
Kết quả, tất cả các đa thức $P(x)$ thỏa mãn đề bài là $P(x)=2 x^{2}-5 x$ và $P(x)=-2 x^{2}+11 x-12.$}
\end{gbtt}

\begin{gbtt}
Xác định tất cả các số nguyên dương $n$ sao cho tồn tại đa thức $P(x)$ hệ số nguyên thỏa mãn
\[\deg P\le 3,\quad P(0)=5,\quad P(n)=11,\quad P(3n)=41.\]
\loigiai{
Áp dụng tính chất $P(a)-P(b)$ chia hết cho $a-b$ với mọi $a,b$ nguyên phân biệt, ta có $6$ chia hết cho $n$ và $30$ chia hết cho $2n.$ Ta tìm ra $n=1$ và $n=3.$ \begin{enumerate}
    \item Với $n=1,$ một trong các đa thức thỏa mãn đề bài là $P(x)=3x^2+3x+5.$
    \item Với $n=3,$ ta đặt $P(x)=ax^3+bx^2+cx+d,$ với $a$ không nhất thiết khác $0.$ Khi đó
    \begin{align*}
    \heva{P(0)&=5\\P(3)&=11\\P(9)&=11}
    \Rightarrow \heva{d&=5\\27a+9b+3c+d&=11\\729a+81b+9c+d&=41}
    \Rightarrow \heva{d&=5\\9a+3b+c&=2\\81a+9b+c&=4}
    \Rightarrow 72a+6b=2.
    \end{align*}
    Vế trái chia hết cho $3,$ trong khi $2$ không chia hết cho $3,$ mâu thuẫn.
\end{enumerate}
Như vậy $n=1$ là giá trị duy nhất của $n$ thỏa yêu cầu.}
\end{gbtt}

\begin{gbtt}
Chứng minh rằng không tồn tại các đa thức $P(x),Q(x)$ có bậc lớn hơn một với hệ số nguyên thỏa mãn $P(x)Q(x)=x^5+2x+1.$
\nguon{India National Olympiad 1999}
\loigiai{
Giả sử tồn tại các đa thức $P(x),Q(x)$ thỏa yêu cầu bài toán và $\deg P\ge \deg Q.$ Do giả thiết $\deg Q\ge 1$ nên 
$$\deg P=3,\quad \deg Q=2.$$
Ta đặt $P(x)=x^3+ax^2+bx+c,Q(x)=x^2+dx+e.$ Với mọi số thực $x,$ ta có
$$\tron{x^3+ax^2+bx+c}\tron{x^2+dx+e}=x^5+2x+1.$$
Ta sẽ thực hiện đồng nhất hệ số lần lượt.
\begin{enumerate}
    \item Đồng nhất hệ số bậc bốn, ta có $a=0.$
    \item Đồng nhất hệ số bậc ba, ta có $ad+e=0,$ nhưng vì $a=0$ nên $e=0.$
    \item Đồng nhất hệ số tự do, ta có $ce=1,$ mâu thuẫn với $e=0.$
\end{enumerate}
Giả sử ban đầu là sai. Bài toán được chứng minh.}
\end{gbtt}

\begin{gbtt}
Chứng minh rằng đa thức $P(x)=x^4+2x^3+2x^2+2$ không thể phân tích thành tích thành hai đa thức hệ số nguyên và có bậc lớn hơn hoặc bằng một.
\loigiai{
Ta xét các trường hợp sau đây.
\begin{enumerate}
    \item Nếu $P(x)=(x+a)\tron{x^3+bx^2+c+d}$ thì $ad=2,$ suy ra $a\in \{-2;-1;1;2\}.$ Tuy nhiên, các số trong tập trên không phải nghiệm của $P(x)$ nên phân tích này không thỏa.
    \item Nếu $P(x)=\tron{x^2+ax+b}\tron{x^2+cx+d},$ ta có
    $$a+c=2,ac+b+d=2,ad+bc=0,bd=2.$$
    Vì $bd=2$ nên $b+d=3$ hoặc $b+d=-3,$ thế trở lại $ac+b+d=2$ thì $ac=5$ hoặc $ac=-1.$ Không có nguyên hai số nào có tổng bằng $2,$ tích bằng $5$ hoặc $-1.$ Trường hợp này cũng không xảy ra. 
\end{enumerate}
Hoàn tất chứng minh.
}
\end{gbtt}

\begin{gbtt}
Tìm tất cả các đa thức $P(x)$ bậc $n\ge 1$ với hệ số nguyên thỏa mãn đồng thời hai điều kiện
\begin{enumerate}[i,]
    \item Các hệ số của $P(x)$ là hoán vị của bộ $\left(0,1,2,\ldots,n\right).$
    \item Đa thức $P(x)$ có $n$ nghiệm hữu tỉ.
\end{enumerate}
\loigiai{
Dựa vào điều kiện thứ hai, ta có thể đặt
\[P(x)=\left(a_nx+b_n\right)\left(a_{n-1}x+b_{n-1}\right)\ldots\left(a_1x+b_1\right).\tag{*}\]
Mặt khác, dựa vào điều kiện thứ nhất, ta chỉ ra
$$P(1)=0+1+\ldots +n=\dfrac{n(n+1)}{2}.$$
Trong (*), cho $x=1$ ta được
\[\dfrac{n(n+1)}{2}=\left(a_n+b_n\right)\left(a_{n-1}+b_{n-1}\right)\ldots\left(a_1+b_1\right).\tag{**}\]
Ta sẽ chứng minh có tối đa $1$ trong $n$ số $b_n,b_{n-1},\ldots,b_1$ bằng $0.$ Thật vậy, trong trường hợp có $2$ trong $n$ số này bằng $0,$ hệ số tự do và hệ số bậc nhất của $P(x)$ đều bằng $0,$ và lúc này vì $P(x)$ chia hết cho $x^2$ nên nó không thể có $n$ nghiệm hữu tỉ, mâu thuẫn. Nhận xét trên cho ta
$$\left(a_nx+b_n\right)\left(a_{n-1}x+b_{n-1}\right)\ldots\left(a_1x+b_1\right)\ge 2^{n-1}.$$
Kết hợp với (**), ta được
$\dfrac{n(n+1)}{2}\ge 2^{n-1}.$
Nhận thấy hàm số mũ hầu như có giá trị lớn hơn hàm đa thức, ta sẽ tìm cách chặn $n.$ Ta chứng minh rằng $$n(n+1)< 2^n, \text{ với mọi }n\ge 4.$$
Với $n=4,$ khẳng định đúng. Giả sử khẳng định đúng với $n=4,5,\ldots,k,$ ta có
    $$2^{k+1}=2\cdot 2^k\ge 2k(k+1)>(k+1)(k+2).$$
Theo nguyên lí quy nạp, khẳng định trên được chứng minh. Ta suy ra $n\le 3.$ 
\begin{enumerate}
    \item Với $n=1,$ ta có đa thức $P(x)=x$ thỏa mãn.
    \item Với $n=2,$ ta có các đa thức $P(x)=x^2+2x,\: P(x)=2x^2+x$ thỏa mãn.
    \item Với $n=3,$ ta có các đa thức    $P(x)=x^3+3x^2+2x,\: P(x)=2x^3+3x^2+x$ thỏa mãn.
\end{enumerate}}
\end{gbtt}

\section{Nghiệm của đa thức}
\subsection*{Ví dụ minh họa}

\begin{bx}
Với $m$ là tham số nguyên, chứng minh rằng đa thức $$P(x)=x^4-3x^3+(4+m)x^2-5x+m$$  không thể có hai nghiệm nguyên phân biệt.
\loigiai{
Giả sử $a$ là một nghiệm nguyên của $P(x).$ Ta có 
$$a^4-3a^3+(4+m)a^2-5a+m=0,$$
\[m\left(a^2+1\right)=-a^4+3a^3-4a^2+5a.\tag{*}\]
Từ đây, ta suy ra $-a^4+3a^3-4a^2+5a$ chia hết cho $a^2+1$. Ta biểu diễn $-a^4+3a^3-4a^2+5a$ dưới dạng
$$-a^{4}+3a^{3}-4a^{2}+5 a=-\left(a^{2}+1\right)\left(a^{2}-3 a+3\right)+2a+3.$$
Nhờ vào biểu diễn trên, ta suy ra
\begin{align*}
\left(a^2+1\right)\mid (2a+3)
&\Rightarrow \left(a^2+1\right)\mid (2a+3)(2a-3)
\\&\Rightarrow \left(a^2+1\right)\mid \left(4a^2+4-13\right)
\\&\Rightarrow\left(a^2+1\right)\mid 13.  
\end{align*}
Rõ ràng $a^2+1>0.$ Ta xét các trường hợp.
\begin{enumerate}
    \item Với $a^2+1=13,$ ta có $a=\pm 2\sqrt{3}$ là số vô tỉ.
    \item Với $a^2+1=1,$ ta có $a=0.$
\end{enumerate}
Thay $a=0$ trở lại (*), ta được $m=0.$ Như vậy
$$P(x)=x^{4}-3 x^{3}+4 x^{2}-5 x=x\left(x^{3}-3 x^{2}+4x-5\right).$$
Phản chứng, giả sử $P(x)$ có hai nghiệm nguyên phân biệt là $a=0$ và $b$.\\
Theo đó, $b$ bắt buộc phải là nghiệm nguyên (khác $0$) của $x^{3}-3 x^{2}+4 x^{2}-5,$ và thế thì
\begin{align*}
    b^3-3b^2+4b-5&=0,\\
    b\left(b^2-3b+4\right)&=5.\tag{**}
\end{align*}
Ta được $b\in\{\pm 1;\pm 5\}.$ Thử với từng trường hợp, ta không tìm ra được $b$ thỏa mãn (**). \\Kết luận, $P(x)$ không thể có hai nghiệm nguyên phân biệt.}
\end{bx}

\begin{bx}
Tìm tất cả các số nguyên $a$ sao cho đa thức $f(x)=x^3+ax+3$  có nghiệm hữu tỉ. 
\nguon{Khảo sát chất lượng trường THCS Archimedes Academy 2021}
\loigiai{
Giả sử $f(x)$ có nghiệm $x_0$ hữu tỉ. Theo đó, ta đặt $x_0=\dfrac{p}{q},$ với $(p,q)=1$ và $q>0.$ Phép đặt này cho ta
$$\left(\dfrac{p}{q}\right)^3+a\left(\dfrac{p}{q}\right)+3=0\Rightarrow p^3+apq^2+3q^3=q^3.$$
Ta nhận thấy cả $apq^2,3q^3$ và $q^3,$ đều chia hết cho $q,$ thế nên $p^3$ cũng chia hết cho $q,$ nhưng do $(p,q)=1$ nên $q=1.$ Lập luận trên cho ta $x_0$ là số nguyên. Do $x_0\ne 0$ nên 
$x_0^3+ax_0+3=0$
hay $$a=-\dfrac{x^3_0+3}{x_0}.$$
Ta có $x_0$ là ước của $3.$ Đến đây, ta lần lượt xét $x_0=-3,-1,1,3$ để chỉ ra tất cả các giá trị thỏa mãn đề bài của $a$ là $a=-8,a=-4,a=2,a=-10.$
}
\end{bx} 

\begin{luuy}
Trong bài toán trên, tác giả đã chứng minh tính chất
\begin{quote}
\it
     Nếu đa thức hệ số nguyên $P(x)$ nhận $\dfrac{p}{q}$ là một nghiệm hữu tỉ (với $(p,q)=1$) thì
\begin{itemize}
    \item Hệ số bậc cao nhất của $P(x)$ chia hết cho $p.$
    \item Hệ số tự do của $P(x)$ chia hết cho $q.$
\end{itemize}
\end{quote}
Từ nay về sau, tác giả sử dụng trực tiếp tính chất này mà không thông qua chứng minh.
\end{luuy}

\begin{bx}
Cho đa thức $P(x)$ khác hằng với hệ số nguyên thỏa mãn $P(0)=-2021.$ Hỏi $P(x)$ có tối đa bao nhiêu nghiệm nguyên phân biệt?
\loigiai{Gọi $r$ là một nghiệm nguyên của $P(x).$ Theo tính chất đã phát biểu, ta có
$$r \in\{\pm 1, \pm 43, \pm 47, \pm 2021\}.$$
Do vậy, trường hợp thu được nhiều nghiệm nhất xảy ra khi $P(x)$ chia hết cho đa thức $$(x-1)(x+1)\left(x\pm 43\right)\left(x\pm 47\right).$$ Trong trường hợp này, $P(x)$ có đúng $4$ nghiệm.}
\end{bx}

\begin{bx}
Với $a,b$ là các số hữu tỉ, xét đa thức $P(x)=x^3+ax^2+bx+6.$ Biết rằng $P(x)$ nhận $\sqrt{3}$ là nghiệm, tìm tất cả các nghiệm còn lại của $P(x).$
\loigiai{Từ giả thiết, ta có $\di P\left(\sqrt{3}\right)=0,$ tức là
$$(3a+6)+(b+3)\sqrt{3}=0.$$
Do $3a+6$ và $b+3$ là các số hữu tỉ, ta bắt buộc có $3a+6=0$ và $b+3=0.$\\
Giải ra, ta tìm được $a=-2,b=-3.$ Với $a=-2,b=-3,$ ta nhận thấy rằng
$$P(x)=x^3-2x^2-3x+6=\left(x-\sqrt{3}\right)\left(x+\sqrt{3}\right)(x-2).$$
Dựa vào đây, ta được các nghiệm còn lại của $P(x)$ là $x=2$ và $x=-\sqrt{3}.$}
\end{bx}

\begin{bx}
Chứng minh rằng không tồn tại đa thức $P(x)$ bậc hai với hệ số nguyên nhận $\sqrt[3]{7}$ làm nghiệm.
\loigiai{
Ta giả sử tồn tại đa thức $P(x)=ax^2+bx+c$ (với $a\ne 0$) nhận $\sqrt[3]{7}$ làm một nghiệm. Theo đó
\begin{align*}
    a\sqrt[3]{49}+b\sqrt[3]{7}+c=0
    &\Rightarrow \left(a\sqrt[3]{7}-b\right)\left(a\sqrt[3]{49}+b\sqrt[3]{7}+c\right)=0
    \\&\Rightarrow 7a^2-bc=\tron{b^2-ca}\sqrt[3]{7}.
\end{align*}
Do $\sqrt[3]{7}$ là số vô tỉ nên bắt buộc $7a^2-bc=0$ và $b^2-ca=0.$ Ta có hệ
$$\heva{&7a^2=bc \\ &b^2=ca}\Rightarrow \heva{&7a^3=abc \\ &b^3=abc}\Rightarrow 7a^3=b^3\Rightarrow \left(\dfrac{b}{a}\right)^3=7.$$
Điều trên là không thể xảy ra. Như vậy, giả sử phản chứng là sai, và bài toán được chứng minh.
}
\begin{luuy}
Ngoài cách nhân với $a\sqrt[3]{7}-b,$ trong bài này ta cũng có thể xử lí đẳng thức
$$a\sqrt[3]{49}+b\sqrt[3]{7}+c=0$$
bằng cách chuyển $-c$ qua vế phải rồi lấy lập phương hai vế.
\end{luuy}
\end{bx}

\begin{bx} \label{dtcan1}
Giả sử $P(x)$ là đa thức khác đa thức không có hệ số hữu tỉ nhận $\sqrt{2}+\sqrt{3}$ làm nghiệm.
\begin{enumerate}[a,]
    \item Hỏi, đa thức $P(x)$ có bậc nhỏ nhất là bao nhiêu?
    \item Tìm tất cả các đa thức $P(x)$ thỏa mãn.
\end{enumerate}
\nguon{Titu Andreescu}
\loigiai{
\begin{enumerate}[a,]
    \item Ta đặt $a=\sqrt{2}+\sqrt{3}.$ Phép đặt này cho ta
\begin{align*}
    a-\sqrt{3}=\sqrt{2}
    &\Rightarrow a^2-2a\sqrt{3}+3=2
    \\&\Rightarrow a^2+1=2a\sqrt{3}
    \\&\Rightarrow \left(a^2+1\right)^2=12a^2\\&
    \Rightarrow a^4-10a^2+1=0
\end{align*}
Biến đổi trên cho ta biết, $a$ là nghiệm của đa thức
$$Q(x)=x^4-10x^2+1.$$
Ta sẽ chứng minh $\min\left(\deg P\right)=\deg Q=4.$ Thật vậy, ta giả sử bậc của $P(x)$ nhỏ hơn $4.$ Đặt
$$P(x)=ax^3+bx^2+cx+d,$$
ở đây $a,b,c,d$ là các số hữu tỉ không đồng thời bằng $0.$ Cho $x=\sqrt{2}+\sqrt{3},$ ta được
$$2b\sqrt{6}+(9a+c)\sqrt{3}+(11a+c)\sqrt{2}+5b+d=0.$$
Lần lượt đặt $2b=A,9a+c=B,11a+c=C,5b+d=D,$ ta có
\[A\sqrt{6}+B\sqrt{3}+C\sqrt{2}+D=0.\tag{1}\]
Chuyển $B\sqrt{3}$ sang vế phải rồi bình phương, ta được
\begin{align*}
    A\sqrt{6}+C\sqrt{2}=-D-B\sqrt{3}
    &\Rightarrow 6A^2+2C^2+4AC\sqrt{3}=D^2+3B^2+2BD\sqrt{3}
    \\&\Rightarrow 2(2AC-BD)\sqrt{3}=D^2+3B^2-6A^2-2C^2.
\end{align*}
Do $\sqrt{3}$ là số vô tỉ, ta bắt buộc phải có 
\[D^2+3B^2=6A^2+2C^2.\tag{2}\]
Một cách tương tự, khi chuyển $C\sqrt{3}$ sang vế phải, ta cũng suy ra được
\[D^2+2C^2=6A^2+3B^2.\tag{3}\]
Kết hợp (2) và (3), ta nhận thấy $2C^2=3B^2.$ Do $B,C$ hữu tỉ, ta có $B=C=0.$ Thế vào (1), ta được
$$A\sqrt{6}+D=0.$$
Lại do $A,D$ đều hữu tỉ, ta suy ra $A=D=0.$ Vì $A=B=C=D=0$ nên $a=b=c=d=0.$\\ Ta được $P(x)\equiv 0,$ trái với giả thiết $P(x)$ khác đa thức không. \\
Như vậy, giả sử phản chứng là sai, và ta chứng minh được $\min\left(\deg P\right)=4.$
    \item Với việc $\deg P\ge 4,$ ta gọi thương và số dư trong phép chia đa thức $P(x)$ cho đa thức $$Q(x)=x^4-10x^2+1$$ lần lượt là $S(x)$ và $R(x),$ trong đó $\deg R\le 3.$ Ta có
    \[P(x)=Q(x)S(x)+R(x).\tag{4}\]
    Trong (4), cho $x=\sqrt{2}+\sqrt{3}$ ta được
    $$P\left(\sqrt{2}+\sqrt{3}\right)=Q\left(\sqrt{2}+\sqrt{3}\right)S\left(\sqrt{2}+\sqrt{3}\right)+R\tron{\sqrt{2}+\sqrt{3}}.$$
     Do $\sqrt{2}+\sqrt{3}$ là nghiệm của cả $P(x)$ và $Q(x)$ nên $P\left(\sqrt{2}+\sqrt{3}\right)=Q\left(\sqrt{2}+\sqrt{3}\right)=0,$ và vì thế $$R\left(\sqrt{2}+\sqrt{3}\right)=0.$$
     Theo đó, $\sqrt{2}+\sqrt{3}$ cũng là một nghiệm của $S(x).$ Theo như câu a, ta bắt buộc phải có $R(x)=0$ (vì nếu $\deg R\ge 0$ thì $\deg R\ge 4,$ trái điều kiện $\deg R\le 3$).
     Tổng kết lại, các đa thức $P(x)$ cần tìm có dạng
     $$P(x)=\left(x^4-10x^2+1\right)S(x).$$
     Trong đó, $S(x)$ là một đa thức hệ số nguyên khác đa thức không.
\end{enumerate}}
\end{bx}

\begin{luuy}
Các bài toán trên là các trường hợp riêng của \chu{bổ đề về bậc nhỏ nhất của đa thức}, đó là
\begin{enumerate}
    \item Với $a,b$ là các số nguyên thỏa mãn $\sqrt{b}$ là số vô tỉ, đa thức nhận $a+\sqrt{b}$ làm nghiệm luôn chia hết cho đa thức $(x-a)^2-b.$
    \item Với $a,b$ là các số nguyên thỏa mãn $\sqrt[3]{b}$ là số vô tỉ, đa thức nhận $a+\sqrt[3]{b}$ làm nghiệm luôn chia hết cho đa thức $(x-a)^3-b.$
    \item Với $a,b$ là các số nguyên thỏa mãn $\sqrt{a}$ và $\sqrt{b}$ là số vô tỉ, đa thức nhận $\sqrt{a}+\sqrt{b}$ làm nghiệm luôn chia hết cho đa thức $x^4-2(a+b)x^2+(a-b)^2.$
\end{enumerate}
\end{luuy}

\begin{bx}
Tìm tất cả các đa thức $P(x)$ với hệ số nguyên thỏa mãn
$$P\tron{1+\sqrt{3}}=2+\sqrt{3}, \quad P\tron{3+\sqrt{5}}=3+\sqrt{5}.$$
\nguon{Zhautykov Mathematical Olympiad 2014}
\loigiai{
Giả sử tồn tại đa thức $P(x)$ thỏa mãn đề bài. \\
Xét đa thức $Q(x)=P(x)-x.$ Rõ ràng $a=3+\sqrt{5}$ là nghiệm của $Q(x).$ Mặt khác,
$$a-3=\sqrt{5}\Rightarrow (a-3)^2=5\Rightarrow a^2-6a+4=0.$$
Theo như nhận xét ở bài trước, ta suy ra tất cả các đa thức $Q(x)$ đều có dạng
\[Q(x)=\left(x^2-6x+4\right)S(x),\tag{*}\]
ở đây, $S(x)$ là một đa thức hệ số nguyên khác đa thức không. \\
Hơn nữa, từ giả thiết ta cũng có thể suy ra $Q\tron{1+\sqrt{3}}=1.$ Trong (*), cho $x=1+\sqrt{3},$ ta được
$$1=\left(2-4\sqrt{3}\right)S\tron{1+\sqrt{3}}.$$
Ta đã biết, $S\tron{1+\sqrt{3}}$ có thể được viết dưới dạng $A+B\sqrt{3},$ trong đó $A,B$ là các số nguyên dương (tham khảo phần \chu{căn thức}). Phép đặt này cho ta
$$1=\left(2-4\sqrt{3}\right)\left(A+B\sqrt{3}\right),$$
hay là
$2A-12B-1=\left(4A-2B\right)\sqrt{3}.$
Do $\sqrt{3}$ là số vô tỉ, ta suy ra 
$$2A-12B-1=0,\:4A-2B=0.$$
Giải hệ trên, ta tìm ra $A=-\dfrac{1}{22}$ và $B=-\dfrac{1}{11},$ mâu thuẫn với điều kiện $A,B$ nguyên.\\ Như vậy, giả sử đã cho là sai, và ta không tìm được đa thức $P(x)$ nào thỏa mãn đề bài.
}
\end{bx}

\subsection*{Bài tập tự luyện}

\begin{btt}
Tìm số nguyên $m$ sao cho đa thức
\[P(x)=x^3-(m+1)x^2+2x+6-m\]
có nhiều nghiệm nguyên phân biệt nhất có thể.
\end{btt}

\begin{btt}
Tìm tất cả các số nguyên $m$ sao cho đa thức 
$$P(x)=x^3+(m+1)x^2-(2 m-1) x-\left(2m^2+m+4\right)$$
tồn tại nghiệm nguyên.
\nguon{Titu Andreescu}
\end{btt}

\begin{btt}
Tìm tất cả các số nguyên dương ${n}$ sao cho đa thức $P(x)=x^{n}+(2+x)^{n}+(2-x)^{n}$ có nghiệm hữu tỉ.
\end{btt}

\begin{btt}
Giả sử $m$ là một nghiệm hữu tỉ chung của hai đa thức
\begin{align*}
    P(x)&=a_{n} x^{n}+a_{n-1} x^{n-1}+\ldots+a_{1} x+a_{0}, \\Q(x)&=b_{n} x^{n}+b_{n-1} x^{n-1}+\ldots+b_{1}
    x+b_{0}.
\end{align*}
Biết rằng $a_{n}-b_{n}$ là một số nguyên tố và $a_{n-1}=b_{n-1}$. Chứng minh $m$ là số nguyên.
\nguon{Cao Đình Huy}
\end{btt}

\begin{btt}
Cho đa thức bậc ba $P(x)=a x^{3}+b x^{2}+c x+d$ với tất cả các hệ số đều nguyên, trong đó $ad$ là số lẻ, còn $abc$ là số chẵn. Chứng minh rằng $P(x)$ có nghiệm vô tỉ.
\nguon{Titu Andreescu}
\end{btt}

\begin{btt}
Cho số nguyên dương $n \geq 2.$ Xét đa thức
$$P(x)=x^{n}+2 x^{n-1}+3 x^{n-2}+\ldots+n x+n+1.$$ 
\begin{enumerate}[a,]
    \item Chứng minh rằng $(x-1)^{2}P(x)=x^{n+2}-(n+2) x+n+1.$
    \item Chứng minh rằng $P(x)$ không có nghiệm hữu tỉ.
\end{enumerate}
\nguon{Titu Andreescu}
\end{btt}

\begin{btt}
Với $a,b$ là các số hữu tỉ, xét đa thức $P(x)=x^{3}+a x+b.$ Giả sử $P(x)$ có nghiệm  $x=1+\sqrt{3}$. Chứng minh rằng $P(x)$ chia hết cho đa thức $x^{2}-2x-2.$
\nguon{Chọn học sinh giỏi Hà Nội 2021}
\end{btt}

\begin{btt}
Cho đa thức $P(x)=x^3+px^2+qx+1,$ với $p,q$ là các số hữu tỉ. Biết rằng $2+\sqrt{5}$ là một nghiệm của $P(x),$ hãy tìm tất cả các giá trị có thể của $p$ và $q.$
\nguon{Hanoi Open Mathematics Competition 2012}
\end{btt}

\begin{btt}
Cho số nguyên dương $a$ không chính phương. Gọi $r$ là một nghiệm thực của phương trình $x^3-2ax+1=0.$ Chứng minh rằng $r+\sqrt{a}$ là một số vô tỉ.
\nguon{China Girls Mathematical Olympiad 2014}
\end{btt}

\begin{btt}
Gọi $\alpha $ là nghiệm dương của phương trình $x^2+x=5$. Với số nguyên dương $n$ nào đó, gọi $c_0,c_1,\ldots ,c_n$ là các số nguyên không âm thỏa mãn đẳng thức $$c_0+c_1\alpha +c_2\alpha ^2+\ldots+c_n\alpha^n=2015.$$
Chứng minh rằng ${{c}_{0}}+{{c}_{1}}+{{c}_{2}}+\ldots+{{c}_{n}}\equiv 2\pmod{3}.$
\nguon{Vietnamese Team Selection Test 2015}
\end{btt}

\begin{btt}
Với các số nguyên $a, b, c$ thỏa mãn $|a|,|b|,|c|\le 10,$ xét đa thức $f(x)=x^3+ax^2+bx+c$ thỏa mãn điều kiện
$$
\left|f\tron{2+\sqrt{3}}\right|<0,0001.
$$
Chứng minh rằng $2+\sqrt{3}$ là một nghiệm của $f(x).$
\nguon{China Girls Mathematical Olympiad 2017}
\end{btt}

\begin{btt}
Tồn tại hay không đa thức $P(x)$ với hệ số nguyên thỏa mãn $$P\left(1+\sqrt[3]{2}\right)=1+\sqrt[3]{2},\quad P\left(1+\sqrt{5}\right)=2+3\sqrt{5}\:?$$ 
\nguon{Vietnam Mathematical Olympiad 2017}
\end{btt}

\begin{btt} \
\begin{enumerate}[a,]
    \item Tìm đa thức $P(x)$ khác hằng, có hệ số hữu tỉ, có bậc nhỏ nhất có thể thỏa mãn $$P\tron{\sqrt[3]{3}+\sqrt[3]{9}}=3+\sqrt[3]{3}.$$
    \item Tồn tại hay không đa thức $P(x)$ khác hằng và có hệ số nguyên thỏa mãn $$P\tron{\sqrt[3]{3}+\sqrt[3]{9}}=3+\sqrt[3]{3} \: ?$$
\end{enumerate}
\nguon{Vietnam Mathematical Olympiad 1997}
\end{btt}

\subsection*{Hướng dẫn bài tập tự luyện}

\begin{gbtt}
Tìm số nguyên $m$ sao cho đa thức
\[P(x)=x^3-(m+1)x^2+2x+6-m\]
có nhiều nghiệm nguyên phân biệt nhất có thể.
\loigiai{
Giả sử $a$ là một nghiệm nguyên của đa thức. Chuyển vế và cô lập $m,$ ta có
$$a^3-a^2+2a+6=m\tron{a^2+1}.$$
Ta nhận thấy rằng $a^3-a^2+2a+6$ chia hết cho $a^2+1.$ Phép chia hết này cho ta
$$(a,m)\in\{(-7,-8);(-2,-2);(-1,1);(0,6);(1,4);(3,3)\}.$$
Tới đây, ta xét các trường hợp sau.
\begin{enumerate}
    \item Với $m=-8,$ đa thức $P(x)=(x+7)\tron{x^2+2}$ có một nghiệm duy nhất.
    \item Với $m=-2,$ đa thức $P(x)=(x+2)\tron{x^2-x+4}$ có một nghiệm duy nhất.
    \item Với $m=1,$ đa thức $P(x)=(x+1)\tron{x^2-3x+5}$ có một nghiệm duy nhất.
    \item Với $m=3,$ đa thức $P(x)=(x-3)\tron{x^2-x-1}$ có một nghiệm nguyên và hai nghiệm vô tỉ.
    \item Với $m=4,$ đa thức $P(x)=(x-1)\tron{x^2-4x-2}$ có một nghiệm nguyên và hai nghiệm vô tỉ. 
    \item Với $m=7,$ đa thức $P(x)=x\tron{x^2-7x+2}$ có một nghiệm nguyên và hai nghiệm vô tỉ.     
\end{enumerate}
Tổng kết lại, tất cả các giá trị kể trên đều thỏa yêu cầu bài toán.}
\end{gbtt}

\begin{gbtt}
Tìm tất cả các số nguyên $m$ sao cho đa thức 
$$P(x)=x^3+(m+1)x^2-(2 m-1) x-\left(2m^2+m+4\right)$$
tồn tại nghiệm nguyên.
\nguon{Titu Andreescu}
\loigiai{
Ta có thế thực hiện phân tích $P(x)+5$ thành nhân tử. Thật vậy
$$P(x)+5=\left(x+m+1\right)\left( x^{2}-2 m+1\right).$$
Theo đó, nếu $P(x)$ có nghiệm nguyên là $a$, ta suy ra
$$\left(a+m+1, a^{2}-2 m+1\right) \in\{(1,5),(-1,-5),(5,1),(-5,-1)\}.$$
Đến đây, ta chia bài toán làm bốn trường hợp.
\begin{enumerate}
    \item Nếu $a+m+1=1$ và $a^{2}-2m+1=5,$ ta có
    $$\heva{&a=-m \\ &m^2-2m-4=0}\Leftrightarrow\heva{&a=-m \\ &(m-1)^2=5.}$$
    Hệ trên không có nghiệm nguyên dương.
    \item Nếu $a+m+1=-1$ và $a^{2}-2m+1=-5,$ ta có
    $$\heva{&a=-m-2 \\ &(m+2)^2-2m+6=0}\Leftrightarrow \heva{&a=-m-2 \\ &m^2+4m+10=0}\Leftrightarrow \heva{&a=-m-2 \\ &(m+2)^2+6=0.}$$
    Hệ trên không có nghiệm nguyên dương.    
    \item Nếu $a+m+1=5$ và $a^{2}-2m+1=1,$ ta có
    $$\heva{&a=-m+4 \\ &(m-4)^2-2m=0}
    \Leftrightarrow \heva{&a=-m+4 \\ &(m-2)(m-8)=0}
    \Leftrightarrow
    \hoac{
         a=-2,&\:m=2  \\
         a=4,&\:m=8.}$$
    \item Nếu $a+m+1=-5$ và $a^{2}-2m+1=-1,$ ta có
    $$\heva{&a=-m-6 \\ &(m+6)^2-2m+2=0}
    \Leftrightarrow \heva{&a=-m-6 \\ &m^2+10m+38=0}\Leftrightarrow \heva{&a=-m-6 \\ &(m+5)^2+13=0.}$$  
    Hệ trên không có nghiệm nguyên dương.  
\end{enumerate}
Tổng kết lại, $m=2$ và $m=8$ là tất cả các giá trị thỏa mãn đề bài.}
\end{gbtt}

\begin{gbtt}
Tìm tất cả các số nguyên dương ${n}$ sao cho đa thức $P(x)=x^{n}+(2+x)^{n}+(2-x)^{n}$ có nghiệm hữu tỉ.
\loigiai{
Một cách hiển nhiên, ta phải có $n$ lẻ. Đồng thời, hệ số cao nhất của $P(x)$ bằng $1,$ và ta suy ra tất cả các nghiệm hữu tỉ của $P(x)$ phải là nghiệm nguyên. Giả sử đa thức có nghiệm $x_0$ nguyên khác $0.$ Ta có
$$\left(x_0\right)^{n}+\left(x_0+2\right)^{n}+\left(2-x_0\right)^{n}=0.$$
Tiếp theo, lấy đồng dư hai vế theo modulo $x_0,$ ta được
$$2^{n+1}\equiv 0\pmod{x_0}.$$
Bắt buộc, $2^{n+1}$ chia hết cho $x_0.$ Ta xét các trường hợp sau.
\begin{enumerate}
    \item Với $x_0=2^k$ ta có
    $2^{kn}+\left(2+2^k\right)^{n}+\left(2-2^k\right)^{n}=0.$ Một cách tương đương, ta nhận được
    $$2^{kn}+\left(2+2^k\right)^{n}=\left(2^k-2\right)^{n},$$  
    mâu thuẫn do $2^{kn}+\left(2+2^k\right)^{n}>\left(2+2^k\right)^{n}>\left(2^k-2\right)^{n}.$
    \item Với $x_0=-2^k$, ta có
    $-2^{kn}+\left(2+2^k\right)^{n}+\left(2-2^k\right)^{n}=0.$  Một cách tương đương, ta nhận được
    $$\left(2^{k-1}+1\right)^{n}=\left(2^{k-1}-1\right)^{n}+2^{(k-1)n}.$$   Lấy đồng dư theo modulo $2^{k-1}$ ở hai vế, ta chỉ ra $2$ chia hết cho $2^{k-1},$ thế nên $k\in\{0;1;2\}.$ \\
    Thử trực tiếp, ta tìm ra $k=2,$ và khi này thì $n=1.$ Đây là giá trị duy nhất của $n$ ta cần tìm.
\end{enumerate}}
\end{gbtt}

\begin{gbtt}
Giả sử $m$ là một nghiệm hữu tỉ chung của hai đa thức
\begin{align*}
    P(x)&=a_{n} x^{n}+a_{n-1} x^{n-1}+\ldots+a_{1} x+a_{0}, \\Q(x)&=b_{n} x^{n}+b_{n-1} x^{n-1}+\ldots+b_{1}
    x+b_{0}.
\end{align*}
Biết rằng $a_{n}-b_{n}$ là một số nguyên tố và $a_{n-1}=b_{n-1}$. Chứng minh $m$ là số nguyên.
\nguon{Cao Đình Huy}
\loigiai{
Từ giả thiết, ta có thể đặt $m=\dfrac{r}{s},$ ở đây $r,s$ là các số nguyên tố cùng nhau và $s>0.$ \\
Do $m=\dfrac{r}{s}$ là một nghiệm $P(x)$ và $Q(x)$ nên ta có
\[r\mid a_0, \quad s\mid a_n.\tag{1}\]
\[r\mid b_0, \quad s\mid b_n.\tag{2}\]
Đối chiếu (1) và (2), ta được $s\mid \tron{a_n-b_n}.$ Do giả thiết $a_n-b_n$ là số nguyên tố, ta xét các trường hợp sau.
\begin{enumerate}
    \item Với $s=1,$ ta có $m=\dfrac{r}{s}=r$ là số nguyên.
    \item Với $s=a_n-b_n,$ ta xét đa thức $S(x)=P(x)-Q(x).$ Rõ ràng $m=\dfrac{r}{a_n-b_n}$ cũng là nghiệm của $S(x),$ thế nên ta có
        $$\tron{a_n-b_n}\left(\dfrac{r}{a_n-b_n}\right)^n+\left(a_{n-1}-b_{n-1}\right)\left(\dfrac{r}{a_n-b_n}\right)^{n-1}+\ldots+\left(a_0-b_0\right)=0.$$
   Với chú ý rằng $a_{n-1}=b_{n-1},$ ta suy ra
      $$\tron{a_n-b_n}\left(\dfrac{r}{a_n-b_n}\right)^n+\left(a_{n-2}-b_{n-2}\right)\left(\dfrac{r}{a_n-b_n}\right)^{n-2}+\ldots+\left(a_0-b_0\right)=0.$$
    Nhân cả hai vế với $\tron{a_n-b_n}^{n-1},$ ta được
    $$r^n+\left(a_{n-2}-b_{n-2}\right)\tron{a_n-b_n}r^{n-2}+\ldots +\left(a_0-b_0\right)\tron{a_n-b_n}^{n-1}=0.$$
    So sánh số dư khi chia hai vế cho $a_n-b_n,$ ta chỉ ra
    $$\tron{a_n-b_n}\mid r^n.$$
    Kết hợp với khẳng định $s\mid\tron{a_n-b_n}, $ ta suy ra $s\mid r,$ tức là $m\in\mathbb{Z}.$
\end{enumerate}
Tổng kết lại, bài toán được chứng minh trong mọi trường hợp.}
\end{gbtt}

\begin{gbtt}
Cho đa thức bậc ba $P(x)=a x^{3}+b x^{2}+c x+d$ với tất cả các hệ số đều nguyên, trong đó $ad$ là số lẻ, còn $abc$ là số chẵn. Chứng minh rằng $P(x)$ có nghiệm vô tỉ.
\nguon{Titu Andreescu}
\loigiai{
Giả sử phản chứng rằng tất cả các nghiệm $P(x)$ là hữu tỉ, gồm $\dfrac{p_1}{q_1},\dfrac{p_2}{q_2},\dfrac{p_3}{q_3}.$ Theo tính chất đã biết, ta có $p_{i} \mid d$ and $q_{i} \mid a,$ với $i=1,2,3.$ Do giả thiết $ad$ lẻ, các $p_{i}$ và $q_{i}$ đều lẻ. Mặt khác, theo định lí Viète cho đa thức bậc ba, ta có thể viết
\begin{align*}
-&\dfrac{b}{a}=\dfrac{p_{1}}{q_{1}}+\frac{p_{2}}{q_{2}}+\dfrac{p_{3}}{q_{3}}=\dfrac{p_{1} q_{2} q_{3}+p_{2} q_{1} q_{3}+p_{3} q_{1} q_{2}}{q_{1} q_{2} q_{3}} \\
&\frac{c}{a}=\frac{p_{1}}{q_{1}} \frac{p_{2}}{q_{2}}+\frac{p_{3}}{q_{3}} \frac{p_{2}}{q_{2}}+\frac{p_{1}}{q_{1}} \frac{p_{3}}{q_{3}}=\frac{p_{1} p_{2} q_{3}+p_{3} p_{2} q_{1}+p_{1} p_{3} q_{2}}{q_{1} q_{2} q_{3}}.
\end{align*}
Các số $p_{1} q_{2} q_{3}+p_{2} q_{1} q_{3}+p_{3} q_{1} q_{2},p_{1} p_{2} q_{3}+p_{3} p_{2} q_{1}+p_{1} p_{3} q_{2},q_{1} q_{2} q_{3}$ đều lẻ, thế nên
$$b=\dfrac{-a\left(p_{1} q_{2} q_{3}+p_{2} q_{1} q_{3}+p_{3} q_{1} q_{2}\right)}{q_1q_2q_3},\quad c=\dfrac{a\left(p_{1} p_{2} q_{3}+p_{3} p_{2} q_{1}+p_{1} p_{3} q_{2}\right)}{q_1q_2q_3}$$ 
cũng là số lẻ, điều này mâu thuẫn với giả thiết $abc$ là số chẵn. \\
Như vậy, giả sử phản chứng là sai, và ta có điều phải chứng minh.}
\end{gbtt}

\begin{gbtt}
Cho số nguyên dương $n \geq 2.$ Xét đa thức
$$P(x)=x^{n}+2 x^{n-1}+3 x^{n-2}+\ldots+n x+n+1.$$ 
\begin{enumerate}[a,]
    \item Chứng minh rằng $(x-1)^{2}P(x)=x^{n+2}-(n+2) x+n+1.$
    \item Chứng minh rằng $P(x)$ không có nghiệm hữu tỉ.
\end{enumerate}
\nguon{Titu Andreescu}
\loigiai{
\begin{enumerate}[a,]
    \item Với mọi $x\ne 1,$ ta có
    \begin{align*}
      x^{n}+2 x^{n-1}+\ldots+n x+n+1 
    &=\left(x^{n}+x^{n-1}+\ldots+1\right)+\ldots+(x+1)+1 \\
    &= \dfrac{x^{n+1}-1}{x-1}+\dfrac{x^{n}-1}{x-1}+\ldots+\dfrac{x^{2}-1}{x-1}+\dfrac{x-1}{x-1} \\
    &= \dfrac{\dfrac{x^{n+2}-1}{x-1}-n-2}{x-1} \\ &=\dfrac{x^{n+2}-(n+2) x+n+1}{(x-1)^{2}}.
    \end{align*}
    Nhân hai vế với $(x-1)^2,$ ta suy ra
    \[(x-1)^2P(x)=x^{n+2}-(n+2) x+n+1,\forall x\ne 1.\tag{*}\]
    Đối với $x=1,$ kiểm tra trực tiếp, ta thấy thỏa mãn (*). Đẳng thức được chứng minh.
    \item Chú ý rằng $P(x)$ không có nghiệm $x=1.$ Chính vì thế, toàn bộ các nghiệm của đa thức $P(x)$ cũng chính là nghiệm của đa thức
    $$Q(x)=x^{n+2}-(n+2) x+n+1.$$
    Giả sử phản chứng rằng $Q(x)$ có nghiệm hữu tỉ là $r=\dfrac{p}{q}$, trong đó $q>0$ và $(p,q)=1.$ Theo tính chất đã biết, ta chỉ ra $q \mid 1$, kéo theo $r=\dfrac{p}{q}$ cũng là số nguyên. Bằng kiểm tra trực tiếp, ta suy ra $-1,0,1$ không là nghiệm của $Q(x),$ thế nên $|r|\ge 2.$ Do $r$ là nghiệm của $Q(x),$ ta có
    $$r^{n+2}=(n+2)r-n-1.$$
    Ta lần lượt có các nhận xét
    \begin{align*}
        \left|(n+2)r-n-1\right|&\le (n+2)|r|+(n+1)\\&
        <2(n+2)|r|,
        \\\left|r^{n+2}\right|&\ge 2^{n+1}|r|.
    \end{align*}
    Dựa vào các nhận xét này, ta suy ra $2^{n+1}<n+2,$ thế nên $n<0,$ một điều vô lí. \\
    Như vậy, giả sử phản chứng là sai. 
\end{enumerate}
Bài toán được chứng minh.}
\end{gbtt}

\begin{gbtt}
Với $a,b$ là các số hữu tỉ, xét đa thức $P(x)=x^{3}+a x+b.$ Giả sử $P(x)$ có nghiệm  $x=1+\sqrt{3}$. Chứng minh rằng $P(x)$ chia hết cho đa thức $x^{2}-2x-2.$
\nguon{Chọn học sinh giỏi Hà Nội 2021}
\loigiai{
Từ giả thiết, ta có $P\tron{1+\sqrt{3}}=0,$ tức là
$$(a+b+10)+(a+6)\sqrt{3}=0.$$
Do $a+b+10$ và $a+6$ đồng thời là các số hữu tỉ, ta bắt buộc có $a+6=0$ và $a+b+10=0.$ Hơn nữa, do
$$P\left(1-\sqrt{3}\right)=(a+b+10)-(a+6)\sqrt{3}=0$$
nên $1-\sqrt{3}$ cũng là nghiệm của $P(x).$ Các kết quả trên chứng tỏ $P(x)$ chia hết cho đa thức $$\left(x-1-\sqrt{3}\right)\left(x-1+\sqrt{3}\right)=x^2-2x-2.$$ 
Toàn bộ bài toán được chứng minh.}
\end{gbtt}

\begin{gbtt}
Cho đa thức $P(x)=x^3+px^2+qx+1,$ với $p,q$ là các số hữu tỉ. Biết rằng $2+\sqrt{5}$ là một nghiệm của $P(x),$ hãy tìm tất cả các giá trị có thể của $p$ và $q.$
\nguon{Hanoi Open Mathematics Competition 2012}
\loigiai{
Từ giả thiết, ta có $P\left(2+\sqrt{5}\right)=0,$ tức là
$$(9p+2q+39)+(4p+q+17)\sqrt{5}=0.$$
Do $9p+2q+39$ và $4p+q+17$ đồng thời là các số hữu tỉ, ta bắt buộc có $9p+2q+39=4p+q+17=0.$ Giải hệ, ta tìm ra $p=-5$ và $q=3.$
}
\end{gbtt}

\begin{gbtt}
Cho số nguyên dương $a$ không chính phương. Gọi $r$ là một nghiệm thực của phương trình $x^3-2ax+1=0.$ Chứng minh rằng $r+\sqrt{a}$ là một số vô tỉ.
\nguon{China Girls Mathematical Olympiad 2014}
\loigiai{Từ giả thiết, ta có
$r^3-2ar+1=0.$
Ta giả sử $r+\sqrt{a}$ là một số hữu tỉ. Đặt $r+\sqrt{a}=b,$ ta có
    $$\left(b-\sqrt{a}\right)^3-2a\left(b-\sqrt{a}\right)+1=0,$$
    $$b^3+ab+1=\sqrt{a}\left(3b^2-a\right).$$
Nếu như $3b^2-a\ne 0,$ ta nhận được $\sqrt{a}=\dfrac{3b^2-a}{b^3+3ab^2-2ab+1}$ là số hữu tỉ, vô lí. Như thế thì
\begin{align*}
b^3+ab+1&=0,\tag{1}\\
3b^2-a&=0.\tag{2}
\end{align*}

Từ (2), ta có $a=3b^2.$ Thế vào (1), ta có
$$4b^3+1=0\Leftrightarrow b=-\sqrt[3]{\dfrac{1}{4}}.$$
Kết quả $b$ vô tỉ ở trên cho ta thấy giả sử phản chứng là sai. Như vậy, bài toán được chứng minh.}
\end{gbtt}

\begin{gbtt}
Gọi $\alpha $ là nghiệm dương của phương trình $x^2+x=5$. Với số nguyên dương $n$ nào đó, gọi $c_0,c_1,\ldots ,c_n$ là các số nguyên không âm thỏa mãn đẳng thức $$c_0+c_1\alpha +c_2\alpha ^2+\ldots+c_n\alpha^n=2015.$$
Chứng minh rằng ${{c}_{0}}+{{c}_{1}}+{{c}_{2}}+\ldots+{{c}_{n}}\equiv 2\pmod{3}.$
\nguon{Vietnamese Team Selection Test 2015}
\loigiai{
Ta xét đa thức
$P(x)=c_nx^n+c_{n-1}x^{n-1}+\ldots+c_1x+c_0-2015.$
Từ giả thiết, ta có $$P\left(\alpha\right)=2015-2015=0.$$ Theo như \chu{bổ đề bề bậc nhỏ nhất của đa thức}, ta chỉ ra tồn tại đa thức $Q(x)$ với hệ số nguyên sao cho $P(x)=\left(x^2+x-5\right)Q(x).$ Cho $x=1$ ta được
$$P(1)=-3Q(1).$$
Tổng các hệ số của $P(x)$ chính là $P(1),$ vì thế $P(1)$ phải là bội của $3,$ và vậy thì
$${{c}_{0}}+{{c}_{1}}+{{c}_{2}}+...+{{c}_{n}}\equiv 2015\equiv 2\pmod{3}.$$
Bài toán đã cho được chứng minh.}
\end{gbtt}


\begin{gbtt}
Với các số nguyên $a, b, c$ thỏa mãn $|a|,|b|,|c|\le 10,$ xét đa thức $f(x)=x^3+ax^2+bx+c$ thỏa mãn điều kiện
$$\left|f\tron{2+\sqrt{3}}\right|<0,0001.$$
Chứng minh rằng $2+\sqrt{3}$ là một nghiệm của $f(x).$
\nguon{China Girls Mathematical Olympiad 2017}
\loigiai{Bằng tính toán trực tiếp, ta nhận thấy
\begin{align*}
f\tron{2+\sqrt{3}}&=(7a+2b+c+26)+\sqrt{3}(4a+b+15),\tag{1}\\
f\left(2-\sqrt{3}\right)&=(7a+2b+c+26)-\sqrt{3}(4a+b+15).\tag{2}
\end{align*}
Lấy tích theo vế của (1) và (2), ta được
\[f\tron{2+\sqrt{3}}f\left(2-\sqrt{3}\right)=(7a+2b+c+26)^2-3(4a+b+15)^2.\tag{3}\]
Giả sử $2+\sqrt{3}$ không là nghiệm của $f(x),$ tức $f\tron{2+\sqrt{3}}\ne 0.$ \\
Với mọi số nguyên $A,B \ne 0$ ta luôn có $A^2-2B^2\ne 0,$ hay là $\left|A^2-2B^2\right|\ge 1.$ Nhận xét này cho ta
$$\left|(7a+2b+c+26)^2-3(4a+b+15)^2\right|\ge 1.$$
Đối chiếu với (3) và kết hợp giả thiết $\left|f\tron{2+\sqrt{3}}\right|<0,0001,$ ta có
\[\left|f\left(2-\sqrt{3}\right)\right|>10000.\tag{4}\]
Mặt khác, dựa vào giả thiết $|a|,|b|,|c|\le 10$ và bất đẳng thức $a\le |a|,$ ta chỉ ra
\begin{align*}
    \left|f\left(2-\sqrt{3}\right)\right|
    &=(7a+2b+c+26)-\sqrt{3}(4a+b+15)
    \\&=\left(7-4\sqrt{3}\right)a+\left(2-\sqrt{3}\right)b+c+26-15\sqrt{3}
    \\&\le\left(7-4\sqrt{3}\right)|a|+\left(2-\sqrt{3}\right)|b|+|c|+26-15\sqrt{3}
    \\&\le 10\left(7-4\sqrt{3}\right)+10\left(2-\sqrt{3}\right)+10+26-15\sqrt{3}  
    \\&\le 126-65\sqrt{3}
    \\&<126.
\end{align*}
Đối chiếu điều vừa thu được với (4), ta thấy mâu thuẫn. \\
Mâu thuẫn này chứng tỏ giả sử là sai, do vậy, $f\tron{2+\sqrt{3}}=0.$ Chứng minh hoàn tất.}
\end{gbtt}

\begin{gbtt}
Tồn tại hay không đa thức $P(x)$ với hệ số nguyên thỏa mãn $$P\left(1+\sqrt[3]{2}\right)=1+\sqrt[3]{2},\quad P\left(1+\sqrt{5}\right)=2+3\sqrt{5}\:?$$ 
\nguon{Vietnam Mathematical Olympiad 2017}
\loigiai{
Giả sử tồn tại đa thức $P(x)$ thỏa mãn đề bài. \\
Xét đa thức $Q(x)=P(x)-x.$ Rõ ràng $a=1+\sqrt[3]{2}$ là nghiệm của $Q(x).$ Mặt khác,
$$a-1=\sqrt[3]{2}\Rightarrow (a-1)^3=2\Rightarrow a^3-3a^2+3a-3=0.$$
Theo như \chu{bổ đề về bậc nhỏ nhất của đa thức}, ta suy ra tất cả các đa thức $Q(x)$ đều có dạng
\[Q(x)=\left(x^3-3x^2+3x-3\right)S(x).\tag{*}\]
Ở đây, $S(x)$ là một đa thức hệ số nguyên khác đa thức không. \\
Ngoài ra, từ giả thiết ta cũng có thể suy ra $Q\left(1+\sqrt{5}\right)=1+2\sqrt{5}.$ Trong (*), cho $x=1+\sqrt{5},$ ta được
$$1+2\sqrt{5}=\left(-2+5\sqrt{5}\right)S\left(1+\sqrt{5}\right).$$
Ta đã biết, $S\left(1+\sqrt{5}\right)$ có thể được viết dưới dạng $A+B\sqrt{5},$ trong đó $A,B$ là các số nguyên dương (tham khảo phần \chu{căn thức}). Phép đặt này cho ta
$$1+2\sqrt{5}=\left(-2+5\sqrt{5}\right)\left(A+B\sqrt{5}\right),$$
hay là
$2A-25B+1=\left(5A-2B-2\right)\sqrt{5}.$
Do $\sqrt{5}$ là số vô tỉ, ta suy ra 
\begin{align*}
    2A-25B+1&=0,
    \\5A-2B-2&=0.
\end{align*}
Giải hệ trên, ta tìm ra $A=\dfrac{52}{121}$ và $B=\dfrac{9}{121},$ mâu thuẫn với điều kiện $A,B$ nguyên.\\ Như vậy, giả sử đã cho là sai. Câu trả lời của bài toán là phủ định.}
\end{gbtt}

\begin{gbtt} \
\begin{enumerate}[a,]
    \item Tìm đa thức $P(x)$ khác hằng, có hệ số hữu tỉ, có bậc nhỏ nhất có thể thỏa mãn $$P\tron{\sqrt[3]{3}+\sqrt[3]{9}}=3+\sqrt[3]{3}.$$
    \item Tồn tại hay không đa thức $P(x)$ khác hằng và có hệ số nguyên thỏa mãn $$P\tron{\sqrt[3]{3}+\sqrt[3]{9}}=3+\sqrt[3]{3} \: ?$$
\end{enumerate}
\nguon{Vietnam Mathematical Olympiad 1997}
\loigiai{
Trong bài toán này, ta có sử dụng bổ đề: Nếu $A,B,C$ là các số hữu tỉ thỏa mãn $A\sqrt[3]{3}+B\sqrt[3]{9}+C=0$ thì $A=B=C=0.$ Về phần chứng minh bổ đề, xin mời bạn đọc nghiên cứu lại phần căn thức.
\begin{enumerate}[a,]
    \item Nếu $P(x)$ có bậc nhất, ta đặt $P(x)=ax+b$ với $a, b$ hữu tỉ. Theo đó $$a\left(\sqrt[3]{3}+\sqrt[3]{9}\right)+b=3+\sqrt[3]{3}\Leftrightarrow(a-1)\sqrt[3]{3}+a \sqrt[3]{9}+b-3=0.$$ 
    Theo như bổ đề đã phát biểu, ta có $a-1=a=0$ ở đây, một điều vô lí. \\
    Nếu $P(x)$ có bậc hai, ta đặt $P(x)=a x^2+b x+c$ với $a, b,c$ hữu tỉ. Theo đó
    $$a\left(\sqrt[3]{3}+\sqrt[3]{9}\right)^2+b\left(\sqrt[3]{3}+\sqrt[3]{9}\right)+c=3+\sqrt[3]{3}$$
    $$(a+b) \sqrt[3]{9}+(3a+b-1) \sqrt[3]{3}+6 a+c-3=0.$$
    Theo như bổ đề đã phát biểu, ta có
    \begin{align*}
        a+b=0,\quad
        3a+b=1,\quad
        6a+c=3.
    \end{align*}
    Giải hệ, ta tìm được $a=\dfrac{1}{2},\ b=-\dfrac{1}{2},\ c=0.$ Đa thức $P(x)$ duy nhất thỏa mãn là
    $$P(x)=\dfrac{1}{2}\left(x^2-x\right).$$
    \item Ta đặt $s=\sqrt[3]{3}+\sqrt[3]{9}$. Lập phương hai vế, ta được
    $$s^3=\left(\sqrt[3]{3}+\sqrt[3]{9}\right)^3=12+9\sqrt[3]{3}+9\sqrt[3]{9}=12+9s.$$
    Theo đó, $s$ là nghiệm của đa thức dưới đây
    $$Q(x)=x^{3}-9 x-12 .$$ 
    Giả sử tồn tại đa thức $P(x)$ hệ số nguyên có bậc $n \geq 3$ sao cho $P(s)=3+\sqrt[3]{3}.$ Thực hiện phép chia đa thức $P(x)$ cho đa thức $Q(x),$ ta có
    $$P(x)=Q(x) T(x)+R(x),$$
    ở đây hệ số các đa thức $T(x)$ và $R(x)$ đều nguyên, và ngoài ra $\deg R \leq 2 .$ Cho $x=\sqrt[3]{3}+\sqrt[3]{9},$ ta được
    $$3+\sqrt[3]{3}=R\left(\sqrt[3]{3}+\sqrt[3]{9}\right).$$
    Theo như kết quả của câu trước, ta có $R(x)=\dfrac{1}{2}\left(x^2-x\right),$ mâu thuẫn với điều kiện các hệ số của $R(x)$ là nguyên. Như vậy, giả sử phản chứng là sai. Câu trả lời là phủ định.
\end{enumerate}}
\end{gbtt}

\section{Đa thức và phương trình bậc hai}

\subsection*{Lí thuyết}

\begin{enumerate}
    \item Cho phương trình $ax^2+bx+c=0$ với $a,b,c$ là các số thực, $a\ne 0.$ Ta đặt $\Delta =b^2-4ac.$ Lúc này
    \begin{itemize}
        \item Phương trình có hai nghiệm phân biệt khi và chỉ khi $\Delta>0$ và hai nghiệm ấy là
        $$x=\dfrac{-b+\sqrt{\Delta}}{2},\quad x=\dfrac{-b-\sqrt{\Delta}}{2}.$$
        \item Phương trình  có nghiệm duy nhất khi và chỉ khi $\Delta=0,$ và nghiệm đó là
        $$x=\dfrac{-b}{2a}.$$
        \item Phương trình  vô nghiệm khi và chỉ khi $\Delta<0.$         
    \end{itemize}
    Đặt biệt hơn, trong điều kiện $a,b,c$ là các số nguyên, phương trình có nghiệm nguyên chỉ khi $\Delta$ là số chính phương. Chiều ngược lại là không đúng, vì chẳng hạn, phương trình
    $$12x^2+7x+1=0$$
    có $\Delta=1$ là số chính phương, thế nhưng hai nghiệm $x=\dfrac{-1}{3}$ và $x=\dfrac{-1}{4}$ của nó đều không nguyên.
    \item \chu{Định lí Viete.} \\Giả sử phương trình bậc hai $ax^2+bx+c=0$ có hai nghiệm $x=x_1$ và $x=x_2.$ Khi đó
    $$x_1+x_2=\dfrac{-b}{a},\quad x_1x_2=\dfrac{c}{a}.$$
\end{enumerate}


\subsection*{Ví dụ minh họa}
\begin{bx}
Xét phương trình $x^2-mx+m+2=0$. Tìm tất cả các giá trị nguyên của $m$ sao cho phương trình có các nghiệm đều là số nguyên.
\loigiai{Ta đã biết, một phương trình bậc hai bất kì có nghiệm nguyên thì $\Delta$ phải là số chính phương. Với phương trình đã cho, ta có
$$\Delta=m^2-4(m+2).$$
Như vậy, ta có thể đặt $m^2-4(m+2)=k^2,$ với $k$ là số nguyên dương. Phép đặt này cho ta
$$m^2-4(m+2)=k^2 \Leftrightarrow(m-2+k)(m-2-k)=12.$$
Ta nhận thấy rằng hai số $m-2+k$ và $m-2-k$ cùng tính chẵn lẻ, do tổng của chúng bằng $$m-2+k+m-2-k=2(m-2)$$ là số chẵn. Đồng thời, $m-2+k\ge m-2-k$. Dựa vào hai nhận xét này, ta lập được bảng giá trị sau
\begin{center}
    \begin{tabular}{c|c|c}
        $m-2+k$ & $6$ & $-2$  \\
    \hline
        $m-2-k$ & $2$ & $-6$ \\
    \hline
        $m$ & $6$ & $-2$ \\
    \end{tabular}
\end{center}
Tổng kết lại, có tất cả hai giá trị của $m$ thỏa mãn đề bài, đó là $m=-2$ và $m=6.$
}
\end{bx}

\begin{bx}
Giải hệ phương trình nghiệm nguyên
\[\heva{&x+y+z=5 \\ &xy+yz+zx=8.}\]
\loigiai{
Hệ phương trình đã cho tương đương với
$$
\heva{&xy+(y+x)z=8 \\ &x+y=5-z}
\Leftrightarrow
\heva{&xy+(5-z)z=8 \\ &x+y=5-z}
\Leftrightarrow
\heva{&xy=z^2-5z+8 \\ &x+y=5-z.}
$$
Như vậy theo định lí $Viete$, $x$ và $y$ là hai nghiệm của phương trình bậc hai ẩn $t$ và tham số $z$ là
\[t^2-(5-z)t+(z^2-5z+8).\tag{*}\label{dathuc.ongloi}\]
Phương trình trên có nghiệm khi và chỉ khi $\Delta \ge 0.$ Ta tính được $\Delta=-(3z^2-10z+7),$ vậy nên
$$3 z^{2}-10 z+7 \leq 0 \Leftrightarrow 1 \leq z \leq \frac{7}{3}.$$ 
Do $z$ nhận giá trị nguyên, ta có $z=1$ hoặc $z=2$.
\begin{enumerate}
    \item Với ${z}=1$, phương trình (\ref{dathuc.ongloi}) trở thành ${t}^{2}-4 {t}+4=0$. Đến đây, ta tìm được ${x}={y}=2$ thỏa mãn.
    \item Với ${z}=1$, phương trình (\ref{dathuc.ongloi}) trở thành ${t}^{2}-3 {t}+2=0$. Đến đây, ta tìm được
    $$(x,y)=(1,2)\text{ và }(x,y)=(2,1).$$
\end{enumerate}
Tổng kết lại, hệ phương trình có các nghiệm nguyên là $(2,2,1),(2,1,2)$ và $(1,2,2).$}
\begin{luuy}
Ví dụ trên là một bài toán về nghiệm nguyên và việc sử dụng định lí $Viete$ giúp ta có lời giải ngắn gọn và dễ hiểu. Thông thường, với các bài toán nghiệm nguyên, ta hay chú ý đến sử dụng các kiến thức số học để giải quyết. Tuy nhiên trong ví dụ này, việc làm ấy lại không đem lại hiệu quả, trong khi đó một định lí đại số lại cho ta một lời giải đẹp.
\end{luuy}
\end{bx}

\begin{bx}
Biết rằng phương trình $x^2-ax+b+2=0$ (với $a,b$ là các số nguyên) có hai nghiệm nguyên. Chứng minh rằng $2a^2+b^2$ là hợp số.
\nguon{Chuyên Toán Quảng Trị 2021}
\loigiai{
Gọi $2$ nghiệm nguyên của phương trình đã cho là $x_1$ và $x_2.$ Theo định lí $Viete$, ta có
$$\heva{&x_1+x_2=a \\ &x_1x_2=b+2}\Rightarrow \heva{&a=x_1+x_2 \\ &b=x_1x_2-2.}$$
Các hệ thức trên cho ta
$$2a^2+b^2=2\left(x_1+x_2 \right)^2+\left(x_1x_2-2\right)^2=x^2_1x^2_2+2x^2_1+2x^2_2+4=\left(x^2_1+2\right)\left(x^2_2+2\right).$$
Do $x^2_1+2\ge 2$ và $x^2_2+2\ge 2,$ ta suy ra $2a^2+b^2$ là hợp số. Chứng minh hoàn tất.}
\end{bx}

\begin{bx}
Tìm số nguyên tố $p$ thỏa mãn $p^3-4p+9$ là số chính phương.
\loigiai{
Từ giả thiết, ta có thể đặt $p^{3}-4 p+9=t^{2},$ với $t$ là số tự nhiên. Theo đó
\[p\left(p^{2}-4\right)=(t-3)(t+3).\tag{*}\label{scamhuycao}\]
Do $p$ là số nguyên tố, một trong hai số $t-3$ và $t+3$ phải chia hết cho $p.$ Ta xét các trường hợp kể trên.
\begin{enumerate}
    \item Nếu $t-3$ chia hết cho $p,$ ta đặt $t-3=pk,$ với $k$ là số tự nhiên. Thế vào (\ref{scamhuycao}), ta được
    $$p\left(p^{2}-4\right)=pk(pk+6) \Leftrightarrow k(pk+6)=p^{2}-4\Leftrightarrow p^2-k^2p-(6k+4)=0.$$
    Coi đây là một phương trình bậc hai theo ẩn $p,$ tham số $k.$ Ta tính ra $$\Delta=k^4+4(6k+4)=k^4+24k+16.$$   
    Mặt khác, với $k>3,$ ta chứng minh được
    $$\left(k^{2}\right)^{2}<k^{4}+24 k+16<\left(k^{2}+4\right)^{2}.$$
    Do $\Delta$ là số chính phương, ta xét các trường hợp sau đây.
    \begin{itemize}
        \item Với $k^{4}+24 k+16=\left(k^{2}+1\right)^{2},$ ta có $2 k^{2}-24 k=15.$ Ta không tìm ra $k$ nguyên.
        \item Với $k^{4}+24 k+16=\left(k^{2}+2\right)^{2},$ ta có $k^{2}-6 k=3.$ Ta không tìm ra $k$ nguyên.
        \item Với $k^{4}+24 k+16=\left(k^{2}+3\right)^{2},$ ta có $6 k^{2}-24k=7.$ Ta không tìm ra $k$ nguyên.
    \end{itemize}
    Các trường hợp trên không cho ta $k$ nguyên, chứng tỏ $k\le 3$ thỏa mãn. \\Thử trực tiếp, ta tìm được $k=3,$ kéo theo $t=36$ và $p=11.$
    \item Nếu $t+3$ chia hết cho $p,$ ta đặt $t+3=pl,$ với $l$ là số tự nhiên. Thế vào (\ref{scamhuycao}), ta được
    $$p\left(p^{2}-4\right)=pl(pl-6) \Leftrightarrow l(pl-6)=p^{2}-4\Leftrightarrow p^2-l^2p+(6k-4)=0.$$Coi đây là một phương trình bậc hai theo ẩn $p,$ tham số $l.$ Ta tính ra $$\Delta=l^4-4(6k-4)=l^4-24l+16.$$ 
    Mặt khác, với $l>3,$ ta chứng minh được
    $$\left(l^{2}-4\right)^{2}<l^{4}-24 l+16<\left(l^{2}\right)^{2}.$$    
    Do $\Delta$ là số chính phương, ta xét các trường hợp sau đây.
    \begin{itemize}
        \item Với $l^{4}-24l+16=\left(l^{2}-1\right)^{2},$ ta có $2 l^{2}-24l=-15.$ Ta không tìm ra $l$ nguyên.
        \item Với $l^{4}-24l+16=\left(l^{2}-2\right)^{2},$ ta có $2 l^{2}-24l=-15.$ Ta không tìm ra $l$ nguyên.
        \item Với $l^{4}-24l+16=\left(l^{2}-3\right)^{2},$ ta có $6 l^{2}-24l=-7.$ Ta không tìm ra $l$ nguyên.
    \end{itemize}
    Các trường hợp trên không cho ta $l$ nguyên, chứng tỏ $l\le 3$ thỏa mãn. \\Thử trực tiếp, ta tìm được $l=3,$ kéo theo $(t,p)=(3,2)$ và $(t,p)=(18,7).$
\end{enumerate}
Kết luận, $p=2,p=7$ và $p=11$ là ba số nguyên tố thỏa mãn yêu cầu.}
\end{bx}

\subsection*{Bài tập tự luyện}

\begin{btt}
Tìm tất cả các cặp số nguyên $(x,y)$ thỏa mãn
$$x^2+5y^2+4xy+4y+2x-3=0.$$
\end{btt}

\begin{btt}
Tìm tất cả các nghiệm nguyên của hệ phương trình
$$\heva{x+y+z&=3 \\ x^3+y^3+z^3&=3.}$$
\end{btt}

\begin{btt}
Cho hai số nguyên phân biệt $p,q.$ Chứng minh rằng ít nhất một trong hai phương trình
$$x^2+px+q=0,\quad x^2+qx+p=0.$$
có hai nghiệm thực (không nhất thiết phân biệt).
\end{btt}

\begin{btt}
Tìm tất cả các số nguyên tố $p,q$ sao cho phương trình $x^2-px+q=0$ có các nghiệm là số nguyên.
\nguon{Chuyên Tin Thái Nguyên 2021}
\end{btt}

\begin{btt}
Với $a,b,c$ là ba số nguyên dương, xét hai phương trình bậc hai sau đây
\begin{align*}
    ax^2+bx+c=0,\\
    ax^2+bx-c=0.
\end{align*}
Giả sử cả hai phương trình trên đều có nghiệm nguyên. Chứng minh rằng $abc$ chia hết cho $30.$
\end{btt}

\begin{btt}
Tìm tất cả các số nguyên tố $p,q$ và số tự nhiên $m$ thỏa mãn
$$\dfrac{pq}{p+q}=\dfrac{m^2+6}{m+1}.$$
\end{btt}

\begin{btt}
Tìm tất cả các số nguyên dương $a,b$ sao cho phương trình
$$x^2-abx+a+b=0$$
có tất cả các nghiệm đều nguyên.
\end{btt}

\begin{btt}
Cho \(m\) và \(n\) là các số nguyên dương, khi đó nếu số 
\[ k=\dfrac{(m+n)^2}{4m(m-n)^2+4}\]
là một số nguyên thì \(k\) là một số chính phương.
\nguon{Turkey National Olympiad 2015 }
\end{btt}

\begin{btt}
Tìm tất cả các số nguyên dương $N$ sao cho $N$ có thể biểu diễn duy nhất một cách biểu diễn ở dạng $\dfrac{x^2+y}{xy+1}$ với $x,y$ là hai số nguyên dương.
\nguon{Chuyên Đại học Sư phạm Hà Nội 2021}
\end{btt}

\begin{btt}
Cho $a,b,c$ là ba số nguyên dương sao cho mỗi số trong ba số đó đều biểu diễn dạng lũy thừa của $2$ với số mũ tự nhiên. Biết rằng phương trình bậc hai $ax^2-bx+c=0$ có hai nghiệm đều là số nguyên. Chứng minh rằng hai nghiệm của phương trình này bằng nhau.
\nguon{Chuyên Đại học Sư phạm Hà Nội 2021}
\end{btt}

\begin{btt}
Tìm tất cả các số tự nhiên $n$ và số nguyên tố $p$ thỏa mãn $n^3=p^2-p-1.$
\end{btt}

\begin{btt}
Tìm tất cả các số nguyên tố $p$ sao cho $\dfrac{p^2-p-2}{2}$ là một số lập phương.
\end{btt}

\begin{btt}
Cho ba số nguyên dương $a,b,c$ thỏa mãn $$c\tron{ac+1}^2=(5a+2b)(2c+b).$$
Chứng minh rằng $c$ là một số chính phương lẻ.
\end{btt}

\subsection*{Hướng dẫn bài tập tự luyện}

\begin{gbtt}
Tìm tất cả các cặp số nguyên $(x,y)$ thỏa mãn
$$x^2+5y^2+4xy+4y+2x-3=0.$$
\loigiai{
Phương trình đã cho tương đương với
$$x^2+(4y+2)x+\tron{5y^2+4y-3}=0.$$
Coi đây là phương trình ẩn $x,$ ta có $\Delta_x^{\prime}=(2y+1)^2-\tron{5y^2+4y-3}=-y^2+4.$ Số này chính phương nên $y\in\{-2;0;2\}.$ Thử từng trường hợp, đáp số của bài toán là $(x,y)=(-5,2),(-3,0),(1,0),(3,-2).$}
\end{gbtt}

\begin{gbtt}
Tìm tất cả các nghiệm nguyên của hệ phương trình
$$\heva{x+y+z&=3 \\ x^3+y^3+z^3&=3.}$$
\loigiai{
Thế $z=3-x-y$ vào phương trình thứ hai, ta được
$$x^3+y^3+(3-x-y)^3=3.$$
Phương trình này tương đương với
$$\tron{-x+3}y^2+\tron{-x^2+6x-9}y+\tron{3x^2-9x+8}=0.$$
Nếu $x=3,$ ta không tìm được $y.$ Nếu $x\ne 3,$ coi đây là phương trình bậc hai ẩn $y$ và ta tính được
$$\Delta_y=9(x-1)^2(x-3)(x+5).$$
Do $\Delta_x$ là số chính phương nên hoặc $x=1,$ hoặc $(x-3)(x+5)$ là số chính phương. Đối với trường hợp thứ hai, ta đặt $(x-3)(x+5)=t^2$ rồi tiến hành tách về $(x+1-t)(x+1+t)=16$ và xét trường hợp. Kết quả, các nghiệm nguyên của hệ phương trình là $(1,1,1),(4,4,-5)$ và hoán vị.}
\end{gbtt}

\begin{gbtt}
Cho hai số nguyên phân biệt $p,q.$ Chứng minh rằng ít nhất một trong hai phương trình
$$x^2+px+q=0,\quad x^2+qx+p=0.$$
có hai nghiệm thực (không nhất thiết phân biệt).
\loigiai{
Giả sử cả hai phương trình đều không có nghiệm thực, thế thì $p^2<4q$ và $q^2<4p.$ Do $p,q>0$ nên khi kết hợp, ta có $p^4<16q^2<64p,$ suy ra $p<4.$ Tương tự thì $q<4.$
\begin{enumerate}
    \item Với $p=3,$ ta có $9<4q$ và $q^2<12,$ suy ra $2,25\le q\le 2\sqrt{3}$ tức $q=3,$ trái giả thiết $p\ne q.$
    \item Với $p=2,$ ta có $4<4q$ và $q^2<8,$ suy ra $1< q\le 2\sqrt{2}$ tức $q=2,$ trái giả thiết $p\ne q.$  
    \item Với $p=1,$ ta có $1<4q$ và $q^2<4,$ suy ra $0,25\le q<2$ tức $q=1,$ trái giả thiết $p\ne q.$        
\end{enumerate}
Nói tóm lại, giả sử phản chứng là sai. Bài toán được chứng minh.}
\end{gbtt}

\begin{gbtt}
Tìm tất cả các số nguyên tố $p,q$ sao cho phương trình $x^2-px+q=0$ có các nghiệm là số nguyên.
\nguon{Chuyên Tin Thái Nguyên 2021}
\loigiai{
Ta tìm ra $\Delta =p^2-4q$. Theo đó, ta có thể đặt $p^2-4q=a^2$ với $a$ là số tự nhiên. Phép đặt này cho ta $$4q=(p-a)(p+a).$$
Dựa vào các nhận xét $p-a\equiv p+a\pmod{2}$ và $0<p-a<p+a$, ta lập nên bảng giá trị sau
    \begin{center}
            \begin{tabular}{c|c|c|c}
            $p-a$ & $2$ & $4$ & $q$  \\
            \hline
            $p+a$ & $2q$ & $q$ & $4$ \\
            \hline
            $p$ & $q+1$ & $\dfrac{q+4}{2}$ & $\dfrac{q+4}{2}$ \\
            \end{tabular}
        \end{center}
\begin{enumerate}
    \item Với $p=q+1,$ ta nhận thấy $p$ và $q$ khác tính chẵn lẻ, thế nên số nhỏ hơn trong hai số đó (là $q$) phải bằng $2.$ Ta tìm ra $(p,q)=(3,2).$
    \item Với $p=\dfrac{q+4}{2},$ ta có $2p=q+4.$ Bắt buộc, $q$ phải là số nguyên tố chẵn, thế nên $q=2$.\\ Thay ngược lại, ta chỉ ra $p=3.$
\end{enumerate}
Tổng kết lại, $(p,q)=(3,2)$ là cặp số nguyên tố duy nhất thỏa mãn yêu cầu đề bài.
}
\end{gbtt}

\begin{gbtt}
Với $a,b,c$ là ba số nguyên dương, xét hai phương trình bậc hai sau đây
\begin{align*}
    ax^2+bx+c=0,\\
    ax^2+bx-c=0.
\end{align*}
Giả sử cả hai phương trình trên đều có nghiệm nguyên. Chứng minh rằng $abc$ chia hết cho $30.$
\loigiai{
Từ giả thiết, ta suy ra cả $b^2-4ac$ và $b^2+4ac$ đều là số chính phương. Theo đó, ta chia bài toán thành các bước làm sau đây.
\begin{enumerate}[\color{tuancolor}\bf\sffamily Bước 1.]
    \item Chứng minh $2 \mid abc.$
\begin{itemize}
    \item Nếu $b$ chẵn, hiển nhiên $abc$ chia hết cho $2.$
    \item Nếu $b$ lẻ thì ${b}^{2} \equiv 1 \pmod{8}$. Lại do ${b}^{2}-4 {ac}$ và $b^{2}+4ac$ là các số chính phương lẻ nên ta lần lượt suy ra
    \begin{align*}
        {b}^{2}-4 {ac}, \: {b}^{2}+4 {ac} \equiv 1\pmod{8} &\Rightarrow 4 {ac} \equiv 0\pmod{8} \\&\Rightarrow 2 \mid {ac} 
        \\&\Rightarrow 2 \mid {abc}.
    \end{align*}
\end{itemize}
    \item Chứng minh $3 \mid abc.$
\begin{itemize}
    \item Nếu ${b}$ chia hết cho $3$, hiển nhiên $abc$ chia hết cho $3.$
    \item Nếu $b$ không chia hết cho $3$ thì ${b}^{2} \equiv 1\pmod{3}$. Lại do ${b}^{2}-4 {ac}$ và $b^{2}+4ac$ là các số chính phương nên ta lần lượt suy ra
    \begin{align*}
        b^2-4ac\equiv 0,1\pmod{3}\Rightarrow 4ac\equiv 1,0\pmod{3}\Rightarrow ac\equiv 1,0\pmod{3},\\
        b^2+4ac\equiv 0,1\pmod{3}\Rightarrow 4ac\equiv 2,0\pmod{3}\Rightarrow ac\equiv 2,0\pmod{3}.
    \end{align*}
    Đối chiếu, ta chỉ ra $ac$ chia hết cho $3,$ vậy nên $abc$ cũng chia hết cho $3.$
\end{itemize}
    \item Chứng minh  $5 \mid abc$.
\begin{itemize}
    \item Nếu $b$ chia hết cho $5$, hiển nhiên $abc$ chia hết cho $5.$
    \item Nếu $b$ không chia hết cho $5$ thì $b^{2} \equiv 1,4\pmod{5}$.
    \begin{itemize}
        \item Với $b^2\equiv 1\pmod{5},$ do ${b}^{2}-4 {ac}$ và $b^{2}+4ac$ là các số chính phương nên ta có
    \begin{align*}
        b^2-4ac\equiv 0,1,4\pmod{5}&\Rightarrow 4ac\equiv 1,0,2\pmod{5}\\&\Rightarrow ac\equiv 4,0,3\pmod{5},\\
        b^2+4ac\equiv 0,1,4\pmod{5}&\Rightarrow 4ac\equiv 4,0,3\pmod{5}\\&\Rightarrow ac\equiv 1,0,2\pmod{5}.
    \end{align*}
    Đối chiếu, ta chỉ ra $ac$ chia hết cho $5$, vậy nên $abc$ cũng chia hết cho $5.$
        \item Với $b^2\equiv 4\pmod{5},$ bằng cách làm tương tự, ta cũng có $abc$ chia hết cho $5$  
    \end{itemize}
\end{itemize}
\end{enumerate}
Thông qua các bước làm bên trên, ta nhận thấy $abc$ chia hết cho $[2,3,5]=30.$ \\
Như vậy, bài toán đã cho được chứng minh.}
\end{gbtt}

\begin{gbtt}
Tìm tất cả các số nguyên tố $p,q$ sao cho tồn tại số tự nhiên $m$ thỏa mãn $$\dfrac{pq}{p+q}=\dfrac{m^2+6}{m+1}.$$
\nguon{Hanoi Opening Mathematical Olympiad 2018}
\loigiai{
Giả sử tồn tại các số $p,q$ thỏa yêu cầu. Đầu tiên, ta đặt $d=(pq,p+q).$ Ta có
\begin{align*}
    \heva{&d\mid pq \\ &d\mid (p+q)}
    \Rightarrow \heva{&\hoac{d\mid p \\ d\mid q}\\ &d\mid (p+q)}
    \Rightarrow \heva{d\mid p \\ d\mid q}
    \Rightarrow d\mid (p,q).
\end{align*}
Tới đây, ta xét các trường hợp sau.
\begin{enumerate}
    \item Nếu $p=q$, ta có nhận xét
    $$p=\dfrac{2pq}{p+q}=\dfrac{2m^2+12}{m+1}=2m-2+\dfrac{14}{m+1}.$$
    Ta chỉ ra được $m+1$ là ước nguyên dương của $14.$ Kiểm tra, ta nhận thấy $p=q=7$ khi $m=1.$
    \item Nếu $p \neq q$, ta sẽ xét tới tính tối giản ở hai vế. Thật vậy, ta nhận xét
    \begin{itemize}
        \item[i,] $(pq,p+q)=1,$ đã chứng minh ở trên.
        \item[ii,] $\left(m^2+6,m+1\right)=1$ hoặc $\left(m^2+6,m+1\right)=7.$
    \end{itemize}
    Theo đó, ta cần xét hai trường hợp sau
    \begin{itemize}
        \item \chu{Trường hợp 1.} Nếu $\left(m^2+6,m+1\right)=1$, ta có $\heva{&{p}+{q}={m}+1 \\ &{pq}={m}^{2}+6}.$ \\
        Dựa vào bất đẳng thức quen thuộc $(p+q)^2\ge 4pq,$ ta chỉ ra $(m+1)^2\ge 4\left(m^2+6\right),$ vô lí.
        \item \chu{Trường hợp 2.} Nếu $\left(m^2+6,m+1\right)=7$, ta có $\heva{&7p+7q=m+1 \\ &7pq=m^2+6}.$ \\
        Căn cứ vào đây, ta suy ra ${p}$ và ${q}$ là hai nghiệm của phương trình
        $$7x^{2}-(m+1) x+m^{2}+1=0.$$
        Ta nhận thấy $\Delta=-27 {m}^{2}+2 {m}-27=-({m}-1)^{2}-\left(26 {m}^{2}+26\right)<0$, và khi ấy phương trình đã cho vô nghiệm.
    \end{itemize}
\end{enumerate}
Tổng kết lại, bộ các số nguyên tố cần tìm là $(p,q)=(7,7).$}
\end{gbtt}

\begin{gbtt}
Tìm tất cả các số nguyên dương $a,b$ sao cho phương trình
$$x^2-abx+a+b=0$$
có tất cả các nghiệm đều nguyên.
\loigiai{
Phương trình đã cho có nghiệm nguyên khi thì $\Delta$ là số chính phương. Ta nhận thấy
$$\Delta=a^2b^2-4a-4b.$$
Không mất tính tổng quát, ta giả sử $a\ge b.$ Ta xét các hiệu sau
\begin{align*}
    a^2b^2-\left(a^2b^2-4a-4b\right)&=4a+4b,
    \\\left(a^2b^2-4a-4b\right)-(ab-2)^2&=4(ab-a-b-1)\\&
    =4\left[(a-1)(b-1)-2\right].
\end{align*}
Hai hiệu trên cùng dương khi mà $a\ge b\ge 3.$ Đánh giá này giúp ta chia bài toán thành các trường hợp sau.
\begin{enumerate}
    \item Với $b\ge 3,$ ta có đánh giá bất đẳng thức
    $$(ab-2)^2<a^2b^2-4a-4b<(ab)^2.$$
    Do $a^2b^2-4a-4b$ chính phương nên ta có $a^2b^2-4a-4b=\left(ab-1\right)^2,$ hay là 
    $$a^2b^2-4a-4b=a^2b^2-2ab+1\Leftrightarrow 2ab-4a-4b=1.$$
    Vế trái là chẵn, còn vế phải là lẻ. Trường hợp này không xảy ra.
    \item Với $b=2,$ ta có $4a^2-4a-8$ là số chính phương. Đặt $4a^2-4a-8=z^2,$ và phép đặt này cho ta
    \begin{align*}
    4a^{2}-4a-8=z^{2} &\Leftrightarrow(2 a-1)^{2}-z^{2}=9 \\&\Leftrightarrow(2 a-1-z)(2a-1+z)=9.    
    \end{align*}
    Giải phương trình ước số trên, ta tìm ra $a=2$ và $a=3.$
    \item Với $b=1,$ ta có $a^2-4a-4$ là số chính phương. Đặt $a^2-4a-4=t^2,$ và phép đặt này cho ta
    \begin{align*}
    a^2-4a-4=t^2
    &\Leftrightarrow (a-2)^2-8=t^2
    \\&\Leftrightarrow (a-2-t)(a-2+t)=8.    
    \end{align*}
\end{enumerate}
Như vậy, có $5$ cặp số $(a,b)$ thỏa mãn đề bài, bao gồm
$(1,5),(5,1),(2,3),(3,2),(2,2).$}
\end{gbtt}

\begin{gbtt}
Cho \(m\) và \(n\) là các số nguyên dương, khi đó nếu số 
\[ k=\dfrac{(m+n)^2}{4m(m-n)^2+4}\]
là một số nguyên thì \(k\) là một số chính phương.
\nguon{Turkey National Olympiad 2015 }
\loigiai{
Với giả thiết đã cho, $k$ là số nguyên dương. Ta xét các trường hợp sau đây
 \begin{enumerate}
     \item Nếu \(m=n\), ta có
     $k=\dfrac{(2n)^2}{4}=n^2.$
     \item Nếu \(m\neq n\), từ giả thiết \(k\) là số nguyên dương, ta lần lượt suy ra
         $$4\mid \left ( m+n \right )^2\Rightarrow 4\mid \left ( m-n \right )^2+4mn\\
         \Rightarrow 4\mid \left ( m-n \right )^2$$
     Ta có \(m-n\equiv m+n\equiv 0\pmod 2\), và như thế, tồn tại các số nguyên dương $a,b$ sao cho
     $$a=\dfrac{m+n}{2},\qquad b=\dfrac{m-n}{2}.$$
     Thế trở lại, ta được
    $k=\dfrac{(2a)^2}{4(a+b)\cdot2b+4}=\dfrac{a^2}{4b^2\left ( a+b \right )+1},$ thế nên là
    $$a^2-\left ( 4kb^2 \right )a-\left ( 4kb^3+k \right )=0.$$
    Coi đây là một phương trình bậc hai theo ẩn $a,$ và ta tính ra
     \begin{align*}
         \Delta_a =\left ( 4kb^2 \right )^{2}+4\left ( 4kb^3+k \right )=16k^2b^4+16kb^3+4k =4\left ( 4k^2b^4+4kb^3+k \right ).
     \end{align*}
     Do $k$ là số nguyên dương, ta cần phải có $4k^2b^4+4kb^3+k$ là số chính phương. Không mất tính tổng quát, ta có thể giả sử \(m>n\), khi đó thì \(b=\dfrac{m-n}{2}>0\). Ta có đánh giá
     \[\left ( 2kb^2+b-1 \right )^2<4k^2b^4+4kb^3+k<\left ( 2kb^2+b+1 \right )^2.\]
     Theo như kiến thức đã học, ta có $4k^2b^4+4kb^3+k=\left ( 2kb^2+b\right)^2.$ Ta tìm ra $k=b^2$ từ đây.
 \end{enumerate}
Trong tất cả các trường hợp, $k$ đều là số chính phương. Bài toán được chứng minh.}
\end{gbtt}

\begin{gbtt}
Tìm tất cả các số nguyên dương $N$ sao cho $N$ có thể biểu diễn duy nhất một cách biểu diễn ở dạng $\dfrac{x^2+y}{xy+1}$ với $x,y$ là hai số nguyên dương.
\nguon{Chuyên Đại học Sư phạm Hà Nội 2021}
\loigiai{
Giả sử tồn tại số nguyên dương $N$ thỏa mãn, tức là tồn tại duy nhất một cặp $(x,y)$ sao cho
    \[x^2-\left(Ny\right)x+y-N=0.\tag{*}\label{csp1}\]
    Tính tồn tại của $x$ chứng tỏ $N^2y^2-4y+4N$ là số chính phương. Ta nhận thấy rằng
    \begin{align*}
       \left(Ny+2\right)^2- \left(N^2y^2-4y+4N\right)&=4Ny+4y-4N+4=4(N+1)(y-1)+8>0, \\
       \left(N^2y^2-4y+4N\right)-\left(Ny-2\right)^2&=4Ny-4y+4N-4=4(N-1)(y+1)\ge 0,
    \end{align*}
    Các đánh giá trên cho ta
    $\left(Ny-2\right)^2\le N^2y^2-4y+4N< \left(Ny+2\right)^2.$
    Đến đây, ta xét các trường hợp sau
    \begin{enumerate}
        \item Với $N^2y^2-4y+4N=\left(Ny-2\right)^2,$ ta có    
        \begin{align*}
        N^2y^2-4y+4N=N^2y^2-4Ny+4
        &\Leftrightarrow 4Ny+4N-4y-4=0
        \\&\Leftrightarrow 4(N-1)(y+1)=0.   
        \end{align*}
        Ta tìm ra $N=1.$ Thay trở lại (\ref{csp1}), ta có
        \begin{align*}
        x^2-xy+y-1=0
        &\Leftrightarrow (x-1)(x-y+1)=0.
        \end{align*}      
        Bằng phân tích như vậy, ta chỉ ra số $1$ có vô hạn dạng biểu diễn thỏa mãn, đó là $$1=\dfrac{x^2+(x+1)}{x(x+1)+1},$$
        với $x$ là một số nguyên dương bất kì.
        \item Với $N^2y^2-4y+4N=\left(Ny-1\right)^2,$ ta có
        $$N^2y^2-4y+4N=N^2y^2-2Ny+1\Leftrightarrow 2Ny-2y+4N=1.$$
        So sánh tính chẵn lẻ của hai vế, ta thấy mâu thuẫn.
        \item Với $N^2y^2-4y+4N=\left(Ny\right)^2,$ ta có $y=N.$ Thay trở lại (\ref{csp1}), ta tìm được $x=N^2.$
        \item Với $N^2y^2-4y+4N=\left(Ny+1\right)^2,$ ta có
        $$N^2y^2-4y+4N=N^2y^2+2Ny+1\Leftrightarrow -2Ny-2y+4N=1.$$
        So sánh tính chẵn lẻ của hai vế, ta thấy mâu thuẫn.  
    \end{enumerate}
Kết luận, tất cả các số nguyên dương $N>1$ đều thỏa mãn yêu cầu bài toán, và cách biểu diễn của mỗi số $N$ này là $N=\dfrac{\left(N^2\right)^2+N}{N^2\cdot N+1}.$}
\end{gbtt}

\begin{gbtt}
Cho $a,b,c$ là ba số nguyên dương sao cho mỗi số trong ba số đó đều biểu diễn dạng lũy thừa của $2$ với số mũ tự nhiên. Biết rằng phương trình bậc hai $ax^2-bx+c=0$ có hai nghiệm đều là số nguyên. Chứng minh rằng hai nghiệm của phương trình này bằng nhau.
\nguon{Chuyên Đại học Sư phạm Hà Nội 2021}
\loigiai{
Từ giả thiết, ta có thể đặt $a=2^m,b=2^n,c=2^p,$ với $m,n,p$ nguyên dương. Phương trình đã cho trở thành
    $$2^mx^2-2^nx+2^p=0.$$
    Phương trình này có hai nghiệm nguyên chỉ khi
    $$\Delta=4^n-2^{m+p+2}=2^{2n}-2^{m+p+2}$$
    là số chính phương, và hiển nhiên $2n\ge m+p+2.$ \\
    Ta đặt $2^{2n}-2^{m+p+2}=q^2,$ với $q$ nguyên dương. Theo đó,
    $$\left(2^n-q\right)\left(2^n+q\right)=2^{m+p+2}.$$
    Rõ ràng, $2^n-q$ và $2^n+q$ đều là các lũy thừa cơ số $2.$ Ta tiếp tục đặt $2^n-q=2^k,2^n+q=2^l,$ với $k,l$ là các số tự nhiên và $k\le l.$ Lấy tổng theo vế, ta được
    $$2^{n+1}=2^k+2^l=2^k\left(2^{l-k}+1\right).$$
    So sánh số mũ của $2$ ở hai vế, ta tìm được $k=l,$ kéo theo $q=0$ và $\Delta=0.$ Nói cách khác, phương trình bậc hai đã cho có nghiệm kép. Bài toán được chứng minh.}
\end{gbtt}


\begin{gbtt}
Tìm tất cả các số tự nhiên $n$ và số nguyên tố $p$ thỏa mãn $n^3=p^2-p-1.$
\loigiai{
Chuyển vế, ta được $\tron{n+1}\tron{n^2-n+1}=p(p-1).$
\begin{enumerate}
    \item Nếu $n+1$ chia hết cho $p$ thì ta có
    $$p(p-1)\ge p\tron{(p-1)^2-(p-1)+1}=p\tron{p^2-3p+3},$$
    suy ra $p=2,$ và khi thế ngược lại ta được $n=1.$
    \item Nếu $n^2-n+1$ chia hết cho $p,$ ta đặt $n^2-n+1=kp.$ Kết hợp với $\tron{n+1}\tron{n^2-n+1}=p(p-1),$ ta có $p-1=k(n+1),$ và thế trở lại $n^2-n+1=kp$ thì
    $$n^2-\tron{k^2+1}n-\tron{k^2+k-1}=0.$$
    Coi đây là phương trình bậc hai ẩn $n,$ khi đó
    $$\Delta_n=\tron{k^2+1}^2+4\tron{k^2+k-1}$$
    là số chính phương. Bằng so sánh
    $$\tron{k^2+1}^2<\tron{k^2+1}^2+4\tron{k^2+k-1}<\tron{k^2+4}^2,$$
    ta tìm được $k=3.$ Theo đó thì $n=11$ và $p=37.$
\end{enumerate}}
\end{gbtt}

\begin{gbtt}
Tìm tất cả các số nguyên tố $p$ sao cho $\dfrac{p^2-p-2}{2}$ là một số lập phương.
\loigiai{
Ta đặt $\dfrac{{p}^{2}-{p}-2}{2}={n}^{3},$ với ${n}$ là một số tự nhiên. Phép đặt này cho ta
$$p^2-p-2=2n^3\Leftrightarrow p^2-p=2n^3+2\Leftrightarrow p(p-1)=2(n+1)\left(n^2-n+1\right).$$
Ta được $p\mid 2(n+1)\left(n^2-n+1\right).$ Ta xét các trường hợp sau.
\begin{enumerate}
    \item Với $p=2,$ bằng kiểm tra trực tiếp, ta thấy thỏa mãn đề bài.
    \item Với $p \mid (n+1),$ ta có $p\le n+1.$ Đánh giá này cho ta
    $$2(n+1)\left(n^2-n+1\right)=p(p-1)\le (n+1)n. $$
    Ta được $2n^2-2n+1\le n.$ Đến đây ta tìm được $n=1$ và $p=2$. 
    \item Với $p \mid \left(n^2-n+1\right),$ ta đặt $n^2-n+1=kp,$ ở đây ${k}$ là số nguyên dương. Phép đặt này cho ta
    $$p(p-1)=2(n+1)\left(n^2-n+1\right)=2(n+1)kp.$$
    Từ đây, ta có $p=2kn+2k+1.$ Thế ngược lại phép đặt, ta chỉ ra
    $$n^2-n+1=k(2kn+2k+1)\Leftrightarrow n^2-\left(2k^2+1\right)n-\left(2k^2+k-1\right)=0.$$
    Coi phương trình trên là một phương trình bậc hai ẩn $n,$ lúc này 
    $$\Delta=\left(2 {k}^{2}+1\right)^{2}+4\left(2 {k}^{2}+{k}-1\right)$$ 
    phải là số chính phương. Đánh giá được $\left(2 k^{2}+1\right)^{2}<\Delta<\left(2 k^{2}+4\right)^{2}$ và $\Delta$ lẻ, ta suy ra $$\Delta=\left(2 k^{2}+3\right)^{2}.$$ Ta lần lượt tìm được $k=3,n=20,p=127.$
\end{enumerate}
Như vậy, có $2$ giá trị của $p$ thỏa mãn đề bài là ${p}=2$ và ${p}=127$.}
\end{gbtt}

\begin{gbtt}
Cho ba số nguyên dương $a,b,c$ thỏa mãn $$c\tron{ac+1}^2=(5c+2b)(2c+b).$$
Chứng minh rằng $c$ là một số chính phương lẻ.
\loigiai{
Đẳng thức đã cho tương đương với
$$2b^2+9cb+10c^2-c(ac+1)^2=0.$$
Nếu $b$ là biến và $a,c$ là tham số, đây sẽ là phương trình bậc hai ẩn $b.$ Ta tính được
$$\Delta_b=c\vuong{c+8(ac+1)^2}.$$
Với việc $\Delta_b$ là số chính phương, ta sẽ nghĩ đến phép đặt $d=\tron{c,c+8(ac+1)^2}.$ Phép đặt cho ta
$$\heva{&d\mid c \\ &d\mid \vuong{c+8(ac+1)^2}}
\Rightarrow \heva{&d\mid c \\ &d\mid \vuong{c\tron{1+8a^2c+16a}+8}}
\Rightarrow d\mid 8.$$
Ta sẽ có $d\in \{1;2;4;8\}.$ Tới đây, ta xét các trường hợp sau.
\begin{enumerate}
    \item Nếu $d=8$ hoặc $d=2,$ ta có $c$ và $c+8(ac+1)^2$ đều là hai lần một số chính phương. \\
    Ta đặt
    $c=2x^2$ và $c+8(ac+1)^2=2y^2.$ Khi đó
    $$2y^2=2x^2+8\tron{2ax^2+1}^2\Leftrightarrow y^2=x^2+4\tron{2ax^2+1}^2.$$
    Thế nhưng, điều này là không thể xảy ra do
    $$\tron{4ax^2+2}^2<x^2+4\tron{2ax^2+1}^2<\tron{4ax^2+3}^2.$$
    \item Nếu $d=4,$ ta có $c$ và $c+8(ac+1)^2$ hai số chính phương chẵn, và ngoài ra
    $$\tron{\dfrac{c}{4},\dfrac{c}{4}+2(ac+1)^2}=1.$$
    Nếu $\dfrac{c}{4}$ là số chẵn thì $\dfrac{c}{4}+2(ac+1)^2$ cũng chẵn, mâu thuẫn.\\ Do đó $\dfrac{c}{4}$ là số chính phương lẻ, và khi ấy $\dfrac{c}{4}\equiv 1\pmod{4}.$ Nhưng lúc này
    $$\dfrac{c}{4}+2(ac+1)^2\equiv 1+2\equiv 3\pmod{4}.$$
    Không có số chính phương nào chia $4$ dư $3.$ Trường hợp này không xảy ra.
    \item Nếu $d=1,$ ta có $c$ là số chính phương. Hoàn toàn tương tự trường hợp trước, ta chỉ ra $c$ lẻ.
\end{enumerate}
Từ tất cả các trường hợp đã xét, bài toán đã cho được chứng minh.}
\end{gbtt} 	%đa thức nguyên
\chapter{Phương trình nghiệm nguyên}

Ở tiểu học, chúng ta đã được làm quen với các bài toán giải phương trình từ rất sớm qua các bài toán “tìm $x$” thân thuộc. Lên cấp hai, các bài toán giải phương trình trở nên đặc sắc hơn và khó hơn rất nhiều, đặc biệt là phương trình nghiệm nguyên. Nếu ở tiểu học, một phương trình chỉ có một ẩn thì ở các bài toán giải phương trình nghiệm nguyên, mỗi phương trình đều có hai, thậm chí ba, bốn ẩn số. Để giải được những phương trình này, chúng ta cần kết hợp nhiều tính chất số học khác nhau, những “mánh khóe” của riêng từng người. \\ \\
Giải phương trình nghiệm nguyên có thể coi là dạng bài thường gặp nhất ở số học THCS. Ở chương V của cuốn sách này, tác giả sẽ phân loại các bài phương trình nghiệm nguyên theo dạng bài gắn với phương pháp giải, được thể hiện rõ trong 12 phần
\begin{itemize}
    \item\chu{Phần 1.} Phương trình ước số.
    \item\chu{Phần 2.} Phép phân tích thành tổng các bình phương.
    \item\chu{Phần 3.} Bất đẳng thức trong phương trình nghiệm nguyên.
    \item\chu{Phần 4.} Phương pháp lựa chọn modulo.
    \item\chu{Phần 5.} Tính chia hết và phép cô lập biến số.
    \item\chu{Phần 6.} Phương trình nghiệm nguyên quy về dạng bậc hai
    \item\chu{Phần 7.} Phương trình với nghiệm nguyên tố.
    \item\chu{Phần 8.} Trở lại với phương pháp kẹp lũy thừa.
    \item\chu{Phần 9.} Phép gọi ước chung.
    \item\chu{Phần 10.} Phương trình chứa ẩn ở mũ.
    \item\chu{Phần 11.} Phương trình chứa căn thức.
    \item\chu{Phần 12.} Bài toán về số tự nhiên và các chữ số.
\end{itemize}

\section{Phương trình ước số}
Phương trình ước số là ứng dụng của kĩ thuật phân tích đa thức thành nhân tử trong đại số. Muốn hiểu tường tận các cách tìm nhân tử, các bạn trước hết cần rèn luyện kĩ năng phân tích, mà sách có đề cập qua ở \chu{chương I}. Dưới đây là một số ví dụ minh họa.

\subsection*{Ví dụ minh họa}

\begin{bx}
Giải phương trình nghiệm nguyên \[(x-2)(3x-2)(5x-2)(7x-2)=945.\]
\loigiai{
Đặt $A=(x-2)(3x-2)(5x-2)(7x-2)$.
\begin{itemize}
\item Nếu $x\geq 3$, ta có $\heva{&x-2\geq 1\\ &3x-2\geq 7\\ & 5x-2\geq 13\\ & 7x-2\geq 19},$ như vậy $A\geq 1\cdot7\cdot13\cdot19=1729$.
\item Nếu $x\leq -2$, ta có $\heva{&x-2\leq -4\\ &3x-2\leq -8\\ & 5x-2\leq -12\\ & 7x-2\leq -16},$ như vậy $A\geq 4\cdot8\cdot12\cdot16=6164$.
\end{itemize}
Theo đó, phương trình đã cho chỉ có thể nhận $-1,0,1,2$ làm nghiệm
\begin{enumerate}
\item Với $x=-1$, ta nhận được $A=(-3)\cdot(-5)\cdot(-7)\cdot(-9)=945$.
\item Với $x=0$, ta nhận được $A=(-2)^{4}=16$.
\item Với $x=1$, ta nhận được $A=(-1)\cdot1\cdot3\cdot5=-15$ (loại).
\item Với $x=2$, ta nhận được $A=0$ (loại).
\end{enumerate}
Dựa vào đây, ta kết luận phương trình có duy nhất một nghiệm nguyên là $x=-1.$}
\begin{luuy}
Không đơn giản chút nào khi cố gắng tạo ra nhân tử $x+1$ từ đa thức bậc bốn tương ứng. Vì lẽ đó, phương pháp xét khoảng giá trị của $x$ dựa trên điều kiện $x$ nguyên là tối ưu hơn cả.
\end{luuy}
\end{bx}

\begin{bx}
Giải phương trình nghiệm nguyên $5x-3y=2xy-11.$
\loigiai{
\begin{enumerate}[\color{tuancolor}\sffamily\bfseries Cách 1.]
\item Rõ ràng, $2x+3\ne 0.$ Phương trình đã cho tương đương với
$$2xy+3y=5x+11\Leftrightarrow (2x+3)y=5x+11\Leftrightarrow y=\dfrac{5x+11}{2x+3}\Leftrightarrow y=2+\dfrac{x+5}{2x+3}.$$
Như vậy, phương trình có nghiệm nguyên chỉ khi $x+5$ chia hết cho $2x+3.$ Ta lần lượt suy ra
$$(2x+3)\mid 2(x+5)\Rightarrow (2x+3)\mid (2x+10)\Rightarrow (2x+3)\mid 7.$$
Lập luận trên cho ta biết $2x+3$ là ước của $7.$ Ta lập được bảng giá trị sau.
\begin{center}
\begin{tabular}{c|c|c|c|c}
    $2x+3$ & $1$ &  $-1$&$7$&$-7$\\ 
    \hline
    $x$ & $-1$ & $-2$ & $2$&$-5$ \\ 
    \hline 
    $y$ & $6$ & $-1$ & $3$&$2$ \\ 
    \end{tabular}            
\end{center}
Tổng kết lại, phương trình đã cho có tất cả $4$ nghiệm nguyên, đó là
\[(-5,2),(-2,-1),(-1,6),(2,3).\]

\item Phương trình đã cho tương đương với
 \begin{align*}
   5x-3y=2xy-11
    &\Leftrightarrow 10x-6y=4xy-22\\
    &\Leftrightarrow 4xy-10x+6y-15=7\\
    &\Leftrightarrow 2x\left(2y-5\right)+3\left(2y-5\right)=7\\
    &\Leftrightarrow \left(2y-5\right)\left(2x+3\right)=7.
\end{align*}
Ta nhận thấy, $2x+3$ và $2y-5$ là ước của $7$. Ta lập bảng giá trị sau.
     \begin{center}
    \begin{tabular}{c|c|c|c|c}
                 $2x+3$ & $1$ &  $-1$&$7$&$-7$\\ 
                 \hline
                 $2y-5$ & $7$ & $-7$ & $1$&$-1$ 
                 \end{tabular}            
    \end{center}
Dựa theo bảng giá trị này, ta tìm ra các nghiệm $(x,y)$ giống như \chu{cách 1}.
\end{enumerate}}
\end{bx}

\begin{bx}
Giải phương trình nghiệm nguyên
\[x^2-2x-11=y^2.\]
\loigiai{Phương trình đã cho tương đương với
\begin{align*}
  x^2-2x+1-12=y^2
    &\Leftrightarrow (x-1)^2-y^2=12 \\
    &\Leftrightarrow (x-1+y)(x-1-y)=12.
\end{align*}
Vế trái phương trình không thay đổi khi thay $y$ bởi $-y$, thế nên ta chỉ cần xét $y \geq 0$. Khi đó
$$x-1+y \geq x-1-y.$$
Mặt khác, do $(x-1+y)-(x-1-y)=2y$ nên $x-1+y$ và $x-1-y$ cùng tính chẵn lẻ. Tích của chúng bằng $12$ nên chúng cùng chẵn. Với các nhận xét trên, ta thu được hai trường hợp, đó là
$$\heva{&x-1+y=6\\&x-1-y=2} \text{ hoặc } \heva{&x-1+y=-2\\&x-1-y=-6.}$$
Mỗi hệ trên lần lượt cho $(x,y)=(5,2)$ và $(x,y)=(-3,2)$.\\
Như vậy, tất cả các nghiệm $(x,y)$ nguyên của phương trình là $(-3,-2),(-3,2),(5,-2)$ và $(5,2)$.}
\end{bx}

\begin{bx}
Tìm tất cả các cặp số nguyên $(x,y)$ thỏa mãn \[x^2+5xy+6y^2+x+2y-2=0.\]
\nguon{Chuyên Toán Hà Nội 2021}
\loigiai{
Giả sử tồn tại cặp $(x,y)$ thỏa yêu cầu. Ta có
    $$(x+2 y)(x+3 y)+x+2 y-2=0\Leftrightarrow(x+2 y)(x+3 y+1)=2.$$
    Căn cứ vào đây, ta lập được bảng giá trị
    \begin{center}
    \begin{tabular}{c|c|c|c|c}
         $x+2y$ & $-2$ & $-1$ & $1$ & $2$   \\
         \hline
         $x+3y+1$  & $-1$ & $-2$ & $2$ & $1$ \\ 
         \hline
         $x$ & $-2$ & $3$ & $1$ & $6$   \\
         \hline
         $y$ & $0$ & $-2$ & $0$ & $-2$   \\         
    \end{tabular}        
    \end{center}
Như vậy, có tổng cộng bốn cặp $(x, y)$ thỏa mãn đề bài, bao gồm $(-2,0),(1,0),(3,-2)$ và $(6,-2).$}
\end{bx}

\begin{bx} Tìm tất cả các số nguyên $a,b$ thỏa mãn \[a^4-2a^3+10a^2-18a-16=4b^2+20b.\]
\nguon{Chuyên Toán Cao Bằng 2021}
\loigiai{
Phương trình đã cho tương đương với
\begin{align*}
    a^4-2a^3+10a^2-18a+9=4b^2+20b+25
    &\Leftrightarrow a^4-2a^3+a^2+9a^2-18a+9=(2b+5)^2
    \\&\Leftrightarrow a^2(a-1)^2+9(a-1)^2=(2b+5)^2
    \\&\Leftrightarrow \left(a^2+9\right)(a-1)^2=(2b+5)^2.
\end{align*}
Giả sử tồn tại cặp $(a,b)$ thỏa yêu cầu. Nếu như $a=1,$ ta suy ra $b=-\dfrac{5}{2},$ mâu thuẫn giả thiết $a,b$ là số nguyên. Còn nếu $a\ne 1,$  $a^2+9$ là số chính phương. Ta đặt $a^2+9=c^2,$ với $c$ nguyên dương. Ta có
$$(c-a)(c+a)=9.$$
Do $c>|a|,$ ta suy ra $c-a>0$ và $c+a>0.$ Dựa vào đây, ta lập được bảng giá trị sau.
\begin{center}
\begin{tabular}{c|c|c|c}
$c+a$ & $9$ & $3$ & $1$ \\ 
\hline
$c-a$ & $1$ & $3$ & $9$\\ 
\hline
$a$ & $4$ & $0$ & $-4$\\ 
\end{tabular}
\end{center}
Lần lượt thay $a=-4,0,4$ trở lại phương trình ban đầu, ta tìm được tất cả 6 cặp $(a,b)$ thỏa yêu cầu, đó là
$$(-4,-15),(-4,10),(0,-4),(0,-1),(4,-1),(4,5).$$
}
\end{bx} 

\begin{bx}\label{hdtbacbar}
Giải phương trình nghiệm nguyên $x^3+y^3+6xy=21.$
\end{bx}
\chu{Nhận xét.}\\
Quan sát thấy các hạng tử $x^3,y^3$ và $6xy$ ở vế trái cho phép ta nghĩ đến việc sử dụng hằng đẳng thức
$$x^3+y^3+z^3-3xyz=(x+y+z)\left(x^2+y^2+z^2-xy-yz-zx\right).$$
Theo đó, khi cho $z=-2,$ ta được
\[x^3+y^3-8+6xy=(x+y-2)\left(x^2+y^2+4-xy+2y+2x\right).\]
\loigiai{
Phương trình đã cho tương đương với
$$x^3+y^3+(-2)^3-3xy\cdot(-2)=13\Leftrightarrow (x+y-2)\left(x^2+y^2+4-xy+2x+2y\right)=13.$$
Ta nhận thấy rằng 
$$2\left(x^2+y^2+4-xy+2x+2y\right)=(x-y)^2+(x+2)^2+(y+2)^2\ge 0.$$
Nhận xét trên cho ta biết, $x+y-2$ là ước nguyên dương của $13,$ tức $x+y-2\in \{1;13\}.$ \\
Ta xét các trường hợp kể trên.
\begin{itemize}
    \item \chu{Trường hợp 1.} Với $x+y-2=1,$ ta có
    \begin{align*}
    \heva{&x+y-2=1 \\ &x^2+y^2+4-xy+2x+2y=13}
    &\Leftrightarrow
    \heva{&y=3-x \\ &x^2+(3-x )^2+4-x(3-x)+2x+2(3-x)=13}
    \\&\Leftrightarrow
    \heva{&y=3-x \\ &3x^2-9x+6=0}
    \\&\Leftrightarrow
    \hoac{&x=1,y=2 \\ &x=2,y=1.}
    \end{align*}
    \item \chu{Trường hợp 2.} Với $x+y-2=13,$ ta có
    \begin{align*}
    \heva{&x+y-2=13 \\ &x^2+y^2+4-xy+2x+2y=1}
    &\Leftrightarrow
    \heva{&y=15-x \\ &x^2+(15-x )^2+4-x(15-x)+2x+2(15-x)=1}
    \\&\Leftrightarrow
    \heva{&y=15-x \\ &3x^2-45x+258=0}
    \\&\Leftrightarrow
    \heva{&y=15-x \\ &4x^2-60x+344=0}
    \\&\Leftrightarrow
    \heva{&y=15-x \\ &(2x-15)^2+89=0.}
    \end{align*} 
    Hệ trên không thể có nghiệm thực.
\end{itemize}
Tổng kết lại, phương trình đã cho có hai nghiệm nguyên, đó là $(2,3)$ và $(3,2).$
}

\subsection*{Bài tập tự luyện}

\begin{btt}
Giải phương trình nghiệm nguyên
\[\left(x^{2}-1\right)\left(x^{2}-11\right)\left(x^{2}-21\right)\left(x^{2}-31\right)=-4224.\]
\end{btt}

\begin{btt}
Giải phương trình nghiệm nguyên
$$x^{4}=24x+9.$$
\end{btt}

\begin{btt}
Giải phương trình nghiệm nguyên $$2xy-x-y+1=0.$$
\end{btt}

\begin{btt}
Giải phương trình nghiệm nguyên dương
\[\dfrac{1}{x}+\dfrac{1}{y}+\dfrac{1}{6xy} = \dfrac{1}{6}.\]
\end{btt}

\begin{btt}
Tìm tất cả các bộ ba số nguyên dương $(x,y,z)$ thỏa mãn đồng thời các điều kiện $$\sqrt{xy}+\sqrt{xz}-\sqrt{yz}=y,\quad  \dfrac{1}{x}+\dfrac{1}{y}-\dfrac{1}{z}=1.$$
\nguon{Chuyên Tin Thanh Hóa 2021}
\end{btt}

\begin{btt}
Cho hình lăng trụ đứng, đáy là tam giác vuông, chiều cao bằng $6.$ Số đo ba cạnh của tam giác đáy là các số nguyên. Số đo diện tích toàn phần của lăng trụ bằng số đo thể tích của lăng trụ. Tính số đo ba cạnh tam giác đáy của lăng trụ.
\nguon{Chuyên Toán Quảng Ninh 2021}
\end{btt}

\begin{btt}
Tìm tất cả các số nguyên dương $x$ sao cho $x^2-x+13$ là số chính phương.
\nguon{Chuyên Tin Bình Định 2021}
\end{btt}

\begin{btt}
Tìm tất cả các số nguyên dương $n$ sao cho hai số $n^2-2n-7$ và $n^2-2n+12$ đều là lập phương của một số nguyên dương nào đó.
\nguon{Chuyên Toán Quảng Bình 2021}
\end{btt}

\begin{btt}
Giải phương trình nghiệm nguyên $$x^2-2x+2y=2(xy+1).$$
\nguon{Chuyên Toán Lào Cai 2021}
\end{btt}

\begin{btt}
Giải phương trình nghiệm nguyên $$(2x+y)(x-y)+3(2x+y)-5(x-y)=22.$$
\nguon{Chuyên Toán Bình Phước 2021}
\end{btt}

\begin{btt}
Tìm tất cả các số nguyên $x, y$ thỏa mãn $$x^2-xy-2y^2+x+y-5=0.$$
\nguon{Chuyên Tin Hà Nội 2021}
\end{btt}

\begin{btt}
Tìm tất cả các cặp số nguyên $(x,y)$ thỏa mãn
$$2 x^{2}-x y+9 x-3 y+4=0.$$
\nguon{Chuyên Toán Lạng Sơn 2021}
\end{btt}

\begin{btt}
Giải phương trình nghiệm nguyên $$y^2+3y=x^4+x^2+18.$$
\nguon{Chuyên Toán Ninh Thuận 2021}
\end{btt}

\begin{btt}
Tìm tất cả các dãy số tự nhiên chẵn liên tiếp có tổng bằng $2010.$
\nguon{Chuyên Quốc Học Huế 2010 $-$ 2011}
\end{btt}

\begin{btt}
Tìm tất cả các nghiệm nguyên của phương trình $$x^{2}-y^{2}\left(x+y^{4}+6 y^{2}\right)=0.$$
\nguon{Chuyên Toán Bắc Giang 2021}
\end{btt}

\begin{btt}
Giải phương trình nghiệm nguyên \[x^2+xy+y^2=\tron{\dfrac{x+y}{3}+1}^3.\]
\end{btt}

\begin{btt}
Tìm các số nguyên dương $a, b, c, d$  thỏa mãn đồng thời các điều kiện
\[{a^2} = {b^3},\quad {c^3} = {d^4},\quad a = d + 98.\]
\nguon{Chuyên Đại học Sư Phạm Hà Nội 2017 $-$ 2018}
\end{btt}

\begin{btt}
Giải phương trình nghiệm nguyên $$(xy-1)^2=x^2+y^2.$$
\nguon{Chuyên Toán Bà Rịa $-$ Vũng Tàu 2021}
\end{btt}

\begin{btt}
Giải phương trình nghiệm nguyên
$$x^{2}(y - 1) + y^{2}(x-1) = 1.$$
\nguon{Polish Mathematical Olympiad 2004}
\end{btt}

\begin{btt}
Giải phương trình nghiệm nguyên $$x^2-xy+y^2=x^2y^2-5.$$
\nguon{Chuyên Khoa học Tự nhiên 2015}
\end{btt}

\begin{btt}
Giải phương trình nghiệm nguyên $$x^3+y^3+6xy=21.$$
\end{btt}

\begin{btt}
Giải phương trình nghiệm nguyên dương $$x^2y^2(y-x)=5xy^2-27.$$
\nguon{Chuyên Toán Nam Định 2021}
\end{btt}
\begin{btt}
Giải phương trình nghiệm nguyên dương $x^2y^2(y-x)=5xy^2-27.$
\nguon{Chuyên Toán Nam Định 2021}
\end{btt}
\begin{btt}
Giải phương trình nghiệm nguyên $$x^2y-xy+2x-1=y^2-xy^2-2y.$$
\nguon{Chuyên Toán Bến Tre 2021}
\end{btt}

\begin{btt}
Giải phương trình nghiệm nguyên $$x^3y-x^3-1=2x^2+2x+y.$$
\nguon{Chuyên Toán Kon Tum 2021}
\end{btt}


\begin{btt}
Giải phương trình nghiệm nguyên $$(x+2)^2(y-2)+xy^2+26=0.$$
\end{btt}

\begin{btt}
Giải phương trình nghiệm nguyên 
\[2xy^2+x+y+1=x^2+2y^2+xy.\]
\nguon{Hanoi Open Mathematics Competitions 2015}
\end{btt}

\begin{btt}
Tìm tất cả các số nguyên dương $m,n$ thỏa mãn 
\[m(m, n)+n^{2}[m, n]=m^{2}+n^{3}-330.\]
\end{btt}

\begin{btt}
Tìm tất cả các số tự nhiên $n$ sao cho số $2^8+2^{11}+2^n$ là số chính phương.
\nguon{Violympic Toán lớp 9}
\end{btt}

\begin{btt}
Tìm tất cả các số nguyên dương $x,y$ thỏa mãn $$x^2-2^y\cdot x-4^{21}\cdot 9=0.$$
\nguon{Chuyên Toán Thừa Thiên Huế 2021}
\end{btt}

\subsection*{Hướng dẫn bài tập tự luyện}

\begin{gbtt}
Giải phương trình nghiệm nguyên
\[\left(x^{2}-1\right)\left(x^{2}-11\right)\left(x^{2}-21\right)\left(x^{2}-31\right)=-4224.\]
\loigiai{
Ta giả sử, phương trình đã cho có nghiệm $x.$ Ta đặt $$A=\left(x^{2}-11\right)\left(x^{2}-21\right)\left(x^{2}-31\right).$$ Vì $A<0$ và là tích của bốn thừa số $x^{2}-1$, $x^{2}-11$, $x^{2}-21$, $x^{2}-31$ nên trong bốn thừa số trên phải có một hoặc ba thừa số âm. Dựa vào nhận định $x^{2}-1>x^{2}-11>x^{2}-21>x^{2}-31$, ta xét hai trường hợp
\begin{enumerate}
\item  Nếu trong $A$ có ba thừa số âm, ta có 
$$x^{2}-1>0>x^{2}-11\Rightarrow 1<x^{2}<11\Rightarrow x^{2}\in \{4;9\}\Rightarrow x\in \{\pm 2;\pm 3\}.$$
\begin{itemize}
\item\chu{Trường hợp 1.} Nếu $x=\pm 2$ thì $A=3\cdot(-7)\cdot(-17)\cdot(-27)=-9639$, mâu thuẫn.
\item\chu{Trường hợp 2.} Nếu $x=\pm 3$ thì $A=8\cdot(-2)\cdot(-12)\cdot(-22)=-4224$, thỏa mãn.
\end{itemize}
\item Nếu trong $A$ có một thừa số âm, ta có  
\begin{align*}
    x^{2}-21>0>x^{2}-31&\Rightarrow 21<x^{2}<31
    \\&\Rightarrow x^{2}=25\\&\Rightarrow A=24\cdot14\cdot4\cdot(-6)=-8064,
\end{align*}
mâu thuẫn.
\end{enumerate}
Tổng kết lại, phương trình đã cho có hai nghiệm nguyên là $x=-3$ và $x=3.$}
\end{gbtt}

\begin{gbtt}
Giải phương trình nghiệm nguyên
$x^{4}=24x+9.$
\loigiai{
Phương trình đã cho tương đương với
\begin{align*}
    x^4-24x-9=0&\Leftrightarrow x^4-27x+3x-9=0
    \\&\Leftrightarrow x(x-3)\left(x^2+3x+9\right)+3(x-3)=0 \\&\Leftrightarrow (x-3)\left(x^{3}+3x^{2}+9x+3\right)=0.
\end{align*}
Ta sẽ chứng minh $x=3$ là nghiệm nguyên duy nhất của phương trình đã cho. Thật vậy.
\begin{enumerate}
\item Với $x\geq 0,$ ta có $x^{3}+3x^{2}+9x+3>0.$
\item Với $x< 0$, ta có $x\le -1,$ và vì thế
\[x^{3}+3x^{2}+9x+3=\tron{x^2+2x+7}\tron{x+1}-4=\bigg((x+1)^2+6\bigg)\tron{x+1}-4\le -4.\]
\end{enumerate}
Như vậy, $x=3$ là nghiệm nguyên duy nhất của phương trình.}
\end{gbtt}

\begin{gbtt}
Giải phương trình nghiệm nguyên $2xy-x-y+1=0.$
\loigiai{
Phương trình đã cho tương đương với
$$4xy-2x-2y+2=0\Leftrightarrow 2x(2y-1)-(2y-1)+1=0\Leftrightarrow (2x-1)(2y-1)=-1.$$
Ta nhận thấy, $2x-1$ và $2y-1$ là ước của $1$. Ta lập bảng giá trị sau.
     \begin{center}
    \begin{tabular}{c|c|c}
        $2x-1$ & $1$ &  $-1$ \\
        \hline
        $x$ & $1$ & $0$ \\
        \hline
        $y$ & $0$ & $1$
    \end{tabular}            
    \end{center}
Như vậy, phương trình đã cho có hai nghiệm nguyên, bao gồm $(0,1)$ và $(1,0)$.}
\end{gbtt}

\begin{gbtt}
Giải phương trình nghiệm nguyên dương
\[\dfrac{1}{x}+\dfrac{1}{y}+\dfrac{1}{6xy} = \dfrac{1}{6}.\]
\loigiai{
Phương trình đã cho tương đương với
$$6y+6x+1=xy\Leftrightarrow xy-6x-6y+36=37\Leftrightarrow (x-6)(y-6)=37.$$
Ta có $x-6$ là ước của $37,$ nhưng do $x-6\ge 5$ nên $x-6\in \{1;37\}.$
\begin{enumerate}
    \item Với $x-6=1,$ ta tìm ra $x=7$ và $y=43.$
    \item Với $x-6=37,$ ta tìm ra $x=43$ và $y=7.$    
\end{enumerate}
Như vậy, phương trình đã cho có hai nghiệm nguyên $(x,y)$ là $(7,43)$ và $(43,7).$}
\end{gbtt}

\begin{gbtt}
Tìm tất cả các bộ ba số nguyên dương $(x,y,z)$ thỏa mãn đồng thời các điều kiện $$\sqrt{xy}+\sqrt{xz}-\sqrt{yz}=y,\quad  \dfrac{1}{x}+\dfrac{1}{y}-\dfrac{1}{z}=1.$$
\nguon{Chuyên Tin Thanh Hóa 2021}
\loigiai{
Điều kiện thứ nhất tương đương với
    $$\left(\sqrt{y}+\sqrt{z}\right)\left(\sqrt{y}-\sqrt{x}\right)=0\Leftrightarrow x=y.$$
    Thay vào điều kiện còn lại, ta được
        \begin{align*}
          \dfrac{2}{x}-\dfrac{1}{z}=1&\Leftrightarrow 2z-x=xz
          \\&\Leftrightarrow xz+x-2z-2=-2\\&\Leftrightarrow x(z+1)-2(z+1)=-2
          \\&\Leftrightarrow (2-x)(z+1)=2.
        \end{align*}
    Từ đây, ta thu được $x=1,z=1,y=1.$ Kết luận $(x,y,z)=(1,1,1)$ là bộ số duy nhất thỏa mãn đề bài.}
\end{gbtt}

\begin{gbtt}
Cho hình lăng trụ đứng, đáy là tam giác vuông, chiều cao bằng $6.$ Số đo ba cạnh của tam giác đáy là các số nguyên. Số đo diện tích toàn phần của lăng trụ bằng số đo thể tích của lăng trụ. Tính số đo ba cạnh tam giác đáy của lăng trụ.
\nguon{Chuyên Toán Quảng Ninh 2021}
\loigiai{Gọi số đo ba cạnh của tam giác đáy là $a,b,c$ với $a,b,c\in\mathbb Z^+$ và $c>b\ge a.$ \\
Tam giác đáy là tam giác vuông, nên theo định lý $Pythagoras,$ ta có 
\[a^2+b^2=c^2. \tag{1} \label{hl1}\] 
Số đo thể tích của lăng trụ bằng $3ab,$ trong khi số đo diện tích của nó bằng $6(a+b+c)+ab.$ Vậy nên
\[ 6(a+b+c)+ab=3ab \Leftrightarrow 3(a+b+c)=ab\Leftrightarrow ab-3a-3b=3c.\tag{2}\label{hl2} \]
Nhân đôi hai vế của (\ref{hl2}) rồi cộng tương ứng vế với (\ref{hl1}), ta được
\begin{align*}
    a^2+b^2+2ab-6a-6b=c^2+6c
    &\Leftrightarrow (a+b)^2-6(a+b)=c^2+6c
    \\&\Leftrightarrow (a+b)^2-6(a+b)+9=c^2+6c+9
    \\&\Leftrightarrow (a+b-3)^2=(c+3)^2
    \\&\Leftrightarrow \hoac{
         a+b-3&=c+3  \\
         a+b-3&=-c-3}
    \\&\Leftrightarrow \hoac{
         c&=a+b-6  \\
         c&=-a-b.}
\end{align*}
Do $a,b,c$ nguyên dương nên ta loại trường hợp $a+b+c=0,$ tức là $c=a+b-6.$ Thế vào (\ref{hl2}), ta được
$$ab-3a-3b=3a+3b-12\Leftrightarrow(a-6)(b-6)=6.$$
Giải phương trình ước số trên rồi thử lại, ta tìm ra có $18$ bộ số đo $3$ cạnh tam giác đáy, bao gồm $$(7,24,25),\ (8,15,17),\ (9,12,15)$$ và các hoán vị của chúng.}
\end{gbtt}

\begin{gbtt}
Tìm tất cả các số nguyên dương $x$ sao cho $x^2-x+13$ là số chính phương.
\nguon{Chuyên Tin Bình Định 2021}
\loigiai{
Từ giả thiết, ta có thể đặt $x^2-x+13=t^2,$ với $t$ nguyên dương. Phép đặt này cho ta
$$4x^2-4x+52=4t^2\Rightarrow (2x-1)^2+51=(2t)^2\Rightarrow (2t-2x+1)(2t+2x-1)=51.$$
Do $x>0$ nên $0<2t-2x+1<2t+2x-1.$ Từ đó, ta lập được bảng giá trị sau.
    \begin{center}
        \begin{tabular}{c|c|c}
        $2t-2x+1$ & $1$ & $3$   \\
        \hline
        $2t+2x-1$ & $51$ & $17$ \\
        \hline
        $x$ & $13$ & $4$  \
        \end{tabular}
    \end{center}
Căn cứ vào bảng, ta kết luận $x=4$ và $x=13$ là các giá trị nguyên dương thỏa mãn yêu cầu.}
\end{gbtt}

\begin{gbtt}
Tìm tất cả các số nguyên dương $n$ sao cho hai số $n^2-2n-7$ và $n^2-2n+12$ đều là lập phương của một số nguyên dương nào đó.
\nguon{Chuyên Toán Quảng Bình 2021}
\loigiai{
Giả sử tồn tại số nguyên dương $n$ thỏa yêu cầu. Ta đặt
$$n^2-2n+12=a^3,\quad n^2-2n-7=b^3,$$
ở đây $a,b$ là các số nguyên dương thỏa $a>b.$ Lấy hiệu theo vế, ta được
$$19=a^3-b^3\Rightarrow (a-b)\left(a^2+ab+b^2\right)=19.$$
Do $0<a-b<a^2+ab+b^2$ và $19$ là số nguyên tố, ta nhận được $a-b=1$ và $a^2+ab+b^2=19.$\\
Thế $b=a-1$ vào $a^2+ab+b^2=19,$ ta được
$$a^2+a(a-1)+(a-1)^2=19\Rightarrow 3a^2-3a-18=0\Rightarrow3(a-3)(a+2)=0\Rightarrow a=3.$$
Thay trở lại $a=3,$ ta tìm ra $n=5.$ Đây là số nguyên dương duy nhất thỏa mãn yêu cầu.
}
\end{gbtt}

\begin{gbtt}
Giải phương trình nghiệm nguyên $x^2-2x+2y=2(xy+1).$
\nguon{Chuyên Toán Lào Cai 2021}
\loigiai{
Phương trình đã cho tương đương
    $$x^2-2x-2=2y(x-1)\Leftrightarrow(x-1)^2-3=2y(x-1)\Leftrightarrow(x-1)(x-2y-1)=3.$$
    Tới đây, ta lập được bảng giá trị
    \begin{center}
        \begin{tabular}{c|c|c|c|c}
            $x-1$ & $-3$ & $-1$ & $1$ & $3$ \\
        \hline
            $x$ & $-2$ & $0$ & $2$ & $4$\\
        \hline
            $y$ & $-1$ & $1$ & $-1$ & $1$\\
        \end{tabular}
    \end{center}
Kết quả, phương trình đã cho có $4$ nghiệm nguyên là $(-2,-1),(0,1),(2,-1),(4,-1).$}
\end{gbtt}

\begin{gbtt}
Giải phương trình nghiệm nguyên $(2x+y)(x-y)+3(2x+y)-5(x-y)=22.$ 
\nguon{Chuyên Toán Bình Phước 2021}
\loigiai{Phương trình đã cho tương đương với
    \begin{align*}
        (2x+y)(x-y)+3(2x+y)-5(x-y)-15=7
        \Leftrightarrow (2x+y-5)(x-y+3)=7.
    \end{align*}
    Đến đây, ta lập bảng giá trị
          \begin{center}
\begin{tabular}{c|c|c|c|c}
$2x+y-5$ & $-7$ & $-1$ & $1$ & $7$ \\
\hline
$x-y+3$ & $-1$ & $-7$ & $7$ & $1$ \\
\hline
$x$ & $-2$ & $-2$ & $\not\in\mathbb{Z}$ & $\not\in\mathbb{Z}$ \\
\hline
$y$ & $2$ & $8$ & $\not\in\mathbb{Z}$ & $\not\in\mathbb{Z}$ \\
\end{tabular}
\end{center}
Kết quả, phương trình đã cho có $2$ nghiệm nguyên phân biệt $(-2,2)$ và $(-2,8).$}
\end{gbtt}

\begin{gbtt}
Tìm tất cả các số nguyên $x, y$ thỏa mãn $x^2-xy-2y^2+x+y-5=0.$
\nguon{Chuyên Tin Hà Nội 2021}
\loigiai{
Giả sử tồn tại cặp $(x,y)$ thỏa yêu cầu. Ta có
    $$(x+y)(x-2y)+(x+y)-5=0\Leftrightarrow(x+y)(x-2 y+1)=5.$$
Căn cứ vào đây, ta lập được bảng giá trị
    \begin{center}
    \begin{tabular}{c|c|c|c|c}
         $x+y$ & $-5$ & $-1$ & $1$ & $5$   \\
         \hline
         $x-2y+1$  & $-1$ & $-5$ & $5$ & $1$ \\ 
         \hline
         $x$ & $-4$ & $\not\in\mathbb{Z}$ & $2$ & $\not\in\mathbb{Z}$   \\
         \hline
         $y$ & $-1$ & $\not\in\mathbb{Z}$ & $-1$ & $\not\in\mathbb{Z}$   \\         
    \end{tabular}        
    \end{center}
Như vậy, có tổng cộng hai cặp $(x, y)$ thỏa mãn đề bài, bao gồm $(-4,-1)$ và $(2,-1).$}
\end{gbtt}

\begin{gbtt}
Tìm tất cả các cặp số nguyên $(x,y)$ thỏa mãn
$$2 x^{2}-x y+9 x-3 y+4=0.$$
\nguon{Chuyên Toán Lạng Sơn 2021}
\loigiai{Phương trình đã cho tương đương với
    \begin{align*}
        2x^2+9x+4=(x+3)y
        &\Leftrightarrow(x+3)(2x+3)-5=(x+3)y
        \\&\Leftrightarrow (x+3)(2x-y+3)=5.
    \end{align*}
    Căn cứ vào biến đổi kể trên, ta lập được bảng giá trị
    \begin{center}
        \begin{tabular}{c|c|c|c}
           $x+3$  & $2x-y+3$ & $x$ & $y$  \\
           \hline
           $-5$ & $-1$ & $-8$ & $-12$ \\
           $-1$ & $-5$ & $-4$ & $0$ \\
           $1$ & $5$ & $-2$ & $-6$ \\
           $5$ & $1$ & $2$ & $6$ 
        \end{tabular}
    \end{center}
    Kết luận, có tất cả $4$ cặp $(x,y)$ thỏa yêu cầu, bao gồm
    \[(-8,-12),\ (-4,0),\ (-2,-6),\ (2,6).\]}
\end{gbtt}


\begin{gbtt}
Giải phương trình nghiệm nguyên $y^2+3y=x^4+x^2+18.$
\nguon{Chuyên Toán Ninh Thuận 2021}
\loigiai{
Phương trình đã cho tương đương với
\begin{align*}
    4y^2+12y=4x^4+4x^2+72
    &\Leftrightarrow 4y^2+12y+9=4x^4+4x^2+1+80
    \\&\Leftrightarrow (2y+3)^2=\left(2x^2+1\right)^2+80
    \\&\Leftrightarrow \left(2y-2x^2+2\right)\left(2y+2x^2+4\right)=80
    \\&\Leftrightarrow \left(y-x^2+1\right)\left(y+x^2+2\right)=20.
\end{align*}
    Ta có các đánh giá.
\begin{enumerate}[i,]
        \item $y-x^2+1<y+x^2+2.$ 
        \item $y-x^2+1$ và $y+x^2+2$ khác tính chẵn lẻ.
\end{enumerate}
    Dựa vào đây, ta lập được bảng giá trị sau
\begin{center}
\begin{tabular}{c|c|c|c}

$y-x^2+1$ & $y+x^2+2$ & $y$ & $x$  \\ 
\hline
$-20$ & $-1$ & $-12$ & $\pm 3$ \\
\hline
$-5$ & $-4$ & $-6$ & $0$ \\
\hline
$1$ & $20$ & $9$ & $\pm 3$ \\
\hline
$4$ & $5$ & $3$ & $0$ 
\end{tabular}
\end{center}
Kết quả, phương trình đã cho có $6$ nghiệm $(x,y)$ phân biệt, bao gồm $$(-3,-12),(-3,9),(0,-6),(0,3),(3,9),(3,-12).$$}
\end{gbtt} %ninhthuan

\begin{gbtt}
Tìm tất cả các dãy số tự nhiên chẵn liên tiếp có tổng bằng $2010.$
\nguon{Chuyên Quốc Học Huế 2010 $-$ 2011}
\loigiai{
Gọi $2x$ là số tự nhiên chẵn đầu tiên của dãy. Khi đó theo giả thiết ta có
\begin{align*}
    2x+\left( 2x+2 \right)+\left( 2x+4 \right)+\cdots+\left( 2x+2y \right)=2010 &\Leftrightarrow x+\left( x+1 \right)+\left( x+2 \right)+\cdots+\left( x+y \right)=1005\\
 &\Leftrightarrow \left( y+1 \right)x+1+2+\cdots+y=1005 \\ 
 &\Leftrightarrow \left( y+1 \right)x+\dfrac{y\left( y+1 \right)}{2}=1005\\
 &\Leftrightarrow \left( y+1 \right)\left( 2x+y \right)=2010. 
\end{align*}
Suy ra $\left( y+1 \right)$ là ước số của $2010=1\cdot2\cdot3\cdot5\cdot67.$
Điều này dẫn tới
$$\left( y+1 \right)\in \left\{ 2;3;5;6;10;15;30;67;134;201;335;402;670;1005;2010 \right\}.$$
Ta lập bảng giá trị dưới đây.
\begin{center}
    \begin{tabular}{c|c|c|l}
       $y+1$  &  $y$ & $2x$ & Dãy số thu được\\
        \hline
        $2$ & $1$ & $1004$ & $1004,1006$\\

        $3$ & $2$ & $668$ & $668, 670, 672$\\

        $5$ & $4$ & $398$ & $398, 400, 402, 404, 406$\\

        $6$ & $5$ & $330$ & $330, 332, 334, 336, 338, 340$\\

        $10$ & $9$ & $192$ & $192, 194, 1948, \ldots, 210$\\

        $15$ & $14$ & $120$ & $120, 122,\ldots, 148$\\

        $30$ & $29$ & $38$ & $38, 40,\ldots, 96$\\

        $\ge 67$ & $\ge 66$ & $<0$ & Không tồn tại
    \end{tabular}
\end{center}
Các dãy thu được trong bảng chính là đáp số của bài toán.}
\end{gbtt}

\begin{gbtt}
Tìm tất cả các nghiệm nguyên của phương trình $$x^{2}-y^{2}\left(x+y^{4}+6 y^{2}\right)=0.$$
\nguon{Chuyên Toán Bắc Giang 2021}
\loigiai{Phương trình đã cho tương đương với
\[x^2-xy^2=y^6+6y^4\Leftrightarrow 4x^2-4xy^2+y^4=4y^6+25y^4\Leftrightarrow\tron{2x-y^2}^2=y^4\tron{4y^2+25}.\tag{*}\label{bgian2021}\]
    Tới đây, ta xét các trường hợp sau.
\begin{enumerate}
        \item Nếu $2x=y^2,$ thế $2x-y^2=0$ trở lại (\ref{bgian2021}) ta được
        $$y^4\tron{4y^2+25}=0.$$
        Ta tìm ra $y=0$ từ đây. Trường hợp này cho ta $(x,y)=(0,0).$
        \item Nếu $2x\ne y^2$ thì $4y^2+25$ là số chính phương. Đặt $4y^2+25=z^2,$ trong đó $z$ nguyên dương. Ta có
        $$z-2y)(z+2y)=25.$$
        Do $z-2y\le z+2y$ nên ta lập được bảng giá trị
        \begin{center}
            \begin{tabular}{c|c|c|c}
                $z-2y$ &  $z+2y$ & $y$ & $x$ \\
                \hline
                $-25$ & $-1$ & $-6$ & $252$ hoặc $-216$ \\
                $-5$ & $-5$ & $0$ & $0$ \\      $5$ & $5$ & $0$ & $0$ \\    
                $1$ & $25$ & $6$ & $252$ hoặc $-216$ \\
            \end{tabular}
        \end{center}
\end{enumerate}
    Kiểm tra trực tiếp từng trường hợp, ta nhận thấy rằng phương trình đã cho có tất cả $5$ nghiệm nguyên là
    \[(-216,-6),(-216,6),(0,0),(252,-6),(252,6).\]}
\end{gbtt}

\begin{gbtt}
Giải phương trình nghiệm nguyên \[x^2+xy+y^2=\tron{\dfrac{x+y}{3}+1}^3.\]
\loigiai{
Dễ thấy $x+y$ chia hết cho $3.$ Ta đặt $$u=\dfrac{x+y}{3},\quad v=x-y,\text{ trong đó }u,v\text{ là các số nguyên}.$$
Để ý thấy $x^2+xy+y^2=\dfrac{3(x+y)^2+(x-y)^2}{4}.$ Phương trình đã cho trở thành 
\[\dfrac{9u^2+v^2}{4}=\tron{u+1}^3
\Leftrightarrow v^2=(u-2)^2(4u+1).\]
Tới đây, ta xét các trường hợp sau.
\begin{enumerate}
    \item Nếu $u=2,$ ta có $v=0.$ Từ đây ta tìm được $x=y=3.$
    \item Nếu $u\ne 2,$ ta có $4u+1$ là số chính phương lẻ. Ta đặt
    $$4u+1=(2k+1)^2,\text{ với }k\text{ là số tự nhiên}.$$
    Lúc này $u=k^2+k.$ Phương trình đã cho trở thành
    $$v^2=(k^2+k-2)^2(2k+1)^2.$$
    Ta xét các trường hợp nhỏ hơn sau.
    \begin{itemize}
        \item\chu{Trường hợp 1.} Với $v=(k^2+k-2)(2k+1)$, ta tìm được $$(x,y)=(k^3+3k^2-1,-k^3+3k+1).$$
        \item\chu{Trường hợp 2.} Với $v=-(k^2+k-2)(2k+1)$, ta tìm được $$(x,y)=(-k^3+3k+1,k^3+3k^2-1).$$
    \end{itemize}
\end{enumerate}
Tổng hợp lại, tất cả nghiệm $(x,y)$ của phương trình là $$(3,3),\quad (-k^3+3k+1,k^3+3k^2-1),\quad (k^3+3k^2-1,-k^3+3k+1),$$
với $k$ là số tự nhiên tùy ý.}
\end{gbtt}

\begin{gbtt}
Tìm các số nguyên dương $a, b, c, d$  thỏa mãn đồng thời các điều kiện
\[{a^2} = {b^3},\quad {c^3} = {d^4},\quad a = d + 98.\]
\nguon{Chuyên Đại học Sư Phạm Hà Nội 2017 $-$ 2018}
\loigiai{
Ta có $b=\tron{\dfrac{a}{b}}^2,$ thế nên $a$ chia hết cho $b,$ và kéo theo $b$ là số chính phương. Đặt $b=z^2,$ khi đó
$$a=\sqrt{b^3}=\sqrt{z^6}=B^3.$$
Ta suy ra $a$ là số lập phương. Chứng minh tương tự, ta có $d$ là số lập phương. Đặt $a=x^3,d=y^3,$ ta có
\[{x^3} = {y^3} + 98 \Leftrightarrow \left( {x - y} \right)\left( {{x^2} + xy + {y^2}} \right) = 98.\]
Do $a>d$ nên ta suy ra được $x-y>0$, như vậy
\[x^2+xy+y^2>x^2-2xy+y^2=(x-y)^2\ge x-y.\]
Đến đây ta xét 2 trường hợp sau.
\begin{enumerate}
    \item  Nếu $x-y=1$ và $x^2+xy+y^2=98,$ ta có hệ
\[\left\{ \begin{gathered}
  x = y + 1 \hfill \\
  {\left( {y + 1} \right)^2} + \left( {y + 1} \right)y + {y^2} = 98 \hfill \\ 
\end{gathered}  \right. \Leftrightarrow \heva{
  &x = y + 1 \hfill \\
  &3{y^2} + 3y - 97 = 0.}\]
 Phương trình $3y^2+3y=97$ không có nghiệm nguyên do vế phải không chia hết cho $3.$ 
    \item  Nếu $x-y=2$ và $x^2+xy+y^2=49,$ ta có hệ
\[
\begin{aligned}
\left\{ \begin{gathered}
  x = y + 2 \hfill \\
  {\left( {y + 2} \right)^2} + \left( {y + 2} \right)y + {y^2} = 49 \hfill \\ 
\end{gathered}  \right. &\Leftrightarrow \left\{ \begin{gathered}
  x = y + 2 \hfill \\
  {y^2} + 2y - 15 = 0\hfill \\ 
\end{gathered}  \right.
\\&\Leftrightarrow\heva{&x=y+2 \\ &(y+3)(y-5)=0}
\\&\Leftrightarrow
\heva{
  &(x,y)=(5,3)\hfill \\
  &(x,y)=(-3,-5).\hfill}
\end{aligned}\]
Đối chiếu với điều kiện $x,y$ nguyên dương, ta tìm ra $x=5$ và $y=3.$\\
Vậy từ đó ta tính được $a = {5^3} = 125,\: d = {3^3} = 27,\: b = 25,\: c = 81.$
\end{enumerate}
Kết luân, có duy nhất một bộ số $(a,b,c,d)$ thỏa mãn yêu cầu bài toán, đó là
$$(a,b,c,d)=(125,25,81,27).$$}
\end{gbtt}

\begin{gbtt}
Giải phương trình nghiệm nguyên $(xy-1)^2=x^2+y^2.$
\nguon{Chuyên Toán Bà Rịa $-$ Vũng Tàu 2021}
\loigiai{
Phương trình đã cho tương đương 
\begin{align*}
    \left(xy\right)^2+1=x^2+y^2+2xy
    \Leftrightarrow \left(xy\right)^2+1=(x+y)^2
    \Leftrightarrow 1=(x+y-xy)(x+y+xy).  
\end{align*}
Tới đây, ta xét các trường hợp sau.
\begin{enumerate}
    \item Với $x+y-xy=x+y+xy=1,$ ta có $xy=0$ và $x+y=1.$ Trường hợp này cho ta $$(x,y)=(0,1),\quad (x,y)=(1,0).$$
    \item Với $x+y-xy=x+y+xy=-1,$ ta có $xy=0$ và $x+y=-1.$ Trường hợp này cho ta $$(x,y)=(0,-1),\quad (x,y)=(-1,0).$$
\end{enumerate}
Như vậy, phương trình đã cho có $4$ nghiệm nguyên phân biệt, bao gồm
$$(-1,0),\ (0,-1),\ (0,1),\ (1,0).$$}
\end{gbtt}

\begin{gbtt}
Giải phương trình nghiệm nguyên
$$x^{2}(y - 1) + y^{2}(x-1) = 1.$$
\nguon{Polish Mathematical Olympiad 2004}
\loigiai{Phương trình đã cho tương đương với
\begin{align*}
    xy(x + y) - (x^{2} + y^{2}) = 1 &\Leftrightarrow xy(x + y) - (x + y)^{2} + 2xy = 1
    \\&\Leftrightarrow xy(x +y +2) = (x + y)^{2} - 4 + 5 
    \\&\Leftrightarrow
    (x+y+2)(xy-x-y+2)=5.
\end{align*}
Từ đây, ta lập bảng giá trị sau
\begin{center}
    \begin{tabular}{c|c|c|c|c}
       $x+y+2$  & $1$  &$5$  &$-1$  &$-5$  \\
       \hline
    $xy-x-y+2$ & $5$  &$1$  &$-5$  &$-1$\\
    \hline
    $x+y$ &$-1$&$3$&$-3$&$-7$\\
    \hline
    $xy$& $4$  &$2$ &$-10$ & $-10$
    \end{tabular}
\end{center}
Bằng cách lập bảng giá trị tương ứng, ta kết luận phương trình đã cho có $4$ nghiệm nguyên là
\[(-5,2),( 2,-5),(1,2),(2,1).\]}
\end{gbtt}

\begin{gbtt}
Giải phương trình nghiệm nguyên $x^2-xy+y^2=x^2y^2-5.$
\nguon{Chuyên Khoa học Tự nhiên 2015}
\loigiai{
Biến đổi phương trình đã cho, ta được
\begin{align*}
    x^2-xy+y^2=x^2y^2-5&\Leftrightarrow4x^2-4xy+4y^2=4x^2y^2-20\\
    &\Leftrightarrow \tron{2x-2y}^2=\tron{2xy-1}^2-21\\
    &\Leftrightarrow\tron{2xy-2x+2y-1}\tron{2xy+2x-2y-1}=21.
\end{align*}
Từ đây, ta lập được bảng giá trị sau.

\begin{center}
    \begin{tabular}{c|c|c|c|c|c|c|c|c}
         $2xy-2x+2y-1$ & $1$   & $21$  & $-1$   &$-21$ &$3$ & $7$&$-3$&$-7$\\
         \hline
         $2xy+2x-2y-1$  & $21$  & $1$  &$-21$  &$-1$&$7$&$3$&$-7$&$-3$\\
         \hline
         $xy$  & $6$   & $6$  &$-5$  &$-5$&$3$&$3$&$-2$&$-2$\\
         \hline
         $x-y$ &$5$ &$-5$  &$-5$ &$5$&$1$&$-1$&$-1$&$1$ 
    \end{tabular}
\end{center}
Không mất tính tổng quát, ta giả sử $x-y\ge0$. Giả sử này cho phép ta xét các trường hợp sau.
\begin{enumerate}
    \item Với $\tron{-xy,x-y}=\tron{-6,5}$, ta có $x,-y$ là nghiệm nguyên của phương trình $$X^2-5X-6=0.$$ Vì $X^2-5X-6=0$ có nghiệm là $\tron{-1,6}$, ta suy ra  các cặp $\tron{x,y}$ là $\tron{-1,-6}, \tron{6,1}$.
    \item Với $\tron{-xy,x-y}=\tron{5,5},$ ta có $x,-y$ là nghiệm của phương trình $$X^2-5X+5=0.$$ Phương trình trên không có nghiệm nguyên.
    \item Với $\tron{-xy,x-y}=\tron{-3,1}$ thì $x,-y$ là nghiệm của phương trình $$X^2-X-3=0.$$ Phương trình trên không có nghiệm nguyên.
     \item Với $\tron{-xy,x-y}=\tron{2,1}$, ta có $x,-y$ là nghiệm nguyên của phương trình $$X^2-X+2=0.$$ Phương trình trên không có nghiệm nguyên.
\end{enumerate}
Như vậy, phương trình đã cho có các nghiệm nguyên $\tron{x,y}$ là  $\tron{-1,-6}, \tron{6,1}$ và hoán vị của chúng.}
\end{gbtt}

\begin{gbtt}
Giải phương trình nghiệm nguyên $x^3-y^3=xy+25.$
\loigiai{
\begin{enumerate}[\color{tuancolor}\sffamily\bfseries Cách 1.]
\item Phương trình đã cho tương đương với
\begin{align*}
    27x^3-27y^3=27xy+675
    &\Leftrightarrow
    (3x)^3-(3y)^3-1-3\cdot3x\cdot(-3y)\cdot(-1)=674
    \\&\Leftrightarrow (3x-3y-1)\left(9x^2+9y^2+1+9xy-3y-3x\right)=674.
\end{align*}
Để cho tiện, ta đặt $3x=z,3y=t.$ Phương trình đã cho trở thành
$$(z-t-1)\left(z^2+t^2+1+zx-z-t\right)=674.$$
Lập luận tương tự như \chu{bài toán \ref{hdtbacbar}}, ta thu được bảng giá trị sau đây.
\begin{center}
    \begin{tabular}{c|c|c|c|c}
         $z-t-1$ & $1$ & $2$ & $337$ & $674$ \\
         \hline
         $z^2+t^2+1+zx-z-t$ & $674$ & $337$ & $2$ & $1$ 
    \end{tabular}
\end{center}
Giải các hệ phương trình thu được trong từng trường hợp, ta chỉ ra phương trình đã cho có hai nghiệm nguyên, đó là $(-3,-4)$ và $(4,3).$ 
\item Phương trình đã cho tương đương
        $$x^3-y^3=xy+25 \Leftrightarrow (x-y)^3+3xy(x-y)=xy+25.$$
Ta đặt $x-y=z,xy=t$, với $z,t$ là các số nguyên thỏa mãn $z^2+4t\ge 0.$ Phương trình trở thành
        $$z^3+3zt=t+25 \Leftrightarrow z^3-25=t(1-3z).$$
Ta lần lượt suy ra
\begin{align*}
   (3z-1)\mid\left(z^3-35\right)&\Rightarrow (3z-1)\mid 27\left(z^3-35\right)\\&\Rightarrow (3z-1)\mid \left(27z^3-1\right)+674\\&\Rightarrow (3z-1)\mid 674 .
\end{align*}
Ta chỉ ra được rằng $3z-1$ là các ước chia cho $3$ dư $2$ của $674.$ \\
Nói cách khác, $3z-1$ nhận một trong các giá trị $-337,-1,2,674.$
\begin{itemize}
    \item Với $3z-1=-337$ hay $z=-112,$ ta có $t=4169,$ mâu thuẫn do $z^2+4t<0.$
    \item Với $3z-1=-1$ hay $z=0,$ ta có $t=-25,$ mâu thuẫn do $z^2+4t<0.$
    \item Với $3z-1=2$ hay $z=1,$ ta có $t=12,$ và ta tìm ra $x=4,y=3$ hoặc $x=-3,y=-4.$
    \item Với $3z-1=674$ hay $z=225,$ ta có $t=-16900,$ mâu thuẫn do $z^2+4t<0.$ 
\end{itemize}
Kết luận, phương trình đã cho có hai nghiệm nguyên, đó là $(-3,-4)$ và $(4,3).$
\end{enumerate}}
\end{gbtt}

\begin{gbtt}
Giải phương trình nghiệm nguyên dương $x^2y^2(y-x)=5xy^2-27.$
\nguon{Chuyên Toán Nam Định 2021}
\loigiai{
Phương trình đã cho tương đương với
    \[xy\left(5y-xy^2+x^2y\right)=27.\tag{*}\label{nd1}\]
Biến đổi trên chứng tỏ $xy$ là ước nguyên dương của $27.$ Ta lần lượt xét các trường hợp sau.
\begin{enumerate}
    \item Với $xy=1,$ ta có $x=y=1.$ Đối chiếu với (\ref{nd1}), ta thấy không thỏa.
    \item Với $xy=3,$ thay trở lại (\ref{nd1}), ta được
    $$5y-3y+3x=9\Leftrightarrow 3x+2y=9.$$
    Lần lượt kiểm tra với $(x,y)=(1,3),(3,1)$ ta thấy chỉ có $(x,y)=(1,3)$ thỏa mãn.
    \item Với $xy=9,$ thay trở lại (\ref{nd1}), ta được
    $$5y-9y+9x=3\Leftrightarrow 9x-4y=3.$$
    Lần lượt kiểm tra với $(x,y)=(1,9),(3,3),(9,1),$ ta thấy chúng đều không thỏa mãn.
    \item  Với $xy=27,$ thay trở lại (\ref{nd1}), ta được
    $$5y-27y+27x=1\Leftrightarrow 27x-22y=3.$$
    Lần lượt kiểm tra với $(x,y)=(1,27),(3,9),(9,3),(27,1),$ ta thấy chúng đều không thỏa mãn.    
\end{enumerate}
Kết luận, $(x,y)=(1,3)$ là nghiệm duy nhất của phương trình.}
\end{gbtt}
\begin{gbtt}
 Tìm các cặp số $\left( x,y \right)$ nguyên dương thoả mãn phương trình
\[{{\left( {{x}^{2}}+4{{y}^{2}}+28 \right)}^{2}}~-\,17\left( {{x}^{4}}+{{y}^{4}} \right)=238{{y}^{2}}+833.\]
\loigiai{
Phương trình đã cho tương đương với
\begin{align*}
    \left( {x}^{2}+4{y}^{2}+28\right)^{2}-17\left( {{x}^{4}}+{{y}^{4}} \right)=238{{y}^{2}}+833 
  &\Leftrightarrow {{\left[ {{x}^{2}}+4\left( {{y}^{2}}+7 \right) \right]}^{2}}=17\left[ {{x}^{4}}+{{\left( {{y}^{2}}+7 \right)}^{2}} \right]
  \\&\Leftrightarrow 16{{x}^{4}}-8{{x}^{2}}\left( {{y}^{2}}+7 \right)+{{\left( {{y}^{2}}+7 \right)}^{2}}=0 
  \\ &\Leftrightarrow {{\left[ 4{{x}^{2}}-\left( {{y}^{2}}+7 \right) \right]}^{2}}=0
  \\&\Leftrightarrow 4{{x}^{2}}-{{y}^{2}}-7=0
  \\&\Leftrightarrow \left( 2x+y \right)\left( 2x-y \right)=7. 
            \end{align*}
Vì $x$ và $y$ là các số nguyên dương nên $2x+y>2x-y$ và $2x+y>0$.\\
Do đó từ phương trình trên ta suy ra được $$\heva{  2x+y&=7  \\
   2x-y&=1}\Leftrightarrow \heva{ x&=2  \\
   y&=3.}$$
Vậy phương trình trên có nghiệm nguyên dương là $\left( x,y \right)=\left( 2,3 \right)$.}
\end{gbtt}
\begin{gbtt}
Giải phương trình nghiệm nguyên $x^2y-xy+2x-1=y^2-xy^2-2y.$
\nguon{Chuyên Toán Bến Tre 2021}
\loigiai{
Phương trình đã cho tương đương với
\begin{align*}
    x^2y+xy^2-xy-y^2+2x+2y=1
    &\Leftrightarrow xy(x+y)-y(x+y)+2(x+y)=1
    \\&\Leftrightarrow (xy-y+2)(x+y)=1.
\end{align*}
Tới đây, ta xét các trường hợp sau.
\begin{enumerate}
    \item Với $x+y=xy-y+2=-1,$ ta có hệ phương trình
    \begin{align*}
        \heva{&x+y=-1 \\ &xy-y+2=-1}
    &\Leftrightarrow \heva{&y=-1-x \\ &x(-1-x)-(-1-x)+2=-1}
    \\&\Leftrightarrow \heva{&y=-1-x \\ &(x+2)(x-2)=0}
    \\&\Leftrightarrow\left[\begin{aligned}
         (x,y)&=(-2,1) \\
         (x,y)&=(2,-3).
    \end{aligned}\right.
    \end{align*}
    \item Với $x+y=xy-y+2=1,$ ta có hệ phương trình
    \begin{align*}
        \heva{&x+y=1 \\ &xy-y+2=1}
    &\Leftrightarrow \heva{&y=1-x \\ &x(1-x)-(1-x)+2=1}
    \\&\Leftrightarrow \heva{&y=1-x \\ &x(x-2)=0}
    \\&\Leftrightarrow\left[\begin{aligned}
         (x,y)&=(0,1) \\
         (x,y)&=(2,-1).
    \end{aligned}\right.
    \end{align*}
\end{enumerate}
Như vậy, phương trình đã cho có $4$ nghiệm nguyên phân biệt, bao gồm
$$(-2,1),(2,-3),(0,1),(2,-1).$$}
\end{gbtt}

\begin{gbtt}
Giải phương trình nghiệm nguyên $x^3y-x^3-1=2x^2+2x+y.$
\nguon{Chuyên Toán Kon Tum 2021}
\loigiai{
Phương trình đã cho tương đương với $$y(x-1)(x^2+x+1)=(x+1)(x^2+x+1).$$
Do $x^2+x+1=\left(x+\dfrac{1}{2}\right)^2+\dfrac{3}{4}>0$ nên phương trình trên tương đương
$$y(x-1)=x+1\Leftrightarrow y(x-1)=x-1+2\Leftrightarrow (y-1)(x-1)=2.$$ 
Tới đây, ta lập được bảng giá trị
\begin{center}
\begin{tabular}{c|c|c|c|c}
    $x-1$ & $1$ & $2$ & $-1$ & $-2$ \\
    \hline
    $y-1$ & $2$ & $1$ & $-2$ & $-1$  \\
    \hline
    $x$ & $2$ & $3$ & $0$ & $-1$ \\
    \hline
    $y$ & $3$ & $2$ & $-1$ & $0$
\end{tabular}
\end{center}
Như vậy, phương trình đã cho có $4$ nghiệm nguyên là $(2,3),(0,-1),(3,2),(-1,0).$}
\end{gbtt}

\begin{gbtt}
Giải phương trình nghiệm nguyên $(x+2)^2(y-2)+xy^2+26=0.$
\loigiai{Phương trình đã cho tương đương với
$$x^2y+4xy+4y-2x^2-8x+xy^2+18=0\Leftrightarrow\tron{x+y+6}\tron{xy-2x+4}=6.$$
Từ đây, ta suy ra $\tron{x+y+6}$ là ước của $6$.\\
Giải các trường hợp trên, ta kết luận phương trình đã cho có $4$ cặp nghiệm nguyên $\tron{x,y}$ là $$\tron{1,-1},\tron{3,-3}, \tron{-10,3},\tron{1,-8}.$$}
\end{gbtt}

\begin{gbtt}
Giải phương trình nghiệm nguyên 
\[2xy^2+x+y+1=x^2+2y^2+xy.\]
\nguon{Hanoi Open Mathematics Competitions 2015}
\loigiai{
Phương trình đã cho tương đương 
$$\tron{x-1}\tron{x+y-2y^2}=1.$$
Từ đây, ta xét các trường hợp sau
\begin{enumerate}
    \item Với $x-1=x+y-2y^2=1,$ ta có $x=2$ kéo theo $$y-2y^2+1=0.$$ Giải ra, ta được $y=1$ là nghiệm nguyên duy nhất.
    \item Với $x-1=x+y-2y^2=-1,$ ta có $x=0$ kéo theo $$y-2y^2+1=0.$$ Giải ra, ta được $y=1$ là nghiệm nguyên duy nhất.
\end{enumerate}
Như vậy, phương trình có $2$ nghiệm nguyên $(x,y)$ là $(2,1),(0,1).$ }
\end{gbtt}

\begin{gbtt}
Tìm tất cả các số nguyên dương $m,n$ thỏa mãn 
\[m(m, n)+n^{2}[m, n]=m^{2}+n^{3}-330.\]
\loigiai{
Vì $(m, n)=d$ nên $d^{2} \mid 2 \cdot 3 \cdot 5 \cdot 11=330 .$ Suy ra $d=1$ và $[m, n]=m n.$ Thế trở lại phương trình, ta được
$$m+mn^3=m^2+n^3-330\Leftrightarrow
(m-1)\left(m-n^3\right)=330.$$
Ta nhận xét $m-1\ge m-n^3>0$ và $330$ chia hết cho $m-1.$ Ta lập bảng giá trị
\begin{center}
    \begin{tabular}{c|c|c|c|c|c|c|c}
        $m-1$ & $330$ & $165$ & $110$ & $66$ & $33$ & $30$ & $22$ \\
        \hline
        $m-n^3$ & $1$ & $2$ & $3$ & $5$ & $10$ & $11$ & $15$ \\
        \hline
        $m$ & $331$ & $166$ & $111$ & $67$ & $34$ & $31$ & $23$ \\
        \hline
        $n^3$ & ${330}$ & ${164}$ & ${108}$ & ${62}$ & ${24}$ & ${20}$ & ${8}$
    \end{tabular}
\end{center}
Căn cứ vào bảng, ta kết luận phương trình có nghiệm nguyên dương duy nhất là $(m,n)=(23,2).$
}

\end{gbtt}

\begin{gbtt}
Tìm tất cả các số tự nhiên $n$ sao cho số $2^8+2^{11}+2^n$ là số chính phương.
\nguon{Violympic Toán lớp 9}
\loigiai{
Giả sử $2^8+2^{11}+2^n=a^2$, trong đó $a$ là số tự nhiên, khi đó ta có
		\[2^n=a^2-48^2=(a+48)(a-48).\]
Từ đây ta có thể đặt
$a+48=2^p,\: a-48=2^q,\text{ trong đó }p>q.$
Lấy hiệu theo vế, ta được
\[2^p-2^q=96\Leftrightarrow 2^q\tron{2^{p-q}-1}=96.\]
Xét số mũ của $2$ ở cả hai vế, ta chỉ ra $q=5.$ Thay ngược lại ta được $p=7,$ kéo theo $n=12.$\\
Đây chính là đáp số bài toán.}
\end{gbtt}

\begin{gbtt}
Tìm tất cả các số nguyên dương $x,y$ thỏa mãn $$x^2-2^y\cdot x-4^{21}\cdot 9=0.$$
\nguon{Chuyên Toán Thừa Thiên Huế 2021}
\loigiai{Với mỗi số nguyên dương $x,$ luôn tồn tại các số nguyên dương $z$ và số nguyên dương lẻ $t$ sao cho $x=2^z t.$ Bằng cách đặt như vậy, phương trình đã cho trở thành
$$
2^{2z}t^2-2^{y+z}t-9\cdot4^{21}=0.$$
Phương trình kể trên tương đương với
\[2^{2z}t\left(t-2^{y-z}\right)=9\cdot4^{21}.\tag{*}\label{huee}\]
Trong hai vế của (\ref{huee}), ta sẽ xét số mũ của lũy thừa cơ số $2.$ Thật vậy
\begin{itemize}
    \item[i,] Cả $t$ và $t-2^{y-21}$ đều lẻ, thế nên số mũ của $2$ ở vế trái là $2z.$
    \item[ii,] Số mũ của $2$ ở vế phải là $2\cdot 21=42.$
\end{itemize}
Do vậy, $z=21.$ Thay $z=21$ vào (\ref{huee}), ta được $t\left(t-2^{y-21}\right)=9.$ Ta có đánh giá
$$0<t-2^{y-21}<t.$$
Đánh giá trên cho ta $t=8$ và $2^{y-21}=8$, tức $y=24.$\\
Kết luận, $(x,y)=\left(9\cdot2^{21},24\right)$ là cặp số nguyên dương duy nhất thỏa mãn yêu cầu.}
\end{gbtt}


\section{Phép phân tích thành tổng các bình phương}
Khi đưa phương trình nghiệm nguyên về dạng tổng các bình phương, tính bị chặn của các trị tuyệt đối được thể hiện. Từ đó, ta sẽ tìm ra được các nghiệm của phương trình ấy. Dưới đây là một số ví dụ minh họa.

\subsection*{Bài tập tự luyện}

\begin{btt}
Giải phương trình nghiệm nguyên dương $$2x^2+4x=19-3y^2.$$
\end{btt}

\begin{btt}
Giải phương trình nghiệm nguyên
$$4x^2+4x+y^2-6y=24.$$
\end{btt}

\begin{btt}
Giải phương trình nghiệm nguyên $$x^2-2y(x-y)=2(x+1).$$
\nguon{Chuyên Toán Tây Ninh 2021}
\end{btt}

\begin{btt}
Tìm tất cả các số nguyên dương $x,y$ thỏa mãn
    $$x^4-x^2+2x^2y-2xy+2y^2-2y-36=0.$$
\nguon{Chuyên Toán Đắk Lắk 2021}    
\end{btt}

\begin{btt}
Giải phương trình nghiệm nguyên 
\[2x^6-2x^3y+y^2=128.\]
\end{btt}

\begin{btt}
Tìm tất cả các số nguyên tố $a\geqslant b\geqslant c\geqslant d$ thỏa mãn
$$a^2+2b^2+c^2+2d^2=2\left(ab+bc-cd+da\right).$$
\nguon{Titu Andreescu}
\end{btt}

\begin{btt}
Phương trình $x^2+2y^2+2z^2-2xy-2yz-2z=4$ có tất cả bao nhiêu nghiệm nguyên?
\end{btt}

\begin{btt}
Tìm tất cả các bộ số nguyên dương $(x, y, z)$ thỏa mãn
$$5\tron{x^{2}+2 y^{2}+z^{2}}=2(5 x y-y z+4 z x),$$
trong đó, ít nhất một trong ba số $x, y, z$ là số nguyên tố.
\nguon{Adrian Andreescu}
\end{btt}

\begin{btt}
Giải bất phương trình nghiệm nguyên $$5x^2+3y^2+4xy-2x+8y+8\le 0.$$
\nguon{Chuyên Toán Đồng Nai 2021}
\end{btt}

\begin{btt}
Tìm tất cả các số nguyên $x,y,z$ sao cho
\[x^2+y^2+z^2+6<xy+3y+4z.\]
\nguon{Chuyên Toán Nghệ An 2019}
\end{btt}

\begin{btt}
Giải phương trình nghiệm nguyên
\[x^2+xy+y^2=3x+y-1.\]
\end{btt}

\begin{btt}
Giải hệ phương trình nghiệm nguyên
\[\heva{&x+y-z=2\\&3x^2+2y^2-z^2=13.}\]
\end{btt}

\begin{btt}
Giải hệ phương trình nghiệm nguyên
\[\heva{&x^2+4y^2+2z^2+2(xz+2x+2z)=396\\ &x^2+y^2=3z.}\]
\nguon{Chuyên Toán Hải Dương 2021}
\end{btt}

\begin{btt}
Tìm tất cả các số nguyên $x,y,z$ thoả mãn \[3x^2+6y^2+z^2+3y^2z^2-18x=6.\]
\nguon{Chuyên Toán Hà Tĩnh 2012}
\end{btt}
\subsection*{Hướng dẫn bài tập tự luyện}

\begin{gbtt}
Giải phương trình nghiệm nguyên dương $2x^2+4x=19-3y^2.$
\loigiai{
Biến đổi phương trình đã cho, ta được
$$2x^2+4x=19-3y^2\Leftrightarrow 2\tron{x-1}^2 +3y^2=21.$$
Dựa vào biến đổi trên, ta suy ra
$$3y^2\le21\Rightarrow y^2\le 7\Rightarrow y^2\in \left\{0;1;4\right\}.$$
Ta xét các trường hợp sau đây:
\begin{enumerate}
    \item Với $y^2=0,$ ta có $2\tron{x-1}^2=21$. Phương trình này vô nghiệm.
    \item Với $y^2=1,$ ta có $2\tron{x-1}^2=18 \Rightarrow \tron{x-1}^2=9$, ta có $\tron{x,y}\in\{(-2,1);(-2,-1);(4,1);(4,-1)\}.$ 
    \item Với $y^2=4,$ ta có $2\tron{x-1}^2=9$. Phương trình vô nghiệm.
\end{enumerate}
Như vậy, phương trình đã cho có các nghiệm $\tron{x,y}$ là $(-2,1), (-2,-1), (4,1), (4,-1)$ .}
\end{gbtt}

\begin{gbtt}
Giải phương trình nghiệm nguyên
\[4x^2+4x+y^2-6y=24.\]
\loigiai{
Biến đổi phương trình đã cho, ta được 
$$4x^2+4x+y^2-6y=24\Leftrightarrow \tron{2x+1}^2+\tron{y-3}^2=34.$$
Dựa vào biến đổi trên kết hợp với $\tron{2x+1}^2$ là số lẻ, ta suy ra
$$\tron{2x+1}^2\le 34\Rightarrow \tron{2x+1}^2\in \left\{1;9;25\right\}.$$
Tới đây, ta xét các trường hợp sau.
\begin{enumerate}
    \item Với $\tron{2x+1}^2=1,$ ta có $\tron{y-3}^2=33$. Phương trình này vô nghiệm.
    \item Với $\tron{2x+1}^2=9$, ta có $\tron{y-3}^2=25$, thế nên $(x,y)\in\{(-2,8);(-2, -2);(1,8);(1,-2)\}.$
    \item Với $\tron{2x+1}^2=25$, ta có $\tron{y-3}^2=9$, thế nên $\tron{x,y}\in\{\tron{-3,0};\tron{-3, 6};\tron{2,0};\tron{2,6}\}.$
\end{enumerate}
Như vậy phương trình đã cho có các nghiệm $\tron{x,y}$ là 
$$(-2,8),\quad(-2, -2),\quad(1,8),\quad(1,-2),\quad\tron{-3,0},\quad\tron{-3, 6},\quad\tron{2,0},\quad\tron{2,6}.$$}
\end{gbtt}

\begin{gbtt}
Giải phương trình nghiệm nguyên $x^2-2y(x-y)=2(x+1).$
\nguon{Chuyên Toán Tây Ninh 2021}
\loigiai{
Phương trình đã cho tương đương với
$$2x^2-4xy+4y^2-4x=4\Leftrightarrow (x-2y)^2+(x-2)^2=8.$$
Dựa vào biến đổi trên, ta lần lượt suy ra $$(x-2)^2\le 8\Rightarrow (x-2)^2\in\left\{0;1;4\right\}.$$
Ta xét các trường hợp sau đây,
\begin{enumerate}
    \item Với $(x-2)^2=0$, thì $(x-2y)^2=8$, vô nghiệm.
    \item Với $(x-2)^2=1$, thì $(x-2y)^2=7$, vô nghiệm.
    \item Với $(x-2)^2=4$, thì $(x-2y)^2=4$, ta được các nghiệm $(x,y)$ là $(0,1),(0,-1),(4,1),(4,3)$.
\end{enumerate}
Vậy phương trình có các nghiệm $(x,y)$ là $(0,1),(0,-1),(4,1),(4,3)$.}
\end{gbtt}

\begin{gbtt}
Tìm tất cả các số nguyên dương $x,y$ thỏa mãn
    $$x^4-x^2+2x^2y-2xy+2y^2-2y-36=0.$$
\nguon{Chuyên Toán Đắk Lắk 2021}    
\loigiai{Phương trình đã cho tương đương với
    $$\left(x^2+y-1\right)^2+(x-y)^2=37.$$
    Có duy nhất một cách biểu diễn $37$ thành tổng hai bình phương, đó là $37^2=1+6^2.$\\
    Đồng thời, ta chỉ ra được $x^2+y-1>x-y.$ Ta xét các trường hợp sau.
\begin{enumerate}
        \item Với $x^2+y-1=6$ và $x-y=1,$ ta có
        $$x^2+(x-1)-1=6\Leftrightarrow x^2+x-8=0.$$
        Ta không tìm được $x$ nguyên dương ở đây.
        \item Với $x^2+y-1=6$ và $x-y=-1,$ ta có
        $$x^2+(x+1)-1=6\Leftrightarrow x^2+x-6=0\Leftrightarrow (x+3)(x-2)=0.$$
        Do $x$ nguyên dương, ta nhận được $x=2$ và $y=3.$
        \item Với $x^2+y-1=1$ và $x-y=-6,$ ta có
        $$x^2+(x+6)-1=1\Leftrightarrow x^2+4=0.$$
        Ta không tìm được $x$ nguyên dương ở đây.
\end{enumerate}
    Như vậy, cặp $(x,y)$ duy nhất thỏa mãn yêu cầu đề bài là $(2,3)$.}
\end{gbtt}

\begin{gbtt}
Giải phương trình nghiệm nguyên 
\[2x^6-2x^3y+y^2=128.\]
\loigiai{
Biến đổi tương đương phương trình đã cho ta được
$$\tron{x^3}^2+\tron{x^3-y}^2=128.$$
Có duy nhất một cách phân tích $128$ thành tổng hai bình phương là $$128=8^2+8^2.$$ 
Từ đây, ta sẽ lập bảng giá trị sau cho $x$ và $y.$
\begin{center}
    \begin{tabular}{c|c|c|c|c}
        $x^3$ & $8$ & $8$ & $-8$ & $-8$ \\
        \hline
        $x^3-y$ & $8$ & $-8$ & $8$ & $-8$ \\
        \hline
        $x$ & $2$ & $2$ & $-2$ & $-2$ \\
        \hline
        $y$ & $0$ & $16$ & $16$ & $0$ \\
    \end{tabular}
\end{center}
Như vậy, phương trình đã cho có $4$ nghiệm là $(2,0),\ (2,16),\ (-2,16)$ và $(-2,0).$}
\end{gbtt}

\begin{gbtt}
Tìm tất cả các số nguyên tố $a\geqslant b\geqslant c\geqslant d$ thỏa mãn
$$a^2+2b^2+c^2+2d^2=2\left ( ab+bc-cd+da \right ).$$
\nguon{Titu Andreescu}
\loigiai{
Phương trình đã cho tương đương với
$$(a-b-d)^{2}+(b-c-d)^{2}=0\Leftrightarrow \heva{a&=b+d \\ b&=c+d.}$$
Từ $a=b+d,$ ta nhận thấy ba số $a,b,d$ không thể cùng lẻ, thế nên phải có một số bằng $2.$ \\
Do $d\le b\le a$ nên $d=2.$ Hệ gồm hai phương trình $a=b+d$ và $b=c+d$ trở thành
\[a=b+2=c+4.\]
Tới đây, ta xét các trường hợp sau.
\begin{enumerate}
    \item Nếu $c\equiv 1\pmod{3}$ thì $b\equiv 0\pmod{3},$ lại do $b$ nguyên tố nên $b=3.$ Từ đây, ta tìm được $c=1,$ mâu thuẫn với điều kiện $c$ nguyên tố.
    \item Nếu $c\equiv 2\pmod{3}$ thì $a\equiv 0\pmod{3},$ lại do $a$ nguyên tố nên $b=3.$ Từ đây, ta tìm được $c=-1,$ mâu thuẫn với điều kiện $c$ nguyên tố.  
    \item Nếu $c\equiv 0\pmod{3}$ thì do $c$ nguyên tố nên $c=3.$ Ta tìm ra $b=5$ và $a=7.$
\end{enumerate}
Như vậy, bộ $(a,b,c,d)=(2,3,5,7)$ là bộ số nguyên tố duy nhất thỏa yêu cầu.}
\end{gbtt}

\begin{gbtt}
Phương trình $x^2+2y^2+2z^2-2xy-2yz-2z=4$ có tất cả bao nhiêu nghiệm nguyên?
\loigiai{
Phương trình đã cho tương đương
$$(x-y)^2+(y-z)^2+(z-1)^2=5.$$
Có một cách phân tích $5$ thành tổng các bình phương là $5=2^2+1^2+0^2.$ Theo đó
$$z-1\in \{0;1;-1;2;-2\}.$$
Ta xét các trường hợp sau
\begin{enumerate}
    \item Với $z-1=0,$ có tất cả $4$ cách chọn giá trị cho $y-z,$ và ứng với mỗi cách chọn giá trị cho $y-z$ có $2$ cách chọn giá trị cho $x-y.$ 
    \item Với $z-1$ bằng $1,-1,2$ hoặc $-2,$ có tất cả $4$ cách chọn giá trị cho cặp $(x-y,y-z),$ đó là
$$(0,A),\ (0,-A),\ (A,0),\ (-A,0).$$
\end{enumerate}
Như vậy phương trình đã cho có tất cả $4\cdot 2\cdot 2+4\cdot 4=24$ nghiệm nguyên.}
\end{gbtt}

\begin{gbtt}
Tìm tất cả các bộ số nguyên dương $(x, y, z)$ thỏa mãn
$$5\tron{x^{2}+2 y^{2}+z^{2}}=2(5 x y-y z+4 z x)$$
trong đó, ít nhất một trong ba số $x, y, z$ là số nguyên tố.
\nguon{Adrian Andreescu}
\loigiai{
Phương trình đã cho tương đương với
$$(x+y-2 z)^{2}+(2 x-3 y-z)^{2}=0
\Leftrightarrow \heva{&x+y=2z \\ &2x=3y+z}
\Leftrightarrow \heva{&x+y=2z \\ &2x=3y+\dfrac{x+y}{2}}
\Leftrightarrow
\heva{&x+y=2z \\ &5y=3z.}
$$
Do $(5,3)=1,$ ta chỉ ra tồn tại số nguyên dương $t$ sao cho 
$$y=3 t, \quad z=5 t,\quad x=7t.$$ 
Với việc một trong ba số $x,y,z$ nguyên tố, ta chỉ ra $t=1,$ và ta kết luận bộ ba $(x,y,z)=(7,3,5)$ là bộ số duy nhất thỏa mãn đề bài.}
\end{gbtt}

\begin{gbtt}
Giải bất phương trình nghiệm nguyên $5x^2+3y^2+4xy-2x+8y+8\le 0.$
\nguon{Chuyên Toán Đồng Nai 2021}
\loigiai{
 Bất phương trình đã cho tương đương
    \[(2x+y)^2+(x-1)^2+2(y+2)^2\le 1.\tag{*}\label{dongnice}\]
    Tổng ba số trong vế trái của (\ref{dongnice}) không vượt quá $1,$ chứng tỏ có ít nhất hai số bằng $0.$ Mặt khác, do 
    $$2x+y=2(x-1)+(y+2)$$
    nên nếu $2$ trong $3$ số kia bằng $0,$ số còn lại chắc chắn cũng bằng $0.$ Ta suy ra
    $$2x+y=x-1=y+2=0\Rightarrow x=1,y=-2.$$
    Như vậy $(x,y)=(1,-2)$ là nghiệm nguyên duy nhất của bất phương trình.}
\end{gbtt}

\begin{gbtt}
Tìm tất cả các số nguyên $x,y,z$ sao cho
\[x^2+y^2+z^2+6<xy+3y+4z.\]
\nguon{Chuyên Toán Nghệ An 2019}
\loigiai{
Bất phương trình đã cho tương đương với
\begin{align*}
    &\tron{x^2-xy+\dfrac{y^2}{4}}+3\tron{\dfrac{y^2}{4}-y+1}+\tron{z^2-4z+4}<1
    \\\Leftrightarrow \: &\tron{x-\dfrac{y}{2}}^2+3\tron{\dfrac{y}{2}-1}^2+(z-2)^2<1
    \\\Leftrightarrow \: & (2x-y)^2+3(y-2)^2+4(z-2)^2<4.
\end{align*}
Trước hết, ta có $4(z-2)^2<4$ nên $z=2.$ Bất phương trình trở thành
$$(2x-y)^2+3(y-2)^2<4.$$
Tiếp theo, ta có $3(y-2)^2<4,$ và ta suy ra $|y-2|\in \{0;1\}.$ Ta lập bảng giá trị
\begin{center}
    \begin{tabular}{c|c|c|c}
        $y-2$ & $2x-y$ & $y$ & $x$ \\
        \hline
        $1$ & $0$ & $3$ & $1,5$ \\
        $-1$ & $0$ & $1$ & $0,5$ \\     
        $0$ & $0$ & $2$ & $1$ \\   
        $0$ & $1$ & $2$ & $1,5$ \\      
        $0$ & $-1$ & $2$ & $0,5$                
    \end{tabular}
\end{center}
Căn cứ vào bảng giá trị, ta kết luận bộ $(x,y,z)=(1,2,2)$ là bộ duy nhất thỏa mãn yêu cầu.}
\end{gbtt}

\begin{gbtt}
Giải phương trình nghiệm nguyên
\[x^2+xy+y^2=3x+y-1.\]
\loigiai{
Phương trình đã cho tương đương với
\begin{align*}
    x^{2}+x y+y^{2}=3 x+y-1 
    &\Leftrightarrow 2 x^{2}+2 x y+2 y^{2}=6 x+2 y-2 
    \\&\Leftrightarrow (x+y)^{2}+(x-3)^{2}+(y-1)^{2}=8.
\end{align*}
Có duy nhất một cách phân tích $8$ thành tổng ba bình phương, đó là
$$8=0^2+2^2+2^2.$$
Tới đây, ta xét các trường hợp sau.
\begin{enumerate}
    \item Với $x+y=0$ hay $y=-x,$ phương trình đã cho trở thành 
    $$(-{y}-3)^{2}+({y}-1)^{2}=8 \Leftrightarrow 2(y+1)^2=0\Leftrightarrow{y}=-1.$$
    Trường hợp này cho ta $(x,y)=(1,-1).$
    \item Với $x-3=0$ hay $x=3,$ phương trình đã cho trở thành  
    $$(y+3)^{2}+(y-1)^{2}=8 \Leftrightarrow 2(y+1)^2=0\Leftrightarrow{y}=-1.$$
    Trường hợp này cho ta $(x,y)=(3,-1).$
    \item Với $y-1=0$ hay $y=1,$ phương trình đã cho trở thành   
    $$({x}+1)^{2}+({x}-3)^{2}=8 \Leftrightarrow 2(x-1)^2=0\Leftrightarrow{x}=1.$$
    Trường hợp này cho ta $(x,y)=(1,1).$    
\end{enumerate}
Kết luận, phương trình đã cho có $3$ nghiệm nguyên là $(1,-1),\ (3,-1)$ và $(1,1).$}
\end{gbtt}

\begin{gbtt}
Giải hệ phương trình nghiệm nguyên
\[\heva{&x+y-z=2\\&3x^2+2y^2-z^2=13.}\]
\loigiai{
Phương trình thứ nhất trong hệ tương đương
$$z=x+y-2.$$
Thế $z=x+y-2$ vào phương trình hai của hệ, ta được
$$3x^2+2y^2-\tron{x+y-2}^2=13\Rightarrow 2x^2 + y^2 +4x+4y-2xy=17\Rightarrow \tron{y-x+2}^2+ \tron{x+4}^2=37.$$
Dựa vào biến đổi trên, ta suy ra
$$\tron{x+4}^2\le37\Rightarrow\tron{x+4}^2\in\left\{0;1;4;9;16;25;36\right\}.$$
Tới đây, ta xét các trường hợp sau.
\begin{enumerate}
    \item Với $\tron{x+4}^2=0$, ta có $\tron{y-x+2}^2=37.$ Phương trình vô nghiệm.
    \item Với $\tron{x+4}^2=1,$ ta có $\tron{y-x+2}^2=36,$  thế nên $$(x,y,z)\in\{\tron{-5,-13,-20};\tron{-5, -1, -8};\tron{-3,-11,-16};\tron{-3,1, -4}\}.$$
    \item Với $\tron{x+4}^2=4$, ta có $\tron{y-x+2}^2=33$. Phương trình vô nghiệm.
    \item Với $\tron{x+4}^2=9$, ta có $\tron{y-x+2}^2=28$. Phương trình vô nghiệm.
    \item Với $\tron{x+4}^2=16$, ta có $\tron{y-x+2}^2=21$. Phương trình vô nghiệm.
    \item Với $\tron{x+4}^2=25$, ta có $\tron{y-x+2}^2=12$. Phương trình vô nghiệm.
     \item Với $\tron{x+4}^2=36,$ ta có $\tron{y-x+2}^2=1,$  thế nên
     $$(x,y,z)\in\{\tron{-10,-13,-25};\tron{-10, -11, -23};\tron{2,-1,-1};\tron{2, 1, 1}\}.$$
\end{enumerate}
Như vậy, hệ phương trình đã cho có $8$ nghiệm nguyên $\tron{x,y,z}$ là 
$$\tron{-5,-13,-20}, \tron{-5, -1, -8},\tron{-3,-11,-16},\tron{-3, 1, -4},$$
$$\tron{-10,-13,-25}, \tron{-10, -11, -23},\tron{2,-1,-1},\tron{2, 1, 1}. $$}
\end{gbtt}

\begin{gbtt}
Giải hệ phương trình nghiệm nguyên
\[\heva{&x^2+4y^2+2z^2+2(xz+2x+2z)=396\\ &x^2+y^2=3z.}\]
\nguon{Chuyên Toán Hải Dương 2021}
\loigiai{
Từ phương trình thứ hai, ta chỉ ra cả $x,y,z$ đều chia hết cho $3,$ đồng thời $z$ không âm. \\
Phương trình thứ nhất trong hệ tương đương với
    $$\tron{x+z+2}^2+\tron{2y}^2+z^2=400.$$
Có đúng hai cách để viết $400$ thành tổng ba số chính phương, đó là
    $$400=0^2+0^2+20^2=0^2+12^2+16^2.$$
Dựa vào các nhận xét kể trên, ta xét các trường hợp sau.
\begin{enumerate}
    \item Nếu $z=0,$ phương trình thứ hai trở thành
        $$x^2+y^2=0.$$
    Ta có $x=y=0.$ Thế ngược lại phương trình thứ nhất, ta thấy không thỏa.
    \item Nếu $z=12,$ một trong hai số $x+y+2, 2y$ phải bằng $0,$ vậy nên
        $$\heva{&\hoac{x+y+2=0 \\ 2y=0}\\&x^2+y^2=36}\Rightarrow \hoac{&x=6,y=0 \\ &x=-6,y=0.}$$
    Thế ngược lại phương trình thứ nhất, ta thấy không thỏa.
\end{enumerate}
Như vậy, hệ đã cho không có nghiệm nguyên.}
\end{gbtt}

\begin{gbtt}
Tìm tất cả các số nguyên $x,y,z$ thoả mãn \[3x^2+6y^2+z^2+3y^2z^2-18x=6.\]
\nguon{Chuyên Toán Hà Tĩnh 2012}
\loigiai{Phương trình đã cho tương đương với
$$3{{\left( x-3 \right)}^{2}}+6{{y}^{2}}+{{z}^{2}}+3{{y}^{2}}{{z}^{2}}=33.$$
Từ đây, ta suy ra $3\mid {{z}^{2}}$ và ${{z}^{2}}\le 33$. Vì  $z$ nguyên nên $z=0$ hoặc $\,\left| z \right|=3$. Ta xét các trường hợp sau
\begin{enumerate}
     \item Với $z=0,$ phương trình trên trở thành $${{\left( x-3 \right)}^{2}}+2{{y}^{2}}=11.$$ 
     Ta suy ra $2{{y}^{2}}\le 11$ nên $\left| y \right|\le 2.$ Ta lập bảng giá trị
     \begin{center}
         \begin{tabular}{c|c|c|c}
             $|y|$ & $0$ & $1$ & $2$ \\
             \hline
             $(x-3)^2$ & $11$ & $9$ & $3$\\
             \hline
             $x$ & $\notin\mathbb{Z}$ & $0$ hoặc $6$ & $\notin\mathbb{Z}$
         \end{tabular}
     \end{center}
     Trường hợp này cho ta $4$ cặp $(x,y)$ là
     $$(0,1),\quad (0,-1),\quad (6,1),\quad (6,-1).$$
     \item  Với $\left| z \right|=3$, phương trình trên trở thành
     $$\left( x-3 \right)^{2}+11y^2=8.$$ 
     Ta suy ra $11{{y}^{2}}\le 8$ nên $y=0.$ Thế trở lại, ta không tìm được $x$ nguyên.
\end{enumerate}
Như vậy, có tất cả $4$ bộ $\left( x,y,z \right)$ thỏa mãn đề bài là
$$\left( 0,1,0 \right),\: \left( 0,-1,0 \right),\:\left( 6,1,0 \right),\:\left( 6,-1,0 \right).$$}
\end{gbtt} %pt ước số + tách tổng bình phương
\section{Phương pháp đánh giá trong phương trình nghiệm nguyên}
Bất đẳng thức là một công cụ mạnh trong việc chặn khoảng cho các biến số, từ đó tìm được nghiệm cho phương trình nghiệm nguyên. Ngoài những đánh giá thông thường hoặc đánh giá kết hợp với bất đẳng thức cổ điển, trong cuốn sách này, chúng ta đã được tìm hiểu một vài phương pháp đánh giá bất đẳng thức khác trong phương trình nghiệm nguyên, như sử dụng đánh giá bất đẳng thức trong chia hết hay sử dụng bổ đề kẹp. Các kĩ thuật nâng cao ấy sẽ được nhắc lại ở phần sau, còn dưới đây là một vài bài tập cơ bản.

\subsection*{Bài tập tự luyện}

\begin{btt}
Giải phương trình nghiệm nguyên $$6x^2+7y^2=229.$$
\end{btt}



\begin{btt}
Giải phương trình nghiệm nguyên dương
\[x^2\tron{y+3}=y\tron{x^2-3}^2.\]
\nguon{Chuyên Toán Phú Thọ 2019}
\end{btt}

\begin{btt}
Giải phương trình nghiệm nguyên dương $$(x+y)^4=40y+1.$$
\end{btt}

\begin{btt}
Tìm tất cả các số nguyên dương $x,y$ thỏa mãn
\[16\tron{x^3-y^3}=15xy+371.\]
\nguon{Chuyên Toán Thái Nguyên 2019}
\end{btt}

\begin{btt}
Tìm tất cả các cặp số nguyên $(x,y)$ thỏa mãn đẳng thức \[\left(x^2-y^2\right)^2=1+20y.\]
\nguon{Chuyên Toán Đà Nẵng 2021}
\end{btt}

\begin{btt}
Giải phương trình nghiệm nguyên dương
$$\left(x^{2}-y^{2}\right)^{2}-6 \min \{x;y\}=2013.$$
\nguon{Titu Andreescu}
\end{btt}

\begin{btt}
Giải phương trình nghiệm nguyên dương
\[\left(x^2+4 y^2+28\right)^2=17\left(x^4+y^4+14 y^2+49\right).\]
\end{btt}

\begin{btt}
Giải phương trình nghiệm nguyên
$$7\left(x^2+xy+y^2\right)=39\left(x+y\right).$$
\end{btt}

\begin{btt}
Giải phương trình nghiệm nguyên
$$19\left(2x^2+2xy+5y^2\right)=65\left(2x+5y\right).$$
\end{btt}

\begin{btt}
Giải phương trình nghiệm nguyên $$12x^2+6xy+3y^2=28(x+y).$$
\nguon{Hanoi Open Mathematics Competitions 2014}
\end{btt}

\begin{btt}
Giải phương trình nghiệm nguyên $$x^3+y^3=(x+y)^2.$$
\end{btt}

\begin{btt}
Giải phương trình nghiệm nguyên dương
$$xy+yz+zx+1=3xyz.$$
\end{btt}

\begin{btt}
Giải phương trình nghiệm nguyên dương
\[4xyz=x+2y+4z+3.\]
\end{btt}

\begin{btt}
Giải phương trình nghiệm nguyên dương
\[x^2+y^2+z^2+xyz=13.\]
\end{btt}

\begin{btt}
Giải phương trình nghiệm nguyên
\[\dfrac{xy}{z}+\dfrac{xz}{y}+\dfrac{yz}{x}=3.\]
\end{btt}

\begin{btt}
Cho các số nguyên dương $x,y,z$ thỏa mãn biểu thức sau nhận giá trị nguyên
$$T=\dfrac{1}{x}+\dfrac{1}{y}+\dfrac{1}{z} +\dfrac{1}{xy}+\dfrac{1}{yz}+\dfrac{1}{zx}.$$
\begin{enumerate}[a,]
    \item Chứng minh rằng $x,y,z$ cùng tính chẵn lẻ.
    \item Tìm tất cả các bộ $x,y,z$ với $x<y<z$ thỏa mãn giả thiết.
\end{enumerate}
\end{btt}

\begin{btt}
Giải phương trình nghiệm nguyên dương
$$101x^3-2019xy+101y^3=100.$$
\nguon{Titu Andreescu}
\end{btt}

\begin{btt}
Tìm tất cả các số nguyên $x,y$ với $y\ge 0$ thỏa mãn
\[x^2+2xy+y!=131.\]
\end{btt}

\begin{btt}
Giải phương trình nghiệm nguyên dương
\[\tron{1+x!}\tron{1+y!}=(x+y)!.\]
\nguon{Tạp chí Toán học và Tuổi trẻ, tháng 10 năm 2017}
\end{btt}

\begin{btt}
Tìm tất cả các số nguyên dương $m,n$ thỏa mãn 
\[m !+n !=(m+n+3)^{2}.\]
\end{btt}

\begin{btt}
Tìm tất cả các số nguyên dương $w$, $x$, $y$ và $z$ sao cho $w!=x!+y!+z!$.
\nguon{Canadian Mathematical Olympiad 1983}
\end{btt}

\begin{btt}
Xét phương trình $x^2+y^2+z^2=3xyz.$
\begin{enumerate}[a,]
    \item Tìm tất cả các nghiệm nguyên dương có dạng $\left( x,y,y \right)$ của phương trình đã cho.
    \item Chứng minh rằng tồn tại nghiệm nguyên dương $\left( a,b,c \right)$ của phương trình và thỏa mãn điều kiện 
    $$\min \left\{ a;b;c \right\}>2017.$$
\end{enumerate}
\nguon{Chuyên toán Vĩnh Phúc 2017 $-$ 2018}
\end{btt}

\begin{btt}
\hfill
\begin{enumerate}[a,]
    \item Cho hai số nguyên $a,b$ thỏa mãn $a^3+b^3>0.$ Chứng minh rằng
    \[a^3+b^3\ge a^2+b^2.\]
    \item Tìm tất cả các số nguyên $x,y,z,t$ thỏa mãn đồng thời
\[x^3+y^3=z^2+t^2\text{ và }z^3+t^3=x^2+y^2.\]
\end{enumerate}
\nguon{Chuyên Toán Phổ thông Năng khiếu 2019}
\end{btt}

\subsection*{Hướng dẫn bài tập tự luyện}

\begin{gbtt}
Giải phương trình nghiệm nguyên $6x^2+7y^2=229.$
\loigiai{
Do $6x^2\ge 0,$ ta có $7y^2\le 229,$ hay là $y^2\le 32.$ Mặt khác, do $y$ là số chính phương lẻ nên 
$$y^2\in \{1;9;25\}.$$ Thử với từng trường hợp, ta thấy chỉ có $y^2=25$ cho $x^2=9$ là số chính phương. \\
Phương trình đã cho có bốn nghiệm nguyên, bao gồm
$(-3,-5),(-3,5),(3,-5) \text{ và } (3,5).$}
\end{gbtt}



\begin{gbtt}
Giải phương trình nghiệm nguyên dương
\[x^2\tron{y+3}=y\tron{x^2-3}^2.\]
\nguon{Chuyên Toán Phú Thọ 2019}
\loigiai{
Phương trình đã cho tương đương với
\[y\vuong{\tron{x^2-3}^2-x^2}=3x^2.\] 
Do $y\ge 1$ nên ta suy ra
\begin{align*}
    \tron{x^2-3}^2-x^2\le 3x^2
    \Rightarrow \tron{x^2-3}^2\le 4x^2
    \Rightarrow \tron{x^2-1}\tron{x^2-9}\le 0
    \Rightarrow 1\le x^2\le 9\Rightarrow 1\le x\le 3.
\end{align*}
Lần lượt thế $x=1,2,3$ trở lại, ta kết luận phương trình có hai nghiệm nguyên dương là $$(x,y)=(1,1),\quad (x,y)=(3,1).$$}
\end{gbtt}

\begin{gbtt}
Giải phương trình nghiệm nguyên dương 
\[(x+y)^4=40y+1.\]
\loigiai{
Điều kiện $x,y$ nguyên dương cho ta $x\geq1,y\geq1$. Ta nhận thấy rằng
$$40y+1=(x+y)^4\ge (1+y)^4.$$
Điều trên chỉ xảy ra khi $y\le 2.$ Thật vậy, nếu $y\ge 3$, ta có
$$(y+1)^4\ge (3+1)^3(y+1)=64(y+1)>40(y+1),$$
mâu thuẫn. Lập luận được $y\le 2,$ ta chỉ ra hoặc $y=1,$ hoặc $y=2.$\\
Thử với từng trường hợp, ta kết luận $(x,y)=(1,2)$ là nghiệm nguyên duy nhất của phương trình.
}
\end{gbtt}

\begin{gbtt}
Tìm tất cả các số nguyên dương $x,y$ thỏa mãn
\[16\tron{x^3-y^3}=15xy+371.\]
\nguon{Chuyên Toán Thái Nguyên 2019}
\loigiai{
Giả sử tồn tại các số nguyên dương $x,y$ thỏa mãn đề bài. Ta có
$$15xy+371=16\tron{x-y}\tron{x^2+xy+y^2}.$$
Do vế trái dương nên $x-y>0$ hay $x-y\ge 1.$ Suy ra
$$15xy+371\ge 16\tron{x^2+xy+16y^2}=1+15xy+16\tron{x^2+y^2}.$$
Chuyển vế, ta được $16\tron{x^2+y^2}\le 370$ hay $x^2+y^2\le 23.$ \\
Ngoài ra, lấy đồng dư modulo $2$ hai vế phương trình đã cho, ta được
$$xy+1\equiv 0\pmod{2}.$$
Ta có $x,y$ cùng lẻ từ đây. Do $x>y\ge 1$ và $x^2<23$ nên chỉ xảy ra khả năng $x=3.$ Thế trở lại, ta có $y=1.$\\ Cặp số duy nhất thỏa mãn yêu cầu là $(x,y)=(3,1).$}
\end{gbtt}

\begin{gbtt}
Tìm tất cả các cặp số nguyên $(x,y)$ thỏa mãn đẳng thức \[\left(x^2-y^2\right)^2=1+20y.\]
\nguon{Chuyên Toán Đà Nẵng 2021}
\loigiai{
Giả sử tồn tại các cặp số nguyên $(x,y)$ thỏa mãn. Rõ ràng $y\ge 0,$ đồng thời khi thay $x$ thành $-x,$ đẳng thức vẫn đúng, thế nên không mất tổng quát, giả sử $x\ge 0.$\\
Ta nhận thấy $x=y$ không thỏa mãn. Trong trường hợp $x\ne y,$ ta suy ra $(x-y)^2\ge 1,$ vì thế
$$1+20y=\left(x^2-y^2\right)^2=(x-y)^2(x+y)^2\ge(x+y)^2\ge y^2.$$
Dựa vào đánh giá trên, ta có
$$y^2\le 20y+1\Rightarrow (y-10)^2\le 101\Rightarrow 10-\sqrt{101}\le y\le 10+\sqrt{101}.$$
Do $y$ là số tự nhiên, ta chọn $y=0,1,2,\ldots,20.$ Trong các số này, $20y+1$ chỉ nhận giá trị là số chính phương với $y=0,y=4,y=6$ và $y=18.$ 
\begin{enumerate}
    \item Với $y=0,$ ta có $x^4=1.$ Do $x\ge0,$ ta chọn $x=1.$
    \item Với $y=4,$ ta có $\left(x^2-16\right)^2=81\Leftrightarrow x^2-16=\pm 9\Leftrightarrow \hoac{&x^2=7 \\ &x^2=25}\Leftrightarrow x=\pm 5.$ \\
    Do $x\ge 0,$ ta chọn $x=5.$
    \item Với $y=6,$ ta có $\left(x^2-36\right)^2=121\Leftrightarrow x^2-36=\pm 11\Leftrightarrow \hoac{&x^2=25 \\ &x^2=47}\Leftrightarrow x=\pm 5.$ \\
    Do $x\ge 0,$ ta chọn $x=5.$  
    \item Với $y=18,$ ta có $\left(x^2-324\right)^2=381\Leftrightarrow x^2-324=\pm 19\Leftrightarrow \hoac{&x^2=305 \\ &x^2=343},$ mâu thuẫn.    
\end{enumerate}
Kết quả, có tất cả $6$ cặp $(x,y)$ thỏa mãn đề bài, bao gồm
$(1,0),(-1,0),(5,4),(-5,4),(5,6),(-5,6).$}
\begin{luuy}
Trong bài toán trên, ta có thể đưa phương trình đã cho về dạng
$$x^{4}-2x^{2}y^{2}+\left(y^{4}-10y-9\right)=0.$$
Việc biến đổi các điều kiện $\Delta^{'}\geq0$ và $\Delta^{'}$ là số chính phương đều không cho ta hiệu quả nhất định. Bắt buộc, ta phải nghĩ đến một phương án sử dụng bất đẳng thức khác, đó chính là cách làm trong bài trên.
\end{luuy}
\end{gbtt}

\begin{gbtt}
    Giải phương trình nghiệm nguyên dương
    $$\left(x^{2}-y^{2}\right)^{2}-6 \min \{x;y\}=2013.$$
\nguon{Titu Andreescu}
\loigiai{
Rõ ràng $x \neq y$. Không mất tính tổng quát, ta có thể giả sử $x<y$, khi ấy ta thu được đánh giá sau
    $$2013+6x=(x-y)^{2}(x+y)^{2}>(x+y)^{2}>4 x^{2}.$$
Đánh giá kể trên cho ta $0<x<23.$ Hơn nữa thì từ
    $$\left(x^{2}-y^{2}\right)^{2}=3(671+2 x)$$
ta có $671+2x$ phải chia hết cho $3,$ kéo theo $x$ chia $3$ dư $2.$ Tới đây, lần lượt thử với $x=2,5,8,\ldots,20,$ ta kết luận tất cả các nghiệm nguyên dương của phương trình là $(2,7)$ và $(7,2).$}
\end{gbtt}

\begin{gbtt}
Giải phương trình nghiệm nguyên dương
\[\left(x^2+4 y^2+28\right)^2=17\left(x^4+y^4+14 y^2+49\right).\]
\loigiai{
Phương trình đã cho tương đương với
$$\left(1\cdot x^{2}+4\cdot\tron{y^2+7}\right)^{2}=\left(1^{2}+4^{2}\right)\left(\left(x^{2}\right)^{2}+\tron{y^2+7}^{2}\right).$$
Vế trái nhỏ hơn vế phải theo như bất đẳng thức $Cauchy - Schwarz.$ Vì thế, dấu bằng phải xảy ra, tức là
$$\dfrac{x^{2}}{1}=\dfrac{y^{2}+7}{4} .$$
Phương trình đã cho, theo đó, tương đương với
$$4x^2=y^2+7\Leftrightarrow(2x+y)(2x-y)=7.$$
Do $0<2x-y<2x+y$ và $7$ là số nguyên tố nên là
$$\heva{
2 x+y=7 \\
2 x-y=1} \Leftrightarrow
\heva{
x&=2 \\
y&=3.}$$
Vậy $(x, y)=(2,3)$ là nghiệm nguyên dương duy nhất của phương trình đã cho.}
\end{gbtt}

\begin{gbtt}\label{dgia.dbac}
Giải phương trình nghiệm nguyên
$7\left(x^2+xy+y^2\right)=39\left(x+y\right).$
\loigiai
{Căn cứ vào phương trình, ta chỉ ra $39(x+y)$ chia hết cho $7,$ lại do $(39,7)=1$ nên $x+y$ chia hết cho $7.$\\
Ta đặt $x+y=7m$, với $m$ là số nguyên. Khi đó, $x^2+xy+y^2=39m$. Ta nhận thấy rằng
$$49m^2=(x+y)^2\le \dfrac{4}{3}\left(x^2+xy+y^2\right)=52m.$$
Ta suy ra $49m^2\le 52m,$ hay $m(52-49m)\ge 0.$ Lại do $m$ nguyên nên $m\in\left\{0;1\right\}$.
\begin{enumerate}
        \item Với $m=0$, ta có $$\heva{&x^2+xy+y^2=0\\&x+y=0}
        \Leftrightarrow \heva{&4x^2+4xy+4y^2=0\\&x+y=0}
        \Leftrightarrow \heva{&3x^2+(x+2y)^2=0\\&x+y=0}
        \Leftrightarrow x=y=0.$$ 
         \item Với $m=1$, ta có
         \begin{align*}
        \heva{&x+y=7\\&x^2+xy+y^2=39}
         &\Leftrightarrow \heva{&y=7-x \\&x^2+x(7-x)+(7-x)^2=39}
         \\&\Leftrightarrow  \heva{&y=7-x \\&x^2-7x+10=0}
         \\&\Leftrightarrow  \heva{&y=7-x \\&(x-2)(x-5)=0}         
         \\&\Leftrightarrow\hoac{&x=2,y=5 \\ &x=5,y=2.}
         \end{align*}
\end{enumerate}
Như vậy, phương trình đã cho có ba nghiệm nguyên, bao gồm 
$(0,0),(2,5)\text{ và }(5,2).$}
\begin{luuy}
Đánh giá bất đẳng thức $(x+y)^2\le \dfrac{4}{3}\left(x^2+xy+y^2\right)$ phía trên được gọi là một đánh giá đồng bậc, và được chứng minh bằng khai triển trực tiếp. Bạn được có thể tự tìm ra một vài đánh giá đồng bậc đẹp đẽ khác, chẳng hạn như
$(x+y)^2\le 4\left(x^2-xy+y^2\right).$
\end{luuy}
\end{gbtt}

\begin{gbtt}
Giải phương trình nghiệm nguyên
\[19\left(2x^2+2xy+5y^2\right)=65\left(2x+5y\right).\]
\loigiai{
Tương tự như \chu{bài \ref{dgia.dbac}}, ta có thể đặt $2x^2+2xy+5y^2=65m$ và $2x+5y=19m$, với $m$ là số nguyên. \\
Mặt khác, áp dụng bất đẳng thức $Cauchy-Schwarz,$ ta có
\begin{align*}
    2x^2+2xy+5y^2&=(x+y)^2+x^2+4y^2
    \\&\ge x^2+4y^2 \\&\ge\dfrac{4}{41}\left(4+\dfrac{25}{4}\right)\left(x^2+4y^2\right)
    \\&\ge \dfrac{4}{41}(2x+5y)^2.
\end{align*}
Ta suy ra $65m\ge \dfrac{4}{41}\cdot(19m)^2$ từ đây, hay là $m(1444m-2665)\le 0.$ \\
Do $m$ nguyên, chỉ có $m=0$ hoặc $m=1$ thỏa mãn.
\begin{enumerate}
        \item Với $m=0$, ta có 
        $$\heva{&2x^2+2xy+5y^2=0 \\ &2x+5y=0}
        \Leftrightarrow \heva{&(x+y)^2+x^2+4y^2=0 \\ &2x+5y=0}
        \Leftrightarrow x=y=0.$$
         \item Với $m=1$, ta có   
         \begin{align*}
        \heva{&2x^2+2xy+5y^2=65 \\ &2x+5y=19}
        &\Leftrightarrow \heva{&y=\dfrac{19-2x}{5} \\ &2x^2+2x\left(\dfrac{19-2x}{5}\right)+5\left(\dfrac{19-2x}{5}\right)^2=65}
        \\&\Leftrightarrow
        \heva{&y=\dfrac{19-2x}{5} \\ &(x-2)(5x-9)=0}\\&
        \Leftrightarrow 
        \heva{&x=2 \\ &y=3.}      
         \end{align*}
\end{enumerate}
Như vậy, phương trình đã cho có hai nghiệm nguyên, bao gồm $(0,0)$ và $(2,3).$}
\end{gbtt}

\begin{gbtt}
Giải phương trình nghiệm nguyên \[12x^2+6xy+3y^2=28(x+y).\]
\nguon{Hanoi Open Mathematics Competitions 2014}
\loigiai{
Phương trình đã cho tương đương với 
\[9x^2=-3\tron{x+y}^2+28\tron{x+y}.\tag{*}\label{homc2014}\]
 Vì $3\mid 9x^2$ nên $3\mid\tron{-3\tron{x+y}^2+28}$ kéo theo $3\mid(x+y).$
Với mọi số nguyên $x,y,$ ta luôn có nhận xét sau 
$$-3\tron{x+y}^2+28\tron{x+y}=9x^2\ge0.$$
Ta suy ra $0\le x+y\le9.$ Từ những đánh giá trên, ta thu được
$$x+y\in \left\{0;3;6;9\right\}.$$
Ta xét các trường hợp sau.
\begin{enumerate}
    \item Với $x+y=0,$ thế trở lại (\ref{homc2014}), ta thu được $x=0$ kéo theo $y=0.$
    \item Với $x+y=3,$ thế trở lại (\ref{homc2014}), ta không tìm được số nguyên $x$ thỏa mãn.
    \item Với $x+y=6,$ thế trở lại (\ref{homc2014}), ta không tìm được số nguyên $x$ thỏa mãn.
    \item Với $x+y=9,$ thế trở lại (\ref{homc2014}), ta thu được $x^2=1$. Các cặp $(x,y)$ thỏa trường hợp này là $(1,8),(-1,10).$
\end{enumerate}
Như vậy, phương trình có $3$ nghiệm nguyên $(x,y)$ là $(0,0),(1,8),(-1,10).$}
\end{gbtt}

\begin{gbtt}
Giải phương trình nghiệm nguyên \[x^3+y^3=(x+y)^2.\]
\loigiai{
Phương trình đã cho tương đương với
$$(x+y)\left(x^2-xy+y^2\right)-(x+y)^2=0\Leftrightarrow (x+y)\left(x^2-xy+y^2-x-y\right)=0.$$
Ta xét các trường hợp sau đây
\begin{enumerate}
    \item Nếu $x+y=0$, phương trình có vô số nghiệm dạng $(x,y)=(a,-a),$ với $a$ là một số nguyên tùy ý.
    \item Nếu $x^2-xy+y^2=x+y$ và $x+y\ne 0,$ ta có
    $$x+y=x^2-xy+y^2\ge \dfrac{1}{4}\left(x+y\right)^2.$$
    Lập luận trên cho ta $4(x+y)\ge (x+y)^2,$ tức $0\le x+y\le 4.$\\ Tuy nhiên, do $x+y\ne 0$ nên $x+y\in\{1;2;3;4\}.$ Ta lập bảng giá trị
    \begin{center}
        \begin{tabular}{c|c|c|c|c}
            $x+y$ & $1$ & $2$ & $3$ & $4$   \\
            \hline
            $x^2-xy+y^2$ & $1$ & $2$ & $3$ & $4$
        \end{tabular}
    \end{center}
    Giải mỗi hệ phương trình trong từng khả năng, ta chỉ ra có $5$ nghiệm nguyên $(x,y)$ là $$(0,1),\, (1,0),\, (1,2),\, (2,1),\, (2,2).$$
\end{enumerate}
Kết luận, tất cả các nghiệm của phương trình đã cho là $(a,-a)$ với $a$ nguyên, cộng thêm các nghiệm $$(0,1),(1,0),(1,2),(2,1),(2,2).$$}
\end{gbtt}

\begin{gbtt}
Giải phương trình nghiệm nguyên dương
\[xy+yz+zx+1=3xyz.\]
\loigiai{
Không mất tính tổng quát, ta giả sử $x\ge y\ge z.$
    Nếu như $x\ge y\ge z\ge 2,$ ta có
    $$\dfrac{1}{x}+\dfrac{1}{y}+\dfrac{1}{z}+\dfrac{1}{xyz}\le \dfrac{1}{2}+\dfrac{1}{2}+\dfrac{1}{2}+\dfrac{1}{2\cdot2\cdot2}=\dfrac{13}{8}<3,$$
    một điều mâu thuẫn. Mâu thuẫn này chứng tỏ $z=1.$ \\
    Thế ngược lại $z=1,$ ta được
    $$xy+x+y+1=3xy\Leftrightarrow 2xy-x-y-1=0\Leftrightarrow (2x-1)(2y-1)=3.$$
    Giải phương trình ước số trên, ta suy ra $x=2,y=1.$ Kết luận, tất cả các nghiệm nguyên dương của phương trình là hoán vị của bộ $(2,1,1).$}
\begin{luuy}
	\nx Trong những bài toán có hai vế phương trình lệch bậc, giả sử sắp thứ tự rất tốt cho việc chặn biến. Cụ thể, trong bài toán trên, giả sử đó cho phép ta tìm được chặn trên $z<2.$
\end{luuy}    
\end{gbtt}


\begin{gbtt}
Giải phương trình nghiệm nguyên dương
\[4xyz=x+2y+4z+3.\]
	\loigiai{
	Giả sử phương trình có nghiệm $(x,y,z).$ Trong bài toán này, ta xét các trường hợp dưới đây.
		\begin{enumerate}
			\item Nếu $x=1,$ thế vào phương trình ban đầu, ta được 
        	$$2y+4z+4\Leftrightarrow 2yz-y-2z-2=0\Leftrightarrow (y-1)(2z-1)=3.$$
            Tới đây, ta lập được bảng giá trị sau cho $y$ và $z$
            \begin{center}
            \begin{tabular}{c|c|c|c}
            $y-1$ & $2z-1$ & $y$ & $z$ \\
                \hline
            $1$ & $3$ & $2$ & $2$ \\
                 \hline
            $3$ & $1$ & $4$ & $2$  
            \end{tabular}
            \end{center} 
			\item Nếu $x \geq 2,$ ta có nhận xét sau từ phương trình ban đầu
			$$2y+4z+3=x(4yz-1) \geq 2(4yz-1)=8yz-2.$$
			Dựa vào nhận xét trên, ta lần lượt suy ra
			$$8yz -2y-4z \leq 5 \Rightarrow (2y-1)(4z-1) \leq 6.$$ 
			Do $4z-1\geq 3$ nên $2y-1 \leq 2,$ kéo theo  $2y\leq 3,$ hay $y=1$. Thế vào phương trình ban đầu ta được 
			$$4xz-x-4z=5 \Leftrightarrow (x-1)(4z-1)=6.$$
			Do $4z-1$ là ước lẻ của $6$ và $4z-1 \geq 3$ nên	$\heva{4z-1&=3\\x-1&=2},$ hay là $\heva{&z=1\\&x=3.}$
		\end{enumerate}
		Kết luận, phương trình đã cho có $3$ nghiệm là $(1,2,2), (1,4,1),(3,1,1).$}
\begin{luuy}
\nx
\begin{enumerate}
		\item Do vai trò của $x,y,z$ không bình đẳng, ta không thể giả sử $1\leq x \leq y \leq z$. Trong bài toán trên, do $x$ nguyên dương nên trước hết ta xét $x=1$, sau đó xét $x\geq 2$. Nhờ $x \geq 2,$ ta có thể giới hạn được $2y \leq 3$.
		\item Ngoài ra, trong trường hợp $x\ge 2$ của bài toán, ta đã sử dụng kĩ thuật "làm mất biến". Theo đó, bước đánh giá $x(4yz-1)\ge 2(4yz-1)$ là nhằm mục đích dùng bất đẳng thức để triệt tiêu biến $x,$ từ đó chỉ ra bất phương trình dạng "tích nhỏ hơn tổng", nơi cực kì thuận lợi cho việc chặn $y.$
\end{enumerate}
\end{luuy}
\end{gbtt}

\begin{gbtt}
Giải phương trình nghiệm nguyên dương
\[x^2+y^2+z^2+xyz=13.\]
\loigiai{
Không mất tính tổng quát, giả sử $1\leq x \leq y \leq z.$ Ta có 
	$$ 13=x^2+y^2+z^2+xyz\geq 3x^2+x^3\geq 4x^2.$$ 
Từ $4x^2\le 13,$ ta chỉ ra $x=1.$ Thế vào phương trình đã cho, ta tiếp tục nhận xét
	$$ 12=y^2+z^2+yz\geq 3y^2. $$ 
Từ $y^2\leq 4,$ ta chỉ ra $y=1$ hoặc $y=2.$ Thử trực tiếp, ta kết luận tất cả các nghiệm nguyên dương của phương trình đã cho là $(1,2,2)$ và hoán vị.}
\end{gbtt}

\begin{gbtt}
Giải phương trình nghiệm nguyên
\[\dfrac{xy}{z}+\dfrac{xz}{y}+\dfrac{yz}{x}=3.\]
\loigiai{
Với điều kiện $xyz\ne 0,$ phương trình đã cho tương đương với
$$x^2y^2+y^2z^2+z^2x^2=3xyz.$$
Ta có $xyz>0,$ nên trong $x,y,z,$ hoặc cả ba số đều dương, hoặc có một số dương, hai số âm. Chú ý rằng nếu đổi dấu hai trong ba số $x,y,z$ thì ta được phương trình tương đương, do đó có thể giả sử $x,y,z$ đều dương. Áp dụng bất đẳng thức quen thuộc $a^2+b^2+c^2 \ge ab + bc + ca$, ta được
$$3xyz = (xy)^2+(xz)^2+(yz)^2 \ge x^2yz+xyz^2+xy^2z = xyz(x+y+z).$$
Do đó $x+y+z\le 3.$ Với việc $x,y,z$ dương, ta chỉ ra $x=y=z=1.$ Kết luận, phương trình đã cho có $4$ nghiệm nguyên là $(1,1,1),\ (1,-1,-1),\ (-1,1,-1)$ và $(-1,-1,1).$}
\end{gbtt}

\begin{gbtt}
Cho các số nguyên dương $x,y,z$ thỏa mãn biểu thức sau nhận giá trị nguyên
$$T=\dfrac{1}{x}+\dfrac{1}{y}+\dfrac{1}{z} +\dfrac{1}{xy}+\dfrac{1}{yz}+\dfrac{1}{zx}.$$
\begin{enumerate}[a,]
    \item Chứng minh rằng $x,y,z$ cùng tính chẵn lẻ.
    \item Tìm tất cả các bộ $x,y,z$ với $x<y<z$ thỏa mãn giả thiết.
\end{enumerate}
 \loigiai{
\begin{enumerate}[a,]
    \item Biến đổi tương đương giả thiết, ta có
    \[\dfrac{yz+zx+xy+z+x+y}{xyz}=T\Leftrightarrow x+y+z+xy+yz+zx = Txyz.\tag{*}\label{bonus111}\]
    Không mất tổng quát, ta chỉ cần quan tâm đến tính chẵn lẻ của $x.$
    \begin{itemize}
        \item \chu{Trường hợp 1.} Nếu $x$ chẵn, ta có $x,xy,zx,Txyz$ đều chẵn, vì thế từ (\ref{bonus111}) ta suy ra
        $$2\mid\tron{y+z+yz}=\tron{(y+1)(z+1)-1}.$$
        Như vậy, $(y+1)(z+1)$ là lẻ, và do đó cả hai số $y$ và $z$ đều chẵn.
        \item \chu{Trường hợp 2.} Nếu $x$ lẻ, bắt buộc các số còn lại đều lẻ, bởi lẽ chỉ cần có một số chẵn thì tất cả các số còn lại đều chẵn.
    \end{itemize}
    Ta đã chứng minh xong ý a.
    \item Từ giả thiết $x<y<z,$ ta xét các trường hợp sau
    \begin{itemize}
        \item \chu{Trường hợp 1.} Nếu $x\ge 3$ thì $y \ge 5$ và $z \ge 7.$ Khi đó
        $$T\le \dfrac{1}{3}+\dfrac{1}{5}+\dfrac{1}{7}+\dfrac{1}{15}+\dfrac{1}{35}+\dfrac{1}{21}.$$
        Do $T$ là số nguyên nên lúc này $T$ không dương, mâu thuẫn.
        \item \chu{Trường hợp 2.} Nếu $x=1$ thì  $y\ge 3$ và $z \ge 5.$ Khi đó
        $$T\le 1+\dfrac{1}{3}+\dfrac{1}{5}+\dfrac{1}{3}+\dfrac{1}{15}+\dfrac{1}{5}= \dfrac{32}{15}.$$
        Do $T > \dfrac{1}{x} = 1$ nên $m=2$. Thay vào (\ref{bonus111}) ta được
        $$1+y+z+y+yz+z =2yz \Leftrightarrow yz-2y-2z = 1\Leftrightarrow (y-2)(z-2) =5.$$
        Ta tìm ra $y=3$ và $z=7$ từ đây.
        \item \chu{Trường hợp 3.} Nếu $x=2$ thì $y\ge 4$ và $z \ge 6.$ Khi đó
        $$T\le \dfrac{1}{2}+\dfrac{1}{4}+\dfrac{1}{6}+\dfrac{1}{8}+\dfrac{1}{24}+\dfrac{1}{12}= \dfrac{7}{6}.$$ 
         Do $T > \dfrac{1}{x} = \dfrac{1}{2}$ nên $T=2.$ Thay vào (\ref{bonus111}) ta được
        $$ 2+y+z+2y+yz+2z =2yz \Leftrightarrow yz-3y-3z = 2\Leftrightarrow (y-3)(z-3) =11.$$
        Ta tìm ra $y=4$ và $z=14$ từ đây.
    \end{itemize}
      Như vậy, tất cả các bộ $(x,y,z)$ thỏa mãn là $(1,3,7)$ và $(2,4,14).$
\end{enumerate}}
\end{gbtt}

\begin{gbtt}
Giải phương trình nghiệm nguyên dương
$$101x^3-2019xy+101y^3=100.$$
\nguon{Titu Andreescu}
\loigiai{Phương trình đã cho tương đương với
$$101\left(x^{3}+y^{3}-20 x y-1\right)+x y+1=0.$$
Ta có $101$ là ước của $xy+1.$ Kết hợp với $xy>0$, ta suy ra 
$$xy+1\geqslant 101\Rightarrow xy\geqslant 100.$$ Hơn nữa, ta có đánh giá sau đây
$$1 \leq \dfrac{x y+1}{101}=1+x y(20-x-y)-(x+y)(x-y)^{2} \leq 1+x y(20-x-y)$$
hay là $x+y \leq 20$, nhưng từ bất đẳng thức $AM-GM$ thì ta phải có $x=y=10$, thử lại thì thấy thỏa mãn phương trình đề bài nên ta kết luận phương trình có duy nhất một nghiệm $\left ( x,y \right )=\left ( 10,10 \right ).$}
\end{gbtt}

\begin{gbtt}
Tìm tất cả các số nguyên $x,y$ với $y\ge 0$ thỏa mãn
\[x^2+2xy+y!=131.\]
\loigiai{Phương trình đã cho tương đương với
$$(x+y)^2=y^2-y!+131.$$
Bằng quy nạp, ta chứng minh được
\[y^2-y!+131<0,\text{ với mọi }y\ge 6.\]
Do $(x+y)^2\ge 0,$ ta chỉ cần phải xét các trường hợp $0\le y\le 5.$\\
Kết quả, có hai cặp $(x,y)$ thỏa yêu cầu là $(1,5), (-11,5).$}
\end{gbtt}

\begin{gbtt}
Giải phương trình nghiệm nguyên dương
\[\tron{1+x!}\tron{1+y!}=(x+y)!.\]
\nguon{Tạp chí Toán học và Tuổi trẻ, tháng 10 năm 2017}
\loigiai{
Trong bài toán này, ta xét các trường hợp sau.
\begin{enumerate}
    \item Nếu $x\ge 2$ và $y\ge 2,$ ta có $1+x!$ và $1+y!$ đều lẻ, kéo theo $(x+y)!$ lẻ, mâu thuẫn với việc $x+y\ge 4.$
	\item Nếu $x=0$ hoặc $y=0,$ ta giả sử $x=0.$ Thế vào phương trình đã cho, ta được
	$$2\tron{1+y!}=y!.$$
	Phương trình tương đương với $y!=-2.$ Ta không tìm được $y$ từ đây.
	\item Nếu $x=1$ hoặc $y=1,$ ta giả sử $x=1.$ Thế vào phương trình đã cho, ta được
	$$2(1+y!) = (y+1)!.$$
	Phương trình tương đương với $y!\cdot(y-1)=2.$ Ta tìm ra $y=2$ từ đây.
\end{enumerate}
Kết luận, phương trình đã cho có $2$ nghiệm tự nhiên là $(1,2)$ và $(2,1).$}
\end{gbtt}

\begin{gbtt}
Tìm tất cả các số nguyên dương $m,n$ thỏa mãn 
\[m !+n !=(m+n+3)^{2}.\]
\loigiai{
Không mất tính tổng quát, ta có thể giả sử $m \leq n$. Chú ý rằng
$$
(2 n+3)^{2} \geq(m+n+3)^{2}>n !.
$$
Mặt khác, với mọi $n\ge 6$ ta có
$$
(2 n+3)^{2} \leq 3 \cdot 4(n-1) n<n !.
$$
Điều này là mâu thuẫn, do vậy $n \leq 5$. Đến đây ta dễ dàng tìm ra được $m=4, n=5$. Do tính đối xứng của phương trình đầu nên $(4,5)$ và $(5,4)$ là hai cặp số cần tìm.}
\end{gbtt}

\begin{gbtt}
Tìm tất cả các số nguyên dương $w$, $x$, $y$ và $z$ sao cho $w!=x!+y!+z!$.
\nguon{Canadian Mathematical Olympiad 1983}
\loigiai{
Không mất tổng quát, ta giả sử $x\le y \le z< w.$ Nếu $y>x$ thì ta có
$$\heva{&x!\mid y! \\ &z!\mid w!}\Rightarrow y!\mid \tron{w!-y!-z!}\Rightarrow y!\mid x!\Rightarrow y\le x.$$
Ta thu được mâu thuẫn ở đây. Do vậy $x=y.$ Thế vào phương trình đã cho, ta được
\[w!=2y!+z!.\tag{*}\label{giaithua11}\]
Tới đây, ta xét hai trường hợp sau.
\begin{enumerate}
		\item Nếu $y<z,$ ta có $y+1\le z$ và $y+1\le w.$ Ta nhận xét rằng
		\begin{align*}
		    \heva{&(y+1)!\mid z! \\ &(y+1)!\mid w!}
		    &\Rightarrow (y+1)!\mid \tron{x!+y!}
		    \\&\Rightarrow (y+1)!\le x!+y!
		    \\&\Rightarrow (y+1)!\le 2y!
		    \\&\Rightarrow y+1\le 2
		    \\&\Rightarrow y=1.
		\end{align*}
		Thế $y=1$ trở lại phương trình (\ref{giaithua11}) ta được $$w!=2+z!.$$
		Do $z<w$ nên $z!\mid w!,$ từ đó $z!\mid 2.$ Ta có $z=1$ hoặc $z=2.$ Thế ngược lại, ta không tìm được $w.$
		\item Nếu $y=z,$ thế trở lại phương trình (\ref{giaithua11}) ta được 
		$$w!=3x!.$$
		Do $w>x$ nên $(x+1)!\mid w!=3x!,$ kéo theo $(x+1)\mid 3.$ Ta có $x=2,$ và ta tìm ra $y=z=2,w=3.$
\end{enumerate}
Như vậy, có duy nhất một bộ số nguyên dương thoả mãn yêu cầu là
$$(x, y, z, w)= (2, 2, 2, 3).$$}
\end{gbtt}

\begin{gbtt}
Xét phương trình $x^2+y^2+z^2=3xyz.$
\begin{enumerate}[a,]
    \item Tìm tất cả các nghiệm nguyên dương có dạng $\left( x,y,y \right)$ của phương trình đã cho.
    \item Chứng minh rằng tồn tại nghiệm nguyên dương $\left( a,b,c \right)$ của phương trình và thỏa mãn điều kiện 
    \[\min \left\{ a;b;c \right\}>2017.\]
\end{enumerate}
\nguon{Chuyên toán Vĩnh Phúc 2017 $-$ 2018}
\loigiai{
\begin{enumerate}[a,]
    \item Giả sử phương trình có nghiệm nguyên dương là $\left( x,y,y \right)$. Thay vào phương trình ta được 
$${{x}^{2}}+{{y}^{2}}+{{y}^{2}}=3x{{y}^{2}}\Leftrightarrow {{x}^{2}}+2{{y}^{2}}=3x{{y}^{2}}.$$
Từ đây suy ra $x$ chia hết cho $y.$ Đặt $x=ty$ với $t$ là số tự nhiên khác $0.$ Phương trình trở thành
$${{t}^{2}}{{y}^{2}}+2{{y}^{2}}=3t\cdot y\cdot {{y}^{2}}\Leftrightarrow {{t}^{2}}+2=3ty.$$
Đến đây ta suy ra $t \mid 2,$ hay là $t\in \left\{ 1;2 \right\}$. Xét hai trường hợp sau.
\begin{itemize}
    \item\chu{Trường hợp 1.} Với $t=1,$ ta có $x=y=1.$
    \item\chu{Trường hợp 2.} Với $t=2,$ ta có $x=2$ và $y=1.$
\end{itemize} 
Như vậy, phương trình đã cho có hai nghiệm nguyên dương dạng $\left( x,y,y \right)$ là $\left( 1,1,1 \right)$ và $\left( 2,1,1 \right)$.
\item Nhận thấy rằng $(1,2,5)$ là một nghiệm của phương trình đã cho, ta giả sử rằng $a=\min \left\{ a;b;c \right\}$ với $a<b<c$ thỏa mãn ${{a}^{2}}+{{b}^{2}}+{{c}^{2}}=3abc.$ Nếu như $(a+d,b,c)$ là nghiệm của phương trình, ta sẽ có
$${{\left( a+d \right)}^{2}}+{{b}^{2}}+{{c}^{2}}=3\left( a+d \right)bc.$$
Rút gọn đi $a^2+b^2+c^2=3abc$ ở hai vế, ta được
$$2ad+d^2=3bcd.$$
Ta nhận xét rằng $d=3bc-2a$ là một số tự nhiên khác $0,$ điều này dẫn đến phương trình có nghiệm $\left( a',b,c \right)$ với $a'=a+d$. Vì $a<b<c$, nên $\min \left\{ a';b;c \right\}>\min \left\{ a;b;c \right\}=a$.
Lặp lại quá trình trên sau không quá $2017$ lần ta được $\min \left\{ a;b;c \right\}>2017$.
\end{enumerate}}
\end{gbtt}

\begin{gbtt} \hfill
\begin{enumerate}[a,]
    \item Cho hai số nguyên $a,b$ thỏa mãn $a^3+b^3>0.$ Chứng minh rằng
    \[a^3+b^3\ge a^2+b^2.\]
    \item Tìm tất cả các số nguyên $x,y,z,t$ thỏa mãn đồng thời
\[x^3+y^3=z^2+t^2\text{ và }z^3+t^3=x^2+y^2.\]
\end{enumerate}
\nguon{Chuyên Toán Phổ thông Năng khiếu 2019}
\loigiai{
\begin{enumerate}[a,]
    \item Từ $a^3+b^3>0,$ ta suy ra  $(a+b)\tron{a^2+ab+b^2}>0.$ Do $$a^2+ab+b^2=\tron{a+\dfrac{b}{2}}^2+\dfrac{3b^2}{4}\ge 0$$ và dấu bằng của bất đẳng thức này không xảy ra nên $a+b>0.$ Lại do $a,b$ nguyên nên $a+b\ge 1.$ \\Ta xét các trường hợp sau.
    \begin{itemize}
        \item \chu{Trường hợp 1.} Nếu $a+b=1$ hay $b=1-a,$ bất đẳng thức trở thành
        $$a^3+(1-a)^3\ge a^2+(1-a)^2\Leftrightarrow a(a-1)\ge 0.$$
        Bất đẳng thức trên đúng với mọi $a$ nguyên, với dấu bằng xảy ra tại $(a,b)=(0,1),(1,0).$
        \item \chu{Trường hợp 2.} Nếu $a+b>1,$ ta có $a+b\ge 2.$ Như vậy    
        $$a^3+b^3=(a+b)\tron{a^2+ab+b^2}\ge 2\tron{a^2+ab+b^2}=\tron{a^2+b^2}+(a+b)^2\ge a^2+b^2.$$
        Trong trường hợp này, dấu bằng không thể xảy ra vì $a+b>0.$
    \end{itemize}
    Bất đẳng thức đã cho được chứng minh trong mọi trường hợp.
    \item Từ giả thiết, ta có $x^3+y^3\ge 0$ và $z^3+t^3\ge 0.$ Ta xét các trường hợp sau đây.
    \begin{itemize}
        \item \chu{Trường hợp 1.} Nếu $x^3+y^3=0$ thì $z^2+t^2=0.$ Ta suy ra $z=t=0.$ Thế trở lại, ta được
        $$x=y=z=t=0.$$
        \item \chu{Trường hợp 2.} Nếu $z^3+t^3=0$ thì $x^2+y^2=0.$ Ta suy ra $x=y=0.$ Thế trở lại, ta được
        $$x=y=z=t=0.$$
        \item \chu{Trường hợp 3.} Nếu $x^3+y^3>0$ và $z^3+t^3>0,$ theo như chứng minh ở câu a, ta thu được
        $$x^3+y^3\ge x^2+y^2,\quad z^3+t^3\ge z^2+t^2.$$
        Ta có các đánh giá
        \begin{align*}
            z^2+t^2=x^3+y^3\ge x^2+y^2,\quad x^2+y^2=z^3+t^3\ge z^2+t^2. 
        \end{align*}
        Dấu bằng trong các đánh giá trên bắt buộc phải xảy ra, tức là
        $$z^2+t^2=z^3+t^3=x^2+y^2=x^3+y^3.$$
        Theo như câu a, dấu bằng này xảy ra khi và chỉ khi 
        $$x(x-1)=y(y-1)=z(z-1)=t(t-1)=0,\: x+y=1,\: z+t=1.$$
        Thế trở lại rồi thử trực tiếp, ta kết luận có tất cả $6$ cặp $(x,y,z,t)$ thỏa yêu cầu, gồm
        $$\left( 0,0,0,0 \right),\: \left( 0,1,0,1 \right),\: \left( 0,1,1,0 \right),\: \left( 1,0,0,1 \right)\,\: \left( 1,0,1,0 \right),\:\left( 1,1,1,1 \right).$$
    \end{itemize}
\end{enumerate}}
\end{gbtt}

\section{Phương pháp lựa chọn modulo}

Mục này được tác giả đưa vào phần đầu phương trình nghiệm nguyên chỉ với một vài bài toán chứng minh phương trình vô nghiệm. Phương pháp này sẽ được nói rõ vào các mục khác trong cùng chương.


\subsection*{Bài tập tự luyện}

\begin{btt}
Giải phương trình nghiệm nguyên $$x^2-5y^2= 27.$$
\end{btt}

\begin{btt}
Giải phương trình nghiệm nguyên $$7x^2-5y^2=3.$$
\end{btt}

\begin{btt}
Chứng minh rằng phương trình $x^3+y^3=2013$ không có nghiệm nguyên.
\end{btt}

\begin{btt}
Giải phương trình nghiệm nguyên $$x^3+y^2-x+3y=2021.$$
\end{btt}

\begin{btt}
Chứng minh rằng phương trình sau không có nghiệm nguyên \[x^2+y^2+z^2=2015.\]
\end{btt}

\begin{btt}
Chứng minh rằng không tồn tại các số nguyên dương $x,y,z$ thỏa mãn \[x^4+y^4=7z^4+5.\] 
\nguon{Chuyên Khoa học Tự nhiên 2012}
\end{btt}

\begin{btt}
Chứng minh rằng không tồn tại các số nguyên dương $x,y,z$ thỏa mãn
\[x^3+10y^3+z^3=2021.\]
\end{btt}

\begin{btt}
Chứng minh rằng phương trình $x^5-y^2=4$ không có nghiệm nguyên.
\nguon{Balkan Mathematical Olympiad 1998}
\end{btt}

\begin{btt}
Chứng minh rằng không tồn tại các số nguyên $x,y$ thỏa mãn
\[12x^2+26xy+15y^2=4617.\]
\nguon{Chuyên Khoa học Tự nhiên 2018}
\end{btt}

\begin{btt}
Chứng minh rằng phương trình $x^3+y^4 = 7$ không có nghiệm nguyên.
\end{btt}

\subsection*{Hướng dẫn bài tập tự luyện}

\begin{gbtt}
Giải phương trình nghiệm nguyên $x^2-5y^2= 27.$
\loigiai{
Lấy đồng dư theo modulo $5$ hai vế, ta được $x^2\equiv 2\pmod{5}.$ Không có số nguyên nào thỏa mãn điều kiện trên, chứng tỏ phương trình đã cho không có nghiệm nguyên.}
\end{gbtt}

\begin{gbtt}
Giải phương trình nghiệm nguyên $7x^2-5y^2=3.$
\loigiai{
Phương trình đã cho tương đương với
$$6x^2-6y^2+\left(x^2+y^2\right)=3.$$
Ta suy ra $x^2+y^2$ chia hết cho $3.$ Theo như kiến thức đã học ở các chương trước, ta có
$$x^2+y^2\equiv 0\pmod{3}\Leftrightarrow \heva{&3\mid x \\ &3\mid y.}$$
Chính vì thế, vế trái chia hết cho $9,$ tuy nhiên do $3$ không chia hết cho $9$ nên phương trình đã cho vô nghiệm nguyên.
}
\end{gbtt}

\begin{gbtt}
Chứng minh rằng phương trình $x^3+y^3=2013$ không có nghiệm nguyên.
\loigiai{
Lập phương một số nguyên chỉ nhận các số dư $0,1,8$ khi chia cho $9.$ Như vậy
$$y^3\equiv 2013-x^3\equiv 6-x^3\equiv 6,5,-2 \equiv 5,6,7\pmod{9}.$$
Lập luận trên chứng tỏ số dư của $y^3$ khi chia cho $9$ khác $0,1,8.$ \\Như vậy, phương trình đã cho không có nghiệm nguyên.
}
\end{gbtt}

\begin{gbtt}
Giải phương trình nghiệm nguyên $x^3+y^2-x+3y=2021$.
\loigiai{
Ta đã biết $x^3-x=x(x+1)(x-1)$ chia hết cho $3$ do đây là tích ba số nguyên liên tiếp. \\
Trong phương trình ban đầu, lấy đồng dư theo modulo $3$ hai vế, ta được
$$y^2\equiv 2\pmod{3}.$$
Đây là điều không thể xảy ra. Phương trình đã cho không có nghiệm nguyên.}   
\end{gbtt}

\begin{gbtt}
Chứng minh rằng phương trình sau không có nghiệm nguyên \[x^2+y^2+z^2=2015.\]
	\loigiai{
	Ta nhận thấy rằng $x^2+y^2+z^2$ là số lẻ nên trong ba số $x^2$, $y^2$, $z^2$ phải có một số lẻ và hai số chẵn, hoặc ba số đều lẻ. Ngoài ra, ta đã biết, một số chính phương chẵn thì chia hết cho $4$, còn số chính phương lẻ thì chia cho $4$ dư $1$ và chia cho $8$ cũng dư $1$. Ta xét các trường hợp sau đây.
\begin{enumerate}
			\item Nếu có một số lẻ và hai số chẵn, vế trái của phương trình $x^2+y^2+z^2=2015$ chia cho $4$ dư $1$, còn vế phải (là $2015$) chia cho $4$ dư $3$, mâu thuẫn.
			\item Nếu cả ba số đều lẻ, vế trái của phương trình $x^2+y^2+z^2=2015$ chia cho $8$ dư $3$, còn vế phải (là $2015$) chia cho $8$ dư $7$, mâu thuẫn.
\end{enumerate}
	Vậy phương trình đã cho không có nghiệm nguyên.}
\begin{luuy}
Bạn đọc có thể lập bảng đồng dư theo modulo $8$ cho bài trên.
\end{luuy}	
\end{gbtt}

\begin{gbtt}
Chứng minh rằng không tồn tại các số nguyên dương $x,y,z$ thỏa mãn \[x^4+y^4=7z^4+5.\] 
\nguon{Chuyên Khoa học Tự nhiên 2012}
\loigiai{
Giả sử tồn tại các số nguyên dương $x,y,z$ thỏa mãn đề bài. Lấy đồng dư theo modulo $16$ hai vế, ta được
\[x^4+y^4+z^4 \equiv 5 \pmod{16}.\tag{1}\label{hsgs1212}\]
Một lũy thừa mũ $4$ chỉ đồng dư với $0$ hoặc $1$  theo modulo $16.$ Với các số $a,b,c$ thuộc tập $\{0;1\}$ thỏa mãn 
$$x^4 \equiv a \pmod{16},\qquad y^4 \equiv b \pmod{16}, \qquad  z^4 \equiv c \pmod{16},$$ 
ta có những nhận xét là
$$\min(a+b+c)=0,\qquad \max(a+b+c)=3.$$ 
Suy luận trên kết hợp với việc $x^4+y^4+z^4 \equiv a+b+c \pmod{16}$ giúp ta chỉ ra
\[x^4+y^4+z^4 \equiv 0,1,2,3 \pmod{16}.\tag{2}\label{hsgs12122}\]
Đối chiếu (1) và (2), ta chỉ ra điều mâu thuẫn. Giả sử phản chứng là sai. Chứng minh hoàn tất.}
\end{gbtt}

\begin{gbtt}
Chứng minh rằng không tồn tại các số nguyên dương $x,y,z$ thỏa mãn
\[x^3+10y^3+z^3=2021.\]
\loigiai{Giả sử tồn tại các số nguyên $x,y,z$ thỏa mãn đề bài. Lấy đồng dư theo modulo $8$ hai vế, ta được
\[x^3+10y^3+z^3\equiv5\pmod{8}.\tag{1}\label{fakehsgs12112}\]
Một lũy thừa mũ $3$ chỉ đồng dư với $0$ hoặc $1$ theo modulo $8$. Với các số  $a,b,c$ thuộc tập $\left\{0;1\right\}$ thỏa mãn
$$x^3 \equiv a \pmod{8},\qquad y^3 \equiv b \pmod{8} \qquad  z^3 \equiv c \pmod{8},$$
ta có nhận xét là
$a+10b+c\in \left\{0;1;2;10;11;12\right\}.$\\
Suy luận trên kết hợp với việc $x^3+10y^3+z^3\equiv a+10b+c\pmod{8}$ giúp ta chỉ ra 
\[x^3+10y^3+z^3\equiv0,1,2,10,11,12\pmod{8}.\tag{2}\label{fakehsgs12122}\]
Đối chiếu (\ref{fakehsgs12112}) và (\ref{fakehsgs12122}), ta chỉ ra điều mâu thuẫn. Giả sử phản chứng là sai. Chứng minh hoàn tất.}
\end{gbtt}

\begin{gbtt}
Chứng minh rằng phương trình $x^5-y^2=4$ không có nghiệm nguyên.
\nguon{Balkan Mathematical Olympiad 1998}
\loigiai{
Theo như kiến thức đã học ở \chu{chương I}, ta biết rằng với mọi số nguyên $x,$ ta luôn có
$$x^5\equiv -1,0,1\pmod{11}.$$
Như vậy, lấy đồng dư theo modulo $11$ hai vế phương trình ban đầu, ta được
$$y^2\equiv 6,7,8\pmod{11}.$$
Tuy nhiên, điều này không thể xảy ra với $y$ nguyên do
$$y^2\equiv0,1,3,4,5,9\pmod{11}.$$
Phương trình đã cho không có nghiệm nguyên!} 
\end{gbtt}

\begin{gbtt}
Chứng minh rằng không tồn tại các số nguyên $x,y$ thỏa mãn
\[12x^2+26xy+15y^2=4617.\]
\nguon{Chuyên Khoa học Tự nhiên 2018}
\loigiai{
Giả sử phương trình đã cho có nghiệm nguyên. Ta có
\[(x+2y)^2+11(x+y)^2=4617.\]
Lấy đồng dư theo modulo $11$ hai vế, ta được $(x+2y)^2\equiv 8\pmod{11}.$ \\
Ta lập bảng đồng dư modulo $11$ như sau
\begin{center}
    \begin{tabular}{c|c|c|c|c|c|c}
       $x+2y$  &  $0$ & $\pm 1$ & $\pm 2$ & $\pm 3$ & $\pm 4$ & $\pm 5$ \\
       \hline
        $(x+2y)^2$ & $0$ & $1$ & $4$ & $9$ & $5$ & $3$
    \end{tabular}
\end{center}
Không có số chính phương nào chia $11$ dư $8.$ Giả sử sai. Bài toán được chứng minh.}
\end{gbtt}

\begin{gbtt}
Chứng minh rằng phương trình $x^3+y^4 = 7$ không có nghiệm nguyên.
\loigiai{
Giả sử phương trình đã cho có nghiệm nguyên. Ta lập bảng đồng dư theo modulo $13$ sau đây.
\begin{center}
    \begin{tabular}{c|c|c|c|c|c|c|c}
       $y$  & $0$ & $\pm 1$ & $\pm 2$ & $\pm 3$ & $\pm 4$ & $\pm 5$ & $\pm 6$\\
       \hline
        $y^4$ & $0$ & $1$ & $3$ & $3$ & $9$ & $1$ & $9$ \\
        \hline
        $x^3=7-y^4$ & $7$ & $6$ & $4$ & $4$ & $11$ & $6$ & $11$
    \end{tabular}
\end{center}
Tiếp theo, ta sẽ lập thêm một bảng đồng dư của $x$ và $x^3$
\begin{center}
    \begin{tabular}{c|c|c|c|c|c|c|c|c|c|c|c|c|c}
       $x$  & $0$ & $1$ & $2$ & $3$ & $4$ & $5$ & $6$ & $7$ & $8$ & $9$ & $10$ & $11$ & $12$ \\
       \hline
       $x^3$ & $0$ & $1$ & $8$ & $3$ & $12$ & $1$ & $8$ & $9$ & $5$ & $9$ & $12$ & $3$ & $12$
    \end{tabular}
\end{center}
Đối chiếu hai bảng đồng dư vừa rồi, ta thấy giả sử phản chứng sai. Bài toán được chứng minh.}
\end{gbtt}
\section{Tính chia hết và phép cô lập biến số}
Đa số, tính chia hết được thể hiện trong các bài toán về phương trình nghiệm nguyên thông qua việc biểu diễn một ẩn theo ẩn (hoặc các ẩn) còn lại. Dưới đây là một vài ví dụ minh họa.

\subsection*{Ví dụ minh họa}

\begin{bx}
    Giải phương trình nghiệm nguyên $5x+11y=125.$
\loigiai{Xét tính chia hết cho $5$ ở cả hai vế, ta chỉ ra $11y$ chia hết cho $5,$ nhưng vì $(11,5)=1$ nên $y$ chia hết cho $5.$ \\
Ta đặt $y=5t,$ với $t$ là số nguyên. Phương trình đã cho trở thành
$$5x+11\cdot5t=125 \Leftrightarrow x+11t=25\Leftrightarrow x=25-11t.$$ 
Như vậy, phương trình đã cho có vô số nghiệm nguyên $(x,y)$, và chúng được biểu diễn dưới dạng
 $$\heva{&x=25-11t \\ &y=5t}, \,\, \text{ $t$ là số nguyên tùy ý.}$$}
\end{bx}

\begin{bx}
Giải phương trình nghiệm nguyên dương  $x^2y+2xy+y=32x.$
\nguon{Chuyên Toán Vĩnh Long 2021}
\loigiai{
Phương trình đã cho tương đương $y(x+1)^2=32x.$ Với giả sử phương trình tồn tại nghiệm $(x,y),$ ta có
    \[(x+1)\mid 32x\Rightarrow(x+1)\mid \bigg(32(x+1)-32x\bigg)\Rightarrow (x+1)\mid 32.\]
Ta nhận được $x+1$ là ước nguyên dương lớn hơn $1$ của $32.$ Ta lập bảng giá trị sau
       \begin{center}
       \begin{tabular}{c|c|c|c|c|c}
            $x+1$ & $2$ & $4$ & $8$ & $16$ & $32$ \\
            \hline
            $x$ & $1$ & $3$ & $7$ & $15$ & $31$ \\ 
            \hline
            $y$ & $8$ & $6$ & $3,5$ & $1,875$ & $0,96875$ 
            \end{tabular}
        \end{center}
    Căn cứ vào bảng giá trị, ta kết luận phương trình có $2$ nghiệm nguyên phân biệt là $(1,8)$ và $(3,6).$}
\end{bx}

\subsection*{Bài tập tự luyện}

\begin{btt}
Giải phương trình nghiệm nguyên $$22x+36y=240.$$
\end{btt}

\begin{btt}
Chứng minh rằng phương trình $33x+1001y=121212$ không có nghiệm nguyên dương.
\end{btt}

\begin{btt}
Giải phương trình nghiệm nguyên $$6x+15y+10z=3.$$
\end{btt}

\begin{btt}
Cho ba số nguyên $a,b,c$ thỏa mãn $$a=b-c=\dfrac{b}{c}.$$ Chứng minh rằng $a+b+c$ là lập phương của một số nguyên.
\nguon{Chuyên Toán Bình Dương 2021}
\end{btt}

\begin{btt}
Giải phương trình nghiệm nguyên $$x^3-xy+1=2y-x.$$
\end{btt}

\begin{btt}
Giải phương trình nghiệm nguyên $$x^2y^2-4x^2y+y^3+4x^2-3y^2+1=0.$$
\nguon{Chuyên Đại học Sư phạm Hà Nội 2019}
\end{btt}

\begin{btt}
Giải phương trình nghiệm nguyên $$\left(x^2y-xy+y\right)(x+y)=3x+1.$$
\nguon{Chuyên Toán Phú Thọ 2021}
\end{btt}

\begin{btt}
Giải phương trình nghiệm tự nhiên
\[x^2y^2(x+y)+x+y=xy+3.\]
\end{btt}

\begin{btt}
Tìm tất cả các cặp số nguyên $\left( {x,y} \right)$ thỏa mãn $\left( {{x^2} - x + 1} \right)\left( {{y^2} + xy} \right) = 3x - 1$.
\nguon{Chuyên Khoa học Tự nhiên Hà Nội năm học 2019 $-$ 2020}
\end{btt}

\begin{btt}
Giải phương trình nghiệm nguyên
\[x^3y+xy^3+2x^2y^2-4x-4y+4=0.\]
\end{btt}

\begin{btt}
Tìm tất cả các cặp số nguyên $(x,y)$ thỏa mãn
$$7(x+2y)^3(y-x)=8y-5x+1.$$
\nguon{Chuyên Toán Ninh Bình 2021}
\end{btt}

\begin{btt}
Tìm tất cả các số nguyên dương \(x,y\) sao cho $(x,y)=1$ và \[2\left ( x^3-x \right )=y^3-y.\]
\end{btt}

\subsection*{Hướng dẫn bài tập tự luyện}

\begin{gbtt}
Giải phương trình nghiệm nguyên $22x+36y=240.$
\loigiai{
Phương trình đã cho tương đương với
$$11x+18=120\Leftrightarrow 11(x-6)+18(y-3)=0.$$
Xét tính chia hết cho $11$ ở cả hai vế, ta chỉ ra $18(y-3)$ chia hết cho $11,$ nhưng vì $(18,11)=1$ nên $y-3$ chia hết cho $11.$ Ta đặt $y-3=11t,$ với $t$ là số nguyên. Phương trình đã cho trở thành
$$11(x-6)+18\cdot11t =0\Leftrightarrow x-6+18t=0\Leftrightarrow x=6-18t.$$ 
Như vậy, phương trình đã cho có vô số nghiệm nguyên $(x,y)$, và chúng được biểu diễn dưới dạng
        $$\heva{&x=-18t+6 \\ &y=11t+3}, \,\, \text{$t$ là số nguyên tùy ý.}$$}
    \begin{luuy}
Việc tạo ra các nhóm $x-6$ và $y-3$ như bên trên là nhờ vào việc nhẩm được $(x,y)=(6,3)$ là một nghiệm riêng của phương trình. Tác giả xin phát biểu và không chứng minh bổ đề
\begin{enumerate}
    \item Phương trình $ax+by=c$ có nghiệm nguyên khi và chỉ khi $c$ chia hết cho $(a,b).$
    \item Cho các số nguyên $a,b,c$ thỏa mãn $(a,b,c)=1.$ Nếu phương trình $ax+by=c$ có nghiệm riêng $\left(x_0,y_0\right),$ nghiệm tổng quát của phương trình sẽ là
        $$\heva{&x=bt+x_0 \\ &y=-at+y_0} \,\, \text{($t$ là số nguyên tùy ý).}$$    
\end{enumerate}
\end{luuy}
\end{gbtt}

\begin{gbtt}
Chứng minh rằng phương trình $33x+1001y=121212$ không có nghiệm nguyên dương.
\loigiai{
Do $121212$ không chia hết cho $(33,1001)=11$ nên theo bổ đề vừa phát biểu, phương trình đã cho không có nghiệm nguyên dương (và thậm chí là nghiệm nguyên).} 
\end{gbtt}

\begin{gbtt}
Giải phương trình nghiệm nguyên $6x+15y+10z=3.$
\loigiai{
Giả sử phương trình có nghiệm $(x,y,z).$ Ta nhận thấy cả $6x,15y$ và $3z$ đều chia hết cho $3,$ vậy nên ta suy ra $10z$ chia hết cho $3,$ nhưng vì $(10,3)=1$ nên $z$ chia hết cho $3.$ Đặt $z=3k$ với $k$ là một số nguyên. Phương trình đã cho trở thành
	$$6x+15y+30k=3 \Leftrightarrow 2x+5y+10k=1\Leftrightarrow 2x+5y=1-10k.$$ 
Với mọi số nguyên $k,$ ta nhận thấy $1-10k$ luôn chia hết cho $(2,5)=1.$ Chỉnh vì lẽ đó, phương trình trên luôn có nghiệm nguyên dương. Cộng thêm việc nhẩm ra nghiệm riêng của phương trình ẩn $x,y$ và tham số $k$
$$2x+5y=1-10k$$ 
là $(x,y)=(-5k-2,1),$ ta kết luận tất cả các nghiệm $(x,y,z)$ của phương trình là $(5t-5k-2, 1-2t, 3k)$, trong đó $t,k$ là những số nguyên tùy ý.
}
\begin{luuy}
	\nx\\
	Trong cách giải trên, ta đã biến đổi phương trình đã cho về dạng $2x+5y=1-10k$. Khi các hệ số của $x$ và $y$ là hai số nguyên tố cùng nhau, ta tiến hành giải phương trình bậc nhất với hai ẩn là $x$ và $y$ như cách giải phương trình bậc nhất dạng $ax+by=c.$
\end{luuy}
\end{gbtt}	

\begin{gbtt}
Cho ba số nguyên $a,b,c$ thỏa mãn $a=b-c=\dfrac{b}{c}.$ Chứng minh rằng $a+b+c$ là lập phương của một số nguyên.
\nguon{Chuyên Toán Bình Dương 2021}
\loigiai{
Xuất phát từ $b-c=\dfrac{b}{c},$ ta có
\[c(b-c)=b\Rightarrow bc-c^2=b\Rightarrow b(c-1)=c^2.\tag{*}\label{bd1}\]
Ta được $c^2$ chia hết cho $c-1,$ chứng tỏ 
$$c-1\mid c^2-1+1=(c-1)(c+1)+1.$$
Do $c-1$ là ước của $1$ và $c\ne 0,$ ta có $c=2.$ Thế vào (\ref{bd1}), ta suy ra $b=4.$ Như vậy
$$a+b+c=b-c+b+c=2b=8.$$
Ta nhận được $a+b+c$ là số lập phương. Bài toán được chứng minh.}
\end{gbtt}

\begin{gbtt}
Giải phương trình nghiệm nguyên $x^3-xy+1=2y-x.$
\loigiai{Phương trình đã cho tương đương với 
$$x^3+x+1=y\tron{2+x}.$$
Ta nhận được $\tron{2+x}\mid \tron{x^3+x+1},$ thế nên 
$$\tron{x+2}\mid\bigg(\tron{x^3+8}+\tron{x+2}-9\bigg)\Rightarrow\tron{x+2}\mid 9.$$
Ta suy ra $x+2$ là ước nguyên của $9$. Ta lập bảng giá trị
\begin{center}
    \begin{tabular}{c|c|c|c|c|c|c}
        $x+2$ & $-9$ &$-3$&$-1$&$1$&$3$&$9$\\
        \hline
         $x$ &$-11$&$-5$&$-3$&$-1$&$1$&$7$\\
         \hline
         $y$&$149$&$43$&$29$&$-1$&$1$&$39$
    \end{tabular}
\end{center}
Như vậy, phương trình đã cho có $6$ nghiệm nguyên $\tron{x,y}$ là 
$$\tron{-11,149},\tron{-5,43},\tron{-3,29},\tron{-1,-1},\tron{1,1},\tron{7,39}.$$
}

\end{gbtt}

\begin{gbtt}
Giải phương trình nghiệm nguyên $x^2y^2-4x^2y+y^3+4x^2-3y^2+1=0.$
\nguon{Chuyên Đại học Sư phạm Hà Nội 2019}
\loigiai{
Biến đổi phương trình đã cho ta được
$$x^2\tron{y^2-4y+4}=-y^3+3y^2-1.$$
Từ đây, ta suy ra $\tron{y^2-4y+4}\mid \tron{-y^3+3y^2-1}.$ Điều này dẫn đến $\tron{y^2-4y+4}\mid \tron{-y^2+4y-1},$ kéo theo $\tron{y^2-4y+4}\mid 3.$ Vì $y^2-4y+4=\tron{y-2}^2\ge0$ nên ta có các trường hợp sau.
\begin{enumerate}
    \item Với $y^2-4y+4=1,$ ta suy ra $y=1$ hoặc $y=3.$ Thử trực tiếp, ta thu được cặp $(x,y)$ nguyên dương thỏa mãn là $(1,1),(3,-1).$
    \item Với $y^2-4y+4=3,$ ta nhận thấy không có $y$ nguyên thỏa mãn.
\end{enumerate}
Như vậy, các nghiệm $(x,y)$ nguyên dương của phương trình là $(1,1)$ và $(3,-1).$}
\end{gbtt}

\begin{gbtt}
Giải phương trình nghiệm nguyên $\left(x^2y-xy+y\right)(x+y)=3x+1.$
\nguon{Chuyên Toán Phú Thọ 2021}
\loigiai{
Phương trình đã cho tương đương với
    $$y\left(x^{2}-x+1\right)(x+y)=3 x-1 .$$
    Ta nhận được $\left(x^{2}-x+1\right)\mid (3x-1),$ thế nên
    \begin{align*}
        \left(x^{2}-x+1\right)\mid (3x-1)(3x-2)&\Rightarrow \left(x^{2}-x+1\right)\mid \left(9x^{2}-9x+9-7\right)\\&\Rightarrow \left(x^{2}-x+1\right)\mid 7.
    \end{align*}
    Lần lượt cho $x^2-x+1$ nhận các giá trị là $-7,-1,1,7$ ta kết luận rằng phương trình đã cho có ba nghiệm nguyên là $(-2,1),(1,-2),(1,1).$}
\end{gbtt}
\begin{gbtt}
Tìm tất cả các cặp số nguyên $\left( {x,y} \right)$ thỏa mãn $\left( {{x^2} - x + 1} \right)\left( {{y^2} + xy} \right) = 3x - 1$.
\nguon{Chuyên Khoa học Tự nhiên Hà Nội năm học 2019 $-$ 2020}
\loigiai{Do $\left( {{x^2} - x + 1} \right)\left( {{y^2} + xy} \right) = 3x - 1$ nên $3x - 1$ chia hết cho ${x^2} - x + 1$, từ đây suy ra $\left( {3x - 1} \right)\left( {3x - 2} \right)$ chia hết cho ${x^2} - x + 1$, điều này đồng nghĩa với việc $9\left( {{x^2} - x + 1} \right) - 7$ chia hết cho ${x^2} - x + 1$. Như vậy ${x^2} - x + 1$ là ước của 7. Ta có nhận xét rằng
$${x^2} - x + 1 = {\left( {x - \dfrac{1}{2}} \right)^2} + \dfrac{3}{4} > 0.$$
Do vậy mà ${x^2} - x + 1 = 1$ hoặc ${x^2} - x + 1 = 7$. Từ đây ta chỉ ra rằng 
$$x \in \left\{ { - 2;0;1;3} \right\}.$$ 
Đến đây, ta xét lần lượt các trường hợp sau.
 \begin{enumerate}
     \item Với $x =  - 2$, thay vào phương trình ban đầu ta được 
     $${y^2} - 2y =  - 1 \Leftrightarrow {\left( {y - 1} \right)^2} = 0\Leftrightarrow y = 1.$$
    \item  Với $x = 0$, thay vào phương trình ban đầu ta được $y^2=  - 1.$ Phương trình này vô nghiệm.
    \item Với $x = 1$, thay vào phương trình ban đầu ta được $${y^2} + y = 2 \Leftrightarrow \left( {y - 1} \right)\left( {y + 2} \right) = 0\Leftrightarrow y \in \left\{ { - 2;1} \right\}.$$
    \item Với $x = 3$, ta có $3x - 1 = 8$ và ${x^2} - x + 1 = 7$ nên $3x - 1$ không chia hết cho ${x^2} - x + 1$.
 \end{enumerate}
Vậy có ba cặp số nguyên thỏa mãn phương trình là $\left( {x,y} \right) = \left( { - 2,1} \right),\left( {1, - 2} \right),\left( {1,1} \right)$.}
\end{gbtt}
\begin{gbtt}
Giải phương trình nghiệm tự nhiên
\[x^2y^2(x+y)+x+y=xy+3.\]
\loigiai{
Phương trình đã cho tương đương với 
\[\tron{x+y}\tron{x^2y^2+1}=xy+3.\tag{*}\label{mistake001}\]
Từ đây, ta nhận được $\tron{x^2y^2+1}\mid\tron{xy+3}.$ Do $xy+3>0,$ phép chia hết kể trên cho ta
    $$x^2y^2+1\le xy+3\Rightarrow (xy)^2-xy-2\le 0\Rightarrow (xy+1)(xy-2)\le 0\Rightarrow -1\le xy\le 2.$$
Do $xy\ge 0$ nên $xy\in\{0;1;2\}.$ Tới đây, ta lập bảng giá trị sau. 
    \begin{center}
    \begin{tabular}{c|c|c|c}
        $xy$ & $0$ & $1$ & $2$ \\
        \hline
        $x+y$ & $3$ & $2$ & $1$ \\
        \hline
        $(x,y)$ & $(3,0)$ hoặc $(0,3)$ & $(1,1)$ & $\notin\mathbb{Z}^2$
    \end{tabular}
    \end{center}
    Căn cứ vào bảng vừa lập, phương trình đã cho có các nghiệm tự nhiên $\tron{x,y}$ là $$\tron{0,3},\quad\tron{3,0},\quad \tron{1,1}.$$}
\end{gbtt}

\begin{gbtt}
Giải phương trình nghiệm nguyên
\[x^3y+xy^3+2x^2y^2-4x-4y+4=0.\]
\loigiai{
Phương trình đã cho tương đương với
$$xy(x^2+2xy+y^2)-4(x+y)+4=0.$$
Đặt $S=x+y,P=xy.$ Phương trình trở thành
$$PS^2-4S+4=0\Leftrightarrow PS^2=4S-4.$$
Ta suy ra $4S-4$ chia hết cho $S,$ hay $S\in\{-4;-2;-1;1;2;4\}.$ Ta lập bảng giá trị
\begin{center}
    \begin{tabular}{c|c|c|c|c|c|c}
      $S$   & $-4$ & $-2$ & $-1$ & $1$ & $2$ & $4$ \\
      \hline
      $P$  & $-1,25$ & $-3$ & $-8$ & $0$ & $1$ & $0,75$ \\
      \hline
      $(x,y)$ & $\notin\mathbb{Z}^2$ & $(-3,1)$ và $(1,-3)$ & $\notin\mathbb{Z}^2$ & $(0,1)$ và $(1,0)$ & $(1,1)$ & $\notin\mathbb{Z}^2$
    \end{tabular}
\end{center}
Kết luận, phương trình đã cho có tất cả $5$ nghiệm nguyên là 
$$(1,0),\quad (0,1),\quad (1,1), \quad (1,-3),\quad (-3,1).$$}
\end{gbtt}

\begin{gbtt}
Tìm tất cả các cặp số nguyên $(x,y)$ thỏa mãn
$7(x+2y)^3(y-x)=8y-5x+1.$
\nguon{Chuyên Toán Ninh Bình 2021}
\loigiai{
Ta đặt $A=x+2y,B=y-x.$ Bằng biểu diễn $8y-5x=A+6B,$ phương trình đã cho trở thành
\[7A^3B=A+6B+1\Leftrightarrow \left(7A^3-6\right)B=A+1.\tag{*}\]
Ta được $\left(7A^3-6\right)\mid \left(A+1\right).$ Ta lần lượt suy ra
$$\left(7A^3-6\right)\mid 7\left(A+1\right)\left(A^2-A+1\right)\Rightarrow \left(7A^3-6\right)\mid \left(7A^3+7\right)\Rightarrow\left(7A^3-6\right)\mid 13.$$
Do $13$ là số nguyên tố, ta có $7A^3-6\in \{\pm 1;\pm 13\}.$ \\Kết hợp với điều kiện $A$ nguyên, ta tìm ra $A=1$ hoặc $A=-1.$ Ta lập bảng giá trị
\begin{center}
\begin{tabular}{c|c|c|c}
    $A$ & $B$ & $x$ & $y$ \\
    \hline
    $-1$ & $0$ & $\not\in\mathbb{Z}$ & $\not\in\mathbb{Z}$ \\
    \hline
    $-1$ & $2$ & $-1$ & $1$
\end{tabular}
\end{center}
Căn cứ vào bảng, ta nhận thấy $(x,y)=(-1,1)$ là nghiệm nguyên duy nhất của phương trình đã cho.}
\end{gbtt}

\begin{gbtt}
Tìm tất cả các số nguyên dương \(x,y\) sao cho $(x,y)=1$ và \[2\left ( x^3-x \right )=y^3-y.\]
\loigiai{
Áp dụng hằng đẳng thức quen thuộc là
$$a^{3}+b^{3}+c^{3}-3abc=(a+b+c)\left(a^{2}+b^{2}+c^{2}-a b-b c-c a\right),$$
ta viết lại phương trình đã cho thành
\begin{align*}
    x^{3}+x^{3}+(-y)^{3}-(2 x-y)=0 
    &\Leftrightarrow x^{3}+x^{3}+(-y)^{3}-3 x \cdot x\cdot(-y)+3 \cdot x \cdot x(-y)-(2 x-y)=0
    \\&\Leftrightarrow (x+x-y)\left(x^{2}+y^{2}+2 x y\right)-(2 x-y)=3 x^{2} y 
    \\&\Leftrightarrow(2 x-y)\left(x^{2}+y^{2}+2 x y-1\right)=3 x^{2} y.
\end{align*}
Từ đây ta suy ra được $(2x-y)\mid 3 x^{2} y$. Ta sẽ có
\[3x^{2}y=3x^{2}(2x-y)+6x^{3}\Rightarrow (2x-y)\mid 6x^{3}.\]
Bây giờ, ta sẽ chứng minh $\tron{2x-y,x^3}=1.$ Thật vậy, nếu $2x-y$ và $x^3$ có ước nguyên tố chung là $p$ thì
$$\heva{&p\mid \tron{2x-y}\\&p\mid x^3}
\Rightarrow\heva{&p\mid \tron{2x-y}\\&p\mid x}
\Rightarrow\heva{&p\mid y\\&p\mid x}
\Rightarrow p\mid (x,y)=1,$$
vô lí do $p$ nguyên tố. Như vậy $2x-y$ và $x^3$ không có ước nguyên tố chung nào, và $\tron{2x-y,x^3}=1.$ \\
Kết hợp với dữ kiện $(2x-y)\mid 6x^{3},$ ta suy ra $6$ chia hết cho $2x-y,$ và
$$2 x-y \in\{1 ; 2,3,6\}.$$ 
Giải lần lượt các trường hợp, ta kết luận phương trình có hai nghiệm nguyên dương là $(1,1)$ và $(4,5).$}
\end{gbtt} %đánh giá + xét mod + cô lập
\section{Phương trình nghiệm nguyên quy về dạng bậc hai}

\subsection*{Lí thuyết}

Trong mục này, chúng ta sẽ ôn tập lại một bổ đề quan trọng đã học ở \chu{chương III}.
\begin{light}
\chu{Bổ đề.} Cho phương trình $ax^2+bx+c=0$ với $a,b,c$ là các số nguyên và $a\ne 0.$ Phương trình có nghiệm nguyên chỉ khi $\Delta$ là số chính phương.
\end{light}
Song, chiều ngược lại là không đúng, vì chẳng hạn, phương trình
    $$12x^2+7x+1=0$$
có $\Delta=1$ là số chính phương, thế nhưng hai nghiệm $x=\dfrac{-1}{3}$ và $x=\dfrac{-1}{4}$ của nó đều không nguyên. \\
Tác giả đã minh họa bổ đề bằng các ví dụ trong chương trước đó. Vì thế, chương này sẽ củng cố các lí thuyết và phương pháp ấy dưới dạng bài tập tự luyện.

\subsection*{Bài tập tự luyện}

\begin{btt}
Giải phương trình nghiệm nguyên
\[x^2+3y^2+4xy+4y+2x-3=0.\]
\end{btt}

\begin{btt}
Cho phương trình $x^2-mx+m+2=0.$ Tìm tất cả các giá trị của $m$ để phương trình đã cho có các nghiệm nguyên.
\nguon{Chuyên Toán Bình Định 2021}
\end{btt}

\begin{btt}
Tìm tất cả các cặp số nguyên $(x,y)$ thỏa mãn
    $$x^2+2y^2-2xy-2x-4y+6=0.$$
\nguon{Chuyên Toán Thanh Hóa 2021}
\end{btt}

\begin{btt}
Giải phương trình nghiệm nguyên dương
\[x^2-y^2=xy+8.\]
\nguon{Chuyên Toán Bình Dương 2018}
\end{btt}

\begin{btt}
Cho $p$ là số nguyên tố sao cho tồn tại các số nguyên dương $x,y$ thỏa mãn
    $$x^3+y^3-p=6xy-8.$$
Tìm giá trị lớn nhất của $p.$
\nguon{Chuyên Toán Lào Cai 2021}
\end{btt}

\begin{btt}
Giải phương trình nghiệm nguyên
$$(mn+8)^{3}+(m+n+5)^{3}=(m-1)^{2}(n-1)^{2}.$$
\end{btt}

\begin{btt}
Tìm tất cả các cặp số nguyên dương $(m, n)$ và số nguyên tố $p$ thỏa mãn đồng thời
$$m+n=2019,\quad \dfrac{4}{m+3}+\dfrac{4}{n+3}=\dfrac{1}{p}.$$
\nguon{Titu Andreescu}
\end{btt}

\begin{btt}
Giải phương trình nghiệm nguyên dương
\[2x(xy-2y-3)=(x+y)(3x+y).\]
\nguon{Đề nghị Olympic 30/4 năm 2018}
\end{btt}

\begin{btt}
Tìm các số tự nhiên $x,y$ thỏa mãn phương trình
\[y^3-2y^2x+x^2+y^2+4x+3y+3=0.\]
\end{btt}

\begin{btt}
Giải phương trình nghiệm nguyên dương \[\tron{x^2-y}\tron{x+y^2}=(x+y)^3.\]
\end{btt}

\begin{btt}
Tìm tất cả các cặp số nguyên $\left ( a,b \right )$ thỏa mãn
\[\left [ b^2+11\left ( a-b \right ) \right ]^2=a^3b.\]
\nguon{Hong Kong Mathematical Olympiads 2014}
\end{btt}

\begin{btt}
Giải hệ phương trình nghiệm nguyên
\[\heva{
\left(x^{2}+1\right)\left(y^{2}+1\right)+\dfrac{z^{2}}{10}&=2010 \\
(x+y)(x y-1)+14 z&=1985.}\]
\nguon{Titu Andreescu}
\end{btt}

\subsection*{Hướng dẫn bài tập tự luyện}

\begin{gbtt}
Giải phương trình nghiệm nguyên
\[x^2+3y^2+4xy+4y+2x-3=0.\]
\loigiai{
Phương trình đã cho tương đương với
\[3y^2+(4x+4)y+\tron{x^2+2x-3}=0.\tag{*}\label{baimodaub2}\]
Coi đây là một phương trình bậc hai theo ẩn $y.$ Ta tính được
$$\Delta'_y=(2x+2)^2-3\tron{x^2+2x-3}=x^2+2x+13.$$
 Phương trình có nghiệm nguyên chỉ khi $\Delta'_y$ là số chính phương. Ta đặt $z^2=x^2+2x+13,$ ở đây $z$ là số tự nhiên. Biến đổi tương đương phép đặt cho ta
 $$(x+1-z)(x+1+z)=-12.$$
Do $x+1-z<x+1+z$ và $x+1-z,x+1+z$ cùng tính chẵn lẻ, ta xét các trường hợp sau.
\begin{enumerate}
    \item Nếu $x+1-z=-6$ và $x+1+z=2$ thì $x=-3.$ Thế vào (\ref{baimodaub2}), ta được $y=0.$ 
    \item Nếu $x+1-z=-2$ và $x+1+z=6$ thì $x=1.$ Thế vào (\ref{baimodaub2}), ta được $y=0.$ 
\end{enumerate}
Kết luận, phương trình đã cho có $2$ nghiệm nguyên là $(-3,0)$ và $(1,0).$}
\end{gbtt}

\begin{gbtt}
Cho phương trình $x^2-mx+m+2=0.$ Tìm tất cả các giá trị của $m$ để phương trình đã cho có các nghiệm nguyên.
\nguon{Chuyên Toán Bình Định 2021}
\loigiai{Giả sử phương trình đã cho có 2 nghiệm nguyên $x_1,x_2$, áp dụng định lí $Viete$, ta có $x_1+x_2=m$, suy ra $m$ là số nguyên. Do phương trình đã cho có nghiệm nguyên nên 
$$\Delta =m^2-4(m+2)=(m-2)^2-12$$ phải là số chính phương. Đặt $(m-2)^2-12=a^2$ với $a\in\mathbb N$. Phép đặt này cho ta $$(m-2-a)(m-2+a)=12.$$ Với chú ý $m-2-a<m-2+a$ và $m-2-a\equiv m-2+a\pmod{2}$, ta có bảng sau
    \begin{center}
            \begin{tabular}{c|c|c}
            $m-2-a$ & $-6$ & $2$   \\
            \hline
            $m-2+a$ & $-2$ & $6$ \\
            \hline
            $m$ & $-2$ & $6$  \\
            \end{tabular}
        \end{center}
Từ đây, ta sẽ đi xem xét các trường hợp trên bảng
\begin{enumerate}
    \item Với $m=-2$, phương trình $x^2+2x=0$ có 2 nghiệm là 0 và $-2$, thỏa mãn.
    \item Với $m=6$, phương trình $x^2-6x+8=0$ có 2 nghiệm là 2 và 4, thỏa mãn.
\end{enumerate}
    Vậy các giá trị của $m$ thỏa mãn yêu cầu đề bài là $m=2,m=4.$}
\end{gbtt}

\begin{gbtt}
Tìm tất cả các cặp số nguyên $(x,y)$ thỏa mãn
    $$x^2+2y^2-2xy-2x-4y+6=0.$$
\nguon{Chuyên Toán Thanh Hóa 2021}
\loigiai{Phương trình đã cho tương đương 
    $$2y^2-(2x+4)y+\left(x^2-2x+6\right)=0.$$
    Coi đây là một phương trình bậc hai theo ẩn $y.$ Ta tính được
    $${\Delta}^{'}_y=(x+2)^2-2\left(x^2-2x+6\right)=-x^2+8x-8.$$
    Phương trình có nghiệm chỉ khi $\Delta'_y$ là số chính phương, và khi ấy $$x^2-8x+8\le 0.$$ Giải bất phương trình nghiệm nguyên này, ta được $2\le x\le 6.$  \\
    Thử với từng trường hợp, ta kết luận phương trình đã cho có $4$ nghiệm nguyên phân biệt là 
    $$(2,1),(2,3),(6,3),(6,5).$$}
\end{gbtt}

\begin{gbtt}
Giải phương trình nghiệm nguyên dương
\[x^2-y^2=xy+8.\]
\nguon{Chuyên Toán Bình Dương 2018}
\loigiai{
Phương trình đã cho tương đương với
$$x^2-xy-\tron{y^2+8}=0.$$
Coi đây là phương trình bậc hai theo ẩn $x.$ Ta tính được
$$\Delta_x=y^2+4\tron{y^2+8}=5y^2+32.$$
Phương trình có nghiệm nguyên chỉ khi $\Delta$ là số chính phương. Tuy nhiên, điều này không thể xảy ra do
$$5y^2+32\equiv 2\pmod{5}.$$
Như vậy, phương trình đã cho không có nghiệm nguyên dương.}
\end{gbtt}


\begin{gbtt}
Cho $p$ là số nguyên tố sao cho tồn tại các số nguyên dương $x,y$ thỏa mãn
    $$x^3+y^3-p=6xy-8.$$
Tìm giá trị lớn nhất của $p.$
\nguon{Chuyên Toán Lào Cai 2021}
\loigiai{
Với các số $x,y,p$ thỏa mãn giả thiết, ta có
    $$x^3+y^3+2^3-3x\cdot y\cdot 2=p\Leftrightarrow (x+y+2)\left(x^2+y^2+4-xy-2x-2y\right)=p.$$
    Do $x,y$ nguyên dương nên ta được $x+y+2\ge 2$ từ lập luận trên, và như vậy
    \begin{align}
        x+y+2&=p, \label{laokai1}\tag{1}\\
        x^2+y^2+4-xy-2x-2y&=1. \label{laokai2}\tag{2}
    \end{align}
Từ (\ref{laokai1}), ta có $y=p-x-2.$ Thế vào (\ref{laokai2}) rồi biến đổi tương đương, ta được
\begin{align*}
3x^2+(6-3p)x+\left(p^2-6p+11\right)=0.   
\end{align*}
Coi phương trình trên là một phương trình bậc hai ẩn $x.$ Ta cần có $\Delta_x$ là số chính phương. Ta tính được
$$\Delta_x=(6-3p)^2-4.3.\left(p^2-6p+11\right)=-3p^2+36p-96.$$
Ta dễ thu được $4\le p\le 8$ từ $\Delta\ge 0.$ Với yêu cầu chọn $p$ là số nguyên tố lớn nhất, ta chọn $p=7.$ \\
Thử với $p=7,$ ta tìm được $(x,y)=(2,3)$ và $(x,y)=(3,2).$ \\
Kết quả, $p=7$ là số nguyên tố thỏa mãn đề bài.}
\end{gbtt}

\begin{gbtt}
Giải phương trình nghiệm nguyên
$$(mn+8)^{3}+(m+n+5)^{3}=(m-1)^{2}(n-1)^{2}.$$
\nguon{Titu Andreescu}
\loigiai{Trước hết $mn+8=x$ và $-(m+n+5)=y$ thì phương trình đã cho trở thành
$$x^{3}-y^{3}=(x+y-2)^{2}.$$
Từ phép đặt thì ta có $x\geq y$, nếu $x=y$ thì $x+y=2$, dẫn đến nghiệm của phương trình là $(x,y)=(1,1)$. Đối với trong trường hợp $x-y=d>0$, ta thu được
$$d\left[(y+d)^{2}+(y+d) y+y^{2}\right]=[2 y+(d-2)]^{2}.$$
Phương trình kể trên tương đương
\[(3 d-4) y^{2}+\left(3 d^{2}-4 d+8\right) y+\left(d^{3}-d^{2}+4 d-4\right)=0.\tag{*}\label{copy.delta}\]
Do $3d-4\ne 0$ nên ta có thể coi (\ref{copy.delta}) là một phương trình bậc hai theo ẩn $y.$ Ta sẽ có 
$$\Delta_y=\left(3 d^{2}-4 d+8\right)^{2}-4(3 d-4)\left(d^{3}-d^{2}+4 d-4\right)=-3 d^{4}+4 d^{3}+48 d.$$
Với việc $\Delta_y$ phải là số chính phương, ta sẽ có 
$$\Delta_y \geqslant 0\Rightarrow \left ( 3d-4 \right )d^2\leqslant 48\Rightarrow d<4\Rightarrow d\in\left \{ 1;2;3 \right \}.$$
Tới đây, ta xét các trường hợp sau.
\begin{enumerate}
    \item Nếu $d=1$ thì $\Delta_y=49,$ và phương trình (\ref{copy.delta})  trở thành $-y^{2}+7 y=0.$ Trường hợp này cho ta 
    $$(x,y)=(1,0),\quad (x,y)=(8,7).$$
    \item Nếu $d=2$ thì $\Delta_y=80$ không là số chính phương nên trường hợp này loại.
    \item Nếu $d=3$ thì $\Delta_y=9,$ và phương trình (\ref{copy.delta})  trở thành $5 y^{2}+23 y+26=0.$ Trường hợp này cho ta 
    $$(x, y)=(1,-2).$$
\end{enumerate}
Thế trở lại các kết quả về $(x,y)$ vào phép đặt, ta kết luận phương trình đã cho có $4$ nghiệm nguyên là $$(1,-7),\ (-7,1),\ (0,-12),\ (-12,0).$$}
\end{gbtt}

\begin{gbtt}
Tìm tất cả các cặp số nguyên dương $(m, n)$ và số nguyên tố $p$ thỏa mãn đồng thời
$$m+n=2019,\quad \dfrac{4}{m+3}+\dfrac{4}{n+3}=\dfrac{1}{p}.$$
\nguon{Titu Andreescu}
\loigiai{
Trước tiên ta đặt $u=m+3, v=n+3$ khi đó thì $u+v=2025$ và tồn tại số nguyên tố $p$ thỏa mãn $$uv=4p(u+v)=4\cdot 2025p.$$ Khi đó, $u,v$ là nghiệm của phương trình bậc hai ẩn $z$
    $$z^{2}-2025 z+4\cdot 2025 p=0.$$
Với việc đây là phương trình bậc hai ẩn $z,$ ta tính được
$$\Delta_z=2025^2-4\cdot4\cdot 2025 p=45^2\tron{2025-16p}.$$
Theo bổ đề đã học, ta có $2025-16p$ là số chính phương. Đặt $2025-16p=w^2,$ ta thu được
$$\tron{45-w}\tron{45+w}=16p.$$
Ta có các nhận xét sau đây.
\begin{enumerate}
    \item[i,] Hai số $45-w$ và $45+w$ không đồng thời chia hết cho $4,$ vì đây là hai số chẵn có tổng (bằng $90$) chia cho $4$ dư $2.$
    \item[ii,] Do $w>0$ nên $45+w>16.$
\end{enumerate}
Từ các nhận xét trên, ta có $45+w=2p$ hoặc $45+w=8p,$ tương ứng với đó là $w=45-8=37$ và $w=45-2=43.$ Kiểm tra từng trường hợp rồi thế trở lại, ta tìm được các bộ $(m,n,p)$ thỏa yêu cầu là
$$\tron{1977,42,11},\quad \tron{42,1977,11},\quad \tron{1842,177,41},\quad \tron{177,1842,11}.$$}
\end{gbtt}

\begin{gbtt}
Giải phương trình nghiệm nguyên dương
\[2x(xy-2y-3)=(x+y)(3x+y).\]
\nguon{Đề nghị Olympic 30/4 năm 2018}
\loigiai{Phương trình đã cho tương đương với
\[(3-2y)x^2+2(4 y+3)x+y^2=0.\]
Do $3-2y\ne 0$ với mọi $y$ nguyên nên ta có thể coi đây là phương trình bậc hai theo ẩn $x.$ Ta tính được
\[\Delta'_y=(2 y+1)(y+3)^{2}.\]
Phương trình có nghiệm nguyên chỉ khi $\Delta'_y$ là số chính phương. Do $(y+3)^2$ là số chính phương dương nên $2y+1$ là số chính phương. Đặt $2y+1=(2k+1)^2,$ ta có $y=2k^2+2k.$ Áp dụng công thức nghiệm của phương trình bậc hai, ta thấy
$$x=\dfrac{4y+3\pm (2k+1)(y+3)}{2y-3}=\dfrac{8k^2+8k+3\pm \tron{2k+1}\tron{2k^2+2k+3}}{4k^2+4k-3}.$$
Tới đây, ta xét các trường hợp sau.
\begin{enumerate}
    \item Nếu $x=\dfrac{8k^2+8k+3- \tron{2k+1}\tron{2k^2+2k+3}}{4k^2+4k-3}$ thì $8k^2+8k+3>\tron{2k+1}\tron{2k^2+2k+3},$ hay là
    $$2k^2(2k-1)<0.$$
    Đây là điều không thể xảy ra do $k$ nguyên dương.
    \item Nếu $x=\dfrac{8k^2+8k+3+ \tron{2k+1}\tron{2k^2+2k+3}}{4k^2+4k-3},$ ta tiếp tục biến đổi để được
    $$x=\dfrac{4k^3+14k^2+16k+6}{4k^2+4k-3}=\dfrac{2k+5}{2}+\dfrac{9}{2k-1}.$$
    Do $x\in \mathbb{N}^*$ nên $9$ chia hết cho $2k-1.$ Ta tìm ra $k=1,\ k=2,\ k=5$ từ đây. Kiểm tra trực tiếp, ta có
    \[(x , y) \in\{(8 , 4);(6 , 12);(8 , 60)\}.\]
\end{enumerate}
Kết luận, phương trình đã cho có $3$ nghiệm nguyên dương là $(6,12),\ (8,4)$ và $(8,60).$}
\end{gbtt}

\begin{gbtt}
Tìm các số tự nhiên $x,y$ thỏa mãn phương trình
\[y^3-2y^2x+x^2+y^2+4x+3y+3=0.\]
\loigiai{
Phương trình đã cho tương đương với
\[x^2-2(y^2-2)x+y^3+y^2+3y+3=0.\]
Coi đây là phương trình bậc hai ẩn $x.$ Ta tính được
$$\Delta'_x=\tron{y^2-2}^2-\tron{y^3+y^2+3y+3}=y^4-y^3-5y^2-3y+1.$$
Phương trình có nghiệm nguyên chỉ khi $\Delta'_x$ là số chính phương. Với $y\ge 10,$ ta có
\begin{align*}
    4\tron{y^4-y^3-5y^2-3y+1}-(2y^2-y-6)^2&=3y^2-24y-32\\&\ge 30y-24y-32\\&\ge 6y-32\\&\ge 60-32\\&>0,\\
    (2y^2-y-5)^2-4\tron{y^4-y^3-5y^2-3y+1}&=-y^2-22y-21<0.
\end{align*}
Như vậy, với mọi $y\ge 10$ ta có
$$(2y^2-y-6)^2<4\tron{y^4-y^3-5y^2-3y+1}<(2y^2-y-5)^2.$$
Khi ấy, theo kiến thức đã học, $y^4-y^3-5y^2-3y+1$ không thể là số chính phương. Ngược lại, với $y=1,2,\ldots,9,$ bằng cách thử trực tiếp rồi đối chiếu, ta kết luận có 3 bộ số tự nhiên thỏa mãn yêu cầu đề bài là $$(x,y)=(4,5),\quad (x,y)=(6,3),\quad (x,y)=(8,3).$$}
\end{gbtt}

\begin{gbtt}
Giải phương trình nghiệm nguyên dương \[\tron{x^2-y}\tron{x+y^2}=(x+y)^3.\]
\loigiai{
Do điều kiện $x,y>0,$ phương trình đã cho tương đương với
\begin{align*}
x^2y^2+x^3-y^3-xy=x^3+3x^2y+3xy^2+y^3
&\Leftrightarrow x^2y^2-3x^2y-3xy^2-2y^3=0
\\&\Leftrightarrow -y\left[2y^2+x(3-x)y+x(3x+1)\right]=0
\\&\Leftrightarrow 2y^2+x(3-y)y+x(3x+1)=0.
\end{align*}
Coi đây là một phương trình bậc hai với ẩn  $y.$ Ta tính được
$$\Delta_y=x^2(3-x)^2-8x(3x+1)=x\left(x^3-6x^2-15x-8\right)=x(x+1)^2(x-8).$$
Do $(x+1)^2\neq0$ nên $\Delta_y$ là số chính phương chỉ khi $x(x-8)$ là số chính phương. \\
Đặt $x(x-8)=a^2,$ với $a$ là số tự nhiên. Phép đặt này cho ta
$$x^2-8x=a^2\Leftrightarrow(x-4)^2=a^2+16 \Leftrightarrow (x-4+a)(x-4-a)=16.$$
Ta thấy $x-4+a$ và $x-4-a$ cùng chẵn và $x-4+a \geq x-4-a$, thế nên ta lập được bảng giá trị sau
\begin{center}
\begin{tabular}{l|c|c|c|c}
$x-4+a$ & $-2$ & $-4$ & $4$ & $8$\\
\hline
$x-4-a$ & $-8$ & $-4$ & $4$ & $2$\\
\hline
$x-4$ & $-5$ & $-4$ & 4 & $5$\\
\hline
$x$ & $-1$ & $0$ & $8$ & $9$\\
\hline
$y$ &  &  & $10$ & $6$ và $21$
\end{tabular}
\end{center}
Đối chiếu điều kiện $x,y$ nguyên dương, phương trình đã cho có ba nghiệm là $(8,10),(9,6)$ và $(9,21).$}
\end{gbtt}

\begin{gbtt}
Tìm tất cả các cặp số nguyên $\left ( a,b \right )$ thỏa mãn
\[\left [ b^2+11\left ( a-b \right ) \right ]^2=a^3b.\]
\nguon{Hong Kong Mathematical Olympiads 2014}
\loigiai{
Bằng khai triển trực tiếp, phương trình đã cho tương đương với
\[\tron{a-b}\tron{ba^2+\left ( b^2-121 \right )a+\left ( b^3-22b^2+121b \right )}=0.\]
Nếu $a=b,$ ta thấy thỏa. Nếu $a\ne b,$ phương trình kể trên tương đương
\[ba^2+\left ( b^2-121 \right )a+\left ( b^3-22b^2+121b \right )=0.\]
Coi đây là phương trình bậc hai theo ẩn $a.$ Ta tính được
\begin{align*}
    \Delta_a&=\left ( b^2-121 \right )^2-4b\left ( b^3-22b^2+121b \right )=-\left ( b-11 \right )^3\left ( 3b+11 \right ).
\end{align*}
Phương trình có nghiệm nguyên chỉ khi $\Delta_a\ge 0,$ và thế thì
\begin{align*}
    \Delta_a\geqslant 0\Rightarrow-\left ( b-11 \right )^3\left ( 3b+11 \right )\geqslant 0\Rightarrow -3\leqslant b\leqslant 11.
\end{align*}
Bằng việc thử trực tiếp các giá trị của \(b\), ta tìm ra \(\left ( a,b \right )=\left ( 0,11 \right )\). \\
Tất cả các cặp $(a,b)$ thỏa yêu cầu là $\left ( a,b \right )=\left ( 0,11 \right )$ và $(a,b)=\left ( k,k \right ),$ với $k$ là một số nguyên tùy ý.}
\end{gbtt}

\begin{gbtt}
Giải hệ phương trình nghiệm nguyên
\[\heva{
\left(x^{2}+1\right)\left(y^{2}+1\right)+\dfrac{z^{2}}{10}&=2010 \\
(x+y)(x y-1)+14 z&=1985.}\]
\nguon{Titu Andreescu}
\loigiai{
Trước tiên, ta lưu ý rằng, tồn tại số $k\in\mathbb{Z}$ sao cho $z=10k$, bởi vì $\dfrac{z^{2}}{10}=2010-\left(x^{2}+1\right)\left(y^{2}+1\right)\in\mathbb{Z}$. Tiếp theo, ta đặt $p=x+y$ và $q=x y-1,$ và khi đó hệ phương trình trở thành
$$\heva{
 p ^ { 2 } + q ^ { 2 } + 1 0 k ^ { 2 } &= 2 0 1 0  \\
 p q + 1 4 0 k &= 1 9 8 5 }
\Leftrightarrow \heva{
p^{2}+q^{2}&=2010-10 k^{2} \\
p q&=1985-140 k.}$$
Ta có nhận xét sau đây
$$0\ge (p-q)^{2}=2010-10 k^{2}-2(1985-140 k)=-10(k-14)^{2}.$$ 
Bắt buộc, ta phải có $k=14.$ Thế trở lại hệ, ta được
$$\heva{p^{2}+q^{2}=50 \\ p q=25} \Leftrightarrow p=q=5.$$
từ đó thì ta thu được giá trị của $x$ và $y$ như sau
    $$\heva{x+y=5 \\ x y=6} \Leftrightarrow (x,y)=(3,2) \text{ hoặc } (x,y)=(2,3).$$
Kết luận, phương trình đã cho có hai nghiệm nguyên là $(2,3,140)$ và $(3,2,140).$
}
\end{gbtt}

\section{Phương trình với nghiệm nguyên tố}

Trong cuốn sách này, các phương trình nghiệm nguyên tố đã được chúng ta nghiên cứu ở \chu{chương II}. Vì thế, mục tương tự ở trong \chu{chương V} sẽ ôn tập lại cho các bạn các kiến thức xung quanh dạng phương trình ấy.

\subsection*{Bài tập tự luyện}

\begin{btt}
Cho dãy số tự nhiên $2,6,30,210,\ldots$ được xác định như sau: 
\begin{it}
Số hạng thứ $k$ bằng tích của $k$ số nguyên tố đầu tiên. 
\end{it}
Biết rằng có hai số hạng của dãy số đó có hiệu bằng $30000$. Tìm hai số hạng đó.
\nguon{Chuyên Toán Thanh Hóa 2016}
\end{btt}

\begin{btt}
Tìm tất cả các cặp số nguyên tố $(p, q)$ thỏa mãn \[7pq^2+p=q^3+43p^3+1.\]
\nguon{Dutch Mathematical Olympiad 2015}
\end{btt}

\begin{btt}
Tìm tất cả các số nguyên tố $p,q,r$ sao cho $pqr=p+q+r+200.$
\nguon{Tạp chí Toán học và Tuổi trẻ}
\end{btt}

\begin{btt}
Tìm tất cả số nguyên tố $p,q,r$ thỏa mãn 
\[(p+1)(q+2)(r+3)=4pqr.\]
\end{btt} 

\begin{btt}
Tìm tất cả bộ ba số nguyên tố $(p,q,r)$ thỏa mãn \[\dfrac{1}{p-1}+\dfrac{1}{q}+\dfrac{1}{r+1}=\dfrac{1}{2}.\]
\nguon{Titu Andreescu}
\end{btt}

\begin{btt}
Tìm các số nguyên tố $ p,q,r$ thỏa mãn đồng thời các điều kiện
\[r>q>p\ge 5,\quad 2p^2-r^2\ge 49,\quad 2q^2-r^2\le 193.\]
\end{btt}

\begin{btt}
Tìm tất cả các bộ ba số nguyên tố $a, b, c$ đôi một phân biệt thỏa mãn điều kiện
\[20abc<30(ab+bc+ca)<21abc.\]
\end{btt} 

\begin{btt}
Tìm số nguyên tố $p$ sao cho tồn tại các số nguyên dương $x,y$ thỏa mãn
\[x\left(y^2-p\right)+y\left(x^2-p\right)=5p.\]
\end{btt}

\begin{btt}
Cho $p$ là số nguyên tố lẻ. Tìm tất cả các số nguyên dương $n$ để $\sqrt{n^{2}-np}$ là số nguyên dương.
\nguon{Spanish Mathematical Olympiad 1997}
\end{btt}

\begin{btt}
Tìm tất cả các số nguyên tố $p$ và số nguyên dương $m$ thỏa mãn \[p^{3}+m(p+2)=m^{2}+p+1.\]
\nguon{Dutch Mathematical Olympiad 2012}
\end{btt}

\begin{btt}
Tìm tất cả các số nguyên tố $x,y,z$ thỏa mãn  \[x^y+1=z.\]
\end{btt}

\begin{btt}
Tìm tất cả các bộ ba số nguyên tố $\left ( p,q,r \right )$ thỏa mãn
    $$p^{2}+2 q^{2}+r^{2}=3pqr.$$
\end{btt}

\begin{btt}
Tìm tất cả các số nguyên $x, y$ và số nguyên tố $p$ thỏa mãn
\[x^2-3xy+p^2y^2=12p.\]
\nguon{France Junior Balkan Mathematical Olympiad Team Selection Test 2017}
\end{btt}

\begin{btt}
Tìm các số nguyên tố $a,b,c,d,e$ sao cho \[a^4+b^4+c^4+d^4+e^4=abcde.\]
\end{btt}

\begin{btt}
Tìm tất cả các số nguyên dương \(a,b,c\) và số nguyên tố \(p\) thỏa mãn phương trình
\[73 p^{2}+6=9 a^{2}+17 b^{2}+17 c^{2}.\]
\nguon{Junior Balkan Mathematical Olympiad Shortlist 2020}
\end{btt}

\begin{btt}
Tìm tất cả bộ ba các số nguyên tố $(p,q,r)$ thỏa mãn \[3p^4-5q^4-4r^2=26.\]
\nguon{Junior Balkan Mathematical Olympiad 2014}
\end{btt}

\begin{btt}
Tìm tất cả các số nguyên tố $p$ và $q$ thỏa mãn 
$$\dfrac{p^{3}-2017}{q^{3}-345}=q^{3}.$$
\nguon{Titu Andreescu}
\end{btt}

\begin{btt}
Tìm các nghiệm nguyên dương của phương trình 
$$x(x+3)+y(y+3)=z(z+3).$$ trong đó $x$ và $y$ là nghiệm nguyên tố.
\end{btt}

\begin{btt}
Tìm tất cả các số nguyên tố $p,q$ sao cho $p^2+q^3$ và $q^2+p^3$ đều là số chính phương.
\nguon{Baltic Way 2011}
\end{btt}

\begin{btt}
Tìm tất cả các số nguyên tố $p,q$ sao cho $p+q$ và $p+4q$ đều là số chính phương.
\nguon{Chuyên Toán Quảng Nam 2019}
\end{btt}


\begin{btt}
Tìm tất cả các số nguyên tố $p$ thỏa mãn $9p+1$ là số chính phương.
\end{btt}

\begin{btt}
Tìm tất cả các số nguyên tố $p$ sao cho $2p^2+27$ là số lập phương.
\end{btt}

\begin{btt}
Tìm tất cả số nguyên tố $p$ và số tự nhiên $n$ thỏa mãn \[n^3=(p+1)^2.\]
\end{btt}

\begin{btt}
Cho các số nguyên tố $p,q$ thỏa mãn $p+q^2$ là số chính phương. Chứng minh rằng
\begin{enumerate}[a,]
    \item $p=2q+1.$
    \item $p^2+q^{2021}$ không phải là số chính phương.
\end{enumerate}
\nguon{Chuyên Toán Quảng Ngãi 2021}    
\end{btt}

\begin{btt}
Tìm tất cả các số nguyên tố $p,q$ thỏa mãn
\[(p-2)\tron{p^2+p+2}=(q-3)(q+2).\]
\end{btt}

\begin{btt}
Tìm tất cả các số nguyên tố $p,q$ thỏa mãn \[p^5+p^3+2=q^2-q.\]
\nguon{Argentina Cono Sur Team Selection Test 2014}
\end{btt}

\begin{btt}
Tìm tất cả các số nguyên tố $p,q$ thỏa mãn
\[q^3+2q^2=6p^4+17p^3+60p^2+8q.\]
\end{btt}

\begin{btt}
Tìm tất cả các số nguyên tố $p,q$ thỏa mãn
\[p^8+7p^6=3q^2+11q.\]
\end{btt}

\begin{btt}
Tìm tất cả các số nguyên dương $n$ và số nguyên tố $p$ thỏa mãn
\[3p^2\tron{p+11}=n^3+n^2-2n.\]
\end{btt}

\begin{btt}
Tìm tất cả các số nguyên dương $n$ và số nguyên tố $p$ thỏa mãn
\[n^5+p^4=p^8+n.\]
\end{btt}

\begin{btt}
Tìm các số nguyên dương $x, y, z$ sao cho $x^{2}+1, y^{2}+1$ đều là các số nguyên tố và
$$\left(x^{2}+1\right)\left(y^{2}+1\right)=z^{2}+1.$$
\nguon{Tạp chí Toán Tuổi thơ, ngày 20 tháng 5 năm 2020}
\end{btt}

\begin{btt}
Tìm tất cả các cặp số nguyên tố $(p,q)$ thỏa mãn 
\[p+q=2(p-q)^2.\]
\nguon{Chuyên Đại học Vinh 2016}
\end{btt}

\begin{btt}
Tìm tất cả các số nguyên tố $p,q,r$ và số tự nhiên $n$ thỏa mãn
\[p^3=q^3+9r^n.\]

\end{btt}

\subsection*{Hướng dẫn bài tập tự luyện}

\begin{gbtt}
Cho dãy số tự nhiên $2,6,30,210,\ldots$ được xác định như sau: 
\begin{it}
Số hạng thứ $k$ bằng tích của $k$ số nguyên tố đầu tiên. 
\end{it}
Biết rằng có hai số hạng của dãy số đó có hiệu bằng $30000$. Tìm hai số hạng đó.
\nguon{Chuyên Toán Thanh Hóa 2016}
\loigiai{
Xét dãy số có dạng $2,2\cdot3,2\cdot3\cdot5,\ldots.$ Giả sử hai số có hiệu bằng $30000$ là
\[a=2\cdot3\cdot5\cdots{{p}_{n}}, \qquad b=2\cdot3\cdot5\cdots{{p}_{m}}\] 
với $p_n$ và $p_m$ là các số nguyên tố, ở đây $n<m.$ Lấy hiệu ta có
\[2\cdot3\cdot5\cdots{{p}_{m}}-2\cdot3\cdot5\cdots{{p}_{n}}=30000. \]
Đẳng thức trên tương đương với
$$2\cdot3\cdot5\cdot{{p}_{n}}\left( {{p}_{n+1}}\cdot{{p}_{n+2}}\cdots{{p}_{m}}-1 \right)=30000.$$
Ta nhận thấy $30000$ chia hết cho $p_n.$ Do
$30000=2^4\cdot 3\cdot 5^4$
và $p_n$ lẻ nên $p_n=3$ hoặc $p_n=5.$
\begin{enumerate}
    \item  Nếu $p_n=3,$ ta có $a=6$ còn $b=30006,$ và $b$ không là tích các số nguyên tố đầu tiên, mâu thuẫn.
    \item Nếu $p_n=5,$ ta có $a=30$ còn $b=30030.$ Thử lại, ta thấy
\[a=2\cdot3\cdot5,\qquad b=2\cdot3\cdot5\cdot7\cdot11\cdot13.\]
\end{enumerate}
Như vậy, hai số hạng thỏa mãn yêu cầu là $30$ và $30030.$}
\end{gbtt}

\begin{gbtt}
Tìm tất cả các cặp số nguyên tố $(p, q)$ thỏa mãn \[7pq^2+p=q^3+43p^3+1.\]
\nguon{Dutch Mathematical Olympiad 2015}
\loigiai{
Giả sử tồn tại các số nguyên tố $(p,q)$ thỏa mãn yêu cầu. Dễ thấy nếu $p,q$ là số lẻ, ta suy ra $7 p q^{2}+p$ là số chẵn, trong khi $q^{3}+43 p^{3}+1$ là số lẻ. Điều này không thể xảy ra, do đó trong $2$ số $p,q$ có một số chẵn. Ta xét các trường hợp sau.
\begin{enumerate}
    \item Với $p=2$, thay vào phương trình đã cho, ta có
    $$7\cdot2q^{2}+2=q^{3}+43\cdot2^{3}+1 \Rightarrow q^3-14q^2+343=0.$$
    Giải phương trình trên, ta thu được $q=7$ là số nguyên duy nhất thỏa mãn.
    \item Với $q=2$, thay vào phương trình đã cho, ta có
    $$7p\cdot2^{2}+p=2^{3}+43p^3+1 \Rightarrow 43p^3-29p+9=0.$$
    Giải phương trình trên, ta nhận thấy không có số nguyên tố $p$ thỏa mãn.
\end{enumerate}
Như vậy, $\tron{p,q}=\tron{2,7}$ là cặp số nguyên tố duy nhất thỏa mãn yêu cầu bài toán.}
\end{gbtt}

\begin{gbtt}
Tìm tất cả các số nguyên tố $p,q,r$ sao cho $pqr=p+q+r+200.$
\nguon{Tạp chí Toán học và Tuổi trẻ}
\loigiai{
Không mất tính tổng quát, giả sử ${p} \leq {q} \leq {r}$. Phương trình đã cho được viết lại thành\
\[({rq}-1)({p}-1)+({r}-1)({q}-1)=202. \tag{*}\label{thtt200}\]
Nếu $p$ lẻ thì $q, r$ cũng lẻ, và khi đó đó $4$ là ước của $({rq}-1)({p}-1)+({r}-1)({q}-1)$, nhưng $202$ không chia hết
cho $4$, vô lí. Vậy ${p}=2,$ và phương trình $(\ref{thtt200})$ trở thành $$2 {rq}-{r}-{q}=202 \Leftrightarrow 4 {rq}-2 {r}-2 {q}+1=405 \Leftrightarrow(2 {q}-1)(2 {r}-1)=5 \cdot 3^{4}.$$
Do $3 \leq 2 {q}-1 \leq 2 {r}-1$ nên $9 \leq(2 {q}-1)^{2} \leq(2 {q}-1)(2 {r}-1)=405,$ và ta tiếp tục suy ra $3 \leq 2 {q}-1 \leq 20$.\\
Ta được $2 {q}-1 \in\{3 ; 5 ; 9 ; 15\}.$ Ta xét các trường hợp kể trên.
\begin{enumerate}
    \item Nếu $2 {q}-1=3$ thì ${r}=68$ không là số nguyên tố.
    \item Nếu $2 {q}-1=5$ thì ${q}=3$ và ${r}=41$ đều là số nguyên tố.
    \item Nếu $2 {q}-1=9$ thì ${q}=5$ và ${r}=23$ đều là số nguyên tố.
    \item Nếu $2 {q}-1=15$ thì ${q}=8$ không là số nguyên tố.
\end{enumerate}
Vậy tất cả các bộ ba số nguyên tố cần tìm là $(2,5,23)$ và $(2,3,41)$ và các hoán vị.}
\end{gbtt}

\begin{gbtt}
Tìm tất cả số nguyên tố $p,q,r$ thỏa mãn 
\[(p+1)(q+2)(r+3)=4pqr.\]
\loigiai{
Giả sử $p,q,r$ là các số nguyên tố thỏa yêu cầu. Ta xét các trường hợp sau đây
\begin{enumerate}
    \item Nếu $r=2$, ta nhận thấy $5(p+1)(q+2)=8pq.$
    Do $(5,8)=1$ và $5$ là ước nguyên tố của $pq,$ ta chỉ ra $p=5$ hoặc $q=5.$ Thử trực tiếp, ta nhận được $(p,q,r)=(7,5,2).$
    \item Nếu ${r}=3$, ta nhận thấy $$({p}+1)({q}+2)=2 {pq}\Leftrightarrow pq-2p-q-2=0\Leftrightarrow (p-1)(q-2)=4.$$
    Do $q$ là các số nguyên tố, ta có $q-2\ne 2$ và $q-2\ne 4.$ Lập luận này cho ta $$\left\{\begin{aligned}{p}-1=4 \\ {q}-2=1\end{aligned} \Rightarrow\left\{\begin{aligned}{p}&=5 \\ {q}&=3.\end{aligned}\right.\right.$$ 
    Bộ số thu được trong trường hợp này là $(p,q,r)=(5,3,3).$
    \item Nếu ${r}>3$, ta nhận thấy $$4 {pqr}=({p}+1)({q}+2)({r}+3)<2 {r}({p}+1)({p}+2).$$
    Ta suy ra
    $2 {pq}<({p}+1)({q}+2)$ từ đây, hay là $$({p}-1)({q}-2)<4.$$
    Do đó ${p}-1<4$ và  ${q}-2<4$ và ${p}$ là số nguyên tố nên ${p}=2$ hoặc ${p}=3.$
\begin{itemize}
    \item\chu{Trường hợp 1.} Với ${p}=2$, ta có $3(q+2)(r+3)=8 {qr}.$ Do $(3,8)=1$ nên $3$ phải là ước nguyên tố của $qr$, lại do $r>3$ nên ta suy ra $q=3$. Thế ngược lại, ta tìm được $r=5.$
    \item\chu{Trường hợp 2.} Với ${p}=3,$ ta có $$({q}+2)({r}+3)=3 {qr}\Leftrightarrow 2 q r-3 q-2 r=6 \Leftrightarrow(q-1)(2 r-3)=9.$$
    Do ${r}>3$ nên $2 {r}-3>3$. Ta suy ra $$\left\{\begin{aligned}2 {r}-3=9 \\ {q}-1=1\end{aligned} \Rightarrow\left\{\begin{aligned} r&=6 \\ q&=2. \end{aligned}\right.\right.$$
    Lúc này, $r=6$ không là số nguyên tố, không thỏa điều kiện bài toán.
\end{itemize}
\end{enumerate}
Kết luận, có ba bộ $(p,q,r)$ thỏa mãn, đó là $(7,5,2),(5.3,3)$ và $(2,3,5)$.}
\end{gbtt} 

\begin{gbtt}
Tìm tất cả bộ ba số nguyên tố $(p,q,r)$ thỏa mãn \[\dfrac{1}{p-1}+\dfrac{1}{q}+\dfrac{1}{r+1}=\dfrac{1}{2}.\]
\nguon{Titu Andreescu}
\loigiai{
Trong bài toán này, ta xét các trường hợp sau.
\begin{enumerate}
    \item Nếu $p\le 3,$ phương trình đã cho không có nghiệm.
    \item Nếu $p=5,$ thế vào phương trình ban đầu ta có
    $$\dfrac{1}{4}+\dfrac{1}{q}+\dfrac{1}{r+1}=\dfrac{1}{2}\Leftrightarrow (q-4)(r-3)=16.$$
    Ta tìm được $q=5$ và $r=19.$
    \item Nếu $p=7,$ thế vào phương trình ban đầu ta có
    $$\dfrac{1}{6}+\dfrac{1}{q}+\dfrac{1}{r+1}=\dfrac{1}{2}\Leftrightarrow (q-3)(r-2)=9.$$
    Ta không tìm được $q,r$ nguyên tố thỏa mãn phương trình trên.
    \item Nếu $p\ge 11,$ ta xét các trường hợp nhỏ hơn sau.
    \begin{itemize}
        \item \chu{Trường hợp 1.} Nếu $q=3,$ thế vào phương trình ban đầu ta có
        $$(p-7)(r-5)=36.$$
        Ta tìm được $p=13$ và $r=11.$
        \item \chu{Trường hợp 2.} Nếu $q=5,$ thế vào phương trình ban đầu ta có
        $$\dfrac{1}{p-1}+\dfrac{1}{5}+\dfrac{1}{r+1}=\dfrac{1}{2}\Leftrightarrow r(3p-7)=13p+3.$$
        Ta không tìm ra cặp $(p,r)$ nguyên tố nào từ đây.
        \item \chu{Trường hợp 3.} Nếu $q\ge 7,$ ta buộc phải có $r=2$ vì nếu $r\ge 3$ thì
                \[\dfrac{1}{p-1}+\dfrac{1}{q}+\dfrac{1}{r+1}\le\dfrac{1}{10}+\dfrac{1}{7}+\dfrac{1}{4}<\dfrac{1}{2},\]
            mâu thuẫn. Với $r=2,$ phương trình đã cho trở thành
            $$(p-7)(q-6)=36.$$
            Ta tìm ra $p=43$ và $r=7$ từ đây.
    \end{itemize}
\end{enumerate} 
Kết luận, tất cả bộ $(p,q,r)$ thỏa mãn là \[(5,5,19),\quad (13,3,11),\quad (43,7,2).\]}
\end{gbtt}

\begin{gbtt}
Tìm các số nguyên tố $ p,q,r$ thỏa mãn đồng thời các điều kiện
\[r>q>p\ge 5,\quad 2p^2-r^2\ge 49,\quad 2q^2-r^2\le 193.\]
\loigiai{
Hai điều kiện thứ hai và thứ ba cho ta
$$2 q^{2}-193 \leq r^{2} \leq 2 p^{2}-49.$$
Do đó $q^{2}-p^{2} \leq 72$. Mặt khác, từ điều kiện thứ nhất, ta chỉ ra ${r} \geq 11$, và vì thế
$$2 {p}^{2} \geq 49+121=170\Rightarrow {p} \geq 11.$$
Vì $(q-p)(q+p) \leq 72$ nên $q-p=2$ hoặc $q-p \geq 4$. Ta xét hai trường hợp kể trên.
\begin{enumerate}
   \item Với ${q}-{p}=2$ và ${q}+{p} \leq 36$, ta có ${p}=11,{q}=13$ hoặc ${p}=17,{q}=19$.
\begin{itemize}
    \item\chu{Trường hợp 1.} Nếu ${p}=11,{q}=13$ thì $145 \leq {r}^{2} \leq 193$, suy ra ${r}=13={q},$ mâu thuẫn.
    \item\chu{Trường hợp 2.} Nếu ${p}=17,{q}=19$ thì $529 \leq {r}^{2} \leq 529$, suy ra ${r}=23.$
\end{itemize}
\item Với $q-p \geq 4$ và $q+p \leq 18$, ta có
$11\le p\le 18,$ thế nên $p=11,14,17.$ Đối chiếu với $q+p\le 18$ rồi xét các trường hợp riêng lẻ của $q,$ ta thấy không thỏa.
\end{enumerate}
Kết luận, các số nguyên tố cần tìm là ${p}=17,{q}=19$ và ${r}=23.$}
\end{gbtt}

\begin{gbtt}
Tìm tất cả các bộ ba số nguyên tố $a, b, c$ đôi một phân biệt thỏa mãn điều kiện
\[20abc<30(ab+bc+ca)<21abc.\]
\loigiai{
Giả sử tồn tại các số nguyên tố $a,b,c$ thỏa mãn đề bài. Chia bất phương trình đã cho $30abc,$ ta được
$$\dfrac{2}{3}<\dfrac{1}{{a}}+\dfrac{1}{{b}}+\dfrac{1}{{c}}<\dfrac{7}{10} .$$
Không mất tính tổng quát, ta giả sử ${a}>{b}>{c}>1$. Theo đó
$$\dfrac{2}{3}<\dfrac{1}{{a}}+\dfrac{1}{{b}}+\dfrac{1}{{c}}<\dfrac{3}{c}.$$
Nhận xét trên cho ta $\dfrac{2}{3}<\dfrac{3}{c}$ hay $2c<9.$ Vì $c$ là số nguyên tố, $c=2$ hoặc $c=3.$
\begin{enumerate}
    \item Với ${c}=2,$ ta lần lượt suy ra $$\dfrac{2}{3}<\dfrac{1}{2}+\dfrac{1}{{a}}+\dfrac{1}{{b}}<\dfrac{7}{10} \Rightarrow \dfrac{1}{6}<\dfrac{1}{{a}}+\dfrac{1}{{b}}<\dfrac{1}{5} \Rightarrow \dfrac{1}{6}<\dfrac{2}{{b}}<\dfrac{2}{5}.$$ Nhận xét trên cho ta ${b} \in\{7 ; 11\}$.
    \begin{itemize}
    \item\chu{Trường hợp 1.} Với ${b}=7$, từ $\dfrac{1}{6}<\dfrac{1}{{a}}+\dfrac{1}{{b}}<\dfrac{1}{5}$ ta suy ra $\dfrac{1}{42}<\dfrac{1}{{a}}<\dfrac{2}{35}.$ Do $a>b$ nên
    $$a\in \{19;23;29;31;37;41\}.$$
    \item\chu{Trường hợp 2.} Với $b=11,$ từ $\dfrac{1}{6}<\dfrac{1}{a}+\dfrac{1}{b}<\dfrac{1}{5}$ ta suy ra $\dfrac{5}{66}<\dfrac{1}{a}<\dfrac{6}{55}.$ Do $a>b$ nên $a=13.$
    \end{itemize}
    \item Với $c=3,$ ta lần lượt suy ra $$\dfrac{2}{3}<\dfrac{1}{3}+\dfrac{1}{{a}}+\dfrac{1}{{b}}<\dfrac{7}{10} \Rightarrow \dfrac{1}{3}<\dfrac{1}{{a}}+\dfrac{1}{{b}}<\dfrac{11}{30} \Rightarrow \dfrac{1}{3}<\dfrac{2}{{b}}<\dfrac{11}{30}.$$
    Nhận xét trên kết hợp với việc $b>c$ cho ta ${b}=5.$ Ta tiếp tục thu được $$\dfrac{1}{3}<\dfrac{1}{{a}}+\dfrac{1}{{5}}<\dfrac{11}{30}\Rightarrow 6<{a}<\dfrac{15}{2} \Rightarrow {a}=7.$$
\end{enumerate}
Vậy có các bộ ba số nguyên tố khác nhau $(a,b,c)$ thoả mãn là $$(19 , 7 , 2),(23 , 7 , 2),(29 , 7 , 2),(31 , 7 , 2),(37 , 7 , 2),(41 , 7 , 2),(13 , 11 , 2),(7 , 5 , 3)$$ và các hoán vị của
nó.}
\end{gbtt} 

\begin{gbtt}
Tìm số nguyên tố $p$ sao cho tồn tại các số nguyên dương $x,y$ thỏa mãn
\[x\left(y^2-p\right)+y\left(x^2-p\right)=5p.\]
\loigiai{
Với các số $x,y,p$ thỏa mãn đề bài, ta có
\begin{align*}
   xy^2+x^2y-px-py=5p\Rightarrow (xy-p)(x+y)=5p.
\end{align*}
Không mất tổng quát, ta giả sử $x\ge y.$ Rõ ràng, $xy-p$ và $x+y$ là các ước số dương của $5p.$ \\
Dựa vào lập luận này, ta xét các trường hợp sau đây.
\begin{enumerate}
    \item Nếu $x+y=1,$ ta không tìm được $x,y$ nguyên dương.
    \item Nếu $x+y=5,$ ta thử trực tiếp $(x,y)=(3,2),(4,1)$ để chỉ ra $p=3$ và $p=2.$
    \item Nếu $x+y=p$ và $xy-p=5,$ ta có
    $$xy=x+y+5\Rightarrow xy-x-y-5=0\Rightarrow (x-1)(y-1)=6.$$
    Do $x-1\ge y-1\ge 1,$ phương trình ước số trên cho ta $x=4,y=3$ hoặc $x=7,y=2.$\\
    Theo đó, ta chỉ tìm ra $p=7$ khi mà $x=4,y=3.$
    \item Nếu $x+y=5p$ và $xy-p=1,$ ta có
    $$5xy=x+y+5\Rightarrow (5x-1)(5y-1)=26.$$
    Trong trường hợp này, ta không chỉ ra được sự tồn tại của $x,y$ nguyên dương.
\end{enumerate}
Như vậy, có tất cả ba số nguyên tố thỏa yêu cầu, đó là $p=2,p=3$ và $p=7.$
}
\end{gbtt}

\begin{gbtt}
Cho $p$ là số nguyên tố lẻ. Tìm tất cả các số nguyên dương $n$ để $\sqrt{n^{2}-np}$ là số nguyên dương.
\nguon{Spanish Mathematical Olympiad 1997}
\loigiai{
Từ giả thiết ta thu được $n^2-np$ là bình phương một số tự nhiên, ta có thể đặt $n^2-np=m^2,$ với $m$ nguyên dương. Phép đặt này cho ta 
\begin{align*}
n^{2}-n p=m^{2} 
&\Rightarrow 4n^2-4np=4m^2
\\&\Rightarrow(2 n-p)^{2}-4 m^{2}=p^{2} \\&\Rightarrow(2 n-p-2 m)(2 n-p+2 m)=p^2.
\end{align*}
Do $0<2 n-p-2 m<2 n-p+2 m$ và $p$ là số nguyên tố nên ta suy ra 
$$\heva{&2n-p-2m=1 \\ &2n-p+2m=p^2}\Rightarrow 4n-2p=p^2+1\Rightarrow n=\dfrac{(p+1)^2}{4}.$$
Thử trực tiếp với số $n=\dfrac{(p+1)^2}{4}$, ta có
$$n^2-np=\left[\dfrac{(p+1)^{2}}{4}\right]^2-\dfrac{p(p+1)^{2}}{4}=\left(\dfrac{p^2-1}{4}\right)^2.$$
Do $p$ lẻ nên $\dfrac{p^2-1}{4}=\dfrac{(p-1)(p+1)}{4}=\dfrac{p-1}{2}\cdot\dfrac{p+1}{2}$ nguyên dương. \\Như vậy, số $n=\dfrac{(p+1)^2}{4}$ là số cần tìm.}
\end{gbtt}

\begin{gbtt}
Tìm tất cả các số nguyên tố $p$ và số nguyên dương $m$ thỏa mãn \[p^{3}+m(p+2)=m^{2}+p+1.\]
\nguon{Dutch Mathematical Olympiad 2012}
\loigiai{
Giả sử tồn tại các số nguyên tố $p$ và số nguyên dương $m$ thỏa mãn yêu cầu bài toán. Ta có
$$p^3=m^2-m(p+2)+p+1\Rightarrow p^3=(m-1)(m-p-1).$$
Do $p$ là số nguyên tố và $m-1>m-p-1,$ chỉ có hai khả năng sau đây xảy ra.
\begin{enumerate}
    \item Với $m-p-1=1$ và $m-1=p^3,$ ta có hệ
    $$
    \heva{&m=p+2 \\ &m=p^3+1}
    \Rightarrow
    p+2=p^3+1 
    \Rightarrow
    p^2(p-1)=1.
    $$
    Ta không tìm được $p$ nguyên dương từ đây.
    \item Với $m-p-1=p$ và $m-1=p^2,$ ta có hệ
    $$
    \heva{m&=2p+1 \\ m&=p^2+1}
    \Rightarrow
    \heva{m&=2p+1 \\ 2p+1&=p^2+1}
    \Rightarrow
    \heva{p&=2 \\ m&=5.}
    $$
\end{enumerate}
Như vậy, cặp $(m,p)=(5,2)$ là cặp số duy nhất thỏa mãn đề bài.}
\end{gbtt}

\begin{gbtt}
Tìm tất cả các số nguyên tố $x,y,z$ thỏa mãn  \[x^y+1=z.\]
\loigiai{
Do $x,y$ là các số nguyên tố nên $x \geqslant 2,y \geqslant 2$. Suy ra ${x^y} + 1 \geqslant 5$ hay $z \geqslant 5$. Điều này dẫn đến $z$ là số lẻ, suy ra ${x^y}$ chẵn. Vì thế ta có $x = 2$. Đến đây, ta xét hai trường hợp
\begin{enumerate}
    \item Nếu $y$ lẻ thì 
    ${x^y} + 1 = {2^y} + {1^y} = \left( {2 + 1} \right)\left( {{2^{y - 1}} - {2^{y - 2}} + {2^{y - 3}} -  \cdots } \right).$
    Suy ra $$3\mid\tron{{x^y} + 1} = z.$$ Điều này là không thể vì $z$ là số nguyên tố lớn hơn hoặc bằng 5.
    \item Nếu $y$  chẵn thì $y=2$, suy ra $z=5$.
\end{enumerate}
Tóm lại, phương trình đã cho có nghiệm duy nhất là $\left( {x,y,z} \right) = \left( {2,2,5} \right)$.
}
\end{gbtt}

\begin{gbtt}
Tìm tất cả các bộ ba số nguyên tố $\left ( p,q,r \right )$ thỏa mãn
    $$p^{2}+2 q^{2}+r^{2}=3pqr.$$
\nguon{Adrian Andreescu}
    \loigiai{
    Nếu cả $p$ và $r$ đều không chia hết cho $3$ thì $p^{2}+2 q^{2}+r^{2} \equiv 1+2 q^{2}+1 \equiv 2,1\pmod 3$, điều này là không thể. Giả sử $r=3$ thì $p^{2}+2 q^{2}=9(pq-1)$, thì ta sẽ xem xét hai trường hợp sau.
    \begin{enumerate}
        \item Nếu $q$ là số lẻ thì $p^{2}+2 q^{2}$ và $9(p q-1)$ đối lập nhau về tính chẵn lẻ, nên vô lí.
        \item Nếu $q=2$ thì ta suy ra được $0=p^{2}-18 p+17=(p-1)(p-17),$ hay $p=17$.
    \end{enumerate}
Do tính đối xứng của $p$ và $r,$ ta kết luận các bộ $(p,q,r)$ thỏa mãn là $(p, q, r)=(17,2,3)$ và $(3,2,17)$.}
\end{gbtt}

\begin{gbtt}
Tìm tất cả các số nguyên $x, y$ và số nguyên tố $p$ thỏa mãn
\[x^2-3xy+p^2y^2=12p.\]
\nguon{France Junior Balkan Mathematical Olympiad Team Selection Test 2017}
\loigiai{
Lấy đồng dư hai vế phương trình theo modulo $3,$ ta được
$$x^2+p^2y^2\equiv 0 \pmod{3}.$$
Dựa vào tính chất đã biết, ta có
$$\heva{&3\mid x \\ &3\mid py}\Rightarrow \heva{&9 \mid x^2 \\ &9\mid 3xy \\ &9\mid p^2y^2}\Rightarrow 9\mid 12p\Rightarrow 3\mid p\Rightarrow p=3.$$
Thay $p=3$ vào phương trình ban đầu, ta được
\[x^2-3xy+9y^2=36.\tag{*}\label{movedto5.1}\]
Ta có nhận xét rằng $$36=x^2-3xy+9y^2=x^2-3xy+\dfrac{9y^2}{4}+\dfrac{27y^2}{4}=\left(x-\dfrac{3y}{2}\right)^2+\dfrac{27y^2}{4}\ge \dfrac{27y^2}{4}.$$
Nhận xét này cho ta $144\ge 27y^2,$ thế nên $|y|\le \sqrt{\dfrac{144}{27}}<3.$ Ta chỉ ra $y\in \{0;\pm 1;\pm 2\}$ từ đây.\\
Ta lập được bảng giá trị tương ứng như sau.
\begin{center}
    \begin{tabular}{c|c|c}
        $y$ &  (\ref{movedto5.1}) sau khi thế & $x$\\
        \hline
        $-2$ & $\quad x^2+6x+36=36\quad$ & $0$ và $6$\\
        $-1$ & $\quad x^2+3x+9=36\quad $ & $\notin \mathbb{Z}$\\
        $0$ & $\quad x^2=36\quad $ & $-6$ và $6$\\
        $1$ & $\quad x^2-3x+9=36\quad $ & $\notin \mathbb{Z}$\\
        $2$ & $\quad x^2-6x+36=36\quad $ & $0$ và $6$
    \end{tabular}
\end{center}
Như vậy, có tổng cộng $6$ bộ $(x,y,p)$ thỏa mãn đề bài, đó là
$$(-6,-2,3),\, (0,2,3),\, (-6,0,3),\,(6,0,3),\,(0,2,3),\,(6,2,3).$$}
\end{gbtt}

\begin{gbtt}
Tìm các số nguyên tố $a,b,c,d,e$ sao cho \[a^4+b^4+c^4+d^4+e^4=abcde.\]
\loigiai{
Ta đã biết, với $p$ là một số nguyên tố khác $5,$ ta có $p^2\equiv -1,1\pmod{5}$, thế nên $p^4\equiv 1\pmod{5}.$\\
Gọi $a$ số các số $5$ ở vế trái. Ta xét các trường hợp sau.
\begin{enumerate}
    \item Nếu $a=0$ thì $VT\equiv 1+1+\cdots+1\equiv 5\equiv 0\pmod{5},$ còn vế phải không chia hết cho $5,$ mâu thuẫn.
    \item Nếu $a=5$ thì $a=b=c=d=e=5.$ Thử lại, ta thấy thỏa mãn.
    \item Nếu $1\le a\le 4$ thì vế phải chia hết cho $5$ và
    $$VT\equiv 5-a\pmod{5}.$$
    Bắt buộc, $5-a\equiv 0\pmod{5},$ vô lí do $1\le a\le 4.$
\end{enumerate}
Kết luận, bộ số nguyên tố duy nhất thoả mãn đề bài là $(a,b,c,d,e)=(5,5,5,5,5).$}
\end{gbtt}

\begin{gbtt}
Tìm tất cả các số nguyên dương \(a,b,c\) và số nguyên tố \(p\) thỏa mãn phương trình
\[73 p^{2}+6=9 a^{2}+17 b^{2}+17 c^{2}.\]
\nguon{Junior Balkan Mathematical Olympiad Shortlist 2020}
\loigiai{
Trong bài toán này, ta xét các trường hợp sau đây.
\begin{enumerate}
    \item Nếu $p\ge 3,$ ta có $p$ là số lẻ, vậy nên $p^2\equiv 1\pmod{8}.$ Lấy đồng dư theo modulo $8$ hai vế, ta được
    \[7\equiv a^{2}+b^{2}+c^{2}\pmod{8}.\]
    Trong đồng dư thức kể trên, các số $a,b,c$ có vai trò tương tự nhau. \\
    Lập bảng đồng dư cho $a^2,b^2,c^2$ theo modulo $8,$ ta có
    \begin{center}
        \begin{tabular}{c|c|c|c}
           $  a^2  $  & $  b^2  $ & $  c^2  $ & $  a^2+b^2+c^2  $\\
            \hline
            0 & 0 & 0 & 0\\
            0 & 0 & 1 & 1\\
            0 & 0 & 4 & 1\\          
            0 & 1 & 1 & 2\\
            0 & 1 & 4 & 5\\
            0 & 4 & 4 & 0\\
            1 & 1 & 1 & 3\\
            1 & 1 & 4 & 6\\
            1 & 4 & 4 & 1         
        \end{tabular}
    \end{center}
    Đối chiểu bảng đồng dư với đồng dư thức kể trên, ta thấy mâu thuẫn.
    \item Nếu $p=2,$ thế vào phương trình đã cho ta được
    $$9a^2+17b^2+17c^2=289.$$
    Với giả sử $b\ge c,$ điều này dẫn đến $b^{2}+c^{2} \leqslant 17,$ nhưng nó chỉ xảy ra khi 
    \[(b, c)\in\{(4,1);(3,2);(3,1);(2,2);(2,1);(1,1)\}.\]
    Kiểm tra trực tiếp, ta thấy chỉ có duy nhất \(\left ( b,c \right )=\left ( 4,1 \right )\) thỏa mãn, từ đó ta thu được \(a=1\).  
\end{enumerate}
Do tính đối xứng của \(b,c\) nên ta thu được các nghiệm của $(1,1,4,2)$ và $(1,4,1,2).$}
\end{gbtt}

\begin{gbtt}
Tìm tất cả bộ ba các số nguyên tố $(p,q,r)$ thỏa mãn $$3p^4-5q^4-4r^2=26.$$
\nguon{Junior Balkan Mathematical Olympiad 2014}
\loigiai{
Giả sử tồn tại các số nguyên tố $p,q,r$ thỏa yêu cầu. Nếu $q\ne 3$ và $r\ne 3,$ ta có
$$26=3p^4-5q^4-4r^2\equiv 0-5-4\equiv -9\equiv 0\pmod{3}.$$
Đồng dư thức trên cho ta $26$ chia hết cho $3,$ vô lí. Như vậy một trong hai số $q$ và $r$ bằng $3.$ \\
Ta xét các trường hợp sau đây.
\begin{enumerate}
    \item Nếu $q=3,$ phương trình đã cho trở thành
    $$3p^4-4r^2=431.$$
    Nếu $p=5,$ ta có $r=19.$ Còn nếu $p\ne 5,$ ta có
    $$431=3p^4-4r^2\equiv 3-4r^2\pmod{5}.$$
    Chuyển vế, ta được $r^2\equiv 3\pmod{5}.$ Đây là điều vô lí.
    \item Nếu $r=3,$ phương trình đã cho trở thành
    $$3p^4-5p^4=62.$$
    Bằng cách xét modulo $5$ hai vế tương tự trường hợp trên, ta cũng chỉ ra điều vô lí.
\end{enumerate}
Kết luận, bộ ba số nguyên tố duy nhất thỏa yêu cầu là $(p,q,r)=(5,3,19).$}
\end{gbtt}

\begin{gbtt}
Tìm tất cả các số nguyên tố $p$ và $q$ thỏa mãn 
$$\dfrac{p^{3}-2017}{q^{3}-345}=q^{3}.$$
\nguon{Titu Andreescu}
\loigiai{
Phương trình đã cho tương đương với
    $$p^{3}-2017=q^{6}-345 q^{3}.$$
Lấy đồng dư theo modulo $7$ hai vế phương trình trên, ta được
$$p^{3}-1 \equiv q^{6}-2 q^{3}\pmod{7} \Rightarrow p^{3} \equiv\left(q^{3}-1\right)^{2}\pmod{7}.$$
Do $q^{3} \equiv 0,1,6\pmod 7$ nên $\left(q^{3}-1\right)^{2} \equiv 0,1,4\pmod 7.$ Kết hợp với đồng dư thức bên trên, ta có $$\left(q^{3}-1\right)^{2} \equiv 0,1\pmod 7.$$ 
Tới đây, ta xem xét các trường hợp sau.
    \begin{enumerate}
        \item Nếu $\left(q^{3}-1\right)^{2} \equiv 0\pmod 7$ thì kéo theo $p^{3} \equiv 0\pmod 7$, nghĩa là $p=7$. Thế trở lại, ta được 
        $$q^{6}-345 q^{3}+1674=0.$$ 
        Phương trình này không có nghiệm nguyên.
        \item Nếu $\left(q^{3}-1\right)^{2} \equiv 1\pmod 7$ thì kéo theo $q^{3} \equiv 0\pmod 7$, nghĩa là $q=7$. Thế trở lại, ta được $$p^{3}-2017=-686.$$ 
        Nghiệm nguyên tố của phương trình trên là $p=11.$
    \end{enumerate}
Như vậy, cặp số nguyên tố duy nhất thỏa yêu cầu là $(p,q)=(11,7).$}
\end{gbtt}
\begin{gbtt}
Tìm các nghiệm nguyên dương của phương trình 
$$x(x+3)+y(y+3)=z(z+3).$$ trong đó $x$ và $y$ là nghiệm nguyên tố.
\loigiai{
Lấy đồng dư theo modulo $3$ hai vế, ta được
$$x^2+y^2\equiv z^2\pmod{3}.$$
Ta sẽ chứng minh rằng $x$ hoặc $y$ bằng $3.$ Thật vậy, nếu như $xy$ không chia hết cho $3,$ ta có
$$z^2\equiv x^2+y^2\equiv 1+1\equiv 2\pmod{3}.$$
Không có số chính phương nào chia $3$ dư $2.$ Giả sử sai, và thế thì $x$ hoặc $y$ bằng $3.$  Không giảm tính tổng quát có thể giả sử $x = 3$. Thay vào phương trình đã cho ta thu được
\[ 18 + {y^2} + 3y = {z^2} + 3z \Leftrightarrow \left( {z - y} \right)\left( {z + y + 3} \right) = 18.\]
Ta có các nhận xét sau đây.
\begin{enumerate}
    \item[i,] $z-y$ và $z+y+3$ khác tính chẵn lẻ do chúng có tổng lẻ.
    \item[ii,] $0<z-y<z+y+3.$
\end{enumerate}
Dựa vào hai nhận xét trên, ta lập bảng giá trị.
\begin{center}
    \begin{tabular}{c|c|c}
       $z-y$  & $1$ & $2$ \\
       \hline
        $z+y+3$ & $18$ & $9$ \\
       \hline
       $y$ & $7$ & $2$ \\
       \hline
       $z$ & $8$ & $4$
    \end{tabular}
\end{center}
Do tính đối xứng của $x$ và $y$ nên ta tìm được $4$ nghiệm của phương trình đã cho là
$$\left( {3,7,8} \right),\:\left( {7,3,8} \right),\:\left( {3,2,4} \right),\:\left( {2,3,4} \right).$$}
\end{gbtt}
\begin{gbtt}
Tìm tất cả các số nguyên tố $p,q$ sao cho $p^2+q^3$ và $q^2+p^3$ đều là số chính phương.
\nguon{Baltic Way 2011}
\loigiai{
Giả sử tồn tại số nguyên tố $p,q$ thỏa mãn. Ta đặt $p^2+q^3=a^2.$ Phép đặt này cho ta
\[q^3=\tron{a-p}\tron{a+p}.\]
Tới đây, ta chia bài toán làm các trường hợp sau.
\begin{enumerate}
    \item Với $q=2,$ ta có $\tron{a-p}\tron{a+p}=8.$ Ta lập bảng giá trị sau đây
    \begin{center}
        \begin{tabular}{c|c|c}
        $a-p$ & $1$ & $2$ \\
        \hline
        $a+p $ & $8$ & $4$ \\
        \hline
        $p$   & $\notin \mathbb{N}$ & $1$ 
        \end{tabular}
    \end{center}
    Bảng giá trị trên không cho ta $p$ nguyên tố. Trường hợp này không xảy ra.
    \item Với $q>2,$ ta xét các trường hợp sau.
    \begin{itemize}
        \item\chu{Trường hợp 1.} Với $p=q,$  ta thế vào $p^2+q^3=a^2$ và thu được
        $$q^2\tron{q+1}=a^2.$$
        Từ đây, ta suy ra $q+1$ là số chính phương. Đặt $q+1=b^2,$ biến đổi tương đương cho ta
        $$q=\tron{b-1}\tron{b+1}.$$
        Vì $q$ là số nguyên tố nên $b-1=1$ và $b+1=q.$ Điều này dẫn đến $b=2$ và kéo theo $q=3.$ Do đó $p=q=3$ là cặp số nguyên tố thỏa mãn.
        \item\chu{Trường hợp 2.} Với $p\ne q,$ ta suy ra $\tron{p,q}=1.$ Ta dễ dàng chứng minh $\tron{a+p,a-p}\mid 2p.$ Lại có, $q$ không chia hết cho $2,p$ nên $\tron{a+p,a-p}=1.$ Từ đây, ta suy ra
        $$\heva{a-p&=1\\a+p&=q^3}\Rightarrow 2p=\tron{q-1}\tron{q^2+q+1}.$$
        Do $p$ là số nguyên tố lẻ nên chỉ xảy ra trường hợp $q-1=2$ và $p=q^2+q+1.$ Điều này dẫn đến $q=3,p=13,$ nhưng khi đó $q^2+p^3=2206$ không là số chính phương, mâu thuẫn.
    \end{itemize}
\end{enumerate}
Như vậy, có duy nhất bộ số nguyên tố thỏa mãn là $(p,q)=(3,3).$}
\end{gbtt}

\begin{gbtt}
Tìm tất cả các số nguyên tố $p,q$ sao cho $p+q$ và $p+4q$ đều là số chính phương.
\nguon{Chuyên Toán Quảng Nam 2019}
\loigiai{
Giả sử tồn tại số nguyên tố $p,q$ thỏa mãn. Ta đặt $p+q=x^2$ và $p+4q=y^2.$ Lấy hiệu theo vế, ta được
$$3q=(y-x)(y+x).$$
Do các ước của $3q$ chỉ có thể là $1,3,q,3q$ nên ta xét các trường hợp sau.
\begin{enumerate}
    \item Nếu $y-x=1$ và $y+x=3q,$ ta có $y=\dfrac{3q+1}{2}.$ Lúc này
    $$p=y^2-4q=\tron{\dfrac{3q+1}{2}}^2-4q=\dfrac{9q^2-10q+1}{4}=\dfrac{(q-1)(9q-1)}{4}.$$
    Rõ ràng $p$ lẻ. Nếu như $q=3,$ ta có $p=13.$ Nếu như $q\ge 5,$ ta có số
    $$p=\tron{\dfrac{q-1}{2}}\tron{\dfrac{9q-1}{2}}$$
    là hợp số do các thừa số của nó không nhỏ hơn $2,$ mâu thuẫn.
    \item Nếu $y-x=3$ và $y+x=q,$ ta có $y=\dfrac{q+3}{2}.$ Lúc này
    $$p=y^2-4q=\tron{\dfrac{q+3}{2}}^2-4q=\dfrac{q^2-10q+9}{4}=\dfrac{(q-1)(q-9)}{4}.$$
    Rõ ràng $q$ lẻ và $p>9.$ Nếu như $q=11,$ ta có $p=5.$ Nếu như $q\ge 13,$ ta có số
    $$p=\tron{\dfrac{q-1}{2}}\tron{\dfrac{q-9}{2}}$$
    là hợp số do các thừa số của nó không nhỏ hơn $2,$ mâu thuẫn.    
    \item Nếu $y-x=q$ và $y+x=3,$ ta có $q<3,$ và do $q$ nguyên tố nên $q=2.$\\
    Lúc này $y-x$ và $y+x$ khác tính chẵn lẻ, mâu thuẫn.
    \item Nếu $y-x=3q$ và $y+x=1,$ ta có $3q<1,$ mâu thuẫn. 
\end{enumerate}
Kết luận, có đúng hai cặp $(p,q)$ thỏa yêu cầu là $(13,3)$ và $(5,11).$}
\end{gbtt}


\begin{gbtt}
Tìm tất cả các số nguyên tố $p$ thỏa mãn $9p+1$ là số chính phương.
\loigiai{
Đặt $9p+1=x^2$ với $x$ là số nguyên dương. Phép đặt này cho ta
$$9p+1=x^2\Leftrightarrow9p=\tron{x-1}\tron{x+1}.$$
Vì các ước của $9p$ chỉ có thể là $1,3,9,p,3p,9p$ và $x-1<x+1$ nên ta xét các trường hợp sau.
\begin{enumerate}
    \item Với $x-1=1$ và $x+1=9p$, ta có $x=2$ và $p=\dfrac{1}{3}$ không là số nguyên tố.
    \item  Với $x-1=3$ và $x+1=3p$, ta có $x=4$ và $p=\dfrac{5}{3}$ không là số nguyên tố.
    \item Với $x-1=9$ và $x+1=p$, ta có $x=10$ và $p=11$. 
    \item Với $x-1=p$ và $x+1=9$, ta có $x=8$ và $p=7$. 
\end{enumerate}
Như vậy, tất các số nguyên tố $p$ thỏa mãn là $7$ và $11.$}
\end{gbtt}

\begin{gbtt}
Tìm tất cả các số nguyên tố $p$ sao cho $2p^2+27$ là số lập phương.
\loigiai{
Giả sử tồn tại số nguyên tố $p$ thỏa mãn đề bài. Ta đặt $2p^2+27=x^3.$ Phép đặt này cho ta
$$2p^2=x^3-27\Rightarrow 2p^2=\tron{x-3}\tron{x^2+3x+9}.$$
Ta dễ dàng nhận thấy $\tron{x-3,x^2+3x+9}\mid 27$ nên  ta xét các trường hợp sau.
    \begin{enumerate}
    \item Nếu $\tron{x-3,x^2+3x+9}=1,$ ta lại xét tiếp các trường hợp sau.
    \begin{itemize}
        \item\chu{Trường hợp 1.} Với $x-3=1$ và $x^2+3x+9=2p^2$, ta suy ra $x=4$ và $2p^2=37,$ vô lí. 
        \item\chu{Trường hợp 2.} Với $x-3=2$ và $x^2+3x+9=p^2$, ta suy ra $x=5$ và $p=7.$
    \end{itemize}
    \item Nếu $\tron{x-3,x^2+3x+9}$ chia hết cho $3,$ ta lần lượt suy ra 
    $$3\mid \tron{x-3}\tron{x^2+3x+9}\Rightarrow 3\mid 2p^2\Rightarrow p=3.$$
    Thế trở lại ta được $2p^2+27=2\cdot3^2+27=45$ không phải là số lập phương.
\end{enumerate}
Như vậy, $p=7$ là số nguyên tố duy nhất thỏa yêu cầu.}
\end{gbtt}

\begin{gbtt}
Tìm tất cả số nguyên tố $p$ và số tự nhiên $n$ thỏa mãn \[n^3=(p+1)^2.\]
\loigiai{
Giả sử tồn tại số nguyên tố $p$ và số tự nhiên $n$ thỏa mãn. Ta sẽ chứng minh $n$ là số chính phương. Ta có
$$\tron{\dfrac{p+1}{n}}^2=n.$$
Rõ ràng $n\ne 0.$ Đặt $\dfrac{p+1}{n}=\dfrac{x}{y},$ trong đó $(x,y)=1$ và $y$ nguyên dương. Ta sẽ có
$$\dfrac{x^2}{y^2}=n\Rightarrow y^2\mid x^2\Rightarrow y\mid x,$$
nhưng do $(x,y)=1$ nên $y=1.$ Nói chung, số $n=\tron{\dfrac{p+1}{n}}^2$ là số chính phương, thế nên $p+1$ là số lập phương. Ta tiếp tục đặt $p+1=a^3.$ Ta có $$p=(a-1)(a^2+a+1).$$
Vì $p$ là một số nguyên tố và $a^2+a+1\ge 3$ nên ta phải có $a-1=1,$ tức là $a=2,$ và như thế $p=7,n=4.$ Như vậy $(n,p)=(4,7)$ là cặp số duy nhất thỏa mãn yêu cầu.}
\end{gbtt}

\begin{gbtt}
Cho các số nguyên tố $p,q$ thỏa mãn $p+q^2$ là số chính phương. Chứng minh rằng
\begin{enumerate}[a,]
    \item $p=2q+1.$
    \item $p^2+q^{2021}$ không phải là số chính phương.
\end{enumerate}
\nguon{Chuyên Toán Quảng Ngãi 2021}   
\loigiai{
 \begin{enumerate}[a,]
    \item Từ giả thiết, ta có thể đặt $p+q^2=a^2,$ với $a$ nguyên dương. Phép đặt trên cho ta
    $$p=(a-q)(a+q).$$
    Do $p$ nguyên tố và $0<a-q<a+q,$ ta suy ra $a-q=1,$ còn $a+q=p.$ Lấy hiệu theo vế, ta được \[p=2q+1.\]
    \item Giả sử $(2q+1)^2+q^{2021}$ là số chính phương. Theo đó, ta có thể đặt $(2q+1)^2+q^{2021}=b^2,$ với $b$ là số nguyên dương. Phép đặt này cho ta
        $$q^{2021}=b^2-(2q+1)^2\Rightarrow q^{2021}=(b-2q-1)(b+2q+1).$$
    Tới đây, ta xét các trường hợp sau.
        \begin{itemize}
            \item\chu{Trường hợp 1.} Nếu $b-2q-1$ và $b+2q+1$ có ước nguyên tố chung là $r,$ ta suy ra 
            $$\heva{&r\mid (b-2q-1) \\ &r\mid (b+2q+1) \\ &r\mid q^{2021}}
            \Rightarrow \heva{&r\mid (4q+2) \\ &r\mid q}\Rightarrow \heva{&r\mid 2 \\ &r\mid q}\Rightarrow q=r=2.$$ \\
            Lúc này, $(2q+1)^2+q^{2021}=5^2+2^{2021}.$ Số này chia cho $5$ dư $2,$ do vậy nó không chính phương.
            \item\chu{Trường hợp 2.} Nếu $b-2q-1$ và $b+2q+1$ nguyên tố cùng nhau, ta suy ra
            $$\heva{&b-2q-1=1 \\ &b+2q+1=q^{2021}} 
            \Rightarrow 4q+2=q^{2021}-1\Rightarrow 4q+3=q^{2021}.$$ \\
            Xét tính chia hết cho $q$ ở cả hai vế, ta được $q=3.$ Thay $q=3$ trở lại đẳng thức $4q+3=q^{2021},$ ta có $15=3^{2021},$ một điều vô lí.
        \end{itemize}
        Mâu thuẫn chỉ ra ở tất cả các trường hợp chứng tỏ giả sử phản chứng là sai. \\Bài toán được chứng minh.
    \end{enumerate}}
\end{gbtt}

\begin{gbtt}
Tìm tất cả các số nguyên tố $p,q$ thỏa mãn
\[(p-2)\tron{p^2+p+2}=(q-3)(q+2).\]
\loigiai{
Giả sử tồn tại các số nguyên tố $p,q$ thỏa mãn đề bài. Giả sử như vậy cho ta
\[p^2\tron{p-1}=\tron{q-2}\tron{q+1}.\tag{*}\label{snt5}\]
Ta nhận thấy rằng một trong các số $q-2,q+1$ chia hết cho $p.$ Ta xét các trường hợp sau. 
\begin{enumerate}
    \item Nếu cả hai số $q+1$ và $q-2$ đều chia hết cho $p,$ ta có $p$ là ước của $3,$ và vì thế $p=3.$\\ Thế $p=3$ trở lại (\ref{snt5}), ta tìm ra $q=5.$
    \item Nếu chỉ một trong hai số $q-2$ và $q+1$ chia hết cho $p^2,$ ta có
    $$\hoac{p^2\le q-2 \\ p^2\le q+1}\Rightarrow p^2\le q+1\Rightarrow q\ge p^2-1.$$
    Phép so sánh này kết hợp với (\ref{snt5}) cho ta
    \begin{align*}
        p^2\tron{p-1}\ge \tron{p^2-3}p^2&\Rightarrow p-1\ge p^2-3\\&\Rightarrow p^2-p-2\le 0\\&\Rightarrow (p+1)(p-2)\le 0\\&\Rightarrow p=2.
    \end{align*}
    Tiếp tục thế $p=2$ trở lại (\ref{snt5}), ta tìm ra $q=3.$
\end{enumerate}
Như vậy, có hai cặp số nguyên tố $(p,q)$ thỏa yêu cầu là $(2,3)$ và $(3,5).$}
\end{gbtt}

\begin{gbtt}
Tìm tất cả các số nguyên tố $p,q$ thỏa mãn \[p^5+p^3+2=q^2-q.\]
\nguon{Argentina Cono Sur Team Selection Test 2014}
\loigiai{
Giả sử tồn tại các số nguyên tố $p,q$ thỏa mãn đề bài. Giả sử như vậy cho ta
\[p^3\tron{p^2+1}=\tron{q+1}\tron{q-2}.\tag{*}\label{acgentila}\]
Ta nhận thấy rằng một trong các số $q+1,q-2$ chia hết cho $p.$ Ta xét các trường hợp sau.
\begin{enumerate}
    \item Nếu cả $q+1$ và $q-2$ chia hết cho $p,$ ta có $p=3$ và $q=17.$
    \item Nếu $q+1$ hoặc $q-2$ chia hết cho $p^3,$ ta có
    $$\hoac{p^3\le q+1 \\ p^3\le q-2}\Rightarrow p^3\le q+1\Rightarrow q\ge p^3-1.$$
    Phép so sánh này kết hợp với (\ref{acgentila}) cho ta
    $$p^3\tron{p^2+1}\ge p^3\tron{p^3-3}\Rightarrow p^2+1\ge p^3-3\Rightarrow p^2(p-1)\le 4.$$
    Với $p\ge 3,$ bất đẳng thức bên trên đổi chiều. Với $p=2,$ ta tìm được $q=17.$
\end{enumerate}
Kết luận, có hai cặp số nguyên tố $(p,q)$ thỏa yêu cầu là $(2,7)$ và $(3,17).$}
\end{gbtt}

\begin{gbtt}
Tìm tất cả các số nguyên tố $p,q$ thỏa mãn
\[q^3+2q^2=6p^4+17p^3+60p^2+8q.\]
\loigiai{
Giả sử tồn tại các số nguyên tố $p,q$ thỏa mãn đề bài. Giả sử như vậy cho ta
\[q^3+2q^2-8q=6p^4+17p^3+60p^2\Rightarrow q\tron{q-2}\tron{q+4}=p^2\tron{6p^2+17p+60}.\tag{*}\label{snt6}\]
Từ đây, ta xét các trường hợp sau
\begin{enumerate}
    \item Nếu một trong ba số $q,q-2,q+4$ chia hết cho $p^2,$ ta có
    $$\hoac{&p^2\le q\\ &p^2\le q-2\\&p^2\le q+4}\Rightarrow p^2\le q+4\Rightarrow q\ge p^2-4.$$
    Phép so sánh kể trên kết hợp với (\ref{snt6}) cho ta
    $$p^2\tron{6p^2+17p+60}\ge \tron{p^2-4}\tron{p^2-6}p^2\Rightarrow 6p^2+17p+60\ge \tron{p^2-4}\tron{p^2-6}.$$
    Nếu $p\ge 5,$ bất đẳng thức bên trên không xảy ra do 
    \begin{align*}
      \tron{p^2-4}\tron{p^2-6}
      &\ge \tron{5p-4}\tron{5p-6}
      =6p^2+17p+60+\tron{19p^2-67p-36}
      \\&\ge 6p^2+17p+60+\tron{95p-67p-3}
      \\&>6p^2+17p+60.
    \end{align*}
    Như vậy $p<5.$ Thử với $p=2,p=3,$ ta tìm được $(p,q)=(3,11).$
    \item Nếu có hai trong ba số $q,q-2,q+4$ chia hết cho $p$, ta có $p$ chẵn, và thế thì $p=2.$ \\
    Thử với $p=2,$ ta không tìm được $q$ nguyên.
\end{enumerate}
Như vậy, các số nguyên tố $(p,q)$ thỏa mãn đề bài là $(3,11).$}
\end{gbtt}

\begin{gbtt}
Tìm tất cả các số nguyên tố $p,q$ thỏa mãn
\[p^8+7p^6=3q^2+11q.\]
\loigiai{
Giả sử tồn tại các số nguyên tố $p,q$ thỏa mãn. Giả sử này cho ta
\[p^6\tron{p^2+7}=q\tron{3q+11}.\tag{*}\label{snt9}\]
Dễ dàng nhận thấy $p=7$ không thỏa mãn (\ref{snt9}) nên $p\ne 7$ và $\tron{p^6,p^2+7}=1$.\\
Vì $q$ là số nguyên tố nên $q\mid p$ hoặc $q\mid p^2+7.$ Ta xét các trường hợp sau.
\begin{enumerate}
    \item Nếu $q\mid p$, ta suy ra $p=q.$ Ta thế trở lại (\ref{snt9}) và thu được
    $$p^8+7p^6=3p^2+11p\Rightarrow p=0.$$
    Điều này mâu thuẫn với điều kiện $p$ là số nguyên tố.
    \item Nếu $q\mid \tron{p^2+7}$, ta suy ra $p^6\mid(3q+11).$ Ta có
    $$\heva{&q\le p^2+7 \\ &p^6\le 3q+11}\Rightarrow \heva{&p^2+7\ge q \\ &3q+11\ge p^6}\Rightarrow p^6\le 3q+11\le 3p^2+32\Rightarrow p<2.$$
    Điều này không thể xảy ra.
\end{enumerate}
Như vậy, không tồn tại các số nguyên tố $(p,q)$ thỏa mãn.}
\end{gbtt}

%nguyệt anh
\begin{gbtt}
Tìm tất cả các số nguyên dương $n$ và số nguyên tố $p$ thỏa mãn
\[3p^2\tron{p+11}=n^3+n^2-2n.\]
\loigiai{
Giả sử tồn tại số nguyên dương $n$ và số nguyên tố $p$ thỏa mãn. Giả sử này cho ta
\[3p^2\tron{p+11}=n\tron{n+2}\tron{n-1}.\tag{*}\label{snt10}\]
Từ đây, ta xét các trường hợp sau.
\begin{enumerate}
    \item Nếu đúng một số trong $n,n-1,n+2$ chia hết cho $p$, chắc chắn số đó chia hết cho $p^2.$ Ta có
    $$\hoac{&p^2\le n \\ &p^2\le n-1 \\ &p^2\le n+2}\Rightarrow p^2\le n+2\Rightarrow n\ge p^2-2\ge p-2.$$
    Phép so sánh này kết hợp với (\ref{snt10}) cho ta
    $$3p^2\tron{p+11}\ge \tron{p-2}p\tron{p-3}\Rightarrow 3p(p+11)\ge \tron{p-2}\tron{p-3}\Rightarrow p^2+19p\le 3.$$
    Không tồn tại số nguyên tố $p$ nào như vậy.
    \item Có ít nhất hai số trong $n,n-1,n+2$ chia hết cho $p.$ Do
    $$n-(n-1)=1,\quad (n+2)-(n-1)=3,\quad (n+2)-n=2$$
    nên $p=2$ hoặc $p=3.$ Thử trực tiếp, ta tìm được $n=7$ khi $p=3.$
    \end{enumerate}
Như vậy, cặp số $(n,p)$ thỏa mãn đề bài là $(7,3).$}
\end{gbtt}

\begin{gbtt}
Tìm tất cả các số nguyên dương $n$ và số nguyên tố $p$ thỏa mãn
\[n^5+p^4=p^8+n.\]
\loigiai{
Giả sử tồn tại số nguyên dương $n$ và số nguyên tố $p$ thỏa mãn đề bài. Giả sử cho ta
\[n\tron{n^2-1}\tron{n^2+1}=p^4\tron{p^4-1}.\tag{*}\label{snt11}\]
Ta xét các trường hợp sau.
\begin{enumerate}
    \item Nếu duy nhất một số trong $n, n^2-1, n^2+1$ chia hết cho $p,$ số đó phải chia hết cho $p^4.$ Khi đó
        $$\hoac{&p^4\le n \\ &p^4\le n^2-1 \\ &p^4\le n^2+1}\Rightarrow p^4\le n^2+1\Rightarrow p^2\le n.$$
    Phép so sánh này kết hợp với (\ref{snt11}) cho ta
    $$n\tron{n^2-1}\tron{n^2+1}\le n^2\tron{n^2-1}.$$
    Không tồn tại số tự nhiên $n$ nào như vậy. Trường hợp này không xảy ra.
    \item Nếu ít nhất hai số trong $n,n^2-1,n^2+1$ chia hết cho $p,$ do
    $$\tron{n,n^2+1}=\tron{n,n^2-1}=1,\quad \tron{n^2-1,n^2+1}\in \{1;2\}$$
    nên $p=2.$ Thử trực tiếp, ta tìm ra $n=3.$
\end{enumerate}
Như vậy, cặp số $(n,p)$ duy nhất thỏa yêu cầu là $(3,2).$}
\end{gbtt}

\begin{gbtt}
Tìm các số nguyên dương $x, y, z$ sao cho $x^{2}+1, y^{2}+1$ đều là các số nguyên tố và
$$\left(x^{2}+1\right)\left(y^{2}+1\right)=z^{2}+1.$$
\nguon{Tạp chí Toán Tuổi thơ, ngày 20 tháng 5 năm 2020}
\loigiai{
Không mất tính tổng quát ta giả sử $x \geq y$. Ta sẽ so sánh $x,y$ và $z.$ Thật vậy
\begin{align*}
    z^2+1&=\left(x^{2}+1\right)\left(y^{2}+1\right)>x^2+1,\\
    z^2+1&=\left(x^{2}+1\right)\left(y^{2}+1\right)\le \left(x^2+1\right)^2<\left(x^2+1\right)^2+1.
\end{align*}
Vì lẽ đó, $y\le x<z\le x^2.$ Ngoài ra, phương trình đã cho tương đương
\[y^{2}\left(x^{2}+1\right)=(z-x)(z+x).\tag{*}\label{vailz}\]
Do $x^2+1$ là số nguyên tố nên hoặc $z-x,$ hoặc $z+x$ chia hết cho $x^2+1.$ Ta xét các trường hợp kể trên.
\begin{enumerate}
    \item Nếu $z-x$ chia hết cho $x^2+1$, ta có $x^2+1\le z-x<z<x^2,$ mâu thuẫn.
    \item Nếu $z+x$ chia hết cho $x^2+1$, ta có $$x^2+1\le z+x<x^2+x<2x^2+2.$$
    Ta suy ra $z+x=x^2+1,$ hay là $z=x^2-x+1$. Thế trở lại (\ref{vailz}), ta có
    $$y^2\left(x^2+1\right)=\left(x^2+1\right)(x+1)^2.$$
    Ta có $y=x+1,$ và khi ấy hai số nguyên tố $x^2+1,\ y^2+1$ khác tính chẵn lẻ. Số nhỏ hơn trong hai số này (là $y^2+1$) phải bằng $2$. Ta tìm ra $y=1,x=2,z=3$ từ đây.
\end{enumerate}
Kết luận, $(x,y,z)=(1,2,3)$ và $(x,y,z)=(2,1,3)$ là hai bộ số thỏa yêu cầu.}
\end{gbtt}

\begin{gbtt}
Tìm tất cả các cặp số nguyên tố $(p,q)$ thỏa mãn 
\[p+q=2(p-q)^2.\]
\nguon{Chuyên Đại học Vinh 2016}
\loigiai{
Giả sử tồn tại các số nguyên tố $p,q$ thỏa yêu cầu. Ta có
$$p+q=2p^2-4pq+2q^2.$$
Lấy đồng dư theo modulo $p$ hai vế, ta chỉ ra
$$q\equiv q^2\pmod{p}\Rightarrow q(2q-1)\equiv 0\pmod{p}.$$
Lấy đồng dư theo modulo $q$ hai vế, ta chỉ ra
$$p\equiv 2p^2\pmod{q}\Rightarrow p(2p-1)\equiv 0\pmod{q}.$$
Do $p\ne q$ nên $2p-1$ chia hết cho $q$ và $2q-1$ chia hết cho $q.$ Giả sử $p\ge q.$ Ta có
$$1\le \dfrac{2q-1}{p}\le \dfrac{2p-1}{p}<2.$$
Do $2q-1$ chia hết cho $p$ nên $2q-1=p.$ Lúc này, $$2p-1=2(2q-1)-1=4q-3$$
chia hết cho $q,$ thế nên $q=3,$ và đồng thời $p=5.$\\ Kiểm tra trực tiếp, ta thấy có hai cặp $(p,q)$ thỏa yêu cầu là $(3,5)$ và $(5,3).$}
\end{gbtt}


\begin{gbtt}
Tìm tất cả các số nguyên tố $p,q,r$ và số tự nhiên $n$ thỏa mãn
\[p^3=q^3+9r^n.\]

\loigiai{
Đầu tiên, nếu cả ba số $p,q,r$ đều lẻ, hai vế phương trình khác tính chẵn lẻ, mâu thuẫn. Do vậy, trong các số $p,q,r$ phải có một số bằng $2.$ Ta xét các trường hợp dưới đây.
\begin{enumerate}
    \item Với $p=2$, ta có $q^3+9r^n=8$. Ta không tìm được $q$ và $r$ từ đây, do
    $$q^3+9r^n\ge 2^3+9\cdot2=26\geq 8.$$
    \item Với $q=2$, ta có $9r^n=p^3-8=\tron{p-2}\tron{p^2+2p+4}.$ \\
    Dễ thấy $\tron{p-2,p^2+2p+4}\in\{1;3\}.$ Ta xét các trường hợp kể trên.
    \begin{itemize}
        \item \chu{Trường hợp 1.} Với $\tron{p-2,p^2+2p+4}=1,$ bằng lập luận được rằng $$p^2+2p+4=(p+1)^2+3$$ không thể chia hết cho $9,$ ta suy ra $p-2$ chia hết cho $9,$ tức là $p\equiv 2\pmod{9}.$ Lúc này        $$p^2+2p+4\equiv 2^2+2\cdot2+4=12\equiv 0\pmod{3}.$$
        Trong trường hợp này, cả $p-2$ và $p^2+2p+4$ đều chia hết cho $3,$ mâu thuẫn.
        \item \chu{Trường hợp 2.} Với $\tron{p-2,p^2+2p+4}=3,$ do $p^2+2p+4$ không thể chia hết cho $9,$ ta có thể đặt $p-2=3x$ và $ p^2+2p+4=3y$ trong đó $x\le y$ và $\tron{x,y}=1$. Phép đặt này cho ta
    $$9r^n=3x\cdot3y\Rightarrow r^n=xy.$$
    Do $\tron{x,y}=1$ và $x\le y$, ta thu được $x=1,$ kéo theo $p=3x+2=5$. Thế trở lại, ta có
    $$5^3=2^3+9r^n\Rightarrow 9r^n=117\Rightarrow r=13.$$
    \end{itemize}
    \item Với $r=2$, ta có $9\cdot2^n=\tron{p-q}\tron{p^2+pq+q^2}.$
    Ta đặt $d=\tron{p-q,p^2+pq+q^2}.$ Theo đó
    $$d\mid \tron{p^2+pq+q^2}-\tron{p-q}^2=3pq.$$
    Ta xét các trường hợp sau đây.
    \begin{itemize}
        \item \chu{Trường hợp 1.} Với $3\mid d$, ta đặt $p-q=3x$ và $p^2+pq+q^2=3y$. Ta chứng minh tương tự ý trên và thu được $x=1,p=5, q=2.$
        \item \chu{Trường hợp 2.}Với $p\mid d$, ta thu được
        $$p\mid\tron{p-q}\tron{p^2+pq+q^2}=9\cdot2^n$$
        Từ đây, ta suy ra $p=2$ hoặc $p=3$. Thử trực tiếp, ta thấy không có số nào thỏa mãn.
        \item \chu{Trường hợp 3.} Với $q\mid d$, tương tự trường hợp trên, ta thấy không có số nào thỏa mãn
        \item \chu{Trường hợp 4.} Với $d=1$, ta dễ dàng chứng minh được $p-q=2^n$ và $p^2+pq+q^2=9$. Vì $p^2+pq+q^2=9$, nên $q\le p<3.$ Nhờ giả thiết $p,q$ là hai số nguyên tố, ta suy ra $p=q=2$. Thông qua thử trực tiếp, ta thấy chúng không thỏa mãn.
    \end{itemize}
\end{enumerate}
Như vậy, có duy nhất một bộ số nguyên tố thỏa yêu cầu là $(p,q,r)=(5,2,13).$}
\end{gbtt} %dùng delta + pt nghiệm nguyên tố
\section{Phương pháp kẹp lũy thừa trong phương trình nghiệm nguyên}
Trong mục này, ta sẽ ôn lại một kiến thức đa học ở \chu{chương III}.
\subsection*{Lí thuyết}
 Giữa hai lũy thừa số mũ $n$ liên tiếp, không tồn tại một lũy thừa cơ số $n$ nào. Hệ quả, với mọi số nguyên $a$ 
    \begin{enumerate}
        \item Không có số chính phương nào nằm giữa $a^2$ và $\left(a+1\right)^2.$
        \item Số chính phương duy nhất nằm giữa $a^2$ và $\left(a+2\right)^2$ là $\left(a+1\right)^2.$    
        \item Có $k-1$ số chính phương nằm giữa $a^2$ và $\left(a+k\right)^2,$ bao gồm \[\left(a+1\right)^2,\left(a+2\right)^2,\ldots,\left(a+k-1\right)^2.\]         
    \end{enumerate}
    Về các kết quả tương tự với số mũ khác, mời bạn đọc tự nghiên cứu và phát biểu.
\subsection*{Bài tập tự luyện}

\begin{btt}
Giải phương trình nghiệm tự nhiên
$$x^2+3x+4=y^2.$$
\end{btt}

\begin{btt}
Giải phương trình nghiệm tự nhiên
$$x^2-5x=y^2-2y-5.$$
\end{btt}

\begin{btt}
Giải phương trình nghiệm nguyên
$$x^3+2x^2+3x+1=y^3.$$
\end{btt}
%nguyệt anh
\begin{btt}
Giải phương trình nghiệm nguyên $$y^{3}-2 x-2=x(x+1)^2.$$
\nguon{Chuyên Toán Hưng Yên 2017}
\end{btt}

\begin{btt}
Giải phương trình nghiệm nguyên
$$x^2+x=y^4+y^3+y^2.$$
\end{btt}

\begin{btt} 
Tìm các số nguyên dương $x,y$ thỏa mãn
$$y^4+2y^3-3=x^2-3x.$$
\nguon{Chuyên Toán Hải Phòng 2021}
\end{btt} %haiphong

\begin{btt}
Giải phương trình nghiệm nguyên dương
$$\min \left\{x^{4}+8 y;\  8 x+y^{4}\right\}=(x+y)^{2}.$$
\nguon{Titu Andreescu}
\end{btt}

\begin{btt}
Giải phương trình nghiệm nguyên dương 
\[a^2+b+3=\left ( b^2-c^2 \right )^2.\]
\nguon{Japan Mathematical Olympiad Final 2019}
\end{btt}

\begin{btt}
Giải hệ phương trình sau trên tập số nguyên
$$\heva{y^3&=x+1 \\ z^3&=x^2-x+2.}$$ 
\end{btt}

\begin{btt}
Giải phương trình nghiệm nguyên dương
\[4y^2+28y+17=7^x.\]
\end{btt}

\begin{btt}
Tìm tất cả các cặp số tự nhiên $x,y$ thỏa mãn
\[2017^x=y^6-32y+1.\]
\nguon{Austrian Mathematical Olympiad 2017}
\end{btt}

\begin{btt}
Giải phương trình nghiệm tự nhiên
$$y^4+6y^3+3y^2-10y+81=3^x.$$
\end{btt}

\begin{btt}
Giải phương trình nghiệm tự nhiên 
$$5^x=y^4+4y+1.$$
\nguon{Tạp chí Toán học và Tuổi trẻ số 440}
\end{btt}

\begin{btt}
Giải phương trình nghiệm tự nhiên
$$x^3+6x+7=3^y.$$
\end{btt}

\begin{btt}
Giải phương trình nghiệm nguyên dương
$$5^x+8x+15=16y^2+16y.$$
\end{btt}


\subsection*{Hướng dẫn bài tập tự luyện}


\begin{gbtt}
Giải phương trình nghiệm tự nhiên
\[x^2+3x+4=y^2.\]
\loigiai{
Với mọi số tự nhiên $x,$ ta có nhận xét
$$(x+1)^2<x^2+3x+4\le (x+2)^2.$$
Do $x^2+3x+4$ là số chính phương, bắt buộc $x^2+3x+4=(x+2)^2,$ vậy nên $x=0.$ \\Kiểm tra trực tiếp, ta kết luận $(x,y)=(0,2)$ là nghiệm tự nhiên duy nhất của phương trình.}
\end{gbtt}

\begin{gbtt}
Giải phương trình nghiệm tự nhiên
\[x^2-5x=y^2-2y-5.\]
\loigiai{
Phương trình đã cho tương đương với $$x^2-5x+6=(y-1)^2.$$
    Với mọi số tự nhiên $x\ge 4,$ ta có nhận xét
    $$(x-3)^2< x^2-5x+6< (x-2)^2.$$
    Do $x^2-5x+6$ là số chính phương, mọi $x\ge 4$ đều không thỏa mãn. Kiểm tra trực tiếp với $x=0,1,2,3,$ ta kết luận $(x,y)=(2,1)$ và $(x,y)=(3,1)$ là hai nghiệm tự nhiên của phương trình.}
\end{gbtt}

\begin{gbtt}
Giải phương trình nghiệm nguyên
\[x^3+2x^2+3x+1=y^3.\]
\loigiai{
Dựa trên một số nhận xét
\begin{align*}
    y^3&=\tron{x^3+3x^2+3x+1}-x^2\\&=(x+1)^3-x^2 \\&\leq (x+1)^3,\\
    y^3&=x^3+2x^2+3x+1\\&=(x^3-3x^2+3x-1)+5x^2+2\\&=(x-1)^3+5x^2+2\\& > (x-1)^3,
\end{align*}
ta chỉ ra $y=x$ hoặc $y=x+1.$
    \begin{enumerate}
        \item Với $y=x$, phương trình đã cho trở thành
        $$x^3=x^3+2x^2+3x+1 \Leftrightarrow 2x^2+3x+1=0\Leftrightarrow (x+1)(2x+1)=0.$$
        Trường hợp này cho ta $x=y=-1.$
        \item Với $y=x+1$, phương trình đã cho trở thành 
        $$(x+1)^3=x^3+2x^2+3x+1 \Leftrightarrow x^2=0\Leftrightarrow x=0.$$
        Trường hợp này cho ta $x=0$ và $y=1.$
    \end{enumerate}
Kết luận, tất cả các cặp $(x,y)$ thỏa mãn đẳng thức là $(-1,-1)$ và $(0,1).$}
\end{gbtt}

%nguyệt anh
\begin{gbtt}
Giải phương trình nghiệm nguyên $y^{3}-2 x-2=x(x+1)^2.$
\nguon{Chuyên Toán Hưng Yên 2017}
\loigiai{
Phương trình đã cho tương đương với
$$y^{3}-2 x-2=x(x+1)^2\Leftrightarrow y^3=x^3+2x^2+3x+2.$$
Với mọi số nguyên $x,$ ta luôn có
$$x^3<x^3+2x^2+3x+2<x^3+6x^2+12x+8.$$
Ta chỉ ra $x^3+2x^2+3x+2=x^3+3x^2+3x+1$, vậy nên $x=1$ hoặc $x=-1$.\\
Thế ngược lại, ta kết luận phương trình đã cho có $2$ nghiệm nguyên $(x,y)$ là $(1,2),(-1,0).$
}
\end{gbtt}


\begin{gbtt}
Giải phương trình nghiệm nguyên
\[x^2+x=y^4+y^3+y^2.\]
\loigiai{
Phương trình đã cho tương đương với 
$$4x^2+4x=4y^4+4y^3+4y^2\Leftrightarrow (2x+1)^2=4y^4+4y^3+4y^2+1.$$
Ta có các đánh giá sau.
    \begin{align*}
        \tron{4y^4+4y^3+4y^2+1}-(2y^2+y)^2&=3y^2+1>0,\\
        (2y^2+y+2)^2-\tron{4y^4+4y^3+4y^2+1}&=5y^2+4y+3>0.
    \end{align*}
Các đánh giá theo hiệu trên cho ta $$(2y^2+y)^2<4y^4+4y^3+4y^2+1<(2y^2+y+2)^2.$$
Do $4y^4+4y^3+4y^2+1$ là số chính phương nên  $$4y^4+4y^3+4y^2+1=(2y^2+y+1)^2.$$ Ta tìm ra $y=0$ hoặc $y=-2.$
    \begin{enumerate}
        \item Với $y=0,$ ta có $(2x+1)^2=1,$ hay là $x=0$ hoặc $x=-1.$
        \item Với $y=-2,$ ta có $(2x+1)^2=49,$ hay là $x=3$ hoặc $x=-4.$
    \end{enumerate}
    Kết luận, phương trình có tất cả $4$ nghiệm nguyên $(x,y)$ là $(0,0),(3,-2),(-1,0),(-4,-2).$}
\end{gbtt}

\begin{gbtt} 
Tìm các số nguyên dương $x,y$ thỏa mãn
$y^4+2y^3-3=x^2-3x.$
\nguon{Chuyên Toán Hải Phòng 2021}
\loigiai{
Phương trình đã cho tương đương với
$$4y^4+8y^3-12=4x^2-12x\Leftrightarrow 4y^4+8y^3-3=(2x-3)^2.$$
Tới đây, ta xét các hiệu
\begin{align*}
    \tron{4y^4+8y^3-3}-\tron{2y^2+2y-1}^2&=4(y-1)\ge 0,\\
    \tron{2y^2+2y}^2-\tron{4y^4+8y^3-3}&=4y^2+3>0.
\end{align*}
Các đánh giá theo hiệu bên trên cho ta biết
$$\tron{2y^2+2y-1}^2\le 4y^4+8y^3-3< \tron{2y^2+2y}^2.$$
Do $4y^4+8y^3-3$ là số chính phương, ta bắt buộc phải có $4y^4+8y^3-3=\tron{2y^2+2y-1}^2,$ tức $y=1.$ \\
Ta tìm ra $x=3$ từ đây. Kết luận, $(x,y)=(1,3)$ là cặp số duy nhất thỏa mãn đề bài.
}
\end{gbtt} %haiphong

\begin{gbtt}
Giải phương trình nghiệm nguyên dương
$$\min \left\{x^{4}+8 y;\  8 x+y^{4}\right\}=(x+y)^{2}.$$
\nguon{Titu Andreescu}
\loigiai{Do đổi chỗ $x$ và $y$ không làm thay đổi dữ kiện bài toán nên không mất tính tổng quát, ta có thể giả sử $$8 x+y^{4}=(x+y)^{2}.$$ Ta viết lại phương trình thành
$$(x+y-4)^{2}=y^{4}-8 y+16.$$
Do đó $y^{4}-8 y+16$ phải là số chính phương. Nếu $y \geq 3$, ta nhận xét được
$$\left(y^{2}-1\right)^{2}<y^4-8y+16<\tron{y^2}^2.$$
Theo kiến thức đã học, $y^4-8y+16$ không là số chính phương lúc này. Chính vì thế, ta có $y\in\{1;2\}.$ Thử lại, ta kết luận phương trình đã cho có các nghiệm nguyên dương là
$$(x, y)=(6,2), \quad(x, y)=(6,1), \quad(x, y)=(1,6), \quad(x, y)=(2,6).$$}
\end{gbtt}

\begin{gbtt}
Giải phương trình nghiệm nguyên dương 
\[a^2+b+3=\left ( b^2-c^2 \right )^2.\]
\nguon{Japan Mathematical Olympiad Final 2019}
\loigiai{
Dễ thấy $b\ne c.$ Ta xét các trường hợp sau đây.
\begin{enumerate}
    \item Nếu $b< a,$ ta có đánh giá
    $$a^2<a^2+b+3<a^2+a+2\le (a+1)^2.$$
    Do $a^2+b+3$ là số chính phương nên dấu bằng trong đánh giá trên phải xảy ra. \\Ta tìm ra $a=1$ từ đây, nhưng khi đó $b<1,$ vô lí.
    \item Nếu $b>a,$ ta có đánh giá
    $$a^2+b+3=\left ( b^2-c^2 \right )^2=(b-c)^2(b+c)^2\ge (b+c)^2\ge (b+1)^2=b^2+2b+1.$$
    Từ đánh giá trên, ta suy ra
    $$a^2\ge b^2+b-2\ge (a+1)^2+(a+1)-2=a^2+3a.$$
    Ta thu được mâu thuẫn.
    \item Nếu $b=a,$ thế trở lại phương trình, ta được
    $$a^2+a+3=\tron{a^2-c^2}^2.$$
    Ta có $a^2+a+3$ là số chính phương. Bằng nhận xét
    $$a^2<a^2+a+3<(a+2)^2$$
    ta chỉ ra $a^2+a+3=(a+1)^2$ hay $a=2.$ Thế trở lại, ta được $b=2$ và $c=1.$
\end{enumerate}
Như vậy phương trình đã cho có duy nhất một nghiệm nguyên dương là $(a,b,c)=(2,2,1).$}
\end{gbtt}
%Châu
\begin{gbtt}
Giải hệ phương trình sau trên tập số nguyên
\[\heva{y^3&=x+1 \\ z^3&=x^2-x+2.}\]
\loigiai{
Nhân lần lượt theo vế, ta thu được
$$y^3z^3=\tron{x+1}\tron{x^2-x+2},$$
hay là $(yz)^3=x^3+x+2.$ Với mọi số nguyên $x$, ta luôn có
    $$x^3-3x^2+3x-1<x^3+x+2<x^3+6x^2+12x+8.$$
Ta thu được $x^3+x+2$ có thể bằng $x^3$ hoặc $\tron{x+1}^3$. Ta xét hai trường hợp sau.
\begin{enumerate}
       \item Với $x^3+x+2=x^3,$ ta có $x=-2.$ Thế trở lại, ta được $(x,y,z)=(-2,-1,2).$
         \item Với $x^3+x+2=(x+1)^3,$ ta có $x=-1.$ Thế trở lại, ta không tìm được $z$ nguyên
\end{enumerate}
    Như vậy, phương trình đã cho có nghiệm nguyên duy nhất là $(-2,-1,2).$}
\end{gbtt}

\begin{gbtt}
Giải phương trình nghiệm nguyên dương
\[4y^2+28y+17=7^x.\]
\loigiai{
Giả sử tồn phương trình có nghiệm nguyên dương $x,y$ thỏa mãn.
Biến đổi phương trình đã cho, ta có
$$4y(y+7)+17=7^x.$$
Vì $y,(y+7)$ khác tính chẵn lẻ nên $2\mid y(y+7)$ kéo theo $8\mid 4y(y+7).$\\
Xét trong hệ đồng dư modulo $8$, ta nhận được
$$7^x=4y(y+7)+17\equiv0+17\equiv1\pmod{8}.$$
Vì $7^x$ chia $8$ dư $1$ nên $x$ chia hết cho $2$. Đặt $x=2a$, thế trở lại phương trình, ta nhận được
$$4y^2+28y+17=\tron{7^a}^2.$$
Với mọi số nguyên dương $y$, ta luôn có
$$(2y+5)^2\le4y^2+28y+17<(2y+7)^2. $$
Do $4y^2+28y+17=\tron{7^x}^2$ là số chính phương nên $4y^2+28y+17$ bằng $(2y+5)^2$ hoặc $(2y+6)^2,$ và thế thì $y=1, \ x=2$. Phương trình đã cho có nghiệm nguyên dương duy nhất là $(x,y)=(2,1).$}
\end{gbtt}

\begin{gbtt}
Tìm tất cả các cặp số tự nhiên $x,y$ thỏa mãn
\[2017^x=y^6-32y+1.\]
\nguon{Austrian Mathematical Olympiad 2017}
\loigiai{Đầu tiên, ta có $y$ là số chẵn, thế nên
$$\heva{&64\mid y^6 \\ &64\mid 32y}\Rightarrow 64\mid \tron{y^6-32y}\Rightarrow 64\mid\tron{2017^x-1}.$$
Tới đây, ta xét các trường hợp sau.
\begin{enumerate}
    \item Nếu $x$ lẻ, ta đặt $x=2k+1.$ Ta có
        $$2017^x=2017^{2k+1}\equiv 33^{2k+1}=33\cdot1089^k\equiv 33\pmod{64}.$$
        Điều này mâu thuẫn với lập luận $2017^x-1$ chia hết cho $64$ ở trên.
    \item Nếu $x$ chẵn, ta có $2017^x$ là số chính phương. \\Với $y=0$ hoặc $y=2,$ thử trực tiếp, ta tìm ra $x=0.$ Với $y\ge 4,$ ta có các nhận xét
        \begin{align*}
           y^6-32y+1&<y^6=\tron{y^3}^2,\\ y^6-32y+1-\tron{y^3-1}^2&=2y^3-32y=2y\tron{y-4}\tron{y+4}\ge 0.
        \end{align*}
        Nhận xét trên kết hợp với chú ý $y^6-32y+1$ là số chính phương cho ta $$y^6-32y+1=\tron{y^3-1}^2.$$ Ta tìm ra $y=4$ từ đây, nhưng không tìm ngược lại được $x$ nguyên.
\end{enumerate}
    Tổng kết lại, có $2$ cặp số tự nhiên $(x,y)$ thỏa mãn đề bài là $(0,0)$ và $(0,2).$}
\end{gbtt}

\begin{gbtt}
Giải phương trình nghiệm tự nhiên
\[y^4+6y^3+3y^2-10y+81=3^x.\]
\loigiai{
Giả sử tồn phương trình có nghiệm tự nhiên $x,y$ thỏa mãn. Ta có
$$y(y-1)(y+2)(y+5)+81=3^x.$$
Lấy đồng dư theo modulo $4$ cho vế trái, ta chỉ ra
$$y(y-1)(y+2)(y+5)+81\equiv y(y-1)(y-2)(y-3)+1\equiv 1\pmod{4}.$$
Do vậy, $3^x$ chia $4$ dư $1$. Xét tính chẵn lẻ của $x,$ ta chỉ ra $x$ chẵn. Đặt $x=2a,$ ta nhận được
$$y^4+6y^3+3y^2-10y+81=\tron{3^a}^2.$$
Với $y=0$ và $y=1,$ ta có $3^x=81$. Từ đây, ta suy ra $(x,y)=(4,0),(4,1).$ Với $y\ge2,$ ta xét các hiệu sau
\begin{align*}
\tron{y^2+3y+2}^2-\tron{y^4+6y^3+3y^2-10y+81}&=10y^2+22y-77>0,\\
\tron{y^4+6y^3+3y^2-10y+81}-\tron{y^2+3y-3}^2&=8y+72>0.
\end{align*}
Các đánh giá theo hiệu bên trên cho ta biết
$$\tron{y^2+3y-3}^2<y^4+6y^3+3y^2-10y+81<\tron{y^2+3y+2}^2.$$
Do $y^4+6y^3+3y^2-10y+81$ là số chính phương, ta thu được các trường hợp sau.
\begin{enumerate}
    \item Với $y^4+6y^3+3y^2-10y+81=\tron{y^2+3y-2}^2,$ ta có
    $$2y^2-2y-77=0.$$
    Phương trình trên không có nghiệm tự nhiên.
    \item Với $y^4+6y^3+3y^2-10y+81=\tron{y^2+3y-1}^2,$ ta có
    $$4y^2+4y-80=0\Leftrightarrow 4(y-4)(y+5)=0.$$
    Ta tìm được $y=4,$ kéo theo $x=6.$
    \item Với $y^4+6y^3+3y^2-10y+81=\tron{y^2+3y}^2,$ ta có
    $$6y^2+10y-81=0.$$
    Phương trình trên không có nghiệm tự nhiên.    
    \item Với $y^4+6y^3+3y^2-10y+81=\tron{y^2+3y+1}^2,$ ta có
    $$8y^2+16y-80=0.$$
    Phương trình trên không có nghiệm tự nhiên.        
\end{enumerate}
Như vậy, phương trình đã cho có các nghiệm tự nhiên là $$(4,0),\quad (4,1),\quad (6,4).$$
}
\end{gbtt}

\begin{gbtt}
Giải phương trình nghiệm tự nhiên 
$$5^x=y^4+4y+1.$$
\nguon{Tạp chí Toán học và Tuổi trẻ số 440}
\loigiai{
Ta sẽ đi chứng minh $x$ chẵn. Thật vậy, nếu như $x$ là số lẻ, ta có 
$$y^{4}+4y=5^x-1\equiv 2-1\equiv 1\pmod{3}.$$
Song, từ việc xét các số dư của $y$ khi chia cho $3,$ ta lập được bảng đồng dư
\begin{center}
    \begin{tabular}{c|c|c|c}
      $y$   & $0$ & $1$ & $2$ \\
      \hline
        $y^4+4y$ & $0$ & $2$ & $0$  
    \end{tabular}
\end{center}
Dựa vào bảng, ta chỉ ra điều mâu thuẫn. Ta có $x$ chẵn. Đặt $x = 2k.$ Phương trình đã cho trở thành
$$\tron{5^k}^2=y^{4}+4y+1.$$
Mặt khác, với $y\ge 3,$ ta lại nhận xét được
\begin{align*}
    y^4+4y+1-y^4&=4y+1>0,\\
    \tron{y^2+1}^2-\tron{y^4+4y+1}&=2y^2-4y=2y(y-2)>0.
\end{align*}
Như vậy $y^4+4y+1$ không là số chính phương với mọi $y\ge 2.$ Với $y=0$ và $y=1,$ kiểm tra trực tiếp rồi thử lại, ta kết luận phương trình đã cho có hai nghiệm tự nhiên là $(0,0)$ và $(2,2).$}
\end{gbtt}


\begin{gbtt}
Giải phương trình nghiệm tự nhiên
\[x^3+6x+7=3^y.\]
\loigiai{
Giả sử phương trình đã cho tồn tại nghiệm tự nhiên $(x,y)$ thỏa mãn. \\
Xét bảng đồng dư modulo $13$, ta có
\begin{center}
    \begin{tabular}{c|c|c|c|c|c|c|c|c|c|c|c|c|c}
        $x$ &  $0$&  $1$&  $2$&  $3$&  $4$&  $5$&  $6$&  $-6$&  $-5$&  $-4$&  $-3$&   $-2$&  $-1$ \\
        \hline
        $x^3+6x+7$&$7$&$1$  &$1$&$0$&$4$&$6$&$12$&$2$&$8$&$10$&$1$&$0$&$0$                    
    \end{tabular}
\end{center}
Ta có nhận xét sau.
\begin{enumerate}
    \item Với $y=3k,$ ta luôn có $3^y\equiv27^k\equiv1\pmod{13}.$
     \item Với $y=3k+1,$ ta luôn có $3^y\equiv27^k\cdot3\equiv3\pmod{13}.$
      \item Với $y=3k,$ ta luôn có $3^y\equiv27^k\cdot9\equiv9\pmod{13}.$
\end{enumerate}
Từ những nhận xét trên kết hợp với $x^3+6x+7=3^y$, ta thu được $3^y\equiv1\pmod{13}$ kéo theo $y=3k.$ \\Thế trở lại phương trình, ta nhận được
$$x^3+6x+7=\tron{3^k}^3.$$
Với mọi số tự nhiên $x$, ta nhận thấy
$$x^3<x^3+6x+7<\tron{x+2}^3.$$
Ta suy ra $x^3+6x+7=(x+1)^3.$ Giải phương trình, ta thu được $x=2$ là nghiệm tự nhiên thỏa mãn, kéo theo $y=3.$ Phương trình đã cho có duy nhất một nghiệm tự nhiên là $(x,y)=(2,3).$}
\end{gbtt}

\begin{gbtt}
Giải phương trình nghiệm nguyên dương
\[5^x+8x+15=16y^2+16y.\]
\loigiai{
Giả sử phương trình đã cho có nghiệm nguyên dương $(x,y)$ thỏa mãn. Xét trong hệ modulo $8.$
\begin{enumerate}
    \item Với $x=2k$, ta có $5^x+8x+15\equiv1+15\equiv0\pmod{8}.$
     \item Với $x=2k+1$, ta có $5^x+8x+15\equiv5+8+15\equiv4\pmod{8}.$
\end{enumerate}
Lại có $16y^2+16y\equiv0\pmod{8}.$ Từ đây, ta suy ra $x=2k.$ Thế trở lại phương trình, ta nhận được
$$\tron{5^k}^2+16k+19=\tron{4y+2}^2.$$
Với mọi số nguyên dương $k>2,$ ta luôn có
$$\tron{5^k}^2<\tron{5^k}^2+16k+19<\tron{5^k+1}^2.$$
Do vậy, không tồn tại $k$ thỏa mãn. Chỉ có $k=1$ hoặc $k=2.$ Thế trở lại, ta tìm được $y=6.$\\ Phương trình đã cho có duy nhất nghiệm nguyên dương $(x,y)$ là $(4,6).$}
\end{gbtt}


\section{Phép gọi ước chung}
Phép gọi ước chung được dùng chủ yếu trong hai trường hợp sau đây.
\begin{enumerate}
    \item Liên hệ giữa sự chênh lệch bậc và tính nguyên tố cùng nhau.
    \item Áp dụng bổ đề về ước chung lớn nhất (xem \chu{chương III}).
\end{enumerate}
Một số tài liệu còn sử dụng phương pháp lùi vô hạn thay thế cho phép gọi ước chung cho phần nửa đầu của mục này. Bạn đọc có thể tham khảo phương pháp ấy trên các tài liệu số học khác.
\subsection*{Ví dụ minh họa}

\begin{bx} \label{baimodaugcd}
Giải phương trình nghiệm nguyên $x^3+2y^3=4z^3.$
\loigiai{
Ta nhận thấy phương trình có nghiệm $(x,y,z)=(0,0,0).$ \\ Nếu như $x\ne 0,$ ba số $x,y,z$ tồn tại ước chung, gọi là $d.$ Ta đặt
$$x=dx_1,y=dy_1,z=dz_1.$$
Một cách hiển nhiên, $(x_1,y_1,z_1)=1.$ Ngoài ra, do $d\ne 0,$ phương trình đã cho trở thành
\[x^3_1+2y^3_1=4z^3_1.\tag{1}\label{lvh1}\]
Rõ ràng $x_1$ chia hết cho $2.$ Đặt $x_1 = 2x_2$ với $x_1$ là số nguyên. Thế vào (\ref{lvh1}), ta được
\[8x^3_2+2y^3_1=4z^3_1\Leftrightarrow 4x^3_2+y^3_1=2z^3_1.\tag{2}\label{lvh2}\]
Rõ ràng $y_1$ chia hết cho $2.$ Tiếp tục đặt $y_1 = 2y_2$ với $y_1$ là số nguyên. Thế vào (\ref{lvh2}), ta được
\[4x^3_2+8y^3_2=2z^3_1\Leftrightarrow 2x^3_2+4y^3_2=z^3_1.\]
Ta tiếp tục nhận được $z_1$ chia hết cho $3.$ Cả $x_1,y_1$ và $z_1$ đều chia hết cho $2,$ chứng tỏ $(x_1,y_1,z_1)\ge 2,$ mâu thuẫn với điều kiện $(x_1,y_1,z_1)=1.$ Trường hợp $x,y,z$ không đồng thời bằng $0$ không xảy ra. Nghiệm nguyên duy nhất của phương trình là $(x,y,z)=(0,0,0).$}
\end{bx}

\begin{bx}
Tìm tất cả các bộ số nguyên dương $(x,y)$ thỏa mãn $$x^3-y^3=95\left(x^2+y^2\right).$$ 
\nguon{Chuyên Đại học Sư phạm Hà Nội 2016}
\loigiai{Nhờ điều kiện $x,y>0,$ ta có thể đặt $d=(x,y).$ Lúc này, tồn tại các số nguyên dương $a,b$ thỏa mãn $(a,b)=1,x=da,y=db.$ Phương trình đã cho trở thành
$$d({a - b})\left( {{a^2} + ab + {b^2}}\right) = 95\left({a^2} + {b^2}\right).$$ 
Từ điều kiện $(a,b) = 1,$ ta dễ dàng chứng minh $\left(a^2+ab+b^2\right)=\left(ab,a^2+b^2\right)= 1.$ Chứng minh này cho ta biết $95$ chia hết cho ${a^2} + ab + {b^2}.$ Ta xét các trường hợp sau đây.
\begin{enumerate}
    \item Nếu $a^2+ab+b^2$ là bội của $5,$ cả $a$ và $b$ đều chia hết cho $5$. Vì lẽ đó, $(a,b)\ge 5,$ mâu thuẫn với điều kiện $(a,b)=1.$
    \item Nếu $a^2+ab+b^2=1,$ ta có $1=a^2+ab+b^2\ge 1+1+1=3,$ mâu thuẫn.
    \item Nếu $a^2+ab+b^2=19,$ ta có
    $$19=a^2+ab+b^2\ge a^2+1+1,$$
    thế nên $a\le 4.$ Thử với $a=1,2,3,4,$ ta tìm ra $(a,b)=(3,2)$ hoặc $(a,b)=(2,3),$ nhưng do $a>b$ nên $a=3,b=2.$ Thay ngược lại, ta tính được $d=65,$ kéo theo $x=195,y=130.$
\end{enumerate}
Kết luận, có duy nhất một cặp số $(x,y)$ thỏa mãn đề bài là $(195,130).$}
\end{bx}

\begin{bx}
Tìm các số nguyên $x,y$ thỏa mãn $x\left(x^2+x+1\right) = 4y\left(y+1 \right).$
\loigiai{
Phương trình đã cho tương đương
$$x^3+x^2+x+1= \left( {2y + 1} \right)^2 \Leftrightarrow \left(x^2+1 \right)\left(x+1 \right) =\left( 2y+1\right)^2.$$ 
Ta nhận thấy $x^2+1\ne 0,$ thế nên giữa $x^2+1$ và $x+1$ tồn tại ước chung. Đặt $\left(x^2+1,x+1\right)=d,$ và phép đặt này cho ta 
\begin{align*}
    \heva{&d\mid \left(x^2+1\right) \\ &d\mid (x+1)}
    &\Rightarrow 
    \heva{&x^2+1\equiv 0\pmod{d} \\ &x\equiv -1\pmod{d}}
    \\&\Rightarrow
    (-1)^2+1\equiv 0\pmod{d}\\&
    \Rightarrow 
    d\mid 2.
\end{align*}
Nhận xét này cho ta $d=1$ hoặc $d=2.$ \begin{enumerate}
    \item Với $d=1,$ theo bổ đề, ta chỉ ra $x^2+1$ là số chính phương, kéo theo $x=0.$ Ta tìm ra $y=0$ hoặc $y=-1$ từ đây.
    \item Với $d=2,$ cả $x^2+1$ và $x+1$ đều chia hết cho $2,$ thế nên $(2y+1)^2=\left(x^2+1 \right)\left(x+1 \right)$ chia hết cho $4,$ mâu thuẫn.
\end{enumerate}
Như vậy, $(x,y)=(0,-1)$ và $(x,y)=(0,0)$ là hai cặp số thỏa mãn đề bài.}
\end{bx}

\subsection*{Bài tập tự luyện}

\begin{btt} Giải phương trình nghiệm nguyên $$x^2+y^2=3z^2.$$
\end{btt}

\begin{btt}
Tìm tất cả các số nguyên $x,y,z$ thỏa mãn $$4x^2+4xy+3y^2=5z^2.$$
\end{btt}


\begin{btt}
Tìm tất cả các cặp số nguyên $(m,n)$ sao cho $$6(m+1)(n-1), \quad (m-1)(n+1)+6,\quad (m+2)(n-2)$$
là các số lập phương.
\end{btt}

\begin{btt}
Giải hệ phương trình nghiệm tự nhiên
$$\heva{&2\left({x}^{2}+{y}^{2}-3 {x}+2 {y}\right)-1=z^2 \\ &5\left({x}^{2}+{y}^{2}+4 {x}+2 {y}+3\right)=t^2.}$$
\nguon{Chuyên Toán Nam Định 2020}
\end{btt}

\begin{btt}
Giải phương trình nghiệm nguyên $$x^2 + y^2 = 6\tron{z^2 + t^2}.$$
\end{btt}

\begin{btt}
Giải phương trình nghiệm nguyên
$$x^2+y^2+z^2=x^2y^2.$$
\end{btt}

\begin{btt} 
Giải phương trình nghiệm nguyên
$$x^2+y^2+z^2=2xyz.$$
\end{btt}

\begin{btt}
  Với $k$ là số nguyên dương, chứng minh rằng không tồn tại các số nguyên $a, b, c$ khác 0 sao cho \[a+b+c=0,\quad ab+bc+ca+{{2}^{k}}=0.\]
\nguon{Chuyên Toán Phổ thông Năng khiếu}
\end{btt}

\begin{btt}
Tìm các số nguyên không âm $a, b, c, d, n$ thỏa mãn
\[a^{2}+b^{2}+c^{2}+d^{2}=7 \cdot 4^{n}.\]
\end{btt}

\begin{btt}
Giải phương trình nghiệm nguyên
\[x^2(x+y)=y^2(x-y)^2.\]
\end{btt}

\begin{btt}
Giải phương trình nghiệm nguyên
\[2x^3=y^3\tron{3x+y+2}.\]
\end{btt}

\begin{btt}
Tìm tất cả các số nguyên dương $a,b$ thỏa mãn $$a^3+b^3=a^2+6ab+b^2.$$
\end{btt}

\begin{btt}
Tìm tất cả các bộ số nguyên $(m, n)$ thỏa mãn phương trình sau
\[m^5-n^5=16mn.\]
\nguon{Junior Balkan Mathematical Olympiad}
\end{btt}

\begin{btt}
Giải phương trình nghiệm nguyên dương
    $$x^{6}-y^{6}=2016 x y^{2}.$$
\nguon{Adrian Andreescu}
\end{btt}

\begin{btt}
Tìm tất cả các số nguyên $x,y$ thỏa mãn $$54x^3-1=y^3.$$
\end{btt}

\begin{btt}
Tìm các số nguyên $x,y$ thỏa mãn $$x^4- 2y^2 = 1.$$
\end{btt}

\begin{btt}
Giải phương trình nghiệm nguyên
    $$2x^2-y^{14}=1.$$
\nguon{Nairi Sedrakyan}
\end{btt}

%nguyệt anh
\begin{btt}
Giải phương trình nghiệm nguyên \[x^4+4x^3+8x^2+8x+3=y^3.\]
\end{btt}

\begin{btt}
Giải phương trình nghiệm tự nhiên
\[x^4+4x^3+7x^2+6x+3=y^3.\]
\end{btt}

\begin{btt}
Tìm tất cả các số nguyên $x,y$ và số nguyên tố $p$ thỏa mãn \[\dfrac{x^4+x^2+1}{p}=y^4.\]
\end{btt}

\begin{btt}
Giải phương trình nghiệm nguyên $$9^x=2y^2+1.$$
\end{btt}

\begin{btt}
Tìm các số nguyên dương $x,y,z$ thoả mãn điều kiện 
$$x^3-y^3=z^2,$$
trong đó $y$ là số nguyên tố và $(z,3)=(x,y)=1.$
\end{btt}

\begin{btt}
Tìm tất cả các số nguyên tố $p$ và hai số nguyên dương $a, b$ sao cho $p^a+p^b$ là số chính phương.
\nguon{Tạp chí Toán học và Tuổi trẻ số 507, tháng 9 năm 2019}
\end{btt}

\subsection*{Hướng dẫn bài tập tự luyện}

\begin{gbtt} Giải phương trình nghiệm nguyên $x^2+y^2=3z^2.$
\loigiai{
Ta nhận thấy phương trình có nghiệm $(0,0,0).$\\ Nếu như $x\ne 0,$ ba số $x,y,z$ tồn tại ước chung, gọi là $d.$ Ta đặt
$$x=dx_1,y=dy_1,z=dz_1.$$
Một cách hiển nhiên, $(x_1,y_1,z_1)=1.$ Ngoài ra, do $d\ne 0,$ phương trình đã cho trở thành
\[x^2_1+y^2_1=3z^2_1.\tag{*}\label{lvh3}\]
Ta đã biết, với mọi số nguyên $a,b,$ nếu $a^2+b^2$ chia hết cho $3$ thì $a$ và $b$ cũng chia hết cho $3.$ Theo đó, cả $x_1$ và $y_1$ đều chia hết cho $3.$ Thực hiện đặt $x_1=3x_2$ và $y_1=3y_2$ rồi thế vào (\ref{lvh3}), ta được
$$9x^2_2+9y^2_2=3z^2_1\Leftrightarrow 3x^2_2+3y^2_2=z^2_1.$$
Rõ ràng, $z_1$ chia hết cho $3.$ Cả $x_1,y_1$ và $z_1$ đều chia hết cho $3,$ chứng tỏ $(x_1,y_1,z_1)\ge 3,$ mâu thuẫn với điều kiện $(x_1,y_1,z_1)=1.$ Trường hợp $x,y,z$ không đồng thời bằng $0$ không xảy ra. Nghiệm nguyên duy nhất của phương trình là $(0,0,0).$}
\end{gbtt}

\begin{gbtt}
Tìm tất cả các số nguyên $x,y,z$ thỏa mãn $4x^2+4xy+3y^2=5z^2.$
\loigiai{Ta nhận thấy $x=y=z=0$ thỏa mãn đề bài. \\ 
Đối với trường hợp $x,y,z$ không đồng thời bằng $0$, ta đặt $d=(x,y,z),$ khi đó tồn tại các số nguyên $m,n,p$ sao cho $(m,n,p)=1$ và $x=dm,y=dn,z=dp.$ Phép đặt này cho ta 
$$4\left(dm\right)^2+4dm\cdot dn+3\left(dn\right)^2=5(dp)^2.$$
Chia cả hai vế cho $d^2,$ ta được
$$4m^2+4mn+3n^2=5p^2\Leftrightarrow (2m+n)^2+2n^2=5p^2.$$
Ta đã biết, một số chính phương chỉ có thể đồng dư $0,1,4$ theo modulo $5.$ Do $$(2m+n)^2+2n^2\equiv 0 \pmod{5},$$ ta xét bảng đồng dư theo modulo $5$ sau
        \begin{center}
            \begin{tabular}{c|c|c|c}
            $(2m+n)^2$ & $0$ & $1$ & $4$\\
            \hline
            $2n^2$ & $0$ & $4$ & $1$\\
            \hline
            $4n^2$& $0$ & $3$ & $2$
            \end{tabular}
        \end{center}
Một số chính phương không thể đồng dư $2$ hoặc $3$ theo modulo $5,$ thế nên đối chiếu với bảng, ta được $n^2\equiv 0 \pmod{5},$ hay $n^2$ chia hết cho $5.$ Ta lần lượt suy ra
$$\heva{&5\mid n^2 \\ &5\mid (2m+n)^2}\Rightarrow \heva{&5\mid n \\ &5\mid (2m+n)}\Rightarrow \heva{&25\mid n^2 \\ &25\mid (2m+n)^2}\Rightarrow 25\mid 5p^2 \Rightarrow 5\mid p^2 \Rightarrow 5\mid p.$$
Cả $m,n,p$ đều chia hết cho $5,$ chứng tỏ $(m,n,p)\ge 5,$ mâu thuẫn với điều kiện phép đặt. Trường hợp $x,y,z$ không đồng thời bằng $0$ không xảy ra. Nghiệm nguyên duy nhất của phương trình là $(0,0,0).$}
\end{gbtt}


\begin{gbtt}
Tìm tất cả các cặp số nguyên $(m,n)$ sao cho $$6(m+1)(n-1), \quad (m-1)(n+1)+6,\quad (m+2)(n-2)$$
là các số lập phương.
\loigiai{
Đặt $a^3=6(m+1)(n-1), b^3=(m-1)(n+1)+6$ và $c^3=(m+2)(n-2).$ Ta dễ dàng chỉ ra
$$a^3=2b^3+4c^3.$$
Áp dụng kết quả của \chu{ví dụ \ref{baimodaugcd}}, ta tìm ra $a=b=c=0.$ Các cặp $(m,n)$ thỏa mãn yêu cầu bài toán là
$$(m,n)=(-1,2),\quad (m,n)=(-2,1).$$
}
\end{gbtt}

\begin{gbtt}
Giải hệ phương trình nghiệm tự nhiên
$$\heva{&2\left({x}^{2}+{y}^{2}-3 {x}+2 {y}\right)-1=z^2 \\ &5\left({x}^{2}+{y}^{2}+4 {x}+2 {y}+3\right)=t^2.}$$
\nguon{Chuyên Toán Nam Định 2020}
\loigiai{
Cộng theo vế hai phương trình trong hệ đã cho, ta nhận được
$$7\tron{x+1}^2+7\tron{y+1}^2=z^2+t^2.$$
Ta nhận thấy $x=y=-1,z=t=0$ không thỏa mãn. Trong trường hợp $(x,y,z,t)\ne (-1,-1,0,0)$ ta đặt $\tron{x+1,y+1,z,t}=d.$ Lúc này, tồn tại các số tự nhiên $x_1,y_1,z_1,t_1$ nguyên tố cùng nhau sao cho
$$7x_1^2+7y_1^2=z_1^2+t_1^2.$$
Lấy đồng dư modulo $7$ hai vế, ta được
$z_1^2+t_1^2\equiv0\pmod{7}.$ Ta đã biết 
$$z_1^2\equiv 0,1,2,4\pmod{4}.$$
Dựa trên chứng minh này, ta lập được bảng đồng dư sau.
\begin{center}
    \begin{tabular}{c|c|c|c|c}
        $z_1^2$ &  $0$ & $1$ & $2$ & $4$ \\
        \hline
        $t_1^2$ &  $0$ & $6$ & $5$&$3$
    \end{tabular}
\end{center}
Theo như bảng đồng dư, chỉ có trường hợp $z_1^2\equiv t_1^2\equiv 0\pmod{7}$ là thỏa mãn. Ta lần lượt suy ra
\begin{align*}
    z_1\equiv t_1\equiv 0\pmod{7}&\Rightarrow z_1^2\equiv t_1^2\equiv 0\pmod{49}\\&\Rightarrow z_1^2+t_1^2\equiv 0\pmod{49}\\&\Rightarrow 7x_1^2+7y_1^2\equiv 0\pmod{49}\\&\Rightarrow x_1^2+y_1^2\equiv 0\pmod{7}.
\end{align*}
Lập luận tương tự, ta có cả $x_1$ và $y_1$ chia hết cho $7,$ thế nên là 
$$\tron{x_1,y_1,z_1,t_1}\ge 7.$$
Điều này mâu thuẫn với điều kiện $\tron{x_1,y_1,z,t}=1.$ Giả sử là sai. Hệ đã cho không có nghiệm nguyên.}
\end{gbtt}

\begin{gbtt}
Giải phương trình nghiệm nguyên 
\[x^2 + y^2 = 6\tron{z^2 + t^2}.\]
\loigiai{
Ta nhận thấy phương trình có nghiệm $(0,0,0,0).$ Nếu như $x\ne 0,$ bốn số $x,y,z,t$ tồn tại ước chung $d.$ Đặt
$$x=dx_1,y=dy_1,z=dz_1,t=dt_1.$$
Một cách hiển nhiên, $(x_1,y_1,z_1,t_1)=1.$ Ngoài ra, do $d\ne 0,$ phương trình đã cho trở thành
\[x^2_1+y^2_1=6z^2_1+6t^2_1.\tag{1}\label{lvh4}\]
Dựa vào (\ref{lvh4}), ta nhận ra $x_1^2 + y_1^2$ chia hết cho $3,$ thế nên cả $x_1$ và $y_1$ đều chia hết cho $3.$ Ta đặt $x_1 = 3x_2$, $y_1 = 3y_2$ với $y_2,y_2$ là các số nguyên. Thế vào (\ref{lvh4}) rồi chia hai vế phương trình cho $3$, ta được
\[9x^2_2+9y^2_2=6z^2_1+6t^2_1\Leftrightarrow 3x^2_2+3y^2_2=2z^2_1+2t^2_1.\tag{2}\label{lvh5}\]
Do $(2,3)=1$ nên dựa vào (\ref{lvh5}), ta nhận ra $z_1^2 + t_1^2$ chia hết cho $3,$ thế nên cả $z_1$ và $t_1$ đều chia hết cho $3.$ Bốn số $x_1,y_1,z_1$ và $t_1$ đều chia hết cho $3,$ chứng tỏ $(x_1,y_1,z_1,t_1)\ge 3,$ mâu thuẫn với điều kiện $(x_1,y_1,z_1,t_1)=1.$ \\
Tổng kết lại, phương trình có nghiệm nguyên duy nhất là $(x,y,z,t)=(0,0,0,0).$}
\end{gbtt}

\begin{gbtt}
Giải phương trình nghiệm nguyên
\[x^2+y^2+z^2=x^2y^2.\]
\loigiai{
Ta nhận thấy $(x,y,z)=(0,0,0)$ là một nghiệm phương trình. \\Ngược lại, nếu một trong ba số $x,y,z$ khác $0,$ ta đặt
$$(x,y,z)=d,\quad x=dm,\quad y=dn,\quad z=dp,$$
trong đó $(m,n,p)=1.$ Do $d\ne 0$ nên phương trình đã cho trở thành
\[m^2+n^2+p^2=d^2m^2n^2.\tag{*}\label{luivohan123}\]
Tới đây, ta xét các trường hợp sau.
\begin{enumerate}
    \item Nếu cả ba số $m,n,p$ đều chẵn thì $(m,n,p)>2,$ mâu thuẫn.
    \item Nếu $m$ và $n$ cùng chẵn, còn $p$ lẻ thì vế trái của (\ref{luivohan123}) lẻ, trong khi vế phải chẵn, mâu thuẫn.
    \item Nếu $m$ và $n$ khác tính chẵn lẻ, ta nhận thấy
    $$m^2+n^2+p^2\equiv \heva{1\pmod{4},&\text{ với }p\text{ chẵn} \\ 2\pmod{4},&\text{ với }p\text{ lẻ}.}$$
    còn $d^2m^2n^2$ chia hết cho $4,$ mâu thuẫn.
    \end{enumerate}    
Tổng kết lại, phương trình có nghiệm nguyên duy nhất là $(x,y,z)=(0,0,0).$}
\end{gbtt}

\begin{gbtt} Giải phương trình nghiệm nguyên
\[x^2 + y^2 + z^2 = 2xyz.\]
\loigiai{
Nếu như phương trình có nghiệm, $x^2 + y^2 + z^2$ phải là số chẵn. Theo đó, trong ba số $x,y,z,$ hoặc có đúng $1$ số chẵn, hoặc cả $3$ số cùng chẵn. Ta xét các trường hợp kể trên.
\begin{enumerate}
    \item Nếu trong ba số $x$, $y$, $z$ có một số chẵn, hai số lẻ, không mất tổng quát, ta giả sử $x$ chẵn, $y$ và $z$ lẻ. Theo như kiến thức đã học, ta có
    $$x^2+y^2+z^2\equiv 0+1+1\equiv 2\pmod{4}.$$
    Tuy nhiên, bởi vì $x$ chẵn nên $2xyz$ chia hết cho $4,$ mâu thuẫn.
    \item Nếu cả ba số $x,y,z$ đều chẵn, ta nhận thấy $(0,0,0)$ thỏa mãn đề bài. Ngược lại, nếu trong $x,y,z$ có một số khác $0,$ ta đặt $x = 2dx_1$, $y = 2dy_1$, $z = 2dz_1,$ trong đó
    $$d=\left(\dfrac{x}{2},\dfrac{y}{2},\dfrac{z}{2}\right).$$
    Ta nhận thấy $(x_1,y_1,z_1)=1.$ Ngoài ra, phép đặt trên cho ta
    $$x_1^2 + y_1^2 + z_1^2 = 4dx_1y_1z_1.$$
    Lập luận tương tự các bài toán trước, ta có cả ba số $x_1,y_1$ và $z_1$ đều chẵn, mâu thuẫn với điều kiện $(x_1,y_1,z_1)=1.$	Trường hợp này không xảy ra.
    \end{enumerate}    
Tổng kết lại, phương trình có nghiệm nguyên duy nhất là $(x,y,z)=(0,0,0).$}
\end{gbtt}

\begin{gbtt}
    Với $k$ là số nguyên dương, chứng minh rằng không tồn tại các số nguyên $a, b, c$ khác 0 sao cho \[a+b+c=0,\quad ab+bc+ca+{{2}^{k}}=0.\]
\nguon{Chuyên Toán Phổ thông Năng khiếu}
 \loigiai{
Đặt $d=\tron{a,b,c}$ và $a=dx,b=dy$ và $c=dz$ trong đó $\tron{x,y,z}=1.$ Thế trở lại giả thiết cho ta
\begin{align}
    d\tron{x+y+z}&=0, \tag{1}\label{ptnk16.1}\\
    d^2\tron{xy+yz+zx}&=-2^k.\tag{2}\label{ptnk16.2}
\end{align}
Từ (\ref{ptnk16.1}) ta suy ra $z=-x-y.$ Thế vào (\ref{ptnk16.2}) ta được
$$d^2\tron{x^2+xy+y^2}=2^k.$$
Cũng vì $z=-x-y$ nên từ $(x,y,z)=1$ ta có $(x,y)=1.$ Ta xét các trường hợp sau.
\begin{enumerate}
    \item Với $x,y$ không cùng tính chẵn lẻ hoặc cùng lẻ, ta có $x^2+xy+y^2$ là số lẻ và là ước dương của $2^k.$\\ Do đó $x^2+xy+y^2=1.$ Biến đổi tương đương cho ta $$\tron{2x+y}^2+3y^2=4.$$
    Vì $x,y$ là các số nguyên nên ta suy ra $3y^2=3$ và $\tron{2x+y}^2=1.$ Từ đây, ta có bảng sau.
    \begin{center}
        \begin{tabular}{c|c|c|c|c}
        $y$ &  $1$ & $-1$ &  $1$ & $-1$\\
        \hline
        $2x+y$ & $1$ &$-1$& $-1$ &$1$\\
        \hline
        $x$ & $0$& $0$ & $-1$ & $1$ \\
        \hline
        $z=-x-y$ & $-1$ & $1$ & $0$ & $0$
    \end{tabular}
    \end{center}
Các giá trị của $x,y,z$ vừa thu được mâu thuẫn với điều kiện $xyz\ne 0.$
    \item Với $x,y$ cùng là số chẵn, ta có $\tron{x,y}\ge 2,$ mâu thuẫn.
\end{enumerate}
Bài toán được chứng minh hoàn tất.}
\end{gbtt}

\begin{gbtt}
Tìm các số nguyên không âm $a, b, c, d, n$ thỏa mãn
\[a^{2}+b^{2}+c^{2}+d^{2}=7 \cdot 4^{n}.\]
\loigiai{
Với $n=0,$ ta tìm ra $(a,b,c,d)$ là các hoán vị của $(2,1,1,1).$ 
Với $n\ge 1,$ ta có $a^{2}+b^{2}+c^{2}+d^{2}$ chia hết cho $4.$ Điều này dẫn tới $a,b,c,d$ cùng tính chẵn lẻ. Ta xét các trường hợp sau.
\begin{enumerate}
    \item Nếu $a,b,c,d$ là số lẻ, ta đặt $$a=2a'+1,b=2b+1,c=2c'+1, d=2d'+1.$$
    Thế trở lại phương trình ban đầu, ta có
    $$4a'\tron{a'+1}+4b'\tron{b'+1}+4c'\tron{c'+1}+4d'\tron{d'+1}=4\tron{7\cdot 4^{n-1}-1}.$$
    Dễ dàng chỉ ra $VT$ chia hết cho $8$ nên ta suy ra $7\cdot4^{n-1}-1$ chia hết cho $2.$ Từ đây ta có $n=1.$ Thế $n=1$ trở lại phương trình ban đầu, ta được
    $$a^2+b^2+c^2+d^2=28.$$
    Phương trình này có hai nghiệm nguyên dương không kể thứ tự là 
    $$(3,3,3,1),\quad (1,1,1,5).$$
    \item Nếu $a,b,c,d$ là số chẵn, ta đặt $\tron{a,b,c,d}=x.$ Khi đó tồn tại các số tự nhiên $a_1,b_1,c_1,d_1$ sao cho $$a=xa_1,\:b=xb_1,\:c=xc_1,\:d=xd_1,\:\tron{a,b,c,d}=1.$$ 
    Phương trình đã cho trở thành
$$x^2\tron{a^2_1+b^2_1+c^2_1+d^2_1}=7\cdot4^n.$$
    Tới đây, ta xét các trường hợp sau.
        \begin{itemize}
            \item\chu{Trường hợp 1.} Nếu trong $a_1,b_1,c_1,d_1$ có $3$ số lẻ và $1$ số chẵn (giả sử là $d_1$), ta có
            $$a_1^2+b_1^2+c_1^2+d_1^2\equiv 3,7\pmod{8}.$$
           Như vậy $a_1^2+b_1^2+c_1^2+d_1^2$ phải là ước của $7.$ Ta suy ra 
           $$\tron{a_1,b_1,c_1,d_1}=\tron{1,1,1,2}\text{ và }x^2=4^n.$$ 
           Bộ số chưa kể thứ tự thu được trong trường hợp này là
            $$\tron{a_1,b_1,c_1,d_1}=\tron{2^n,2^n,2^n,2^{n+1}}.$$
            \item\chu{Trường hợp 2.} Nếu trong $a_1,b_1,c_1,d_1$ có $3$ số chẵn và $1$ số lẻ, ta có 
            $$a_1^2+b_1^2+c_1^2+d_1^2\equiv 3,7\pmod{8}.$$
            Tới đây, ta lập luận tương tự trường hợp trước để chỉ ra sự không thoả mãn.
            \item\chu{Trường hợp 3.} Nếu $a_1,b_1,c_1,d_1$ đều lẻ, ta sẽ có
            $$a_1^2+b_1^2+c_1^2+d_1^2\equiv 4\pmod{8}.$$
            Từ đó $4^n\equiv 4\pmod{8}.$ Do $x$ là luỹ thừa của $2$ nên từ phương trình ta có $x=2^{n-1},$ đồng thời
            $$a_1^2+b_1^2+c_1^2+d_1^2=28.$$
        Phương trình này có hai nghiệm nguyên dương không kể thứ tự là 
    $$(3,3,3,1),\quad (1,1,1,5).$$
          \end{itemize} 
\end{enumerate}
Tổng kết lại, các bộ số nguyên không âm $(a,b,c,d)$ thỏa mãn là $$\tron{2^n,2^n,2^n,2^{n+1}},\:\tron{3\cdot2^n,3\cdot2^n,3\cdot2^n,2^n},\:\tron{2^n,2^n,2^n,5\cdot2^n}$$ và toàn bộ các hoán vị của chúng.
}
\end{gbtt}
\begin{gbtt}
Giải phương trình nghiệm nguyên
\[x^2(x+y)=y^2(x-y)^2.\]
\loigiai{
Đầu tiên, ta quan sát thấy $(x,y)=(0,0)$ là một nghiệm của phương trình. Với ít nhất một trong hai số $x,y$ khác $0$, ta đặt $(x,y)=d$, khi đó tồn tại hai số $a,b$ nguyên thỏa mãn
$$(a,b)=1,\quad x=da,\quad y=db.$$ 
Phương trình đã cho trở thành
\[(da)^2(da+db)=(db)^2(da-db)^2\Leftrightarrow a^2(a+b)=db^2(a-b)^2.\tag{*}\label{stolecualam}\]
Ta suy ra $a^2(a+b)$ chia hết cho $b^2,$ nhưng do $(a,b)=1$ nên $a+b$ chia hết cho $b^2,$ và $a$ chia hết cho $b.$ Lại do $(a,b)=1$ nên $b=1$ hoặc $b=-1.$ Ta xét các trường hợp kể trên.
\begin{enumerate}
    \item Với $b=1,$ thế vào phương trình (\ref{stolecualam}) ta được
    $$a^2(a+1)=d(a-1)^2.$$
    Ta suy ra $a^2(a+1)$ chia hết cho $a-1,$ và thế thì $a\in \{-1;0;2;3\}.$ Lần lượt thế trở lại rồi kiểm tra trực tiếp, ta tìm được $(x,y)=(27,9)$ và $(x,y)=(24,12).$
    \item Với $b=-1,$ thế vào phương trình (\ref{stolecualam}) ta được
    $$a^2(a-1)=d(a+1)^2.$$    
    Ta suy ra $a^2(a-1)$ chia hết cho $a+1,$ và thế thì $a\in \{1;0;-2;-3\}.$ Lần lượt thế trở lại rồi kiểm tra trực tiếp, ta không tìm được cặp $(x,y)$ nào thỏa mãn.
\end{enumerate}
Kết luận, phương trình đã cho có hai nghiệm nguyên $(x,y)$ là $(27,9)$ và $(24,12).$}
\end{gbtt}

\begin{gbtt}
Giải phương trình nghiệm nguyên
\[2x^3=y^3\tron{3x+y+2}.\]
\loigiai{
Đầu tiên, ta quan sát thấy $(x,y)=(0,0)$ là một nghiệm của phương trình. Với ít nhất một trong hai số $x,y$ khác $0$, ta đặt $(x,y)=d$, khi đó tồn tại hai số $a,b$ nguyên thỏa mãn
$$(a,b)=1,\quad x=da,\quad y=db.$$ 
Phương trình đã cho trở thành
\[2d^3a^3=d^3b^3\tron{3da+db+2}\Leftrightarrow 2a^3=b^3\tron{3da+db+2}.\tag{*}\label{baidatche}\]
Vì $\tron{a,b}=1$ nên $b^3\mid 2$. Từ đây, ta suy ra $b=\pm1.$ Ta xét $2$ trường hợp sau.
\begin{enumerate}
    \item Với $b=1$, thế vào phương trình (\ref{baidatche}) ta được
    $$2\tron{a^3-1}=d\tron{3a+1}.$$
Ta nhận thấy $(3a+1)\mid 2\tron{a^3-1}.$ Ta dễ dàng tìm ra $3a+1$ là ước của $56.$ Lần lượt thử từng trường hợp, ta thu được $5$ cặp $(a,d)$ tương ứng là
$$(0,-2),(-1,2),(-3,7),(-5,18),(-19,245).$$
Thế trở lại, ta được các bộ số $(x,y)$ là $$(0,-2), (-2,2), (-21,7), (-90,18),(-4655,245).$$
\item Với $b=-1$, thế vào phương trình (\ref{baidatche}) ta được
$$2\tron{a^3+1}=-d(3a-1).$$
Tính tương tự \chu{trường hợp 1}, ta tìm được $5$ bộ số nguyên $(x,y)$ thỏa mãn là
$$(-2,2),\ (0,-2), \ (0,0),\ (4,2),\ (468,52).$$
\end{enumerate}
Như vậy, phương trình đã cho có $8$ nghiệm nguyên $(x,y)$ thỏa mãn là 
$$(0,0),\ (0,-2),\ (-2,2), \ (-21,7),\ (-90,18),\ (-4655,245),\ (4,2),\ (468,52).$$}
\end{gbtt}

\begin{gbtt}
Tìm tất cả các số nguyên dương $a,b$ thỏa mãn $a^3+b^3=a^2+6ab+b^2.$
\loigiai{
Đầu tiên, ta quan sát thấy bộ $(a,b)=(0,0)$ thỏa mãn đề bài bài toán.\\
Với ít nhất một trong hai số $a,b$ khác $0,$ ta đặt $d=(a,b),$ khi đó tồn tại hai số $x,y$ nguyên dương thỏa mãn $(x,y)=1$ sao cho $a=dx,b=dy.$ Phép đặt này cho ta
\[d^3\left(x^3+y^3\right)=d^3x^2+6d^2xy+d^2y^2.\tag{*}\label{datd.lechbac.1}\]
Do $d^2\ne 0$ nên một cách tương đương, ta có
$$d(x+y)\left(x^2-xy+y^2\right)=x^2+6xy+y^2.$$
Ta suy ra $x^2+6xy+y^2$ chia hết cho $x^2-xy+y^2,$ thế nên $7xy$ cũng chia hết cho $x^2-xy+y^2.$ Áp dụng kết quả $\left(xy,x^2-xy+y^2\right)=1$ quen thuộc, ta suy ra $x^2-xy+y^2$ là ước của $7.$ Tới đây, ta xét hai trường hợp sau.
\begin{enumerate}
    \item Với $x^2-xy+y^2=1,$ ta biến đổi phương trình trên tương đương về thành 
    $$\left(x-\dfrac{y}{2}\right)^2+\dfrac{3y^2}{4}=1.$$
    Dựa theo chú ý $\dfrac{3y^2}{4}\le 1,$ ta có $y=0$ hoặc $y=1.$ 
    \begin{itemize}
        \item \chu{Trường hợp 1. }Với $y=0,$ ta có $x=1.$ Thế vào (\ref{datd.lechbac.1}), ta được $d=1,$ và khi đó $(a,b)=(1,0).$
        \item \chu{Trường hợp 2. }Với $y=0,$ ta có $x=1.$ Bằng lập luận tương tự, ta chỉ ra $(a,b)=(0,1).$
    \end{itemize}
    \item Với $x^2-xy+y^2=7,$ bằng cách chỉ ra $\dfrac{3y^2}{4}\le 7$ tương tự, ta có $y\in \{1;2;3\}.$
    \begin{itemize}
        \item \chu{Trường hợp 1.} Với $y=1,$ ta có $x=3.$ Thế vào (\ref{datd.lechbac.1}), ta được $d=1,$ và khi đó $(a,b)=(3,1).$
        \item \chu{Trường hợp 2.} Với $y=3,$ ta có $x=1.$ Bằng lập luận tương tự, ta chỉ ra $(a,b)=(1,3).$ 
        \item \chu{Trường hợp 3. }Với $y=2,$ ta không tìm được $x$ thỏa mãn.
    \end{itemize}
\end{enumerate} 
Kết luận, có tất cả $4$ bộ $(a,b)$ thỏa mãn đề bài, đó là $(0,1),(1,0),(1,3)$ và $(3,1).$}
\end{gbtt}

\begin{gbtt}
Tìm tất cả các bộ số nguyên $(m, n)$ thỏa mãn phương trình sau
\[m^5-n^5=16mn.\]
\nguon{Junior Balkan Mathematical Olympiad}
\loigiai{
Giả sử tồn tại các số nguyên $m,n$ thỏa phương trình. Trong bài toán này, ta xét các trường hợp sau đây.
    \begin{enumerate}
        \item Nếu một trong hai giá trị \(m,n\) bằng \(0\) thì giá trị còn lại cũng bằng \(0\), khi đó bộ $(m, n)=(0,0)$ là một nghiệm của bài toán.
        \item Nếu $m n \neq 0$, ta đặt $d=(m, n),m=d a, n=d b,$ trong đó $(a, b)=1$. Thế trở lại, ta được
        \[d^{3} a^{5}-d^{3} b^{5}=16ab.\]
        Vì thế, từ phương trình trên ta thu được $a\mid d^{3} b^{5}$ và do đó $a\mid d^{3}.$ Tương tự thì ta cũng có $b\mid d^{3},$ mà do $(a, b)=1$ nên ta thu được $ab\mid d^{3}.$ Đặt $d^{3}=a b r.$ Thay vào phương trình ban đầu ta thu được
        \[a b r a^{5}-a b r b^{5}=16ab \Rightarrow r\left(a^{5}-b^{5}\right)=16.\]
        Do đó ta phải có \(a^5-b^5\) là ước của \(16\), nghĩa là
        \[a^{5}-b^{5}=\left \{ \pm 1;\pm 2; \pm 4; \pm 8; \pm 16 \right \}.\]
        Giá trị nhỏ nhất của $\left|a^{5}-b^{5}\right|$ là \(1\) hoặc \(2\). Thật vậy, ta xem xét các trường hợp
        \begin{itemize}
            \item\chu{Trường hợp 1.} Nếu $\left|a^{5}-b^{5}\right|=1$ thì $a=\pm 1$ và $b=0$ hoặc $a=0$ và $b=\pm 1,$ mâu thuẫn.  
            \item\chu{Trường hợp 2.} Nếu $\left|a^{5}-b^{5}\right|=2$ thì $a=1$ và $b=-1$ hoặc $a=-1$ và $b=1$. Khi đó, $r=-8$ và $d^{3}=-8$ hay \(d=-2\), kéo theo $(m, n)=(-2,2)$.
            \item\chu{Trường hợp 3.} Nếu $\left|a^{5}-b^{5}\right|>2,$ không mất tổng quát, giả sử $a>b.$ Nếu như $b\ne \pm1$ thì
            $$\left|a^{5}-b^{5}\right|\ge (b+1)^{5}-b^{5}=\left|5 b^{4}+10b^{3}+10b^{2}+5 b+1\right| \geq 31,$$
            mâu thuẫn. Do đó $b=\pm 1.$ Thử lại, ta không tìm được $a.$
        \end{itemize}
    \end{enumerate}
  Do đó tất cả các bộ số nguyên thỏa mãn đề bài là $(m, n)=(0,0),(-2,2)$.}
\end{gbtt}

\begin{gbtt}
Giải phương trình nghiệm nguyên dương
    $$x^{6}-y^{6}=2016 x y^{2}.$$
\nguon{Adrian Andreescu}
\loigiai{
Trước tiên ta đặt $(x, y)=d,$ khi đó tồn tại các số nguyên dương $u,v$ nguyên tố cùng nhau sao cho $x=k u$ và $y=k v.$ Thế trở lại phương trình rồi rút gọn, ta được
    $$ k^{3}\left(u^{6}-v^{6}\right)=2016 u v^{2}.$$
Lấy đồng dư modulo $u$ hai vế của phương trình kết hợp điều kiện $(u,v)=1,$ ta có
    $$-k^{3} v^{6} \equiv 0 \pmod{u}\Rightarrow u\mid k^3v^6\Rightarrow \Rightarrow u \mid k^{3}.$$
Lấy đồng dư modulo $v^2$ hai vế của phương trình kết hợp điều kiện $(u,v)=1,$ ta có
    $$-k^{3} u^{6} \equiv 0 \pmod{v^2}\Rightarrow v^2\mid k^3u^6 \Rightarrow v^{2} \mid k^{3}.$$
Với việc $\left(u, v^{2}\right)=1$, ta chỉ ra $u v^{2} \mid k^{3}$. Đối chiếu lại phương trình, ta nhận thấy $u^6-v^6$ là ước của $2016.$ Nhờ chú ý thêm $u>v,$ ta xét các trường hợp sau.
    \begin{enumerate}
        \item Nếu $u\geq 4,$ ta có $u^{6}-v^{6} \geq 4^{6}-3^{6}=3367>2016,$ mâu thuẫn do $\tron{u^6-v^6}\mid 2016.$
        \item Nếu $1\leq v<u\leq 3,$ ta chỉ cần kiểm tra điều kiện bài toán với $(u,v)=(3,1),\ (3,2),\ (2,1).$
        \begin{itemize}
            \item\chu{Trường hợp 1.} Với $\left ( u,v \right )=\left ( 3,1 \right),$ ta có $u^6-v^6=3^6-1=728$ không là ước của $2016.$
            \item\chu{Trường hợp 2.} Với $\left ( u,v \right )=\left ( 3,2 \right ),$ ta có $u^6-v^6=3^6-2^6=665$ không là ước của $2016.$
            \item\chu{Trường hợp 3.} Với $\left ( u,v \right )=\left ( 2,1 \right ),$ ta có $u^6-v^6=2^6-1=63$ là ước của $2016$ và khi ấy $$k^3=\dfrac{2016}{63}\cdot 2\cdot 1=64\Rightarrow k=4.$$
        \end{itemize}
    \end{enumerate}
Như vậy, phương trình đã cho có duy nhất một nghiệm nguyên dương là $(x,y)=(8,4).$}
\end{gbtt}

\begin{gbtt}
Tìm tất cả các số nguyên $x,y$ thỏa mãn $54x^3-1=y^3.$
\loigiai{
Xét phép biến đổi hệ quả của phương trình đã cho
\begin{align*}
    216{x^3}\left( {54{x^3} - 1} \right) = 216{x^3}{y^3} 
    &\Rightarrow {\left( {6{x^3} - 1} \right)^2} = {\left( {6xy} \right)^3} + 1
    \\&\Rightarrow 
    {\left( {6{x^3} - 1} \right)^2} = (6xy+1)\left(36x^2y^2-6xy+1\right).
\end{align*}
Đặt $z = 18{x^3},t = 6xy,$ phương trình trở thành
$${\left(6z-1\right)^2} = (t+1)\left(t^2-t+1\right).$$
Tương tự như bài vừa rồi, ta chỉ ra $\left(t^2-t+1,t+1\right)\in\{1;3\}.$ Ta xét các trường hợp kể trên.
\begin{enumerate}
    \item Với $d=1,$ theo bổ đề, ta chỉ ra $t+1$ và $t^2-t+1$ đều là các số chính phương. Tương tự bài trước, ta có $t=0$ hoặc $t=1.$ Thử trực tiếp, ta tìm ra $(x,y)=(0,-1).$
    \item Với $d=3,$ ta được $3\mid (t+1)$ và $3\mid \left(t^2-t+1\right).$ Tương tự bài trước, ta nhận được mâu thuẫn.
\end{enumerate}
Kết luận, phương trình có nghiệm nguyên duy nhất là $(x,y)=(0,-1).$}
\end{gbtt}

\begin{gbtt}
Tìm các số nguyên $x,y$ thỏa mãn 
\[x^4- 2y^2 = 1.\]
\loigiai{
Không mất tính tổng quát, ta giả sử  $x,y \geqslant 0$ .\\
Rõ ràng $x$ là số lẻ. Đặt $x=2k+1,$ và phép đặt này cho ta
$$\left( {4{k^2} + 4k} \right)\left( {4{k^2} + 4k + 2} \right) = 2{y^2} \Leftrightarrow 4\left( {{k^2} + k} \right)\left( {2{k^2} + 2k + 1} \right) = {y^2}.$$
Với chú ý $\left( {{k^2} + k,2{k^2} + 2k + 1} \right) = 1$ và $2{k^2} + 2k + 1\ge 0,$ ta  suy ra $${k^2} + k,\quad 2{k^2} + 2k + 1$$ là hai số chính phương. Đặt $k^2+k=m^2, $ với $m$ là số nguyên dương. Ta có
$${k^2} + k = {m^2} \Leftrightarrow 4{k^2} + 4k + 1 = 4{m^2} + 1 \Leftrightarrow \left( {2k + 1 - 2m} \right)\left( {2k + 1 + 2m} \right) = 1.$$ 
Biến đổi trên cho ta $2k+1-2m=2k+1+2m=1,$ thế nên $m=k=0.$\\ Ta tìm ra $(x,y)=(1,0)$ và $(x,y)=(-1,0)$ là các cặp số thỏa mãn đề bài.}
\end{gbtt}

\begin{gbtt}
Giải phương trình nghiệm nguyên
    $$2x^2-y^{14}=1.$$
\nguon{Nairi Sedrakyan}
\loigiai{
Phương trình đã cho tương đương với
    $$x^2=\tron{\dfrac{y^2+1}{2}}\left(\left(y^{2}\right)^{6}-\left(y^{2}\right)^{5}+\cdots-y^{2}+1\right).$$
Rõ ràng $y$ lẻ. Đặt $d=\tron{\dfrac{y^2+1}{2},\left(y^{2}\right)^{6}-\left(y^{2}\right)^{5}+\cdots-y^{2}+1}.$ Do $y^2\equiv -1\pmod{d}$ nên là
$$\left(y^{2}\right)^{6}-\left(y^{2}\right)^{5}+\cdots-y^{2}+1\equiv 1-1+1-1+1-1+1\equiv 1\pmod{d}.$$
Như vậy $d=1.$ Theo kiến thức đã học, tồn tại các số nguyên dương $a,b$ sao cho
$$y^2+1=2a^2,\quad \left(y^{2}\right)^{6}-\left(y^{2}\right)^{5}+\cdots-y^{2}+1=b^2.$$
Đến đây, ta đặt $t=y^2.$ Khi đó, với $t\geq 4$ thì ta có các đánh giá dưới đây
\begin{align*}
(16b)^{2} &=\left(16 t^{3}-8 t^{2}+6 t-5\right)^{2}+140 t^{2}-196 t+231 \\
&>\left(16 t^{3}-8 t^{2}+6 t-5\right)^{2},\\
(16 b)^{2} &=\left(16 t^{3}-8 t^{2}+6 t-4\right)^{2}-\left(32 t^{3}-156 t^{2}+208 t-240\right) \\
&>\left(16 t^{3}-8 t^{2}+6 t-4\right)^{2}.
\end{align*}
Cuối cùng, ta chỉ cần kiểm tra các giá trị $t=1,2,3 $, nhưng do $t=y^{2}$ là số chính phương nên chỉ có duy nhất $t=1$ thỏa mãn. Các nghiệm nguyên của phương trình sẽ là
$$(-1,-1),\quad (-1,1),\quad (1,-1),\quad (1,1).$$}
\end{gbtt}

%nguyệt anh
\begin{gbtt}
Giải phương trình nghiệm nguyên \[x^4+4x^3+8x^2+8x+3=y^3.\]
\loigiai{Cộng thêm $1$ vào hai vế, phương trình đã cho tương đương với
    $$\tron{x^2+2x+2}^2=\tron{y+1}\tron{y^2-y+1}.$$
Đặt $d=\tron{y+1,y^2-y+1}.$ Phép đặt này cho ta
\begin{align*}
\heva{&d\mid\tron{y+1}\\&d\mid \tron{y^2-y+1}\\&d\mid \tron{x^2+2x+2}}
&\Rightarrow 
\heva{&y\equiv -1\pmod{d}\\& y^2-y+1\equiv 0\pmod{d}\\&d\mid \tron{x^2+2x+2}}
\\&\Rightarrow 
\heva{&(-1)^2+1+1\equiv 0\pmod{d}\\&d\mid \vuong{(x+1)^2+1}}
\\&\Rightarrow
\heva{&d\mid 3\\&d\mid \vuong{(x+1)^2+1}}
\\&\Rightarrow d=1.
\end{align*}
Theo như bổ đề đã học, ta có $y^2-y+1$ là số chính phương. Ta đặt $y^2-y+1=n^2,$ với $n$ tự nhiên.\\
Phép đặt này cho ta
$$4y^2-4y+4=4n^2\Rightarrow (2y-1)^2+3=(2n)^2\Rightarrow (2n-2y+1)(2n+2y-1)=3.$$
Tới đây, ta lập được bảng giá trị sau
    \begin{center}
    \begin{tabular}{c|c|c|c|c}
        $2n-2y+1$     &  $3$ & $1$ &$-1$ &$-3$\\
        \hline
        $2n+2y-1$     &  $1$ & $3$ &$-3$ &$-1$\\
        \hline
         $y$     &  $0$ & $3$&  $0$ & $3$\\
          \hline
         $x$     &  $-1$ & $\notin \mathbb{N}$&  $-1$ &  $\notin \mathbb{N}$\\
        
            \end{tabular}
        \end{center}
    Như vậy, phương trình đã cho có nghiệm tự nhiên duy nhất là $(-1,0).$}
\end{gbtt}

\begin{gbtt}
Giải phương trình nghiệm tự nhiên
\[x^4+4x^3+7x^2+6x+3=y^3.\]
\loigiai{
Phương trình đã cho tương đương với
    $$\tron{x^2+3x+3}\tron{x^2+x+1}=y^3.$$
Ta đặt $\tron{x^2+3x+3,x^2+x+1}=d.$ Do $x^2+x+1=x(x+1)+1$ nên $d$ lẻ. Ngoài ra
$$\heva{&d\mid\tron{x^2+3x+3}\\&d\mid \tron{x^2+x+1}}
\Rightarrow
\heva{&d\mid(2x+2)\\&d\mid \tron{x^2+x+1}}
\Rightarrow
\heva{&d\mid(x+1)\\&d\mid \tron{x^2+x+1}}
\Rightarrow d\mid 1\Rightarrow d=1.
$$
Theo bổ đề đã học, cả $x^2+3x+3$ và $x^2+x+1$ đều là số lập phương. Ta đặt $$x^2+3x+3=m^3,\quad x^2+x+1=n^3.$$ 
Lấy hiệu theo vế, ta được
$$m^3-n^3=2x+2\Rightarrow(m-n)\tron{m^2+mn+n^2}=2x+2\Rightarrow m^2+mn+n^2<2x+2.$$
Tuy nhiên, ta lại có
    $$m^2+mn+n^2>m^2+n^2=\sqrt[3]{\tron{x^2+3x+3}^2}+\sqrt[3]{\tron{x^2+x+1}^2}\ge (x+1)+(x+1)=2x+2.$$
Hai lập luận trên mâu thuẫn nhau. Kết luận, phương trình đã cho không có nghiệm tự nhiên.}
\end{gbtt}

\begin{gbtt}
Tìm tất cả các số nguyên $x,y$ và số nguyên tố $p$ thỏa mãn \[\dfrac{x^4+x^2+1}{p}=y^4.\]
\loigiai{Nếu thay $x$ bởi $-x,$ kết quả bài toán vẫn không bị ảnh hưởng. Hơn nữa, $x=0$ không thỏa mãn đề bài, do vậy ta chỉ cần xét bài toán trên với $x$ nguyên dương. Ta phân tích
$$\dfrac{x^4+x^2+1}{p}=\dfrac{\left(x^2+1\right)^2-x^2}{p}=\dfrac{\left(x^2+x+1\right)\left(x^2-x+1\right)}{p}.$$
Đặt $d=\left(x^2+x+1,x^2-x+1\right).$ Phép đặt này cho ta
$$\heva{&d\mid \left(x^2+x+1\right) \\ &d\mid \left(x^2-x+1\right)}\Rightarrow\heva{&d\mid 2x \\ &d\mid \left(x^2-x+1\right)}\Rightarrow \heva{&d\mid 2x \\ &d\mid \left(4x^2-4x+4\right)}\Rightarrow d\mid 4.$$
Do $x^2+x+1=x(x+1)+1$ là số lẻ nên $d$ cũng lẻ. Kết hợp với $d\mid 4,$ ta được $d=1.$\\
Đến đây, ta xét các trường hợp sau.
\begin{enumerate}
    \item Nếu $p$ là ước của $x^2-x+1,$ ta có
    $$\left(\dfrac{x^2-x+1}{p}\right)(x^2+x+1)=y^4.$$
    Lập luận được $\left(x^2-x+1,x^2+x+1\right)=1$ ở trên cho ta $$\left(\dfrac{x^2-x+1}{p},x^2+x+1\right)=1.$$ Ta được
    $x^2+x+1$ chính phương. Nhờ vào đánh giá
    $$x^2<x^2+x+1<(x+1)^2,$$
    ta loại trừ được trường hợp kể trên. 
    \item Nếu $p$ là ước của $x^2+x+1,$ bằng cách làm tương tự trường hợp trên, ta chỉ ra
    $x^2-x+1$ chính phương. Nhờ vào đánh giá
    $$(x-1)^2<x^2-x+1\le x^2,$$
    ta tìm ra được $x=1.$ Thay ngược lại, ta dễ dàng chỉ ra $p=3$ và $y=\pm 1.$
\end{enumerate}
Như vậy, có $4$ bộ $(x,y,p)$ thỏa mãn đề bài,  gồm
$(1,1,3),(1,-1,3),(-1,1,3),(-1-1,3).$}
\end{gbtt}

\begin{gbtt}
Giải phương trình nghiệm nguyên $9^x=2y^2+1.$
\loigiai{
Trong trường hợp phương trình đã cho có nghiệm $(x,y),$ ta nhận thấy rằng $x\ge 0,$ bởi vì nếu $x<0$ thì $y$ không là số nguyên. Ngoài ra 
$$\left(3^x-1\right)\left(3^x+1\right)=2y^2.$$
Trong hai số chẵn liên tiếp $3^x-1$ và $3^x+1,$ chắc chẵn có một số chia hết cho $4,$ và số còn lại chia cho $4$ được dư là $2.$ Ta xét các trường hợp kể trên.
\begin{enumerate}
    \item Nếu $x$ chẵn, $3^x-1$ chia hết cho $4,$ còn $3^x+1\equiv 2\pmod{4},$ thế nên dựa vào đẳng thức
    $$\left(\dfrac{3^x-1}{4}\right)\left(\dfrac{3^x+1}{2}\right)=\left(\dfrac{y}{2}\right)^2,$$
    ta chỉ ra $\dfrac{3^x-1}{4}$ là số chính phương, hay $3^x-1$ là số chính phương. Do
    $$\left(3^{\frac{x}{2}}-1\right)^2\le 3^x-1<\left(3^{\frac{x}{2}}\right)^2$$
    nên bắt buộc $\left(3^{\frac{x}{2}}-1\right)^2= 3^x-1,$ hay $x=0.$ Thay ngược lại, ta tìm được $y=0.$
    \item Nếu $x$ lẻ, $3^x+1$ chia hết cho $4,$ còn $3^x-1\equiv 2\pmod{4},$ thế nên dựa vào đẳng thức
    $$\left(\dfrac{3^x+1}{4}\right)\left(\dfrac{3^x-1}{2}\right)=\left(\dfrac{y}{2}\right)^2,$$ 
    ta chỉ ra $\dfrac{3^x+1}{4}$ là số chính phương. Đặt $3^x+1=4z^2,$ với $z$ nguyên dương. Phép đặt này cho ta
    $$3^x=(2z-1)(2z+1).$$
    Cả $2z-1$ và $2z+1$ lúc này đều là lũy thừa số mũ tự nhiên của $3.$\\ Tiếp tục đặt $2z-1=3^u,2z+1=3^v,$ với $u,v$ là các số tự nhiên, ta được
    $$2=(2z+1)-(2z-1)=3^v-3^u=3^u\left(3^{v-u}-1\right).$$
    Số mũ của $3$ trong phân tích của hai số $2$ và $3^u\left(3^{v-u}-1\right)$ lần lượt là $0$ và $u,$ thế nên $u=0.$\\ Ta lần lượt tìm ra $z=1,x=1,y=\pm 2.$
\end{enumerate}
Kết luận, phương trình đã cho có $3$ nghiệm nguyên, đó là $(0,0),(1,-2)$ và $(1,2).$}
\end{gbtt}

\begin{gbtt}
Tìm các số nguyên dương $x,y,z$ thoả mãn điều kiện 
$$x^3-y^3=z^2,$$
trong đó $y$ là số nguyên tố và $(z,3)=(x,y)=1.$
\loigiai{
Giả sử tồn tại các số $x,y,z$ thỏa mãn đề bài. Phương trình đã cho tương đương với
$$(x-y)\tron{x^2+xy+y^2}=z^2.$$
Đặt $d=\tron{x-y,x^2+xy+y^2}.$ Phép đặt này cho ta
\begin{align*}
\heva{&d\mid (x-y) \\ &d\mid \tron{x^2+xy+y^2} \\ &d\mid z}
&\Rightarrow \heva{&x\equiv y\pmod{d} \\ &x^2+xy+y^2\equiv 0\pmod{d}\\ &d\mid z}
\\&\Rightarrow \heva{&d\mid 3x^2 \\ &d\mid 3y^2\\ &d\mid z}
\\&\Rightarrow \heva{&d\mid 3(x,y)^2\\ &d\mid z}
\\&\Rightarrow \heva{&d\mid 3\\ &d\mid z}
\\&\Rightarrow d=1.    
\end{align*}
Ta suy ra $x-y$ và $x^2+xy+y^2$ là số chính phương từ đây. Ta đặt \[x^2+xy+y^2=t^2,\quad x-y=k^2\] với $k,t\in \mathbb{N}.$ Phép đặt này cho ta
$$3{{y}^{2}}=4{{t}^{2}}-4{{x}^{2}}-{4xy}-{{y}^{2}}=\left( 2t+2x+y \right)\left( 2t-2x-y \right).$$
Do $y$ là số nguyên tố và $2t+2x+y>2t+2y+y>3y,$ ta xét các trường hợp sau đây.
\begin{enumerate}
    \item Với $2t+2x+y=3y^2$ và $2t-2x-y=1,$ ta có
    \[3y^2-1=2(2x-y)=4k^2+2y.\] Từ đây ta suy ra $(y-1)(3y+1)=4k^2.$ Dễ thấy $(y-1,3y+1)\in\{1;2;4\}.$
    \begin{itemize}
        \item Nếu $(y-1,3y+1)=2$, khi đó ta phải có $$3y+1=2a^2$$ với $a\in\mathbb{Z}$ nào đó. Dẫn đến $a^2+1$ chia hết cho $3,$ không thể xảy ra.
        \item Nếu $(y-1,3y+1)=1$ hoặc $4$, khi đó ta phải có $3y+1=a^2$ với $a\in\mathbb{Z}$. Dẫn đến \[y=\dfrac{(a-1)(a+1)}{3}. \]
        Lại do $y$ là số nguyên tố nên $\dfrac{a+1}{3}=1$ hoặc $\dfrac{a-1}{3}=1.$ Ta tìm được $$y=5,\, k=4,\, x=9.$$ Thay vào ta thấy không có $z$ thỏa mãn.
    \end{itemize}
    \item Với $2t+2x+y=y^2$ và $2t-2x-y=3,$ lấy hiệu theo vế ta có
    \[y^2-3=2\left(2x+y\right).\tag{*}\label{bdscpp}\]
    Phương trình (\ref{bdscpp}) trở thành
    $$y^2-3=2\tron{2k^2+3y}\Leftrightarrow{{\left( y-3 \right)}^{2}}-4{{k}^{2}}=12\Leftrightarrow \left( y-3+2k \right)\left( y-3-2k \right)=12.$$
    Từ đó tìm được $y=7$, thay vào ta có $x=8,z=13$.
\end{enumerate}
Như vậy, có duy nhất bộ $\left( x,y,z \right)=\left( 8,7,13 \right)$ thỏa mãn yêu cầu bài toán.}
\end{gbtt}

\begin{gbtt}
Tìm tất cả các số nguyên tố $p$ và hai số nguyên dương $a, b$ sao cho $p^a+p^b$ là số chính phương.
\nguon{Tạp chí Toán học và Tuổi trẻ số 507, tháng 9 năm 2019}
\loigiai{
Giả sử $p^a+p^b=c^2$. Ta xét các trường hợp sau.
\begin{enumerate}
    \item Nếu $a=b,$ ta có $c^2=2p^a.$ Ta suy ra $4\mid c^2,$ kéo theo $p=2.$ Ngoài ra, $a$ phải là số lẻ.
    \item Nếu $a\ne b,$ không mất tổng quát, ta giả sử $a>b.$ Ta có
    $$c^2=p^a+p^b = p^b\tron{p^{a-b}+1}.$$
    Do $\tron{p^b,p^{a-b}+1}=1$ nên cả $p^b$ và $p^{a-b}+1$ là số chính phương. Ta đặt
    $$p^b=x^2,\quad p^{a-b}+1=y^2.$$
    Từ $p^{a-b}+1=y^2,$ ta có $p^{a-b}=(y-1)(y+1).$ Cả $y-1$ và $y+1$ đều là luỹ thừa của $p.$ Ta đặt
    $$y+1=p^v,\: y-1=p^u,\text{ trong đó }u>v\ge 0.$$
    Lấy hiệu theo vế, ta được $p^v\tron{p^{u-v}-1}=2.$ Từ đây ta suy ra $p^v=2$ hoặc $p^v=1.$
    \begin{itemize}
        \item \chu{Trường hợp 1.} Nếu $p^v=2,$ ta có $p=2,v=1,$ đồng thời $u-v=1.$ Ta lần lượt tìm ra $$u=2,\quad a=2k+3,\quad b=2k.$$
        \item \chu{Trường hợp 2.} Nếu $p^v=1,$ ta có $v=0,$ đồng thời $p^{u-v}=3.$ Ta lần lượt tìm ra $$p=3,\quad u=1,\quad a=2k+1,\quad b=2k.$$        
    \end{itemize}
\end{enumerate}
Như vậy, các bộ ba số $(p,a,b)$ cần tìm là 
$$(2,2k-1,2k-1),\:\: (2,2k+3,2k),\:\: (2,2k,2k+3),\:\: (3,2k+1,2k),\:\: (3,2k,2k+1),$$
trong đó $k$ là một số nguyên dương tuỳ ý.}
\end{gbtt}



 %kẹp + đặt gcd
\section{Phương trình chứa ẩn ở mũ}

\subsection{Phương pháp đánh giá}

\subsubsection*{Bài tập tự luyện}

\begin{btt}
Tìm tất cả các số nguyên dương $n$ thỏa mãn
\[1^n+9^n+10^n=5^n+6^n+11^n.\]
\end{btt}

\begin{btt}
Tìm tất cả các số nguyên dương $x,y,z,t$ và $n$ thỏa mãn
\[n^x+n^y+n^z=n^t.\]
\end{btt}

\begin{btt}
Chứng minh rằng không tồn tại các số nguyên dương $x,y,z$ thỏa mãn
\[x^x+y^y=z^z.\]
\end{btt}

\begin{btt}
Tìm tất cả các số nguyên dương $x,y$ thỏa mãn
\[x^{x^{x^{x}}}=\tron{19-y^x}y^{x^y}-74.\]
\end{btt}

\begin{btt}
Giải phương trình nghiệm nguyên dương  $$4^{y}+4^{y}+2^{xy}-2^{x^{2}}-2^{y^{2}}=16.$$
\nguon{Cao Đình Huy}
\end{btt}

\begin{btt}
Tìm tất cả các nghiệm nguyên dương của phương trình 
\[x^y+y^z+z^x=2\tron{x+y+z}.\]
\end{btt}

\begin{btt}
Tìm tất cả các số tự nhiên $n$ sao cho $3^n+n^2$ là số chính phương.
\end{btt}

\begin{btt}
Tìm tất cả các số nguyên dương $m,n$ và số nguyên tố $p$ thỏa mãn
\[n^{2p}=2n^2+m^2+p+2.\]
\end{btt}

\subsubsection*{Hướng dẫn bài tập tự luyện}

\begin{gbtt}
Tìm tất cả các số nguyên dương $n$ thỏa mãn
\[1^n+9^n+10^n=5^n+6^n+11^n.\]
\loigiai{
Vì $1^n,10^n,5^n,6^n$ và $11^n$ lần lượt có chữ số tận cùng là $1,0,5,6$ và $1.$ Xét modulo $10$ cả hai vế, ta được
\begin{align*}
    1^n+9^n+10^n\equiv 1+9^n+0\equiv9^n+1\pmod{10},\\
    5^n+6^n+11^n\equiv 5+6+1\equiv2\pmod{10}.
\end{align*}
Từ đây, ta suy ra $9^n\equiv 1\pmod{10},$ kéo theo $2\mid n.$ Thử trực tiếp, ta nhận được $n=2$ và $n=4$ là hai nghiệm của phương trình. Với mọi $n\ge 6,$ ta có
$$n10^{n-1}\ge6\cdot10^4\cdot10^{n-5}\ge 9^5\cdot10^{n-5}\ge9^n.$$
Xét vế phải của phương trình, ta thu được
$$VP>11^n=(10+1)^n=10^n+n\cdot10^{n-1}+\cdots+1\ge10^n+9^n+1=VT.$$
Điều này không thể xảy ra. Tất cả các số nguyên dương $n$ cần tìm là $n=2$ và $n=4.$}
\end{gbtt}

\begin{gbtt}
Tìm tất cả các số nguyên dương $x,y,z,t$ và $n$ thỏa mãn
\[n^x+n^y+n^z=n^t.\]
\loigiai{
Chia cả hai vế của phương trình cho $n^t,$ ta được
$$n^{x-t}+n^{y-t}+n^{z-t}=1.$$
Vì mỗi hạng tử vế trái đều dương và bé hơn $1$ nên $x,y,z<t.$ Với $n\ge4$, ta suy ra
$$n^{x-t}+n^{y-t}+n^{z-t}<\dfrac{3}{n}\le \dfrac{3}{4}<1.$$
Điều này không thể xảy ra nên $n\le 3.$ Ta xét các trường hợp sau.
\begin{enumerate}
    \item Với $n=3,$ ta có bất phương trình sau
    $$n^{x-t}+n^{y-t}+n^{z-t}\le\dfrac{3}{3}=1.$$
    Từ đây, ta suy ra dấu bằng của bất phương trình trên phải xảy ra. Do đó $x+1=y+1=z+1=t.$
    \item Với $n=2,$ không mất tính tổng quát, ta giả sử $x\le y\le z.$ Từ đây, ta suy ra
    $$3\cdot2^{z-t}\ge 2^{x-t}+2^{y-t}+2^{z-t}=1.$$
    Từ đây, ta suy ra $2^{z-t}\ge \dfrac{1}{3}$ hay $t-z<2.$ Điều này dẫn tới $t-z=1$ hay $z=t-1.$\\ Chứng minh tương tự, ta thu được 
    $$x=t-1,\qquad y=t-1.$$
    \item Với $n=1,$ thử trực tiếp ta thấy không thỏa mãn phương trình.
\end{enumerate}
Như vậy, bộ số nguyên dương $(x,y,z,t,n)$ thỏa mãn là
$$(x,x,x,x+1,3),\quad (x,x,x+1,x+2,2),\quad(x,x+1,x,x+2,2),\quad(x+1,x,x,x+2,2).$$
trong đó $x$ là số nguyên dương.
}
\end{gbtt}

\begin{gbtt}
Chứng minh rằng không tồn tại các số nguyên dương $x,y,z$ thỏa mãn
\[x^x+y^y=z^z.\]
\loigiai{
Giả sử tồn tại các số nguyên dương $x,y,z$ thỏa mãn phương trình. Ta dễ dàng nhận thấy 
$$z^z>x^x,\quad  z^z>y^y,$$ 
kéo theo $z>x,y.$ Vì $x,y,z$ là các số nguyên nên $z\ge x+1$ và $z\ge y+1.$ Từ đó ta có
$$z^z\ge\tron{x+1}^{x+1}=(x+1)\tron{x+1}^{x}\ge2\tron{x+1}^x>2x^x.$$
Chứng minh tương tự, ta thu được $z^z>2y^y.$ Cộng theo vế hai bất phương trình, ta có 
$$2z^z>2x^x+2y^y,$$
hay $z^z>x^x+y^y,$ mâu thuẫn. Giả sử sai. Bài toán được chứng minh.}
\end{gbtt}

\begin{gbtt}
Tìm tất cả các số nguyên dương $x,y$ thỏa mãn
\[x^{x^{x^{x}}}=\tron{19-y^x}y^{x^y}-74.\]
\loigiai{
Vì vế trái lớn hơn $0$ nên ta dễ dàng suy ra $y^x<19.$ Xét $y=1,$ thế trở lại phương trình, ta có
$$x^{x^{x^{x}}}=\tron{19-1}\cdot1-74<0.$$
Điều này không thể xảy ra. Do đó $y\ge2,$ kéo theo $x\le 4.$  Ta xét các trường hợp sau.
\begin{enumerate}
    \item Với $x=1,$ thế trở lại phương trình cho ta 
    $$1=(19-y)y-74,$$
    hay $y^2-19y+75=0.$ Phương trình này không có nghiệm nguyên.
    \item Với $x\ge 2,$ kết hợp $y^x<19,$ ta nhận được các cặp số $(x,y)$ là $$(2,2),\quad (2,3),\quad (2,4),\quad (3,2),\quad (4,2).$$ Thử trực tiếp, ta thu được $(x,y)=(2,3)$ là bộ số thỏa mãn duy nhất.
\end{enumerate}
Như vậy, có duy nhất cặp số nguyên $(x,y)$ thỏa mãn là $(2,3).$
}
\end{gbtt}

\begin{gbtt}
Giải phương trình nghiệm nguyên dương  $$4^{y}+4^{y}+2^{xy}-2^{x^{2}}-2^{y^{2}}=16.$$
\nguon{Cao Đình Huy}
\loigiai{
Với $x,y,z>0$, ta sẽ chứng minh bất đẳng thức
$$2^{x^{2}}+2^{y^{2}}+2^{z^{2}}\geq 2^{xy}+2^{yz}+2^{zx}.$$
Thật vậy, sử dụng bài toán quen thuộc $\left ( a+b+c \right )^{2}\geq 3(ab+bc+ca)$, ta có
\begin{align*}
    \left ( 2^{x^{2}}+2^{y^{2}}+2^{z^{2}} \right )^{2}
    &\geq 3\left ( 2^{x^{2}+y^{2}}+2^{y^{2}+z^{2}}+2^{z^{2}+x^{2}} \right )
    \\&\geq 3\bigg[ (2^{xy})^{2}+(2^{yz})^{2}+(2^{zx})^{2}\bigg]
    \\&\geq 3\cdot \dfrac{(2^{xy}+2^{yz}+2^{zx})^{2}}{3}\\&=(2^{xy}+2^{yz}+2^{zx})^{2}\\&=(2^{xy}+2^{yz}+2^{zx})^{2}.
\end{align*}
Do đó $2^{x^{2}}+2^{y^{2}}+2^{z^{2}}\geq 2^{xy}+2^{yz}+2^{zx}.$ Khi $z=2,$ bất đẳng thức trở thành $$2^{x^{2}}+2^{y^{2}}+16\geq 2^{xy}+2^{2x}+2^{2y}.$$
Đối chiếu với phương trình đã cho, ta nhận thấy đẳng thức phải xảy ra, tức là $x=y=2.$\\
Đây là nghiệm nguyên dương duy nhất của phương trình.}
\begin{luuy}
Ta không nhất thiết cần tới điều kiện $x,y$ nguyên dương để giải bài toán vừa rồi.
\end{luuy}
\end{gbtt}

\begin{gbtt}
Tìm tất cả các nghiệm nguyên dương của phương trình 
\[x^y+y^z+z^x=2\tron{x+y+z}.\]
\loigiai{
Giả sử phương trình có các nghiệm nguyên dương $(x,y,z).$ Ta xét các trường hợp sau.
\begin{enumerate}
    \item Nếu $x,y,z$ không nhỏ hơn $2,$ ta có nhận xét.
    $$VT=x^y+y^z+z^x\ge x^2+y^2+z^2\ge 2(x+y+z)=VP.$$
    Dấu bằng của đánh giá trên xảy ra. Ta có $x=y=z=2.$
    \item Nếu có ít nhất một trong ba số $x,y,z$ nhỏ hơn bằng $1$ (giả sử là $z$), phương trình trở thành
    $$x^y=2x+y+1.$$
    Từ đây, ta xét các trường hợp sau.
    \begin{itemize}
        \item\chu{Trường hợp 1.} Với $y<4,$ ta có $y\in\{1;2;3\}.$ Ta lập bảng.
        \begin{center}
            \begin{tabular}{c|c|c}
                $y$   &  Phương trình sau khi thế & $x$\\
                \hline
                $1$ & $x=2x+2$ & $-2$\\
                \hline
                $2$ & $x^2=2x+3$ & $3$ \\
                \hline
                $3$ & $x^3=2x+4$ & $2$
            \end{tabular}
        \end{center}
        Trường hợp này cho ta hai bộ số $(x,y,z)$ thỏa mãn là $(3,2,1)$ và $(2,3,1).$
        \item\chu{Trường hợp 2.} Với $x<4,$ ta có $x\in\{1;2;3\}.$ Ta lập bảng. 
        \begin{center}
            \begin{tabular}{c|c|c}
                $x$   &  Phương trình sau khi thế & $y$\\
                \hline
                $1$ & $1=y+3$ & $-2$\\
                \hline
                $2$ & $2^y=y+5$ & $3$ \\
                \hline
                $3$ & $3^y=y+7$ & $2$
            \end{tabular}
        \end{center}        
        Trường hợp này cho ta hai bộ số $(x,y,z)$ thỏa mãn là $(2,3,1)$ và $(3,2,1).$
        \item \chu{Trường hợp 3.} Với $x,y\ge 4$, bằng quy nạp, ta chứng minh được
        $$x^y>xy,\text{ với mọi }x,y\ge 4.$$
        Chứng minh trên cho ta $x^y>xy\ge 2x+2y+2(x+y-8)>2x+y+1,$ vô lí.
    \end{itemize}
\end{enumerate}
Như vậy, phương trình có $7$ nghiệm nguyên dương, gồm $(2,2,2),(1,2,3)$ và hoán vị của chúng.}
\end{gbtt}

%nguyệt anh
\begin{gbtt}
Tìm tất cả các số tự nhiên $n$ sao cho $3^n+n^2$ là số chính phương.
\loigiai{
Giả sử tồn tại số nguyên dương $m$ thỏa mãn $n^{2}+3^{n}=m^{2}.$ Ta viết lại
$$(m-n)(m+n)=3^{n}.$$ 
Khi đó, tồn tại số tự nhiên $k$ sao cho $m-n=3^{k}$ và $m+n=3^{n-k}.$\\
Vì $m+n\ge m-n$ nên $k\le n-k,$ và do đó $n-2 k \geq 0.$ Ta xét các trường hợp sau.
\begin{enumerate}
    \item Nếu $n=2k=0$ thì các dấu bằng phía trên phải xảy ra, tức là $m+n=m-n$ hay $n=0.$
    \item Nếu $n-2k=1$ thì ta có
    \begin{align*}
        2 n=(m+n)-(m-n)=3^{n-k}-3^{k}&=3^{k}\left(3^{n-2 k}-1\right)=2\cdot3^{k}.
    \end{align*}
    Vậy $n=2 k+1=3^{k}.$ Hơn nữa, do $$3^{k}=(1+2)^{k}=1+2 k+\cdots+2^{k}>2 k+1$$ nên ta suy ra $k=0,1,$ và do đó $n=1$ hoặc $n=3.$
    \item Nếu $n-2 k>1$ thì ta lần lượt suy ra
    \begin{align*}
        k\le n-k-2
        \Rightarrow 3^{k} \leq 3^{n-k-2}
        \Rightarrow 2 n&=3^{n-k}-3^{k} \\&\geq 3^{n-k}-3^{n-k-2}\\&=3^{n-k-2}\left(3^{2}-1\right)
        \\&\ge8\cdot 3^{n-k-2}\\&\geq 8(1+2(n-k-2))\\&=16 n-16 k-24.
    \end{align*}
    Từ loạt đánh giá trên, ta chỉ ra $8k+12\ge 7n\ge 7(2k+2)=14k+14,$ vô lí.
\end{enumerate}
Như vậy, chỉ có $n=0,\ n=1$ và $n=3$ là các số tự nhiên thỏa mãn đề bài.}
\end{gbtt}

\begin{gbtt}
Tìm tất cả các số nguyên dương $m,n$ và số nguyên tố $p$ thỏa mãn
\[n^{2p}=2n^2+m^2+p+2.\]
\loigiai{
Giả sử tồn tại các số nguyên $m,n,p$ thỏa phương trình. Ta có
$$2n^2+p+2=\left(n^p-m\right)\left(n^p+m\right)\ge n^p+m+1\ge n^p+2.$$
Do $n^p>m^2\ge 1$ nên $n\ge 2.$ Từ đây, ta tiếp tục suy ra
$$p+1\ge n^2\left(n^{p-2}-2\right)\ge 4\left(2^{p-2}-2\right)=2^p-8.$$
Bằng quy nạp, ta chứng minh được $2^x>x+9,$ với mọi $x\ge 4.$ \\
Như vậy $p\le 3,$ tức $p=2$ hoặc $p=3.$ Ta xét hai trường hợp kể trên. 
\begin{enumerate}
    \item Với $p=2,$ thay vào phương trình ban đầu rồi biến đổi tương đương, ta có
    $$\left(n^2-m-1\right)\left(n^2+m-1\right)=5.$$
    Giải phương trình ước số trên, ta thu được $n=2$ và $m=2.$
    \item Với $p=3,$ thay vào phương trình ban đầu rồi biến đổi tương đương, ta có
    $$(n-2)\left(n^5+2n^4+4n^3+8n^2+14n+20\right)+\left(m^2+51\right)=0.$$
    Vế trái phương trình trên luôn lớn hơn $0.$ Trường hợp này không thể xảy ra.
\end{enumerate}
Tóm lại, $(m,n,p)=(2,2,2)$ là bộ số duy nhất thỏa mãn đề bài.}
\end{gbtt}


\subsection{Phương pháp xét tính chia hết}

\subsubsection*{Bài tập tự luyện}

\begin{btt}
Tìm tất cả các số nguyên dương $x,y$ thỏa mãn $$x^5+x^4=7^y-1.$$
\nguon{Belarusian National Olympiad 2007}
\end{btt}

\begin{btt}
Chứng minh rằng với mọi số nguyên dương $a$ và $b$ thì $$(36a+b)(a+36b)$$ không thể là một lũy thừa số mũ tự nhiên của $2.$
\end{btt}

\begin{btt}
Tìm tất cả các cặp số nguyên dương $(a,b)$ sao cho $$(x+y)(xy+1)$$ là một lũy thừa số mũ tự nhiên của $3$.
\end{btt}

\begin{btt}
Tìm tất cả các cặp số nguyên dương $(a,b)$ sao cho $$\tron{a^2+b}\tron{a+b^2}$$ là một lũy thừa số mũ nguyên dương của $2.$
\nguon{Tạp chí Pi, tháng 5 năm 2017, Kvant 301}
\end{btt}

\begin{btt}
Tìm tất cả bộ ba các số nguyên dương $(a,b,c)$ thỏa mãn $$\left(a^3+b\right)\left(a+b^3\right)=2^c.$$
\nguon{Turkey Junior Balkan Mathematical Olympiad Team Selection Test 2014}
\end{btt}

\begin{btt}
Tìm tất cả các bộ ba số nguyên dương $(a,b,c)$ thỏa mãn \[\tron{a^5+b}\tron{a+b^5}=2^c.\]
\end{btt}

\begin{btt}
Giả sử $n$ là một số nguyên dương thỏa mãn tồn tại $a, b, c$ nguyên dương sao cho 
$$7^{n}=(a+b c)(b+a c).$$ 
Chứng minh rằng $n$ là số chẵn.
\end{btt}

\begin{btt}
Giải phương trình nghiệm tự nhiên \[n^x+n^y=n^z.\]
\end{btt}

\begin{btt}
Tìm các số nguyên dương $x$ và $y$ khác nhau sao cho
\[x^y=y^x.\]
\end{btt}

\begin{btt}
Giải phương trình nghiệm nguyên dương
$$x^y=y^{x-y}.$$
\nguon{Junior Balkan Mathematical Olympiad 1998}
\end{btt}

\begin{btt}
Phương trình sau có bao nhiêu nghiệm nguyên dương?
\[\tron{x^y-1}\tron{z^t-1}=2^{200}.\]
\end{btt}

\subsubsection*{Hướng dẫn bài tập tự luyện}

\begin{gbtt}
Tìm tất cả các số nguyên dương $x,y$ thỏa mãn $$x^5+x^4=7^y-1.$$
\nguon{Belarusian National Olympiad 2007}
\loigiai{
Giả sử tồn tại các số nguyên dương $x,y$ thỏa mãn yêu cầu. Ta viết lại phương trình thành
$$\tron{x^3-x+1}\tron{x^2+x+1}=7^y.$$
Ta suy ra cả $x^3-x+1$ và $x^2+x+1$ đều là lũy thừa số mũ nguyên dương của $7.$ Ta đặt
$$x^3-x+1=7^a,\quad x^2+x+1=7^b.$$
Ta xét các trường hợp sau đây.
\begin{enumerate}
    \item Nếu $a\ge b$ thì $7^a$ chia hết cho $7^b,$ và
    \begin{align*}
        \tron{x^2+x+1}\mid\tron{x^3-x+1}
        &\Rightarrow \tron{x^2+x+1}\mid\tron{\tron{x-2}\tron{x^2+x+1}-\tron{x-2}}
        \\&\Rightarrow \tron{x^2+x+1}\mid\tron{x-2}.
    \end{align*}
    Với $x=1,x=2,$ kiểm tra trực tiếp, ta tìm được $y=2$ khi $x=2.$ Với $x\ge 3,$ ta có
    $$x^2+x+1\le x-2\Rightarrow x^2+3\le 0.$$
    Đây là điều không thể xảy ra.
    \item Nếu $a<b,$ lập luận tương tự, ta chỉ ra $x^2+x+1\ge 7\tron{x^3-x+1}.$ \\
    Không tồn tại số nguyên dương $x$ nào thỏa mãn điều này.
\end{enumerate}
Như vậy, cặp số $(x,y)=(2,2)$ là cặp số duy nhất thỏa yêu cầu.}
\begin{luuy}
\begin{enumerate}
    \item  Bài toán trên thuộc mảng tính chất lũy thừa của một số nguyên tố. Các kĩ thuật sử dụng tính chất của chúng đã được nói rõ ở \chu{chương II}.
    \item Ở trong các phép chia hết như $x-2$ chia hết cho $x^2+x+1$ phía trên, ta có thể tìm được $x$ mà không cần phải sử dụng bất đẳng thức.
\end{enumerate}    
\end{luuy}
\end{gbtt}

\begin{gbtt}
Chứng minh rằng với mọi số nguyên dương $a$ và $b$ thì $(36a+b)(a+36b)$ không thể là một lũy thừa số mũ tự nhiên của $2.$
\loigiai{
Không mất tính tổng quát, ta giả sử $a\ge b.$ Giả sử cho ta $36a+b\ge a+36b.$ Ta sẽ chứng minh bài toán bằng phản chứng. Nếu như $(36a+b)(a+36b)=2^c$ trong đó $c$ là số nguyên dương, ta đặt 
$$36a+b=2^m, \:a+36b=2^n$$ 
ở đây $m,n$ là các số tự nhiên thỏa mãn $m\ge n>0.$ Lấy hiệu theo vế, ta được
$$35(a-b)=2^n\tron{2^{m-n}-1}.$$
Vì $\tron{35,2^n}=1$ nên $2^n\mid (a-b)$. Trong trường hợp $a>b,$ ta nhận thấy rằng
$$a-b\ge 2^n\Rightarrow a>2^n\Rightarrow a+36b>2^n.$$
Điều này trái với lập luận trên của ta. Do đó $a=b$, nhưng lúc này $36a+b=37a$ là lũy thừa số mũ tự nhiên của $2,$ vô lí. Do đó giả sử sai, và bài toán được chứng minh.}
\end{gbtt}

\begin{gbtt}
Tìm tất cả các cặp số nguyên dương $(a,b)$ sao cho $$(x+y)(xy+1)$$ là một lũy thừa số mũ tự nhiên của $3$.
\loigiai{
Rõ ràng, $xy+1$ và $x+y$ đều là lũy thừa cơ số $3.$ \\
Ta đặt $3^a=xy+1,\:3^b=x+y,$ với $a,b$ là các số nguyên dương. Lần lượt lấy tổng và hiệu theo vế, ta được
\begin{align*}
    3^b\left(3^{a-b}+1\right)&=(x+1)(y+1),\tag{1}\label{luythuadayne.1}\\
3^b\left(3^{a-b}-1\right)&=(x-1)(y-1).\tag{2}\label{luythuadayne.2}
\end{align*}

Hai số $x+1$ và $x-1$ có hiệu bằng $2,$ thế nên trong chúng phải có một số không là bội của $3.$ 
\begin{enumerate}
    \item Nếu $x+1$ không chia hết cho $3,$ từ (\ref{luythuadayne.1}), ta suy ra $y+1$ chia hết cho $3^b,$ thế nên
    $$y+1\ge 3^b=x+y.$$
    Ta bắt buộc phải có $x=1.$ Lúc này, ta tìm ra $y=3^a-1.$
    \item Nếu $x-1$ không chia hết cho $3,$ từ (\ref{luythuadayne.2}), ta suy ra $y-1$ chia hết cho $3^b.$ Khi $y\ge 2,$ ta có
    $$y-1\ge 3^b=x+y.$$ 
    Nhận xét trên là một mâu thuẫn, thế nên bắt buộc $y=1.$ Lúc này, ta tìm ra $x=3^a-1.$
\end{enumerate}
Kết luận, các bộ $(x,y)$ thỏa mãn đề bài có dạng $\left(3^a-1,1\right)$ và $\left(1,3^a-1\right),$ trong đó $a$ nguyên dương.}
\end{gbtt}

\begin{gbtt}
Tìm tất cả các cặp số nguyên dương $(a,b)$ sao cho $\tron{a^2+b}\tron{a+b^2}$ là một lũy thừa số mũ nguyên dương của $2.$
\nguon{Tạp chí Pi, tháng 5 năm 2017, Kvant 301}
\loigiai
{Rõ ràng, $a^2+b$ và $a+b^2$ đều là lũy thừa cơ số $2.$ Không mất tổng quát, ta giả sử $a\ge b.$ Ta đặt
$$2^x=a^2+b,\:2^y=a+b^2,$$ 
trong đó $x,y$ là các số nguyên dương. Phép đặt này cho ta 
\[2^{y}\left(2^{x-y}-1\right)=2^x-2^y=a^2+b-a-b^2=(a-b)(a+b-1).\tag{1}\label{pi.p35.1}\]
Cũng từ $a^2+b=2^x,$ ta suy ra $a-b$ là số chẵn và $a+b-1$ là số lẻ, thế nên (\ref{pi.p35.1}) cho ta $2^y\mid (a-b).$ \\
Tới đây, ta xét hai trường hợp.
\begin{enumerate}
    \item Với $x=y,$ thế ngược lại, ta tìm ra $a=b=1.$ 
    \item Với $x> y,$ ta có $a>b.$ Ta nhận xét rằng
    $$a-b\ge 2^y=a+b^2.$$
    Nhận xét trên dẫn ta đến $b^2+b\ge 0,$ một điều vô lí. Trường hợp này không cho $(a,b)$ phù hợp.
\end{enumerate}
Kết quả, $(a,b)=(1,1)$ là cặp số duy nhất thỏa mãn yêu cầu bài toán.}
\end{gbtt}

\begin{gbtt}
Giả sử $n$ là một số nguyên dương thỏa mãn tồn tại $a, b, c$ nguyên dương sao cho 
$$7^{n}=(a+b c)(b+a c).$$ 
Chứng minh rằng $n$ là số chẵn.
\loigiai{
Với các số $n,a,b$ thỏa mãn đẳng thức đã cho, ta có thể đặt
\[a+bc=7^p,\quad b+ac=7^q,\tag{1}\label{abcpq}\]
trong đó $p,q$ là các số tự nhiên. Ta giả sử rằng $a\ge b,$ như thế $p\le q.$\\ Lấy tổng và hiệu theo vế trong (\ref{abcpq}) ta được
\begin{align}
    7^p\tron{7^{q-p}+1}&=(a+b)(c+1),\tag{2}\label{abcpq.2}\\
    7^p\tron{7^{q-p}-1}&=(a-b)(c-1).\tag{3}\label{abcpq.3}
\end{align}
Hai số $c+1$ và $c-1$ có hiệu là $2,$ thế nên trong chúng phải có một số không là bội của $7.$
\begin{enumerate}
    \item Nếu $c+1$ không chia hết cho $7,$ từ (\ref{abcpq.2}) ta có $a+b$ chia hết cho $7^p,$ thế nên
    $$a+b\ge 7^p=a+bc.$$
    Chuyển vế và rút gọn, ta được $b(c-1)\le 0,$ kéo theo $c=1.$ Như thế thì
    $$7^n=(a+bc)(b+ac)=(a+b)(b+a)=(a+b)^2.$$
    Ta suy ra $n$ chẵn từ đây.
    \item Nếu $c-1$ không chia hết cho $7,$ từ (\ref{abcpq.3}) ta có $a-b$ chia hết cho $7^p.$ Ta xét các trường hợp nhỏ hơn.
    \begin{itemize}
        \item \chu{Trường hợp 1.} Với $a=b,$ ta có
        $$7^n=(a+bc)(b+ac)=(b+bc)(b+bc)=(b+bc)^2.$$
    Ta suy ra $n$ chẵn từ đây.       
        \item \chu{Trường hợp 2.} Với $a>b,$ kết hợp với phép chia hết ở trên ta có
        $$a-b\ge 7^q=a+bc.$$
        Chuyển vế, ta được $b(c+1)\le 0,$ vô lí.
    \end{itemize}
\end{enumerate}
Bài toán được chứng minh trong mọi trường hợp.
}
\end{gbtt}

\begin{gbtt}
Tìm tất cả bộ ba các số nguyên dương $(a,b,c)$ thỏa mãn $$\left(a^3+b\right)\left(a+b^3\right)=2^c.$$
\nguon{Turkey Junior Balkan Mathematical Olympiad Team Selection Test 2014}
\loigiai{
Rõ ràng, $a^3+b$ và $a+b^3$ đều là lũy thừa cơ số $2.$ \\
Ta đặt $2^x=a^3+b,\:2^y=a+b^3,$ với giả sử $x\ge y\ge 1.$ Lấy tổng và hiệu theo vế, ta được
\begin{align*}
    2^y\left(2^{x-y}+1\right)&=(a+b)\tron{a^2-ab+b^2+1},\tag{1}\label{turjeyjuni.1}\\
2^y\left(2^{x-y}-1\right)&=(a-b)\tron{a^2+ab+b^2-1}.\tag{2}\label{turjeyjuni.2}
\end{align*}

Ta cũng không khó chỉ ra cả $a$ và $b$ đều lẻ. Tới đây, ta xét các trường hợp sau
\begin{enumerate}
    \item Nếu $x=y$ thì ta tìm được $(a,b,c)=(1,1,2).$
    \item Nếu $x>y$ và $a\equiv b\pmod{4}$ thì $a\equiv b\equiv 1\pmod{4}$ hoặc $a\equiv b\equiv 3\pmod{4}.$ Khi ấy
    $$a+b\equiv 2\pmod{4},\quad a^2-ab+b^2+1\equiv 2\pmod{4}.$$
    Lập luận này kết hợp với (\ref{turjeyjuni.1}) cho ta $2^y=4$ hay $y=2.$ Ta có $a+b^3=4$ nên $(a,b)=(3,1),$ mâu thuẫn với điều kiện $a\equiv b\pmod{4}.$
    \item Nếu $x>y$ và $a\not\equiv b\pmod{4}$ thì tương tự trường hợp vừa rồi, ta chỉ ra
    $$a-b\equiv 2\pmod{4}.$$    
    Lập luận này kết hợp với (\ref{luythuadayne.2}) cho ta $2\tron{a^2+ab+b^2-1}$ chia hết cho $2^y=a+b^3.$ Ta có
    $$\tron{a+b^3}\mid 2b\tron{a^2+ab+b^2-1}=2\bigg[(ab-1)(a+b)+a+b^3\bigg].$$
    Ta có $a+b^3$ là ước của $2(ab-1)(a+b),$ nhưng do $ab-1$ chia cho $4$ dư $2$ và $a+b^3$ là lũy thừa của $2$ nên $4(a+b)$ chia hết cho $a+b^3.$ Dựa trên so sánh
    $$a+b^3\le 4(a+b)\le 4\tron{a+b^3},$$
    ta sẽ đi xét các trường hợp sau.
    \begin{itemize}
        \item\chu{Trường hợp 1.} Nếu $4(a+b)=a+b^3,$ ta có $a=\dfrac{b^3-4b}{3},$ và lúc này
        $$2^y=b^3+a=b^3+\dfrac{b^3-4b}{3}=\dfrac{b(b-1)(b+1)}{3}.$$
        Với việc tồn tại một cặp hai trong ba số $b,b-1,b+1$ là lũy thừa của $2,$ ta có $b\in\{1;2;3\}.$ Từ đây dễ dàng tìm ra $(a,b,c)=(5,3,2)$ là kết quả trong trường hợp này.
        \item\chu{Trường hợp 2.} Nếu $4(a+b)=2\tron{a+b^3},$ ta có $a=b^3-2b,$ và lúc này  
        $$2^y=b^3+a=b^3+b^3-2b=2b(b-1)(b+1).$$ 
        Rõ ràng, không tồn tại số $b$ nào để cho cả $b,b-1$ và $b+1$ đều là lũy thừa của $3.$
        \item\chu{Trường hợp 3.} Nếu $4(a+b)=3\tron{a+b^3},$ ta có $a=3b^3-4b,$ và lúc này  
        $$2^y=b^3+a=4b^3-4b=4b(b-1)(b+1).$$ 
        Rõ ràng, không tồn tại số $b$ nào để cho cả $b,b-1$ và $b+1$ đều là lũy thừa của $3.$     
        \item\chu{Trường hợp 4.} Nếu $4(a+b)=4\tron{a+b^3},$ ta có $b=1.$ Ta không tìm ra $a,c$ từ đây.     
    \end{itemize}
\end{enumerate}
Kết luận, tất cả các bộ $(a,b,c)$ thỏa mãn đề bài là $(1,1,2),(5,3,2)$ và $(3,5,2).$}
\end{gbtt}

\begin{gbtt}
Tìm tất cả các bộ ba số nguyên dương $(a,b,c)$ thỏa mãn \[\tron{a^5+b}\tron{a+b^5}=2^c.\]
\loigiai{
Không mất tính tổng quát, ta giả sử $a\ge b\ge 1.$ Điều này cho ta $a^5+b\ge a+b^5\ge 2.$\\
Từ phương trình đã cho, ta suy ra $a,b$ cùng tính chẵn lẻ. Khi đó, tồn tại số nguyên dương $m\ge n$ thỏa mãn
\[a^5+b=2^m,\qquad a+b^5=2^n.\tag{*}\label{1515115}\]
Cộng theo vế hai phương trình trên, ta được
$$a^5+b^5+a+b=2^m+2^n.$$
Biến đổi tương đương cho ta
\[\tron{a+b}\tron{a^4-a^3b+a^2b^2-ab^3+b^4+1}=2^n\tron{2^{m-n}+1}.\tag{**}\label{123333}\]
Tới đây, ta xét các trường hợp sau.
\begin{enumerate}
    \item Với $a,b$ cùng là số chẵn, ta có $a^4-a^3b+a^2b^2-ab^3+b^4+1$ là số lẻ.\\
    Từ đây và (\ref{123333}), ta suy ra $2^n\mid (a+b).$ Lại có
    $$0<a+b\le a+b^5=2^n.$$
    Do đó $a+b=2^n$ hay $b=b^5.$ Điều này dẫn tới $b=1,$ mâu thuẫn với $b$ là số chẵn.
    \item Với $a,b$ cùng là số lẻ, ta suy ra $a^4-a^3b+a^2b^2-ab^3+b^4+1$ chia $4$ dư $2.$\\
    Từ đây và (\ref{123333}), ta suy ra $2^{n-1}\mid (a+b).$ Mặt khác, ta luôn có
    $$0<a+b\le a+b^5=2^n.$$
    Do đó $a+b=2^n$ hoặc $a+b=2^{n-1}.$ Ta xét tiếp tới các trường hợp nhỏ hơn sau.
    \begin{itemize}
        \item\chu{Trường hợp 1.} Với $a+b=2^n,$ kết hợp với $a+b^5=2^n,$ ta có
        $$a+b=a+b^5\Leftrightarrow b^5=b\Leftrightarrow b(b-1)(b+1)\tron{b^2+1}=0.$$
        Do $b>0$ nên $b=1.$ Thế trở lại (\ref{1515115}) cho ta
        $$a^5+1=2^m,\quad a+1=2^n.$$
        Lấy thương theo vế hai phương trình, ta được
        $$a^4-a^3+a^2-a+1=2^{m-n}.$$
        Vì $a$ là số lẻ  nên $a^4-a^3+a^2-a+1=2^{m-n}$ là số lẻ. Kéo theo $m=n.$ Từ đó ta được 
        $$a^5+1=a+1\Leftrightarrow a^5=a\Leftrightarrow a(a-1)(a+1)\tron{a^2+1}=0.$$
        Do $a>0$ nên $a=1.$ Thay $a=b=1$ vào phương trình đã cho, ta tìm được $c=2.$
        \item\chu{Trường hợp 2.} Với $a=b=2^{n-1},$ kết hợp với $a+b^5=2^n,$ ta có
        $$b^5-b=2^n-2^{n-1}\Leftrightarrow b\tron{b^4-1}=2^{n-1}.$$
        Ta có $b$ là luỹ thừa của $2.$ Do $b$ lẻ nên $b=1,$ nhưng khi đó $2^{n-1}=0,$ vô lí.
    \end{itemize}
\end{enumerate}
Kết luận, có duy nhất bộ số nguyên dương thỏa mãn đề bài là $(a,b,c)=(1,1,2).$}
\end{gbtt}


\begin{gbtt}
Giải phương trình nghiệm tự nhiên \[n^x+n^y=n^z.\]
\loigiai{
Không mất tính tổng quát, giả sử $x\le y.$ Rõ ràng $x\le y<z.$ Phương trình đã cho tương đương với
\[1+n^{y-x}=n^{z-x}.\]
Tới đây, ta xét các trường hợp sau.
\begin{enumerate}
    \item Nếu $y-x>0,$ xét tính chia hết cho $n$ ở hai vế ta có $n=1.$ Thử lại, ta thấy vô lí.
    \item Nếu $y=x,$ thế vào phương trình ta được $n^z=2,$ và khi đó $n=2,\, z=1.$
\end{enumerate}
Vậy tất cả các nghiệm của phương trình có dạng
$$(n,x,y,z)=(2,k,k,k+1),$$ 
trong đó $k$ là số tự nhiên tùy ý.}
\end{gbtt}

\begin{gbtt}
Tìm các số nguyên dương $x$ và $y$ khác nhau sao cho
\[x^y=y^x.\]
\loigiai{
Trước hết, ta sẽ đi chứng minh bổ đề sau
\begin{light}
\chu{Bổ đề.} Cho các số nguyên dương $a,b,c,d$ thỏa mãn $a^c=b^d.$ \\Lúc này $c\le d$ khi và chỉ khi $a$ chia hết cho $b.$
\end{light}
\chu{Chứng minh.}\\ 
\chu{Chiều thuận.} Nếu $a$ chia hết cho $b,$ ta lần lượt suy ra
    $$a\ge b\Rightarrow a^c\ge b^c\Rightarrow b^d\ge b^c\Rightarrow d\ge c.$$
\chu{Chiều đảo.} Nếu $c\le d,$ ta lần lượt suy ra
$$b^c\mid b^d\Rightarrow b^c\mid a^c\Rightarrow b\mid a.$$
Như vậy, bổ đề được chứng minh.\\
\chu{Quay lại bài toán.}\\
Trong bài toán này, không mất tổng quát, ta giả sử $x\ge y.$ Khi đó, theo bổ đề đã biết, $x$ chia hết cho $y.$ \\Đặt $x=ky,$ và phương trình đã cho trở thành
$$(ky)^y=y^{ky}\Leftrightarrow (ky)^y=\tron{y^k}^y\Leftrightarrow ky=y^k\Leftrightarrow k=y^{k-1}.$$
Bằng quy nạp, ta chứng minh được rằng
$$y^{k-1}>k,\text{ với mọi }y\ge 2\text{ và }k\ge 3.$$
Dựa vào lập luận trên, ta chỉ cần kiểm tra trực tiếp các cặp $(y,k)$ sau
    $$(1,k),\ (2,1),\ (2,2),\ (2,3).$$
Chỉ có các trường hợp $(y,k)=(1,1),(2,1),(2,2)$ là thỏa mãn $k=y^{k-1}.$ \\Thử lại, ta thu $(2,4)$ và $(4,2)$ là các cặp số thỏa yêu cầu.}
\end{gbtt}

\begin{gbtt}
Giải phương trình nghiệm nguyên dương
$$x^y=y^{x-y}.$$
\nguon{Junior Balkan Mathematical Olympiad 1998}
\loigiai{
Từ bổ đề đã phát biểu ở bài toán trên, ta xét các trường hợp sau.
\begin{enumerate}
    \item Nếu $y\ge x-y$ hay $2y\ge x,$ ta có $y$ chia hết cho $x.$ Đặt $y=kx.$ Phương trình đã cho trở thành
    $$x^{kx}=(kx)^{x-kx}\Leftrightarrow \tron{x^k}^x=\tron{(kx)^{1-k}}^x\Leftrightarrow x^k=(kx)^{1-k}.$$
    Vế phải phương trình là số nguyên dương chỉ khi $k=1,$ và khi ấy $x=y.$ Thế trở lại phương trình, ta được $x^x=1.$ Trường hợp này cho ta $(x,y)=(1,1).$
    \item Nếu $y\le x-y$ hay $2y\le x,$ ta có $x$ chia hết cho $y.$ Đặt $x=ly.$ Phương trình đã cho trở thành
    $$\tron{ly}^y=y^{ly-y}\Leftrightarrow \tron{ly}^y=\tron{y^{l-1}}^y\Leftrightarrow ly=y^{l-1}\Leftrightarrow l=y^{l-2}.$$
    Bằng quy nạp, ta chứng minh được rằng
    $$y^{l-2}>l,\text{ với mọi }y\ge 2\text{ và }l\ge 5.$$
    Dựa vào lập luận trên, ta chỉ cần kiểm tra trực tiếp các cặp $(y,l)$ sau
    $$(1,l),\ (2,1),\ (2,2),\ (2,3), \ (2,4).$$
    Chỉ có các trường hợp $(y,l)=(1,1),(2,4)$ là thỏa mãn $l=y^{l-2}.$ Thử lại, ta có $(x,y)=(8,2).$
\end{enumerate}
Như vậy, phương trình đã cho có hai nghiệm nguyên dương là $(1,1)$ và $(8,2).$}
\end{gbtt}


\begin{gbtt}
Phương trình sau có bao nhiêu nghiệm nguyên dương?
\[\tron{x^y-1}\tron{z^t-1}=2^{200}.\]
\loigiai{
Từ phương trình đã cho, ta suy ra $x,z$ không thể cùng chẵn. Ta đặt 
$$x^y-1=2^m,\quad  z^t-1=2^n$$
với $m,n$ là các số tự nhiên. Ta xét các trường hợp sau.
\begin{enumerate}
    \item Nếu $x$ là số chẵn, ta có $2^m$ là số lẻ nên $m=0,$ từ đây ta suy ra $x=2,y=1.$\\
    Ngoài ra, ta còn có $z^t-1=2^{200},$ kéo theo $z>1$ và $z$ lẻ số lẻ. 
    \begin{itemize}
        \item\chu{Trường hợp 1.1.} Với $t\ge 3$ là số lẻ, ta có 
        $$z^t-1=\tron{z-1}\tron{z^{t-1}+\cdots+1}.$$
        Dễ dàng chỉ ra $z^{t-1}+\cdots+1$ là số lẻ, vô lí.
        \item\chu{Trường hợp 1.2.} Với $t$ chẵn, đặt $t=2a.$ Ta có
        $$\tron{z^a-2^{100}}\tron{z+2^{100}}=1.$$
        Ta suy ra $z^a-2^{100}=z+2^{100}=1,$ vô lí.
        \item\chu{Trường hợp 1.3.} Với $t=1,$ ta nhận được $z=2^{200}+1.$
    \end{itemize}
    \item Nếu $z$ là số chẵn, ta làm tương tự trường hợp trên để nhận được
    $$\tron{x,y,z,t}=\tron{2^{200}+1,1,2,1}$$
    \item Nếu $x$ và $z$ là số lẻ, ta lại xét tiếp tới các trường hợp sau.
    \begin{itemize}
        \item \chu{Trường hợp 3.1.} Nếu $y$ là số chẵn, ta có 
        $$\tron{x^{y/2}-1}\tron{x^{y/2}+1}=2^m.$$
        Ta suy ra cả $x^{y/2}-1$ và $x^{y/2}+1$ đều là lũy thừa của $2.$ Do đây là hai số chẵn liên tiếp nên
        $$x^{y/2}-1=2,\quad x^{y/2}+1=4.$$
        Từ đây ta có $x=3,y=2,$ thế trở lại thì ta có
        $$z^t-1=2^{198}.$$
        Đến đây, ta xử lí tương tự \chu{trường hợp 1.1} để chỉ ra $z=2^{198}+1$ và $t=1.$
        \item \chu{Trường hợp 3.2.} Nếu $t$ là số chẵn, ta lập luận tương tự trường hợp trước để chỉ ra 
        $$\tron{x,y,z,t}=\tron{2^{198}+1,1,3,2}.$$
        \item \chu{Trường hợp 3.3.} Nếu $y$ và $t$ là số lẻ, ta xử lí tương tự \chu{trường hợp 1.1} để chỉ ra
        $$\tron{x,y,z,t}=\tron{2^n+1,1,2^{200-n},1},$$
        trong đó $n$ là số nguyên tùy ý thuộc $\{1;2;\ldots;199\}.$
    \end{itemize}
\end{enumerate}
Như vậy, phương trình đã cho có $203$ nghiệm số nguyên dương.}
\end{gbtt}

\subsection{Phương pháp lựa chọn modulo}
\subsubsection*{Ví dụ minh họa}
\begin{bx}
Giải phương trình nghiệm tự nhiên $2^x-5^y=1.$
\loigiai{
Giả sử phương trình đã cho có nghiệm tự nhiên $(x,y).$ Ta xét các trường hợp sau đây.
\begin{enumerate}
    \item Với $x>1,$ ta có $2^x$ chia hết cho $4.$ Lúc này
    $$5^y\equiv 2^x-1\equiv -1\pmod{4},$$
    vô lí do $5^y=(4+1)^y\equiv 1\pmod{4}.$
    \item Với $x=1,$ kiểm tra trực tiếp, ta tìm ra $y=0.$
    \item Với $x=0,$ kiểm tra trực tiếp, ta không tìm được $y$ tự nhiên do lúc này $5^y=0.$
\end{enumerate}
Kết luận, $(x,y)=(1,0)$ là nghiệm tự nhiên duy nhất của phương trình đã cho.}
\begin{luuy}
Việc dự đoán được nghiệm $(x,y)=(1,0)$ trong bài toán này là vô cùng quan trọng. Dự đoán kể trên cho phép ta nghĩ đến việc sử dụng đồng dư thức $2^x\equiv 0\pmod{4}$ với $x>1$ để chứng minh rằng không tìm được $y$ ở trong trường hợp ấy.
\end{luuy}
\end{bx} 

\begin{bx}
Giải phương trình nghiệm tự nhiên $2^x+3=y^2.$
\loigiai{
Ta xét các trường hợp sau đây.
\begin{enumerate}
    \item Nếu $x \geq 2$ thì $2^x$ chia hết cho $4$, và vế trái chia $4$ dư $3$, còn  $y$ lẻ nên vế phải chia $4$ dư $1$, mâu thuẫn.
    \item Nếu $x=1$ thì $y^2=5,$ kéo theo $y$ vô tỉ, mâu thuẫn.
    \item Nếu $x=0,$ ta tìm được $y=2.$
\end{enumerate}
Kết luận, $(x,y)=(0,2)$ là nghiệm tự nhiên duy nhất của phương trình.}
\end{bx}

\begin{bx}\label{bai2mu}
Giải phương trình nghiệm tự nhiên $3^x-2^y=1.$
\loigiai{Giả sử phương trình đã cho có nghiệm tự nhiên $(x,y)$.\\
Với $y\le 1,$ bằng kiểm tra trực tiếp, ta tìm ra $(x,y)=(1,1).$\\
Với $y>1,$ ta có $2^y$ chia hết cho $4.$ Ta xét các trường hợp sau đây.
\begin{enumerate}
	\item Nếu $x$ lẻ, ta đặt $x=2k+1.$ Phép đặt này cho ta 
	$$3^x=3\cdot 9^k\equiv 3\pmod 4.$$ 
	Căn cứ vào đây, ta suy ra $2^y=3^x-1\equiv 2\pmod 4,$ một điều vô lí.
	\item Nếu $x$ chẵn, ta đặt $x=2k.$ Phép đặt này cho ta  $$2^y=\left(3^k-1\right)\left(3^k+1\right).$$ 
    Nhờ vào biến đổi trên, ta nhận thấy cả $3^k-1$ và $3^k+1$ đều là lũy thừa số mũ tự nhiên của $2.$ Ta đặt
    $$3^k-1=2^u,\quad 3^k+1=2^v,$$
    trong đó $0\le u<v$ và $u+v=y.$ Lấy hiệu theo vế, ta thu được
    $$2=2^v-2^u=2^u\tron{2^{v-u}-1}.$$ 
    So sánh số mũ của $2$ ở các vế, ta tìm ra $u=1,$ kéo theo $k=1,y=3$ và $x=2.$
\end{enumerate}
Như vậy, phương trình đã cho có hai nghiệm là $(1,1)$ và $(2,3).$}
\begin{luuy}
\begin{enumerate}
    \item Việc chứng minh $x$ là số chẵn phía trên cho phép ta tạo ra các nhân tử $3^k-1$ và $3^k+1,$ để từ đó tiến hành xét hiệu hai vế. Loạt bài toán dưới đây là một vài ví dụ điển hình.
    \item Ngoài cách xét hiệu hai vế sau bước đặt
        $$3^k-1=2^u,\quad 3^k+1=2^v,$$
    chúng ta còn có thể tiến hành bài toán bằng cách khác. Cụ thể, hai số chẵn $3^k-1$ và $3^k+1$ không thể cùng chia hết cho $4,$ thế nên một trong hai số ấy bằng $2.$ Lần lượt xét các trường hợp $3^k-1=2$ và $3^k+1=2,$ ta sẽ chỉ ra $k,x,y$ thỏa yêu cầu.
\end{enumerate}
\end{luuy}
\end{bx}

\begin{bx} \label{bai3mu}
Giải phương trình nghiệm tự nhiên $3^x+4^y=5^z.$
\loigiai{Giả sử phương trình đã cho có nghiệm tự nhiên $(x,y,z)$. Ta chia bài toán thành các trường hợp sau đây.
\begin{enumerate}
    \item Với $x=0,$ phương trình đã cho trở thành 
    \[1+4^y=5^z.\tag{*}\label{3,4,5,chau}\]
    Ta xét tới các trường hợp nhỏ hơn sau.
    \begin{itemize}
        \item Với $y=0$ hoặc $y=1,$ bằng kiểm tra trực tiếp, ta tìm được $z=1$ khi $y=1.$
        \item Với $y>1$, khi đó $4^y$ chia hết cho $8.$ Lấy đồng dư modulo $8$ hai vế của (\ref{3,4,5,chau}), ta được
        $$1\equiv 5^z\pmod{8}.$$
        Kết quả quen thuộc ở đây cho ta $z$ là số chẵn. Ta đặt $z=2k.$ Lúc này
        $$1+4^y=5^{2k}\Leftrightarrow \left(5^k-2^y\right)\left(5^k+2^y\right)=1.$$
        Bắt buộc, $5^k-2^y=5^k+2^y=1.$ Đây là điều không thể nào xảy ra.
    \end{itemize}
    \item Với $x\ge 0$ khi đó $3^x$ chia hết cho 3. Lấy đồng dư theo modulo $3$ hai vế phương trình đã cho, ta được
    $$5^z\equiv 3^x+4^y\equiv 1+0\equiv 1\pmod 3.$$ 
    Do đó, $z$ là số chẵn. Ta đặt $z=2k.$ Phương trình đã cho trở thành $$3^x=5^{2k}-4^y\Leftrightarrow 3^x=\left(5^k-2^y\right)\left(5^k+2^y\right).$$
    Cả $5^k-2^y$ và $5^k+2^y$ đều lừa lũy thừa số mũ tự nhiên của $3.$ Chính vì thế, ta có thể đặt
    $$5^k-2^y=3^u,\quad 5^k+2^y=3^v,$$
    trong đó $0\le u<v$ và $u+v=x.$ Lấy hiệu theo vế, ta được
    $$2^{y+1}=3^v-3^u=3^u\tron{3^{v-u}-1}.$$
    So sánh số mũ của $3$ ở hai vế, ta chỉ ra $u=0,$ và lúc này
    $$2^{y+1}=3^v-1.$$
    Theo như lời giải của \chu{ví dụ \ref{bai2mu}}, ta có $v=1,y=0$ hoặc $v=2,y=2.$ 
\begin{itemize}
	\item Với $v=1$ và $y=0$, ta có $5^k=3^v-2^y=2$, mâu thuẫn.
	\item Với $v=2$ và $y=2$, ta có $5^k=5$ nên $k=1$. Ta tìm ra $(x,y,z)=(2,2,2)$.
\end{itemize}
\end{enumerate}
Như vậy, phương trình đã cho có hai nghiệm tự nhiên là $(0,1,1),(2,2,2).$}
\end{bx}

\subsubsection*{Bài tập tự luyện}
\begin{btt}
Giải phương trình nghiệm tự nhiên \[9^x+1=2^y.\]
\end{btt}

\begin{btt}
Giải phương trình nghiệm tự nhiên \[5\cdot 3^x+11=4^y.\]
\end{btt}

\begin{btt}
Tìm tất cả các cặp số tự nhiên $(x,y)$ thỏa mãn $$5\cdot 3^x+11=2^y.$$
\end{btt}

\begin{btt}
Giải phương trình nghiệm tự nhiên $$2^x-7^y=1.$$
\end{btt}

\begin{btt}
Tìm các số $x,y$ nguyên dương thỏa mãn $$3^x+29=2^y.$$
\nguon{Chuyên Khoa học Tự nhiên 2021}
\end{btt}

\begin{btt}
Giải phương trình nghiệm tự nhiên $$5^x+48=y^2.$$ 
\end{btt}

\begin{btt}
Giải phương trình nghiệm nguyên $$2^x-1=y^2.$$  
\end{btt}

\begin{btt}
Tìm tất cả các cặp số tự nhiên $x,y$ sao cho 
\[x^{3}=1993 \cdot 3^{y}+2021.\]
\nguon{Chuyên Toán Nghệ An 2021}
\end{btt}

\begin{btt}
Tìm tất cả các số tự nhiên $x,y,z$ thoả mãn
\[3^x+5^y-2^z=\left(2z+3\right)^3.\]
\nguon{Tạp chí Toán học và Tuổi trẻ số 510, tháng 12 năm 2019}
\end{btt}

\begin{btt}
Giải phương trình nghiệm nguyên 
$$x^2y^5-2^x5^y=2015+4xy.$$
\nguon{Saudi Arabia Mathematical Olympiad 2015}
\end{btt}

\begin{btt}
Tìm tất cả các bộ số nguyên $(a, b, c, d)$ sao cho
\[a^{2}+35=5^{b}6^{c}7^{d}.\]
\nguon{Đề thi chọn đội tuyển học sinh giỏi quốc gia 2017 $-$ 2018 Tỉnh Đắk Lắk}
\end{btt}

\begin{btt}
Giải phương trình nghiệm nguyên dương $$2^x-3^y=1.$$
\end{btt}

\begin{btt}
Giải phương trình nghiệm nguyên dương $$5^x-2^y=9.$$
\end{btt}

\begin{btt}
Giải phương trình nghiệm nguyên dương $$3^x-2^y=5.$$
\end{btt}

\begin{btt}
Giải phương trình nghiệm tự nhiên $$2^x-7^y=1.$$
\end{btt}

\begin{btt}
Tìm tất cả các cặp số nguyên dương $\left ( m,n \right )$ thỏa mãn phương trình
\[125\cdot 2^n-3^m=271.\]
\nguon{Junior Balkan Mathematical Olympiad 2018 Shortlist}
\end{btt}

\begin{btt}
Giải phương trình nghiệm nguyên dương $$7^x+3^y=2^z.$$
\end{btt}

\begin{btt}
Giải phương trình nghiệm nguyên dương $$2^x+3^y=5^z.$$
\end{btt}

\begin{btt}
Giải phương trình nghiệm tự nhiên $$2^x+5^y=7^z.$$
\end{btt}

\begin{btt}
Giải phương trình nghiệm nguyên dương $$5^x+12^y=13^z.$$
\end{btt}

\begin{btt}
Tìm tất cả các số nguyên dương $x,y,z$ thỏa mãn $$3^x+2^y=1+2^z.$$
\nguon{Chọn học sinh giỏi thành phố Hà Nội 2021}
\end{btt}

\begin{btt}
Tìm tất cả các bộ số tự nhiên $(x,y,z)$ thỏa mãn $$2^x+3^y=z^2.$$
\end{btt}

\begin{btt}
Tìm tất cả các số nguyên dương $m,n$ thỏa mãn
\[10^n-6^m=4n^2.\]
\nguon{Tigran Akopyan}
\end{btt}

\begin{btt}
Giải phương trình nghiệm tự nhiên $$2^x+7^y=z^3.$$
\end{btt}

\begin{btt}
Giải phương trình nghiệm tự nhiên $$2^xx^2=9y^2+6 y+16.$$
\end{btt}


\subsubsection*{Hướng dẫn bài tập tự luyện}

\begin{gbtt}
Giải phương trình nghiệm tự nhiên \[9^x+1=2^y.\]
\loigiai{
Giả sử phương trình đã cho có nghiệm tự nhiên $(x,y).$ Ta nhận thấy rằng
$$2^y=9^x+1=(8+1)^x+1\equiv 1+1\equiv 2\pmod{8}.$$
Chỉ có $y=1$ thỏa mãn đồng dư thức trên. \\
Thế trở lại, ta kết luận $(x,y)=(0,1)$ là nghiệm tự nhiên của phương trình.}
\end{gbtt} 

\begin{gbtt}\label{bai1mu}
Giải phương trình nghiệm tự nhiên 
\[5\cdot 3^x+11=4^y.\]
\loigiai{Giả sử phương trình đã cho có nghiệm tự nhiên $(x,y).$ Ta xét các trường hợp sau đây.
\begin{enumerate}
    \item  Với $x>0$, ta có $3^x$ chia hết cho $3.$ Lúc này 
    $$4^y\equiv 11+5\cdot3^x\equiv 11\equiv 2\pmod{3},$$
    vô lí do $4^y=(3+1)^y\equiv 1\pmod{3}.$
    \item Với $x=0,$ kiểm tra trực tiếp, ta tìm ra $y=2.$    
\end{enumerate}
Kết luận, $(x,y)=(0,2)$ là nghiệm tự nhiên duy nhất của phương trình đã cho.}
\end{gbtt}

\begin{gbtt}
Tìm tất cả các cặp số tự nhiên $(x,y)$ thỏa mãn  \[5\cdot 3^x+11=2^y.\]
\loigiai{Giả sử phương trình đã cho có nghiệm tự nhiên $(x,y).$ Ta xét các trường hợp sau đây.
\begin{enumerate}
    \item Nếu $ y$ lẻ, đặt $y=2k+1,$ ở đây $k$ là số nguyên dương. Ta có
    \[2^y=2^{2k+1}=2\cdot 4^k\equiv \pm 2\pmod 5.\]
    Tuy nhiên, do $\ddu{5\cdot 3^x+11}{1}{5}$ nên ta suy ra $\ddu{1}{\pm2}{5},$ một điều vô lí.
    \item Nếu $y$ chẵn, áp dụng \chu{bài \ref{bai1mu}}, ta tìm được ta tìm được $x=0,z=2$, và thế thì $y=4.$
\end{enumerate}
Kết luận, $(x,y)=(0,4)$ là nghiệm tự nhiên duy nhất của phương trình đã cho.}
\end{gbtt}

\begin{gbtt}
Giải phương trình nghiệm tự nhiên \[2^x-7^y=1.\]
\loigiai{Giả sử phương trình đã cho có nghiệm tự nhiên $(x,y).$ Ta xét các trường hợp sau đây.
\begin{enumerate}
    \item Với $x\ge 4$, ta có $2^x$ chia hết cho $16.$
    \begin{itemize}
        \item\chu{Trường hợp 1.} Nếu $y$ chẵn, ta đặt $y=2k.$ Phép đặt này cho ta $$7^y+1=7^{2k}+1=49^k+1\equiv 2\pmod{16}.$$ 
        Căn cứ vào đây, ta suy ra $0\equiv 2^x\equiv 7^y+1\equiv 2\pmod{16},$ một điều vô lí.
        \item\chu{Trường hợp 2.} Nếu $y$ lẻ, ta đặt $y=2k+1.$ Phép đặt này cho ta $$7^y+1=7^{2k+1}+1=7\cdot49^k+1\equiv 8\pmod{16}.$$ 
        Căn cứ vào đây, ta suy ra $0\equiv 2^x\equiv 7^y+1\equiv 8\pmod{16},$ một điều vô lí.
    \end{itemize}
    \item Với $x\in\{0;1;2;3\},$ bằng kiểm tra trực tiếp, ta tìm ra $y=1$ khi $x=3$ và $y=0$ khi $x=1.$
\end{enumerate}
Như vậy, phương trình đã cho có hai nghiệm tự nhiên là $(1,0)$ và $(3,1).$}
\end{gbtt}

\begin{gbtt}
Tìm các số $x,y$ nguyên dương thỏa mãn  \[3^x+29=2^y.\]
\nguon{Chuyên Khoa học Tự nhiên 2021}
\loigiai{
Giả sử tồn tại các số nguyên dương $x,y$ thỏa mãn yêu cầu bài toán.\\
    Với $x=1,$ kiểm tra trực tiếp, ta nhận được $y=5.$ Với $x\ge 2,$ ta có
    $$3^x+29\equiv 2\pmod{9}\Rightarrow 2^y\equiv 2\pmod{9}.$$
    Xét các số dư khi chia cho $6$ của $y,$ ta nhận được $y\equiv 1\pmod{6}.$ \\Bằng cách đặt $y=6z+1$ (trong đó $z$ là số tự nhiên), ta chỉ ra
    $$2^y=2^{6z+1}=2\cdot64^z\equiv 2\pmod{7}.$$
    Căn cứ vào đẳng thức $3^x+29=2^y,$ ta tiếp tục suy ra
    $$3^x+29\equiv 2\pmod{7}\Rightarrow 3^x\equiv 1\pmod{7}.$$
    Xét các số dư khi chia cho $6$ của $x,$ ta nhận được $x\equiv 0\pmod{6}.$ \\Tiếp tục đặt $x=6t$ (trong đó $t$ là số tự nhiên), ta chỉ ra 
    $$3^x+29=3^{6y}+29=729^t+29\equiv 6\pmod{8}.$$
    Ta được $2^y\equiv6\pmod{8}$ từ đây, là điều không thể xảy ra. \\
    Kết luận, $(x,y)=(1,5)$ là cặp số nguyên dương duy nhất thỏa yêu cầu.}
\end{gbtt}

\begin{gbtt}
Giải phương trình nghiệm tự nhiên  \[5^x+48=y^2.\]
\loigiai{
Ta xét các trường hợp sau đây.
\begin{enumerate}
    \item Nếu $x \geq 1$ thì $5^x+48$ chia cho $5$ được dư là $3,$ trong khi đó $y^2\equiv 0,1,4\pmod{5},$ mâu thuẫn.
    \item Nếu $x=0,$ ta tìm được $y=7.$
\end{enumerate}
Kết luận, $(x,y)=(0,7)$ là nghiệm tự nhiên duy nhất của phương trình.}
\end{gbtt}

\begin{gbtt}
Giải phương trình nghiệm nguyên \[2^x-1=y^2.\]
\loigiai{
Nếu như $x\le -1,$ ta có $y^2=2^x-1$ không phải là số nguyên, do
$$-1<2^x-1\le 2^{-1}-1=-\dfrac{1}{2}.$$
Nếu như $x$ là số tự nhiên, ta xét các trường hợp sau đây
\begin{enumerate}
    \item Nếu $x\ge 3$ thì $2^x-1$ chia cho $8$ dư $7,$ nhưng $y^2$ khi chia cho $8$ chỉ có thể dư $0,1,4,$ mâu thuẫn.
    \item Nếu $x=2,$ ta tìm được $y=\pm \sqrt{3}$ không là số nguyên.
    \item Nếu $x=1,$ ta tìm được $y=\pm 1.$
    \item Nếu $x=0,$ ta tìm được $y=0.$
\end{enumerate}
Kết luận, phương trình đã cho có ba nghiệm nguyên là $(0,0),\ (1,-1)$ và $(1,1).$}
\end{gbtt}

\begin{gbtt}
Tìm tất cả các cặp số tự nhiên $x,y$ sao cho 
\[x^{3}=1993 \cdot 3^{y}+2021.\]
\nguon{Chuyên Toán Nghệ An 2021}
\loigiai{
 Dựa vào tính chất đã biết $x^3\equiv 0,1,8 \pmod{9},$ ta có các đánh giá
    $$1993\cdot 3^y+2021 \equiv 0,1,8 \pmod{9}\Rightarrow 4\cdot 3^y\equiv 3,4,5 \pmod{9}.$$
    Ta xét các trường hợp kể trên.
\begin{enumerate}
    \item Nếu $4\cdot 3^y\equiv 3\pmod{9},$ ta có $y=1.$ Thay ngược lại, ta tìm ra $x=20.$
    \item Nếu $4\cdot 3^y\equiv 4\pmod{9},$ ta có $y=0.$ Thay ngược lại, ta tìm ra $x=\sqrt[3]{4041}$ không là số nguyên.     
    \item Nếu $4\cdot 3^y\equiv 5\pmod{9},$ ta không tìm được $y$ nguyên dương thỏa mãn.  
\end{enumerate}
Như vậy, cặp số tự nhiên $(x,y)$ duy nhất thỏa mãn là $(x,y) =(20,1).$}
\end{gbtt}

\begin{gbtt}
Tìm tất cả các số tự nhiên $x,y,z$ thoả mãn
\[3^x+5^y-2^z=\left(2z+3\right)^3.\]
\nguon{Tạp chí Toán học và Tuổi trẻ số 510, tháng 12 năm 2019}
\loigiai{
Ta thấy $3^x,$ $5^y$ và $2z+3$ đều là những số nguyên lẻ, do đó 
$$2^z=3^x+5^y-(2z+3)^3$$
là số lẻ, và ta suy ra $z=0.$ Thay trở lại phương trình, ta có
$$3^x+5^y=28.$$
Ta có $5^y<28,$ hay là $y\le 2.$ Xét ba trường hợp sau.
		\begin{enumerate}
			\item Nếu $y=0$ thì $5^{y}=1$ và $3^{x}=27,$ suy ra $x=3.$
			\item Nếu $y=1$ thì $5^{y}=5$ và $3^{x}=23,$ không có số tự nhiên $x$ nào thỏa mãn.
			\item Nếu $y=2$ thì $5^{y}=25$ và $3^{x}=3,$ suy ra $x=1.$
		\end{enumerate}
Vậy có hai bộ số tự nhiên $(x,y,z)$ thỏa mãn yêu cầu bài toán là $(3,0,0)$ và $(1,2,0).$}
\end{gbtt}

\begin{gbtt}
Giải phương trình nghiệm nguyên 
$$x^2y^5-2^x5^y=2015+4xy.$$
\nguon{Saudi Arabia Mathematical Olympiad 2015}
\loigiai{Nếu $x<0$ hoặc $y<0$ thì $x^2y^5-2015-4xy$ nguyên nhưng $2^x5^y$ không là số nguyên do
$$0<2^x5^y\le \dfrac{1}{2}\cdot\dfrac{1}{5}=\dfrac{1}{10}.$$
Nếu $x=0$ hoặc $y=0,$ phương trình cũng không có nghiệm. Vì thế nên $x, y$ là các số nguyên dương. Trước hết, ta có một vài nhận xét sau
\begin{enumerate}
    \item[i,] Vì $2^x5^y+2015+4xy$ là một số nguyên lẻ nên $x, y$ phải là các số nguyên lẻ.
    \item[ii,] Vì khi lấy modulo $5$ hai vế phương trình đã cho, ta có
    $$x^2y\equiv 4xy \pmod{5}$$
    nên $x \equiv 0,4\pmod{5}$ hoặc $y \equiv 0\pmod{5}.$
    \item[iii,] Vì khi lấy modulo $8$ hai vế phương trình đã cho, ta có
    $$y-2^{x} \cdot 5 \equiv-1+4\Leftrightarrow y \equiv 5\left(2^{x}-1\right)\pmod 8 .$$
    nên có hai trường hợp là $x=1$ và $y \geq 5$ hoặc $x \geq 3$ và $y \equiv 3\pmod 8.$
\end{enumerate}
Ta sẽ xem xét các trường hợp kể trên.
\begin{enumerate}
    \item Nếu $x=1$ và $y \geq 5$ thì phương trình trở thành 
    $$y^{5}-2 \cdot 5^{y}=2015+4y.$$ và phương trình này vô nghiệm vì vế trái âm mà vế phải dương. 
    \\(Có thể chứng minh vế trái âm bằng phép quy nạp với $y \geq 5$).
    \item Nếu $x=3$ thì $y \equiv 3 \pmod 8 .$ Kết hợp với nhận xét $y \equiv 0 \pmod 5,$ ta chỉ ra $y \geq 35$ và $9 \cdot y^{5}-8 \cdot 5^{y}$ là số nguyên âm nên phương trình vô nghiệm.
    \item Nếu $x \geq 5$ và $y \neq 3$ thì $y \geq 5.$ Trong trường hợp này ta có $x^{2}<2^{x}$ và $y^{5} \leq 5^{y}$ (chứng  minh bằng phép quy nạp) vì thế nên $x^{2} y^{5}-2^{x} 5^{y}$ là số nguyên âm. Phương trình vô nghiệm.
    \item Nếu $x \geq 7$ và $y=3$ thì vì $x \equiv 0,4\pmod 5$ nên $x \geq 9 .$ Do đó $$x^{2} \cdot 3^{5}-2^{x} 5^{3}<3^{5}\left(x^{2}-2^{x-2}\right)$$
    là số nguyên âm. Phương trình đã cho vô nghiệm.
    \item Nếu $x=5$ và $y=3,$ thế trở lại, ta thấy thỏa mãn.
\end{enumerate}
Vậy nghiệm nguyên $(x,y)$ duy nhất của phương trình là $(5,3).$}
\begin{luuy}
Ở trong bài toán trên, ta đã áp dụng một ý tưởng xuất hiện trong các bài trước, đó là sử dụng đồng dư dùng để chặn $x,y.$ Từ phương trình, ta dễ dàng quan sát được rằng phương trình vô nghiệm với $x,y$ đủ lớn, vì vậy phép chặn trên giúp ta giảm bớt số trường hợp cần xét. 
\end{luuy}
\end{gbtt}

\begin{gbtt}
Tìm tất cả các bộ số nguyên $(a, b, c, d)$ sao cho
\[a^{2}+35=5^{b}6^{c}7^{d}.\]
\nguon{Đề thi chọn đội tuyển học sinh giỏi quốc gia 2017 $-$ 2018 Tỉnh Đắk Lắk}
\loigiai{ 
Giả sử tồn tại bộ số nguyên $(a,b,c,d)$ thỏa mãn yêu cầu. Vì $a^2+35$ là số nguyên nên
$$5^b6^c7^d\in \mathbb{Z}.$$
Trước hết, cả $b,c,d$ đều là số tự nhiên. Ta sẽ chứng minh rằng $b\le 1$ và $d\le 1.$ Thật vậy.
\begin{enumerate}
    \item[i,] Nếu $b \geq 2$ thì $25 \mid \left(a^{2}+35\right) \Rightarrow 5 \mid a \Rightarrow 25 \mid a^{2} \Rightarrow 25 \mid 35,$ vô lí.
    \item[ii,] Nếu $d \geq 2$ thì $49 \mid \left(a^{2}+35\right) \Rightarrow 7 \mid a \Rightarrow 49 \mid a^{2}\Rightarrow 49 \mid 35,$ vô lí.
\end{enumerate}
Do đó $b, d \in\{0 ; 1\}.$ Ta xét các trường hợp sau.
\begin{enumerate}
    \item Nếu $b=0$ và $d=1,$ thế vào phương trình ban đầu ta được
    $$a^{2}+35=7\cdot 6^{c}.$$
    Lấy đồng dư theo modulo $5$ hai vế, ta được
    $$a^2+35\equiv 0,1,-1\pmod{5}, 7\cdot 6^c\equiv 2\pmod{5}.$$
    Trường hợp này không xảy ra.
    \item Nếu $b=1$ và $d=0,$ thế vào phương trình ban đầu ta được $$a^{2}+35=5\cdot 6^{c}.$$ 
    Lấy đồng dư theo modulo $8$ hai vế, ta được
    $$5\cdot 6^c=a^2+35\equiv 3,4,7\pmod{8}.$$
    Nếu như $c\ge 3,$ đồng dư thức trên không xảy ra vì $5\cdot6^c$ chia hết cho $8.$ Vì thế $c\le 2.$ \\
    Thử trực tiếp $c=0,1,2,$ ta không tìm được $a$ nguyên tương ứng.
    \item Nếu $b=0$ và $d=0,$ thế vào phương trình ban đầu ta được
    $$a^2+35=6^c.$$
    Lấy đồng dư theo modulo $7$ hai vế, ta được
    $$6^c\equiv -1,1\pmod{7}, a^2+35\equiv 0,1,2,4\pmod{7}.$$
    Đối chiếu, ta chỉ ra $6^c\equiv 1\pmod{7}$ hay $c$ chẵn. Đặt $c=2k.$ Phương trình trở thành
    $$a^2+35=6^{2k}\Leftrightarrow \left(6^{k}-|a|\right)\left(6^{k}+|a|\right)=35.$$
    Do $0<6^k-|a|<6^k+|a|$ nên 
    $$\heva{6^k-|a|&=5 \\ 6^k+|a|&=7}\text{  hoặc }\heva{6^k-|a|&=1 \\ 6^k+|a|&=35.}$$
    Thử trực tiếp và thay trở lại, ta tìm ra $(a , b , c , d)=(\pm 1 , 0 , 2 , 0).$
    \item Nếu $b=1$ và $d=1,$ thế vào phương trình ban đầu ta được \[a^{2}+35=35\cdot 6^{c}.\tag{**}\label{huydago}\]
    Từ phương trình, ta suy ra $a$ chia hết cho $35.$ Đặt $a=35k.$ Phương trình (\ref{huydago}) trở thành
    $$(35k)^2+35=35\cdot 6^{c}\Leftrightarrow 35k^2+1=6^c.$$
    Lấy đồng dư theo modulo $7$ hai vế, ta được
    $$6^c=35k^2+1\equiv 1\pmod{7}.$$
    Như vậy $c$ chẵn. Đặt $c=2m,$ ta sẽ có
    $$35k^2+1=36^m.$$
    Lấy đồng dư theo modulo $8$ hai vế, ta được
    $$36^m=35k^2+1\equiv 3k^2+1\equiv 1,4,5\pmod{8}.$$
    Nếu như $m\ge 1$ thì $36^m$ chia hết cho $8,$ mâu thuẫn. Như vậy $m=1.$ Thế trở lại, ta tìm ra
$$(a, b , c , d)=(0 , 1 , 0 , 1) , (\pm 35 , 1 , 2 , 1).$$
\end{enumerate}
Kết luận, có $5$ bộ $(a , b , c , d)$ thỏa mãn yêu cầu bài toán là
$$(-1 , 0 , 2 , 0),\ (1 , 0 , 2 , 0),\ (0 , 1 , 0 , 1) ,\ (-35 , 1 , 2 , 1),\ (35 , 1 , 2 , 1).$$}
\end{gbtt}

\begin{gbtt}\label{bai4mu}
Giải phương trình nghiệm nguyên dương \[2^x-3^y=1.\]
\loigiai{Giả sử phương trình đã cho có nghiệm nguyên dương $(x,y)$.\\
Lấy đồng dư theo modulo $3$ hai vế, ta thu được
$$1\equiv 2^x-3^y\equiv (-1)^x\pmod{3}.$$
Lập luận trên dẫn đến $x$ là số chẵn. Ta đặt $x=2k.$ Phép đặt này cho ta
$$2^{2k}-1=3^y\Rightarrow \left(2^k-1\right)\left(2^k+1\right)=3^y.$$
Nhờ vào biến đổi trên, cả $2^k-1$ và $2^k+1$ đều là lũy thừa số mũ tự nhiên của $3.$ Ta đặt
    $$2^k-1=3^u,\quad 2^k+1=3^v,$$
    trong đó $0\le u<v$ và $u+v=y.$ Lấy hiệu theo vế, ta thu được
    $$2=3^v-3^u=3^u\tron{3^{v-u}-1}.$$ 
So sánh số mũ của $3$ ở các vế, ta tìm ra $u=0,$ kéo theo $k=1,x=2$ và $y=1.$\\
Như vậy, phương trình đã cho có duy nhất một nghiệm là $(x,y)=(2,1).$}
\end{gbtt}

\begin{gbtt}
Giải phương trình nghiệm nguyên dương \[5^x-2^y=9.\]
\loigiai{Giả sử phương trình đã cho có nghiệm nguyên dương $(x,y)$.\\
Với $y\le 2,$ bằng kiểm tra trực tiếp, ta thấy không thỏa mãn.\\
Với $y\ge 3$, ta có $2^y$ chia hết cho $8.$
\begin{enumerate}
\item Nếu $x$ lẻ, ta đặt $x=2k+1.$ Khi đó, $1\equiv 5^x\equiv 5\cdot 9^k\equiv 5\pmod 8,$ mâu thuẫn.
\item Nếu $x$ chẵn, ta đặt $x=2k.$ Khi đó, $$2^y=5^{2k}-9=\left(5^k-3\right)\left(5^k+3\right).$$ 
Dựa vào đây, ta chỉ ra cả $5^k-3$ và $5^k+3$ đều là lũy thừa số mũ tự nhiên của $2.$ Ta đặt
$$5^k-3=2^u,\quad 5^k+3=2^v,$$
trong đó $0\le u<v$ và $u+v=y.$ Lấy hiệu theo vế, ta được
$$6=2^v-2^u=2^u\tron{2^{v-u}-1}.$$
So sánh số mũ của $2$ ở các vế, ta tìm ra $u=1.$ \\
Với việc $u=1,$ ta tiếp tục nhận được $k=1,x=2,y=4.$
\end{enumerate}
Như vậy, phương trình đã cho có nghiệm duy nhất là $(x,y)=(2,4).$}
\end{gbtt}

\begin{gbtt}
Giải phương trình nghiệm nguyên dương \[3^x-2^y=5.\]
\loigiai{Giả sử phương trình đã cho có nghiệm nguyên dương $(x,y)$.\\
Với $y=1,$ ta không tìm được $x$ nguyên.\\
Với $y>1,$ ta thực hiện bài toán theo các bước làm sau đây.
\begin{enumerate}[\color{tuancolor}\bf\sffamily Bước 1.]
    \item Ta chứng minh $x$ chẵn.\\
    Do $y>1,$ ta có $2^y$ chia hết cho $4.$ Lấy đồng dư modulo $4$ hai vế phương trình ban đầu, ta được
    $$3^x\equiv 1\pmod{4}\Rightarrow (-1)^x\equiv 1\pmod{4}.$$
    Đồng dư thức trên cho ta biết $x$ là số chẵn.
    \item Ta chứng minh $y$ chẵn.\\
    Với việc $y>1,$ ta chỉ ra $x\ge 2.$ Lấy đồng dư modulo $3$ hai vế phương trình ban đầu, ta được
    $$-2^y\equiv -1\pmod{3}\Rightarrow (-1)^y\equiv 1\pmod{3}.$$
    Đồng dư thức trên cho ta biết $y$ là số chẵn.    
\end{enumerate}
Dựa vào hai chứng minh trên, ta có thể đặt $x=2u,y=2v,$ trong đó $u,v\in \mathbb{N}^*.$ Phương trình trở thành $$3^{2u}-2^{2v}=5\Leftrightarrow \left(3^u-2^v\right)\left(3^u+2^v\right)=5.$$ 
Nhận xét được $0<3^u-2^v<3^u+2^v$ cho ta
	$$3^u+2^v=5,\quad 3^u-2^v=1.$$
Lần lượt lấy tổng và hiệu hai vế, ta chỉ ra $3^u=3,2^v=2,$ thế nên $u=v=1$, và $x=y=2.$\\
Như vậy, phương trình đã cho có nghiệm nguyên dương duy nhất là $(x,y)=(2,2).$}
\begin{luuy}
Do $5$ chưa phải số chính phương, bài toán này không thể được hoàn thành nếu như chưa chứng minh được $y$ là số chẵn. Hướng đi chứng minh cả $x$ và $y$ là số chẵn này mở ra các cách làm mới cho loạt bài tập phương trình chứa ẩn ở mũ.
\end{luuy}
\end{gbtt}

\begin{gbtt}
Giải phương trình nghiệm tự nhiên \[2^x-7^y=1.\]
\loigiai{
Với $y=0,$ ta tìm ra $x=1.$ Với $y\ge 1,$ ta có $7^y$ chia hết cho $7.$ Ta xét các trường hợp sau đây.
\begin{enumerate}
    \item Nếu $x$ chia cho $3$ dư $1,$ ta đặt $x=3z+1.$\\ Lấy đồng dư theo modulo $7$ hai vế phương trình đã cho, ta được
    $$2^{3z+1}\equiv 1\pmod{7}\Rightarrow 2\cdot8^z\equiv 1\pmod{7}\Rightarrow 2\equiv1 \pmod{7},$$
    một điều không thể xảy ra.
    \item Nếu $x$ chia cho $3$ dư $2,$ ta đặt $x=3z+2.$\\ Lấy đồng dư theo modulo $7$ hai vế phương trình đã cho, ta được
    $$2^{3z+2}\equiv 1\pmod{7}\Rightarrow 4\cdot8^z\equiv 1\pmod{7}\Rightarrow 4\equiv1 \pmod{7},$$
    một điều không thể xảy ra.
    \item Nếu $x$ chia hết cho $3,$ ta đặt $x=3z.$ Phép đặt này cho ta
$$2^{3z}-7^y=1\Rightarrow \tron{2^z-1}\tron{2^{2z}+2^z+1}=7^y.$$
Lập luận trên chứng tỏ cả $2^z-1$ và $2^{2z}+2^z+1$ đều là lũy thừa của $7.$\\ Tiếp tục đặt $2^z-1=7^t,$ ta nhận thấy rằng
$$2^{2z}+2^z+1=\tron{7^t+1}^2+\tron{7^t+1}+1=7^{2t}+3\cdot7^t+3.$$
Số kể trên không thể chia hết cho $7$ nếu như $t\ge 1.$ Vì thế, nếu $2^{2z}+2^z+1$ là lũy thừa của $7,$ bắt buộc $t=0.$ Việc tìm ra $t$ này kéo theo $x=3,y=1.$
\end{enumerate}
Kết luận, phương trình đã cho có $2$ nghiệm tự nhiên $(x,y)$ là $(1,0)$ và $(3,1).$}
\begin{luuy}
Ngoài cách thế $2^z=7^t+1$ vào biểu thức $2^{2z}+2^z+1,$ độc giả còn có thể tiến hành bài toán theo cách xét ước chung lớn nhất của $2^z-1$ và $2^{2z}+2^z+1.$ Cách làm này sẽ được thể hiện ở một bài toán trong cùng tiểu mục phương trình chứa ẩn ở mũ.
\end{luuy}
\end{gbtt}
\begin{gbtt}
Tìm tất cả các cặp số nguyên dương $\left ( m,n \right )$ thỏa mãn phương trình
\[125\cdot 2^n-3^m=271.\]
\nguon{Junior Balkan Mathematical Olympiad 2018 Shortlist}
\loigiai{
Trước tiên, ta xem xét phương trình trong modulo $5$ thì thu được
\[3^m\equiv -1\pmod{5},\]
vì thế $m=4k+2$, trong đó $k$ là một số nguyên dương nào đó.\\
Tiếp theo, ta xét phương trình trong modulo $7$ thì thu được
\[ 2^n+2^{2k+1}\equiv 2\pmod{7}.\]
Đồng dư thức $2^s\equiv 1,2,4\pmod{7}$, tương ứng với $s\equiv 0,1,2\pmod{3}$. Do đó ta phải có 
$$3\mid n,\quad 3\mid (2k+1).$$
Đặt $m=3x,n=3y.$ Thế vào phương trình ban đầu ta được
\[5^3\cdot 2^{3x}-3^{3y}=271.\]
Phương trình bên trên tương đương với
\[\left ( 5\cdot 2^x-3^y \right )\left ( 25\cdot 2^{2x}+5\cdot 2^x\cdot 3^y+3^{2y} \right )=271.\]
Điều này dẫn đến $25\cdot 2^{2x}+5\cdot 2^x\cdot 3^y+3^{2y}\leqslant 271$, và do đó $25\cdot 2^{2x}\leqslant 271$, hay $x<2$, nghĩa là $x=1$ do $x$ là số nguyên dương. Thay trở lại, ta thu được $y=2$. Do đó ta suy ra phương trình chỉ có nghiệm duy nhất là $\left ( m,n \right )=\left ( 6,3 \right )$.}
\end{gbtt}
\begin{gbtt} 
Giải phương trình nghiệm nguyên dương \[7^x+3^y=2^z.\]
\loigiai{
Giả sử phương trình đã cho có nghiệm nguyên dương $(x,y,z).$\\
Để có thể tạo ra các nhân tử, ta sẽ chứng minh rằng $y,z$ là các số chẵn.
\begin{enumerate}[\color{tuancolor}\bf\sffamily Bước 1.]
    \item Ta chứng minh $y$ chẵn. \\ Ta nhận xét được $2^z\ge 7+3=10,$ nên là $z\ge 4.$ \\
    Tiếp theo, ta xét bảng đồng dư của $7^x$ và $3^y$ theo modulo $8$ sau
         \begin{center} \begin{tabular}{c|c|c|c|c}
            $x$ & lẻ & lẻ & chẵn & chẵn \\
            \hline
            $7^x$ & $7$ & $7$ & $1$ & $1$\\
            \hline
            $y$ & lẻ & chẵn & lẻ & chẵn \\
            \hline
            $3^y$ & $3$ & $1$ & $3$ & $1$\\
            \hline
            $7^x+3^y$ & $2$ & $0$ & $4$ & $2$
            \end{tabular}
        \end{center}
    Do $7^x+3^y=2^z$ chia hết cho $8$ nên ta được $x$ lẻ, $y$ chẵn từ bảng trên.
    \item Ta chứng minh $z$ chẵn. \\ 
    Xét đồng dư thức
    $2^z\equiv 7^x+3^y\equiv 1 \pmod{3}.$ Đồng dư thức trên, rõ ràng, cho ta $z$ chẵn.
\end{enumerate}
Bằng các chứng minh trên, ta có thể đặt $y=2u,z=2v,$ với $u$ và $v$ nguyên dương. Phương trình trở thành
$$7^{x}+3^{2u}=2^{2v}\Leftrightarrow 7^x=\tron{2^v-3^u}\tron{2^v+3^u}.$$
Cả $2^v-3^y$ và $2^v+3^u$ đều là lũy thừa cơ số $7.$ Chính vì thế, ta có thể đặt $$2^v-3^u=7^m,\qquad 2^v+3^u=7^n,$$ 
trong đó $0\le m<n$ và $m+n=x.$ Lấy hiệu theo vế, ta được
$$2.3^u=7^n-7^m=7^m\tron{7^{n-m}-1}.$$
Vế trái không thể là bội của $7,$ chứng tỏ $m=0.$ Từ đây, ta suy ra
    $$2^v-3^u=1.$$
Nghiệm nguyên dương $(v,u)=(2,1)$ ở phương trình trên là kết quả đã xuất hiện trong \chu{bài \ref{bai4mu}}. \\
Kết luận, $(x,y,z)=(1,2,4)$ là nghiệm nguyên dương duy nhất của phương trình.}
\end{gbtt} 

\begin{gbtt}\label{phannguyen3}
Giải phương trình nghiệm nguyên dương \[2^x+3^y=5^z.\]
\loigiai{Giả sử phương trình đã cho có nghiệm nguyên dương $(x,y,z)$.\\
Ta có $2^x\equiv 5^z\equiv 2^z\pmod 3$, do vậy $x$ và $z$ cùng tính chẵn lẻ. Ta xét các trường hợp sau đây.
\begin{enumerate}
    \item Nếu $x$ và $z$ cùng chẵn, ta đặt $x=2x'$. Phương trình đã cho trở thành \[4^{x'}+3^y=5^z.\] 
    Theo như \chu{ví dụ \ref{bai3mu}}, ta tìm được $x'=y=z=2,$ thế nên $x=4,y=2,z=2.$
    \item Nếu $x$ và $z$ cùng lẻ, ta xét tiếp tới các trường hợp nhỏ hơn sau.
\begin{itemize}
        \item\chu{Trường hợp 1.} Với $x\ge 3,$ ta có $2^x$ chia hết cho $8.$ Đồng thời, do $x$ lẻ nên $5^x\equiv 5\pmod{8}.$ lấy đồng dư theo modulo $8$ hai vế, ta được
        $$3^y\equiv 5\pmod{8}.$$
        Điều trên là không thể xảy ra với cả trường hợp $y$ chẵn và $y$ lẻ.
        \item\chu{Trường hợp 2.} Với $x=1,$ phương trình đã cho trở thành
        \[2+3^y=5^z.\tag{*}\label{2,3,5,chauu}\]
        Lấy đồng dư theo modulo $5$ hai vế, ta được 
        $$2+3^y\equiv 0\pmod{5}\Rightarrow 3^y\equiv 3\pmod{5}.$$
        Ta nhận được $y$ lẻ, vì nếu $y$ chẵn thì $3^y=9^{\frac{y}{2}}\equiv -1,1\pmod{5}.$ 
        \begin{itemize}
            \item \chu{Khả năng 2.1.} Nếu $y=1,$ ta tìm ra $z=1.$
            \item \chu{Khả năng 2.2.} Nếu $y\ge 3,$ ta có $3^y$ chia hết cho $9.$ Lấy đồng dư hai vế của (\ref{2,3,5,chauu}), ta được
            $$2\equiv 5^z\pmod{9}.$$
            Từ đây, bằng cách xét số dư của $z$ khi chia cho $6$ và để ý $\ddu{5^6}{1}{9}$, ta có $\ddu{z}{5}{6}.$ Mặt khác, do $\ddu{5^6}{1}{7}$ nên khi lấy đồng dư modulo $7$ hai vế của (\ref{2,3,5,chauu}), ta được
            \[2\equiv 5^z-3^y\equiv 5^5-3^y\equiv 3-3^y\pmod 7.\] 
            Ta suy ra $\ddu{3^y}{1}{7}$. Đến đây, ta lại xét số dư của $y$ khi chia cho $6$ với chú ý $3^6\equiv 1\pmod 7.$ Cách xét này cho ta biết $y$ chia hết cho $6,$ kéo theo $y$ là số chẵn, mâu thuẫn với nhận xét $y$ lẻ.
        \end{itemize}
\end{itemize}
\end{enumerate}
Kết luận, phương trình đã cho có $2$ nghiệm là $(4,2,2),(1,1,1).$}
\begin{luuy}
Việc trình bày vắn tắt ở \chu{trường hợp 2} trong phần lời giải trên là có cơ sở. Bạn đọc có thể tham khảo lời giải của câu số học trong đề \chu{chuyên Khoa học Tự nhiên 2021} để hiểu rõ hơn phương pháp.
\end{luuy}
\end{gbtt}

\begin{gbtt}\label{phannguyen5/8}
Giải phương trình nghiệm tự nhiên \[2^x+5^y=7^z.\]
\loigiai{Giả sử phương trình đã cho có nghiệm tự nhiên $(x,y,z)$.
\begin{enumerate}
    \item Với $x=0,$ ta có $1+5^y=7^z.$
    \begin{itemize}
	    \item \chu{Trường hợp 1.1.} Nếu $y=0,$ ta không tìm được $z$ nguyên.
	    \item \chu{Trường hợp 1.2.} Nếu $y>0$, lấy đồng dư theo modulo $5,$ ta có
	    $$7^z\equiv 5^y+1\equiv 1\pmod 5.$$ Bắt buộc, $z$ là số chẵn. Đặt $z=2k,$ và phép đặt này cho ta \[5^y=7^{2k}-1=\left(7^k-1\right)\left(7^k+1\right).\]
	    Cả $7^k-1$ và $7^k+1$ lúc này đều là lũy thừa số mũ tự nhiên của $5.$ Ta đặt
		$$7^k-1=5^u,\quad 7^k+1=5^v.$$
		Lấy hiệu theo vế tương tự các bài toán trước, ta chỉ ra điều vô lí trong trường hợp này.
    \end{itemize}
    \item Với $x=1,$ ta có $2+5^y=7^z.$	
    \begin{itemize}
	    \item\chu{Trường hợp 2.1.} Nếu $y\ge 2,$ ta chỉ ra $7^z\equiv 1,24,7,18\pmod{25},$ phụ thuộc vào tính chẵn lẻ của $y.$ Mặt khác, khi lấy đồng dư theo modulo $25$ ở $2+5^y=7^z,$ ta được
	    $$7^y\equiv 2\pmod{5}.$$
	    Các lập luận trên mâu thuẫn nhau.
	    \item \chu{Trường hợp 2.2.} Nếu $y=1$, ta tìm được $z=1.$
	    \item \chu{Trường hợp 2.3.} Nếu $y=0$, ta không tìm được $z$ nguyên.
    \end{itemize}
    \item Với $x\ge 2,$ ta có $7^z=5^y+2^x\equiv 1\pmod 4,$ kéo theo $z$ là số chẵn. Ta đặt $z=2z'.$
    \begin{itemize}
	    \item \chu{Trường hợp 3.1.} Với $y=0$, ta có $$2^x=\left(7^{z'}-1\right)\left(7^{z'}+1\right).$$ 
	    Bằng cách chỉ ra các lũy thừa số mũ tự nhiên của $2$ rồi xét hiệu, ta không tìm ra được $x,z'$ nguyên trong trường hợp này.
	    \item \chu{Trường hợp 3.2.} Với $y>0$, ta có 
	    $$2^x\equiv 7^z\equiv 2^z\pmod 5.$$ Mặt khác, do $z$ là số chẵn nên $2^z\equiv -1,1\pmod{5},$ chứng tỏ $x$ cũng là số chẵn. \\
	    Đặt $x=2x'$, và phép đặt này cho ta
	    \[5^y=7^{2z'}-2^{2x'}=\left(7^{z'}-2^{x'}\right)\left(7^{z'}+2^{x'}\right).\]
	     Bằng cách chỉ ra các lũy thừa số mũ tự nhiên của $5$ rồi xét hiệu, ta tìm được $7^{z'}-1=2^{x'}.$\\
	     Tuy nhiên, $7^{z'}-1$ chia hết cho $3,$ và vì thế, $2^{x'}$ cũng chia hết cho $3,$ mâu thuẫn.
\end{itemize}
\end{enumerate}
Kết luận, phương trình có duy nhất một nghiệm tự nhiên là $(x,y,z)=(1,1,1).$}
\end{gbtt}

\begin{gbtt}
Giải phương trình nghiệm nguyên dương \[5^x+12^y=13^z.\]
\loigiai{Giả sử phương trình đã cho có nghiệm nguyên dương $(x,y,z).$\\ Ta có $5^x\equiv 13^z\equiv 1\pmod 3$, do vậy $x$ là số chẵn. Ta đặt $x=2x',$ với $x'$ là số nguyên dương.
\begin{enumerate}
	\item Nếu $y\ge 2$, ta có $12^y$ chia hết cho $8.$ Lấy đồng dư theo modulo $8$ hai vế, ta chỉ ra
	$$5^z\equiv 13^z=5^x+12^y\equiv 1+0\equiv 1\pmod{8}.$$
	Ta suy ra $z$ là số chẵn. Đặt $z=2z'$, và phép đặt này cho ta
	$$13^z=169^{z'}\equiv -1,1\pmod{5}.$$
	Đồng dư thức kể trên kết hợp với phương trình ban đầu cho ta
	$$12^y\equiv -1,1\pmod{5}.$$
	Do đó, $y$  cũng là số chẵn. Đặt $y=2y'$, và ta có 
	\[5^x=13^{2z'}-12^{2y'}=\left(13^{z'}-12^{y'}\right)\left(13^{z'}+12^{y'}\right).\] 
	Vì $\left(13^{z'}-12^{y'},13^{z'}+12^{y'}\right)=1$ nên $13^{z'}-12^{y'}=1$.
	\begin{itemize}
		\item \chu{Trường hợp 1.} Nếu $y'=1$, ta nhận được $z'=1,$ kéo theo $x=y=z=2.$
		\item \chu{Trường hợp 2.} Nếu $y'>1$, ta nhận được $12^{y'}$ chia hết cho $8,$ thế nên 
		$$13^{z'}\equiv 1\pmod{8}.$$
		Đồng dư thức trên cho ta $z'$ là số chẵn. Tuy nhiên, khi $z'$ chẵn, ta có điều vô lí là $$(13+1)\mid\tron{ 13^{z'}-1}=12^{y'}\Rightarrow 7\mid 12^{y'}.$$
	\end{itemize}
\item Nếu $y=1$, ta có $5^x+12=13^z$, kéo theo $\ddu{13^z}{2}{5}$. Xét số dư của $z$ khi chia cho $4$ và để ý $\ddu{13^4}{1}{5}$, ta suy ra $z$ chia $4$ dư $3.$ Đặt $z=4k+3,$ và ta có
\[	13\left(13^{4k+2}+1\right)=25\left(5^{2x'-2}+1\right).\]
Bởi vì $(13,25)=1,$ biến đổi trên cho ta
$$(-1)^{x'-1}\equiv 25^{x'-1}\equiv -1\pmod{13}.$$
Dựa vào đồng dư thức trên, ta biết $x'$ là số chẵn. Tiếp tục đặt $x'=2m$ do $$\tron{13^2+1}\mid \tron{13^{4k+2}+1}$$ nên $13^{4k+2}+1$ chia hết cho $17,$ và thế thì
\[5^{4m+2}+1\equiv 13^{4k+2}+1\equiv 0\pmod {17}\Rightarrow 8\cdot 13^m+1\equiv 0\pmod{17}.\] Bằng cách xét số dư của $m$ khi chia cho $4,$ ta thấy $8\cdot 13^m+1$ không chia hết cho $17,$ vô lí.
\end{enumerate}
Kết luận, phương trình đã cho có nghiệm duy nhất là $(x,y,z)=(2,2,2).$}
\end{gbtt}
%hello anh
\begin{gbtt}
Tìm tất cả các số nguyên dương $x,y,z$ thỏa mãn \[3^x+2^y=1+2^z.\]
\nguon{Chọn học sinh giỏi thành phố Hà Nội 2021}
\loigiai{Với $y=1,$ ta có bài toán sau đây.
\begin{light}
\begin{it}
"Tìm tất cả các số nguyên dương $x,y,z$ thỏa mãn
$7^x+3^y=2^z$".
\end{it}
\end{light}
Bài toán cho ta kết quả $(x,y,z)=(1,1,2).$ Phần còn lại, ta tiến hành với $y\ge 2.$  \\
Ta nhận thấy, do $3^{x}>1$ nên $2^{z}>2^{y}$, hay là $z>y\ge 2.$ Từ đây, ta đánh giá được
$$3^x=2^z+1-2^y\equiv 1 \pmod{3}.$$
Ta lập tức thu được $x$ chẵn. Bằng kiểm tra trực tiếp, ta nhận thấy $(x,y,z)=(2,3,4)$ thỏa mãn đề bài, thế nên ta nghĩ đến việc xét các trường hợp sau.
\begin{enumerate}
    \item Nếu $y\ge 4,$ do $16$ là ước của cả $2^y$ và $2^z$ nên
    $$3^x=2^z+1-2^y\equiv 1 \pmod{16}.$$
    Do $x$ chẵn, $x$ chia $4$ có thể dư $2$ hoặc $4.$ Tuy nhiên, nếu $x=4m+2,$ ta sẽ có
    $$3^{x}=3^{4m+2}=9\cdot 81^m \equiv 9 \pmod{16}.$$
    Đồng dư thức này mâu thuẫn với việc $3^x\equiv 1\pmod{16},$ thế nên $x$ chia hết cho $4.$ \\
    Đặt $x=4n,$ với $n$ nguyên dương. Ta có
    $$2^y\left(2^{z-y}-1\right)=2^z-2^y=3^x-1=81^n-1\equiv 0 \pmod{5}.$$
    Do $\left(2^y,5\right)=1$ nên ta suy ra $2^{z-y}\equiv 1 \pmod{5}$ từ đây. Ta đã biết $2^4\equiv 1\pmod{5},$ vì thế với $x-y \equiv a \pmod{4}$ và $2^{x-y}\equiv b \pmod{5},$ ta lập được bảng sau.
         \begin{center}           \begin{tabular}{c|c|c|c|c}
            $a$ & $0$ & $1$ & $2$ & $3$ \\
            \hline
            $b$ & $1$ & $2$ & $4$ & $8$
        \end{tabular}
        \end{center}
    Bảng trên cho ta $4\mid (x-y),$ và như vậy, $3\mid 2^{x-y}-1=3^x-1.$\\ Điều này là không thể xảy ra. Trường hợp này bị loại.
    \item Nếu $y=3,$ đặt $x=2t,$ và phép đặt này cho ta
    $$3^{2t}=2^z-2^y+1=2^z-7.$$
    Ta được $2^z\equiv 1\pmod{3},$ tức $z$ chẵn. Đặt $z=2a,$ ta thu được
    $$\left(2^a-3^t\right)\left(2^a+3^t\right)=7\Rightarrow\heva{2^a-3^t&=1 \\ 2^a+3^t&=7} \Rightarrow \heva{t&=1 \\ a&=2.}$$
    Thay ngược lại, ta tìm ra $x=2,z=4.$ 
    \item Nếu $y=2,$ thế trở lại phương trình ban đầu, ta có
    $$3^x+4=1+2^z \Rightarrow 3^x-3=2^z.$$
    $2^z$ không thể chia hết cho $3,$ chứng tỏ không tồn tại $x,y,z$ trong trường hợp này.
\end{enumerate}
Như vậy tất cả các bộ số $(x,y,z)$ thỏa mãn đề bài là $(1,1,2)$ và $(2,3,4).$}
\end{gbtt}

\begin{gbtt}
	Tìm tất cả các bộ số tự nhiên $(x,y,z)$ thỏa mãn \[2^x+3^y=z^2.\]
\loigiai{Giả sử phương trình đã cho có nghiệm tự nhiên $(x,y,z)$.
\begin{enumerate}
    \item Nếu $y=0$, ta có $2^x=(z-1)(z+1).$ Lúc này, bạn đọc tự chỉ ra các lũy thừa của $2,$ lấy hiệu theo vế và tìm ra $x=z=3.$
    \item Nếu $y>0$, ta có $z^2=2^x+3^y$ không chia hết cho 3 nên $z^2$ chia $3$ dư $1.$ Xét modulo $3$ hai vế, ta có
    $$2^x\equiv 1\pmod{3}\Rightarrow (-1)^x\equiv 1\pmod{3},$$
    thế nên $x$ là số chẵn. Đặt $x=2m,$ và phép đặt này cho ta
    \[3^y=\left(z-2^m\right)\left(z+2^m\right).\]
    Cả $z-2^m$ và $z+3^m$ đều là lũy thừa cơ số $3.$ Chính vì thế, ta có thể đặt 
    $$z-2^m=3^u,\qquad z+2^m=3^v,$$ 
    trong đó $0\le u<v$ và $u+v=y.$ Lấy hiệu theo vế, ta được
    $$2^{m+1}=3^v-3^u=3^u\tron{3^{v-u}-1}.$$ 
    So sánh số mũ của lũy thừa cơ số $2$ tại hai vế, ta chỉ ra $u=0,$ kéo theo $3^v-2^{m+1}=1$. \\
    Theo như \chu{ví dụ \ref{bai2mu}}, ta nhận thấy $v=m+1=1$ hoặc $v=2,m+1=3.$
\begin{itemize}
	\item \chu{Trường hợp 1.} Với $v=m+1=1$, ta tìm ra $x=0,y=1,z=2.$
	\item \chu{Trường hợp 2.} Với $v=2,m+1=3$, ta tìm ra $x=4,y=2,z=5.$
\end{itemize}
\end{enumerate}
Như vậy, phương trình đã cho có $3$ nghiệm là $(3,0,3),(0,1,2),(4,2,5).$}
\end{gbtt}

\begin{gbtt}
Tìm tất cả các số nguyên dương $m,n$ thỏa mãn
    \[10^n-6^m=4n^2.\]
\nguon{Tigran Akopyan}
\loigiai{
Giả sử tồn tại cặp $(m,n)$ thỏa yêu cầu. Nếu $n=1,$ ta có $m=1.$ Nếu $n>1,$ ta xét các trường hợp sau.
\begin{enumerate}
    \item Nếu $n$ lẻ, lấy đồng dư modulo $8$ hai vế phương trình đã cho ta có
    $$10^{n}-6^{m} \equiv-6^{m}\pmod 8,\quad 4 n^{2} \equiv 4\pmod 8.$$
    Ta suy ra $-6^m\equiv 4\pmod{8},$ và như thế thì $m=2.$ Thế trở lại, ta có $10^n-4n^2=36,$ vô lí vì 
    $$10^n-4n^2>36,\text{ với mọi }n\ge 3.$$
    \item Nếu $n$ chẵn, ta đặt $n=2k.$ Phương trình đã cho trở thành
    $$\left(10^{k}-4 k\right)\left(10^{k}+4 k\right)=6^{m}.$$
    Với việc không tìm được $n$ khi $k=1,2,$ ta sẽ xét $k>2.$ Dễ thấy khi ấy $m\ge 4.$ Chia hai vế cho $16,$ phương trình bên trên tương đương
    \[\left(2^{k-2} \cdot 5^{k}-k\right)\left(2^{k-2} \cdot 5^{k}+k\right)=2^{m-4} \cdot 3^{m}.\]
    Nếu $k$ là số lẻ thì $m=4$, nhưng do $2^{k-2} \cdot 5^{k}+k \geq 2 \cdot 5^{3}+3>3^{4}$ nên dẫn đến mâu thuẫn. Do đó $k$ là số chẵn. Ta đặt $k=2^{\alpha} h$, trong đó $\alpha, h$ là các số nguyên dương và $h$ lẻ. 
    \begin{itemize}
        \item \chu{Trường hợp 1.} Nếu $\alpha \geq k-2$ thì $2^{k-2} \mid k$, nghĩa là $2^{k-2} \leq k$, từ đây ta tìm được các giá trị  $k$ thỏa mãn là $k=3,4.$ Thế trở lại, ta thấy mâu thuẫn.
        \item \chu{Trường hợp 2.} Nếu $\alpha<k-2$ thì phương trình được viết lại thành
        $$2^{2 \alpha}\left(2^{k-2-\alpha} \cdot 5^{k}-h\right)\left(2^{k-2-\alpha} \cdot 5^{k}+h\right)=2^{m-4} \cdot 3^{m}$$
        và do tính duy nhất của phân tích ra thừa số nguyên tố nên ta có $2\alpha=m-4$, do đó từ quan hệ $k>\alpha+2$ thì ta thu được $n>2 \alpha+4=m$. Nhưng từ đánh giá này, ta thu được $$10^{n}=6^{m}+4 n^{2}<6^{n}+4 n^{2},$$ mâu thuẫn với điều kiện $n>1.$
          \end{itemize}
\end{enumerate}
Kết luận, $(m,n)=(1,1)$ là cặp số duy nhất thỏa yêu cầu.}
\end{gbtt}

\begin{gbtt}
Giải phương trình nghiệm tự nhiên \[2^x+7^y=z^3.\]
\loigiai{Giả sử phương trình đã cho có nghiệm tự nhiên $(x,y,z)$.
\begin{enumerate}
    \item Nếu $y=0,$ phương trình đã cho trở thành
    $$2^x=(z-1)\tron{z^2+z+1}.$$
    Cả $z-1$ và $z^2+z+1$ đều là lũy thừa cơ số $2,$ và ngoài ra $z^2+z+1>z-1.$ Vì thế
    $$\tron{z-1}\mid \tron{z^2+z+1}.$$
    Bài toán quen thuộc này cho ta $z=2$ hoặc $z=4.$ Thử trực tiếp, không tìm được $x$ nguyên.
    \item Nếu $y\ge 1,$ ta có $7^y$ chia hết cho $7.$ Vì thế
    $$2^x\equiv z^3\pmod 7.$$
    Do $z^3\equiv 0,1,-1\pmod{7}$ nên dễ thấy $x$ chia hết cho $3.$ Ta đặt $x=3k.$ Phép đặt này cho ta
    \[7^y=\left(z-2^k\right)\left(z^2+2^kz+2^{2k}\right).\]
    Cả $z-2^k$ và $z^2+2^kz+2^{2k}$ đều là lũy thừa của cơ số $7.$ Ta tiếp tục đặt	
    $$z-2^k=7^u,\qquad z^2+2^kz+2^{2k}=7^v,$$
    trong đó $0\le u<v$ và $u+v=y.$ 
    \begin{itemize}
        \item \chu{Trường hợp 1.} Nếu $u>0$, ta có $\ddu{z}{2^k}{7}$, thế nên là \[0\equiv z^2+2^kz+2^{2k}\equiv 3\cdot 2^{2k}\pmod 7,\]
        một điều không thể xảy ra.
        \item \chu{Trường hợp 2.} Nếu $u=0,$ ta có $z=2^k+1$. Thế vào $z^2+2^kz+2^{2k}=7^v,$ ta được
        \[\left(2^k+1\right)^2+2^k\left(2^k+1\right)+2^{2k}=7^v\Rightarrow 3\left(4^k+2^k\right)=7^v-1.\]
        Do $4^3\equiv 2^3\equiv 1\pmod 7$ nên từ đây ta dễ dàng suy ra $k$ chia hết cho $3,$ dẫn đến $3\left(4^k+2^k\right)$ chia hết cho $8.$ Lập luận này cho ta \[\ddu{7^v}{1}{8}.\] Với $\ddu{7^2}{1}{8}$, ta nhận xét được $v$ là số chẵn. Bây giờ, ta tiếp tục xét hai khả năng sau.
        \begin{itemize}
    \item \chu{Khả năng 1.} Với $k=0,$ ta tìm ra $(x,y,z)=(0,1,2).$
	\item \chu{Khả năng 2.} Với $k$ là số nguyên dương chẵn, ta có $k$ chia hết cho $6.$ Đặt $k=6m,$ ta được
	\[3\left(4^k+2^k\right)=3\left(2^{12m}+2^{6m}\right)\equiv 3(1+1)\equiv 6\pmod 9.\] 
	Do đó $\ddu{7^v}{7}{9}$. Dựa theo chú ý $\ddu{7^3}{1}{9}$, ta suy ra $b\equiv 1\pmod 3$, nhưng do $v$ chẵn nên nên ta có thể đặt $v=6n+4.$ Các đồng dư thức
	$$2^6\equiv 7^6\equiv -1\pmod{13},\quad \ddu{4^6}{1}{13}$$ 
	cho ta một số kết quả
	\begin{align*}
	    3\left(4^k+2^k\right)&=3\left(4^{6m}+2^{6m}\right)\equiv 0,6\pmod{13},\\
	    7^b-1&=7^{6n+4}-1\equiv 3,8\pmod{13}.
	\end{align*}
	Hai kết quả bên trên mâu thuẫn nhau.
	\item \chu{Khả năng 3.} Với $k$ là số lẻ, ta đặt $k=6m+3$, $k$. Hoàn toàn tương tự \chu{khả năng 1}, ta chỉ ra $b$ chia hết cho $6.$ Đặt $b=6n,$ và phép đặt này cho ta
	\begin{align*}
	    3\left(4^k+2^k\right)&=3\left(4^{6m+3}+2^{6m+3}\right)\equiv 8,12\pmod{13},\\
	    7^b-1&=7^{6n}-1\equiv 0,11\pmod{13}.
	\end{align*}
	Hai kết quả trên mâu thuẫn nhau.
\end{itemize}
\end{itemize}
\end{enumerate} 
Như vậy, phương trình đã cho có nghiệm tự nhiên duy nhất là $(x,y,z)=(0,1,2).$}
\end{gbtt}

\begin{gbtt}
Giải phương trình nghiệm tự nhiên \[2^xx^2=9y^2+6y+16.\]
\loigiai{
Giả sử phương trình đã cho có nghiệm tự nhiên $(x,y).$ Rõ ràng $x$ không chia hết cho $3.$ Lấy đồng dư theo modulo $3$ hai vế, ta được
    $$2^xx^2\equiv (-1)^x\cdot1\equiv (-1)^x\pmod{3},\qquad 9y^2+6y+16\equiv 1\pmod{3}.$$
    Căn cứ vào đây, ta suy ra $x$ phải là số chẵn. Bằng cách đặt $x=2z,$ phương trình đã cho trở thành
    $$2^{2z}\cdot (2z)^2=(3y+1)^2+15\Leftrightarrow \left(2^{z+1}\cdot z-3y-1\right)\left(2^{z+1}\cdot z+3y+1\right)=15.$$
    Ta nhận thấy rằng do $2^{z+1}\cdot z+3y+1>0$ nên $0<2^{z+1}\cdot z-3y-1<2^{z+1}\cdot z-3y-1.$ Nhận xét này giúp ta chia bài toán làm hai trường hợp.
    \begin{enumerate}
        \item Nếu $2^{z+1}\cdot z-3y-1=1$ và $2^{z+1}\cdot z+3y+1=15,$ lấy tổng và hiệu theo vế, ta được
        $$2^{z+2}\cdot z=16,\qquad 6y+2=14.$$
        Ta không tìm được $z$ từ đây, bởi vì nếu $z=0$ hoặc $z=1$ thì $2^{z+2}\cdot z<16,$ còn nếu $z\ge 2$ thì $$2^{z+2}\cdot z\ge 32>16.$$
        \item Nếu $2^{z+1}\cdot z-3y-1=3$ và $2^{z+1}\cdot z+3y+1=5,$ bằng cách lấy tổng và hiệu theo vế tương tự, ta tìm ra $z=1$ và $y=0.$   
    \end{enumerate}
Kết luận, $(x,y)=(2,0)$ là nghiệm tự nhiên duy nhất của phương trình.}
\end{gbtt}


\subsection{Phương pháp kết hợp xét tính chia hết và xét modulo}

\subsubsection*{Ví dụ minh họa}

\begin{bx}
Giải phương trình nghiệm tự nhiên $$5^x3^y=3z^2+2z-1.$$
\nguon{Tạp chí Pi tháng 11 năm 2017}
\loigiai{
Biến đổi tương đương phương trình đã cho, ta được
$$5^x3^y=3z^2+2z-1\Leftrightarrow5^x3^{y}=\tron{3z-1}\tron{z+1}.$$
Giả sử phương trình đã cho có nghiệm tự nhiên $(x,y,z).$ Ta xét các trường hợp sau.
\begin{enumerate}
    \item Với $y\ge1$, ta có $3\mid 5^x3^y=\tron{3z-1}\tron{z+1}$. Vì $\tron{3z-1,3}=1$ và $\tron{3z-1,z+3}=1$ nên 
    $$3^y=z+1,\quad  5^x=3z-1.$$ 
    Từ đây, ta thu được
    $3^{y+1}=5^x+4.$ Lấy đồng dư theo modulo $5$ cả hai vế, ta có
    $$3^{y+1}\equiv4\pmod{5}.$$
    Bắt buộc, $y+1$ là số chẵn. Đặt $y+1=2k$ rồi thể trở lại, ta chỉ ra
    $$3^{2k}=5^x+4\Leftrightarrow \tron{3^k-2}\tron{3^k+2}=5^x.$$
    Biến đổi trên cho ta biết cả $3^k-2$ và $3^k+2$ đều là lũy thừa của $5.$ Tuy nhiên, hai số này không cùng chia hết cho $5,$ ép buộc một trong hai số ấy bằng $1.$ Ta nhận được $3^k-2=1,$ và ta lần lượt chỉ ra $k=1, x=1, y=2, z=2.$
    \item Với $y=0,$ ta có $5^x=\tron{3z-1}\tron{z+1}$. Từ đây, ta suy ra $\tron{3z-1}, \tron{z+1}$ là các lũy thừa của $5$. Đặt 
    $3z-1=5^a$, $z+1=5^b$ trong đó $a,b$ là số tự nhiên. Phép đặt này cho ta
    $$3\cdot5^b=5^a+4.$$
    Ta xét tiếp tới các trường hợp nhỏ hơn sau.
    \begin{itemize}
        \item\chu{Trường hợp 1.} Với $b\ge1$, ta xét modulo $5$ cả hai vế và có
        $$5^a+4\equiv0\pmod{5}\Rightarrow a=0\Rightarrow z=\dfrac{2}{3}.$$
        $z$ lúc này không phải số nguyên, mâu thuẫn.
        \item\chu{Trường hợp 2.} Với $b=0$, thế trở lại, ta được 
        $3=5^b+4,$ mâu thuẫn.
    \end{itemize}
\end{enumerate}
    Như vậy, phương trình đã cho có nghiệm tự nhiên duy nhất là $\tron{x,y,z}=\tron{1,2,2}.$} 
\end{bx}

\begin{bx}
Giải phương trình nghiệm nguyên dương $$7^x=3\cdot2^y+1.$$
\nguon{Chuyên Toán Hà Nam 2020}
\loigiai{
Với $y\le 2,$ ta tìm ra $x=1$ khi $y=2.$\\ Với $y\ge 3$, ta có $3\cdot 2^y$ chia hết cho $8$. Lấy đồng dư theo modulo $8$ hai vế phương trình ban đầu, ta có
$$(-1)^x\equiv 7^x\equiv 3\cdot2^y+1 \equiv1\pmod{8}.$$
Bắt buộc, $z$ là số chẵn. Ta đặt $z=2t,$ và phép đặt này cho ta
\[3\cdot 2^y=\left(7^k-1\right)\left(7^k+1\right).\]
Ta có các nhận xét
\begin{enumerate}[i,]
    \item $7^k-1$ và $7^k+1$ là hai số lẻ liên tiếp nên chúng không cùng chia hết cho $4.$
    \item $7^k-1$ và $7^k+1$ có hiệu bằng $2$ nên chúng không cùng chia hết cho $3.$
\end{enumerate}
Các nhận xét trên cho phép ta xét các trường hợp sau.
\begin{enumerate}
        \item Với $7^k-1=6$ và $7^k+1=2^{y-1},$ ta tìm được $k=1,y=4,x=2.$
        \item Với $7^k-1=3\cdot 2^{y-1},7^k+1=2,$  ta có $k=0,$ nhưng khi ấy $3\cdot 2^{y-1}=0,$ mâu thuẫn.
        \item Với $7^k-1=2,7^k+1=3\cdot 2^{y-1},$ ta có $7^k=3,$ và $k$ không nguyên.
        \item Với $7^k-1=2^{y-1},7^k+1=3\cdot 2,$ ta có $7^k=5,$ và $k$ không nguyên.
\end{enumerate}
Như vậy, phương trình đã cho có $2$ nghiệm nguyên dương là $(1,2)$ và $(2,4).$}
\end{bx}

\begin{bx}
Giải phương trình nghiệm nguyên dương \[x!+1=5^y.\]
\loigiai{
Nếu $x\ge 5$ thì $x!$ chia hết cho $5$, do đó $5^y=x!+1$ chia $5$ dư $1$, vô lí. Nếu $x<5,$ ta lập bảng giá trị
\begin{center}
    \begin{tabular}{c|c|c|c|c}
       $x$  & $4$ & $3$ & $2$ & $1$ \\
       \hline
        $5^y=x!+1$ & $25$ & $7$ & $3$ & $2$ \\
       \hline
       $y$ & $2$ & $\notin\mathbb{Z}$ & $\notin\mathbb{Z}$ & $\notin\mathbb{Z}$
    \end{tabular}
\end{center}
Như vậy phương trình có nghiệm nguyên duy nhất là $(x,y)=(4,2).$}
\end{bx}

\subsubsection*{Bài tập tự luyện}


\begin{btt}
Giải phương trình nghiệm nguyên dương $$2^x5^y+25=z^2.$$
\end{btt}

\begin{btt}
Giải phương trình nghiệm tự nhiên $$2^x3^y+9=z^2.$$
\end{btt}

\begin{btt}
Giải phương trình nghiệm nguyên dương $$3^x7^y+8=z^3.$$
\end{btt} 

\begin{btt}
Giải phương trình nghiệm tự nhiên $$5^x7^y+4=3^z.$$
\end{btt}

\begin{btt}
Giải phương trình nghiệm nguyên dương $$2^{x+1}3^y+5^z=7^t.$$
\end{btt}

\begin{btt}
Cho các số nguyên dương $m,n,k,s$ thỏa mãn
$$\tron{3^m-3^n}^2=2^k+2^s.$$
Tìm giá trị lớn nhất của tích $mnks.$
\end{btt}

\begin{btt}
Tìm tất cả nghiệm nguyên dương của phương trình $$x!+5=y^{z+1}.$$
\end{btt}

\begin{btt}
Giải phương trình nghiệm nguyên dương $$x!+5^y=7^z.$$
\end{btt}

\begin{btt}
Tìm nghiệm nguyên dương của phương trình 
\[1!+2!+\cdots+(x+1)!=y^{z+1}.\]
\end{btt}

\begin{btt}
Tìm tất cả các số tự nhiên $p,q,r,s>1$ thỏa mãn
$$p!+q!+r!=2^s.$$
\nguon{Indian IMO Training Camp 2017}
\end{btt}

\subsubsection*{Hướng dẫn bài tập tự luyện}

\begin{gbtt}
Giải phương trình nghiệm nguyên dương \[2^x5^y+25=z^2.\]
\loigiai{Phương trình đã cho tương đương với
$$2^x5^y=(z-5)(z+5).$$
Giả sử phương trình đã cho có nghiệm nguyên dương $(x,y,z)$. Ta có một vài nhận xét sau đây.
    \begin{enumerate}
    \item[i,] $y\ge 2,$ bởi vì $z^2$ là số chính phương chia hết cho $25.$
    \item[ii,] $x\ge 2,$ bởi vì nếu $x=1$ thì $2^x5^y+25\equiv 3\pmod{4}$ và không là số chính phương.    
    \item[iii,] Hai số chẵn $z-5$ và $z+5$ không cùng chia hết cho $4.$
    \item[iv,] Hai số $z-5$ và $z+5$ không cùng chia hết cho $25,$ nhưng cùng chia hết cho $5.$
    \end{enumerate} 
    Dựa vào các nhận xét này, ta tiếp tục chia bài toán thành các trường hợp nhỏ hơn.
\begin{enumerate}
    \item Với $z-5=2\cdot5^{y-1},z+5=5\cdot 2^{x-1},$ lấy hiệu theo vế, ta được
        $$5\cdot2^{x-1}-2\cdot5^{y-1}=10\Leftrightarrow 2^{x-2}-5^{y-2}=1.$$
        Bài toán này đã được giải quyết ở phần bài tập trước. Đáp số là $(x,y)=(3,2).$
    \item Với $z-5=5\cdot2^{x-1},z+5=2\cdot 5^{y-1},$ lấy hiệu theo vế, ta được
        $$5\cdot2^{x-1}-2\cdot5^{y-1}=-10\Leftrightarrow 2^{x-2}-5^{y-2}=-1.$$
        Bài toán này đã được giải quyết ở phần bài tập trước. Đáp số là $(x,y)=(4,3).$   
    \item Với $z-5=5\cdot2,z+5=2^{x-1}\cdot 5^{y-1},$ ta tìm ra $z=15,$ kéo theo $x=3,y=2.$
    \item Với $z+5=5\cdot2,z-5=2^{x-1}\cdot 5^{y-1},$ ta tìm ra $z=5,$ mâu thuẫn.
\end{enumerate}
Kết luận, phương trình đã cho có các nghiệm nguyên dương là $(3,2,15),(4,3,45).$}
\end{gbtt}

\begin{gbtt}
Giải phương trình nghiệm tự nhiên \[2^x3^y+9=z^2.\]
\loigiai{
Giả sử phương trình đã cho có nghiệm nguyên dương $(x,y,z)$. Ta có $$(z-3)(z+3)=2^x3^y.$$
Đặt $d=(z-3,z+3).$ Phép đặt này cho ta $d\mid (z-3)$ và $d\mid (z+3)$ nên $d\mid 6.$ Ta suy ra
$$d\in\{1;2;3;6\}.$$ 
Ta xét các trường hợp sau
\begin{enumerate}
    \item Với $d=1$, ta lại xét tới các trường hợp nhỏ hơn sau.
      \begin{itemize}
          \item\chu{Trường hợp 1.1.} Nếu $z-3=1$ và $z+3=2^x3^y,$ ta có $z=4$ nên $7=2^x3^y,$ vô lí.
          \item\chu{Trường hợp 1.2.} Nếu $z-3=2^x$ và $z+3=3^y,$ ta có $y=0,\ z=-2,$ vô lí.
          \item\chu{Trường hợp 3.} Nếu $z-3=3^y$ và $z+3=2^x,$ ta có $y=0$ vì nếu $y\ge 1$ thì
          $$3\mid (z-3)\Rightarrow 3\mid (z+3)\Rightarrow 3\mid 2^x\Rightarrow x=0.$$
          Với $y=0,$ ta có $z=4$ dẫn đến $2^x=7,$ vô lí.
      \end{itemize}
      \item Với $d=2$, ta lại xét tới các trường hợp nhỏ hơn sau.
      \begin{itemize}
          \item\chu{Trường hợp 2.1.} Nếu $z-3=2$ và $z+3=2^{x-1}2^y,$ ta có $z=5.$ Thể trở lại, ta được
          $$2^{x-1}3^y=8.$$ 
          Lần lượt xét số mũ của $2$ và $3$ ở hai vế, ta tìm ra $x=4$ và $y=0.$
          \item\chu{Trường hợp 2.2.} Nếu $z-3=2^{x-1}$ và $z+3=2\cdot 3^y,$ lấy hiệu theo vế ta có
          $$2^{x-1}+6=3^y.$$
          Do vế trái không chia hết cho $3$, ta tìm ra $x=1$ nhưng lúc này $3^y=7,$ vô lí.
          \item\chu{Trường hợp 2.3.} Nếu $z-3=2\cdot 3^y$ và $z+3=2^{x-1},$ lấy hiệu theo vế ta có
          $$2\cdot 3^y+6=2^{x-1}.$$
          Do vế phải không chia hết cho $3$ nên $3^y$ không chia hết cho $3.$ Ta tìm ra $(x,y,z)=(4,0,5).$
          \item\chu{Trường hợp 2.4.}  Nếu $z-3=2^{x-1}3^y$ và $z+3=2$, ta có $z=-1,$ vô lí.
      \end{itemize}
      \item Với $d=3$, ta lại xét tới các trường hợp nhỏ hơn sau.
      \begin{itemize}
          \item\chu{Trường hợp 3.1.} Nếu $z-3=3$ và $z+3=2^x3^{y-1},$ ta tìm được $(x,y,z)=(0,3,6).$
          \item\chu{Trường hợp 3.2.} Nếu $z-3=3\cdot 2^x$ và $z+3=3^{y-1},$ lấy hiệu theo vế ta có
          $$3\cdot2^x+6=3^{y-1}.$$
          Nếu $x\ge 1,$ vế trái là số chẵn nên $3^{y-1}$ chẵn, vô lí. Như vậy $x=0,$ và $y=3,\ z=6.$
          \item\chu{Trường hợp 3.3.} Nếu $z-3=3^{y-1}$ và $z+3=3\cdot 2^x,$ lấy hiệu theo vế ta có
          $$3^{y-1}+6=3\cdot 2^x.$$
          Xét tính chẵn lẻ ở hai vế, ta thấy $2^x$ lẻ nên $x=0,$ nhưng khi đó $3^{y-1}=-3,$ vô lí.
          \item\chu{Trường hợp 3.4.} Nếu $z-3=2^x3^{y-1}$ và $z+3=3$, ta có $z=0$ và $2^x3^{y-1}=-3,$ vô lí.
     \end{itemize}
    \item Với $d=6$, ta lại xét tới các trường hợp nhỏ hơn sau.
    \begin{itemize}
        \item\chu{Trường hợp 4.1.} Nếu $z-3=6$ và $z+3=2^{x-1}3^{y-1},$ ta tìm được $z=9,$ dẫn đến $x=3,\ y=2.$
        \item\chu{Trường hợp 4.2.} Nếu $z-3=2\cdot 3^{y-1}$ và $z+3=2^{x-1}\cdot 3,$ lấy hiệu theo vế ta có
        $$1=2^{x-2}-3^{y-2}.$$ 
        Do $2^{x-2}=1+3^{y-2}\equiv 2,4\pmod{8}$ nên $x\in \{3;4\}.$ Trường hợp này cho ta các bộ 
        $$(x,y,z)=(3,2,9),\quad (x,y,z)=(4,3,21).$$
        \item\chu{Trường hợp 4.3.} Nếu $z-3=3\cdot 2^{x-1}$ và $z+3=2\cdot 3^{y-1},$ lấy hiệu theo vế ta có 
        $$1=3^{y-2}-2{x-2}.$$ 
        Bằng việc xét các trường hợp nhỏ $x=2,x=3,x=4$ rồi xét tới tính chia hết cho $8,$ ta chỉ ra $y-2$ chẵn. Cách làm quen thuộc này cho ta các bộ
        $$(x,y,z)=(3,3,15),\quad (x,y,z)=(5,4,51).$$
         \item\chu{Trường hợp 4.4.} Nếu $z-3=2^{x-1}3^{y-1}$ và $z+3=6$, ta có $z=3,$ và $2^{x-1}3^{y-1}=0,$ vô lí.
     \end{itemize}
\end{enumerate}
Tổng kết lại, phương trình đã cho có $6$ nghiệm tự nhiên là
\[(4,0,5),\, (0,3,6),\, (3,2,9),\, (4,3,21),\, (3,3,15),\, (5,4,51).\]}
\end{gbtt}

\begin{gbtt}
Giải phương trình nghiệm nguyên dương \[3^x7^y+8=z^3.\]
\loigiai{Phương trình đã cho tương đương với \[3^x7^y=(z-2)(z^2+2z+4).\]
Giả sử phương trình đã cho có nghiệm nguyên dương $(x,y,z).$ Ta có một vài nhận xét sau đây.
\begin{enumerate}
    \item[i,] $\tron{z-2,z^2+2z+4}\in \{1;3\},$ thế nên chỉ $1$ trong hai số $z-2$ và $z^2+2z+4$ chia hết cho $7^y.$ 
    \item[ii,] Số $z^2+2z+4$ không thể chia hết cho $9,$ vậy nên $z-2$ chia hết cho $3^{x-1}$ (hoặc thậm chí $3^x$).
    \item[iii,] $z-2<z^2+2z+4.$
\end{enumerate}
Dựa vào các nhận xét này, ta tiếp tục chia bài toán thành các trường hợp nhỏ hơn.
\begin{enumerate}
	\item Nếu $\tron{z-2,z^2+2z+4}=1$, ta xét $2$ trường hợp nhỏ hơn sau đây.
	\begin{itemize}
	    \item \chu{Trường hợp 1.} $z-2=1,\ z^2+2z+4=3^x7^y.$ Trường hợp này cho ta $z=3,$ nhưng khi đó ta không tìm thấy được $x,y$ nguyên dương tương ứng.
	    \item \chu{Trường hợp 2.} $z-2=3^x,\ z^2+2z+4=7^y.$ Trong trường hợp này, ta có $z\equiv 2\pmod{3}$ nên
	    $$7^y=z^2+2z+4\equiv 2^2+2\cdot2+4\equiv0\pmod{3}.$$
	    Đây là điều không thể nào xảy ra.
	\end{itemize}
	\item Nếu $\tron{z-2,z^2+2z+4}=3,$ ta xét $2$ trường hợp nhỏ hơn sau đây.
	\begin{itemize}
	    \item \chu{Trường hợp 1. }$z-2=3^{x-1},\ z^2+2z+4=3\cdot 7^y.$ Trong trường hợp này, ta có
	    \[3^{2x-3}+2\cdot 3^{x-1}+4=7^y.\]
	    Lấy đồng dư theo modulo $4$ hai vế, ta được
	    $$(-1)^y\equiv 7^y\equiv 3+2+0\equiv 1\pmod{4}.$$
	    Theo đó, $y$ là số chẵn. Ta đặt $y=2t.$ Phép đặt này cho ta
	    \[3^{2x-3}+2\cdot 3^{x-1}+4=7^{2z}\Rightarrow 3^{x-1}\tron{3^{x-2}+2}=\tron{7^t-2}\tron{7^t+2}.\]
	    Do $7^z-2\equiv -1\pmod{3}$ và $7^z+2\equiv 0\pmod{3}$ nên $7^t+2$ chia hết cho $3^{x-1}.$ Đồng thời, $3^{x-2}+2$ chia hết cho $7^t-2.$ Hai lập này cho ta biết
	    $$3^{x-2}+2\ge 7^t-2=7^z+2-4\ge 3^{x-1}-4.$$
	    Chỉ có $x=2$ và $x=3$ mới thỏa mãn nhận xét trên. Thế ngược lại, ta lần lượt tìm ra $t=1,y=2$ và $z=11$ trong trường hợp $x=3.$
	    \item \chu{Trường hợp 2. }$z-2=3^{x-1}7^y,\ z^2+2z+4=3.$ Trường hợp này không cho ta $z$ dương.
	\end{itemize}
\end{enumerate}
Như vậy, phương trình đã cho có nghiệm nguyên dương duy nhất là $(x,y,z)=(3,2,11).$}
\end{gbtt} 

\begin{gbtt}
Giải phương trình nghiệm tự nhiên \[5^x7^y+4=3^z.\]
\loigiai{
Giả sử phương trình đã cho có nghiệm tự nhiên $(x,y,z).$ Với $x=1$, ta có $7^y+4=3^z$. Ta nhận thấy 
$$7^y+4\equiv2\pmod{3}, \qquad 3^z\equiv0,1\pmod{3}.$$
Điều này không thể xảy ra. Ta suy ra $x\ge 1$. Lấy đồng dư theo modulo $5$ hai vế, ta có
$$3^z-4=5^x7^y\equiv0\pmod{5}\Rightarrow3^z\equiv4\pmod{5}.$$
Từ đây, ta tiếp tục suy ra $z=2k$ với $k$ là số tự nhiên. Thế $z=2k$ vào phương trình đã cho, ta được
$$5^x7^y+4=3^{2k}\Rightarrow 5^x7^y=\tron{3^k-2}\tron{3^k+2}.$$
Khoảng cách bằng $4$ giữa hai số lẻ $3^k-2,3^k+2$ chứng tỏ $\tron{3^k-2,3^k+2}=1$. \\
Ngoài ra, ta còn có $3^k-2<3^k+2.$ Nhận xét này cho phép ta xét các trường hợp sau.
\begin{enumerate}
    \item Với $3^k-2=1$ và $3^k+2=5^x7^y,$ ta lần lượt tìm được $k=1,x=1,y=0,z=2.$
    \item Với $3^k-2=5^x$ và $3^k+2=7^y,$ ta có
    $3^k+2=7^y\equiv 1\pmod{3},$ vô lí.
    \item Với $3^k-2=7^y$ và $3^k+2=5^x,$ ta nhận thấy rằng $7^y+4=5^x.$ Ta có
    $$5^x\equiv1\pmod{4}\Rightarrow 7^y+4\equiv1\pmod{4}\Rightarrow7^y\equiv1\pmod{4}.$$
    Suy ra $y$ chẵn. Ta đặt $y=2y_1.$ Ta xét các trường hợp sau
    \begin{itemize}
        \item \chu{Trường hợp 1.} Với $x\ge 2$, ta có $5^y\equiv0\pmod{25}$, kéo theo $7^{2y_1}+4\equiv0\pmod{25}$.\\ Điều này không thể xảy ra.
        \item \chu{Trường hợp 2.} Với $x=1$, ta có $3^k+2=5$ từ đây, ta tìm ra $k=1,z=2,y=0.$
    \end{itemize}
\end{enumerate}
Như vậy, phương trình đã cho có nghiệm tự nhiên duy nhất là $(1,0,2)$.}
\end{gbtt}

\begin{gbtt}
Giải phương trình nghiệm nguyên dương \[2^{x+1}3^y+5^z=7^t.\]
\loigiai{
Giả sử phương trình đã cho có nghiệm nguyên dương $(x,y,z,t).$ Ta có một vài nhận xét sau đây.
\begin{enumerate}[i,]
\item $z$ là số chẵn. Thật vậy, nếu $z$ là số lẻ thì $2^{x+1}3^y+5^z$ chia $3$ dư $2$, trong khi đó $7^t$ chia $3$ dư $1$, vô lí.
\item $t$ cũng là số chẵn. Thật vậy, nếu $t$ là số lẻ thì $7^t$ chia $4$ dư $3$, mà $2^{x+1}3^y+5^z$ chia $4$ dư $1,$ vô lí.
\end{enumerate}
Đặt $t=2a,z=2b.$ Phép đặt này cho ta \[2^x3^y=\left(7^a-5^b\right)\left(7^a+5^b\right).\]
Ta cũng dẫn dàng nhận xét được $\left(\dfrac{7^a-5^b}{2},\dfrac{7^a+5^b}{2}\right)=1,$ lại do $\dfrac{7^a-5^b}{2}\cdot \dfrac{7^a+5^b}{2}=2^{x-1}3^y$ nên ta xét các trường hợp sau.
\begin{enumerate}
    \item Nếu $\dfrac{7^a-5^b}{2}=1$ và $\dfrac{7^a+5^b}{2}=2^{x-1}3^y,$ ta có $$7^a=2+5^b.$$ Áp dụng \chu{bài \ref{phannguyen5/8}}, ta tìm được $a=1,\ b=1.$ Lúc này $x=2,\ y=1,\ z=2,\ t=2.$
    \item Nếu $\dfrac{7^a-5^b}{2}=2^{x-1}$ và $\dfrac{7^a+5^b}{2}=3^y,$ ta có
    $$7^a-5^b=2^x.$$ 
    Áp dụng \chu{bài \ref{phannguyen5/8}}, ta được $a=b=x=1.$ Khi đó $12=3^y,$ vô lí.
    \item Nếu $\dfrac{7^a-5^b}{2}=3^y$ và $\dfrac{7^a+5^b}{2}=2^{x-1},$ ta có
    $$5^b=2^{x-1}-3^y,\quad 7^a=3^y+2^{x-1}.$$ Vì $7^a\equiv 1\pmod 3$ nên $2^{x-1}\equiv 1\pmod 3$ nên $x-1$ chẵn và $5^b\equiv 2^{x-1}-3^y\equiv 1\pmod 3,$ điều này dẫn đến $b$ cũng là số chẵn. Ta phân tích
    \[3^y=\left(2^{\frac{x-1}{2}}-5^{\frac{b}{2}}\right)\left(2^{\frac{x-1}{2}}+5^{\frac{b}{2}}\right).\] Vì $\left(2^{\frac{x-1}{2}}-5^{\frac{b}{2}},2^{\frac{x-1}{2}}+5^{\frac{b}{2}}\right)=1$  nên $2^{\frac{x-1}{2}}-5^{\frac{b}{2}}=1,\ 2^{\frac{x-1}{2}}+5^{\frac{b}{2}}=3^y.$ Khi đó $2^{\frac{x-1}{2}}-3^y=1.$ \\Áp dụng kết quả \chu{bài \ref{bai4mu}}, ta được $\dfrac{x-1}{2}=2,\ y=1,$ suy ra $5^b=1$ và $b=0,$ mâu thuẫn.
    \item Nếu $\dfrac{7^a-5^b}{2}=2^{x-1}3^y$ và $\dfrac{7^a+5^b}{2}=1,$ ta có
    $$7^a+5^b=2.$$ 
    Ta tìm ra $a=b=0$ nên $z=t=0,$ vô lí.
\end{enumerate}
Kết luận, phương trình có nghiệm nguyên dương duy nhất là $(2,1,2,2).$}
\end{gbtt}

\begin{gbtt}
Cho các số nguyên dương $m,n,k,s$ thỏa mãn
$$\tron{3^m-3^n}^2=2^k+2^s.$$
Tìm giá trị lớn nhất của tích $mnks.$
\loigiai{
Không mất tính tổng quát, ta giả sử $m>n$ và $k>s.$ Từ đây, ta có
$$3^{2n}\tron{3^{m-n}-1}^2=2^s\tron{2^{k-s}+1}.$$
Vì $2^s$ là số chẵn nên ta thu được
$$3^{2n}a=2^{k-s}+1,\quad \tron{3^{m-n}-1}^2=2^sa$$
với $a\in\mathbb{N}.$ Vì $3^{2n}a=2^{k-s}+1$ nên $a$ là số lẻ. Kết hợp với $\tron{3^{m-n}-1}^2=2^sa,$ ta suy ra $s=2l$ và $a=b^2$ với $l,b$ là số tự nhiên. Thế trở lại, ta có
$$3^{2n}b^2=2^{k-s}+1,\qquad 3^{m-n}=2^lb+1.$$
Chuyển vế phương trình đầu cho ta
$$\tron{3^nb-1}\tron{3^nb+1}=2^{k-s}.$$
Đặt $3^nb-1=2^{\alpha}$ và $3^nb+1=2^{\beta}$ với $\alpha,\beta\in\mathbb{N}.$ Điều này chỉ ra
$$2^{\beta}-2^{\alpha}=2.$$
Giải phương trình trên, ta nhận được $\alpha=1$ và $\beta=2.$ Kéo theo $b=1,n=1$ và $k-s=3.$ Do đó
$3^{m-2}=2^l+1,$
với $s=2l.$ Ta xét các trường hợp sau.
\begin{enumerate}
    \item Với $m-1$ là số chẵn, ta suy ra $m-1=2, l=3.$ Điều này dẫn tới $$m=3,\quad n=1,\quad s=6,\quad k=9,\quad mnks=162.$$
    \item Với $m-1$ là số lẻ, ta suy ra $3^{m-1}$ chia $4$ dư $3.$ Do đó $l=1$ và kéo theo
    $$ m=2, \quad s=2,\quad k=5,\quad mnks=20.$$
\end{enumerate}
Như vậy, giá trị lớn nhất của $mnks$ là $162.$
}

\end{gbtt}

\begin{gbtt}
Tìm tất cả nghiệm nguyên dương của phương trình $x!+5=y^{z+1}.$
\loigiai{
Nếu $x\ge 10,$ ta lần lượt suy ra
$$5\mid x!\Rightarrow 5\mid y^{z+1}\Rightarrow 5\mid y\Rightarrow 25\mid y^{z+1}\Rightarrow 25\mid \tron{x!+5}.$$
Đây là điều vô lí. Do đó, $x<10.$ Thử lại, phương trình có nghiệm nguyên dương duy nhất là $(4,5,2).$}
\end{gbtt}

\begin{gbtt}
Giải phương trình nghiệm nguyên dương \[x!+5^y=7^z.\]
\loigiai{
Giả sử phương trình đã cho có nghiệm nguyên dương $(x,y,z).$ Ta xét các trường hợp sau đây.
\begin{enumerate}
    \item Nếu $x\ge 5,$ ta có $7^z=x!+5$ chia hết cho $5,$ vô lí.
    \item Nếu $x=1,$ ta có $x!+5^y=1+5^y$ là số chẵn, dẫn đến $7^z$ là số chẵn, vô lí.
    \item Nếu $x=2,$ ta có $2+5^y=7^z.$ Bài toán quen thuộc này cho ta kết quả $y=z=1.$
    \item Nếu $x=3,$ ta có $6+5^y=7^z.$ Bài toán quen thuộc này không cho ta $y,z$ nguyên.
    \item Nếu $x=4,$ ta có $24+5^y=7^z.$ Bài toán quen thuộc này cho ta $y=z=2.$
\end{enumerate}
Kết luận, phương trình có $2$ nghiệm nguyên dương là $(2,1,1)$ và $(4,2,2).$}
\end{gbtt}

\begin{gbtt}
Tìm nghiệm nguyên dương của phương trình 
\[1!+2!+\cdots+(x+1)!=y^{z+1}.\]
\loigiai{
Đặt $f(x)=1!+2!+\cdots+(x+1)!.$ Ta xét các trường hợp sau.
\begin{enumerate}
    \item Với $z=1,$ ta có $f(x)$ là số chính phương. Nếu như $x\ge 4$ thì
    $$VT\equiv f(3)\equiv 2\pmod{5}.$$
    Không có số chính phương nào chia $2$ dư $5,$ thế nên $x\le 3.$ Thử trực tiếp, ta tìm ra $(x,y,z)=(2,3,1).$
    \item Với $z\ge 2,$ ta có $z+1\ge 3.$ Từ $f(x)=y^{z+1},$ ta suy ra $y^3\mid f(x).$ Nếu $x\ge 8$ thì
    $$VT\equiv f(7)\equiv 9\pmod{27}.$$
    Ta có $y$ chia hết cho $3$ nhưng $y^{z+1}\equiv 9\pmod{27},$ mâu thuẫn. \\ Như vậy $x\le 7.$ Thử trực tiếp, ta không tìm được $y$ và $z$ tương ứng. 
\end{enumerate}
Kết luận, phương trình đã cho có nghiệm nguyên dương duy nhất là $(x,y,z)=(2,3,1).$}
\end{gbtt}

\begin{gbtt}
Tìm tất cả các số tự nhiên $p,q,r,s>1$ thỏa mãn
$$p!+q!+r!=2^s.$$
\nguon{Indian IMO Training Camp 2017}
\loigiai{
Không mất tổng quát, ta giả sử $p\ge q\ge r.$ Nếu $r\ge 3,$ ta có
$$2^s=p!+q!+s!\equiv 0 \pmod{3}.$$
Đây là một điều vô lí vì $\left(2^s,3\right)=1.$ Ta có $r=2.$ Thế $r=2$ vào phương trình ban đầu, ta được
\[p!+q!+2=2^s.\tag{*}\]
Lấy đồng dư theo modulo $4$ hai vế, ta có
$$p!+q!+2\equiv 0\pmod{4}.$$
Rõ ràng, $p\ge q\ge 4$ là điều không thể xảy ra, vì lúc này, $p!+q!$ chia hết cho $4.$\\
Mâu thuẫn trên chứng tỏ $q\le 3.$ Ta xét các trường hợp sau.
\begin{enumerate}
    \item Với $q=3,$ thế trở lại vào (*), ta được
    $$p!+8=2^s\Leftrightarrow p!=2^s-8.$$
    Lấy đồng dư theo modulo $16$ hai vế, ta suy ra
    $$p!\equiv 2^s-8\equiv 8\left(2^{s-3}-1\right)\equiv 8\pmod{16}.$$
    Ta suy ra $p=4$ hoặc $p=5$ từ đây.
    \begin{itemize}
        \item\chu{Trường hợp 1.} Với $p=4,$ ta có $2^s=4!+3!+2!=32,$ thế nên $s=5.$
        \item\chu{Trường hợp 2.} Với $p=5,$ ta có $2^s=5!+3!+2!=128,$ thế nên $s=7.$
    \end{itemize}
    \item Với $q=2,$ thế trở lại vào (*), ta được
    $$p!+4=2^s\Leftrightarrow p!=2^s-4.$$
    Lấy đồng dư theo modulo $8$ hai vế, ta suy ra
    $$p!\equiv 2^s-4\equiv 4\left(2^{s-2}-1\right)\equiv 4\pmod{8}.$$
    Ta không tìm được $p$ từ đây.
\end{enumerate}
Tổng kết lại, có tất cả 12 bộ $(p,q,r,s)$ thỏa mãn đề bài, đó là $$(2,3,4,5),(2,3,5,7)$$ và các hoán vị theo $(p,q,r)$ của chúng.}
\end{gbtt} %pt chứa ẩn ở mũ
\section{Phương trình chứa căn thức}

\subsection*{Bài tập tự luyện}
\begin{btt}
Giải phương trình nghiệm nguyên \[\sqrt{9x^2+16x+96}=3x-16y-24.\]
\end{btt}

\begin{btt}
Giải phương trình nghiệm nguyên \[y^2=1+\sqrt{9-x^2-4x}.\]
\end{btt}

\begin{btt}
Giải phương trình nghiệm nguyên $$xy-7\sqrt{x^2+y^2}=1.$$
\nguon{Adrian Adreescu} 
\end{btt}

\begin{btt}
Giải phương trình nghiệm nguyên dương $$xy+yz+zx-5\sqrt{x^2+y^2+z^2}=1.$$
\nguon{Titu Andreescu}
\end{btt}

\begin{btt}
Giải phương trình nghiệm tự nhiên $$2\sqrt{x}-3\sqrt{y}=\sqrt{48}.$$
\end{btt}

\begin{btt}
Giải phương trình nghiệm nguyên $$\sqrt{x}+\sqrt{x+3}=y.$$
\end{btt}
 
\begin{btt}
Giải phương trình nghiệm nguyên dương $$\sqrt{x}+\sqrt{y}=\sqrt{1980}.$$
\end{btt}

\begin{btt}
Giải phương trình nghiệm nguyên dương
$$\sqrt{x}+\sqrt{y}=\sqrt{z+2\sqrt{2}}.$$
\end{btt}

\begin{btt}
Giải phương trình nghiệm tự nhiên
\[xy+3x+\sqrt{4x-1}=\sqrt{y+2}+4y.\]
\end{btt}

\begin{btt}
Tìm tất cả các số nguyên dương $x,y$ thỏa mãn
\[1+\sqrt{x+y+3}=\sqrt{x}+\sqrt{y}.\]
\nguon{Chuyên Khoa học Tự nhiên 2015}
\end{btt}

\begin{btt}
Tìm tất cả các số hữu tỉ $x,y$ thỏa mãn
$$\sqrt{2\sqrt{3}-3}=\sqrt{3x\sqrt{3}}-\sqrt{y\sqrt{3}}.$$
\nguon{Chọn học sinh giỏi Vĩnh Phúc 2012 $-$ 2013}
\end{btt}

\begin{btt}
Giải phương trình nghiệm tự nhiên
\[2x\sqrt{x}-2y\sqrt{y}=7\sqrt{xy}.\]
\end{btt}

\begin{btt}
Giải phương trình nghiệm nguyên dương
\[\sqrt{4x^3+14x^2+3xy-2y}+\sqrt{y^2-y+3}=z.\]
\end{btt} 

\begin{btt}
Giải phương trình nghiệm nguyên
\[\sqrt{x+\sqrt{x+\sqrt{x+\sqrt{x}}}}=y.\]
\end{btt}

\begin{btt}
Giải phương trình nghiệm nguyên
\[y=\sqrt[3]{2+\sqrt{x}}+\sqrt[3]{2-\sqrt{x}}.\]
\end{btt}

\begin{btt}
Giải phương trình nghiệm nguyên \[\dfrac{4}{y}+\sqrt[3]{4-x}=\sqrt[3]{4+4\sqrt{x}+x}+\sqrt[3]{4-4\sqrt{x}+x}.\]
\end{btt}

\begin{btt}
Tìm bộ số nguyên dương $(x,y,z)$ nhỏ nhất thỏa mãn điều kiện
\[\dfrac{\sqrt{x}}{2}=\dfrac{\sqrt[3]{y}}{3}=\dfrac{\sqrt[5]{z}}{5}.\]
\end{btt}

\begin{btt}
Tìm tất cả các số nguyên $x,y$ khác $0$ thỏa mãn 
\[\dfrac{1}{\sqrt[3]{x}}+\dfrac{2}{\sqrt{y}}=\dfrac{4}{9}.\]
\end{btt}


\begin{btt}
Tìm tất cả các số nguyên tố $p,q,r,s,t$ thỏa mãn $$p+\sqrt{q^{2}+r}=\sqrt{s^{2}+t}.$$
\nguon{Kazakhstan Mathematical Olympiad 2012, Grade 9}
\end{btt}

\subsection*{Hướng dẫn bài tập tự luyện}
\begin{gbtt}
Giải phương trình nghiệm nguyên \[\sqrt{9x^2+16x+96}=3x-16y-24.\]
\loigiai{
Với điều kiện $3x-16y-24\ge 0,$ phương trình đã cho tương đương
\begin{align*}
9x^2+16x+96=(3x-16y-24)^2
&\Leftrightarrow  81x^2+9\cdot 16x+864=9(3x-16y-24)^2
\\&\Leftrightarrow (9x+8)^2-(3(3x-16y-24))^2=-800
\\&\Leftrightarrow (18x-48y-64)(48y+80)=-800
\\&\Leftrightarrow(9x-24y-32)(3y+5)=-25.
\end{align*}
Do $3y+5$ là ước của $25$ và chia cho $3$ dư $2,$ ta lập được bảng giá trị sau đây
\begin{center}
\begin{tabular}{c|c|c|c}
$3y+5$ & $-1$ & $5$ & $-25$ \\ 
\hline 
$y$ & $-2$ & $0$ & $-10$ \\ 
\hline 
$x$ & $1$ & $3$ & $-23$ \\ 
\end{tabular} 
\end{center}
Đối chiếu với $3x-16y-24\ge 0,$ phương trình có hai nghiệm nguyên thỏa mãn là 
$$(-23,-10),\quad (1,-2).$$}
\end{gbtt}

\begin{gbtt}
Giải phương trình nghiệm nguyên $y^2=1+\sqrt{9-x^2-4x}.$
\loigiai{
Với $x$ nguyên, điều kiện xác định của phương trình là $-5\le x\le 1.$ \\
Giả sử phương trình có nghiệm $(x,y).$ Ta nhận thấy rằng
$$y^2=1+\sqrt{13-(x+2)^2}\le 1+\sqrt{13},$$
thế nên $0\le y^2\le 4.$
\begin{enumerate}
    \item Với $y=0,$ ta tìm được $x=\pm 2.$
    \item Với $y^2=1,$ ta tìm được $x=-2\pm \sqrt{13},$ mâu thuẫn.
    \item Với $y^2=4,$ ta tìm được $x=0$ hoặc $x=-4.$
\end{enumerate}
Kết luận, phương trình đã cho có đúng bốn nghiệm nguyên là $$(0,-2),(0,2),(-4,2),(-4,-2).$$
}
\end{gbtt}

\begin{gbtt}
Giải phương trình nghiệm nguyên $xy-7\sqrt{x^2+y^2}=1.$
\nguon{Adrian Adreescu} 
\loigiai{
Ta sẽ chỉ cần tìm các nghiệm dương của phương trình.\\
Với điều kiện $xy\ge 1,$ phương trình đã cho tương đương với
\begin{align*}
    \left(xy-1\right)^2=49\left(x^2+y^2\right)
    & \Leftrightarrow \left(xy-1\right)^2+98xy=49(x+y)^2
    \\&\Leftrightarrow (xy)^2+96xy+1=49(x+y)^2
    \\&\Leftrightarrow (xy)^2+96xy+2304=49(x+y)^2+2303
    \\&\Leftrightarrow (xy+48)^2=(7x+7y)^2+2303
    \\&\Leftrightarrow  (xy-7x-7y+48)(xy+7x+7y+48)=2303.
\end{align*}
Do $1\le xy-7x-7y+48<xy+7x+7y+48$ và $2303=7^2.47,$ ta xét các trường hợp sau
\begin{itemize}
    \item \chu{Trường hợp 1.} $xy-7x-7y+48=1$ và  $xy+7x+7y+48=2303.$
    \item \chu{Trường hợp 2.} $xy-7x-7y+48=7$ và  $xy+7x+7y+48=329.$
    \item \chu{Trường hợp 3.} $xy-7x-7y+48=47$ và  $xy+7x+7y+48=49.$
\end{itemize}
Giải mỗi hệ trên, ta chỉ ra được các cặp $(x,y)$ thỏa mãn đề bài là $(8,15)$ và $(15,8).$ \\
Kết luận, phương trình có $4$ nghiệm nguyên là $(-15,-8),(-8,-15),(8,15)$ và $(15,8).$}
\end{gbtt}

\begin{gbtt}
Giải phương trình nghiệm nguyên dương \[xy+yz+zx-5\sqrt{x^2+y^2+z^2}=1.\]
\nguon{Titu Andreescu}
\loigiai{
Đặt $x+y+z=u,\sqrt{x^2+y^2+z^2}=v.$ Ta có $xy+yz+zx=\dfrac{u^2-v^2}{2}.$ \\ 
Phương trình đã cho trở thành
\begin{align*}
    \dfrac{u^2-v^2}{2}-5v=1
    &\Leftrightarrow u^2=v^2+10v+2
    \Leftrightarrow u^2+23=(v+5)^2
    \\&\Leftrightarrow (v+5-u)(v+5+u)=23.
\end{align*}
Do $23$ là số nguyên tố và $0< v+5-u<v+5+u,$ chỉ có trường hợp $v+5-u=1$ và $v+5+u=23$ là xảy ra, thế nên $v=7,$ và
$$x^2+y^2+z^2=49.$$
Chỉ có một cách biểu diễn $49$ dưới dạng tổng ba số chính phương, đó là 
$$49=6^2+3^2+2^2.$$
Thử với các hoán vị của $(2,3,6),$ ta thấy tất cả chúng đều thỏa phương trình, và đây cùng là toàn bộ nghiệm nguyên của phương trình đã cho.
}
\end{gbtt}

\begin{gbtt}
Giải phương trình nghiệm tự nhiên $2\sqrt{x}-3\sqrt{y}=\sqrt{48}.$
\loigiai{
Giả sử phương trình đã cho có nghiệm $(x,y).$ Chuyển vế và bình phương, ta có
\begin{align*}
    4x^2=9y^2+6y\sqrt{48}+48.
\end{align*}
Nếu như $y=0,$ ta tìm được $x=12.$ Còn đối với $y\ne 0,$ ta có
$$\sqrt{48}=\dfrac{4x^2-9y^2-48}{6y}.$$
Vế trái là số vô tỉ, trong khi vế phải hữu tỉ, mâu thuẫn.\\
Kết luận, $(x,y)=(12,0)$ là nghiệm duy nhất của phương trình.
}
\end{gbtt}

\begin{gbtt}
Giải phương trình nghiệm nguyên $\sqrt{x}+\sqrt{x+3}=y.$
\loigiai{
Giả sử phương trình đã cho có nghiệm $(x,y).$ \\
Theo như bổ đề đã học ở \chu{chương III}, cả $x$ và $x+3$ là số chính phương. Ta đặt
$$x+3=z^2,\quad x=t^2,$$
trong đó $z$ và $t$ là hai số nguyên dương. Trừ theo vế, ta được
$$(z-t)(z+t)=3.$$
Do $0<z-t<z+t,$ ta suy ra $z-t=1$ và $z+t=3.$ Ta tính được $z=2,$ kéo theo $x=1.$ \\
Kết luận, phương trình có nghiệm nguyên duy nhất là $(x,y)=(1,3).$
}
\end{gbtt}

Tác giả xin phép nhắc lại kiến thức đã học ở \chu{chương III}. 
\begin{light}
Với các số tự nhiên $a,b,$ ta có các khẳng định sau.
\begin{enumerate}
    \item Nếu $\sqrt{a}+\sqrt{b}$ là số tự nhiên thì $a,b$ là các số chính phương.
    \item Nếu $\sqrt{a}-\sqrt{b}$ là số tự nhiên thì hoặc $a,b$ là các số chính phương, hoặc $a=b.$
    \item Nếu $\sqrt{a}$ là số hữu tỉ thì $a$ là số chính phương.  
\end{enumerate}    
Tương tự, với các số hữu tỉ không âm $a,b,$ ta có các khẳng định sau.
\begin{enumerate}
    \item Nếu $\sqrt{a}+\sqrt{b}$ là số tự nhiên thì $a,b$ là bình phương các số hữu tỉ.
    \item Nếu $\sqrt{a}-\sqrt{b}$ là số hữu tỉ thì hoặc $a,b$ là bình phương các số hữu tỉ, hoặc $a=b.$
    \item Nếu $\sqrt{a}$ là số hữu tỉ thì $a$ là bình phương một số hữu tỉ.     \end{enumerate}
Đối với một số khẳng định tương tự dạng căn bậc cao hơn và phần chứng minh chúng, mời bạn đọc tự nghiên cứu.
 \end{light}
 
\begin{gbtt}\label{laoth1}
Giải phương trình nghiệm nguyên dương $\sqrt{x}+\sqrt{y}=\sqrt{1980}.$
\loigiai{
Giả sử phương trình đã cho có nghiệm $(x,y).$ Chuyển vế và bình phương, ta có
\begin{align*}
    x=1980+y-2\sqrt{1980y}
    &\Rightarrow x=1980+y-12\sqrt{55y}
    \\&\Rightarrow \sqrt{55y}=\dfrac{y+1980-x}{12}.
\end{align*}
Do $\sqrt{55y}$ là số hữu tỉ nên theo như bổ đề, ta chỉ ra $55y$ là số chính phương, vậy nên $y$ phải là $55$ lần một số chính phương. Chứng minh tương tự, $x$ cũng là $55$ lần một số chính phương.\\
Ta đặt $x=55z^2,y=55t^2,$ với $z,t$ là hai số nguyên dương. Khi ấy
$$z\sqrt{55}+t\sqrt{55}=6\sqrt{55}\Rightarrow z+t=6.$$
Không mất tổng quát, ta giả sử $y\le x$ và $a\le b.$ Ta lập được bảng giá trị sau
\begin{center}
\begin{tabular}{c|c|c|c}
$a$ & $b$ & $x=55a^2$ & $y=55b^2$ \\ 
\hline 
$1$ & $5$ & $55$ & $1375$ \\ 
\hline 
$2$ & $4$ & $220$ & $880$ \\ 
\hline 
$3$ & $3$ & $495$ & $495$ 
\end{tabular} 
\end{center}
Dựa theo bảng giá trị, ta kết luận phương trình đã cho có các nghiệm nguyên là $$(55,1375),(1375,55),(220,880),(880,220),(495,495).$$
}
\end{gbtt}

\begin{gbtt}
Giải phương trình nghiệm nguyên dương
$\sqrt{x}+\sqrt{y}=\sqrt{z+2\sqrt{2}}.$
\loigiai{
Phương trình đã cho tương đương với
$$x+y+2\sqrt{xy}=z+2\sqrt{2}\Leftrightarrow x+y-z=\sqrt{8}-\sqrt{4xy}.$$
Hiệu hai căn thức $\sqrt{8}$ và $\sqrt{4xy}$ là một số nguyên, vậy nên ta chia bài toán làm các trường hợp sau.
\begin{enumerate}
    \item Nếu $4xy=8$ và $x+y=z,$ ta tìm được các bộ $(x,y,z)=(1,2,3)$ và $(x,y,z)=(2,1,3).$
    \item Nếu $4xy$ và $8$ là số chính phương, ta thấy ngay điều mâu thuẫn.
\end{enumerate}
Kết luận, $(x,y,z)=(1,2,3)$ và $(x,y,z)=(2,1,3)$ là hai nghiệm nguyên dương của phương trình.}
\end{gbtt}

\begin{gbtt}
Giải phương trình nghiệm tự nhiên
\[xy+3x+\sqrt{4x-1}=\sqrt{y+2}+4y.\]
\loigiai{
Giả sử phương trình đã cho có nghiệm tự nhiên $(x,y).$ Giả sử này cho ta
$$\sqrt{4x-1}-\sqrt{y+2}=4y-xy-3x.$$
Theo như bổ đề đã học, ta chia bài toán làm hai trường hợp sau.
\begin{enumerate}
    \item Nếu $4x-1=y+2,$ ta có
    $$\heva{4x-1&=y+2 \\ 4y&=xy-3x}\Rightarrow \heva{y&=4x-3 \\ 4(4x-3)&=x(4x-3)+3x}\Rightarrow\hoac{x&=1,y=1 \\ x&=3,y=9.}$$
    \item Nếu $4x-1\ne y+2,$ ta có cả $4x-1$ và $y+2$ là số chính phương, vô lí do $4x-1\equiv 3\pmod{4}.$
\end{enumerate}
Như vậy, phương trình đã cho có hai nghiệm tự nhiên là $(1,1)$ và $(3,9).$}
\end{gbtt}

\begin{gbtt}
Tìm tất cả các số nguyên dương $x,y$ thỏa mãn
\[1+\sqrt{x+y+3}=\sqrt{x}+\sqrt{y}.\]
\nguon{Chuyên Khoa học Tự nhiên 2015}
\loigiai{Giả sử phương trình đã cho có nghiệm $(x,y)$ thỏa mãn. Ta có
\begin{align*}
    1+\sqrt{x+y+3}=\sqrt{x}+\sqrt{y}
    &\Rightarrow \left(1+\sqrt{x+y+3}\right)^2=\left(\sqrt{x}+\sqrt{y}\right)^2
    \\&\Rightarrow 1+x+y+3+2\sqrt{x+y+3}=x+y+2\sqrt{xy}
    \\&\Rightarrow 4+2\sqrt{x+y+3}=2\sqrt{xy}
    \\&\Rightarrow \sqrt{xy}-\sqrt{x+y+3}=2.
\end{align*}
Theo như nội dung bổ đề, ta cần phải xét hai trường hợp từ đây.
\begin{enumerate}
    \item Nếu $xy=x+y+3,$ ta có $0=2,$ vô lí.
    \item Nếu cả $xy$ và $x+y+3$ đều chính phương, kết hợp $x+y+3$ chính phương với phương trình đã cho, ta suy ra được $x,y$ đều là các số chính phương. Đặt $x=a^2,y=b^2$, với $a,b$ là các số nguyên dương. \\
    Từ $\sqrt{xy}-\sqrt{x+y+3}=2,$ ta có $\sqrt{x+y+3}=ab-2.$ Phương trình đã cho trở thành
    $$ab-1=a+b \Leftrightarrow (a-1)(b-1)=2.$$ 
    Phương trình ước số trên cho ta hai nghiệm là $(2,3)$ và $(3,2).$\\
    Thế ngược lại, ta tìm được $(4,9)$ và $(9,4)$ là hai cặp số  thỏa mãn đề bài.
\end{enumerate}}
\end{gbtt}

\begin{gbtt}
Tìm tất cả các số hữu tỉ $x,y$ thỏa mãn
$$\sqrt{2\sqrt{3}-3}=\sqrt{3x\sqrt{3}}-\sqrt{y\sqrt{3}}.$$
\nguon{Chọn học sinh giỏi Vĩnh Phúc 2012 $-$ 2013}
\loigiai{Giả sử tồn tại cặp số $(x,y)$ hữu tỉ thỏa mãn đẳng thức. Ta có
\begin{align*}
\sqrt{2\sqrt{3}-3}=\sqrt{3x\sqrt{3}}-\sqrt{y\sqrt{3}}  
&\Rightarrow 2 \sqrt{3}-3=3 x \sqrt{3}+y \sqrt{3}-6 \sqrt{x y} 
\\& \Rightarrow\sqrt{3}(3x+y-2)=6 \sqrt{x y}-3 
\\& \Rightarrow 3(3x+y-2)^{2}=36 x y-36 \sqrt{x y}+9
\\&\Rightarrow \sqrt{xy}=\dfrac{12 x y+3-(3 x+y-2)^{2}}{12}.
\tag{*}\label{vp1213}
\end{align*}
Do $x, y$ là các số hữu tỉ, nên từ $(\ref{vp1213})$ ta suy ra $\sqrt{xy}$ là số hữu tỉ. 
\begin{enumerate}
    \item Nếu $3x+y-2 \ne 0$, vế trái của $\sqrt{3}(3x+y-2)=6 \sqrt{x y}-3$ là một số vô tỉ, trong khi vế phải của nó hữu tỉ, điều này vô lí.
    \item Nếu $3x+y-2=0$, kết hợp với $(*),$ ta được
    \begin{align*}
    \heva{&3x+y=2 \\ &12\sqrt{xy}=12xy+3}
    &\Leftrightarrow \heva{&3x+y=2 \\ &3\left(2\sqrt{xy}-1\right)^2=0}\\&
    \Leftrightarrow \heva{&y=2-3x \\ &xy=\dfrac{1}{4}}
    \\&\Leftrightarrow \heva{&y=2-3x \\ &4x(2-3x)-1=0} \\&
    \Leftrightarrow \heva{&x=\dfrac{1}{2},y=\dfrac{1}{2} \\ &x=\dfrac{1}{6},y=\dfrac{3}{2}.}    
    \end{align*}
\end{enumerate}
Thử lại, ta nhận thấy chỉ có $x=y=\dfrac{1}{2}$ là thỏa mãn. Bài toán kết thúc.
}
\end{gbtt}

\begin{gbtt}
Giải phương trình nghiệm tự nhiên
\[2x\sqrt{x}-2y\sqrt{y}=7\sqrt{xy}.\]
\loigiai{
Giả sử phương trình đã cho tồn tại nghiệm tự nhiên $(x,y)$ thỏa mãn.\\
Ta bình phương hai vế của phương trình đã cho và nhận được
$$4x^3-8xy\sqrt{xy}+4y^3=49xy.$$
Từ đây, ta suy ra $8xy\sqrt{xy}=4x^3+4y^3-49xy.$ Vì $x,y$ là số tự nhiên nên $\sqrt{xy}$ là số tự nhiên. \\
Thế trở lại phương trình đã cho, theo bổ đề, ta có $2$ trường hợp sau.
\begin{enumerate} 
    \item Với $x\sqrt{x}=y\sqrt{y},$ ta có $x=y=0.$
    \item Với $x,y$ là số chính phương khác $0,$ ta đặt $x=u^2, y=v^2.$ Phương trình đã cho trở thành
    $$2u^3-2v^3=7uv.$$
    Đặt $u=dm$ và $v=dn$ trong đó $m,$ $n$ là các số tự nhiên sao cho $\tron{m,n}=1$. Phép đặt này cho ta
    $$2d^3m^3-2d^3n^3=7d^2mn.$$
    Phương trình kể trên tương đương với
    \[2d\tron{m^3-n^3}=7mn.\tag{*}\label{pttn1}\]
    Theo như kiến thức đã học ở \chu{chương I}, ta chứng minh được $\tron{mn,m^3-n^3}=1.$ Kết hợp với (\ref{pttn1}), ta chỉ ra $7$ chia hết cho $m^3-n^3.$ Ta xét các trường hợp sau.
    \begin{itemize}
        \item \chu{Trường hợp 1.} Với $m^3-n^3=1$, ta có
        $$\tron{m-n}\tron{m^2+mn+n^2}=1.$$
        Biến đổi kể trên cho ta $m-n=m^2-mn+n^2=1,$ thế nên $m=1,n=0.$ Thế trở lại (\ref{pttn1}), ta được $1d=0,$ vô lí.
        \item \chu{Trường hợp 2.} Với $m^3-n^3=7$, ta có
        $$\tron{m-n}\tron{m^2+mn+n^2}=7.$$
        Do $m-n<m<m^2+mn+n^2$ nên $$\heva{&m-n=1\\&m^2+mn+n^2=7.}$$ 
        Giải hệ, ta tìm ra $m=2,n=1,$ thế vào (\ref{pttn1}) thì $d=1$ và lúc này $x=4,y=1.$
    \end{itemize}
\end{enumerate}
Như vậy, phương trình đã cho có $2$ nghiệm tự nhiên là $(0,0)$ và $(4,1).$}
\end{gbtt}

\begin{gbtt}
Giải phương trình nghiệm nguyên dương
\[\sqrt{4x^3+14x^2+3xy-2y}+\sqrt{y^2-y+3}=z.\]
\loigiai{
Giả sử phương trình đã cho tồn tại nghiệm $(x,y,z)$ thỏa mãn. Theo như bổ đề đã biết, hai số
$$4x^3+14x^2+3xy-2y,\quad y^2-y+3$$ 
đều là chính phương. Ngoài ra, ta còn nhận thấy rằng
$$(y-1)^2< y^2-y+3< (y+1)^2.$$
Ta suy ra $y^2-y+3=y^2,$ hay là $y=3,$ và lúc này
$$4x^3+14x^2+3xy-2y=4x^3+14x^2+9x-6=(x+2)\left(4x^2+6x-3\right)$$
là một số chính phương. Ta đặt $d=\left(x+2,4x^2+6x-3\right).$ Phép đặt này cho ta
$$
\heva{&d\mid (x+2)\\ &d\mid \left(4x^2+6x-3\right)}
\Rightarrow
\heva{&d\mid (x+2)\\ &d\mid \left(2(x+2)(2x-1)+1\right)}
\Rightarrow d=1.
$$
Theo như kiến thức đã học ở \chu{chương III}, ta thu được $4x^2+6x-3$ là số chính phương. Do
$$(2x+1)^2\le 4x^2+6x-3< (2x+3)^2$$
và $4x^2+6x-3$ lẻ nên $4x^2+6x-3=(2x+1)^2,$ hay là $x=2.$ Với $x=2,y=3,$ thay trở lại, ta tìm được $z=13.$ Kết luận, phương trình đã cho có nghiệm nguyên dương duy nhất là $(x,y,z)=(2,3,13).$}
\end{gbtt} 
% fact
\begin{gbtt}\label{ptvt1}
Giải phương trình nghiệm nguyên\[\sqrt{x+\sqrt{x+\sqrt{x+\sqrt{x}}}}=y.\]
\loigiai{
Với điều kiện xác định là $x\ge 0,$ phương trình đã cho tương đương $$\sqrt{x+\sqrt{x+\sqrt{x}}}=y^2-x.$$
Đặt $y^2-x=z.$ Bình phương rồi chuyển vế, phương trình trở thành $$\sqrt{x+\sqrt{x}}=z^2-x.$$
Tiếp tục đặt $z^2-x=t.$ Bình phương hai vế, phương trình trở thành
$$x+\sqrt{x}=t^2.$$
Với giả sử phương trình có nghiệm nguyên dương, ta chỉ ra $\sqrt{x}$ là số tự nhiên, chứng tỏ $x$ là một số chính phương. Đặt $\sqrt{x}=u,$ ta có
$$u^2+u=t^2.$$
Bằng đánh giá $u^2\le u^2+u<(u+1)^2,$ ta chỉ ra $u=0,$ kéo theo $x=0.$\\
Kết luận, $(x,y)=(0,0)$ là nghiệm nguyên duy nhất của phương trình.
}
\end{gbtt}

\begin{gbtt}\label{laoth2}
Giải phương trình nghiệm nguyên
\[y=\sqrt[3]{2+\sqrt{x}}+\sqrt[3]{2-\sqrt{x}}.\]
\loigiai{
Giả sử phương trình đã cho có nghiệm nguyên. Ta nhận thấy rằng
$$y=\sqrt[3]{2+\sqrt{x}}+\sqrt[3]{2-\sqrt{x}}>\sqrt[3]{-2+\sqrt{x}}+\sqrt[3]{2-\sqrt{x}}=0,$$
thế nên $y>0.$  Lấy lập phương hai vế phương trình đã cho, ta được
\begin{align*}
    y^3=4+3\left(\sqrt[3]{2+\sqrt{x}}+\sqrt[3]{2-\sqrt{x}}\right)\sqrt[3]{4-x}
    \Rightarrow 
    y^3=4+3y\sqrt[3]{4-x}    
    \Rightarrow \sqrt[3]{4-x}=\dfrac{y^3-4}{3y}.
\end{align*}
Nhờ điều kiện xác định là $x\ge 0,$ ta chỉ ra
$$\dfrac{y^3-4}{3y}\le \sqrt[3]{4}.$$
Quy đồng, ta có $y^3\le\sqrt[3]{4}y+4,$ thế nên ta dễ dàng thu được $y\le 1$ nhờ vào phản chứng.  \\
Các nhận xét trên cho ta $0<y\le 1,$ vậy nên $y=1.$ Thế trở lại, ta tìm ra $x=5.$\\
Kết luận, $(x,y)=(5,2)$ là nghiệm nguyên duy nhất của phương trình đã cho.
}

\end{gbtt}
\begin{gbtt}
Giải phương trình nghiệm nguyên \[\dfrac{4}{y}+\sqrt[3]{4-x}=\sqrt[3]{4+4\sqrt{x}+x}+\sqrt[3]{4-4\sqrt{x}+x}.\]
\loigiai{
Đặt $\sqrt[3]{2+\sqrt{x}}=z,\sqrt[3]{2-\sqrt{x}}=t.$ Phương trình đã cho trở thành
$$\dfrac{z^3+t^3}{y}+zt=z^2+t^2.$$
Do $z$ và $t$ không đồng thời bằng $0,$ ta có
$$\dfrac{z^3+t^3}{y}=z^2-zt+t^2\Leftrightarrow \dfrac{(z+t)\left(z^2-zt+t^2\right)}{y}=z^2-zt+t^2\Leftrightarrow z+t=y.$$
Từ đó, ta thu được
$y=\sqrt[3]{2+\sqrt{x}}+\sqrt[3]{2-\sqrt{x}}.$ \\
Bài toán quen thuộc này cho ta kết quả $(x,y)=(1,5).$}
\end{gbtt}

\begin{gbtt}
Tìm bộ số nguyên dương $(x,y,z)$ nhỏ nhất thỏa mãn điều kiện
\[\dfrac{\sqrt{x}}{2}=\dfrac{\sqrt[3]{y}}{3}=\dfrac{\sqrt[5]{z}}{5}.\]
\loigiai{
Giả sử tồn tại bộ số nguyên dương $(x,y,z)$ thỏa mãn điều kiện. Lúc này
$$\tron{\dfrac{\sqrt{x}}{2}}^6=\tron{\dfrac{\sqrt[3]{y}}{3}}^6\Rightarrow \dfrac{x^3}{2^6}=\dfrac{y^2}{3^6}\Rightarrow3^6x^3=2^6y^2.$$
Vì $2^6y^2$ là số chính phương ta có $3^6x^3$ là số chính phương, và như vậy $x$ cũng là số chính phương.\\
Cũng từ lập luận $x$ chẵn, ta nhận thấy rằng $x\ge 4.$ Với $x=4,$ thế trở lại, ta có
$$\dfrac{\sqrt[3]{y}}{3}=\dfrac{\sqrt[5]{z}}{5}=1 \Rightarrow y=3^3=27, \quad z=5^5=3125.$$
Như vây, bộ ba số nguyên dương $(x,y,z)$ nhỏ nhất thỏa mãn điều kiện $\tron{4,27,3125}.$}
\end{gbtt}

\begin{gbtt}
Tìm tất cả các số nguyên $x,y$ khác $0$ thỏa mãn 
\[\dfrac{1}{\sqrt[3]{x}}+\dfrac{2}{\sqrt{y}}=\dfrac{4}{9}.\]
\loigiai{
Phương trình đã cho tương đương với
\begin{align*}
   \dfrac{1}{\sqrt[3]{x}}=\dfrac{4}{9}-\dfrac{2}{\sqrt{y}}
   &\Leftrightarrow \dfrac{1}{x}=\left(\dfrac{4}{9}-\dfrac{2}{\sqrt{y}}\right)^3
   \\&\Leftrightarrow \dfrac{1}{x}=\dfrac{64}{729}+\dfrac{16}{3y}-\dfrac{8}{y\sqrt{y}}-\dfrac{32}{27\sqrt{y}}
   \\&\Leftrightarrow\dfrac{1}{x}-\dfrac{64}{729}-\dfrac{16}{3y}=-\dfrac{8\sqrt{y}}{y^2}-\dfrac{32\sqrt{y}}{27y}
   \\&\Leftrightarrow \dfrac{64}{729}+\dfrac{16}{3y}-\dfrac{1}{x}=\sqrt{y}\left(\dfrac{8}{y^2}+\dfrac{32}{27y}\right).
\end{align*}
Ta đươc $\sqrt{y}$ hữu tỉ, và ta suy ra $y$ chính phương. Kết hợp với $\dfrac{1}{\sqrt[3]{x}}+\dfrac{2}{\sqrt{y}}=\dfrac{4}{9},$ ta được $\sqrt[3]{x}$ hữu tỉ, và suy ra $x$ là số lập phương. Bằng lập luân trên, ta đặt $$x=a^3,y=b^2,\text{ ở đây }a\in\mathbb{Z},b\in\mathbb{N}^*$$ 
Phép đặt này cho ta
\begin{align*}
    \dfrac{1}{a}+\dfrac{2}{b}=\dfrac{4}{9}\Leftrightarrow \dfrac{1}{a}=\dfrac{4}{9}-\dfrac{2}{b}\Leftrightarrow \dfrac{1}{a}=\dfrac{4b-18}{9b}\Leftrightarrow a=\dfrac{9b}{4b-18}
\end{align*}
Với việc $a$ là số nguyên, ta được 
\begin{align*}
    (4b-18)\mid 9b
    &\Rightarrow (4b-18)\mid 36b
    \\&\Rightarrow (4b-18)\mid 9(4b-18)+162
    \\&\Rightarrow (4b-18)\mid 162
    \\&\Rightarrow (2b-9)\mid 81.
\end{align*}
Căn cứ vào điều này, ta lập ra bảng giá trị sau.
       \begin{center}
            \begin{tabular}{c|c|c|c|c|c|c|c}
            $2b-9$ & $-3$ & $-1$ & $1$ & $3$ & $9$ & $27$ & $81$ \\
            \hline
            $b$ & $3$ & $4$ & $5$ & $6$ & $9$ & $18$ & $45$\\
            \hline
            $a$ & $-4,5$ & $-18$ & $22,5$ & $9$ & $4,5$ & $3$ & $2,5$ \\ 
            \end{tabular}
        \end{center}
Đối chiếu với phép đặt $x=a^3,y=b^2,$ ta nhận thấy có $3$ cặp số nguyên $(x,y)$ thỏa đề, bao gồm $$(-5832,16),(729,36),(27,324).$$}
\end{gbtt}

\begin{gbtt}
Tìm tất cả các số nguyên tố $p,q,r,s,t$ thỏa mãn $$p+\sqrt{q^{2}+r}=\sqrt{s^{2}+t}.$$
\nguon{Kazakhstan Mathematical Olympiad 2012, Grade 9}
\loigiai{
Giả sử tồn tại các số nguyên tố $p,q,r,s,t$ thỏa mãn đẳng thức. Ta nhận thấy $\sqrt{s^{2}+t}-\sqrt{q^{2}+r}$ là số nguyên dương, thế nên theo bổ đề, hai số $s^2+t$ và $q^2+r$ chính phương. Ta đặt
$$s^2+t=x^2,\quad q^2+r=y^2.$$
Từ $s^2+t=x^2,$ ta có $t=(x-s)(x+s).$ Dựa vào nhận xét
$$0<x-s<x+s,$$
ta chỉ ra $x-s=1,$ còn $x+s=t,$ và vì thế $t=2s+1.$ \\
Chứng minh tương tự, ta cũng có thể chỉ ra $r=2q+1.$ Thể trở lại đẳng thức ban đầu, ta được
$$p+\sqrt{q^2+2q+1}=\sqrt{s^2+2s+1}\Rightarrow p+q=s.$$
Tính chẵn lẻ khác nhau của $p,q$ cho phép ta xét những trường hợp sau đây.
\begin{enumerate}
    \item Với $p=2,$ ta có
    $t=2s+1, r=2q+1, q+2=s.$ \\
    Nếu như $q\ge 5,$ từ $q+2=s$ và $t=2s+1,$ ta lần lượt suy ra
    $$q\equiv 5\pmod{6},\quad s\equiv 1\pmod{6}\Rightarrow t\equiv 3\pmod{6},$$
    vô lí do $t$ là số nguyên tố lớn hơn $5.$ \\
    Vì vậy, bắt buộc $q\le 4,$ tức $q=3$ hoặc $q=2.$ Kiểm tra trực tiếp, ta tìm ra $$(p,q,r,s,t)=(2,3,7,5,11).$$
    \item Với $q=2,$ ta có $t=2s+1,r=5,p+2=s.$ \\
    Bằng cách xét số dư của $p$ khi chia cho $6$ như trường hợp trước, ta thấy trường hợp $p\ge 5$ không thể xảy ra, thế nên $p=2$ hoặc $p=3.$ Kiểm tra trực tiếp, ta tìm ra $$(p,q,r,s,t)=(3,2,5,5,11).$$
\end{enumerate}
Kết luận, có hai bộ $(p,q,r,s,t)$ thỏa yêu cầu là
$$(2,3,7,5,11),\: (3,2,5,5,11).$$
}
\end{gbtt}

\section{Bài toán về cấu tạo số}

Phát triển từ ý tưởng các bài toán cấu tạo số ở cấp học dưới, các bài toán về số tự nhiên và các chữ số trong cuốn sách mở ra thêm nhiều hướng đi mới cho dạng bài tập này, có thể kể đến như chặn giá trị cho chữ số. Dưới đây là một vài bài tập tự luyện.

\subsubsection*{Bài tập tự luyện}
\begin{btt}
Tìm các số tự nhiên $\overline{abc}$ với các chữ số khác nhau sao cho
\[9a = 5b + 4c.\]
\end{btt}

\begin{btt}
Tìm tất cả các số tự nhiên có $4$ chữ số $\overline{abcd}$ thỏa mãn đồng thời các điều kiện $\overline{abcd}$ chia hết cho $3$ và $\overline{abc}-\overline{bda}=650.$
\nguon{Chuyên Toán Hải Dương 2021}
\end{btt}

\begin{btt}
Tìm các số tự nhiên có ba chữ số, biết rằng nếu cộng chữ số hàng trăm với $n,$ đồng thời trừ các chữ số hàng chục và đơn vị cho $n,$ ta được một số gấp $n$ lần số ban đầu.
\end{btt}

\begin{btt}
Tìm tất cả các số nguyên dương $x,y$ thỏa mãn đồng thời các tính chất
\begin{enumerate}[i,]
    \item $x$ và $y$ đều có hai chữ số.
    \item $x=2 y.$
    \item Một chữ số của $y$ thì bằng tổng hai chữ số của $x$, còn chữ số kia thì bằng trị tuyệt đối của hiệu hai chữ số của $x$.
\end{enumerate}
\end{btt}

\begin{btt}
Tìm các số tự nhiên có bốn chữ số và bằng tổng các bình phương của số tạo bởi hai chữ số đầu và số tạo bởi hai chữ số cuối, biết rằng hai chữ số cuối giống nhau.
\end{btt}

\begin{btt}
Tìm tất cả các số có $5$ chữ số $\overline{abcde}$ sao cho $\sqrt[3]{\overline{abcde}}=\overline{ab}$.
\nguon{Chuyên Toán Thái Nguyên 2016} 
\end{btt}

\begin{btt}
Tìm hai số chính phương có bốn chữ số, biết rằng mỗi chữ số của số thứ nhất đều lớn hơn chữ số cùng hàng của số thứ hai cùng bằng một số.
\end{btt}

\begin{btt}
Tìm số tự nhiên $\overline{abc}$ thỏa mãn điều kiện $\overline{abc}=\tron{a+b}^24c.$
\end{btt}
\begin{btt}
Hãy tìm tất cả các chữ số nguyên dương $a,b,c$ đôi một khác nhau thỏa mãn 
\[\dfrac{\overline{ab}}{\overline{ca}}=\dfrac{b}{c}.\]
\nguon{ Chuyên Quốc Học Huế năm 2011}
\end{btt}
\begin{btt}
Số nguyên dương $n$ được gọi là số bạch kim nếu $n$ bằng tổng bình phương các chữ số của nó.
\begin{enumerate}[a,]
    \item Chứng minh rằng không tồn tại số bạch kim có $3$ chữ số.
    \item Tìm tất cả các số nguyên dương $n$ là số bạch kim.
\end{enumerate}
\nguon{Phổ thông Năng khiếu 2008}
\end{btt}

\begin{btt}
Tìm hai số tự nhiên liên tiếp, mỗi số có hai chữ số, biết rằng nếu viết số lớn trước số nhỏ thì ta được một số chính phương.
\end{btt}

\begin{btt}
Tìm các số tự nhiên có bốn chữ số và bằng bình phương của tổng của số tạo bởi hai chữ số đầu và số tạo bởi hai chữ số cuối của số đó (viết theo thứ tự cũ).
\end{btt}

\begin{btt}
Tìm các số tự nhiên có bốn chữ số thỏa mãn đồng thời các điều kiện. 
\begin{enumerate}
\item[i,] Hai chữ số đầu như nhau, hai chữ số cuối như nhau.
\item[ii,] Số cần tìm  bằng tích của hai số, mỗi số gồm hai chữ số như nhau.
\end{enumerate}
\end{btt}

\subsection*{Hướng dẫn bài tập tự luyện}

\begin{gbtt}
Tìm các số tự nhiên $\overline{abc}$ với các chữ số khác nhau sao cho
\[9a = 5b + 4c.\]
\loigiai
{Giả sử tồn tại số tự nhiên $\overline{abc}$ thỏa mãn đề bài. Ta có
$$9a = 5b + 4c \Leftrightarrow 9a - 9c = 5b - 5c \Leftrightarrow 9(a - c) = 5(b - c).$$    
Do $(5,9)=1$ nên $b-c$ chia hết cho $9.$ Với việc $-9\le b-c\le 9$ và giả thiết $b,c$ là hai chữ số phân biệt, ta xét các trường hợp sau đây.
\begin{enumerate}
   \item Với $b-c=9,$ ta có $b=9$ và $c=0.$
   Kiểm tra, ta tìm ra $\overline{abc}=590.$
   \item Với $b-c=-9,$ ta có $b=0$ và $c=9.$
   Kiểm tra, ta tìm ra $\overline{abc}=409.$
\end{enumerate}
Vậy tất các số cần tìm là $590$ và $409$.}
\end{gbtt}

\begin{gbtt}
Tìm tất cả các số tự nhiên có $4$ chữ số $\overline{abcd}$ thỏa mãn đồng thời hai điều kiện là $\overline{abcd}$ chia hết cho $3$ và $\overline{abc}-\overline{bda}=650.$
\nguon{Chuyên Toán Hải Dương 2021}
\loigiai{
Giả sử tồn tại số tự nhiên $\overline{abcd}$ thỏa mãn đề bài. Rõ ràng, $c=a$ và từ giả thiết, ta nhận thấy $a\ge 7.$\\
Ta xét các trường hợp sau.
\begin{enumerate}
    \item Với $a=c=7,$ ta có
        $$650=\overline{abc}-\overline{bda}=\overline{7b7}-\overline{bd7}=700-90b-10d\le 700-90\cdot1=630,$$
        một điều mâu thuẫn.
    \item Với $a=c=8,$ ta có
        $$650=\overline{abc}-\overline{bda}=\overline{8b8}-\overline{bd8}=800-90b-10d.$$
        Từ đây, ta suy ra $9b+d=15,$ tức $b=1$ và $c=6.$\\
        Kiểm tra trực tiếp, ta thấy số $\overline{abcd}=8186$ không chia hết cho $3,$ mâu thuẫn.
    \item Với $a=c=9,$ ta có
        $$650=\overline{abc}-\overline{bda}=\overline{9b9}-\overline{bd9}=900-90b-10d.$$
        Từ đây, ta suy ra $9b+d=25.$ Do $b,d$ là các chữ số, chỉ có trường hợp $b=2,d=7$ xảy ra.\\ Kiểm tra trực tiếp, ta thấy số $\overline{abcd}=9297$ chia hết cho $3.$
\end{enumerate}
Kết luận, $\overline{abcd}=9297$ là số tự nhiên duy nhất thỏa yêu cầu.}
\end{gbtt}

\begin{gbtt}
Tìm các số tự nhiên có ba chữ số, biết rằng nếu cộng chữ số hàng trăm với $n,$ đồng thời trừ các chữ số hàng chục và đơn vị cho $n,$ ta được một số gấp $n$ lần số ban đầu.
\loigiai{
Gọi số phải tìm là $x.$ Khi thêm $n$ vào hàng trăm, bớt $n$ ở hàng chục và hàng đơn vị, số đó sẽ tăng thêm
$$100n-10n-n=89n.$$
Số mới gấp $n$ lần số cũ, chứng tỏ $nx-x=89n,$ hay là $(n-1)x=89n.$ \\
Do $(n,n-1)=1$ nên $89$ chia hết cho $n-1.$ Với chú ý $0\le n\le 9,$ ta tìm ra $n=2.$ Số cần tìm là $178.$}
\end{gbtt}

\begin{gbtt}
Tìm tất cả các số nguyên dương $x,y$ thỏa mãn đồng thời các tính chất
\begin{enumerate}[i,]
    \item $x$ và $y$ đều có hai chữ số.
    \item $x=2 y.$
    \item Một chữ số của $y$ thì bằng tổng hai chữ số của $x$, còn chữ số kia thì bằng trị tuyệt đối của hiệu hai chữ số của $x$.
\end{enumerate}
\loigiai{Giả sử tồn tại các số nguyên dương $x=\overline{a b}, y=\overline{c d}$ thỏa yêu cầu. Ta sẽ có $x=2 y,d=a+b,c=|a-b|$. Ta lần lượt xét các trường hợp sau đây.
\begin{enumerate}
    \item Nếu $c=a-b, d=a+b,$ ta có $10 a+b=2(10 c+d)=2(10 a-10 b+a+b),$ hay là $19 b=12 a.$ Điều này mâu thuẫn với điều kiện $0\le b,a\le 10.$
    \item Nếu $c=b-a, d=a+b,$ ta có 
    $$10 a+b=2(10 c+d)=2(10 b-10 a+a+b) \Leftrightarrow 28 a=21 b \Leftrightarrow 4 a=3 b.$$
    Trong trường hợp này, ta tìm được $\overline{ab}\in\{34;68\}.$ Thử trực tiếp, ta chỉ ra được $\overline{cd}=17$ khi $\overline{ab}=34.$ 
\end{enumerate}
Kết luận, cặp số duy nhất thỏa yêu cầu bài toán là $(34,17).$}
\end{gbtt}

\begin{gbtt}
 Tìm các số tự nhiên có bốn chữ số và bằng tổng các bình phương của số tạo bởi hai chữ số đầu và số tạo bởi hai chữ số cuối, biết rằng hai chữ số cuối giống nhau.
 \loigiai{
Gọi số cần tìm là $\overline{abcc}.$ Từ giả thiết, ta có
$$\overline{abcc} = \overline{ab}^2 + \overline{cc}^2.$$
Ta tiếp tục đặt $\overline{ab} = x$, $\overline{cc} = y.$ Dựa vào phép đặt này, ta lại có 
$$100x + y = x^2 + y^2\Rightarrow 4x^2-400x+4y^2-4y=0\Rightarrow 4(x-50)^2+(2y-1)^2=10001.$$
Phân tích trên cho ta $(2y-1)^2\le 10001,$ hay là $y\le 50.$ Song, với việc $y$ chia hết cho $11,$ có bốn trường hợp xảy ra, là $y=11,y=22,y=33,y=44.$ Thử trực tiếp từng trường hợp, ta chỉ ra tất cả các số tự nhiên cần tìm là $1233$ và $8833.$}
\end{gbtt}

\begin{gbtt}
Tìm tất cả các số có $5$ chữ số $\overline{abcde}$ sao cho $\sqrt[3]{\overline{abcde}}=\overline{ab}$.
\nguon{Chuyên Toán Thái Nguyên 2016} 
\loigiai{Từ giả thiết, ta có $1000\overline{ab} + \overline{cde} = (\overline{ab})^3.$  Đặt $m=\overline{ab},n=\overline{cde}.$ Phép đặt này giúp ta lần lượt suy ra $$1000m+n=m^3 \Rightarrow m^3 \geq 1000m \Rightarrow m^2 \geq 1000 \Rightarrow m \geq 32.$$
Nếu như $m\geq 33,$ ta nhận xét $$n=m(m^2-1000)\geq 33\cdot 89=2937>1000.$$
Điều này mâu thuẫn với điều kiện $n$ có không quá ba chữ số. Theo đó, chỉ khả năng $m=32$ xảy ra. Ta tìm được $\overline{abcde}=32^3=32768.$}
\end{gbtt}

\begin{gbtt}
Tìm hai số chính phương có bốn chữ số, biết rằng mỗi chữ số của số thứ nhất đều lớn hơn chữ số cùng hàng của số thứ hai cùng bằng một số.
\loigiai{
Gọi hai số chính phương lần lượt là $x^2 = \overline{abcd}$ và $y^2 = \overline{a'b'c'd'}.$
Từ giả thiết, ta có $$a - a' = b - b' = c - c' = d - d' = m,$$ với $m \leq 8$ và $32 \leq y < x \leq 99,$ đồng thời
  $$x^2 - y^2 = 1111m \Rightarrow (x + y)(x - y) = 11\cdot101m.$$
Do $11,101$ đều là số nguyên tố và
$$ x - y \leq 99 - 32 = 67, \quad x + y \leq 99 + 98 = 197$$
  nên $\heva{& x + y = 101 \\& x - y = 11m} \Rightarrow \heva{& x = \dfrac{101 + 11m}{2} \\& y = \dfrac{101 - 11m}{2}}\Rightarrow 101 - 11m \geq 64 \Rightarrow m \leq \dfrac{34}{11}.$ \\
Từ $2y=101-11m,$ ta cũng nhận thấy $m$ là số lẻ. Ta xét các trường hợp sau.
\begin{enumerate}
    \item Với $m = 1,$ ta có  $\heva{& x = 56 \\& y = 45} \Rightarrow \heva{& x^2 = 3136 \\& y^2 = 2025.}$
    \item Với $m = 3,$ ta có  $\heva{& x = 67 \\& y = 34} \Rightarrow \heva{& x^2 = 4489 \\& y^2 = 1156.}$
\end{enumerate}
Như vậy có hai cặp số thỏa mãn đề bài, gồm $(3136,2025)$ và $(4489,1156).$}
\end{gbtt}

\begin{gbtt}
Tìm số tự nhiên $\overline{abc}$ thỏa mãn điều kiện $\overline{abc}=\tron{a+b}^24c.$
\loigiai{
Từ giả thiết bài toán ta có $100a+10b+c=4c{{\left( a+b \right)}^{2}}$. Do đó ta được 
\[c\vuong{4(a+b)^2-1}=10\vuong{(a+b)+9a}.\tag{*}\label{cts01}\]
Nếu $c$ chia hết cho $5,$ ta có $c=0$ hoặc $c=5,$ nhưng cả hai trường hợp này đều mâu thuẫn với giả thiết.\\ Do đó $c$ không chia hết cho $5,$ và ta suy ra
$$5\mid \vuong{4{{\left( a+b \right)}^{2}}-1}.$$
Vì $4(a+b)^2$ là số chẵn nên nó phải có tận cùng là $6,$ kéo theo $(a+b)^2$ phải có tận cùng là $4$ hoặc $9.$\\
Ngoài ra, ta cũng có $c$ chẵn. Do $c$ chẵn và $c\ne 0$ nên $c\ge 2.$ Kết hợp với (\ref{cts01}) ta có
$$4(a+b)^2-1\le  \dfrac{10(a+b)+90a}{2}\le \dfrac{10\cdot9\cdot 2+90\cdot9}{2}=495.$$
Từ các kết quả trên ta nhận được  ${\left( a+b \right)}^{2}\in \left\{ 4;9;49;64 \right\}$ hay \[a+b \in \left\{2;3;7;8\right\}.\tag{**}\label{cts02}\]
Tới đây, ta xét các trường hợp sau.
\begin{enumerate}
    \item Nếu $a+b\in \left\{ 2;7;8 \right\}$ thì $a+b$ có dạng $3k\pm 1,$ khi đó $4{{\left( a+b \right)}^{2}}-1$ chia hết cho $3.$ Lại có \[\left( a+b \right)+9a=3k\pm 1+9a\] không chia hết cho $3$ nên $10\left[ \left( a+b \right)+9a \right]$ không chia hết cho $3,$ suy ra $c\notin \mathbb{N},$ mâu thuẫn.
    \item Nếu $a+b=3$ ta có $c=\dfrac{10\left( 3+9a \right)}{35}=\dfrac{6\left( 1+3a \right)}{7}$. \\
    Ta suy ra $1+3a$ chia hết cho $7.$ Thử với $a=1,2,3,\ldots,9,$ ta tìm ra $a=2$ và $a=9.$
    \begin{itemize}
        \item \chu{Trường hợp 1.} Nếu $a=2,$ ta có $b\in \{1;5;6\}.$ Thế trở lại (\ref{cts01}), ta tìm ra $c=6$ khi $b=1.$
        \item \chu{Trường hợp 2.} Nếu $a=9,$ đối chiếu với (\ref{cts02}), ta có $b<0,$ mâu thuẫn.      
    \end{itemize}
\end{enumerate}
Vậy số $\overline{abc}=216$ là số tự nhiên cần tìm.}
\end{gbtt}
\begin{gbtt}
Hãy tìm tất cả các chữ số nguyên dương $a,b,c$ đôi một khác nhau thỏa mãn 
\[\dfrac{\overline{ab}}{\overline{ca}}=\dfrac{b}{c}.\]
\nguon{ Chuyên Quốc Học Huế năm 2011}
\loigiai{Giả sử tồn tại các bộ số $\left(a,b,c \right)$ thỏa mãn đề bài. 
Khi đó đẳng thức đã cho trở thành \[\left( 10a+b \right)c=\left( 10c+a \right)b\tag{*}\label{quochoc11}\]
hay $2\cdot 5\cdot c\left( a-b \right)=b\left( a-c \right)$. 
Ta suy ra $5$ là ước số của $bac$. Vì $1\le a,b,c\le 9$ và $a\ne c$ nên
$$b=5 \quad\text{hoặc}\quad a-c=5 \quad \text{ hoặc }\quad c-a=5.$$
Ta đi xét các trường hợp sau.
\begin{enumerate}
\item Với $b=5$ ta có $2c\left( a-5 \right)=a-c.$ Biến đổi tương đương cho ta
\[ c=\dfrac{a}{2a-9}\Leftrightarrow 2c=1+\dfrac{9}{2a-9}.\]
Vì $a\ne 5$ nên $2a-9=3$ hoặc $2a-9=9.$ Trong trường hợp này, ta tìm ra
$$\overline{abc}=652,\quad \overline{abc}=951.$$
\item Với $a-c=5$ ta có $a=c+5.$ Thế vào (\ref{quochoc11}) ta được
$$2c\left( c+5-b \right)=b\Leftrightarrow b=\dfrac{2{{c}^{2}}+10c}{2c+1}\Leftrightarrow 2b=2c+9-\dfrac{9}{2c+1}.$$
Ta được $2c+1=3$ hoặc $2c+1=9$ vì $c\ne 0$. 
Trong trường hợp này, ta tìm ra
$$\overline{abc}=641,\quad \overline{abc}=984.$$
\item Với $c-a=5$ ta có $c=a+5.$ Thế vào (\ref{quochoc11}) ta được
$$2\left( a+5 \right)\left( a-b \right)=-b\Leftrightarrow b=\dfrac{2{{a}^{2}}+10a}{2a-9}.$$
Từ đó suy ra $2b=2a+19+\dfrac{9.19}{2a-9}>9,$ mâu thuẫn.
\end{enumerate}
Vậy các bộ số  $\left( a,b,c \right)$ thỏa mãn yêu cầu bài toán là 
$\left( 6,5,2 \right),\ \left( 9,5,1 \right),\ \left( 6,4,1 \right),\ \left( 9,8,4 \right).$}
\end{gbtt}
\begin{gbtt}
Một số nguyên dương $n$ được gọi là số bạch kim nếu $n$ bằng tổng bình phương các chữ số của nó.
\begin{enumerate}[a,]
    \item Chứng minh rằng không tồn tại số bạch kim có $3$ chữ số.
    \item Tìm tất cả các số nguyên dương $n$ là số bạch kim.
\end{enumerate}
\nguon{Phổ thông Năng khiếu 2008}
\loigiai{
\begin{enumerate}[a,]
    \item Gọi số bạch kim có $3$ chữ số là $\overline{abc}.$ Từ giả thiết, ta có
    $$\overline{abc}= a^2+b^2+c^2.$$
    Từ đây, ta suy ra điều sau
    $$100a+10b+c=a^2+b^2+c^2\Rightarrow a\tron{100-a}+b\tron{10-b}+c-c^2=0.$$
    Vì $a,b,c\le9$ nên $91\le(100-a), 0\le(10-b), c^2\le 81.$ Ta nhận thấy $0<(100-a)-c^2.$ \\
    Từ những lập luận vừa rồi, ta thu được
    $$\vuong{a\tron{100-a}-c^2}+b\tron{10-b}+c>0.$$
    Điều này mâu thuẫn với lập luận trên của ta. Vậy nên không tồn tại số bạch kim có $3$ chữ số.
    \item Ta sẽ chứng minh không tồn tại số bạch kim có số chữ số lớn hơn $2.$ Thật vậy, ta gọi số bạch kim có $n$ chữ số là $\overline{a_1a_2\cdots a_n}$ trong đó $3\le n, 1\le a_1\le 9$ và $0\le a_2,a_3,\cdots, a_n\le9.$ Điều kiện ở giả thiết cho ta
    \[a_1\tron{10^{n-1}-a}+a_2\tron{10^{n-2}-a_2}+\cdots+a_n-a^2_n=0\tag{*}\label{platinumhcm}.\]
    Mặt khác, từ những điều kiện $1\le a_1\le 9$ và $0\le a_2,a_3,\cdots, a_n\le9,$ ta suy ra  $81<a_1\tron{10^{n-1}-a}$ và $0<10^{n-2}-a_2,\cdots,10-a_{n-1}.$ Các nhận xét kể trên giúp ta đánh giá được vế trái của (\ref{platinumhcm})
    $$\vuong{a_1\tron{10^{n-1}-a}-a^2_n}+a_2\tron{10^{n-2}-a_2}+\cdots+a_n>0.$$
    Đánh giá trên là một mâu thuẫn, chứng tỏ một số bạch kim chỉ có thể có $1$ hoặc $2$ chữ số. Ta xét các trường hợp sau.
    \begin{itemize}
        \item \chu{Trường hợp 1.} Nếu $n=1,$ ta gọi số bạch kim có một chữ số là $a.$ Từ giả thiết, ta có
        $a=a^2.$
        Vì $a$ là số nguyên dương nên $a=1.$ 
        \item \chu{Trường hợp 2.} Nếu $n\ge 2,$  ta gọi số bạch kim có hai chữ số là $\overline{ab}.$ Từ giả thiết, ta có
        $\overline{ab}=a^2+b^2,$
        hay là $10a+b=a^2+b^2.$ Từ đây, ta suy ra
        $$b\tron{b-1}=a\tron{10-a}.$$
        Xét tính chẵn lẻ của hai vế, ta chỉ ra $a$ chẵn. Thử trực tiếp với $a=2,4,6,8,$ ta không tìm được $b$ thỏa mãn.
    \end{itemize}
Như vậy, có duy nhất số bạch kim là $1.$
\end{enumerate}}
\end{gbtt}

\begin{gbtt}
Tìm hai số tự nhiên liên tiếp, mỗi số có hai chữ số, biết rằng nếu viết số lớn trước số nhỏ thì ta được một số chính phương.
\loigiai{Gọi hai số tự nhiên phải tìm là $x$ và $x + 1$, số chính phương là $n^2$, trong đó $x$, $n$ thuộc $\mathbb{N}$. Rõ ràng $10 \leq x \leq 98; \, 32 \leq n \leq 99.$ Từ giả thiết, ta có
$$100(x + 1) + x = n^2 \Leftrightarrow 101x + 100 = n^2 \Leftrightarrow (n + 10)(n - 10) = 101x.$$
Với chú ý $(n + 10)(n - 10)$ chia hết cho số nguyên tố $101$, ta thấy tồn tại một thừa số chia hết cho $101$. Việc chặn $32\le n\le 99$ ở ban đầu cho ta
$$ 22 \leq n - 10 \leq 89, \quad 42 \leq n + 10 \leq 109.$$
Quan sát, ta nhận ra chỉ tồn tại trường hợp $n+10=101.$ Ta lần lượt tìm được $n = 91$ và $n^2 = 91^2 = 8281$. Như vậy hai số cần tìm là $81$ và $82.$}
\end{gbtt}

\begin{gbtt}
Tìm các số tự nhiên có bốn chữ số và bằng bình phương của tổng của số tạo bởi hai chữ số đầu và số tạo bởi hai chữ số cuối của số đó (viết theo thứ tự cũ).
\loigiai{
Gọi số cần tìm là $\overline{abcd}.$ Từ giả thiết, ta có 
$$\overline{abcd} = \left( \overline{ab} + \overline{cd} \right)^2.$$
Ta đặt $\overline{ab} = x$, $\overline{cd} = y$ trong đó $10\le x,y\le 99.$ Phép đặt này cho ta
$$100x + y = (x + y)^2 \Leftrightarrow 99x = (x + y)^2 - (x + y) \Leftrightarrow 99x = (x + y)(x + y - 1).$$
Tới đây, ta xét các trường hợp sau.
\begin{enumerate}
    \item Nếu một trong hai thừa số $x + y$, $x + y - 1$ chia hết cho $99,$ do $32 \leq x + y \leq 99$ nên $$31 \leq x + y - 1 \leq 98.$$ 
    Chỉ tồn tại khả năng $x+y=99.$ Ta tìm ra $\overline{abcd}=99^2=9801.$
    \item Nếu trong hai thừa số $x + y$, $x + y - 1,$ không có thừa số nào chia hết cho $99,$ một thừa số phải chia hết cho $11,$ còn thừa số còn lại chia hết cho $9.$
    \begin{itemize}
        \item \chu{Trường hợp 1.} Nếu $11\mid \tron{x+y}$ và $9\mid\tron{x+y-1},$ ta có
        $$\heva{& (x + y) \in \{33; 44; 55; 66; 77; 88 \} \\& (x + y - 1) \in \{36; 45; 54; 63; 72; 81; 90 \}}.$$
        Đối chiếu các dòng trong hệ, ta tìm ra $x+y=55,$ và như vậy $\overline{abcd}=55^2=3025.$
        \item \chu{Trường hợp 2.} Nếu $9\mid \tron{x+y}$ và $11\mid\tron{x+y-1},$ ta có
        $$\heva{& (x + y) \in \{36; 45; 54; 63; 72; 81; 90 \} \\& (x + y - 1) \in \{33; 44; 55; 66; 77; 88 \}}.$$
        Đối chiếu các dòng trong hệ, ta tìm ra $x+y=45,$ và như vậy $\overline{abcd}=45^2=2025.$
   \end{itemize}
\end{enumerate}
Thử trực tiếp từng trường hợp, ta kết luận tất cả các số cần tìm là $2025,3025$ và $9801.$}
\end{gbtt}

\begin{gbtt}
Tìm các số tự nhiên có bốn chữ số thỏa mãn đồng thời các điều kiện. 
\begin{enumerate}
\item[i,] Hai chữ số đầu như nhau, hai chữ số cuối như nhau.
\item[ii,] Số cần tìm  bằng tích của hai số, mỗi số gồm hai chữ số như nhau.
\end{enumerate}
\loigiai{
Gọi số cần tìm là $\overline{xxyy}.$ Từ giả thiết, ta có
   \begin{align*}
    \overline{xxyy} = \overline{aa} \cdot \overline{bb} & \Leftrightarrow 1100x + 11y = 11a \cdot 11b \text{ (với $a$, $b$ thuộc $\mathbb{N}^*$)}\\
    & \Leftrightarrow 100x + y = 11ab \\
    & \Leftrightarrow 99x + x + y = 11ab.
   \end{align*}
Xét tính chia hết cho $11$ ở cả hai vế, ta nhận thấy rằng $x+y$ chia hết cho $11.$ Với việc $2\le x+y\le 18,$ chỉ khả năng $x+y=11$ xảy ra. Ta lập bảng giá trị sau đây.
    \begin{center}
    \begin{tabular}{c|c|c|c|c|c|c|c|c}
    $\overline{xy}$ & $29$ & $38$ & $47$ & $56$ & $65$ & $74$ & $83$ & $92$\\
    \hline
    $ab$ & $19$  & $28$  & $37$  & $46$  & $55$  & $64$  & $73$  & $82$\\
    \end{tabular}
    \end{center}
Trong các tích $ab$ ở bảng trên, chỉ có hai trường hợp cho $a,b$ là hai số có một chữ số, đó là
   $$28 = 4 \cdot 7, \hspace*{0.5cm} 64 = 8 \cdot 8.$$
Kiểm tra trực tiếp, ta nhận thấy các số cần tìm là $3388 = 44 \cdot 77$ và $7744 = 88 \cdot 88.$}
\end{gbtt} %pt chứa căn thức + cấu tạo số
\chapter{Bất đẳng thức trong số học}

Bất đẳng thức là một dạng toán khó trong phân môn Đại số. Ta đã biết đến các bất đẳng thức đại số nổi tiếng như bất đẳng thức $AM-GM$ hay bất đẳng thức $Cauchy-Schwarz.$ Ở chương này, chúng ta sẽ cùng tìm hiểu về một số dạng toán và phương pháp để chứng minh các bất đẳng thức trong Số học.
\\ \\
Chương VI được chia thành 3 phần.
\begin{itemize}
    \item\chu{Phần 1.} Bất đẳng thức đối xứng nhiều biến.
    \item\chu{Phần 2.} Bất đẳng thức và phương pháp xét modulo.
    \item\chu{Phần 3.} Bất đẳng thức liên quan đến ước và bội.
\end{itemize}

\section{Bất đẳng thức đối xứng nhiều biến}
\subsection*{Ví dụ minh hoạ}

\begin{bx}
Cho các số nguyên dương $x,y$ thỏa mãn $x+y=99.$ Tìm giá trị lớn nhất và giá trị nhỏ nhất của tích $xy.$
\end{bx}
\nx{\\Nếu bài toán trên cho $x,y$ là các số thực không âm, bằng kiến thức đã học về bất đẳng thức, ta chỉ ra được
$$xy\ge 0,\quad xy\le\dfrac{(x+y)^2}{4}=\dfrac{99^2}{4}.$$
Với $x,y$ là các số nguyên dương, dấu bằng xảy ra ở cả hai chiều lớn và nhỏ nhất sẽ tiệm cận nhất với dấu bằng trong trường hợp số thực. Thông qua thử một vài trường hợp, ta dự đoán
\begin{itemize}
    \item[i,] $\min xy = 98,$ đạt tại $(x,y)=(1,98)$ hoặc $(x,y)=(98,1).$
    \item[ii,] $\max xy = 2450,$ đạt tại $(x,y)=(44,45)$ hoặc $(x,y)=(45,44).$    
\end{itemize}
Ta sẽ chứng minh các nhận định trên.}
\loigiai{
Nhờ giả thiết $x\ge 1,y\ge 1,$ ta có $(x-1)(y-1)\ge 0.$ Từ đó ta suy ra
$$xy\ge x+y-1=99-1=98.$$
Tiếp theo, ta sẽ đi chứng minh rằng $xy\le 2352,$ tức đi chứng minh $x(99-x)\le 2352.$ Bằng phân tích nhân tử, ta nhận thấy bất đẳng thức này tương đương với
$$(x-49)(x-50)\ge 0.$$
Bất đẳng thức trên đúng $(x-49)(x-50)$ là tích hai số tự nhiên liên tiếp. Ta kết luận
\begin{itemize}
    \item[i,] Giá trị nhỏ nhất của $xy$ là $98,$ đạt tại $(x,y)=(1,98)$ hoặc $(x,y)=(98,1).$
    \item[ii,] Giá trị lớn nhất của $xy$ là $2450,$ đạt tại $(x,y)=(44,45)$ hoặc $(x,y)=(45,44).$    
\end{itemize}}

\begin{bx}
Cho $45$ số nguyên dương $a_1,a_2,\cdots,a_{45}$  thỏa mãn điều kiện 
$$a^3_1+a^3_2+\cdots+a^3_{45}\le 10^6.$$
Chứng minh ta luôn tìm được ít nhất $2$ số bằng nhau trong $45$ số trên.
\loigiai{
Ta giả sử phản chứng rằng, $45$ số nguyên dương trên phân biệt. Không mất tính tổng quát, ta giả sử thêm
$$a_1\le a_2\le \cdots \le a_{45}.$$
Tính phân biệt của $45$ số nguyên dương kể trên cho ta
$$a_1\ge 1,\,a_2\ge 2,\,a_3\ge 3,\ldots,a_{44}\ge 44, \,a_{45}\ge 45.$$
Dựa vào lập luận trên, ta chỉ ra
\begin{align*}
    a^3_1+a^3_2+\cdots+a^3_{45}
    &\ge 1^3+2^3+\cdots+45^3
    \\&\ge \dfrac{45(45+1)(2\cdot 45+1)}{6}
    \\&\ge 1071225.
\end{align*}
Điều này mâu thuẫn với giả thiết. Giả sử tất cả các số phân biệt là sai, và ta có điều phải chứng minh.}
\end{bx}

\begin{bx}
Cho các số nguyên dương $a,b,c,d$ thỏa mãn $a>b>c>d$ và $ab=cd.$
\begin{enumerate}[a,]
    \item Chứng minh rằng tồn tại các số nguyên dương $x,y,z,t$ sao cho $$a=xt,b=yz,c=xz,d=yt.$$
    \item Chứng minh rằng giữa $a$ và $d$ có ít nhất một số chính phương.
\end{enumerate}
\loigiai{
\begin{enumerate}[a,]
    \item Ta đặt $x=(a,c).$ Lúc này, tồn tại các số nguyên dương $t,z$ thỏa mãn $$(t,z)=1,a=xt,c=xz.$$ Kết hợp với $ab=cd,$ phép đặt này cho ta
    $xt.b=xz.d,$
    hay là 
    $bx=dz.$ \\
    Ta nhận thấy $t\mid dz,$ nhưng do $(t,z)=1$ nên $t\mid d.$ Tiếp tục đặt $d=yt,$ ta được $b=yz.$ Bằng các cách đặt như vậy, ta chỉ ra được sự tồn tại của các số nguyên dương $x,y,z,t$ sao cho $$a=xt,b=yz,c=xz,d=yt.$$
    Ta có điều phải chứng minh.
    \item
    Để giải quyết bài toán, ta sẽ chứng minh rằng
    $\sqrt{d}-\sqrt{a}>1.$\\
    Với cách đặt như câu a, bất đẳng thức trên trở thành
    \[\sqrt{xt}-\sqrt{yz}>1.\tag{*}\]
    Nếu như $y=z,$ hiển nhiên giữa $a$ và $d$ có một số chính phương (là $b=yz=y^2$).\\
    Đối với trường hợp $y\ne z,$ ta sẽ tìm cách so sánh $x,y,z,t.$ Thật vậy
    $$
    \heva{&a>d \\ &a>c}
    \Rightarrow \heva{&xt>yt \\ &xt>xz}
    \Rightarrow \heva{&x>y \\ &t>z}
    \Rightarrow \heva{&x\ge y+1 \\ &t\ge z+1.}
    $$
    Để chứng minh (*), ta sẽ phải chứng minh
    \[\sqrt{(y+1)(z+1)}-\sqrt{yz}>1.\tag{**}\]
    Bất đẳng thức (**) tương đương với
    \begin{align*}
        \sqrt{(y+1)(z+1)}>\sqrt{yz}+1
        &\Leftrightarrow (y+1)(z+1)>yz+1+2\sqrt{yz}
        \\&\Leftrightarrow yz+y+z+1>yz+1+2\sqrt{yz}
        \\&\Leftrightarrow y+z>2\sqrt{yz}
        \\&\Leftrightarrow \left(\sqrt{y}-\sqrt{z}\right)^2>0.
    \end{align*}
    Do $y\ne z,$ ta nhận được $\left(\sqrt{y}-\sqrt{z}\right)^2>0$ đúng. Bài toán được chứng minh.
\end{enumerate}
}
\begin{luuy}
Việc xét hiệu $\sqrt{a}-\sqrt{d}$ để chứng minh giữa $a$ và $d$ không chứa số chính phương nào là chìa khóa của bài toán. Đối với bài toán tương tự, bạn đọc có thể tham khảo phần bài tập tự luyện.
\end{luuy}
\end{bx}
\subsection*{Bài tập tự luyện}
\begin{btt}
Cho các số nguyên dương $x,y,z$ thỏa mãn $x+y+z=169.$ Tìm giá trị lớn nhất và giá trị nhỏ nhất của tích $xyz.$
\end{btt}

\begin{btt}
Cho các số nguyên dương $x,y,z$ thỏa mãn $x+y+z=53.$ Tìm giá trị lớn nhất và giá trị nhỏ nhất của biểu thức $P=xy+yz+zx.$
\end{btt}

\begin{btt}
Cho các số tự nhiên $x,y,z$ thỏa mãn $x+y+z=40.$ Tìm giá trị lớn nhất của biểu thức $P=xy^2+yz^2+zx^2.$
\end{btt}

\begin{btt}
Cho $55$ số nguyên dương $a_1,a_2,\ldots,a_{55}$  thỏa mãn điều kiện 
$$a_1\tron{a_1-1}\tron{a_1+1}+a_2\tron{a_2-1}\tron{a_2+1}+\cdots+a_{55}\tron{a_{55}-1}\tron{a_{55}+1}\le 2\cdot 10^6.$$
Chứng minh ta luôn tìm được ít nhất $2$ số bằng nhau trong $55$ số trên.
\end{btt}

\begin{btt}
Cho $100$ số nguyên dương $a_1,a_2,\ldots,a_{100}$ thỏa mãn điều kiện
$$\dfrac{1}{a_1}+\dfrac{1}{a_2}+\cdots+\dfrac{1}{a_{100}}\ge 15.$$
Chứng minh rằng có ít nhất ba số bằng nhau trong $100$ số trên.
\end{btt}

\begin{btt}
Tìm các bộ số tự nhiên $ {{a}_{1}},{{a}_{2}},\ldots,{{a}_{2014}}$ thỏa mãn điều kiện sau.
$${{a}_{1}}+{{a}_{2}}+{{a}_{3}}+\cdots+{{a}_{2014}}\ge {{2014}^{2}},\quad
  a_{1}^{2}+a_{2}^{2}+a_{3}^{2}+\cdots+a_{2014}^{2}\le {{2014}^{3}}+1.$$
\nguon{Đề thi vào lớp 10 thành phố Hà Nội 2014 $-$ 2015}
\end{btt}

\begin{btt}
Tìm tất cả các số nguyên dương $n$ sao cho tồn tại dãy số $x_1,x_2,\ldots,x_n$ thỏa mãn 
\[{{x}_{1}}+{{x}_{2}}+\cdots +{{x}_{n}}=5n-4\quad \text{ và }\quad \dfrac{1}{{{x}_{1}}}+\dfrac{1}{{{x}_{2}}}+\cdots +\dfrac{1}{{{x}_{n}}}=1.\]
\end{btt}

\begin{btt}
Cho $k$ và $n_{l}<n_{2}<\cdots<n_{k}$ là các số nguyên dương lẻ. Chứng minh rằng
\[n_{1}^{2}-n_{2}^{2}+n_{3}^{2}-n_{4}^{2}+\cdots+n_{k}^{2} \geq 2 k^{2}-1.\]
\end{btt}

\begin{btt}
Cho $n$ là số tự nhiên $n \geq 2$ và $n$ số nguyên $x_{1}, x_{2}, \ldots, x_{n}$ thỏa mãn điều kiện $$x_{1}^{2}+x_{2}^{2}+\ldots+x_{n}^{2}+n^{3} \leq(2 n-1)\left(x_{1}+x_{2}+\cdots+x_{n}\right)+n^{2} .$$
\begin{enumerate}[a,]
    \item Chứng minh rằng các số $x_{i}$ $(i=1,2,\ldots, {n})$ đều là các số nguyên dương.
    \item  Chứng minh rằng $S=x_{1}+x_{2}+\cdots+x_{n}+n+1$ không là số chính phương.
\end{enumerate}
\end{btt}

\begin{btt}
Cho dãy vô hạn các số nguyên dương $1<n_{1}<\cdots<n_{k}<\cdots$ thỏa mãn không có hai số nào là hai số nguyên dương liên tiếp. Chứng minh rằng, với mỗi $m$ nguyên dương cho trước, giữa hai số $n_{1}+\cdots+n_{m}$ và $n_{1}+\cdots+n_{m+1}$ luôn có ít nhất một số chính phương.
\nguon{Titu Andresscu}
\end{btt}

\subsection*{Hướng dẫn bài tập tự luyện}

\begin{gbtt}
Cho các số nguyên dương $x,y,z$ thỏa mãn $x+y+z=169.$ Tìm giá trị lớn nhất và giá trị nhỏ nhất của tích $xyz.$
\loigiai{
Không mất tính tổng quát, ta giả sử $x\le y\le z.$ \\
Nhờ giả thiết $x\ge 1,y\ge 1,$ ta chỉ ra
$(x-1)(y-1)\ge 0.$ Một cách tương đương, ta có
$$xy\ge x+y-1.$$
Nhân hai vế bất đẳng thức cho $z,$ ta được
$$xyz\ge (x+y-1)z=(168-z)z.$$
Ta dự đoán dấu bằng ở chiều giá trị nhỏ nhất là $(x,y,z)=(1,1,167),$ vậy nên ta chỉ cần chứng minh thêm
$$(168-z)z\ge 167.$$
Khai triển rồi phân tích nhân tử, bất đẳng thức trên tương đương với
$$(z-1)(z-167)\le 0.$$
Bất đẳng thức trên đúng do
$$z-1\ge 0,\quad z-167\le z+(x-1)+(y-1)-169\le 0.$$ 
Tiếp theo, ta sẽ đi tìm giá trị lớn nhất của $xyz.$ Ta dự đoán dấu bằng ở chiều giá trị lớn nhất là $$(x,y,z)=(56,56,57),$$ vậy nên ta nghĩ đến việc áp dụng bất đẳng thức $AM-GM$ như sau
$$xyz\le \dfrac{z(x+y)^2}{4}=\dfrac{z(169-z)^2}{4}.$$
Để kết thúc bài toán, ta chỉ cần chứng minh
$$z(169-z)^2\le 178752.$$
Khai triển rồi chuyển vế, ta thấy bất đẳng thức trên tương đương
$$(z-57)(719z+12544)\ge 0.$$
Bất đẳng thức trên đúng do $z\ge \dfrac{168}{3}.$ Cuối cùng, ta kết luận.
\begin{itemize}
    \item[i,] Giá trị nhỏ nhất của $xyz$ là $167,$ đạt tại $(x,y,z)=(1,1,167)$ và các hoán vị.
    \item[ii,] Giá trị lớn nhất của $xyz$ là $178752,$ đạt tại $(x,y,z)=(56,56,57)$ và các hoán vị.    
\end{itemize}
}
\begin{luuy}
Bài toán tương tự của dạng này đã xuất hiện trong đề thi vào chuyên Phổ thông Năng khiếu $2016$
\begin{quote}
\it 
Cho $x,y,z$ là các số tự nhiên thỏa mãn $x+y+z=2017.$ Tìm giá trị lớn nhất của $xyz.$
\end{quote}
\end{luuy}
\end{gbtt} 

\begin{gbtt} \label{ineq1}
Cho các số nguyên dương $x,y,z$ thỏa mãn $x+y+z=53.$ Tìm giá trị lớn nhất và giá trị nhỏ nhất của biểu thức $P=xy+yz+zx.$
\loigiai{
Không mất tính tổng quát, ta giả sử $x\ge y\ge z.$ Rõ ràng $x,y,z$ không đồng thời bằng nhau, và ta nhận thấy
$$z\le 17,\quad x\ge 18.$$
Tương tự như các bài tập trước, ta chỉ ra
$yz\ge y+z-1.$ Đánh giá trên cho ta
$$P\ge x(y+z)+(y+z)-1=x(53-x)+52-x.$$
Ta dự đoán dấu bằng xảy ra tại $(x,y,z)=(51,1,1).$ Vì lẽ đó, ta sẽ đi chứng minh 
$$x(53-x)+52-x\ge 103$$
Chuyển vế và phân tích, bất đẳng thức trên tương đương với
$$(x-51)(x-1)\le 0.$$
Bất đẳng thức hiển nhiên đúng, do
$$x-1\ge 0,\quad x-51=x-(x+y+z-2)=(1-y)+(1-z)\le 0.$$
Tiếp theo, ta sẽ đi tìm giá trị lớn nhất của $P.$ Áp dụng bất đẳng thức Cauchy, ta có
$$xy+yz+zx\le \dfrac{(x+y)^2}{4}+z(x+y)=\dfrac{(53-z)^2}{4}+z(53-z).$$
Để kết thúc bài toán, ta cần chứng minh
$$\dfrac{(53-z)^2}{4}+z(53-z)\le 936.$$
Chuyển vế và phân tích, bất đẳng thức trên tương đương với
$$(x-17)(3x-55)\ge 0.$$
Bất đẳng thức trên đúng do $x\ge 18.$ Cuối cùng, ta có kết luận.
\begin{itemize}
    \item[i,] Giá trị nhỏ nhất của $P$ là $103,$ đạt tại $(x,y,z)=(1,1,51)$ và các hoán vị.
    \item[ii,] Giá trị lớn nhất của $P$ là $936,$ đạt tại $(x,y,z)=(18,18,17)$ và các hoán vị.    
\end{itemize}
}
\end{gbtt}

\begin{gbtt}
Cho các số tự nhiên $x,y,z$ thỏa mãn $x+y+z=40.$ Tìm giá trị lớn nhất của biểu thức $P=xy^2+yz^2+zx^2.$
\loigiai{
Không mất tính tổng quát, ta giả sử $y$ là số nằm giữa $x$ và $z.$ Giả sử này giúp ta lần lượt suy ra
\begin{align*}
    (y-x)(y-z)\le 0
    &\Rightarrow x(y-x)(y-z)\le 0
    \\&\Rightarrow xy^2+zx^2\le x^2y+xyz
    \\&\Rightarrow xy^2+yz^2+zx^2\le y\left(x^2+xz+z^2\right)
    \\&\Rightarrow xy^2+yz^2+zx^2\le y\left(x+z\right)^2
    \\&\Rightarrow xy^2+yz^2+zx^2\le y\left(40-y\right)^2.
\end{align*}
Tiếp theo, ta sẽ chứng minh rằng
$y\left(40-y\right)^2\le 9477.$ Bất đẳng thức này tương đương với
$$(y-13)\left(y^2-67y+729\right)\ge 0,$$
luôn đúng do $y^2-67y+729$ đổi dấu khi đi qua $y=13.$ \\
Dấu bằng xảy ra chẳng hạn tại $(x,y,z)=(0,13,27).$ Như vậy, $\max{P}=9477.$
}
\end{gbtt}

%nguyệt anh
\begin{gbtt}
Cho $55$ số nguyên dương $a_1,a_2,\ldots,a_{55}$  thỏa mãn điều kiện 
$$a_1\tron{a_1-1}\tron{a_1+1}+a_2\tron{a_2-1}\tron{a_2+1}+\cdots+a_{55}\tron{a_{55}-1}\tron{a_{55}+1}\le 2\cdot 10^6.$$
Chứng minh ta luôn tìm được ít nhất $2$ số bằng nhau trong $55$ số trên.
\loigiai{
Ta sẽ chứng minh bài toán bằng phản chứng. Giả sử không tồn tại $2$ số bằng nhau trong $55$ số trên.\\
Không mất tính tổng quát, ta giả sử
$1\le a_1< a_2< \cdots <a_{55}.$
Từ đây, ta suy ra
\begin{align*}
    &a_1\tron{a_1-1}\tron{a_1+1}+a_2\tron{a_2-1}\tron{a_2+1}+\cdots+a_{55}\tron{a_{55}-1}\tron{a_{55}+1}\\
    \ge \,&1\tron{1-1}\tron{1+1}+2\tron{2-1}\tron{2+1}+\cdots+55\tron{55-1}\tron{55+1}\\
    \ge \,&0 + 1\cdot2\cdot3+2\cdot3\cdot4+\cdots+54\cdot55\cdot56\\
    \ge \,&\dfrac{1}{4}\vuong{1\cdot2\cdot3\cdot\tron{4-1}+2\cdot3\cdot4\tron{5-1}+\cdots+54\cdot55\cdot56\cdot\tron{57-53}}\\
    \ge \,&\dfrac{1}{4}\tron{-6+1\cdot2\cdot3\cdot4-1\cdot2\cdot3\cdot4+2\cdot3\cdot4\cdot5-\cdots-53\cdot54\cdot55\cdot56+54\cdot55\cdot56\cdot57}\\
    \ge \,&\dfrac{1}{4}\tron{54\cdot55\cdot56\cdot57-6}
    \\>&\,2\cdot10^6.
\end{align*}
Điều này mâu thuẫn với giả thiết. Giả sử sai nên tồn tại ít nhất hai số bằng nhau  trong $55$ số trên.
}
\end{gbtt}

%nguyệt anh
\begin{gbtt}
Cho $100$ số nguyên dương $a_1,a_2,\ldots,a_{100}$ thỏa mãn điều kiện
$$\dfrac{1}{a_1}+\dfrac{1}{a_2}+\cdots+\dfrac{1}{a_{100}}\ge 15.$$
Chứng minh rằng có ít nhất ba số bằng nhau trong $100$ số trên.
\loigiai{
Ta sẽ chứng minh bài toán bằng phản chứng. Giả sử không tồn tại $3$ số bằng nhau trong $100$ số trên.\\
Không mất tính tổng quát, ta giả sử $a_1\le a_2<a_3\le a_4<\cdots < a_{99}\le a_{100}.$ Từ đây, ta suy ra
\begin{align*}
    \dfrac{1}{a_1}+\dfrac{1}{a_2}+\cdots+\dfrac{1}{a_{100}}&\le 2\tron{ \dfrac{1}{a_1}+\dfrac{1}{a_2}+\cdots+\dfrac{1}{a_{50}}}\\
    &\le 2\tron{\dfrac{1}{1}+\dfrac{1}{2}+\cdots+\dfrac{1}{50}}.
\end{align*}
Ta sẽ chứng minh rằng
$1+\dfrac{1}{2}+\dfrac{1}{3}+\cdots+\dfrac{1}{49}+\dfrac{1}{50}<\dfrac{15}{2}.$
Thật vậy
\begin{align*}
    1+\dfrac{1}{2}+\cdots+\dfrac{1}{50}
    &=\tron{1+\dfrac{1}{2}+\dfrac{1}{3}+\dfrac{1}{4}}+\tron{\dfrac{1}{5}+\cdots+\dfrac{1}{8}}+\cdots+\tron{\dfrac{1}{33}+\cdots+\dfrac{1}{64}}+\tron{\dfrac{1}{65}+\cdots+\dfrac{1}{100}}
    \\&<\dfrac{25}{12}+4\cdot\dfrac{1}{4}+8\cdot \dfrac{1}{8}+\cdots+32\cdot\dfrac{1}{32}+36\cdot\dfrac{1}{64}
    \\&<\dfrac{25}{12}+1+1+\cdots+1+\dfrac{9}{16}
    \\&<\dfrac{319}{48}
    \\&<\dfrac{7}{2}.
\end{align*}
Các lập luận trên cho ta điều mâu thuẫn với giả thiết. Giả sử sai, và ta có điều phải chứng minh.}
\end{gbtt}



\begin{gbtt}
Tìm các bộ số tự nhiên $ {{a}_{1}},{{a}_{2}},\ldots,{{a}_{2014}}$ thỏa mãn điều kiện sau.
$${{a}_{1}}+{{a}_{2}}+{{a}_{3}}+\cdots+{{a}_{2014}}\ge {{2014}^{2}},\quad
  a_{1}^{2}+a_{2}^{2}+a_{3}^{2}+\cdots+a_{2014}^{2}\le {{2014}^{3}}+1.$$
\nguon{Đề thi vào lớp 10 thành phố Hà Nội 2014 $-$ 2015}
\loigiai{
Ta viết lại hệ điều kiện đã cho thành
\begin{align}
    -2\cdot 2014\left( {{a}_{1}}+{{a}_{2}}+{{a}_{3}}+\cdots+{{a}_{2014}} \right)&\le -{{2\cdot 2014\cdot 2014}^{2}}, \tag{1}\label{pt1hn14}\\
    a_{1}^{2}+a_{2}^{2}+a_{3}^{2}+\cdots+a_{2014}^{2}&\le {{2014}^{3}}+1. \tag{2}\label{pt2hn14}
\end{align}
Lấy (\ref{pt2hn14}) cộng theo vế với (\ref{pt1hn14}) ta được
\begin{align*}
    &a_{1}^{2}+a_{2}^{2}+a_{3}^{2}+\cdots+a_{2014}^{2}-2\cdot2014\left( {{a}_{1}}+{{a}_{2}}+{{a}_{3}}+\cdots+{{a}_{2014}} \right)\le {{2014}^{3}}+1-{{2\cdot2014\cdot2014}^{2}} \\ 
  \Leftrightarrow \:&a_{1}^{2}+a_{2}^{2}+a_{3}^{2}+\cdots
  +a_{2014}^{2}-2.2014\left( {{a}_{1}}+{{a}_{2}}+{{a}_{3}}+\cdots+{{a}_{2014}} \right)+{{2014.2014}^{2}}\le 1 \\ 
 \Leftrightarrow \:&{{\left( {{a}_{1}}-2014 \right)}^{2}}+{{\left( {{a}_{2}}-2014 \right)}^{2}}+\cdots+{{\left( {{a}_{2014}}-2014 \right)}^{2}}\le 1. 
\end{align*}
Tới đây, ta xét các trường hợp sau.
\begin{enumerate}
    \item  Nếu ${{\left( {{a}_{1}}-2014 \right)}^{2}}+{{\left( {{a}_{2}}-2014 \right)}^{2}}+\cdots+{{\left( {{a}_{2014}}-2014 \right)}^{2}}=1,$ do \[{{a}_{1}},{{a}_{2}},\ldots,{{a}_{2014}}\in \mathbb{N}, \qquad {{\left( {{a}_{1}}-2014 \right)}^{2}}, {{\left( {{a}_{2}}-2014 \right)}^{2}},\ldots,{{\left( {{a}_{2014}}-2014 \right)}^{2}}\ge 0.\] 
    nên trong $2014$ số chính phương
    \[{{\left( {{a}_{1}}-2014 \right)}^{2}},\: {{\left( {{a}_{2}}-2014 \right)}^{2}},\ldots,{{\left( {{a}_{2014}}-2014 \right)}^{2}}\] 
    có đúng một số nhận giá trị là $1$ và các số còn lại nhận giá trị là $0$. Không mất tổng quát, giả sử 
\[\heva{ & {{\left( {{a}_{1}}-2014 \right)}^{2}}=1 \\ 
 & {{\left( {{a}_{2}}-2014 \right)}^{2}}=\cdots={{\left( {{a}_{2014}}-2014 \right)}^{2}}=0 .}\]
Từ hệ ở trên ta suy ra một trong hai trường hợp sau xay ra
\begin{align*}
    {{a}_{1}}=2013&,\quad{{a}_{2}}={{a}_{3}}=\cdots={{a}_{2014}}=2014, \\ 
     a_1=2015&,\quad {a}_{2}={{a}_{3}}=\cdots={{a}_{2014}}=2014. 
\end{align*}
Thử lại các trường hợp trên ta thấy không thỏa mãn.
\item Nếu ${{\left( {{a}_{1}}-2014 \right)}^{2}}+\tron{a_2-2014}^2+\cdots+{{\left( {{a}_{2014}}-2014 \right)}^{2}}=0$, ta có 
$${{\left( {{a}_{1}}-2014 \right)}^{2}}={{\left( {{a}_{2}}-2014 \right)}^{2}}=\cdots={{\left( {{a}_{2014}}-2014 \right)}^{2}}=0.$$
Do đó ${{a}_{1}}={{a}_{2}}={{a}_{3}}=\cdots={{a}_{2014}}=2014$. Thử lại ta thấy thỏa mãn yêu cầu bài toán.
\end{enumerate}
Như vây, bộ số tự nhiên thỏa mãn yêu cầu bài toán là \[\left( {{a}_{1}},{{a}_{2}},\ldots,{{a}_{2014}} \right)=\left( 2014,2014,\ldots,2014 \right).\]}
\end{gbtt}

\begin{gbtt}
Tìm tất cả các số nguyên dương $n$ sao cho tồn tại dãy số $x_1,x_2,\ldots,x_n$ thỏa mãn 
\[{{x}_{1}}+{{x}_{2}}+\cdots +{{x}_{n}}=5n-4\quad \text{ và }\quad \dfrac{1}{{{x}_{1}}}+\dfrac{1}{{{x}_{2}}}+\cdots +\dfrac{1}{{{x}_{n}}}=1.\]
\loigiai{
Không mất tính tổng quát ta giả sử ${{x}_{1}}\le {{x}_{2}}\le \cdots \le {{x}_{n}}.$ Theo bất đẳng thức $AM-GM$ ta có
$$5n-4=\left( {{x}_{1}}+{{x}_{2}}+\cdots +{{x}_{n}} \right)\left( \dfrac{1}{{{x}_{1}}}+\dfrac{1}{{{x}_{2}}}+\cdots +\dfrac{1}{{{x}_{n}}} \right)\ge n\sqrt[n]{{{x}_{1}}\cdots{{x}_{n}}}\cdot n\sqrt[n]{\dfrac{1}{{{x}_{1}}\cdots{{x}_{n}}}}={{n}^{2}}.$$
Do đó ta được ${{n}^{2}}-5n+4\le 0,$ hay 
$n\in \left\{ 1;2;3;4 \right\}.$ Ta xét các trường hợp sau.
\begin{enumerate}
    \item Với $n=1,$ ta có $x_1=1.$
    \item Với $n=2,$ ta có $x_1+x_2=6$ và $\dfrac{1}{x_1}+\dfrac{1}{x_2}=1$ hay $$\heva{x_1+x_2&=6\\x_1x_2&=6.}$$
    Hệ này không có nghiệm nguyên.
    \item  Với $n=3,$ ta có $x_1+x_2+x_3=11$ và $\dfrac{1}{x_1}+\dfrac{1}{x_2}+\dfrac{1}{x_3}=1.$ Ta thấy có $\tron{x_1,x_2,x_3}=(2,3,6)$ thỏa mãn.
    \item Với $n=4$ thì bất đẳng thức trên xảy ra dấu bằng nên $x_1=x_2=x_3=x_4=4.$
\end{enumerate}
Như vậy $n=1,n=3,n=4$ là tất cả các giá trị của $n$ thỏa mãn đề bài.}
\end{gbtt}

%nguyệt anh
\begin{gbtt}
Cho $k$ và $n_{l}<n_{2}<\cdots<n_{k}$ là các số nguyên dương lẻ. Chứng minh rằng
\[n_{1}^{2}-n_{2}^{2}+n_{3}^{2}-n_{4}^{2}+\cdots+n_{k}^{2} \geq 2 k^{2}-1.\]
\loigiai{
Biến đổi vế trái của bất đẳng thức, ta có
\begin{align*}
    n_{1}^{2}-n_{2}^{2}+n_{3}^{2}-n_{4}^{2}+\cdots+n_{k}^{2}
    &=n_{1}^{2}+\tron{n^2_3-n^2_2}+\tron{n^2_5-n^2_4}+\cdots+\tron{n^2_{k}-n^2_{k-1}}
    \\&=n^2_1+\tron{n_3-n_2}\tron{n_3+n_2}+\cdots+\tron{n_{k}-n_{k-1}}\tron{n_{k}+n_{k-1}}.
\end{align*}
Vì $n_{l}<n_{2}<\cdots<n_{k}$ là các số nguyên dương lẻ nên 
$$n_3-n_2\ge 2,\, n_5-n_4\ge 2,\ldots,\, n_k-n{k-1}\ge 2.$$
Từ nhận xét trên, ta suy ra
\begin{align*}
    n^2_1+\tron{n_3-n_2}\tron{n_3+n_2}+\cdots+\tron{n_{k}-n_{k-1}}\tron{n_{k}+n_{k-1}}&\ge n^2_1 +2\tron{n_2+n_3+\cdots n_{k}}\\&\ge 1+2\vuong{3+5+\cdots+\tron{2k-1}}\\&=2\vuong{1+3+\cdots+\tron{2k-1}}-1\\&=2k^2-1.
\end{align*}
Dấu bằng xảy ra khi và chỉ khi $n_1=1,n_2=3,\ldots,n_k=2k+1.$ Bài toán được chứng minh.}
\end{gbtt}

\begin{gbtt}
Cho $n$ là số tự nhiên $n \geq 2$ và $n$ số nguyên $x_{1}, x_{2}, \ldots, x_{n}$ thỏa mãn điều kiện $$x_{1}^{2}+x_{2}^{2}+\ldots+x_{n}^{2}+n^{3} \leq(2 n-1)\left(x_{1}+x_{2}+\cdots+x_{n}\right)+n^{2} .$$
\begin{enumerate}[a,]
    \item Chứng minh rằng các số $x_{i}$ $(i=1,2,\ldots, {n})$ đều là các số nguyên dương.
    \item  Chứng minh rằng $S=x_{1}+x_{2}+\cdots+x_{n}+n+1$ không là số chính phương.
\end{enumerate}
\loigiai{
\begin{enumerate}[a,]
    \item Từ giả thiết, ta có
    \begin{align*}
        &\quad{\:}\left[x_{n}^{2}-(2 n-1) x_{n}+n^{2}-n\right]+\ldots+\left[x_{n}^{2}-(2 n-1) x_{n}+n^{2}-n\right] \leq 0
        \\&\Leftrightarrow\left(x_{1}-n\right)\left(x_{1}-n+1\right)+\ldots+\left(x_{n}-n\right)\left(x_{n+1}-n+1\right) \leq 0.
    \end{align*}
Mặt khác, với mọi $i$ nguyên dương, ta có tích $\left(x_{i}-n\right)\left[x_{i}-(n-1)\right]$ là tích của hai số nguyên liên tiếp nên nó không âm. Đánh giá này giúp ta suy ra
$$\left(x_{1}-n\right)\left(x_{1}-n+1\right)+\ldots+\ldots+\left(x_{n}-n\right)\left(x_{n+1}-n+1\right) \ge 0.$$
Đối chiếu các đánh giá, ta được
$$\left(x_{1}-n\right)\left(x_{1}-n+1\right)+\ldots+\ldots+\left(x_{n}-n\right)\left(x_{n+1}-n+1\right) = 0.$$

Dấu bằng xảy ra chỉ khi mỗi số $x_i$ bằng $n$ hoặc $n-1.$ Bài toán được chứng minh.
    \item Dựa vào kết quả $x_{i} \in\{n ; n-1\}$ với mọi $i\in \left[1;n\right]$ ở câu a, ta có
     $$n(n-1) \leq x_{1}+x_{2}+\ldots+x_{n} \leq n^{2}.$$
    Cộng thêm $n+1$ vào mỗi vế, ta được
    $$n^2+1\le S\le n^2+n+1.$$
    Tuy nhiên, do $n^2+1>n^2$ và $n^2+n+1<(n+1)^2$ nên
    $$n^2<S<(n+1)^2.$$
    Theo bổ đề, $S$ không phải số chính phương. Bài toán được chứng minh.
\end{enumerate}}
\end{gbtt}

\begin{gbtt}
Cho dãy vô hạn các số nguyên dương $1<n_{1}<\cdots<n_{k}<\cdots$ thỏa mãn không có hai số nào là hai số nguyên dương liên tiếp. Chứng minh rằng, với mỗi $m$ nguyên dương cho trước, giữa hai số $n_{1}+\cdots+n_{m}$ và $n_{1}+\cdots+n_{m+1}$ luôn có ít nhất một số chính phương.
\nguon{Titu Andresscu}
\loigiai{Để giải quyết bài toán, ta sẽ chứng minh rằng
$$\sqrt{n_{1}+n_{2}+\cdots+n_{m+1}}-\sqrt{n_{1}+n_{2}+\cdots+n_{m}}>1.$$
Bằng cách nhân liên hợp, bất đẳng thức trên tương đương với
\[n_{m+1}>\sqrt{n_{1}+n_{2}+\cdots+n_{m+1}}+\sqrt{n_{1}+n_{2}+\cdots+n_{m}}. \tag{*}\]
Áp dụng bất đẳng thức $Cauchy - Schwarz$ cho hai số, ta có
$$VP^2(*)< 2\left(2 n_{1}+2 n_{2}+\cdots+2 n_{m}+n_{m+1}\right).$$
Chú ý rằng, bất đẳng thức trên không xảy ra dấu bằng. Để chứng minh được (*), ta chỉ cần chỉ ra
\[n_{m+1}^{2} \geqslant 2\left(2 n_{1}+2 n_{2}+\cdots+2 n_{m}+n_{m+1}\right). \tag{**}\]
Từ giả thiết, ta có $n_{1} \leqslant n_{2}-2, \leqslant n_{2} \leqslant n_{3}-2, \ldots, n_{m} \leqslant n_{m+1}-2,$ vậy nên
$$n_{1} \leqslant n_{m+1}-2 m, n_{2} \leqslant n_{m+1}-2(m-1), \ldots, n_{m} \leqslant n_{m+1}-2.$$
Từ đây ta suy ra
\begin{align*}
   VP(**)
   &\le  2\left[2\left(n_{m+1}-2 m+n_{m+1}-2(m-1)+\cdots+n_{m+1}-2\right)+n_{m+1}\right]
   \\&\le  2(2 m+1) n_{m+1}-4 m(m+1).
\end{align*}
Ta sẽ chứng minh rằng $n_{m+1}^{2} \geqslant 2(2 m+1) n_{m+1}-4 m(m+1).$ Bất đẳng thức ở trên tương đương với
$$
\left(n_{m+1}-2 m-1\right)^{2} \geqslant 1.
$$
Vì $n_{m+1}\ge n_1+2m\ge 2m+2,$ nên bất đẳng thức trên là đúng. Bài toán được chứng minh.}
\end{gbtt}

\section{Bất đẳng thức liên quan đến ước và bội}

\subsection*{Lí thuyết}

Trước khi đi tìm hiểu phần này, chúng ta sẽ nhắc lại một vài kiến thức đã học ở các chương trước.

\begin{enumerate}
    \item Cho $n$ số nguyên dương $a_1,a_2,\ldots,a_n.$ Đặt $d=\tron{a_1,a_2,\ldots,a_n}.$ Khi đó ta có thể viết
    $$a_1=db_1,\quad a_2=db_2,\ldots,a_n=db_n,$$
    trong đó $b_1,b_2,\ldots,b_n$ là các số nguyên dương nguyên tố cùng nhau.
    \item Cho hai số nguyên dương $a,b.$ Khi đó nếu $a$ chia hết cho $b$ thì $a\ge b.$
\end{enumerate}

Ngoài hai bổ đề trên, một số tính chất khác về tính chia hết liên quan đến ước, bội chung cũng được đề cập ở trong phần này.

\subsection*{Bài tập tự luyện}

\begin{btt}
Cho $43$ số nguyên dương có tổng bằng $2021$. Gọi $ d $ là ước chung lớn nhất của các số đó. Tìm giá trị lớn nhất của $d.$
\end{btt}

\begin{btt}
Cho $30$ số nguyên dương phân biệt có tổng bằng $2016$. Gọi $ d $ là ước chung lớn nhất của các số đó. Tìm giá trị lớn nhất của $ d$.
\end{btt}

\begin{btt}
Cho các số nguyên dương $a,b$ thỏa mãn $\dfrac{a+1}{b}+\dfrac{b+1}{a}$ cũng là một số nguyên dương. Chứng minh rằng 
$(a,b)\le \sqrt{a+b}.$
\end{btt}

\begin{btt}
Cho các số thực dương $x,y,z$ thỏa mãn $\dfrac{x+1}{y}+\dfrac{y+1}{z}+\dfrac{z+1}{x}$ là một số nguyên. Chứng minh rằng $(x,y,z)\leq \sqrt[3]{x y+y z+z x}.$
\nguon{Baltic Way 2018}
\end{btt}

\begin{btt}
Cho $a,b$ là các số nguyên dương. Chứng minh rằng
$$\min \{(a, b+1);(a+1, b)\} \leq \dfrac{\sqrt{4 a+4 b+5}-1}{2}.$$
\nguon{Japan Junior Mathematical Olympiad Finals 2019}
\end{btt}

\begin{btt}
Cho $a,b$ là hai số nguyên phân biệt khác $0$ thỏa mãn $ab(a+b)$ chia hết cho $a^{2}+a b+b^{2}$. Chứng minh rằng $|a-b|>\sqrt[3]{a b}$.
\end{btt}

\begin{btt}
Chứng minh với mọi số nguyên dương $m > n$ thì ta có bất đẳng thức
\[\left[ {m,n} \right] + \left[ {m + 1,n + 1} \right] > \dfrac{{2mn}}{{\sqrt {m - n} }}.\]
\nguon{Saint Peterburg Mathematical Olympiad 2001}
\end{btt}

\begin{btt}
Cho hai số nguyên dương $m,n$ phân biệt. Chứng minh rằng
$$(m, n)+(m+1, n+1)+(m+2, n+2) \leq 2|m-n|+1.$$
\nguon{India National Olympiad 2019}
\end{btt}

\begin{btt}
Cho $a_{0}<a_{1}<a_{2}<\ldots<a_{n}$ là các số nguyên dương. Chứng minh rằng
$$\dfrac{1}{\left[a_{0}, a_{1}\right]}+\dfrac{1}{\left[a_{1}, a_{2}\right]}+\ldots+\dfrac{1}{\left[a_{n-1}, a_{n}\right]} \leqslant 1-\dfrac{1}{2^{n}}.$$
\nguon{Kvant}
\end{btt}

\subsection*{Hướng dẫn bài tập tự luyện}
\begin{gbtt}
Cho $43$ số nguyên dương có tổng bằng $2021$. Gọi $ d $ là ước chung lớn nhất của các số đó. Tìm giá trị lớn nhất của $d.$
\loigiai{
Giả sử $43$ số tự nhiên đã cho là $a_1\le a_2\le \cdots\le a_{43}.$ Ta nhận thấy rằng
$$43 a_{1} \leq a_{1}+a_{2}+a_{3}+\cdots \cdots+a_{43}=2021.$$
Nhận xét trên cho ta ${a}_{1} \leq47.$ Do $d$ là ước của $a_1,$ ta có $d\le 47.$\\
Ứng với $d=47,$ ta có bộ số dưới đây thỏa mãn yêu cầu bài toán
		$$a_1=a_2=\cdots=a_{43}=47.$$
	Như vậy, giá trị lớn nhất của $d$ là $47.$}
\end{gbtt}

\begin{gbtt}
	Cho $30$ số nguyên dương phân biệt có tổng bằng $2016$. Gọi $ d $ là ước chung lớn nhất của các số đó. Tìm giá trị lớn nhất của $ d$.
	\loigiai{
		Giả sử $30$ số tự nhiên đã cho là $ a_1,a_2,a_3,\ldots ,a_{30}.$ \\
		Với mỗi $i=\overline{1,30},$ ta đặt $a_i=dk_i,$ ở đây $k_i$ là các số nguyên dương phân biệt. Phép đặt này cho ta $$d\left(k_1+k_2+k_3+\cdots+k_{30}\right)=2016.$$
		Nhờ vào điều kiện các $k_i$ dương, ta suy ra
		$$2016=d\left(k_1+k_2+k_3+\cdots+k_{30}\right)\ge d\left(1+2+\cdots+30\right)=\dfrac{30\cdot31d}{2}=465d.$$
		Ta được $2016\ge 465d.$ Giải bất phương trình trên, ta nhận thấy $d\le 4.$ \\
		Ứng với $d=4,$ ta có bộ số dưới đây thỏa mãn yêu cầu bài toán:
		$$a_1=4,\: a_2=8,\:a_3=12,\cdots,\:a_{28}=112,\:a_{29}=116,\:a_{30}=276.$$
	Như vậy, giá trị lớn nhất của $d$ là $4.$
	}
	\begin{luuy}
	Một cách tổng quát, với $k$ số nguyên dương $a_1,a_2,\cdots,a_k$ có tổng bằng $n$, ta có
$$\max{\left(a_1,a_2,\cdots,a_k\right)}
=\max\limits_{d\in\mathbb{N}}
\left\{
d \text{ là ước của } n\mid d\le\dfrac{2n}{k(k+1)}
\right\},
$$
với $\left(a_1,a_2,\cdots,a_k\right)$ là ước chung lớn nhất của $k$ số đã cho.
\end{luuy}
\end{gbtt}

\begin{gbtt}
Cho các số nguyên dương $a,b$ thỏa mãn $\dfrac{a+1}{b}+\dfrac{b+1}{a}$ cũng là một số nguyên dương. Chứng minh rằng 
$(a,b)\le \sqrt{a+b}.$
\nguon{Spanish Mathematical Olympiad 1996}
\loigiai{
Đặt $d=(a,b),$ lúc này tồn tại các số nguyên dương $m,n$ sao cho
$a=md, b=nd,(m, n)=1.$
Ta có
$$\dfrac{a+1}{b}+\dfrac{b+1}{a}=\dfrac{\left(m^{2}+n^{2}\right) d+(m+n)}{m n d}.$$
Do $\dfrac{a+1}{b}+\dfrac{b+1}{a}$ là số nguyên, ta thu được $d\mid (m+n),$ và như vậy, $d\le m+n.$ Từ đây, ta có
$$\sqrt{a+b}=\sqrt{d(m+n)}\ge d.$$
Dấu bằng xảy ra chẳng hạn tại $a=4,b=12.$ Bài toán được chứng minh.}
\end{gbtt}

\begin{gbtt}
Cho các số thực dương $x,y,z$ thỏa mãn $\dfrac{x+1}{y}+\dfrac{y+1}{z}+\dfrac{z+1}{x}$ là một số nguyên. Chứng minh rằng $(x,y,z)\leq \sqrt[3]{x y+y z+z x}.$
\nguon{Baltic Way 2018}
\loigiai{
Ta đặt $d=\tron{x,y,z}$. Lúc này, tồn tại các số nguyên dương $m,n,p$ sao cho
$$x=md,\quad y=nd, \quad z=pd, \quad \tron{m,n,p}=1.$$
Phép đặt này cho ta 
$$\dfrac{x+1}{y}+\dfrac{y+1}{z}+\dfrac{z+1}{x}= \dfrac{d\tron{m^2p+n^2m+p^2n}+ \tron{mp+nm+pn}}{mnpd}.$$
Do $\dfrac{x+1}{y}+\dfrac{y+1}{z}+\dfrac{z+1}{x}$ là số nguyên, ta thu được $d\mid \tron{mp+nm+pn}$.\\
Vì $d\mid \tron{mp+nm+pn}$ nên $d\le mp+nm+pn$. Từ đây, ta có
$$\sqrt[3]{{xy+yz+zx}}=\sqrt[3]{{d^2\tron{mn+np+pm}}} \ge d.$$
Dấu bằng xảy ra chẳng hạn tại $z=11, y=22, z=33$. Bài toán được chứng minh.}
\end{gbtt}

\begin{gbtt}
Cho $a,b$ là các số nguyên dương. Chứng minh rằng
$$\min \{(a, b+1);(a+1, b)\} \leq \dfrac{\sqrt{4 a+4 b+5}-1}{2}.$$
\nguon{Japan Junior Mathematical Olympiad Finals 2019}
\loigiai{
Đặt $d_1=(a, b+1)$ và $d_2=(a+1, b)$. Từ đây, ta suy ra
$d_{1}$ và $d_{2}$ là ước của $a+b+1.$\\
Ta dễ dàng chỉ ra $\tron{d_1,d_2}=1$, kéo theo $\left|d_{1}-d_{2}\right| \ge 1$ và
$$
d_{1} d_{2} \mid (a+b+1)
$$
Do đó, $d_{1} d_{2} \leq a+b+1$. Không mất tính tổng quát, ta giả sử $d_{2} \geq d_{1}+1$,  dẫn đến
$$
d_{1}\left(d_{1}+1\right) \leq d_{1} d_{2} \leq a+b+1.
$$
Từ đây, ta suy ra $d_{1} \leq \dfrac{-1+\sqrt{4 a+4 b+5}}{2}.$ Ta xét trường hợp dấu bằng đẳng thức xảy ra. Ta có 
$$\tron{a,b+1}=d,\quad (a+1,b)=d+1,\quad a+b+1=d(d+1).$$
Xét trong hệ đồng dư modulo $d,d+1$ cho ta 
$$a\equiv d\pmod{d},\qquad a\equiv -1\pmod{d+1}.$$
Từ nhận xét trên, ta chỉ ra $a\equiv d\pmod{d\tron{d+1}}.$ Tương tự, ta thu được
$$b\equiv -d-1\pmod{d\tron{d+1}}.$$
Điều này cho ta $a\ge d$ và $b\ge d^2-1.$ Vì $a+b+1=d\tron{d+1}$, dẫn đến $\tron{a,b}=\tron{d,d^2-1}.$\\
Dấu bằng xảy ra chẳng hạn khi $a=2,b=3.$ Bất
đẳng thức được chứng minh.}
\end{gbtt}

\begin{gbtt}
Cho $a,b$ là hai số nguyên phân biệt khác $0$ thỏa mãn $ab(a+b)$ chia hết cho $a^{2}+a b+b^{2}$. Chứng minh rằng $|a-b|>\sqrt[3]{a b}$.
\nguon{Rusia Mathematical Olympiad 2001}
\loigiai{
Đặt $d=(a,b),$ đồng thời đặt thêm $a=dx,\ b=dy.$ Khi đó $(x,y)=1$ và 
$$\dfrac{a b(a+b)}{a^{2}+a b+b^{2}}=\dfrac{d x y(x+y)}{x^{2}+x y+y^{2}}.$$
Từ giả thiết, ta suy ra $dxy(x+y)$ chia hết cho $x^2+x y+y^2.$ Trước hết, ta đi chứng minh.
\[\left(x,x^2+x y+y^2\right)=\left(y,x^2+x y+y^2\right)=\left(x+y,x^2+x y+y^2\right)=1.\tag{*}\label{usineq}\]
Ta nhận thấy rằng
$\left(x,x^2+x y+y^2\right)=\left(x,y^2\right).$
Do $(x,y)=1,$ hai số $x$ và $y^2$ không có ước nguyên tố chung, thế nên $\left(x,x^2+x y+y^2\right)=\left(x,y^2\right)=1.$ Một cách tương tự, ta có
$$\left(x,x^2+x y+y^2\right)=\left(y,x^2+x y+y^2\right)=1.$$
Để hoàn tất chứng minh (\ref{usineq}), ta đặt $\left(x+y,x^2+xy+y^2\right)=d.$ Phép đặt này cho ta
\begin{align*}
    \heva{&d\mid (x+y) \\ &d\mid\left(x^2+xy+y^2\right)}&\Rightarrow \heva{&d\mid\left(x^2+xy+y^2-y(x+y)\right) \\ &d\mid\left(x^2+xy+y^2-x(x+y)\right)}\\&\Rightarrow \heva{&d\mid x^2\\ &d\mid y^2}
    \\&\Rightarrow d\mid \left(x^2,y^2\right)=1.
\end{align*}
Kết hợp chứng minh ở (\ref{usineq}) và  $dxy(x+y)$ chia hết cho $x^2+x y+y^2,$ ta chỉ ra $\left(x^{2}+x y+y^{2}\right)\mid d,$ thế nên $$d \geqslant x^{2}+x y+y^{2}.$$ 
Dựa vào nhận xét trên, ta có đánh giá
\begin{align*}
    |a-b|^{3}=d^{3}|x-y|^{3}&=d^{3}|x-y|\left(x^{2}+x y+y^{2}\right)\\&\geqslant d^{2}\left(x^{2}+x y+y^{2}\right)\\&\ge a^2+ab+b^2\\&>ab.
\end{align*}
Như vậy, bài toán được chứng minh.}
\end{gbtt}

\begin{gbtt}
Chứng minh với mọi số nguyên dương $m > n$ thì ta có bất đẳng thức
\[\left[ {m,n} \right] + \left[ {m + 1,n + 1} \right] > \dfrac{{2mn}}{{\sqrt {m - n} }}.\]
\nguon{Saint Peterburg Mathematical Olympiad 2001}
\loigiai{
Biến đổi vế trái bất đẳng thức, ta được
\begin{align*}
    \left[ {m,n} \right] + \left[ {m + 1,n + 1} \right]
&=\dfrac{{mn}}{{\left( {m,n} \right)}} + \dfrac{{\left( {m + 1} \right)\left( {n + 1} \right)}}{{\left( {m + 1,n + 1} \right)}}
\\&=\dfrac{{mn}}{{\left( {m-n,n} \right)}} + \dfrac{{\left( {m+1} \right)\left( {n + 1} \right)}}{{\left( {m-n,n + 1} \right)}}.
\end{align*}
Từ giả thiết, ta có thể đặt $m=n+k,$ ở đây $k$ là số nguyên dương. \\Bằng phép đặt này, bất đẳng thức cần chứng minh trở thành
\[\dfrac{mn}{(k,n)}+\dfrac{(m+1)(n+1)}{(k,n+1)}\ge \dfrac{2mn}{\sqrt{k}}.\tag{*}\label{usineq2}\]
Áp dụng bất đẳng thức $AM-GM,$ ta có
$$VT(\ref{usineq2})>\dfrac{mn}{(k,n)}+\dfrac{m}{(k,n+1)}\ge\dfrac{2mn}{\sqrt{(k,n)(k,n+1)}}.$$
Để kết thúc bài toán bằng việc suy ra $(k,n)(k,n+1)\le k$, ta sẽ đi chứng minh
$$(k,n)(k,n+1)\mid k.$$
Thật vậy, ta có các đánh giá sau
\begin{itemize}
    \item[i,] $(n,n+1)=1\Rightarrow ((k,n),(k,n+1))=1.$
    \item[ii,] $(k,n)\mid k,\quad (k,n+1)\mid k.$
\end{itemize}
Các đánh giá trên cho ta hay, $(k,n)(k,n+1)\mid k.$ Và theo đó, toàn bộ bài toán được chứng minh.}
\end{gbtt}

\begin{gbtt}
Cho hai số nguyên dương $m,n$ phân biệt. Chứng minh rằng
$$(m, n)+(m+1, n+1)+(m+2, n+2) \leq 2|m-n|+1.$$
\nguon{India National Olympiad 2019}
\loigiai{
Không mất tính tổng quát, ta giả sử $m>n$. Đặt $k=m-n$, bất đẳng thức cần chứng minh trở thành
$$\tron{m,k}+\tron{m+1,k}+\tron{m+2,k}\le 2k+1.$$
Với $k=1,\ k=2,$ bài toán trở nên đơn giản.\\
Với $k>2,$ chỉ tối đa một ước chung ở vế trái bằng $k.$ Như vậy
$$\tron{m,k}+\tron{m+1,k}+\tron{m+2,k}\le k+\dfrac{k}{2}+\dfrac{k}{2}=2k<2k+1.$$
Bất đẳng thức cũng được chứng minh trong trường hợp $k>2$ này.}
\end{gbtt}

\begin{gbtt}
Cho $a_{0}<a_{1}<a_{2}<\ldots<a_{n}$ là các số nguyên dương. Chứng minh rằng
$$\dfrac{1}{\left[a_{0}, a_{1}\right]}+\dfrac{1}{\left[a_{1}, a_{2}\right]}+\ldots+\dfrac{1}{\left[a_{n-1}, a_{n}\right]} \leqslant 1-\dfrac{1}{2^{n}}$$
\nguon{Kvant}
\loigiai{
Trong bài toán này, ta xét hai trường hợp.
\begin{enumerate}
    \item Nếu $a_{n}\le 2^{n},$ ta nhận xét rằng với hai số nguyên dương $x,y$ bất kì, ta có
    $$[x,y](x,y)=xy.$$
    Áp dụng đẳng thức trên, ta chỉ ra
    $$\dfrac{1}{\left[a_{i-1}, a_{i}\right]}=\dfrac{\left(a_{i-1}, a_{i}\right)}{a_{i-1} a_{i}} \leqslant \dfrac{a_{i}-a_{i-1}}{a_{i-1} a_{i}}=\dfrac{1}{a_{i-1}}-\dfrac{1}{a_{i}}.$$
    Cho $i$ chạy từ $1$ đến $n$ rồi cộng theo vế các bất đẳng thức, ta được
    $$\dfrac{1}{\left[a_{0}, a_{1}\right]}+\dfrac{1}{\left[a_{1}, a_{2}\right]}+\ldots+\dfrac{1}{\left[a_{n-1}, a_{n}\right]} \leqslant \dfrac{1}{a_{0}}-\dfrac{1}{a_{n}} \leqslant 1-\dfrac{1}{2^{n}},$$
    và bài toán được chứng minh trong trường hợp trên.
    \item Nếu $a_{n}> 2^{n},$ ta sẽ chứng minh bài toán trong trường hợp này bằng quy nạp. \\
    Với $n=1,$ ta có $\left[a_{0}, a_{1}\right] \geqslant[1,2]=2,$ do đó
    $$\dfrac{1}{\left[a_{0}, a_{1}\right]} \leqslant \dfrac{1}{2}=1-\dfrac{1}{2}.$$
    Giả sử bất đẳng thức cần chứng minh đúng tới $n=k.$ Giả sử này cho ta
    $$\dfrac{1}{\left[a_{0}, a_{1}\right]}+\dfrac{1}{\left[a_{1}, a_{2}\right]}+\ldots+\dfrac{1}{\left[a_{k-1}, a_{k}\right]} \leqslant 1-\dfrac{1}{2^{k}}.$$
    Cộng hai vế bất đẳng thức trên với $\dfrac{1}{\left[a_{k-1}, a_{k}\right]},$ ta được
    $$\dfrac{1}{\left[a_{0}, a_{1}\right]}+\dfrac{1}{\left[a_{1}, a_{2}\right]}+\ldots+\dfrac{1}{\left[a_{k}, a_{k+1}\right]} \leqslant 1-\dfrac{1}{2^{k}}+\dfrac{1}{\left[a_{k}, a_{k+1}\right]}.$$    
    Bằng đánh giá
    $\dfrac{1}{\left[a_{k}, a_{k+1}\right]}\le \dfrac{1}{a_{k+1}}<\dfrac{1}{2^{k+1}},$
    ta chỉ ra
    $$\dfrac{1}{\left[a_{0}, a_{1}\right]}+\dfrac{1}{\left[a_{1}, a_{2}\right]}+\ldots+\dfrac{1}{\left[a_{k}, a_{k+1}\right]} \leqslant 1-\dfrac{1}{2^{k}}+\dfrac{1}{2^{k+1}}=1-\dfrac{1}{2^{k+1}}.$$
    Theo nguyên lí quy nạp, bài toán cũng được chứng minh trong trường hợp này.
\end{enumerate}
Dấu bằng xảy ra tại
$a_i=2^i,i=\overline{1,n}.$
Toàn bộ bài toán được chứng minh.}
\end{gbtt}

\section{Bất đẳng thức và phương pháp xét modulo}


Có rất nhiều phương trình, chẳng hạn $x^2-6^y=1$ tuy có vô hạn nghiệm trên tập hợp số thực nhưng lại vô nghiệm trong tập hợp số nguyên dương. Những tính chất số học đã hạn chế khá đáng kể tập giá trị của rất nhiều biểu thức. Chẳng hạn, ta có thể chứng minh được $x^2-6^y\ge3$ trong trường họp $x^2-6^y>0$ và $x,y$ nguyên dương. Điều này đã mở ra một hướng tiếp cận mới và ngày càng phổ biến trong số học, là sử dụng bất đẳng thức số học.
\\ \\
Trước tiên, ta sẽ đi tìm hiểu ví dụ đã nêu ở phần đặt vấn đề.
\subsection*{Ví dụ minh họa}

\begin{bx}
Cho các số nguyên dương $x,y$ phân biệt thỏa mãn $x^2>6^y.$ Tìm giá trị nhỏ nhất của biểu thức $P=x^2-6^y.$
\loigiai{
Với dự đoán $\min\limits_{x,y\in\mathbb{N}^*,\,x^2>6^y}\tron{x^2-6^y}=3$ như đã đặt vấn đề, ta xét các trường hợp sau.
\begin{enumerate}
    \item Nếu $x^2=6^y+1,$ lấy đồng dư theo modulo $5$ hai vế ta có $x^2\equiv 1+1\equiv2\pmod{5},$ vô lí.
    \item Nếu $x^2=6^y+2,$ lấy đồng dư theo modulo $5$ hai vế ta có $x^2\equiv 1+2\equiv3\pmod{5},$ vô lí.
    \item Nếu $x^2=6^y+3,$ ta thấy có cặp $(x,y)=(3,1)$ thỏa mãn.
\end{enumerate}
Như vậy $\min P=3,$ đạt được chẳng hạn tại $x=3$ và $y=1.$}
\begin{luuy}
Thông qua ví dụ mở đầu bên trên, chúng ta đã hình dung được một cách chứng minh $\min A=m,$ với $A$ là một biểu thức số học cho trước. Một số bài tập tự luyện dưới đây minh họa rõ hơn cho phương pháp này.
\end{luuy}
\end{bx}


\subsection*{Bài tập tự luyện}

\begin{btt}
Cho $x,y$ là hai số nguyên dương thỏa mãn ${{x}^{2}}+{{y}^{2}}+10$ chia hết cho $xy.$
\begin{enumerate}[a,]
 \item Chứng minh rằng $x$ và $y$ là hai số lẻ và nguyên tố cùng nhau.
   \item Chứng minh rằng $k=\dfrac{{{x}^{2}}+{{y}^{2}}+10}{xy}$ chia hết cho $4$ và $k\ge 12.$
\end{enumerate}
 \nguon{Chuyên Toán Phổ thông Năng khiếu}
\end{btt}

\begin{btt}
Cho số nguyên dương $n.$ Gọi $d$ là một ước nguyên dương của $2^n+15.$ Tìm giá trị nhỏ nhất của $d,$ biết $d$ có thể được biểu diễn dưới dạng $3x^2-4xy+3y^2,$ trong đó $x,y$ là các số nguyên.
\end{btt}

\begin{btt}
Cho $x,y$ là các số nguyên không đồng thời bằng $0.$ Tìm giá trị nhỏ nhất của biểu thức
\[F=\left|5x^2+11xy-5y^2\right|.\]
\nguon{Chọn học sinh giỏi Toán 9 Hà Tĩnh 2017 $-$ 2018}
\end{btt}

\begin{btt}
Tìm số nguyên tố $p$ nhỏ nhất sao cho tồn tại số nguyên dương $n$ thỏa mãn $x^2+5x+23$ chia hết cho $p.$
\nguon{Brazilian Math Olympiad 2003}
\end{btt}

\begin{btt}
Cho hai số nguyên dương $p$ và $q$ thỏa mãn $\sqrt{11}-\dfrac{p}{q}>0$. Chứng minh rằng
$$\sqrt{11}-\dfrac{p}{q}> \dfrac{1}{2p q}.$$
\nguon{Baltic Way 2018}
\end{btt}

\begin{btt}
Cho hai số nguyên dương $a$ và $b$ thỏa mãn điều kiện $a\sqrt{3} >b \sqrt{7}.$ Chứng minh rằng  \[a\sqrt{3}-b\sqrt{7} >\dfrac{1}{a+b}.\]
\nguon{Tạp chí Pi, tháng 5 năm 2017}
\end{btt}

\begin{btt}
Cho hai số nguyên dương $m$ và $n$ thỏa mãn $\sqrt{11}-\dfrac{m}{n}>0$. Chứng minh rằng
$$\sqrt{11}-\dfrac{m}{n} \geq \dfrac{3\tron{\sqrt{11}-3}}{m n}.$$
\nguon{Chuyên Toán Hà Nội 2020}
\end{btt}

\begin{btt}
Cho $x, y$ là các số tự nhiên khác $0.$ Tìm giá trị nhỏ nhất của biểu thức \[A=\left| 36^{2x}-5^y \right|.\]
\nguon{Chuyên Toán Thanh Hóa 2014}
\end{btt}

\begin{btt}
Cho $m,n$ là các số nguyên dương. Tìm giá trị nhỏ nhất của biểu thức $$A=\Big|2^m-181^n\Big|.$$ 
\nguon{Middle European Mathematical Olympiad 2017}
\end{btt}

\subsection*{Hướng dẫn bài tập tự luyện}

\begin{gbtt}
Cho $x,y$ là hai số nguyên dương thỏa mãn ${{x}^{2}}+{{y}^{2}}+10$ chia hết cho $xy.$
\begin{enumerate}[a,]
 \item Chứng minh rằng $x$ và $y$ là hai số lẻ và nguyên tố cùng nhau.
   \item Chứng minh rằng $k=\dfrac{{{x}^{2}}+{{y}^{2}}+10}{xy}$ chia hết cho $4$ và $k\ge 12.$
\end{enumerate}
 \nguon{Chuyên Toán Phổ thông Năng khiếu}
\loigiai{
\begin{enumerate}[a,]
    \item Ta giả sử phản chứng rằng $x$ chẵn. Khi đó $x^2+y^2+10$ là số chẵn, kéo theo $y$ chẵn. Ta có
    $$x^2+y^2+10\equiv 2\pmod{4},\quad xy\equiv 0\pmod{4}.$$
    Suy ra $x^2+y^2+10$ không chia hết cho $xy,$ vô lí. Như vậy giả sử phản chứng là sai, và ta có $xy$ lẻ.\\
    Tiếp theo, ta đặt $d=\left( x,y \right)$ và  $x=d{{x}_{0}},\: y=d{{y}_{0}}$ trong đó $\tron{x_0,y_0}=1$. 
    Từ đó ta có
    $${{x}^{2}}+{{y}^{2}}+10={{d}^{2}}x_{0}^{2}+{{d}^{2}}y_{0}^{2}+10$$
    chia hết cho ${{d}^{2}}{{x}_{0}}{{y}_{0}}$ nên suy ra $d^2\mid 10,$ kéo theo $d=1$ hay $x$ và $y$ nguyên tố cùng nhau.
    \item Đặt $x=2m+1,y=2n+1$ với $m,n$ là số tự nhiên. Khi đó ta có \[k=\dfrac{4\left( {{m}^{2}}+{{n}^{2}}+m+n+3 \right)}{\left( 2m+1 \right)\left( 2n+1 \right)}.\]
    Do $\tron{4,(2m+1)(2n+1)}=1$ và $k$ là số tự nhiên nên
    $$(2m+1)(2n+1)\mid\tron{m^2+n^2+m+n+3}.$$
    Ta suy ra $k$ chia hết cho $4$ từ đây. Tiếp theo, ta sẽ chứng minh $k\ge 12.$ Giả sử $k<12.$
    \begin{itemize}
        \item \chu{Trường hợp 1.} Nếu $k=4,$ ta có $x^2+y^2+10=4xy,$ hay là
        $$(x-2y)^2+10=3y^2.$$
        Lấy đồng dư theo modulo $3$ hai vế, ta có $(x-2y)^2\equiv 2\pmod{3},$ vô lí.
        \item \chu{Trường hợp 2.} Nếu $k=8,$ ta có $x^2+y^2+10=8xy,$ hay là
        $$(x-4y)^2+10=15y^2.$$
        Lấy đồng dư theo modulo $3$ hai vế, ta có $(x-4y)^2\equiv 2\pmod{3},$ vô lí.        
    \end{itemize}
    Giả sử sai, và ta có $k\ge 12.$ Dấu bằng xảy ra chẳng hạn tai $x=y=1.$ Bất đẳng thức được chứng minh.
\end{enumerate} }
\end{gbtt}

%nguyệt anh
\begin{gbtt}
Cho số nguyên dương $n.$ Gọi $d$ là một ước nguyên dương của $2^n+15.$ Tìm giá trị nhỏ nhất của $d,$ biết $d$ có thể được biểu diễn dưới dạng $3x^2-4xy+3y^2,$ trong đó $x,y$ là các số nguyên.
\loigiai{
Ta dễ dàng chỉ ra $2^n+15$ không chia hết cho $2,3,5$ nên $d$ cũng không chia hết cho $2,3,5.$ Ta sẽ tìm $d$ bằng cách loại dần đi các trường hợp nhỏ.
\begin{enumerate}
    \item Với $d=1,$ thế trở lại giả thiết cho ta
    $$3x^2-4xy+3y^2=1,$$
    hay $(3x-2y)^2+5y^2=3.$ Vì $x,y$ nguyên nên $\tron{3x-2y}^2=3,$ vô lí. 
        \item Với $d=7,$ ta suy ra $2^n+15$ chia hết cho $7$ hay $2^n$ chia $7$ dư $6.$\\
        Xét bảng đồng dư modulo $7$ đưới đây, ta có
    \begin{center}
        \begin{tabular}{c|c|c|c|c|c|c|c}
            $n$ & $0$&$1$ & $2$ & $3$ &$4$&$5$&$6$  \\
            \hline
              $2^n$ & $1$&$2$ & $4$ & $1$ &$2$&$4$&$1$  \\
        \end{tabular}
    \end{center}
    Không có số nguyên dương $n$ thỏa mãn.
    \item Với $d=11,$ ta có $(3x-2y)^2+5y^2=33.$ Từ đây, ta suy ra
    $$5y^2\le 33,$$
    dẫn đến $y^2\in\left\{0,1,4\right\}. $ Kéo theo $\tron{3x-2y}^2\in\left\{33;28;13\right\},$ vô lí.
    \item Với $d=13,$ ta có 
    $(3x-2y)^2+5y^2=39.$ Chứng minh tương tự, không có số nguyên $x,y$ thỏa mãn.
    \item Với $d=17,$ ta có $(3x-2y)^2+5y^2=51.$ Chứng minh tương tự, không có số nguyên $x,y$ thỏa mãn.
    \item Với $d=19,$ ta có $(3x-2y)^2+5y^2=57.$ Chứng minh tương tự, không có số nguyên $x,y$ thỏa mãn.
    \item Với $d=23,$ ta có $(3x-2y)^2+5y^2=69.$ Từ đây, ta nhận được trường hợp thỏa mãn là
    $$x=-2,\quad y=1.$$
    Chọn $n=3,$ ta thu được $2^n+15=23$ chia hết cho $d=23.$
\end{enumerate}
Như vậy, số nguyên $d$ nhỏ nhất là $23.$
}
\end{gbtt}


\begin{gbtt}
Cho $x,y$ là các số nguyên không đồng thời bằng $0.$ Tìm giá trị nhỏ nhất của biểu thức
\[F=\left|5x^2+11xy-5y^2\right|.\]
\nguon{Chọn học sinh giỏi Toán 9 Hà Tĩnh 2017 $-$ 2018}
\loigiai{
Gọi giá trị nhỏ nhất cần tìm là $m,$ đồng thời đặt
$$f\tron{x;y}=\left|5x^2+11xy-5y^2\right|.$$
Ta chia bài toán thành các bước làm sau.
\begin{enumerate}[\color{tuancolor}\bf\sffamily Bước 1.]
    \item Chứng minh $m$ là số lẻ và $m\le 5.$ \\
    Đầu tiên, do $f\tron{1;0}=5$ nên $m\le 5.$ Hơn nữa, ta còn có
    $$f\tron{2x;2y}=4f\tron{x;y}.$$
    Ta suy ra giá trị nhỏ nhất của $F$ không đạt được khi $x,y$ cùng chẵn, và vì thế $m$ lẻ.
    \item Chứng minh $m\ne 1.$ \\
    Nếu $m=1,$ tồn tại các số nguyên $x,y$ sao cho
    $$\left|5x^2+11xy-5y^2\right|=1.$$
    Biến đổi tương đương, ta được
    \begin{align*}
         \left|5x^2+11xy-5y^2\right|=1
         &\Leftrightarrow 100 x^{2}+220 x y-100 y^{2}=\pm 20 
         \\&\Leftrightarrow(10 x+11 y)^{2}-221 y^{2}=\pm 20
         \\&\Leftrightarrow(10 {x}+11 {y})^{2} \pm 20=221 {y}^{2}.
    \end{align*}
    Do $221y^2$ chia hết cho $13$ nên khi lấy đồng dư modulo $13$ hai vế, ta được
    $$(10x+11y)^2\equiv 6,7\pmod{13}.$$
    Đây là điều không thể xảy ra. Thật vậy, ta quan sát bảng đồng dư modulo $13$ dưới đây.
    \begin{center}
        \begin{tabular}{c|c|c|c|c|c|c|c}
           $A$  & $0$ & $\pm 1$ & $\pm 2$ & $\pm 3$ & $\pm 4$ & $\pm 5$ & $\pm 6$ \\
           \hline
            $A^2$ & $0$ & $1$ & $4$ & $9$ & $5$ & $3$ & $3$
        \end{tabular}
    \end{center}
    Không có số chính phương nào chia $13$ dư $6$ hoặc $7.$ Như vậy $m\ne 1.$
    \item Chứng minh $m\ne 3.$\\
        Nếu $m=3,$ tồn tại các số nguyên $x,y$ sao cho
    $$\left|5x^2+11xy-5y^2\right|=3.$$
    Một các tương tự, biến đổi tương đương ta thu được
    $$(10 x+11 y)^{2} \pm 60=221 y^{2}.$$
    Căn cứ vào bảng đồng dư đã xét ở bước 2, ta thấy trường hợp $m=3$ cũng không xảy ra.
\end{enumerate}
Thông qua các bước làm trên, ta kết luận $\min F=5,$ đạt được chẳng hạn khi $x=1$ và $y=0.$}
\end{gbtt}

\begin{gbtt}
Tìm số nguyên tố $p$ nhỏ nhất sao cho tồn tại số nguyên dương $n$ thỏa mãn $x^2+5x+23$ chia hết cho $p.$
\nguon{Brazilian Math Olympiad 2003}
\loigiai{
Ta sẽ tìm giá trị của $p$ thông qua thử một vài giá trị nhỏ. Trước hết, ta viết
$$4\tron{x^2+5x+23}=(2x+5)^2+67.$$
Đặt $2x+5=A.$ Ta lần lượt xét biểu thức trên trong các modulo của $p.$
\begin{enumerate}
    \item Nếu $p=3,$ ta có $A^2+67$ chia hết cho $3,$ vô lí do
    $$A^2+67\equiv 1,2\pmod{3}.$$
    \item Nếu $p=5,$ ta có $A^2+67$ chia hết cho $5,$ vô lí do 
    $$A^2+67\equiv 2,3,1\pmod{5}.$$
    \item Nếu $p=7,$ ta có $A^2+67$ chia hết cho $7,$ vô lí do 
    $$A^2+67\equiv 4,5,6,1\pmod{7}.$$
    \item Nếu $p=11,$ ta có $A^2+67$ chia hết cho $11.$ Ta lập bảng đồng dư modulo $11$
    \begin{center}
            \begin{tabular}{c|c|c|c|c|c|c}
            $A$ & $0$ & $\pm 1$ & $\pm 2$ & $\pm 3$ & $\pm 4$ & $\pm 5$ \\
            \hline
            $A^2$ & $0$ & $1$ & $4$ & $9$ & $5$ & $3$ \\
            \hline
            $A^2+67$ & $1$ & $2$ & $5$ & $10$ & $6$ & $4$             
            \end{tabular}
        \end{center}
        Ta có $A^2+67$ không chia hết cho $11$ với mọi $A,$ mâu thuẫn.
    \item Nếu $p=13,$ ta có $A^2+67$ chia hết cho $13.$ Ta lập bảng đồng dư modulo $13$
    \begin{center}
        \begin{tabular}{c|c|c|c|c|c|c|c}
           $A$  & $0$ & $\pm 1$ & $\pm 2$ & $\pm 3$ & $\pm 4$ & $\pm 5$ & $\pm 6$ \\
           \hline
            $A^2$ & $0$ & $1$ & $4$ & $9$ & $5$ & $3$ & $3$ \\
            \hline 
            $A^2+67$ & $2$ & $3$ & $6$ & $11$ & $7$ & $5$ & $5$
        \end{tabular}
    \end{center}    
    Ta có $A^2+67$ không chia hết cho $13$ với mọi $A,$ mâu thuẫn.
    \item Nếu $p=17,$ ta chỉ ra với $x=31$ hay $A=67$ thì $x^2+5x+23$ chia hết cho $17.$   
\end{enumerate}
Kết luận, $p=17$ là số nguyên tố cần tìm.}
\end{gbtt}



\begin{gbtt}
Cho hai số nguyên dương $p$ và $q$ thỏa mãn $\sqrt{11}-\dfrac{p}{q}>0$. Chứng minh rằng
$$\sqrt{11}-\dfrac{p}{q}> \dfrac{1}{2p q}.$$
\nguon{Baltic Way 2018}
\loigiai{Bất đẳng thức cần chứng minh tương đương với
$$q\sqrt{11}-p\ge \dfrac{1}{2p}.$$
Ta sẽ chứng minh rằng $11q^2-p^2\ge 2.$ Từ giả thiết $\sqrt{11}-\dfrac{p}{q}>0$, ta có $11q^2>p^2$, hay là $11q^2-p^2 \geq 1$. \\Ta xét bảng đồng dư theo modulo $11$ sau
  \begin{center}
            \begin{tabular}{c|c|c|c|c|c|c}
            $q$ & $0$ & $\pm 1$ & $\pm 2$ & $\pm 3$ & $\pm 4$ & $\pm 5$ \\
            \hline
            $q^2$ & $0$ & $1$ & $4$ & $9$ & $5$ & $3$ \\
            \hline
            $11q^2-p^2$ & $0$ & $10$ & $7$ & $2$ & $6$ & $8$
            \end{tabular}
        \end{center}
    Căn cứ vào dòng cuối cùng của bảng, ta suy ra $11q^2-p^2\not\equiv 1\pmod{11},$ và vì thế $11q^2-p^2\ge 2.$\\
    Từ đây, ta suy ra
    $$q\sqrt{11}-p=\dfrac{11q^2-p^2}{q\sqrt{11}+p}\ge\dfrac{2}{q\sqrt{11}+p}>\dfrac{2}{p}.$$
    Vế trái lớn hơn vế phải. Bất đẳng thức được chứng minh.}
\end{gbtt}

\begin{gbtt}
Cho hai số nguyên dương $a$ và $b$ thỏa mãn điều kiện $a\sqrt{3} >b \sqrt{7}.$ Chứng minh rằng  \[a\sqrt{3}-b\sqrt{7} >\dfrac{1}{a+b}.\]
\nguon{Tạp chí Pi, tháng 5 năm 2017}
\loigiai{
Ta sẽ chứng minh rằng $3a^2-7b^2\ge 3.$ Từ giả thiết $a\sqrt{3}>b\sqrt{7}$, ta có $3a^2>7b^2$, hay là $3a^2-7b^2 \geq 1$.\\ Ta xét bảng đồng dư theo modulo $7$ sau
  \begin{center}
            \begin{tabular}{c|c|c|c|c}
            $a$ & $0$ & $\pm 1$ & $\pm 2$ & $\pm 3$\\
            \hline
            $3a^2$ & $0$ & $3$ & $5$ & $6$\\
            \hline
            $3a^2-7b^2$ & $0$ & $3$ & $5$ & $6$
            \end{tabular}
        \end{center}
    Căn cứ vào dòng cuối cùng của bảng, ta suy ra $3a^2-7b^2\not\equiv 1,2\pmod{7}.$ Kết hợp với  $3a^2-7b^2 \geq 1,$ ta thu được $3a^2-7b^2\ge 3.$
    Tiếp theo, biến đổi vế trái bất đẳng thức cần chứng minh, ta có
$$a\sqrt{3} -b\sqrt{7} = \dfrac{3a^2-7b^2}{a\sqrt{3}+b\sqrt{7}} \geq \dfrac{3}{a\sqrt{3}+b\sqrt{7}} 
=\dfrac{1}{a\dfrac{\sqrt{3}}{3} + b\dfrac{\sqrt{7}}{3}} > \dfrac{1}{a+b}.$$
Tổng kết loại, bài toán được chứng minh.}
\end{gbtt}

\begin{gbtt}
Cho hai số nguyên dương $m$ và $n$ thỏa mãn $\sqrt{11}-\dfrac{m}{n}>0$. Chứng minh rằng
$$\sqrt{11}-\dfrac{m}{n} \geq \dfrac{3\tron{\sqrt{11}-3}}{m n}.$$
\nguon{Chuyên Toán Hà Nội 2020}
\loigiai{Bất đẳng thức cần chứng minh tương đương với
\[\sqrt{11} n-m \geq \dfrac{3\left(\sqrt{11}-3\right)}{m}\Leftrightarrow 11 n^{2} \geq m^{2}+6\left(\sqrt{11}-3\right)+\dfrac{9\left(\sqrt{11}-3\right)^{2}}{m^{2}}.\tag{*}\]
Đến đây, ta xét các trường hợp sau.
\begin{enumerate}
    \item Với $m=1,$ bất đẳng thức được chứng minh do
    $$VP(*)=1+6\left(\sqrt{11}-3\right)+9\left(\sqrt{11}-3\right)^{2}<11 \leq 11 n^{2}.$$
    \item Với $m=2,$ bất đẳng thức được chứng minh do 
    $$VP(*)=4+6\left(\sqrt{11}-3\right)+\dfrac{9\left(\sqrt{11}-3\right)^{2}}{4}<11 \leq 11 n^{2}.$$
    \item Với $m=3,$ ta có đánh giá sau cho vế phải của (*)
    $$VP(*) \leq m^{2}+6\left(\sqrt{11}-3\right)+\dfrac{9\left(\sqrt{11}-3\right)^{2}}{9}=m^{2}+2.$$
    Ta sẽ chứng minh rằng $11n^2\ge m^2+2.$ Thật vậy, từ giả thiết ta suy ra $11 n^{2}-m^{2}>0$ và do đó $$11 n^{2}-m^{2} \geq 1.$$ Ta xét bảng đồng dư theo modulo $11$ sau
        \begin{center}
           \begin{tabular}{c|c|c|c|c|c|c}
            $m$ & $0$ & $\pm 1$ & $\pm 2$ & $\pm 3$ & $\pm 4$ & $\pm 5$ \\
            \hline
            $m^2$ & $0$ & $1$ & $4$ & $9$ & $5$ & $3$ \\
            \hline
            $11n^2-m^2$ & $0$ & $10$ & $7$ & $2$ & $6$ & $8$ 
            \end{tabular}
        \end{center}
   Căn cứ vào dòng cuối cùng của bảng, ta suy ra $11n^2-m^2\not\equiv 1 \pmod{11}.$ \\Kết hợp với chứng minh $11n^2-m^2\ge 1$ ở trên, ta được $11n^2-m^2=2.$ 
\end{enumerate}
Dấu bằng xảy ra khi và chỉ khi $n=1,m=3.$ Bất đẳng thức đã cho được chứng minh.}
\end{gbtt}

\begin{gbtt}
Cho $x, y$ là các số tự nhiên khác $0.$ Tìm giá trị nhỏ nhất của biểu thức \[A=\left| 36^{2x}-5^y \right|.\]
\nguon{Chuyên Toán Thanh Hóa 2014}
\loigiai{
Đặt $k=2x$ nên $k$ là số chẵn. Ta đi tìm giá trị nhỏ nhất của 
$$A=\left| {{36}^{k}}-{{5}^{y}} \right|.$$
Dễ thấy $A=11$ khi  $k=2,y=2$ hay $x=1,y=2$. Ta chứng minh $\min A=11.$\\ 
Thật vậy, do $A=\left| {{36}^{k}}-{{5}^{y}} \right|$ nên ta suy ra
\begin{center}
    $A={{36}^{k}}-{{5}^{y}}$ hoặc $A={{5}^{y}}-{{36}^{k}}.$ 
\end{center}
Giả sử $\min A<11.$ Ta xét các trường hợp sau.
\begin{enumerate}
    \item Xét trường hợp $A={{36}^{k}}-{{5}^{y}}$. Khi đó ta thấy $A$ luôn có chữ số tận cùng là $1.$\\
    Nếu $A<11$ và $A$ có chữ số tận cùng là 1 thì $A=1,$ kéo theo  ${{36}^{k}}-{{5}^{y}}=1.$
\begin{itemize}
    \item\chu{Trường hợp 1.} Nếu $y$ là số chẵn, ta có \[{{36}^{k}}\equiv 0\pmod{3},\qquad {{5}^{y}}\equiv 1 \pmod{3}.\] Từ đây, ta suy ra $1={{36}^{k}}-{{5}^{y}}\equiv 0-1\equiv 2\pmod {3},$ vô lí.
    \item\chu{Trường hợp 2.} Nếu $y$ là số lẻ, ta có \[{{36}^{k}}\equiv 0\pmod{4},\qquad {{5}^{y}}\equiv 1\pmod{4}.\] Từ đây, ta suy ra $1={{36}^{k}}-{{5}^{y}}\equiv0-1 \equiv 3\pmod{4},$ vô lí.
\end{itemize}
\item Xét trường hợp $A={{5}^{y}}-{{36}^{k}}$. Ta thấy $A={{5}^{y}}-{{36}^{k}}$ có chữ số tận cùng là $9.$\\
Nếu $A<11$ và $A$ có chữ số tận cùng là $9$ thì $A=9,$ kéo theo
\[{{5}^{y}}-{{36}^{k}}=9.\]
Ta suy ra $5^y$ chia hết cho $3$, vô lí.
\end{enumerate}
Vậy giá trị nhỏ nhất của $A$ là $11,$ đạt được khi $x=1,y=2$.}
\end{gbtt}

\begin{gbtt}
Cho $m,n$ là các số nguyên dương. Tìm giá trị nhỏ nhất của biểu thức $$A=\Big|2^m-181^n\Big|.$$ 
\nguon{Middle European Mathematical Olympiad 2017}
\loigiai{
Đầu tiên, dễ thấy $A$ lẻ. Ta sẽ đi tìm giá trị nhỏ nhất của $A$ bằng cách loại dần các giá trị nguyên đủ nhỏ.
\begin{enumerate}
    \item Nếu $2^m-181^n=-1,$ lấy đồng dư theo modulo $3$ ta được
    $$2^m =181^n-1 \equiv 0 \pmod{3}.$$
    Ta suy ra $2^m$ chia hết cho $3,$ vô lí.
    \item Nếu $2^m-181^n=1,$ lấy đồng dư theo modulo $4$ ta được
    $$2^m =181^n+1 \equiv 2 \pmod{4}.$$
    Điều này là không thể xảy ra với $m\ge 2.$ Đối với $m=1,$ ta có $181^n=1,$ kéo theo $n=0,$ vô lí.
    \item Nếu $2^m-181^n=-3,$ lấy đồng dư theo modulo $4$ ta được
    $$2^m=181^n-3 \equiv 2 \pmod{4}.$$
    Điều này là không thể xảy ra với $m\ge 2.$ Đối với $m=1,$ ta có $181^n=5,$ vô lí.
    \item Nếu $2^m-181^n=3,$ trước hết ta nhận thấy do
    $$2^m \equiv 1, 2, 4, 8 \pmod{15},\quad 81^n+3 \equiv 4 \pmod{15}$$ 
    nên $2^m \equiv 4 \pmod{15}$. Xét các số dư của $m$ khi chia cho $4,$ ta có $m\equiv 2\pmod{4}.$ 
    \\Đặt $m=4k+2.$ Phép đặt này cho ta
    		\begin{align*}
			 2^m=181^n+3
			& \Rightarrow 2^{4k+2} \equiv (-1)^n+3 \pmod{13}\\
			&\Rightarrow 4 \cdot 16^k \equiv (-1)^n+3 \pmod{13}\\
			&\Rightarrow 4 \cdot 3^k \equiv (-1)^n+3 \pmod{13}.
		\end{align*}
			Ta có $(-1)^n+3 \equiv 2, 4 \pmod{13}$ và $4 \cdot 3^k \equiv 4, 12, 10 \pmod{13}$. Do đó $(-1)^n+3 \equiv 4 \pmod{13}$, suy ra $n$ phải chẵn hay $n=2q$, trong đó $q$ là số nguyên dương. Như vậy
		\[
		\begin{aligned}
		A &=\left|2^m-181^n\right|\\&=\left|2^{4k+2}-181^{2q}\right|\\& =\left|\left(2^{2k+1}-181^q\right)\left(2^{2k+1}+181^q\right)\right| \\&\ge 183.
		\end{aligned}
		\]	
		Dấu bằng của bất đẳng thức trên thậm chí không xảy ra. Ta có $A>183$ trong trường hợp này.
		\item Nếu $2^m-181^n=\pm 5,$ lấy đồng dư theo modulo $15$ hai vế ta được
		$$2^m\equiv 5,10\pmod{15}.$$
		Khi xét các số dư của $m$ khi chia cho $4,$ ta nhận thấy $2^m\equiv 1,2,4,8\pmod{15},$ mâu thuẫn.
		\item Nếu $2^m-181^n=\pm 7,$ ta chỉ ra $m=15$ và $n=2$ thoả mãn.
\end{enumerate}
Như vậy $\min A=7,$ đạt được tại $m=15$ và $n=2.$}
\end{gbtt} 	%bất đẳng thức số học
\chapter{Hàm phần nguyên, phần lẻ}

Phần nguyên và phần lẻ là hai khái niệm mới hoàn toàn ở toán trung học cơ sở. Đây là những vấn đề số học khó và mang nhiều ý nghĩa lớn. Gần đây, hàm phần nguyên và phần lẻ đã bắt đầu xuất hiện nhiều hơn ở trong các đề thi học sinh giỏi qua các bài toán hay và đẹp. Ở chương VII, tác giả muốn giới thiệu những tính chất cơ bản nhất của hai hàm số này.

\section*{Định nghĩa, tính chất}

\begin{light}
\chu{Định nghĩa 1.} Phần nguyên của một số thực $x$ $\big($thường được kí hiệu là $[x]\big)$ là số nguyên lớn nhất và nhỏ hơn với $x.$
\end{light}

Một vài ví dụ về phần nguyên có thể kể đến như
 $$[2,4]=2,\quad [3]=3,\quad [-12,27]=-13,\quad \vuong{\sqrt{2}}=1.$$
 
\begin{light}
\chu{Định nghĩa 2.} Phần lẻ của một số thực $x$ $\big($được kí hiệu là $\{x\}\big)$ hàm số được định nghĩa theo công thức $\{x\}=x-\vuong{x}.$
\end{light} 
Một vài ví dụ về phần lẻ có thể kể đến như
 $$\{2,4\}=0,4,\quad \{3\}=0,\quad \{-12,27\}=0,73,\quad \left\{\sqrt{2}\right\}=0,414123562\ldots$$

Dưới đây là một vài tính chất quan trọng của phần nguyên và phần lẻ.
\begin{light}
\begin{enumerate}[\color{blue!60!black}\sffamily\bfseries Tính chất 1.]
    \item Cho số thực $x$ và số nguyên $n.$ Các công thức dưới đây là tương đương nhau
    \begin{multicols}{3}
    \begin{itemize}
        \item[i,] $\vuong{x}=n.$
        \item[ii,] $n\le x< n+1.$
        \item[iii,] $x-1< n\le x.$
    \end{itemize}
    \end{multicols}
    \item Với mọi số thực $x,$ ta luôn có $0\le\{x\}<1.$
\end{enumerate}
\end{light}

\section{Tính toán phần nguyên}

\subsection*{Ví dụ minh họa}

\begin{bx}
Cho số nguyên dương $n.$ Hãy tính $\vuong{\tron{\sqrt{n}+\sqrt{n+1}}^{2}}.$ 
\loigiai{
Khai triển biểu thức trong dấu phần nguyên, ta có
$$\tron{\sqrt{n}+\sqrt{n+1}}^2=2n+1+2\sqrt{n(n+1)}=2n+1+\sqrt{4n(n+1)}.$$
Dựa trên so sánh $(2n)^2<4n(n+1)<(2n+1)^2,$ ta chỉ ra
$$4n+1<\tron{\sqrt{n}+\sqrt{n+1}}^2<4n+2.$$
Theo như tính chất đã biết về phần nguyên, ta kết luận rằng
$$\vuong{\tron{\sqrt{n}+\sqrt{n+1}}^2}=4n+1.$$}
\end{bx}

\begin{bx}
Tính tổng $A=\bigg[\sqrt{1}\bigg]+\bigg[\sqrt{2}\bigg]+\bigg[\sqrt{3}\bigg]+\ldots+\bigg[\sqrt{101}\bigg].$
\loigiai{
Với mọi số nguyên dương $x,$ ta có 
    $$\bigg[\sqrt{x}\bigg]=n\Leftrightarrow n^2\le x<(n+1)^2.$$
    Ứng với mỗi số nguyên dương $n,$ có tất cả $2n+1$ nguyên  $x$ thỏa mãn $n^2\le x<(n+1)^2,$ thế nên có đúng $2n+1$ số nguyên $x$ thỏa mãn $\bigg[\sqrt{x}\bigg]=n.$ Dựa vào nhận xét này, ta chỉ ra
    \[\begin{aligned}
    A
    &=1(2\cdot 1+1)+2(2\cdot 2+1)+3(2\cdot 3+1)+\ldots+9(2\cdot 9+1)+10\cdot 2\\
    &=2\tron{1^2+2^2+\ldots+9^2}+\tron{1+2+\ldots+9}+20\\
    &=\dfrac{2\cdot9\cdot10\cdot19}{6}+\dfrac{9\cdot10}{2}+20=635.  
    \end{aligned}\]}
\end{bx}

\subsection*{Bài tập tự luyện}

\begin{btt}
Cho số nguyên dương $n.$ Tính giá trị biểu thức $\vuong{\sqrt{4 n^{2}+\sqrt{16 n^{2}+8 n+3}}}.$
\end{btt}

\begin{btt}
Với mỗi số nguyên dương $n,$ ta đặt
$$x_n=\vuong{\dfrac{n+1}{\sqrt{2015}}}-\vuong{\dfrac{n}{\sqrt{2015}}}.$$
Hỏi trong dãy $x_1,x_2,\ldots,x_{2014}$ có bao nhiêu số bằng $0$?
\nguon{Hanoi Open Mathematics Competitions 2014}
\end{btt}

\begin{btt}
Tính tổng
$B=\bigg[\sqrt[3]{1}\bigg]+\bigg[\sqrt[3]{2}\bigg]+\bigg[\sqrt[3]{3}\bigg]+\ldots+\bigg[\sqrt[3]{1000}\bigg].$
\end{btt}

\begin{btt}
Tìm phần nguyên của số $$A=1+\dfrac{1}{\sqrt{2}}+\dfrac{1}{\sqrt{3}}+\dfrac{1}{\sqrt{4}}+\ldots+\dfrac{1}{\sqrt{1000000}}.$$
\end{btt}

\begin{btt}
Tìm phần nguyên của số  $$B=\dfrac{1}{\sqrt[3]{4}}+\dfrac{1}{\sqrt[3]{5}}+\dfrac{1}{\sqrt[3]{6}}+\ldots+\dfrac{1}{\sqrt[3]{1000000}}.$$
\end{btt}

\begin{btt}
Biết rằng số $A_n$ sau đây có $n$ dấu căn $(n\ge 1)$. Hãy tính phần nguyên của
\[A_n=\sqrt{2+\sqrt{2+\ldots+\sqrt{2+\sqrt{2}}}}.\]
\end{btt}

\begin{btt}
Biết rằng số $B_n$ sau đây có $n$ dấu căn $(n\ge 1)$. Hãy tính phần nguyên của
\[B_n=\sqrt[3]{6+\sqrt[3]{6+\ldots+\sqrt[3]{6+\sqrt[3]{6}}}}.\]
\end{btt}

\begin{btt}
Với mỗi số nguyên tố $p,$ chứng minh rằng $$S_p=\left[\sqrt{2}+\sqrt[3]{\dfrac{3}{2}}+\sqrt[4]{\dfrac{4}{3}}+\cdots+\sqrt[p+1]{\dfrac{p+1}{p}}\right]$$ 
cũng là một số nguyên tố.
\end{btt}

\begin{btt}
Cho số thực $a\ge\dfrac{1+\sqrt{5}}{2}$ và số nguyên dương $n.$ Tính giá trị biểu thức \[A=\vuong{\dfrac{1+\vuong{\dfrac{1+n{a^2}}{a}}}{a}}.\]
\end{btt}

\subsection*{Hướng dẫn bài tập tự luyện}

\begin{gbtt}
Cho số nguyên dương $n.$ Tính giá trị biểu thức $\vuong{\sqrt{4 n^{2}+\sqrt{16 n^{2}+8 n+3}}}.$
\loigiai{Nhận xét $4n+1<\sqrt{(4n+1)^2+2}=\sqrt{16n^2+8n+3}<\sqrt{16n^2+16n+4}=4n+2$ cho ta $$(2n+1)^2=4n^2+4n+1<4n^2+\sqrt{16n^2+8n+3}<4n^2+4n+2<4n^2+8n+4=(2n+2)^2.$$ Khai căn theo tất cả các vế, ta được 
$$2n+1<\sqrt{4 n^{2}+\sqrt{16 n^{2}+8 n+3}}<2n+2.$$
Do vậy, $\vuong{\sqrt{4 n^{2}+\sqrt{16 n^{2}+8 n+3}}}=2n+1.$}
\end{gbtt}

\begin{gbtt}
Với mỗi số nguyên dương $n,$ ta đặt
$$x_n=\vuong{\dfrac{n+1}{\sqrt{2015}}}-\vuong{\dfrac{n}{\sqrt{2015}}}.$$
Hỏi trong dãy $x_1,x_2,\ldots,x_{2014}$ có bao nhiêu số bằng $0$?
\nguon{Hanoi Open Mathematics Competitions 2014}
\loigiai{
Với mọi số nguyên dương $n,$ ta luôn có
$$0\le \vuong{\dfrac{n+1}{\sqrt{2015}}}-\vuong{\dfrac{n}{\sqrt{2015}}} \le 1.$$
Do đó, $0\le x_n\le 1.$ Ta lại có
\begin{align*}
  x_1+x_2+\ldots+x_{2014}&=\vuong{\dfrac{2}{\sqrt{2015}}}-\vuong{\dfrac{1}{\sqrt{2015}}}+\vuong{\dfrac{3}{\sqrt{2015}}}-\vuong{\dfrac{2}{\sqrt{2015}}}+\ldots+\vuong{\dfrac{2015}{\sqrt{2015}}}-\vuong{\dfrac{2014}{\sqrt{2015}}}\\
  &=\vuong{\dfrac{2015}{\sqrt{2015}}}-\vuong{\dfrac{1}{\sqrt{2015}}}=\vuong{\dfrac{2015}{\sqrt{2015}}}.  
\end{align*}

Ta nhận thấy rẳng $44^2<2015<45^2.$ Điều này dẫn đến $\vuong{\dfrac{2015}{\sqrt{2015}}}=44.$ Từ đây, ta suy ra có $44$ dãy số bằng $1$ nên số dãy số bằng $0$ là $2014-44=1970.$\\
Như vậy, trong $2014$ dãy số có $1970$ số bằng $0.$
}
\end{gbtt}

\begin{gbtt}\label{phannguyen1}
Tính tổng
$B=\bigg[\sqrt[3]{1}\bigg]+\bigg[\sqrt[3]{2}\bigg]+\bigg[\sqrt[3]{3}\bigg]+\ldots+\bigg[\sqrt[3]{1000}\bigg].$
\loigiai{
Với mọi số nguyên dương $x,$ ta có 
    $$\bigg[\sqrt[3]{x}\bigg]=n\Leftrightarrow n^3\le x<(n+1)^3.$$
    Ứng với mỗi số nguyên dương $n,$ có tất cả $(n+1)^3-n^3=3n^2+3n+1$ số nguyên  $x$ thỏa mãn $$n^3\le x<(n+1)^3,$$ thế nên có đúng $3n^2+3n+1$ số nguyên $x$ thỏa mãn $\bigg[\sqrt[3]{x}\bigg]=n.$ Dựa vào nhận xét này, ta chỉ ra
    \[\begin{aligned}
    B&=1\tron{3\cdot 1^2+3\cdot 1+1}+2\tron{3\cdot 2^2+3\cdot 2+1}+\ldots +9\tron{3\cdot 9^2+3\cdot 9+1}+10
    \\&=3\tron{1^3+2^3+\ldots+9^3}+3\tron{1^2+2^2+\ldots+9^2}+\tron{1+2+\ldots+9}+10
    \\&=3\tron{\dfrac{9\cdot10}{2}}^2+3\tron{\dfrac{9\cdot 10\cdot 19}{6}}+\dfrac{9\cdot10}{2}+10=6985.
    \end{aligned}\]}
\end{gbtt}

\begin{gbtt}
	Tìm phần nguyên của số $A=1+\dfrac{1}{\sqrt{2}}+\dfrac{1}{\sqrt{3}}+\dfrac{1}{\sqrt{4}}+\ldots+\dfrac{1}{\sqrt{1000000}}.$
	\loigiai{
Trước tiên ta sẽ đi chứng minh rằng, với mọi số nguyên dương $n,$ ta luôn có
$$2\sqrt{n+1}-2\sqrt{n}<\dfrac{1}{\sqrt{n}}<2\sqrt{n}-2\sqrt{n-1}.$$
	Thật vậy, ta có $2\sqrt{n+1}-2\sqrt{n}=2\left(\sqrt{n+1}-\sqrt{n}\right)=\dfrac{2}{\sqrt{n+1}+\sqrt{n}}<\dfrac{2}{\sqrt{n}+\sqrt{n}}=\dfrac{1}{\sqrt{n}}$.\\
	Bất đẳng thức thứ hai được chứng minh tương tự. Áp dụng vào bài toán, ta được
	\begin{align*}
		A&>1+2\left[\left(\sqrt{3}-\sqrt{2}\right)+\left(\sqrt{4}-\sqrt{3}\right)+...+\left(\sqrt{1000001}-\sqrt{1000000}\right)\right]\\ 
		A&<1+2\left[\left(\sqrt{2}-1\right)+\left(\sqrt{3}-\sqrt{2}\right)+...+\left(\sqrt{1000000}-\sqrt{999999}\right)\right]
	\end{align*}
Các kết quả kể trên cho ta biết
\begin{align*}
    & A>1+2\left(\sqrt{1000001}-\sqrt{2}\right)>1998\\ 
	& A<1+2\left(\sqrt{1000000}-1\right)<1+2.999=1999.
\end{align*}
Ta có $1998<A<1999.$ Kết quả, phần nguyên của số thực $A$ bằng $1998.$}
\end{gbtt}

\begin{gbtt}
Tìm phần nguyên của số  $B=\dfrac{1}{\sqrt[3]{4}}+\dfrac{1}{\sqrt[3]{5}}+\dfrac{1}{\sqrt[3]{6}}+\ldots+\dfrac{1}{\sqrt[3]{1000000}}.$
\loigiai{
Với mọi số nguyên dương $n.$ ta có
	$$\left(1+\dfrac{2}{3n}\right)^3=1+\dfrac{2}{n}+\dfrac{4}{3n^2}+\dfrac{8}{27n^3}>1+\dfrac{2}{n}+\dfrac{1}{n^2}=\left(1+\dfrac{1}{n}\right)^2.$$
Lấy căn bậc ba các vế, ta lần lượt chỉ ra
	\[1 + \dfrac{2}{{3n}} > {\left( {1 + \dfrac{1}{n}} \right)^{\dfrac{2}{3}}} \Rightarrow {n^{\dfrac{2}{3}}} + \dfrac{2}{{3{n^{\dfrac{1}{3}}}}} > {\left( {n + 1} \right)^{\dfrac{2}{3}}} \Rightarrow \dfrac{1}{{\sqrt[3]{n}}} > \dfrac{3}{2}\left[ {\sqrt[3]{{{{\left( {n + 1} \right)}^2}}} - \sqrt[3]{{{n^2}}}} \right]\tag{1}\label{ceilng.1}\] 
Hoàn toàn tương tự, ta chứng minh được
\[\dfrac{1}{\sqrt[3]{n}}<\dfrac{3}{2}\left[\sqrt[3]{n^2}-\sqrt[3]{\left(n-1\right)^2}\right].\tag{2}\label{ceilng.2}\]
Kết hợp (\ref{ceilng.1}) và (\ref{ceilng.2}), ta có $$\dfrac{3}{2}\left[\sqrt[3]{\left(n+1\right)^2}-\sqrt[3]{n^2}\right]<\dfrac{1}{\sqrt[3]{n}}<\dfrac{3}{2}\left[\sqrt[3]{n^2}-\sqrt[3]{\left(n-1\right)^2}\right].$$
Lần lượt cho $n=1,2,\ldots,1000000$ rồi lấy tổng, ta nhận thấy $14996<B<14997.$ Vậy $[B]=14996.$}
\end{gbtt}

\begin{gbtt}
Biết rằng số $A_n$ sau đây có $n$ dấu căn $(n\ge 1)$. Hãy tính phần nguyên của
\[A_n=\sqrt{2+\sqrt{2+\ldots+\sqrt{2+\sqrt{2}}}}.\]
\loigiai{
Ta có $A_n>\sqrt{2}>1$ với mọi $n.$ Ngoài ra, ta chứng minh được
$$A_{n+1}=\sqrt{2+A_n}.$$
Ta đi chứng minh bằng quy nạp rằng $A_n<2$ với mọi $n\in\mathbb{Z^+}.$ Với $n=1$, ta có $$A_n=A_1=\sqrt{2}<2.$$ Do vậy, khẳng định đúng với $n=1$. Giả sử khẳng định đúng với $n=1,2,\ldots,k.$ Với $n=k+1$, ta có \[A_{k+1}=\sqrt{2+A_k}<\sqrt{2+2}=2.\]
Khẳng định cũng đúng với $n=k+1$ nên nó đúng với mọi số nguyên dương $n$. Vậy ta đã chứng minh được $1<A_n<2$. Do đó $[A_n]=1$ với mọi $n.$}
\end{gbtt}

\begin{gbtt}
Biết rằng số $B_n$ sau đây có $n$ dấu căn $(n\ge 1)$. Hãy tính phần nguyên của
\[B_n=\sqrt[3]{6+\sqrt[3]{6+\ldots+\sqrt[3]{6+\sqrt[3]{6}}}}.\]
\loigiai{
Ta có $B_n>\sqrt[3]{6}>1$ với mọi $n.$ Ngoài ra, ta còn chứng minh được
$$B_{n+1}=\sqrt[3]{6+B_n}.$$
Ta đi chứng minh bằng quy nạp rằng $B_n<2$ với mọi $n\in\mathbb{Z^+}.$ Với $n=1$, ta có $$B_n=B_1=\sqrt[3]{6}<2.$$ Do vậy, khẳng định đúng với $n=1$. Giả sử khẳng định đúng với $n=1,2,\ldots,k.$ Với $n=k+1$, ta có
\[B_{k+1}=\sqrt[3]{6+B_k}<\sqrt[3]{6+2}=2.\]
Khẳng định cũng đúng với $n=k+1$ nên nó đúng với mọi số nguyên dương $n$. Vậy ta đã chứng minh được $1<B_n<2$. Do đó $[B_n]=1$ với mọi $n.$}
\end{gbtt}

\begin{gbtt}
Với mỗi số nguyên tố $p,$ chứng minh rằng $$S_p=\left[\sqrt{2}+\sqrt[3]{\dfrac{3}{2}}+\sqrt[4]{\dfrac{4}{3}}+\cdots+\sqrt[p+1]{\dfrac{p+1}{p}}\right]$$ 
cũng là một số nguyên tố.
\loigiai{
Hiển nhiên $S_{p}>p$. Ngoài ra, khi áp dụng bất đẳng thức $AM-GM$ cho mỗi bộ $k+1$ số dương, ta có $$\sqrt[k+1]{\dfrac{k+1}{k}}<\dfrac{\dfrac{k+1}{k}+1+1+\cdots+1}{k+1}=1+\dfrac{1}{k(k+1)},$$ 
trong đó trên tử số có $k$ số $1$ không kể  $\dfrac{k+1}{k},$ với $k=1,2, \ldots, p$. Đánh giá trên cho ta biết $$S_{p}<p+\dfrac{1}{1.2}+\dfrac{1}{2.3}+\cdots+\dfrac{1}{p(p+1)}<p+1 .$$ 
Ta có $p<S_p<p,$ và như vậy $\vuong{S_p}=p$ cũng là số nguyên tố. Bài toán được chứng minh.}
\begin{luuy}
Bài toán tương tự của bài này cũng đã từng xuất hiện trên \chu{tạp chí Toán học và Tuổi trẻ số 364}:
\begin{quote}
    \it Tính phần nguyên của $S,$ biết rằng
\[S=\sqrt{\dfrac{2+1}{2}}+\sqrt[3]{\dfrac{3+1}{3}}+\sqrt[4]{\dfrac{4+1}{4}}\ldots+\sqrt[n]{\dfrac{n+1}{n}}.\]
\end{quote}
\end{luuy}
\end{gbtt}

\begin{gbtt}
Cho số thực $a\ge\dfrac{1+\sqrt{5}}{2}$ và số nguyên dương $n.$ Tính giá trị biểu thức \[A=\vuong{\dfrac{1+\vuong{\dfrac{1+n{a^2}}{a}}}{a}}.\]
\loigiai{
Từ giả thiết $a\ge\dfrac{1+\sqrt{5}}{2}$ ta được $a^2-a-1\ge 0$ hay $a\ge\dfrac{1}{a}+1.$ Ta nhận thấy rằng \[\vuong{\dfrac{1+n{a^2}}{a}}=\dfrac{1}{a}+na-\alpha \text{ với }0\le\alpha <1.\]
Dựa vào nhận xét bên trên, ta chỉ ra
$$\vuong{\dfrac{1+\vuong{\dfrac{1+n{a^2}}{a}}}{a}}=\vuong{\dfrac{1+\dfrac{1}{a}+na-\alpha}{a}}=\vuong{\left(1+\dfrac{1}{a}-\alpha\right)\dfrac{1}{a}+n}.$$
Với các đánh giá
$\left(1+\dfrac{1}{a}-\alpha\right)\dfrac{1}{a}\le\left(a-\alpha\right)\dfrac{1}{a}=1-\dfrac{\alpha}{a}<1$ và $\left(1+\dfrac{1}{a}-\alpha\right)\dfrac{1}{a}>\dfrac{1}{a^2}>0,$ ta có
$$\vuong{\dfrac{1+\vuong{\dfrac{1+n{a^2}}{a}}}{a}}=n.$$
Như vậy, giá trị biểu thức đã cho là $A$ bằng $n.$}
\end{gbtt}

\section{Giải phương trình có chứa phần nguyên}

\subsection*{Ví dụ minh họa}

\begin{bx}
Giải phương trình $\left[\dfrac{x-3}{2}\right]=\left[\dfrac{x-2}{3}\right] $.
\loigiai{
	Ta giả sử phương trình đã cho có nghiệm nguyên. Ta đặt $$\vuong{\dfrac{x-3}{2}}=n,$$
	trong đó $n$ là số nguyên. Theo đó
	$$\heva{
		& n\le\dfrac{x-3}{2}<n+1\\ 
		& n\le\dfrac{x-2}{3}<n+1}
		\Rightarrow\heva{
		& 2n+3\le x<2n+5\\ 
		& 3n+2\le x<3n+5}
        \Rightarrow\heva{
		& 3n+2<2n+5\\ 
		& 2n+3<3n+5}\Rightarrow -2<n<3.$$
	Do $n$ là số nguyên nên $n\in\{-1;0;1;2\}.$ Ta xét các trường hợp kể trên.
	\begin{enumerate}
		\item Nếu $n=-1,$ ta có $1\le x<3$ và $-1\le x<2$ nên $x\in[1,2)$.
		\item Nếu $n=0,$ ta có $3\le x<5$ và $2\le x<5$ nên $x\in[3,5)$.
		\item Nếu $n=1,$ ta có $5\le x<7$ và $5\le x<8$ nên $x\in[5,7)$.
		\item Nếu $n=2,$ ta có $7\le x<9$ và $8\le x<11$ nên $x\in[8,9).$
	\end{enumerate}
	Như vậy, tập nghiệm của phương trình đã cho là $S=[1,2)\cup[3,7)\cup[8,9)$.}
\end{bx}

\begin{bx}
Tìm tất cả các nghiệm thực của phương trình
$4{x^2} - 40\left[ x \right]  + 51 = 0$.
\loigiai{
Trước tiên, ta có nhận xét rằng
	$$\left( {2x - 3} \right)\left( {2x - 17} \right) = 4{x^2} - 40x + 51 \leqslant 4{x^2} - 40\left[ x \right]  + 51 = 0.$$
	Ta suy ra $\dfrac{3}{2} \leqslant x \leqslant \dfrac{{17}}{2}$ và $1 \leqslant \left[ x \right]  \leqslant 8$. Như vậy $x = \dfrac{{\sqrt {40\left[ x \right]  + 51} }}{2}.$ Lấy phần nguyên hai vế, ta được 
    \[\left[ x \right]  = \left[ {\dfrac{{\sqrt {40\left[ x \right]  + 51} }}{2}} \right]\tag{*}\label{clmm}.\]
	Lần lượt thay $\left[ x \right]  \in \left\{ {1,2,3,4,5,6,7,8} \right\}$ vào (\ref{clmm}), ta thấy $\left[ x \right]  = \left\{ {2,6,7,8} \right\}$ thỏa mãn. Bằng phép thay như trên, ta tìm được tập nghiệm của phương trình đã cho là
		\[S = \left\{ {\dfrac{{\sqrt {29} }}{2};\dfrac{{\sqrt {189} }}{2};\dfrac{{\sqrt {229} }}{2};\dfrac{{\sqrt {269} }}{2}} \right\}\]}
\end{bx}

\subsection*{Bài tập tự luyện}

\begin{btt}
Giải phương trình sau trên tập số thực
$$\vuong{\dfrac{{2x - 1}}{3}}+ \vuong{\dfrac{{4x + 1}}{6}} = \dfrac{{5x - 4}}{3}.$$
\end{btt}

\begin{btt}
Giải phương trình nghiệm nguyên
	$$\di\bigg[ {\dfrac{x}{2}} \bigg]  + \bigg[ {\dfrac{x}{3}} \bigg]  + \bigg[ {\dfrac{x}{5}} \bigg]  = x.$$
\end{btt}

\begin{btt}
Giải phương trình sau trên tập số thực
\[x^3-\left[ x\right]=3.\]
\end{btt}

\begin{btt}
Giải phương trình sau trên tập số thực \[x^4=2x^2+\left[x\right].\]
\end{btt}

\begin{btt}
Giải phương trình nghiệm tự nhiên $$x=8\left[\sqrt[4]{x}\right]+3.$$
\end{btt}

\begin{btt}
Giải phương trình sau trên tập số thực
\[\left[ x\left[ x\right]\right]=1.\]
\end{btt}

\begin{btt}
Giải hệ phương trình sau trên tập số thực \[\left\{\begin{array}{l}x+[y]+\left\{z\right\}=3,9 \\ y+[z]+\left\{x\right\}=3,5 \\ z+[x]+\left\{y\right\}=2\end{array}\right.\]
\end{btt}

\subsection*{Hướng dẫn bài tập tự luyện}

\begin{gbtt}
Giải phương trình sau trên tập số thực
\[\vuong{\dfrac{{2x - 1}}{3}}+ \vuong{\dfrac{{4x + 1}}{6}} = \dfrac{{5x - 4}}{3}.\]
\loigiai{
	Trước hết ta đặt $\dfrac{{2x - 1}}{3} = y,$ và ta có $x = \dfrac{{3y + 1}}{2}.$ Thay vào phương trình, ta được
	\[[y] +\vuong{{y + \dfrac{1}{2}}} = \dfrac{{5y - 1}}{2} \Rightarrow [2y]  = \dfrac{{5y - 1}}{2}\]
	Bây giờ, ta tiếp đặt 
	$\di\dfrac{{5y - 1}}{2} = t.$ Ta lại có 
	$$y = \dfrac{{2t + 1}}{5} \Rightarrow \left[ {\dfrac{{4t + 2}}{5}} \right]  = t \Rightarrow 0 \leqslant \dfrac{{4t + 2}}{5} - t < 1.$$ 
	Do $t$ là số nguyên nên ta suy ra được 
	\[t \in \left\{ { - 2, - 1,0,1,2} \right\} \Rightarrow y \in \left\{ { - \dfrac{3}{5}, - \dfrac{1}{5},\dfrac{1}{5},\dfrac{3}{5},1} \right\}\]
	Kết luận, tập nghiệm của phương trình đã cho là
	$\di S = \left\{ { - \dfrac{2}{5},\dfrac{1}{5},\dfrac{4}{5},\dfrac{7}{5},2} \right\}.$}
\end{gbtt}

\begin{gbtt}
Giải phương trình nghiệm nguyên
	$\di\bigg[ {\dfrac{x}{2}} \bigg]  + \bigg[ {\dfrac{x}{3}} \bigg]  + \bigg[ {\dfrac{x}{5}} \bigg]  = x.$
\nguon{Canada 1998}	
\loigiai{
	Vì vế trái là một số nguyên nên $x$ cũng phải là một số nguyên. Ta đặt $x = 30q + r,$ trong đó $r$ là thương của phép chia $x$ chia $30$. Phương trình ban đầu trở thành
	\[31q + \bigg[ {\dfrac{r}{2}} \bigg]  + \bigg[ {\dfrac{r}{3}} \bigg]  + \bigg[ {\dfrac{r}{5}} \bigg]  = 30q + r\] 
	Chuyển vế, ta thu được phương trình tương đương
	\[q = r - \left( {\bigg[ {\dfrac{r}{2}} \bigg]  + \bigg[ {\dfrac{r}{3}} \bigg]  + \bigg[ {\dfrac{r}{5}} \bigg]} \right)\]
	Như vậy, tập nghiệm của phương trình đã cho là
	$$S=\left\{x \:|\: x=30\tron{r -  {\left[ {\dfrac{r}{2}} \right]  - \left[ {\dfrac{r}{3}} \right]  - \left[ {\dfrac{r}{5}} \right]} }+r,r=0,1,2,\ldots,29\right\}.$$
	}
\end{gbtt}

\begin{gbtt}
Giải phương trình sau trên tập số thực
\[x^3-\left[ x\right]=3.\]
\loigiai{
	Đặt $\left[ x\right]=n.$ Từ phương trình đã cho, ta có $x=\sqrt[3]{n+3}.$ Kết hợp với định nghĩa hàm phần nguyên, ta được
		$$n+1>\sqrt[3]{n+3}\ge n\Rightarrow{(n+1)^3}>n+3\ge{n^3}\Rightarrow\heva{& n+3\ge{n^3}\\ 
		&{(n+1)^3}-n-3>0}\Rightarrow n=1.$$
	Thế trở lại $n=1,$ ta tìm ra $x=\sqrt[3]{4}.$ Đây cũng là nghiệm thực duy nhất của phương trình đã cho.}
\end{gbtt}

\begin{gbtt}
Giải phương trình sau trên tập số thực \[x^4=2x^2+\left[x\right].\]
\loigiai{
Phương trình đã cho tương đương với $[x]=x^4-2x^2.$ Ta xét các trường hợp sau.
    \begin{enumerate}
        \item Với $x^2\le 2,$ ta có nhận xét rằng
        \[- \sqrt 2  \leqslant x \leqslant \sqrt 2  \Rightarrow \left[ x \right]  \leqslant 1 \Rightarrow \left[ x \right]  \in \left\{ { - 1;0;1} \right\}.\]
        Thế trở lại $[x]=-1,[x]=0$ và $[x]=1$ vào phương trình ban đầu, ta lần lượt tìm ra $x=0$ và $x=1.$
        \item Với $x^2>2,$ ta có nhận xét rằng
		\begin{align*}
			{x^2} > 2 \Rightarrow \left[ x \right]  > 0 \Rightarrow x > \sqrt 2  &\Rightarrow {x^2}\left( {{x^2} - 2} \right) = \dfrac{{\left[ x \right] }}{x} \leqslant 1 \Rightarrow {x^2} - 2 \leqslant \dfrac{1}{x} < 1\\&
			\Rightarrow x < \sqrt 3  \Rightarrow \sqrt 2  < x < \sqrt 3  \Rightarrow \left[ x \right]  = 1
		\end{align*}
		Thế trở lại phương trình, ta được
		$x = \sqrt {1 + \sqrt 2 }.$
    \end{enumerate}
    Như vậy, tập nghiệm của phương trình đã cho là $S=\left\{0;1;\sqrt{1+\sqrt{2}}\right\}.$}
\end{gbtt}

\begin{gbtt}
Giải phương trình nghiệm tự nhiên \[x=8\left[\sqrt[4]{x}\right]+3.\]
\loigiai{
	Đặt $\sqrt[4]{x}=n+y,$ trong đó $n$ là số tự nhiên và $0\le y<1.$ Phương trình đã cho trở thành 
	$$(n+y)^4=8n+3\Leftrightarrow y=\sqrt[4]{8n+3}-n.$$ Như vậy ta cần tìm $n$ sao cho $\sqrt[4]{8n+3}-n\in[0,1)$. Ta lần lượt suy ra $$\sqrt[4]{8n+3}-n\ge 0\Rightarrow 8n+3>n^4\Rightarrow n\in\{0,1,2\}.$$
	Tới đây, ta xét các trường hợp sau.
	\begin{enumerate}
		\item Với $n=0,$ ta có $y=\sqrt[4]{3},$ thỏa mãn điều kiện $0\le y<1.$
		\item Với $n=1,$ ta có $y=\sqrt[4]{11}-1,$ thỏa mãn điều kiện $0\le y<1.$
		\item Với $n=2,$ ta có $y=\sqrt[4]{19}-2,$ thỏa mãn điều kiện $0\le y<1.$
	\end{enumerate}
Như vậy, tập nghiệm tự nhiên của phương trình đã cho là $S=\{3,11,19\}$.}
\end{gbtt}

\begin{gbtt}
Giải phương trình sau trên tập số thực
\[\left[ x\left[ x\right]\right]=1.\]
\loigiai{
Từ định nghĩa về phần nguyên, ta có $1\le x\left[ x\right] <2.$ Ta xét các trường hợp sau đây.
	\begin{enumerate}
		\item Nếu $x<-1$ thì $\left[ x\right]\le-2$ và $x\left[ x\right] >2$, mâu thuẫn.
		\item Nếu $x=-1$ thì $\left[ x\right]=-1$ và $x\left[ x\right]=\left(-1\right)\left(-1\right)=1$ và $\left[ x\left[ x\right]\right]=1$, thỏa mãn.
		\item Nếu $-1<x<0$ thì $\left[ x\right]=-1$ và $x\left[ x\right]=-x<1$, mâu thuẫn.
		\item Nếu $0\le x<1$ thì $\left[ x\right]=0$ và $x\left[ x\right]=0<1$, mâu thuẫn.
		\item Nếu $1\le x<2$ thì $\left[ x\right]=1$ và $x\left[ x\right]=\left[ x\right]=1$, thỏa mãn.
		\item Nếu $x\ge 2$ thì $\left[ x\right]\ge 2$ và $x\left[ x\right]=2x\ge 4,$ mâu thuẫn.
\end{enumerate}
Vậy tập nghiệm của phương trình là $S=\left[1;2\right)\cup\{-1\}$.}
\end{gbtt}

\begin{gbtt}
Giải hệ phương trình sau trên tập số thực \[\left\{\begin{array}{l}x+[y]+\left\{z\right\}=3,9 \\ y+[z]+\left\{x\right\}=3,5 \\ z+[x]+\left\{y\right\}=2\end{array}\right.\]
\loigiai{
Ta đặt $[x]=a,\{x\}=\alpha,[y]=b.\{y\}=\beta,[z]=c,\{z\}=\gamma.$ Hệ phương trình đã cho trở thành
\[\heva{a+b+\alpha+\gamma&=3,9 \\
b+c+\beta+\alpha&=3,5 \\
c+a+\gamma+\beta&=2}\tag{1}\]
Cộng theo vế ba phương trình trong hệ, ta được
\[2\tron{a+b+c+\alpha+\beta+\gamma}=9,4.\]
Phương trình kể trên tương đương
\[a+b+c+\alpha+\beta+\gamma=4,7\tag{2}\]
Trừ (2) cho từng phương trình trong (1), ta nhận được hệ
\[\heva{a+\gamma&=1,2\\b+\alpha&=2,7\\c+\beta&=0,8}\]
Do $0\le \alpha,\beta,\gamma<1$ nên là
$$0,2<a\le 1,2,\quad 1,7<b\le 2,7,\quad -0,2<c\le 0,8.$$
Điều kiện phép đặt $a,b,c$ nguyên cho ta $a=1,b=2,c=0.$ Thể trở lại, ta tìm ra $\alpha=0,7,\beta=0,8,\gamma=0,2.$\\ Kết luân, hệ phương trình đã cho có nghiệm duy nhất là $(x;y;z)=(1,7;2,8;0,2).$
}
\end{gbtt}

\section{Phần nguyên và các bài toán đồng dư số mũ lớn}

\subsection*{Ví dụ minh họa}
\begin{bx}\
\label{phannguyen2} 
\begin{enumerate}[a,] 
    \item Với mỗi số nguyên dương $n,$ ta đặt \[S_{n}=\tron{5+2\sqrt{6}}^{n}+\tron{5-2\sqrt{6}}^{n}.\] Chứng minh $S_{n+4}$ và $S_{n}$ là các số nguyên có cùng chữ số tận cùng.
    \item Tìm chữ số hàng đơn vị của $\left[\tron{\sqrt{3}+\sqrt{2}}^{48}\right].$
    \item Tìm chữ số tận cùng của $\left[\tron{\sqrt{3}+\sqrt{2}}^{250}\right].$
\end{enumerate}    
\loigiai{
Trước hết, hết, tác giá xin phát biểu một bổ đề tương tự bổ đề đã được học ở \chu{chương IV}:
\begin{light}
\begin{quote}
 \it   Cho hai số nguyên dương $a$ và $b.$ Chứng minh rằng ứng với mỗi số tự nhiên $n,$ tồn tại các số nguyên $x_n$ và $y_n$ sao cho
\begin{align*}
    &\left(a+\sqrt{b}\right)^n=x_n+y_n\sqrt{b},
    \\&\left(a-\sqrt{b}\right)^n=x_n-y_n\sqrt{b}.
\end{align*}
\end{quote}
\end{light}
Ta định nghĩa $S_0=\tron{5+2\sqrt{6}}^{0}+\tron{5-2\sqrt{6}}^{0}=2.$
\begin{enumerate}[a,]
    \item Áp dụng bổ đề trên, ta thấy $S_n$ là số nguyên. Với mọi số nguyên dương $n$, ta có\[\begin{aligned}
     10S_{n+1}&=\vuong{\tron{5+2\sqrt{6}}^{n+1}+\tron{5-2\sqrt{6}}^{n+1}}\vuong{\tron{5+2\sqrt{6}}+\tron{5-2\sqrt{6}}}\\
     &=\tron{5+2\sqrt{6}}^{n+2}+\tron{5+2\sqrt{6}}^{n}+\tron{5-2\sqrt{6}}^{n+2}+\tron{5-2\sqrt{6}}^{n}\\
     &=S_{n+2}+S_{n}.
    \end{aligned}
    \]
      Suy ra $S_{n+2}=10S_{n+1}-S_n$ với mọi $n$. Khi đó
      \[\begin{aligned}
      S_{n+4}&=10S_{n+3}-S_{n+2}\\
      &=10S_{n+3}-\tron{10S_{n+1}-S_n}\\
      &=10S_{n+3}-10S_{n+1}+S_n\equiv S_n\pmod{10}.
      \end{aligned}
      \]Do vậy $S_{n+4}$ và $S_n$ có cùng chữ số tận cùng. Đây là điều phải chứng minh.
      \item Từ câu a ta suy ra $S_{4k+i}\equiv S_{i}\pmod{10}$ với mọi $k\in\mathbb{N}, i\in\{0,1,2,3\}.$\\
      Do $0<5-2\sqrt{6}<1$ nên $0<\tron{5-2\sqrt{6}}^n<1$ với mọi $n,$ và như vậy \[S_n-1<\tron{5+2\sqrt{6}}^n<S_n.\] 
      Nhận xét trên cho ta $\vuong{\tron{5+2\sqrt{6}}^n}=S_n-1.$ Theo đó $$\vuong{\tron{\sqrt{3}+\sqrt{2}}^{48}}=\vuong{\tron{5+2\sqrt{6}}^{24}}=S_{24}-1\equiv S_0-1\equiv 1\pmod {10}.$$ Vậy chữ số hàng đơn vị của số $\vuong{\tron{\sqrt{3}+\sqrt{2}}^{48}}$ là $1.$
      \item Ta có $\left[(\sqrt{3}+\sqrt{2})^{250}\right]=\vuong{\tron{5+2\sqrt{6}}^{125}}=S_{125}-1\equiv S_1-1\equiv 9\pmod{10}.$ \\Vậy chữ số tận cùng của $\left[\tron{\sqrt{3}+\sqrt{2}}^{250}\right]$ là $9.$
\end{enumerate}}
\end{bx}

\subsection*{Bài tập tự luyện}

\begin{btt}
Với mọi số nguyên dương $n,$ chứng minh rằng $\left[\tron{2+\sqrt{3}}^{n}\right]$ là một số lẻ.
\end{btt}

\begin{btt}
Tìm số dư của $x_{n}=\left[\tron{4+\sqrt{15}}^{n}\right]$ khi chia cho $8.$
\end{btt}

\begin{btt}
Chứng minh rằng $\left[\tron{\sqrt{3}+\sqrt{2}}^{2 n}\right]$ không chia hết cho $5$ với mọi số tự nhiên $n.$
\end{btt}

\begin{btt}
Chứng minh rằng trong biểu diễn
thập phân của số $\left(8+3 \sqrt{7}\right)^{7}$ có bảy chữ số $9$ liền sau dấu phẩy.
\end{btt}

\begin{btt}
Tìm số nguyên tố $p$ nhỏ nhất để $\left[\tron{3+\sqrt{p}}^{2 n}\right]+1$ chia hết cho $2^{n+1}$ với mọi $n$ tự nhiên.
\nguon{Tạp chí Toán học và Tuổi trẻ, tháng 2, 2005}
\end{btt}

\subsection*{Hướng dẫn tập tự luyện}

\begin{gbtt}
Với mọi số nguyên dương $n,$ chứng minh rằng $\left[\tron{2+\sqrt{3}}^{n}\right]$ là một số lẻ.
\loigiai{
Ta đặt $S_n=\tron{2+\sqrt{3}}^n+\tron{2-\sqrt{3}}^n.$ Khi đó $S_0=2,S_1=4.$ Với mọi số tự nhiên $n,$ ta có
\[
\begin{aligned}
4S_{n+1}&=\vuong{\tron{2+\sqrt{3}}^{n+1}+\tron{2-\sqrt{3}}^{n+1}}\vuong{\tron{2+\sqrt{3}}+\tron{2-\sqrt{3}}}\\
&=\tron{2+\sqrt{3}}^n+\tron{2-\sqrt{3}}^n+\tron{2+\sqrt{3}}^{n+1}+\tron{2-\sqrt{3}}^{n+1}\\
&=S_{n+2}+S_{n}.
\end{aligned}
\]
Nhận xét trên cho ta $S_{n+2}=4S_{n+1}-S_n$ với mọi $n$. Do $S_0,S_1$ là hai số chẵn nên ta dễ dàng suy ra được $S_n$ là số chẵn với mọi $n$. Ngoài ra, ta có $0<2-\sqrt{3}<1.$ Hoàn toàn tương tự ý b \chu{ví dụ \ref{phannguyen2}}, ta chỉ ra \[\left[\tron{2+\sqrt{3}}^{n}\right]=S_n-1\] với mọi $n$ nguyên dương. Lại do $S_n$ là số chẵn nên $\left[\tron{2+\sqrt{3}}^{n}\right]$ là số lẻ. Chứng minh hoàn tất.}
\end{gbtt}

\begin{gbtt}
Tìm số dư của $x_{n}=\left[\tron{4+\sqrt{15}}^{n}\right]$ khi chia cho $8.$
\loigiai{
Ta đặt $S_n=\tron{4+\sqrt{15}}^{n}+\tron{4-\sqrt{15}}^{n}.$ Chứng minh tương tự các bài toán trên, ta có $S_n$ là số nguyên và $S_{n+2}=8S_{n+1}-S_n$ với mọi $n,$ như thế thì
\[
\begin{aligned}
S_{n+4}=8S_{n+3}-S_{n+2}=8S_{n+3}-(8S_{n+1}-S_n)=8S_{n+3}-8S_{n+1}+S_n
\end{aligned}
\]Do đó $S_{n+4}\equiv S_n\pmod 8$ với mọi $n$, kéo theo
\[S_{4k+i}\equiv S_i\pmod 8\quad \text{ với mọi }k\in\mathbb{N}, i\in\{0,1,2,3\}.\]
Với việc $0<4+\sqrt{15}<1,$ ta dễ dàng suy ra $x_n=\left[\tron{4+\sqrt{15}}^{n}\right]=S_n-1.$ Bước tính toán cuối cùng của ta sẽ là xét theo $n$ số dư của nó khi chia cho $4.$ Với mọi số nguyên $k$ thì 
\begin{multicols}{2}
\begin{itemize}
    \item $x_{4k}\equiv S_0-1\equiv 1\pmod 8$
    \item $x_{4k+1}\equiv S_1-1\equiv 7\pmod 8$
    \item $x_{4k+2}\equiv S_2-1\equiv 5\pmod 8$
    \item $x_{4k+3}\equiv S_3-1\equiv 7\pmod 8$
\end{itemize}
\end{multicols}
Phép liệt kê trên cũng chính là kết luận bài toán.}
\end{gbtt}

\begin{gbtt}
Chứng minh rằng $\left[\tron{\sqrt{3}+\sqrt{2}}^{2 n}\right]$ không chia hết cho $5$ với mọi số tự nhiên $n.$
\loigiai{
Ta đặt $S_n=\tron{5+2\sqrt{6}}^{n}+\tron{5-2\sqrt{6}}^{n}.$
Theo như \chu{ví dụ \ref{phannguyen2}}, với mọi $n\in\mathbb{N}$ thì \[S_{n+2}=10S_{n+1}-S_n.\]Suy ra $S_{n+2}\equiv S_n\pmod {5}$ với mọi $n\in\mathbb{N}.$ Theo đó, với mọi số tự nhiên $k,$ ta có
\[\begin{cases}
     S_{2k+1}\equiv S_1\equiv 0\pmod 5.\\
     S_{2k}\equiv S_0\equiv 2\pmod 5.
\end{cases}\]
Cũng theo \chu{ví dụ \ref{phannguyen2}}, ta có  $\left[\tron{\sqrt{3}+\sqrt{2}}^{2 n}\right]=\vuong{\tron{5+2\sqrt{6}}^{n}}=S_n-1.$ Nhận xét này cho ta 
\begin{itemize}
    \item $\left[\tron{\sqrt{3}+\sqrt{2}}^{2 n}\right]\equiv -1\pmod 5$ với mọi $n$ lẻ,
    \item $\left[\tron{\sqrt{3}+\sqrt{2}}^{2 n}\right]\equiv 1\pmod 5$ với mọi $n$ chẵn.
\end{itemize}
Từ đây suy ra $\left[\tron{\sqrt{3}+\sqrt{2})^{2 n}}\right]$ không chia hết cho 5 với mọi số tự nhiên $n$.\\ Bài toán được chứng minh.}
\end{gbtt}

\begin{gbtt}
Chứng minh rằng trong biểu diễn
thập phân của số $\left(8+3 \sqrt{7}\right)^{7}$ có bảy chữ số $9$ liền sau dấu phẩy.
\loigiai{
Áp dụng bổ đề, ta suy ra $\left(8+3 \sqrt{7}\right)^{7}+\left(8-3 \sqrt{7}\right)^{7}$ là một số tự nhiên. Ta đặt $$a=\left(8+3 \sqrt{7}\right)^{7}+\left(8-3 \sqrt{7}\right)^{7},$$ trong đó $a$ là một số tự nhiên. Do $0<8-3 \sqrt{7}<0,1$ nên là 
$$\left(8+3 \sqrt{7}\right)^{7}=a-\left(8-3 \sqrt{7}\right)^{7}>a-(0,1)^7.$$ 
Bắt buộc, $\left(8+3 \sqrt{7}\right)^{7}$ có bảy chữ số $9$ liền sau dấu phẩy. Bài toán được chứng minh.}
\end{gbtt}

\begin{gbtt}
Tìm số nguyên tố $p$ nhỏ nhất để $\left[\tron{3+\sqrt{p}}^{2 n}\right]+1$ chia hết cho $2^{n+1}$ với mọi $n$ tự nhiên.
\nguon{Tạp chí Toán học và Tuổi trẻ, tháng 2, 2005}
\loigiai{
Kiểm tra trực tiếp, ta thấy $p=2,p=3$ không thỏa. Ta sẽ chứng minh $p=5$ thỏa mãn yêu cầu đề bài. Với mỗi số tự nhiên $n,$ ta đặt\[S_n=\left(14+6\sqrt{5}\right)^n+\left(14-6\sqrt{5}\right)^n.\] Phép đặt trên cho ta $S_0=2, S_1=28,$ và
$28S_{n+1}=S_{n+2}+16S_n$ nên $S_{n+2}=28S_{n+1}-16S_{n}.$ 
Bây giờ, ta đi chứng minh $S_n$ chia hết cho $2^{n+1}$ với mọi $n$ bằng quy nạp.
\begin{itemize}
    \item Với $n=0$, ta có $S_n=S_0=2$ chia hết cho $2^1.$
    \item Với $n=1$, ta có $S_n=S_1=28$ chia hết cho $2^2.$    
    \item Giả sử khẳng định trên đúng đến $n=k\ge 1.$ Khi đó 
    $$
    \heva{&2^{k+1}\mid S_k \\ &2^k\mid S_{k-1}}
    \Rightarrow
    \heva{&2^{k+2}\mid 28S_k \\ &2^{k+2}\mid 16S_{k-1}} \Rightarrow   
    2^{k+2}\mid \tron{28S_{k}-16S_{k-1}}=S_{k+1}.
    $$
    Vậy khẳng định cũng đúng với $n=k+1.$
\end{itemize}
Theo nguyên lí quy nạp, khẳng định được chứng minh. Ta có $0<14-6\sqrt{5}<1$, và khi chứng minh tương tự ý b \chu{ví dụ \ref{phannguyen2}}, ta thu được \[\left[\tron{3+\sqrt{p}}^{2 n}\right]+1=\left[\tron{14+6\sqrt{5}}^n\right]+1=S_n\] chia hết cho $2^{n+1}$ với mọi $n$ tự nhiên.\\
Vậy số nguyên tố nhỏ nhất thỏa mãn yêu cầu bài toán là $p=5.$
}
\end{gbtt}
 	%hàm phần nguyên, phần lẻ
\begin{thebibliography}{21}
    \addcontentsline{toc}{chapter}{Tài liệu tham khảo}
    \bibitem{} Các đề thi chuyên, đề thi chọn đội tuyển, đề thi học sinh giỏi các cấp ở Việt Nam.
    \item \credba{Phan Đức Chính, Tôn Thân, Vũ Hữu Bình, Trần Đình Châu, Ngô Hữu Dũng, Phạm Gia Đức, Nguyễn Duy Thuận}{Sách giáo khoa Toán 8}{NXB Giáo Dục Việt Nam}
    \item \credba{Phan Đức Chính, Tôn Thân, Vũ Hữu Bình, Trần Phương Dung, Ngô Hữu Dũng, Lê Văn Hồng, Nguyễn Như Thảo}{Sách giáo khoa Toán 9}{NXB Giáo Dục Việt Nam}
    \bibitem{} \credbon{Vũ Hữu Bình}{Nâng cao và phát triển toán 6 $-$ tập 1}{NXB Giáo Dục}{2008}   
    \bibitem{} \credbon{Vũ Hữu Bình}{Nâng cao và phát triển toán  8 $-$ tập 1}{NXB Giáo Dục}{2008} 
    \bibitem{} \credbon{Vũ Hữu Bình}{Phương trình nghiệm nguyên và kinh nghiệm giải}{NXB Giáo Dục}{2008} 
    \bibitem{} \credbon{Phan Huy Khải}{Chuyên đề 2 $-$ Số học và dãy số}{NXB Giáo Dục}{2006} \bibitem{} \credbon{Phan Huy Khải}{Chuyên đề 3 $-$ Các bài toán cơ bản của số học}{NXB Giáo Dục}{2006}  
    \bibitem{} \credbon{Phan Huy Khải}{Chuyên đề 4 $-$ Các bài toán về hàm số học}{NXB Giáo Dục}{2006}.
    \bibitem{} \credbon{Lê Anh Vinh (chủ biên), Hoàng Đỗ Kiên, Lê Phúc Lữ, Phạm Đức Hiệp}{Định hướng bồi dưỡng học sinh năng khiếu toán: Tập 3 $-$ Số học}{NXB Đại học Quốc gia Hà Nội}{2021} 
    \bibitem{} \credbon{Titu Andreescu and Dorin Andrica and Zuming Feng}{104 Number Theory Problems : From the Training of the USA IMO Team}{Birkhäuser}{2006}
    \bibitem{} \credbon{Adrian Andreescu and  Vinjai Vale}{111 Problems in Algebra and Number Theory}{TX, United States}{2016} 
    \bibitem{} \credba{Titu Andreescu and Dorin Andrica}{An introduction to Diophant Equations}{Birkhauser}
    \bibitem{} \credhai{Hayk Sedrakyan and Nairi Sedrakyan}{Number theory through exercises}
    \bibitem{} \credba{Hayk Sedrakyan and Nairi Sedrakyan}{The Stair $-$ Step Approach in Mathematics}{Problem Books in Mathematics, Springer}
    \bibitem{} \credba{Titu Andreescu and Razvan Gelca}{Mathematical Olympiad Challenges}{Birkhäuser}.
    \bibitem{} \credba{Titu Andreescu and Vlad Crisan}{Mathematical Induction $-$ A powerful and elegant method of proof}{XYZ Press}
    \bibitem{} \credba{Titu Andreescu, Gabriel Dospinescu and Oleg Mushkarov}{Number Theory, Concepts and Problems}{XYZ Press}.
    \bibitem{} \credba{Ellina Grigorieva}{Methods of Solving Number Theory Problems}{Birkhäuser}.
    \bibitem{} Art of Problem Solving, \url{https://artofproblemsolving.com/community}
    \bibitem{} Trang web Diễn đàn Toán học \href{diendantoanhoc.org}{diendantoanhoc.org}.    
    \bibitem{} Group "Hướng tới Olympic Toán Việt Nam": \url{fb.com/groups/243634219684049}.
    \bibitem{} Huy Cao's blog, \url{https://julielltv.wordpress.com/}
    \bibitem{} Tạp chí Pi.
    \bibitem{} Tạp chí Toán Tuổi thơ.  
    \bibitem{} Tạp chí Toán học và Tuổi trẻ.
    \bibitem{} Tạp chí Epsilon.    
    \bibitem{} Tạp chí Kvant, \url{http://kvant.mccme.ru/} 
    \bibitem{} Tạp chí Mathematical Reflections,\\
    \url{https://www.awesomemath.org/mathematical-reflections/}
\end{thebibliography}
Ngoài ra có một số bài toán và tài liệu chúng tôi tham khảo và sưu tầm trên Internet nhưng không xác minh được rõ tác giả của chúng, do vậy sẽ không trích dẫn ở đây. Mong tác giả của những bài toán và tài liệu này hết sức thông cảm.
 	%lời kết, tài liệu tham khảo
\input{0.Quy nạp} 


\end{document}