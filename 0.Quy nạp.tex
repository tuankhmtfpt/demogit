\chapter*{Phụ lục\\
Phương pháp quy nạp
}
\addcontentsline{toc}{chapter}{Phụ lục. Phương pháp quy nạp}
\section{Phương pháp quy nạp toán học}
Khi tính một vài tổng $n$ số nguyên dương lẻ đầu tiên, ta nhận thấy rằng 
\begin{align*}
	1&=1,\\
	4&=3+1,\\
	9&=1+3+5,\\
	16&=1+3+5+7,\\
	25&=1+3+5+7+9.
\end{align*}
Ta có thể dự đoán kết quả từ việc tính các kết quả trên là $n^{2}$. Tuy nhiên ta không thể kết luận luôn được mà cần phải chứng minh dự đoán là đúng nếu thực tế nó đúng.\\

Trong toán học, quy nạp là một kĩ thuật chứng minh cực kì quan trọng. Người ta dùng nó để chứng minh và kiểm định những giả định giống như ví dụ ở trên. Quy nạp toán học thường được sử dụng để chứng minh các kết quả về những đối tương rời rạc, ví dụ như các bất đẳng thức, đẳng thức; rộng hơn là các bài toán số học, tổ hợp, các bài toán lí thuyết đồ thị, cây; trong tin học là độ phức tạp tính toán hay là tính đúng đắn của các chương trình. Hãy nhớ rằng quy nạp không phải là một công cụ để phát triển một định lí, mà chỉ dùng để chứng minh một kết quả hay mệnh đề nào đó.\\

Có rất nhiều định lí phát biểu rằng $P(n)$ đúng với mọi số $n$ nguyên dương, trong đó $P(n)$ là một mệnh đề chứa biến, ví dụ mệnh đề $n\le 2^n$. Quy nạp là kĩ thuật để ta chứng minh các mệnh đề thuộc loại như vậy.
\subsection*{Lý thuyết}
\begin{dx}[Phương pháp quy nạp]
Quá trình chứng minh một mệnh đề nào đó bằng quy nạp toán học (ví dụ mệnh đề $P(n)$ đúng với $n\ge 0$ nguyên dương) là một quá trình gồm hai bước.
	\begin{itemize}
		\item \chu{Bước 1.} Chỉ ra mệnh đề $P(n_0)$ đúng.
		\item \chu{Bước 2.} Chỉ ra $P(k)$ đúng, thì $P(k+1)$ cũng đúng với mọi số tự nhiên $k\ge {{n}_{0}}$.
	\end{itemize}
	Khi đó mệnh đề $P(n)$ đúng với mọi số tự nhiên $n\ge {{n}_{0}}$. 
\end{dx}
Trong phương pháp này, $P(k)$ ta gọi là giả thiết quy nạp. Nếu diễn đạt như một quy tắc suy luận thì kĩ thuật chứng minh này được phát biểu rằng
$\left[ {P\left( 1 \right) \wedge \forall k\left( {P\left( k \right) \to P\left( {k + 1} \right)} \right)} \right] \to \forall nP\left( n \right)$.
Chú ý rằng trong khi chứng minh bằng quy nạp toán học, ta không giả sử $P(k)$ đúng với mọi $k$ nguyên dương, mà ta chỉ chứng tỏ rằng giả sử $P(k)$ là đúng thì khi đó $P(k+1)$ cũng đúng, như vậy quy nạp toán học không phải là cách suy lí luẩn quẩn.\\


\chu{Quay lại bài toán ban đầu.} Ta cần chứng minh tổng của $n$ số tự nhiên lẻ đầu tiên là $n^2$. Ta gọi $P(n)$ là mệnh đề "Tổng $n$ số lẻ đầu tiên là $n^2$". Đầu tiên ta cần thực hiện bước 1 là chỉ ra $P(1)$ đúng. Sau đó phải chứng minh bước quy nạp tức cần chứng tỏ $P(k+1)$ đúng nếu giả sử $P(k)$ đúng.
\begin{enumerate}[\color{tuancolor}\bf\sffamily Bước 1.]
	\item Ta dễ dàng thấy một điều hiển nhiên đúng rằng $P(1)=1^2.$
	\item Ta giả sử rằng $P(k)$ đúng với một số nguyên dương $k$ nào đó, tức là
	$$1+3+5+\cdots+(2k-1)=k^2.$$
	Ta cần chứng minh $P(k+1)$, là đúng, với giả thiết $P(k)$ đúng. \\
	Mệnh đề $P(k+1)$ cho ta thấy rằng 
	\[1+3+5+...+(2k-1)+(2k+1)=(k+1)^2.\]
	Vì giả thiết quy nạp là đúng nên ta suy ra
	\begin{align*}
		1+3+5+...+(2k-1)+(2k+1)&=[1+3+...+(2k-1)]+(2k+1)
		\\&=k^2+(2k+1)\\&=(k+1)^2.
	\end{align*}
	Điều này chứng tỏ $P(k+1)$ được suy ra từ $P(k)$.
\end{enumerate}
Như vậy theo nguyên lý quy nạp thì giả thiết quy nạp đúng. \\

Vừa rồi chúng tôi đã trình bày cho bạn đọc về ý tưởng của phương pháp quy nạp, sau đây là một số các bài toán nhằm giúp bạn đọc hiểu rõ cũng như thành thạo hơn trong việc sử dụng phương pháp này để giải một số bài tập cụ thể.
\subsection{Phương pháp quy nạp trong các bài toán chứng minh đẳng thức}
\begin{bx}
Chứng minh rằng với mọi số nguyên dương $k$ thì ta luôn có 
\[1 \cdot 2 \cdot 3+2 \cdot 3 \cdot 4+\ldots+k \cdot(k+1) \cdot(k+2)=\dfrac{k(k+1)(k+2)(k+3)}{4}.\tag{*}\]
\loigiai{
Như ý tưởng đã trình bày ở trên, ta sẽ tiến hành theo 2 bước sau.
\begin{enumerate}[\color{tuancolor}\bf\sffamily Bước 1.]
    \item Với $k=1$ thì ta có, 
    $VT = 1\cdot 2 \cdot3 = \dfrac{1\cdot 2 \cdot3 \cdot 4}{4}.$
    Như vậy (*) đúng với $k=1.$
    \item Ta giả sử rằng (*) đúng với $k = n>1.$ 
    \[1 \cdot 2 \cdot 3+2 \cdot 3 \cdot 4+\ldots+n \cdot(n+1) \cdot(n+2)=\dfrac{n(n+1)(n+2)(n+3)}{4} \tag{**}\]
    bây giờ nhiệm vụ là sẽ đi chứng minh (*) đúng với $k= n+1$, hay 
    \[1 \cdot 2 \cdot 3+2 \cdot 3 \cdot 4+\ldots+(n+1) \cdot(n+2) \cdot(n+3)=\dfrac{(n+1)(n+2)(n+3)(n+4)}{4}.\]
    Cộng vào hai vế của (**) với $(n+1) \cdot(n+2) \cdot(n+3)$, ta được
    \[
    \begin{aligned}
 1 \cdot 2 \cdot 3+\ldots+(n+1)(n+2)(n+3) 
&= \dfrac{n(n+1)(n+2)(n+3)}{4}+(n+1)(n+2)(n+3) \\
&=(n+1)(n+2)(n+3)\left(\dfrac{n}{4}+1\right) \\
&= \dfrac{(n+1)(n+2)(n+3)(n+4)}{4}.
\end{aligned}
\]
Vậy (*) cũng đúng với $k=n+1.$
\end{enumerate}
Theo nguyên lí quy nạp, ta chỉ ra điều phải chứng minh.}
\begin{luuy}
Hoàn toàn tương tự, bạn đọc cũng có thể dùng phương pháp quy nạp để chứng minh các đẳng thức
\[\begin{aligned}
&\sum_{i=1}^{n} i^{1}=\dfrac{n (n+1)}{2}, \\
&\sum_{i=1}^{n} i^{2}=\dfrac{n (n+1) (2 n+1)}{6}, \\
&\sum_{i=1}^{n} i^{3}=\dfrac{n^{2} (n+1)^{2}}{4}, \\
&\sum_{i=1}^{n} i^{4}=\dfrac{n (2 n+1) (n+1) \left(3 n^{2}+3 n-1\right)}{30}, \\
&\sum_{i=1}^{n} i^{5}=\dfrac{n^{2} \left(2 n^{2}+2 n-1\right) (n+1)^{2}}{12}, \\
&\sum_{i=1}^{n} i^{6}=\dfrac{n (2 n+1) (n+1) \left(3 n^{4}+6 n^{3}-3 n+1\right)}{42}, \\
&\sum_{i=1}^{n} i^{7}=\dfrac{n^{2} \left(3 n^{4}+6 n^{3}-n^{2}-4 n+2\right) (n+1)^{2}}{24}.
\end{aligned}\]
Trong đó, kí hiệu $\di\sum$ ở đây (đọc là \textit{sigma}). Hiểu đơn giản thì
\[\sum_{i=1}^{n} f(x) = f(1) + f(2) + f(3) + \cdots +f(n).\]
\end{luuy}
\end{bx}

\begin{bx}
Chứng minh rằng với mọi số nguyên dương $n$ ta luôn có 
\[\left(1-\dfrac{1}{k^{2}}\right)\left(1-\dfrac{1}{(k+1)^{2}}\right) \ldots\left(1-\dfrac{1}{(2 k-1)^{2}}\right)=1-\dfrac{1}{2 k-1}. \tag{*}\]
\loigiai{
Với $k=1,k=2,$ mệnh đề (*) đúng. Ta giả sử rằng (*) đúng với $k=n>2$, tức là
\[\left(1-\dfrac{1}{n^{2}}\right)\left(1-\dfrac{1}{(n+1)^{2}}\right) \ldots\left(1-\dfrac{1}{(2 n-1)^{2}}\right)=1-\dfrac{1}{2 n-1}. \tag{**}\]
Ta sẽ cần chứng minh (1) đúng với $k=n+1$, hay 
\[\left(1-\dfrac{1}{(n+1)^{2}}\right)\left(1-\dfrac{1}{(n+1)^{2}}\right) \ldots\left(1-\dfrac{1}{(2 n+1)^{2}}\right)=1-\dfrac{1}{2 n+1}.\]
Nhân cả hai vế của (**) với đại lượng 
\[\left(1-\dfrac{1}{(2 n)^{2}}\right)\left(1-\dfrac{1}{(2 n+1)^{2}}\right)\tron{\dfrac{n^2}{n^2-1}},\]
khi đó ta được
\[\begin{aligned}
\left(1-\dfrac{1}{(n+1)^{2}}\right) \ldots\left(1-\dfrac{1}{(2 n+1)^{2}}\right) 
&=\left(1-\dfrac{1}{2 n-1}\right)\left(1-\dfrac{1}{(2 n)^{2}}\right)\left(1-\dfrac{1}{(2 n+1)^{2}}\right)\tron{\dfrac{n^2}{n^2-1}} \\
&= \dfrac{2(n-1)}{2 n-1} \cdot \dfrac{(2 n-1)(2 n+1)}{4 n^{2}} \cdot \dfrac{4 n(n+1)}{(2 n+1)^{2}} \cdot \dfrac{n^{2}}{(n-1)(n+1)} \\
& =\dfrac{2 n}{2 n+1} \\
&= 1-\dfrac{1}{2 n+1}.
\end{aligned}\]
Theo nguyên lí quy nạp, ta có điều phải chứng minh.}
\end{bx}
\begin{bx}
Chứng minh rằng với mọi số $n$ nguyên dương thì ta luôn có 
\[\sum_{i=1}^{n}(-1)^{i} i^{2}=\dfrac{(-1)^{n} n(n+1)}{2}. \tag{*}\]
\loigiai{
Mệnh đề (*) đúng với $n=1.$ Ta giả sử rằng (*) đúng với $n= k >1.$ Ta có 
\[\begin{aligned}
\sum_{i=1}^{k+1}(-1)^{i} i^{2} &=\sum_{i=1}^{k}(-1)^{i} i^{2}+(-1)^{k+1}(k+1)^{2} \\
&=\dfrac{(-1)^{k} k(k+1)}{2}+(-1)^{k+1}(k+1)^{2} \\
&=\dfrac{(-1)^{k}(k+1)}{2}(k-2(k+1)) \\
&=\dfrac{(-1)^{k}(k+1)}{2}(-k-2) \\
&=\dfrac{(-1)^{k+1}(k+2)}{2}.
\end{aligned}\]
Như vậy, (1) cũng đúng với $n=k+1$. Theo nguyên lí quy nạp thì đẳng thức được chứng minh.
}
\end{bx}

\begin{bx}
Với $n$ là số nguyên dương, hãy tính giá trị của biểu thức 
\[A(n)=\dfrac{1}{1.2 .3}+\dfrac{1}{2.3 .4}+\dfrac{1}{3.4 .5}+\cdots+\dfrac{1}{n(n+1)(n+2)}.\]
\loigiai{
Trước tiên, ta có một số nhận xét sau
\begin{align*}
    A(1) &= \dfrac{1}{{1.2.3}} = \dfrac{{1(1 + 3)}}{{4(1 + 1)(1 + 2)}}\\
    A(2) &= \dfrac{1}{{1.2.3}} + \dfrac{1}{{2.3.4}} = \dfrac{{2(2 + 3)}}{{4(2 + 1)(2 + 2)}}\\
    A(3) &= \dfrac{1}{{1.2.3}} + \dfrac{1}{{2.3.4}} + \dfrac{1}{{3.4.5}} = \dfrac{{3(3 + 3)}}{{4(3 + 1)(3 + 2)}}.
\end{align*}
Như vậy, ta có thể dự đoán được rằng 
\[A(n)=\dfrac{n(n+3)}{4(n+1)(n+2)}. \tag{*}\]
Ta sẽ đi chứng minh điều này bằng quy nạp. Hiển nhiên với $n = 1$ thì mệnh đề trên đúng. \\ Ta giả sử rằng (*) đúng với $n=k >1$, tức là 
\[A(k)=\sum_{i=1}^{k} \dfrac{1}{i(i+1)(i+2)}=\dfrac{k(k+3)}{4(k+1)(k+2)}.\]
Cộng hai vế với $\dfrac{1}{(k+1)(k+2)(k+3)},$ ta được
\[\begin{aligned}
A(k+1) &=\sum_{i=1}^{k+1} \dfrac{1}{i(i+1)(i+2)} \\
&=\sum_{i=1}^{k} \dfrac{1}{i(i+1)(i+2)}+\dfrac{1}{(k+1)(k+2)(k+3)} \\
&=\dfrac{k(k+3)}{4(k+1)(k+2)}+\dfrac{1}{(k+1)(k+2)(k+3)} \\
&=\dfrac{k(k+3)^{2}+4}{4(k+1)(k+2)(k+3)}\\&=\dfrac{k^{3}+6 k^{2}+9 k+4}{4(k+1)(k+2)(k+3)} \\
&=\dfrac{(k+1)^{2}(k+4)}{4(k+1)(k+2)(k+3)}\\&=\dfrac{(k+1)(k+4)}{4(k+2)(k+3)} .
\end{aligned}\]
Như vậy, (*) cũng đúng với $n=k+1$ nên theo nguyên lí quy nạp, ta có điều phải chứng minh.}
\begin{luuy}
Hoàn toàn tương tự, bạn đọc có thể tính giá trị của các biểu thức
\begin{align*}
    B(n)&=\dfrac{3}{{1 \times 2 \times 2}} + \dfrac{4}{{2 \times 3 \times {2^2}}} + \dfrac{5}{{3 \times 4 \times {2^3}}} +  \ldots \dfrac{{(n + 2)}}{{n \times (n + 1) \times {2^n}}},\\
    C(n)&=\dfrac{5}{{1 \times 2 \times 3}} + \dfrac{6}{{2 \times 3 \times 4}} + \dfrac{7}{{3 \times 4 \times 5}} +  \ldots  + \dfrac{{n + 4}}{{n(n + 1)(n + 2)}}.
\end{align*}
\end{luuy}
\end{bx}
\begin{bx}
Chứng minh rằng với mọi số nguyên $k \ge 1$ ta luôn có 
\[\left[\dfrac{1}{2}\right]+\left[\dfrac{2}{2}\right]+\ldots+\left[\dfrac{k}{2}\right]=\left[\dfrac{k}{2}\right] \cdot\left[\dfrac{k+1}{2}\right]. \tag{*}\]
Trong đó $[x]$ được kí hiệu là hàm phần nguyên của $x$, được định nghĩa như sau
\[[x] = z \Leftrightarrow \left\{ \begin{gathered}
  z \leqslant x < z + 1; \hfill \\
  z \in \mathbb{Z} \hfill \\ 
\end{gathered}  \right. \Leftrightarrow \left\{ \begin{gathered}
  0 \leqslant x - z < 1 \hfill \\
  z \in \mathbb{Z}. \hfill \\ 
\end{gathered}  \right.\]
Chẳng hạn $[6,9] = 6;\, [9,12] = 9; \,[21,2] = 21$.
\loigiai{
Với $k=1$ thì ta có $\di \left[\dfrac{1}{2}\right]=\left[\dfrac{1}{2}\right] \cdot\left[\dfrac{2}{2}\right]=0$. Với $k=2$ thì ta có $\di \left[\dfrac{1}{2}\right]+\left[\dfrac{2}{2}\right]=1 \text { và  }\left[\dfrac{2}{2}\right] \cdot\left[\dfrac{3}{2}\right]=1$.\\
Như vậy (*) đúng với $n=1,2.$ Ta giả sử rằng (*) đúng với $k=n >2$, hay 
\[\left[\dfrac{1}{2}\right]+\left[\dfrac{2}{2}\right]+\ldots+\left[\dfrac{n}{2}\right]=\left[\dfrac{n}{2}\right] \cdot\left[\dfrac{n+1}{2}\right].\]
Cộng $\di \left[\dfrac{n+1}{2}\right]+\left[\dfrac{n+2}{2}\right]$ vào cả hai vế của đẳng thức trên, ta thu được
\[\begin{aligned}
\left[\dfrac{1}{2}\right]+\left[\dfrac{2}{2}\right]+\ldots+\left[\dfrac{n+2}{2}\right] &=\left[\dfrac{n}{2}\right] \cdot\left[\dfrac{n+1}{2}\right]+\left[\dfrac{n+1}{2}\right]+\left[\dfrac{n+2}{2}\right] \\ &=\left[\dfrac{n}{2}\right] \cdot\left[\dfrac{n+1}{2}\right]+\left[\dfrac{n}{2}\right]+\left[\dfrac{n+1}{2}\right]+1 \\
&=\left(\left[\dfrac{n}{2}\right]+1\right)\left(\left[\dfrac{n+1}{2}\right]+1\right) \\
&=\left[\dfrac{n+2}{2}\right] \cdot\left[\dfrac{n+3}{2}\right] .
\end{aligned}\]
Như vậy, (*) cũng đúng với $k= n+1$, theo nguyên lý quy nạp thì ta có điều phải chứng minh.
}
\end{bx}
\subsubsection*{Bài tập tự luyện}
\begin{bai}
Cho $a$ là một số thực khác $1.$ Chứng minh rằng
$$\di \sum_{j=0}^{n} a^{j}=\dfrac{a^{n+1}-1}{a-1}.$$
\end{bai}
\begin{bai}
Chứng minh rằng với mọi số nguyên dương $n$, ta luôn có $$\di \sum_{i=1}^{n} \dfrac{1}{i(i+1)}=\dfrac{n}{n+1}.$$
\end{bai}
\begin{bai}
Chứng minh rằng với mọi số nguyên dương $n$, ta luôn có $$\di \sum_{k=1}^{n} k \cdot k !=(n+1) !-1.$$
\end{bai}
\begin{bai}
Chứng minh rằng với mọi số nguyên dương $n$, ta luôn có 
\[\sum_{r=1}^{n} r\left(\dfrac{1}{2}\right)^{r}=2-(n+2)\left(\dfrac{1}{2}\right)^{n}.\]
\end{bai}
\begin{bai}
Chứng minh rằng với mọi số nguyên dương $n$, ta luôn có 
\[\sum_{r=1}^{n} r 2^{r}=(n-1)(2)^{n+1}+2.\]
\end{bai}
\begin{bai}
Chứng minh rằng với mọi số nguyên dương $n$, ta luôn có 
\[2+5+8+\cdots+3n-1=\dfrac{n(3n+1)}{2}.\]
\end{bai}
\begin{bai}
Chứng minh rằng với mọi số nguyên dương $n$, ta luôn có 
\[\dfrac{1}{3}+\dfrac{2}{9}+\dfrac{3}{27}+\cdots+\dfrac{n}{3^n}=\dfrac{3}{4}-\dfrac{2n+3}{4.3^n}.\]
\end{bai}
\begin{bai}
Chứng minh rằng với mọi số nguyên dương $n$, ta luôn có 
\[\dfrac{1}{3}+\dfrac{2}{9}+\dfrac{3}{27}+\cdots+\dfrac{n}{3^n}=\dfrac{3}{4}-\dfrac{2n+3}{4.3^n}.\]
\end{bai}
\begin{bai}
Chứng minh rằng với mọi số nguyên dương $n$, ta luôn có 
\[\left( {\dfrac{1}{2}} \right)\left( {\dfrac{4}{3}} \right)\left( {\dfrac{9}{4}} \right) \cdots \left( {\dfrac{{{n^2}}}{{n + 1}}} \right) = \dfrac{{n!}}{{n + 1}}.\]
\end{bai}
\begin{bai}
Chứng minh rằng với mọi số thực $x\ne 1$, ta luôn có
\[1+2 x+3 x^{2}+4 x^{3}+\ldots+n x^{n-1}=\dfrac{1-(n+1) x^{n}+n x^{n+1}}{(1-x)^{2}}.\]
\end{bai}
\begin{bai}
Chứng minh rằng với mọi số nguyên dương $n$, ta luôn có 
\[1-\dfrac{1}{2}+\dfrac{1}{3}-\dfrac{1}{4}+\ldots+\dfrac{1}{2 n-1}-\dfrac{1}{2 n}=\dfrac{1}{n+1}+\dfrac{1}{n+2}+\ldots+\dfrac{1}{2 n}.\]
\end{bai}
\begin{bai}
Chứng minh rằng với mọi số nguyên dương $n$, ta luôn có 
\[(n+1) !=1+\dfrac{1 !^{2}}{0 !}+\dfrac{2 !^{2}}{1 !}+\ldots+\dfrac{n !^{2}}{(n-1) !}.\]
\end{bai}
\begin{bai}
Chứng minh rằng với mọi số nguyên dương $n$, ta luôn có 
\[\dfrac{1}{4\cdot1^{2}-1}+\dfrac{1}{4\cdot2^{2}-1}+\dfrac{1}{4\cdot3^{2}-1}+\ldots+\dfrac{1}{4 n^{2}-1}=\dfrac{n}{2 n+1}.\]
\end{bai}
\begin{bai}
Chứng minh rằng với mọi số nguyên dương $n$, ta luôn có 
\begin{enumerate}[a,]
    \item $1^{2}+4^{2}+7^{2}+\cdots+(3 n-2)^{2}=\dfrac{1}{2} n\left(6 n^{2}-3 n-1\right).$
    \item $2^{2}+5^{2}+8^{2}+\cdots+(3 n-1)^{2}=\dfrac{1}{2} n\left(6 n^{2}+3 n-1\right).$
\end{enumerate}
\end{bai}
\begin{bai}
Chứng minh rằng với mọi số nguyên dương $n$, ta luôn có 
\begin{enumerate}[a,]
    \item $\di \sum_{r=1}^{n} r(r+1)=\dfrac{1}{3} n(n+1)(n+2)$.
    \item $\di \sum_{r=1}^{n} r(r+1)(r+2)=\dfrac{1}{4} n(n+1)(n+2)(n+3).$
    \item $\di \sum_{r=1}^{n} r(r+1)(r+2) \ldots(r+p-1)=\dfrac{1}{p+1} n(n+1)(n+2) \ldots(n+p)$.
\end{enumerate}
\end{bai}

\subsection{Phương pháp quy nạp trong các bài toán chứng minh bất đẳng thức}
\begin{bx}
Chứng minh rằng với mọi số nguyên $n\ge 5,$ ta luôn có $2^n>n^2+1.$
\loigiai{
Với $n=5,$ ta có $2^n=32>26=n^2+1.$ Giả sử $$2^n>n^2+1,\text{ với }n=5,6,\ldots,k.$$
Ta sẽ chứng minh bài toán đúng với $n=k+1.$ Theo như giả thiết quy nạp, ta có
$$2^{k+1}=2\cdot2^k>2\tron{k^2+1}=2k^2+2=(k+1)^2+1+k(k-2)>(k+1)^2+1.$$
Bất đẳng thức cũng đúng với $n=k+1.$ Theo nguyên lí quy nạp, bài toán được chứng minh.}
\begin{luuy}
Bản chất của bài toán trên là kết quả 
\begin{quote}
    \it Giá trị hàm mũ lớn hơn $1$ luôn lớn hơn giá trị hàm đa thức với biến số đủ lớn.
\end{quote}
\end{luuy}
\end{bx}
\begin{bx}[Bất đẳng thức Bernoulli]
	Xét số thực $x \geq -1$. Chứng minh rằng với mọi số nguyên dương $n$ ta có \[(1+x)^n \geq 1+nx .\tag{1}\]
	\loigiai{
	Với $n=1$ thì (1) trở thành $1+x \geq 1+x$, đúng. Giả sử (1) đúng với $n=k\geq 1$, tức là $1+x^k \geq 1+kx$. Ta chứng minh (1) đúng với $n=k+1$, tức là $(1+x)^{k+1} \geq 1+(k+1)x.$ Thật vậy, $$(1+x)^{k+1}=(1+x)^k(1+x) \geq (1+kx)(1+x)=1+(k+1)x+kx^2 \geq 1+(k+1)x.$$
	Vậy, theo nguyên lí quy nạp thì bất đẳng thức được chứng minh.
}
\end{bx}
\begin{bx}
Chứng minh rằng với mọi số nguyên $n\ge 1$ ta luôn có
\[\dfrac{1}{1^{2}}+\dfrac{1}{2^{2}}+\dfrac{1}{3^{2}}+\cdots+\dfrac{1}{n^{2}}<2. \tag{1}\]
\loigiai{
\chu{Cách 1.}
Hiển nhiên với $n = 2$ và $n=1$ thì mệnh đề (1) đúng, giả sử mệnh đúng với $n=k >2$, hay 
\[\sum_{k=1}^{n} \dfrac{1}{k^{2}}<2-\dfrac{1}{n}.\]
Cộng $\dfrac{1}{(n+1)^2}$ vào cả hai vế của giả thiết quy nạp, ta được
\[\sum_{k=1}^{n+1} \dfrac{1}{k^{2}}<2-\dfrac{1}{n}+\dfrac{1}{(n+1)^{2}}=2-\dfrac{(n+1)^{2}-n}{n(n+1)^{2}}.\]
Chú ý rằng 
\[2-\dfrac{(n+1)^{2}-n}{n(n+1)^{2}}=2-\dfrac{n^{2}+n+1}{n(n+1)^{2}}<2-\dfrac{n(n+1)}{n(n+1)^{2}}=2-\dfrac{1}{n+1}.\]
Suy ra 
\[\sum_{k=1}^{n+1} \dfrac{1}{k^{2}}<2-\dfrac{1}{n+1} <2.\]
Như vậy, mệnh đề (1) đúng với $n= k+1$ nên theo nguyên lí quy nạp ta có điều cần chứng minh.\\
\chu{Cách 2.}
Ta có nhận xét rằng, với mọi $k>0$ ta luôn có 
\[\dfrac{1}{k}-\dfrac{1}{k+1}=\dfrac{k+1-k}{k(k+1)}=\dfrac{1}{k(k+1)}.\]
Do đó
\[\begin{aligned}
\dfrac{1}{1.2}+\dfrac{1}{2.3}+\cdots+\dfrac{1}{(n-1) n} =\dfrac{1}{1}-\dfrac{1}{2}+\dfrac{1}{2}-\dfrac{1}{3}+\cdots+\dfrac{1}{n-1}-\dfrac{1}{n} 
=1-\dfrac{1}{n} 
=\dfrac{n-1}{n}.
\end{aligned}\]
Bây giờ ta xét với $k \ge 2$ thì ta luôn có $\di\dfrac{1}{k^{2}}<\dfrac{1}{(k-1) k}.$ Do vậy 
\[\begin{aligned}
\dfrac{1}{1^{2}}+\dfrac{1}{2^{2}}+\dfrac{1}{3^{2}}+\cdots+\dfrac{1}{n^{2}} <1+\dfrac{1}{1.2}+\dfrac{1}{2.3}+\cdots+\dfrac{1}{(n-1) n} 
=1+\dfrac{n-1}{n} 
<2.
\end{aligned}\]
Bài toán được giải quyết.}
\end{bx}
\begin{bx}[Bất đẳng thức đối với các số điều hòa]
Các số điều hòa \index{các số điều hòa} $H_j,j=1,2,...$ được định nghĩa như sau $\di H_j=1+\dfrac{1}{2}+\dfrac{1}{3}+\cdots+\dfrac{1}{j}.$ Dùng quy nạp toán học chứng minh rằng $H_{2^n}\ge 1+\dfrac{1}{2}$.
\loigiai{
Với $n=0$ ta có $H_{2^0}=1\ge 1+\dfrac{0}{2}$, như vậy mệnh đề đúng với $n=0$. \\
Giả sử mệnh đề đúng với $k>0$, khi đó ta có
\[\begin{aligned}
	H_{2^{k+1}} &=1+\dfrac{1}{2}+\dfrac{1}{3}+\ldots+\dfrac{1}{2^{k}}+\dfrac{1}{2^{k}+1}+\ldots+\dfrac{1}{2^{k+1}} \\
	&=H_{2^{k}}+\dfrac{1}{2^{k}+1}+\ldots+\dfrac{1}{2^{k+1}}\\
	&\geq\left(1+\dfrac{k}{2}\right)+\dfrac{1}{2^{k}+1}+\ldots+\dfrac{1}{2^{k+1}}\\&
	\geq\left(1+\dfrac{k}{2}\right)+\dfrac{1}{2} \\&
	=1+\dfrac{k+1}{2}.
\end{aligned}\]
Như vậy theo nguyên lý quy nạp, ta có điều phải chứng minh.
}
\end{bx}

\begin{bx}
Xét các số $a, b$ không âm. Chứng minh rằng với mọi số nguyên dương $n$ ta đều có 
\[\dfrac{a^n+b^n}{2} \geq \left(\dfrac{a+b}{2}\right)^n. \tag{1}\]
\loigiai{
    Với $n=1$ thì (1) trở thành $\dfrac{a+b}{2} \geq \dfrac{a+b}{2}$, hiển nhiên mệnh đề này đúng. \\
    Giả sử (1) đúng với $n=k  \geq 1$, tức là $\dfrac{a^k+b^k}{2} \geq \left(\dfrac{a+b}{2}\right)^k$.\\
	Ta cần chứng minh (1) đúng với $n=k+1$, hay 
	\[\dfrac{a^{k+1}+b^{k+1}}{2} \geq \left(\dfrac{a+b}{2}\right)^{k+1}\cdot\tag{2}\]
	Ta có $$\left(\dfrac{a+b}{2}\right)^{k+1} =\dfrac{a+b}{2}\cdot\left(\dfrac{a+b}{2}\right)^k \leq\dfrac{a+b}{2}\cdot\dfrac{a^k+b^k}{2}$$
	Mặt khác thì 
	$$2\tron{a^{k+1}+b^{k+1}}-(a+b)\tron{a^k+b^k}=(a-b)\tron{a^k-b^k} \geq 0$$
	nên $\dfrac{a^{k+1}+b^{k+1}}{2} \geq \dfrac{a+b}{2}\cdot\dfrac{a^k+b^k}{2}$.\\
	Như vậy, (1) đúng với $n=k+1$ nên theo nguyên lí quy nạp ta có điều phải chứng minh.
}
\end{bx}
\begin{bx}[Bổ đề Titu] Với $a_{1}, a_{2}, \ldots, a_{n} \in \mathbb{R}$ và  $b_{1}, b_{2}, \ldots, b_{n}$ là các số thực dương, chứng minh rằng
\[\dfrac{a_{1}^{2}}{b_{1}}+\dfrac{a_{2}^{2}}{b_{2}}+\ldots+\dfrac{a_{n}^{2}}{b_{n}} \geq \dfrac{\left(a_{1}+a_{2}+\ldots+a_{n}\right)^{2}}{b_{1}+b_{2}+\ldots+b_{n}}.\]
\loigiai{
Xét $n=1$, đây là một trường hợp tầm thường. Để ý rằng, nếu như ta chứng minh được với trường hợp $n=2$ thì bất đẳng thức ban đầu sẽ được chứng minh, bởi vì
\[\dfrac{{a_1^2}}{{{b_1}}} + \dfrac{{a_2^2}}{{{b_2}}} +  \ldots  + \dfrac{{a_n^2}}{{{b_n}}} \geqslant \underbrace {\dfrac{{{{\left( {{a_1} + {a_2} +  \ldots  + {a_{n - 1}}} \right)}^2}}}{{{b_1} + {b_2} +  \ldots  + {b_{n - 1}}}}}_{\text{ giả thiết quy nạp đúng với } (n-1) } + \dfrac{{a_n^2}}{{{b_n}}} \geqslant \dfrac{{{{\left( {{a_1} + {a_2} +  \ldots  + {a_n}} \right)}^2}}}{{{b_1} + {b_2} +  \ldots  + {b_n}}}\]
Bây giờ ta sẽ đi chứng minh với $n=2$, hay là
$$
\dfrac{a_{1}^{2}}{b_{1}}+\dfrac{a_{2}^{2}}{b_{2}} \geq \dfrac{\left(a_{1}+a_{2}\right)^{2}}{b_{1}+b_{2}}.
$$
Nhân cả hai vế bất đẳng thức trên với $b_{1} b_{2}\left(b_{1}+b_{2}\right)$, ta được
$$
\left(a_{1}^{2} b_{2}+a_{2}^{2} b_{1}\right)\left(b_{1}+b_{2}\right) \geq\left(a_{1}+a_{2}\right)^{2} b_{1} b_{2}.
$$
Khai triển ở cả hai vế, suy ra
$$
\left(a_{1}^{2}+a_{2}^{2}\right) b_{1} b_{2}+a_{1}^{2} b_{2}^{2}+a_{2}^{2} b_{1}^{2} \geq\left(a_{1}^{2}+a_{2}^{2}\right) b_{1} b_{2}+2 a_{1} b_{1} a_{2} b_{2}.
$$
Điều này là đúng bởi vì $a_{1}^{2} b_{2}^{2}+a_{2}^{2} b_{1}^{2} \geq 2 a_{1} b_{1} a_{2} b_{2}$. Như vậy bất đẳng thức được chứng minh.
}
\end{bx}
\begin{bx}
Cho $a_1, a_2,\dots,a_n$ thuộc khoảng $(0; 1)$ và $n\in \mathbb{Z}, n\ge 2$. Chứng minh rằng
	\begin{equation}(1 - a_1 )(1 - a_2 )\dots(1 - a_n ) > 1 - a_1  - a_2  - \cdots - a_n.\tag{1}\end{equation}
	\loigiai{
	Khi $n=2$, bất đẳng thức (1) trở thành
	$$(1 - a_1 )(1 - a_2 ) > 1 - a_1  - a_2  \Leftrightarrow 1 - a_1  - a_2  + a_1 a_2  > 1 - a_1  - a_2  \Leftrightarrow a_1 a_2  > 0\,\text{(đúng)}.$$
	Giả sử (1) đúng khi $n=k$ $(k=2, 3,\dots)$, tức là
	\begin{equation}(1 - a_1 )(1 - a_2 )\dots(1 - a_k ) > 1 - a_1  - a_2  - \cdots - a_k.\tag{2}\end{equation}
	Khi đó
	$$\begin{aligned}
		(1 - a_1 )(1 - a_2 )\dots(1 - a_k )(1 - a_{k + 1} )
		>& (1 - a_1  - a_2  - \cdots  - a_k )(1 - a_{k + 1} ) \\ 
		>& 1 - a_1  - a_2  - \cdots  - a_k  - a_{k + 1}  + a_{k + 1} (a_1  + a_2  + \cdots  + a_k ) \\ 
		>& 1 - a_1  - a_2  - \cdots  - a_k  - a_{k + 1}.
	\end{aligned}$$ 
	Vậy (1) đúng khi $n=k+1$, do đó (1) đúng với mọi $n\in \mathbb{Z}, n\ge 2$. 
	}
\end{bx}
\subsubsection*{Bài tập tự luyện}
\setcounter{bai}{0}
\begin{bai}
Chứng minh rằng với mọi số nguyên dương $n$ ta đều có $\dfrac{1}{2}\cdot\dfrac{3}{4}\cdot\dfrac{5}{6}\cdots\dfrac{2n-1}{2n} \leq \dfrac{1}{\sqrt{3n+1}}.$
\end{bai}
\begin{bai}
	Cho các số $a_1, a_2, \ldots,a_n$ không âm thỏa mãn $a_1+a_2+\cdots+a_n \leq \dfrac{1}{2}$. Chứng minh rằng với mọi số nguyên dương $n$ ta có $(1-a_1)(1-a_2)\cdots(1-a_n) \geq \dfrac{1}{2}.$
\end{bai}
\begin{bai}
Cho $x$ là số thực dương bất kì, $n$ là số nguyên dương, chứng minh rằng
	$$\di\dfrac{{{x}^{n}}\left( {{x}^{n+1}}+1 \right)}{{{x}^{n}}+1}\le {{\left( \dfrac{x+1}{2} \right)}^{2n+1}}.$$
\end{bai}
\begin{bai}
Cho các số thực ${{a}_{1}},{{a}_{2}},...,{{a}_{n}}\ge 1$. Chứng minh rằng 
$$\di\dfrac{1}{1+{{a}_{1}}}+\dfrac{1}{1+{{a}_{2}}}+\cdot \cdot \cdot +\dfrac{1}{1+{{a}_{n}}}\ge \dfrac{n}{1+\sqrt[n]{{{a}_{1}}.{{a}_{2}}\cdots{{a}_{n}}}}.$$
\end{bai}
\begin{bai}
Chứng minh rằng với mọi số nguyên dương $n>1$ ta có $\dfrac{1}{n+1}+\dfrac{1}{n+2}+\ldots+\dfrac{1}{2n}>\dfrac{13}{24}.$
\end{bai}
\begin{bai}
Chứng minh rằng với mọi số nguyên $n>1$ ta đều có $\dfrac{1}{\sqrt{1}}+\dfrac{1}{\sqrt{2}}+\ldots+\dfrac{1}{\sqrt{n}}>\sqrt{n}.$
\end{bai}
\begin{bai}
Chứng minh rằng với mọi số nguyên $n\ge 2$, ta có
	$\di 1 + \dfrac{1}{{{2^2}}} + \dfrac{1}{{{3^2}}} + \cdots + \dfrac{1}{{{n^2}}} > \dfrac{{3n}}{{2n + 1}}.$
\end{bai}
\begin{bai}
	Cho các số thực dương $a_1,{a_2},...,{a_n}$ thỏa mãn $a_1\cdot a_2\cdots a_n=x$. Chứng minh $$a_1^3+a_2^3+a_3^3+\cdots+a_n^3\le{x^3}+n-1.$$
\end{bai}
\begin{bai}
Chứng minh bất đẳng thức sau đúng với mọi $n\in\mathbb{N}$ ta có 
	$1+\dfrac{1}{2^2}+\dfrac{1}{3^2}+...+\dfrac{1}{n^2}<\dfrac{79}{48}$.
\end{bai}
\begin{bai}
Chứng minh rằng với mọi $n\ge 1,n\in\mathbb{N}$ , ta có
	$\dfrac{1}{n+1}+\dfrac{1}{n+2}+...+\dfrac{1}{2n}<\dfrac{7}{10}$.
\end{bai}
\begin{bai}
Chứng minh rằng với mọi số tự nhiên $n\ge 2$ ta có $n^n>\left(n+1\right)^{n-1}$.
\end{bai}
\begin{bai}
Cho $ n$ số thực dương $x_1,x_2,...,x_n$ có tích bằng 1. Chứng minh rằng $x_1+x_2+\cdots+x_n\ge n$.
\end{bai}
\begin{bai}
Cho các số $a_1,a_2,...,a_n \ge 0$ thỏa mãn $a_1a_2...a_n=1$ và $k$ là một hằng số dương tùy ý sao cho $k \ge n-1$. Chứng minh rằng $\di\sum\limits_{i = 1}^n {\dfrac{1}{{k + {a_i}}}}  \leqslant \dfrac{n}{{k + 1}}.$
\end{bai}
\begin{bai}
Cho $x_1, x_2, ..., x_n$ là $n$ số
	không âm $n\in \mathbb{Z}, n\ge 4$ và tổng của chúng bằng 1. Chứng minh rằng $${x_1}{x_2} + {x_2}{x_3} + \cdots + {x_n}{x_1} \leqslant \dfrac{1}{4}$$
\end{bai}
\begin{bai}
Chứng minh rằng với mọi số nguyên dương $n, a$ là số thực không âm, thì 
	$$\sqrt {a + 1 + \sqrt {a + 2 + \cdots + \sqrt {a + n} } }  < a + 3.$$
\end{bai}
\begin{bai}
Chứng minh rằng $\displaystyle{\dfrac{1 ! 2 !+2 ! 3 !+\ldots+n !(n+1) !}{n \sqrt[n]{(1 !)^{2} \ldots(n !)^{2}}} \geqslant 2 \sqrt[2 n]{n !}}$.
\end{bai}

\begin{bai}
Với $n \geq 3$ là số nguyên dương, $x_{1}, x_{2}, \ldots, x_{n}$ là các số thực dương. Chứng minh rằng
$$
\dfrac{x_{1}}{x_{1}+x_{2}}+\dfrac{x_{2}}{x_{2}+x_{3}}+\ldots+\dfrac{x_{n}}{x_{n}+x_{1}}<n-1.
$$
\end{bai}

\begin{bai}[Bất đẳng thức Cauchy $-$ Schwarz]
Cho $a_{1}, \ldots, a_{n}, b_{1}, \ldots, b_{n}$ là các số thực. Chứng minh rằng
$$
\left(a_{1}^{2}+\ldots+a_{n}^{2}\right)\left(b_{1}^{2}+\ldots+b_{n}^{2}\right) \geq\left(a_{1} b_{1}+\ldots+a_{n} b_{n}\right)^{2}.
$$
\end{bai}
\begin{bai}[Bất đẳng thức AM $-$ GM]
Cho $x_{1}, x_{2}, \ldots x_{n}$ là các số thực dương. Chứng minh rằng
$$
\dfrac{x_{1}+\ldots+x_{n}}{n} \geq \sqrt[n]{x_{1} x_{2} \ldots x_{n}}.
$$
\end{bai}
%à từ từ
%cứ để đó đi
%để anh bảo anh Tuấn trước 1 câu
\subsection{Phương pháp quy nạp trong các bài toán số học}
\begin{bx}
Chứng minh rằng với mọi số nguyên $n$ không âm, ta luôn có
\begin{multicols}{2}
\begin{enumerate}[a,]
    \item $n^3+2n$ chia hết cho $3.$
    \item $n^5-n$ chia hết cho $5.$
    \item $n^3-n$ chia hết cho $6.$
    \item $n^2-1$ chia hết cho $8,$ trong đó $n$ lẻ.
\end{enumerate}
\end{multicols}
\loigiai{
\begin{enumerate}[a,]
    \item  Với $n=1,$ ta có $n^3+2n=3$ chia hết cho $3.$
    \\Giả sử khẳng định đã cho đúng với $n=k>1.$ Ta sẽ chứng minh $3\mid (k+1)^3+2(k+1).$ Thật vậy
    $$(k+1)^3+2(k+1)=\tron{k^3+2k}+3\tron{k^2+k+1}.$$
    Vì $k^3+2k$ và $3\tron{k^2+k+1}$ chia hết cho $3$ nên $(k+1)^3+2(k+1)$ cũng chia hết cho $3.$  \\
    Theo nguyên lí quy nạp, khẳng định được chứng minh.
    \item  Với $n=1,$ ta có $n^5-n=0$ chia hết cho $5.$
    \\Giả sử khẳng định đã cho đúng với $n=k>1.$ Ta sẽ chứng minh $5\mid \tron{k+1}^5-\tron{k+1}.$ Thật vậy
    $$(k+1)^5-(k+1)=\tron{k^5-k}+5\tron{k^4+2k^3+2k^2+k}.$$
   Vì $\tron{k^5-k}$ và $5\tron{k^4+2k^3+2k^2+k}$ chia hết cho $5$ nên $(k+1)^5-(k+1)$ cũng chia hết cho $5.$  \\
    Theo nguyên lí quy nạp, khẳng định được chứng minh.
     \item  Với $n=1,$ ta có $n^3-n=0$ chia hết cho $6.$
    \\Giả sử khẳng định đã cho đúng với $n=k>1.$ Ta sẽ chứng minh $6\mid \tron{k+1}^3-\tron{k+1}.$ Thật vậy
    $$(k+1)^3-(k+1)=\tron{k^3-k}+3\tron{k^2+k}=\tron{k^3-k}+3k\tron{k+1}.$$
    Lại có $k,k+1$ là hai số nguyên liên tiếp nên $2\mid k(k+1).$
    Vì $k^3-k$ và $3k\tron{k+1}$ chia hết cho $6$ nên $(k+1)^3-(k+1)$ cũng chia hết cho $6.$  \\
    Theo nguyên lí quy nạp, khẳng định được chứng minh.
     \item  Đặt $n=2m+1$ với $K$ là tự nhiên. Nếu $m=0,$ ta có $n^2-1=\tron{2m+1}^2-1=0$ chia hết cho $8.$
    \\Giả sử khẳng định đã cho đúng với $m=k>1.$ Ta sẽ chứng minh $8\mid \tron{2k+3}^2-1.$ Thật vậy
    $${2k+3}^2-1=\tron{2k+2}\tron{2k+4}=4(k+1)(k+2).$$
    Vì $k+1,k+2$ là hai số liên tiếp nên $2\mid(k+1)(k+2).$ Điều này dẫn tới $8\mid 4(k+1)(k+2).$\\
    Theo nguyên lí quy nạp, khẳng định được chứng minh.
\end{enumerate}
%ok a
%làm theo form này nhé

}
%cứ làm đi, anh sửa để đồng bộ với mấy phần còn lại thôi
\end{bx}
\begin{bx}
Chứng minh rằng với mọi số tự nhiên $n,$ ta có $ 2^{5 n+3}+5^{n} \cdot 3^{n+2}$ chia hết cho $17.$
\loigiai{
Đặt $P(n)$ là mệnh đề $17 \mid 2^{5 n+3}+5^{n} \cdot 3^{n+2}$.
Dễ thấy $2^{3}+5^{0} \cdot 3^{2}=17$, do vậy $P(0)$ đúng.\\
Giả sử rằng $P(n)$ đúng với mọi $n \geq 0$, ta cần chứng minh rằng $$17 \mid\left(2^{5 n+8}+5^{n+1} \cdot 3^{n+3}\right) .$$ Ta biến đổi như sau
$$
\begin{aligned}
&2^{5 n+8}=2^{5 n+3} \cdot 2^{5}=2^{5 n+3}(34-2)=2^{5 n+3} \cdot 34-2^{5 n+3} \cdot 2, \\
&5^{n+1} \cdot 3^{n+3}=5^{n} \cdot 3^{n+2} \cdot 15=5^{n} \cdot 3^{n+2} \cdot 17-5^{n} \cdot 3^{n+2} \cdot 2.
\end{aligned}
$$
Từ biến đổi trên, ta dễ thấy rằng
$$
17 \mid 2^{5 n+3} \cdot 2+5^{n} \cdot 3^{n+2} \cdot 2.
$$
Như vậy, $P(n+1)$ đúng, theo quy nạp ta suy ra điều cần chứng minh.
}
\end{bx}
\begin{bx}
	Chứng minh rằng với $n$ là số nguyên dương thì $(4n)!$ chia hết cho ${24^n}$.
	\loigiai{ Ta kí hiệu $P$ là mệnh đề
		"Nếu $n\in \mathbb{N}^*$  thì $(4n)!$ chia hết cho ${24^n}$".
		\begin{itemize}
			\item Dễ dàng nhận thấy $4! = 1.2.3.4 = 24$, chia hết cho $24^1$ nên mệnh đề $P$ đúng với $n=1$.
			\item Ta giả sử mệnh đề $P$ đúng tới $n=k$, hay 
			\[24^k \mid (4k)!. \tag{1}\]
			Bây giờ ta cần chứng minh 
			\[{24^{k + 1}}\mid \left( {4(k + 1)} \right)!\]
			\item Ta có 
			$\left( 4(k + 1) \right)! = (4k + 4)! = (4k)!(4k + 1)(4k + 2)(4k + 3)(4k + 4).$
			Như vậy ta sẽ cần chứng minh
			$$24 \mid (4k + 1)(4k + 2)(4k + 3)(4k + 4) = A.$$
			Ta có $(4k + 1)(4k + 2)(4k + 3)(4k + 4)$ chia hết cho $3$ (vì có chứa tích 3 số nguyên liên tiếp) và chia hết cho $8$ (vì có chứa tích hai số chẵn liên tiếp), mà $3$ và $8$ nguyên tố cùng nhau nên $3 \cdot 8 \mid A$ hay $24 \mid  A$.
		\end{itemize}
	 Như vậy mệnh đề cũng đúng với $n=k+1$, theo nguyên lí quy nạp, ta có điều phải chứng minh.}
\end{bx}
\begin{bx}
    Chứng minh rằng  $B_{n}=a^{4^{n}}+a^{3}-a-1$ chia hết cho $(a-1)(a+1)\left(a^{2}+a+1\right)$, trong đó $a>1, n \ge 1$ là các số tự nhiên.
    \loigiai{
    Gọi $P$ là mệnh đề "$B_{n}=a^{4^{n}}+a^{3}-a-1$ chia hết cho $(a-1)(a+1)\left(a^{2}+a+1\right)$".\\
    Nếu $n=1$, thì $B_{1}=a^{4}+a^{3}-a-1=(a-1)(a+1)\left(a^{2}+\right.$ $a+1)$, hiển nhiên mệnh đề đúng. Giả sử mệnh đề $P$ đúng với $n=k \geq 1$ hay $B_{k}$ chia hết cho $(a-1)(a+1)\left(a^{2}+a+1\right)$. Ta kí hiệu $K(a+1)$ là biểu thức luỹ thừa của $a+1$. Khi đó ta có
    \[\begin{aligned}
    B_{k+1} &=a^{4^{k+1}}+a^{3}-a-1\\&=\left(a^{4}-a+a\right)^{4^{k}}+a^{3}-a-1 \\
    &=\left[a\left(a^{3}-1\right)+a\right]^{k}+a^{3}-a-1\\&=K(a-1)\left(a^{2}+a+1\right)
    +a^{4^{k}}+a^{3}-a-1\\&=K(a-1)\left(a^{2}+a+1\right)+B_{k}.
    \end{aligned}\]
Mặt khác ta lại có 
    \[\begin{aligned}
    B_{k+1} &=\left(a^{4}+a-a\right)^{4^{k}}+a^{3}-a-1\\&=\left[a\left(a^{3}+1\right)-a\right]^{4^{k}}+a^{3}-a-1 \\
    &=K(a+1)+a^{4^{k}}+a^{3}-a-1\\&
    =K(a+1)+B_{k}.
    \end{aligned}\]
    Không khó đề nhận ra rằng $a-1, a+1$ và $a^{2}+a+1$ là các số nguyên tố cùng nhau, do vậy mà $B_{k+1}$ sẽ chia hết cho tích của chúng. Như vậy mệnh đề $P$ đúng với $n=k+1$ nên theo nguyên lí quy nạp ta có điều phải chứng minh.
    }
\end{bx}
\begin{bx}
    Với $x$ là số thực sao cho $x + \dfrac{1}{x} \in \mathbb{Z}$, chứng minh rằng với mọi số nguyên dương $n$, thì
	\[x^n + \dfrac{1}{x^n} \in \mathbb{Z}.\tag{1}\]
	\loigiai{Ta sẽ chứng minh (1) đúng bằng phương pháp quy nạp.
		\begin{itemize}
			\item Theo giả thiết thì (1) đúng khi $n=1$. \\
			Ta có $x^2 + \dfrac{1}{x^2} = \left( x + \dfrac{1}{x} \right)^2 - 2 \in \mathbb{Z}$, suy ra $(1)$ đúng khi $n=2$.
			\item Giả sử (1) đúng tới $n=k$ $(k\in \mathbb{N}^*, k\ge 2)$, tức là $x^k + \dfrac{1}{x^k} \in \mathbb{Z}$.\hfill(2)\\
			Ta cần chứng minh (1) cũng đúng với $n=k+1$, tức là chứng minh 
			\[x^{k+1} + \dfrac{1}{x^{k+1}} \in \mathbb{Z}.\tag{3}\]
			Ta có
			\begin{align*}
					\left( x^k + \dfrac{1}{x^k} \right)\left( x + \dfrac{1}{x} \right) &= \left( x^{k + 1} + \dfrac{1}{x^{k + 1}} \right) + \left( x^{k - 1} + \dfrac{1}{x^{k - 1}} \right)\\
					\Leftrightarrow x^{k + 1} + \dfrac{1}{x^{k + 1}} &= \left( x^k + \dfrac{1}{x^k} \right)\left( x + \dfrac{1}{x} \right) - \left( x^{k - 1} + \dfrac{1}{x^{k - 1}} \right).\tag{4}
			\end{align*}
		Mà theo giả thiết thì $x + \dfrac{1}{x} \in \mathbb{Z}$, theo giả thiết quy nạp thì 
			$x^k + \dfrac{1}{x^k},\,\, x^{k - 1} + \dfrac{1}{x^{k - 1}}$
			là những số nguyên nên từ (4) suy ra (3) đúng, tức là (1) đúng khi $n=k+1$. 
		\end{itemize}
	Như vậy theo nguyên lí quy nạp suy ra (1) đúng với mọi số nguyên dương $n$.}
\end{bx}
\begin{bx} Chứng minh rằng mọi số hữu tỉ dương $r<1$ có thể được viết dưới dạng
$$r=\dfrac{1}{q_{1}}+\dfrac{1}{q_{2}}+\ldots+\dfrac{1}{q_{m}},$$
trong đó $q_{1}, \ q_{2}, \ldots,\ q_{m}$ là các số nguyên phân biệt thỏa mãn 
\[q_{1}\mid q_{2}, q_{2}\mid q_{3}, \ldots, q_{m-1} \mid q_{m}.\]
\loigiai{
Ta viết $r=\dfrac{p}{q},$ trong đó $(p,q)=1.$ Ta sẽ chứng minh bài toán bằng quy nạp theo $p.$ \\
    Trước hết, khẳng định đúng với $p=1.$ Ta giả sử khẳng định đúng với $p\le n-1.$ Với $p=n,$ đặt $$k=1+\left[\dfrac{q}{n}\right],$$
    khi đó $0<kn-q<n<q.$ Theo giả thiết quy nạp, do $kn-q<n$ nên ta có thể biểu diễn
    $$\dfrac{kn-q}{q}=\dfrac{1}{q_{1}}+\dfrac{1}{q_{2}}+\ldots+\dfrac{1}{q_{m}}.$$
    Khi đó số $\dfrac{n}{q}$ viết được thành tổng như sau
    $$\dfrac{n}{q}=\dfrac{1}{k}+\dfrac{kn-q}{k q}=\dfrac{1}{k}+\dfrac{1}{k q_{1}}+\dfrac{1}{k q_{2}}+\ldots+\dfrac{1}{k q_{m}}.$$
    Do $q_{1}\mid q_{2}, q_{2}\mid q_{3}, \ldots, q_{m-1} \mid q_{m}$ nên
    $$k\mid kq_1,\ kq_1\mid kq_2,\ kq_2\mid kq_3,\ldots,kq_{m-1}\mid kq_m.$$
    Theo nguyên lí quy nạp, bài toán được chứng minh.}
\end{bx}

\begin{bx}
Chứng minh rằng với mọi số nguyên dương $n$ thì $$2^{n}\mid \tron{3+\sqrt{5}}^n+\tron{3-\sqrt{5}}^n.$$
\loigiai{
Gọi $P$ là mệnh đề $"A_n=\tron{3+\sqrt{5}}^n+\tron{3-\sqrt{5}}^n$ chia hết cho $2^n".$\\
\begin{itemize}
    \item Nếu $n=1$ và $n=2,$ thì $A_1=6$ và $A_2=28$, hiển nhiên mệnh đề đúng.\\
    \item Giả sử mệnh đề $P$ đúng với $n=k$ và $n=k-1$ trong đó $k\ge 2,$ hay
    \begin{align*}
        2^{k}&\mid \tron{3+\sqrt{5}}^k+\tron{3-\sqrt{5}}^k=B,\\ 2^{k-1}&\mid \tron{3+\sqrt{5}}^{k-1}+\tron{3-\sqrt{5}}^{k-1}=C.
    \end{align*}
    Bây giờ ta cần chứng minh
    $$2^{k+1}\mid \tron{3+\sqrt{5}}^{k+1}+\tron{3-\sqrt{5}}^{k+1}=D.$$
    \item Biến đổi $D,$ ta nhận được
    $$\tron{3+\sqrt{5}}^{k+1}+\tron{3-\sqrt{5}}^{k+1}=6\vuong{\tron{3+\sqrt{5}}^{k}+\tron{3-\sqrt{5}}^{k}}-4\vuong{\tron{3+\sqrt{5}}^{k-1}+\tron{3-\sqrt{5}}^{k-1}}.$$
    Vì vậy, ta cần chứng minh
    \begin{align*}
        2^{k+1} &\mid 6\vuong{\tron{3+\sqrt{5}}^{k}+\tron{3-\sqrt{5}}^{k}}=6B,\\
    2^{k+1}&\mid 4\vuong{\tron{3+\sqrt{5}}^{k-1}+\tron{3-\sqrt{5}}^{k-1}}=4C.
    \end{align*}
   Ta thấy điều này hiển nhiên đúng do $2^{k}\mid B$ và $2^{k-1}\mid C.$
\end{itemize}
Như vậy, mệnh đề cũng đúng với $n=k+1,$ theo nguyên lí quy nạp, ta có điều phải chứng minh.
}
\end{bx}

\begin{bx}
	Chứng minh rằng 
	$\left( 1 + \sqrt 2  \right)^n  + \left( 1 - \sqrt 2  \right)^n, \,\, \sqrt 2 \left[ \left( 1 + \sqrt 2  \right)^n  - \left( 1 - \sqrt 2  \right)^n  \right]$
	đều là các số chẵn với mọi số nguyên dương chẵn $n$.
	\loigiai{Ta sẽ chứng minh bằng quy nạp rằng 
			\begin{align*}
				&\left( 1 + \sqrt 2  \right)^n  + \left( 1 - \sqrt 2  \right)^n =2a_n,\\
				&\left( 1 + \sqrt 2  \right)^n  - \left( 1 - \sqrt 2  \right)^n  = b_n\sqrt 2 
		\end{align*}
	với mọi số nguyên dương chẵn $n$. Trong đó $a_n$, $b_n$ là các số tự nhiên.
		\begin{itemize}
			\item Kiểm tra với $n=2$. Ta có 
			{\allowdisplaybreaks
				\begin{align*}
					&\left( 1 + \sqrt 2  \right)^2  + \left( 1 - \sqrt 2  \right)^2  = 2\left( 1^2  + \sqrt 2 ^2  \right) = 2.3.\\
					&\left( 1 + \sqrt 2  \right)^2  - \left( 1 - \sqrt 2  \right)^2  = 4.1.\sqrt 2 =4\sqrt2.
			\end{align*}}Vậy khẳng định đúng khi $n=2$.
			\item Với số nguyên dương chẵn $k$ bất kỳ, giả sử điều khẳng định đã đúng tại $n=k$.  Khi đó, tại số chẵn tiếp theo $n=k+2$ ta có
			
				\begin{align*}
					\left( 1 + \sqrt 2  \right)^{k + 2}  + \left( 1 - \sqrt 2  \right)^{k + 2}
					&= \left( 1 + \sqrt 2  \right)^2 \left( 1 + \sqrt 2  \right)^k  + \left( 1 - \sqrt 2  \right)^2 \left( 1 - \sqrt 2  \right)^k \\
					&= 3\left[ \left( 1 + \sqrt 2  \right)^k  + \left( 1 - \sqrt 2  \right)^k  \right] + 2\sqrt 2 \left[ \left( 1 + \sqrt 2  \right)^k  - \left( 1 - \sqrt 2  \right)^k \right]\\
					&= 3.2a_k + 2\sqrt 2 b_k\sqrt 2  \\&= 2\left( 3a_k + 2b_k \right)\\&=2a_{k+1},
			\end{align*}trong đó $a_{k + 1}  = 3a_k  + 2b_k  \in \mathbb{N}$.
			Ta có
				\begin{align*}
					\left( 1 + \sqrt 2  \right)^{k + 2}  - \left( 1 - \sqrt 2  \right)^{k + 2} 
					&=\left( 1 + \sqrt 2  \right)^2 \left( 1 + \sqrt 2  \right)^k  - \left( 1 - \sqrt 2  \right)^2 \left( 1 - \sqrt 2  \right)^k \\
					&= 3\left[ \left( 1 + \sqrt 2  \right)^k  - \left( 1 - \sqrt 2 \right)^k  \right] + 2\sqrt 2 \left[ \left( 1 + \sqrt 2  \right)^k  + \left( 1 - \sqrt 2  \right)^k  \right]\\
					&= 3.b_k \sqrt 2  + 2\sqrt 2.2a_k  \\&= \left( 3b_k  + 4a_k  \right)\sqrt 2  \\&= b_{k + 1} \sqrt 2,
			\end{align*}
		trong đó $b_{k + 1}  = 4a_k  + 3b_k  \in \mathbb{N}$.
		\end{itemize}
		Bài toán được chứng minh.
	}
\end{bx}

\begin{bx}
Hãy tìm chữ số tận cùng của $A_n=2^{2^n}+1$ với mọi số nguyên $n\ge2.$
\loigiai{
 Ta sẽ chứng minh $A_n$ có tận cùng là $7$ bằng quy nạp. 
\begin{itemize}
    \item Nếu $n=2,$ thì $A_2=2^4+1=17$ có tận cùng là $7,$ hiển nhiên mệnh đề đúng.
    \item Giả sử mệnh đề đúng tới $n=k\ge 2,$ hay
    $A_k=2^{2^k}+1$ có tận cùng là $7,$ kéo theo $2^{2^k}$ có tận cùng là $6.$\\
    Bây giờ ta cần chứng minh mệnh đề đúng với $n=k+1$ hay $A_{k+1}=2^{2^{k+1}}+1$ có tận cùng là $7.$
    \item Biến đổi $A_{k+1},$ ta có
    $$2^{2^{k+1}}+1=2^{2^k\cdot2}+1=\tron{2^{2^k}}^2+1.$$
    Vì $2^{2^k}$ có tận cùng là $6$ nên $\tron{2^{2^k}}^2$ cũng có tận cùng là $6.$ Điều này dẫn tới $\tron{2^{2^k}}^2+1$ có tận cùng bằng $7.$
\end{itemize}
Như vậy, mệnh đề đúng với $n=k+1,$ theo nguyên lí quy nạp, $A_n$ có tận cùng là $7$ với $n\ge 2.$}
\end{bx} 
\begin{bx}
Cho dãy số thực $a_1,a_2,\ldots,a_{2000}$ thỏa mãn điều kiện
$$a^3_{1}+a^3_{2}+\ldots+a^3_{n}=\tron{a_1+a_2+\ldots+a_n}^2,$$
với mọi $1\le n\le 2000. $ Chứng minh rằng mọi số hạng của dãy là số nguyên.
\nguon{Russia Mathematical Olympiad 2000}
\loigiai{
Ta sẽ chứng minh quy nạp $a_n$ là một số nguyên và 
$$a_1+a_2+\ldots+a_n=\dfrac{N_N\tron{N_n+1}}{2},$$
với $N_n$ là số nguyên không âm. 
\begin{itemize}
    \item Nếu $n=1,$ thì $a_1=1$ hoặc $a_0,$ kéo theo $N_n=1$ hoặc $N_n=0.$ Hiển nhiên mệnh đề đúng.
    \item Giả sử, mệnh đề đúng với $n=k,$ hay là 
    $$a^3_{1}+a^3_{2}+\ldots+a^3_{k}=\tron{a_1+a_2+\ldots+a_k}^2,\quad a_1+a_2+\ldots+a_k=\dfrac{N_k\tron{N_k+1}}{2}.$$
    Bây giờ ta sẽ chứng minh mệnh đề đúng với $n=k+1,$ hay
    $$a^3_{1}+a^3_{2}+\ldots+a^3_{k}=\tron{a_1+a_2+\ldots+a_{k+1}}^2,\quad a_1+a_2+\ldots+a_{k+1}=\dfrac{N_{k+1}\tron{N_{k+1}+1}}{2}.$$
    \item Biến đổi đẳng thức cần chứng minh, ta có
    $$\tron{\dfrac{N_k\tron{N_k+1}}{2}+a_{k+1}}^2=\tron{\dfrac{N_k\tron{N_k+1}}{2}}^2+a^3_{k+1}.$$
    Biến đổi tương đương cho ta
    $$a_{k+1}\tron{a_{k+1}-N_k-1}\tron{a_{k+1}+N_k}=0.$$
    Từ đây, ta suy ra $a_{k+1}\in\left\{0,N_k+1,-N_k\right\}.$ Điều này dẫn tới, $a_{k+1}$ là số nguyên.\\
    Ta xét các trường hợp sau.
    \begin{enumerate}
        \item Với $a_{k+1}=0,$ ta suy ra $N_{k+1}=N_k$ là số nguyên không âm.
        \item Với $a_{k+1}=N_k+1,$ ta thu được
        $$a_1+a_2+\ldots+a_{k+1}=\dfrac{N_k\tron{N_k+1}}{2}+\tron{N_k+1}=\dfrac{\tron{N_{k}+1}\tron{N_k+2}}{2}.$$
        Từ đây, ta suy ra $N_{n+1}=N_n+1$ là số nguyên.
        \item Với $a_{k+1}=-N_k,$ ta nhận được
        $$a_1+a_2+\ldots+a_{k+1}=\dfrac{N_k\tron{N_k+1}}{2}-N_k=\dfrac{N_k\tron{N_k-1}}{2}.$$
         Từ đây, ta suy ra $N_{n+1}=N_n-1$ là số nguyên.
    \end{enumerate}
\end{itemize}
Như vậy, mệnh đề đúng với $n=k+1,$ theo nguyên lí quy nạp, ta suy ra điều phải chứng minh.}
\end{bx}
\begin{bx}
Chứng minh rằng với mọi số nguyên dương $n$ thì $3^{2^n}-1$ chia hết cho $2^{n+2}$ và không chia hết cho $2^{n+3}.$
\loigiai{
Xét mệnh đề  với mọi số nguyên dương $n$ thì $3^{2^n}-1$ chia hết cho $2^{n+2}$ và không chia hết cho $2^{n+3}.$
\begin{itemize}
    \item Nếu $n=1,$ thì $3^{2^n}-1=8,$ hiển nhiên mệnh đề đúng.
    \item Giả sử mệnh đề đúng tới $n=k,$ hay là 
    $$2^{k+2}\mid3^{2^k}-1,\qquad 2^{k+3}\nmid 3^{2^k}-1.$$
    Bây giờ, ta cần chứng minh
    $$2^{k+3}\mid3^{2^{k+1}}-1,\qquad 2^{k+4}\nmid 3^{2^{k+1}}-1.$$
    \item Biến đổi $3^{2^{k+1}}-1,$ ta có
    $$3^{2^{k+1}}-1=\tron{3^{2^k}-1}\tron{3^{2^k}+1}.$$
    Vì $2^{k+2}\mid \tron{3^{2^k}-1}$ nên $2\mid\tron{3^{2^k}+1}.$ Điều này dẫn tới 
    $$2^{k+3}=2^{k+2}\cdot2=\tron{3^{2^k}-1}\tron{3^{2^k}+1}=3^{2^{k+1}}-1.$$
    Mặt khác, vì $\tron{3^{2^k}-1,3^{2^k}+1}=2$ nên $4\nmid 3^{2^k}+1,$ kéo theo
    $$2^{k+4}=2^{k+2}\cdot4\nmid \tron{3^{2^k}-1}\tron{3^{2^k}+1}=3^{2^{k+1}}-1.$$
\end{itemize}
Như vậy, mệnh đề đúng với $n=k+1,$ theo nguyên lí quy nạp, ta có điều phải chứng minh.}
\end{bx}
\begin{bx}
Chứng minh rằng $A_n=4^{n+1}+5^{2n-1}$ chia hết cho $21$ với mọi $n$ nguyên dương.
\loigiai{
Với $n=1$ và $n=2,$  ta có $A_1=21$ và $A_2=189$ chia hết cho $21.$
    \\Giả sử khẳng định đã cho đúng với $n=1,2,\ldots,k$ trong đó $k\ge 2.$ Ta sẽ chứng minh $21\mid 4^{n+2}+5^{2n+1}.$ Thật vậy,
    $$4^{n+2}+5^{2n+1}=29\tron{4^{k+1}+5^{2k-1}}-100\tron{4^k+5^{2k-3}}.$$
    Vì $4^{k+1}+5^{2k-1}$ và $4^k+5^{2k-3}$ chia hết cho $21$ nên $4^{n+2}+5^{2n+1}$ cũng chia hết cho $21.$  \\
    Theo nguyên lí quy nạp, khẳng định được chứng minh.
}
\end{bx}
\begin{bx}
Chứng minh rằng với $n$ là số nguyên dương thì $\tron{n^2}!$ chia hết cho $\tron{n!}^n.$
\loigiai{
Với $n=1,$  ta có $n^2!=1$ chia hết cho $\tron{n!}^n=1.$
    \\Giả sử khẳng định đã cho đúng với $n=1,2,\ldots,k.$ Ta sẽ chứng minh $\tron{\tron{k+1}!}^{k+1}\mid \tron{k+1}^2!.$ Ta có
    $$\tron{k+1}^2!=k^2!(k^2+1)(k^2+2)\ldots(k+1)^2.$$ 
    Vì $k^2+1,k^2+2,\ldots,k^2+k$ là $k$ số nguyên liên tiếp nên  $k!\mid k^2+1,k^2+2,\ldots,k^2+k.$ Điều này dẫn đến
    $$\tron{k!}^{k+1}\mid \tron{k+1}^2!.$$\\
    Bây giờ, ta cần chứng minh $\tron{k+1}^{k+1}\mid\tron{k+1}^2!.$ Thật vậy,
$$\dfrac{\tron{k+1}^2-k-1}{k+1}+1=k+1.$$
Từ đây, ta suy ra có $k+1$ số nguyên chia hết cho $k+1$ trong $\tron{k+1}^2!.$ Do đó, $\tron{k+1}^{k+1}\mid\tron{k+1}^2!.$\\
    Theo nguyên lí quy nạp, khẳng định được chứng minh.
}
\end{bx}
\begin{bx}
Chứng minh rằng với $a,n$ là những số nguyên dương thì $(a+1)^n-an-1$ chia hết cho $a^2.$
\loigiai{
Với $n=1,$ ta có $(a+1)^n-an-1=0$ chia hết cho $a^2.$
    \\Giả sử khẳng định đã cho đúng với $n=1,2,\ldots,k.$\\
    Ta sẽ chứng minh $a^2\mid (a+1)^{k+1}-a(k+1)-1.$ 
    Thật vậy
    $$(a+1)^{k+1}-a(k+1)-1=(a+1)\tron{(a+1)^{k}-ak-1}+a^2n.$$
    Vì $(a+1)^{k}-ak-1$ và $a^2n$ chia hết cho $a^2$ nên $(a+1)^{k+1}-a(k+1)-1$ cũng chia hết cho $a^2.$  \\
    Theo nguyên lí quy nạp, khẳng định được chứng minh.
}
\end{bx}

\begin{bx}
Cho $a,b,c,d$ là các số nguyên dương thỏa mãn đồng thời các điều kiện.
\begin{enumerate}
    \item[i,] $a-b+c-d$ là số lẻ.
    \item[ii,] $a-b+c-d$ là ước của $a^{2}-b^{2}+c^{2}-d^{2}.$
\end{enumerate}
Chứng minh $a-b+c-d$ rằng là ước của $a^{n}-b^{n}+c^{n}-d^{n}$, với mọi số nguyên dương $n$.
\nguon{Ibero America 2012}
\loigiai{
Từ giả thiết $a-b+c-d$ là một ước lẻ của $a^{2}-b^{2}+c^{2}-d^{2}.$ Lại có
$$(a-b+c-d)\mid\tron{(a+c)^2-(b+d)^2}$$
nên $a-b+c-d$ là ước lẻ của $2(a c-b d)$, kéo theo
\[(a-b+c-d) \mid (a c-b d).\]
Bây giờ ta sẽ chứng minh bằng quy nạp rằng, $a-b+c-d$ là ước của $a^{n}-b^{n}+c^{n}-d^{n}$ với mọi số nguyên dương $n.$ Với $n=1,2,$ khẳng định đúng. Giả sử khẳng định đúng đến $n=k-1\geqslant 3,$ và ta có
    \[(a-b+c-d) \mid \tron{a^{n}-b^{n}+c^{n}-d^{n}},\text{ với }n=1,2,\ldots,k-1.\]
Ta đặt $e=a-b+c-d$, thì từ  $a^{k-1}+c^{k-1} \equiv b^{k-1}+d^{k-1}\pmod{e}$ và $a+c \equiv b+d\pmod{e}$ ta có
    \[(a+c)\left(a^{k-1}+c^{k-1}\right) \equiv(b+d)\left(b^{k-1}+d^{k-1}\right)\pmod{e}.\]
    Khai triển và sắp xếp lại thứ tự ta được
    \[a^{n}-b^{n}+c^{n}-d^{n} \equiv b d\left(b^{n-2}+d^{n-2}\right)-a c\left(a^{n-2}+c^{n-2}\right) \equiv 0\pmod{e},\]
    với đồng dư thức cuối cùng là kết quả của các đồng dư thức $bd\equiv a c\pmod{e}$ và $$b^{n-2}+c^{n-2} \equiv a^{n-2}+c^{n-2}\pmod{e}.$$
Từ đây theo nguyên lí quy nạp toán học, ta thu được kết quả cần chứng minh. }
\end{bx}
\subsubsection*{Bài tập tự luyện}
\setcounter{bai}{0}
\begin{bai}
	Chứng minh các mệnh đề sau
	\begin{itemize}
		\item  $u_n=n^3+3n^2+5n$ chia hết cho $3$, $\forall n\in \mathbb{N}^*$. 
		\item $v_n=3^{2n+1}+2^{n+2}$ chia hết cho $7$, $\forall n\in \mathbb{N}^*$.
	\end{itemize}
\end{bai}
\begin{bai}
	Chứng minh các mệnh đề sau
	\begin{itemize}
		\item  $4^n+15n-1$ chia hết cho $9$, với mọi số tự nhiên $n$.
		\item   $4^{2n}-3^{2n}-7$ chia hết cho $84$, với mọi $n\in \mathbb{N}^*$.
	\end{itemize}
\end{bai}
\begin{bai}
	Chứng minh các mệnh đề sau đây đúng với mọi $n\in \mathbb{N}^*$.
	\begin{itemize}
		\item  $n^3+2n$ chia hết cho $3$.
		\item  $7.2^{2n-2}+3^{2n-1}$ chia hết cho $5$.
		\item  $n^3+11n$ chia hết cho $6$.
	\end{itemize}
\end{bai}
\begin{bai}
	Chứng minh các mệnh đề sau đây đúng với mọi $n\in \mathbb{N}^*$.
	\begin{itemize}
		\item  $13^n-1$ chia hết cho $6$.
		\item   $11^{n+1}+12^{2n-1}$ chia hết cho $133$.
		\item  $5.2^{3n-2}+3^{3n-1}$ chia hết cho $19$.
	\end{itemize}
\end{bai}
\begin{bai}
	Chứng minh rằng $3 \mid n^3+2n$ với mọi $n$ là số nguyên không âm.
\end{bai}
\begin{bai}
	Chứng minh rằng $5 \mid n^5-n$ với mọi $n$ là số nguyên không âm.
\end{bai}
\begin{bai}
	Chứng minh rằng $6 \mid n^3-n$ với mọi $n$ là số nguyên không âm.
\end{bai}
\begin{bai}
	Chứng minh rằng $8 \mid n^2-1$ với mọi $n$ là số nguyên dương lẻ.
\end{bai}
\begin{bai}
	Chứng minh bằng quy nạp rằng tập hợp $n$ phần tử có $\dfrac{n(n-1)}{2}$ tập con chứa đúng hai phần tử trong đó $n$ là số nguyên lớn hơn hoặc bằng 2.
\end{bai}
\begin{bai}
	Chứng minh rằng $4^{n+1}+5^{2n-1}$ chia hết cho 21 với mọi $n$ nguyên dương.
\end{bai}
\begin{bai}%Câu 12
	Chứng minh rằng với mọi $ n\ge 2$, ta luôn có $a_n=\left(n+1\right)\left(n+2\right)...\left(n+n\right)$ chia hết cho $2^n$
\end{bai}
\begin{bai}
Chứng minh rằng với mọi $n\in N$ thì ta có $8\mid 5^{n+1}+2\cdot3^n+1.$
\end{bai}

\begin{bai}
Chứng minh rằng một số tạo bởi $3^n$ chữ số $1$ chia hết cho $3^n.$
\end{bai}
\begin{bai}
Chứng minh rằng với mọi số nguyên dương $a,b,c$ thỏa mãn $a^2+b^2=c^2$ thì $E,F$ chia hết cho $D,$ trong đó $k$ là số nguyên lớn hơn $2$ và
$$E_k=a^{2^k}+b^{2^k}+c^{2^k},\quad F_k=(ab)^{2^k}+(bc)^{2^k}+(ca)^{2^k},\quad D=\dfrac{a^4+b^4+c^4}{2}.$$
\end{bai}
\begin{bai}
Chứng minh rằng với mọi số nguyên dương $n$ thì $2^{3^n}+1$ chia hết cho $3^{n+1}$ và không chia hết cho $3^{n+2}.$
\end{bai}
\begin{bai}
 Chứng minh rằng với các số nguyên dương $a,b$ thì tồn tại vô số số nguyên $n$ thỏa mãn
$$n^2\mid a^n-n^n.$$
\end{bai}
\begin{bai}
 Cho $a,b$ là các số thực thỏa mãn $a+b,a^2+b^2,a^3+b^3,a^4+b^4$ là các số nguyên. Chứng minh rằng với mọi số nguyên dương $n$ thì $a^n+b^n$ là số nguyên.
\end{bai}
\begin{bai}
Chứng minh rằng với mọi số nguyên dương $m,n$ thì luôn tồn tại số nguyên $k$ sao cho $2^k-m$ có ít nhất $n$ ước nguyên tố.
\end{bai}
\begin{bai}
	Cho $ a,b,c,d,m$ là các số tự nhiên sao cho $ a+d$, $ (b-1)c$, $ ab-a+c$ chia hết cho $ m$. Chứng minh rằng $x_n=a.b^n+cn+d$ chia hết cho $ m$ với mọi số tự nhiên $ n$.
\end{bai}
\begin{bai}
	Cho $q$ là một số tự nhiên chẵn lớn hơn 0. Chứng minh rằng với mỗi số nguyên không âm $n$, số $q^{(q+1)^{n}}+1$ chia hết cho $(q+1)^{n+1}$ và không
	chia hết cho $(q+1)^{n+2}$.
\end{bai}