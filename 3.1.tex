\chapter{Số chính phương, số lập phương, căn thức}

Cũng như số nguyên tố, số chính phương là một “loại số” mới được giới thiệu lần đầu ở bậc trung học cơ sở. Số chính phương (tiếng Anh là \chu{square number}) còn được gọi là số hình vuông bởi nó là diện tích của một hình vuông có cạnh là số tự nhiên. Có rất nhiều tính chất thú vị xoay quanh số chính phương như chúng là những số tự nhiên duy nhất có một lượng lẻ ước nguyên dương, hay tổng của dãy những số lẻ đầu tiên là một số chính phương, tổng của lập phương các số tự nhiên đầu tiên cũng là số chính phương,...\\ \\
Trong khi đó, căn thức trong một vài trường hợp, là phép toán ngược của phép lũy thừa. Mối liên hệ mật thiết giữa các căn bậc và lũy thừa, chẳng hạn như căn bậc hai và số chính phương, đã kiến tạo nên những tính chất số học hay và đẹp.
\\ \\
Ở chương III của quyển sách này, tác giả muốn tập trung nghiên cứu các bài toán xoay quanh hai "loại số" này và các lũy thừa có số mũ cao hơn, cùng với đó tìm hiểu thêm tính số học của căn thức. Chương III được chia làm 5 phần, tương ứng với 5 dạng bài tập khác nhau.
\begin{itemize}
    \item\chu{Phần 1.} Biến đổi đại số.
    \item\chu{Phần 2.} Ứng dụng của đồng dư thức.
    \item\chu{Phần 3.} Phương pháp kẹp lũy thừa.
    \item\chu{Phần 4.} Ước chung lớn nhất và tính chất lũy thừa.
    \item\chu{Phần 5.} Căn thức trong số học.
\end{itemize}

\section*{Các định nghĩa}
\begin{dx}
Số chính phương là số viết được thành bình phương một số tự nhiên.
\end{dx}
Một vài số chính phương đầu dãy có thể kể đến như
\begin{multicols}{4}
\begin{itemize}
    \item $0=0^2$ 
    \item $1=1^2$ 
    \item $4=2^2$ 
    \item $9=3^2$ 
    \item $16=4^2$
    \item $25=5^2$
    \item $36=6^2$
    \item $49=7^2$
    \item $64=8^2$
    \item $81=9^2$
    \item $100=10^2$
    \item $121=11^2$
\end{itemize}
\end{multicols}

\begin{dx}
Số lập phương là số viết được thành lập phương một số nguyên.
\end{dx}
Một vài số lập phương không âm đầu dãy có thể kể đến như
\begin{multicols}{4}
\begin{itemize}
    \item $0=0^3$ 
    \item $1=1^3$ 
    \item $8=2^3$ 
    \item $27=3^3$ 
    \item $64=4^3$
    \item $125=5^3$
    \item $216=6^3$
    \item $343=7^3$
    \item $512=8^3$
    \item $729=9^3$
    \item $1000=10^3$
    \item $1331=11^3$
\end{itemize}
\end{multicols}

\begin{dx}
Cho số nguyên dương $n.$ Căn bậc $n$ của một số $x$ là một số $r$ sao cho $r^n=x.$
\end{dx}

Một số có thể có $0,1$ hoặc $2$ căn bậc $n.$ Chẳng hạn
\begin{enumerate}
    \item Số $5$ có hai căn bậc hai là $\sqrt{5}$ và $-\sqrt{5}.$
    \item Số $-12$ không có căn bậc hai nào.  
    \item Số $27$ có một căn bậc ba là $3.$
    \item Với mọi số nguyên dương $n,$ số $0$ có một căn bậc $n$ là chính nó.
\end{enumerate}

\section{Biến đổi đại số}

Trong mục này, ta sử dụng các cách biến đổi đại số đã học nhằm đưa biểu thức về dạng chính phương hoặc lập phương.

\subsection*{Bài tập tự luyện}

\begin{btt}
Cho $a,b$ và $c$ là các số nguyên thỏa mãn $ab + bc + ca = 1$. \\
Chứng minh rằng $\tron{1+a^2}\tron{1+b^2}\tron{1+c^2}$ là số chính phương.
\end{btt}

\begin{btt}
Chứng minh rằng tổng của tích của bốn số tự nhiên liên tiếp và $1$ luôn là số chính phương.
\nguon{Vũ Hữu Bình}
\end{btt}

\begin{btt} \label{scp1.111}
Chứng minh rằng ${A}=1^{3}+2^{3}+3^{3}+\cdots+2016^{3}$ là số chính phương.
\end{btt}

\begin{btt}
Chứng minh rằng $N=\underbrace{11\ldots1}_{1995} \underbrace{00\ldots0}_{1994}5+1$ là một số chính phương.
\nguon{Vietnamese National Mathematical Olympiad 1995, Group A}	
\end{btt}

\begin{btt}
Chứng minh rằng $\underbrace {11\ldots11}_{2021}\underbrace {22\ldots22}_{2022}5$
là một số chính phương.
\end{btt}

\begin{btt} 
Chứng minh rằng với mỗi số nguyên $n \geq 6$ thì $$a_{{n}}=1+\dfrac{2\cdot6\cdot10 \cdots(4 n-2)}{(n+5)(n+6) \cdots(2 n)}$$ là một
số chính phương.
\nguon{Chuyên Đại học Sư phạm Hà Nội 2014}
\end{btt}

\begin{btt}
Cho 2 số nguyên ${a}, {b}$ thỏa mãn ${a}^{2}+{b}^{2}+1=2({ab}+{a}+{b})$. Chứng minh $a$ và $b$ là
hai số chính phương liên tiếp.
\nguon{Chuyên Đại học Sư phạm Hà Nội 2016}
\end{btt}

\begin{btt}
Tìm các số nguyên $n$ sao cho $A=n^{4}+n^{3}+n^{2}$ là số chính phương.
\end{btt}

\begin{btt}
Tìm tất cả các số nguyên $n$ sao cho $$A=n^6-3n^5-3n^4+20n^3-48n^2+96n-80$$ là số lập phương.
\end{btt}

\begin{btt}
Chứng minh rằng không tồn tại số nguyên dương $n$ sao cho tích
$$\left(1^4+1^2+1\right)\left(2^4+2^2+1\right)\cdots\left(n^4+n^2+1\right)$$
là một số chính phương.
\nguon{Tuymada 2019}
\end{btt}

\begin{btt}
Tìm tất cả các số tự nhiên $n$ thỏa mãn $2 n+1,3 n+1$ là các số chính phương và $2n+9$ là số nguyên tố. 
\nguon{Chuyên Toán Hà Nội 2017}
\end{btt}

\begin{btt}
Cho 2 số hữu tỉ $a, b$ thoả mãn điều kiện $a^{3}+4 a^{2} b=4 a^{2}+b^{4}$. Chúng minh rằng $\sqrt{a}-1$ là bình phương của 1 số hữu tỉ.
\nguon{Polish Mathematical Olympiad 2010}
\end{btt}

\begin{btt}
Chứng minh rằng nếu $n$ là tổng của ba số chính phương thì $3n$ được viết dưới dạng tổng của bốn số chính phương.
\nguon{Titu Andresscu}
\end{btt}

\begin{btt}
Cho các số nguyên $a$, $b$, $c$, $d$ thỏa mãn $a + b = c + d$. Chứng minh rằng $a^2 + b^2 + c^2 + d^2$ là tổng của ba số chính phương.
\end{btt}

\begin{btt}
 Cho các số nguyên ${a}, {b}$ và số nguyên tố ${p}$ thỏa mãn $\dfrac{{a}^{2}+{b}^{2}}{{p}} \in \mathbb{Z}$. Cho biết ${p}$ là
tổng của hai số chính phương. Chứng minh rằng $\dfrac{{a}^{2}+{b}^{2}}{{p}}$ cũng là tổng của hai số chính phương.
\nguon{Chọn học sinh giỏi tỉnh Thanh Hóa}
\end{btt}


\subsection*{Hướng dẫn bài tập tự luyện}

\begin{gbtt}
Cho $a,b$ và $c$ là các số nguyên thỏa mãn $ab + bc + ca = 1$. \\
Chứng minh rằng $(1+a^2)(1+b^2)(1+c^2)$ là số chính phương.
\loigiai{Từ giả thiết $ab+bc+ca=1,$ ta có
\begin{align*}
1+a^{2}&=ab+bc+ac+a^2=(a+b)(a+c),\\
1+b^{2}&=ab+bc+ac+b^2=(a+b)(b+c),\\
1+c^{2}&=ab+bc+ac+c^2=(c+b)(a+c).
\end{align*}
Lấy tích theo vế, ta có
$(1+a^2)(1+b^2)(1+c^2)=\left[(a+b)(a+c)(b+c)\right]^2$ là số chính phương. \\Bài toán được chứng minh.}
\end{gbtt}

\begin{gbtt}
Chứng minh rằng tổng của tích của bốn số tự nhiên liên tiếp và $1$ luôn là số chính phương.
\nguon{Vũ Hữu Bình}
\loigiai{
Dựa vào nhận xét 
\begin{align*}
    n(n+1)(n+2)(n+3)+1
    &=\left(n^2+3n\right)\left(n^2+3n+2\right)+1
    \\&=\left(n^2+3n+1\right)^2-1+1
    \\&=\left(n^2+3n+1\right)^2,
\end{align*}
bài toán đã cho được chứng minh.
}
\end{gbtt}

\begin{gbtt} \label{scp1.111}
Chứng minh rằng ${A}=1^{3}+2^{3}+3^{3}+\cdots+2016^{3}$ là số chính phương.
\loigiai{
Ta nhận thấy rằng
\begin{align*}
    A&=\left(\dfrac{1}{2}\right)^{2}\cdot 4\cdot 1+\left(\dfrac{2}{2}\right)^{2}\cdot 4 \cdot 2+\left(\dfrac{3}{2}\right)^{2}\cdot 4\cdot 3+\left(\dfrac{4}{2}\right)^{2}\cdot 4\cdot 4+\cdots+\left(\dfrac{2016}{2}\right)^{2}\cdot 4\cdot 2016
    \\&=\left(\dfrac{1}{2}\right)^{2}\left(2^{2}-0^{2}\right)+\left(\dfrac{2}{2}\right)^{2}\left(3^{2}-1^{2}\right)+\left(\dfrac{3}{2}\right)^{2}\left(4^{2}-2^{2}\right)+\cdots+\left(\dfrac{2016}{2}\right)^{2}\left(2017^{2}-2015^{2}\right)
    \\&=\left(\dfrac{1\cdot 2}{2}\right)^{2}-\left(\dfrac{0\cdot 1}{2}\right)^{2}+\left(\dfrac{2\cdot 3}{2}\right)^{2}-\left(\dfrac{1\cdot 2}{2}\right)^{2}+\cdots+\left(\dfrac{2016\cdot 2017}{2}\right)^{2}-\left(\dfrac{2015\cdot 2016}{2}\right)^{2}
    \\&=\left(\dfrac{2016\cdot 2017}{2}\right)^{2}
    \\&=(1008\cdot 2017)^{2}.
\end{align*}
Như vậy, ${A}$ là số chính phương. Bài toán được chứng minh.}
\end{gbtt}
\begin{gbtt}
Chứng minh rằng $N=\underbrace{11\ldots1}_{1995} \underbrace{00\ldots0}_{1994}5+1$ là một số chính phương.
\nguon{Vietnamese National Mathematical Olympiad 1995, Group A}	
\loigiai{
Biến đổi biểu thức đã cho, ta có
	\begin{align*}
	\underbrace{11\ldots1}_{1995} \underbrace{00\ldots0}_{1994}5+1&=\dfrac{{10}^{1995}-1}{9}\cdot\left({10}^{1995}+5\right)+1
	\\&=\dfrac{{\left({10}^{1995}\right)}^{2}+4\cdot{10}^{1995}-5}{9}+1\\
	&=\dfrac{{\left({10}^{1995}\right)}^{2}+4\cdot{10}^{1995}+4}{9}
	\\&={\left(\dfrac{{10}^{1995}+2}{3}\right)}^{2}.
	\end{align*} 
	Do $10^{1995}+2$ chia hết cho $3,$ ta suy ra $N$ là số chính phương. Bài toán được chứng minh.}	
\end{gbtt}

\begin{gbtt}
Chứng minh rằng $\underbrace {11\cdots11}_{2021}\underbrace {22\cdots22}_{2022}5$
là một số chính phương.
\loigiai{
Biến đổi biểu thức đã cho, ta có
	\begin{align*}
	\underbrace {11\cdots11}_{2021}\underbrace {22\cdots22}_{2022}5
	&=10^{2023}\cdot\underbrace{11\cdots11}_{2021}+10\cdot\underbrace{22\cdots22}_{2022}+5
	\\&=\dfrac{10^{2023}\tron{10^{2021}-1}}{9}+\dfrac{20\tron{10^{2022}-1}}{9}+5
	\\&=\dfrac{10^{4024}-10^{2023}+20\cdot 10^{2022}-20+45}{9}
	\\&=\dfrac{10^{4024}+10\cdot 10^{2022}+25}{9}
	\\&=\tron{\dfrac{10^{2012}+5}{3}}^2.
	\end{align*} 
	Do $10^{2012}+5$ chia hết cho $3,$ ta suy ra $N$ là số chính phương. Bài toán được chứng minh.}
\end{gbtt}

\begin{gbtt} Chứng minh rằng với mỗi số nguyên $n \geq 6$ thì $$a_{{n}}=1+\dfrac{2\cdot6 \cdot10 \cdots(4 n-2)}{(n+5)(n+6) \cdots(2 n)}$$ là một
số chính phương.
\nguon{Chuyên Đại học Sư phạm Hà Nội 2014}
\loigiai{
Với mọi $n\ge 6,$ ta nhận thấy rằng
$$
\begin{aligned}
a_{n} &=1+\dfrac{2^{n} \cdot[1\cdot3 \cdot5 \cdots(2 n-1)] \cdot(n+4) !}{(2 n) !}
\\&=1+\dfrac{2^{n}(n+4) !}{2 \cdot 4 \cdot 6 \cdots 2 n} \\
&=1+\dfrac{2^{n} \cdot 1 \cdot 2 \cdot 3 \cdots n(n+1)(n+2)(n+3)(n+4)}{2^{n} \cdot 1 \cdot 2 \cdot 3 \cdot 4 \cdots n} \\
&=1+(n+1)(n+2)(n+3)(n+4) \\
&=1+\left(n^2+5n+4\right)\left(n^2+5n+6\right) \\
&=\left(n^{2}+5 n+5\right)^{2}-1+1
\\&=\left(n^{2}+5 n+5\right)^{2}.
\end{aligned}
$$
Như vậy, $a_n$ là số chính phương với mọi $n\ge 6.$ Bài toán được chứng minh.}
\end{gbtt}

\begin{gbtt}
Cho 2 số nguyên ${a}, {b}$ thỏa mãn ${a}^{2}+{b}^{2}+1=2({ab}+{a}+{b})$. Chứng minh $a$ và $b$ là
hai số chính phương liên tiếp.

\nguon{Chuyên Đại học Sư phạm Hà Nội 2016}

\loigiai{
Từ giả thiết, ta có \begin{align*}
    a^{2}+b^{2}+1=2(ab+a+b) 
    &\Leftrightarrow a^2+b^2+1-2ab+2a-2 b=4 a\\
    &\Leftrightarrow(a-b+1)^2=4a.
\end{align*}
Ta được $a$ là số chính phương. Đặt ${a}={x}^{2},$ ở đây $x$ là số tự nhiên. Phép đặt này cho ta
$$
\left(x^{2}-b+1\right)^{2}=4 x^{2} \Rightarrow x^2-b+1=2 x \Rightarrow b=(x-1)^{2}.
$$
Vậy $a$ và $b$ là hai số chính phương liên tiếp. Bài toán được chứng minh.}
\end{gbtt}

\begin{gbtt}
Tìm các số nguyên $n$ sao cho $A=n^{4}+n^{3}+n^{2}$ là số chính phương.
\loigiai{Ta có ${A}={n}^{4}+{n}^{3}+{n}^{2}={n}^{2}\left({n}^{2}+{n}+1\right).$ Ta xét các trường hợp sau.
\begin{enumerate}
    \item Nếu $A=0,$ ta thu được $n=0.$ 
    \item Nếu ${A} \neq 0$ thì ${n}^{2}+{n}+1$ là số chính phương. Đặt $n^{2}+n+1=k^{2},$ với $k$ là số tự nhiên. Ta có
\begin{align*}
4\left(n^{2}+n+1\right)=4 k^{2} &\Leftrightarrow(2 n+1)^{2}+3=(2 k)^{2} \\&\Leftrightarrow(2k-2 n-1)(2 k+2 n+1)=3. \end{align*}
Tới đây, ta lập được bảng giá trị sau.
\begin{center}
    \begin{tabular}{c|c|c}
       $2k-2n-1$  & $2k+2n+1$ & $n$ \\
       \hline
       $-3$  & $-1$ & $0$ \\
       \hline       
       $-1$  & $-3$ & $-1$ \\
        \hline
       $1$  & $3$ & $0$ \\
       \hline       
       $3$  & $1$ & $-1$        
    \end{tabular}
\end{center}
Giải phương trình ước số trên, ta tìm được $n=-1$ và $n=0,$ nhưng  $A\ne 0$ nên ta chỉ chọn $n=-1.$ 
\end{enumerate}
Tóm lại, tất cả các giá trị thỏa mãn đề bài của $n$ là $n=-1$ và $n=0.$}
\end{gbtt}

\begin{gbtt}
Tìm tất cả các số nguyên $n$ sao cho $$A=n^6-3n^5-3n^4+20n^3-48n^2+96n-80$$ là số lập phương.
\loigiai{
Ta phân tích được $A=(n-2)^3\left(n^3+3n^2+3n+10\right).$ Với $A=0,$ ta có $n=2,$ còn với $A \neq 0$ thì $$n^3+3n^2+3n+10$$ phải là số lập phương. Đặt $n^3+3n^2+3n+10=m^3.$ Ta có
$$
(n+1)^3+9=m^3\Leftrightarrow (m-n-1)\left[m^2+m(n+1)+(n+1)^2\right]=9.
$$
Do $m^2+m(n+1)+(n+1)^2\ge 0,$ ta xét các trường hợp sau
\begin{enumerate}
    \item Với $m-n-1=1$ và $m^2+m(n+1)+(n+1)^2=9,$ ta có $m=n+2.$ Như vậy
     $$(n+2)^2+(n+2)(n+1)+(n+1)^2=9\Leftrightarrow 3n^2+9n-2=0.$$
     Phương trình trên không có nghiệm nguyên. Trường hợp này không xảy ra.
    \item Với $m-n-1=3$ và $m^2+m(n+1)+(n+1)^2=3,$ ta có $m=n+4.$ Như vậy
    \begin{align*}
        (n+4)^2+(n+4)(n+1)+(n+1)^2=9\Leftrightarrow 3(n+1)(n+4)=0\Leftrightarrow\hoac
          {&n=-1  \\
          &n=-4.}
    \end{align*}
    \item Với $m-n-1=9$ và $m^2+m(n+1)+(n+1)^2=1,$ ta có $m=n+10.$ Như vậy 
     $$(n+10)^2+(n+10)(n+1)+(n+1)^2=9\Leftrightarrow 3n^2+33n+102=0.$$
     Phương trình trên không có nghiệm nguyên. Trường hợp này không xảy ra.     
\end{enumerate}
Tóm lại, tất cả các giá trị thỏa mãn đề bài của $n$ là $n=-1$ và $n=-4.$}
\end{gbtt}

\begin{gbtt}
Chứng minh rằng không tồn tại số nguyên dương $n$ sao cho tích
$$\left(1^4+1^2+1\right)\left(2^4+2^2+1\right)\cdots\left(n^4+n^2+1\right)$$
là một số chính phương.
\nguon{Tuymada 2019}
\loigiai{Khai triển
$k^4+k^2+1=\left(k^2+k+1\right)\left(k^2-k+1\right),\forall k\in \mathbb{N^*}$
cho ta nhận xét
\begin{align*}
    &{\quad}\left(1^4+1^2+1\right)\left(2^4+2^2+1\right)\cdots\left(n^4+n^2+1\right)=3^2\cdot7^2\cdots\left(n^2-n+1\right)^2\left(n^2+n+1\right).
\end{align*}
Từ đây, ta suy ra $n^2+n+1$ là số chính phương. Đặt $n^2+n+1=m^2.$ Phép đặt này cho ta
\begin{align*}
    4n^2+4n+4=4m^2
    &\Rightarrow (2m+1)^2+3=(2m)^2
    \\&\Rightarrow (2m-2n-1)(2m+2n+1)=3
    \\&\Rightarrow \heva{&2m-2n-1=1 \\ &2m+2n+1=3}
    \\&\Rightarrow \heva{&m=1 \\ &n=0.}
\end{align*}
Do giả thiết $n$ nguyên dương, $n=0$ không thỏa mãn. Bài toán được chứng minh.}
\end{gbtt}

\begin{gbtt}
Tìm tất cả các số các số tự nhiên $n$ thỏa mãn $2 n+1,\ 3 n+1$ là các số chính phương và $2n+9$ là số
nguyên tố. 
\nguon{Chuyên Toán Hà Nội 2017}
\loigiai{
Từ giả thiết, ta có thể đặt $2n+1=x^2,3n+1=y^2,$ ở đây $x,y$ là các số nguyên dương. \\
Ta sẽ chọn các tham số $a,b$ sao cho
$$ax^2+by^2=a(2n+1)+b(3n+1)=2n+9.$$
Giải hệ $\heva{&2a+3b=2 \\ &a+b=9},$ ta được $a=25,b=-16.$ Bây giờ, ta xét phân tích
$$2n+9=25x^2-16y^2=(5x-4y)(5x+4y).$$
Do $2n+9$ là số nguyên tố nên trong $5x-4y,\ 5x+4y$ phải có một số bằng $1,$ nhưng vì $$0<5x-4y<5x+4y$$ nên $5x-4y=1.$ Kết hợp với phép đặt $2n+1=x^2,3n+1=y^2,$ ta thu được hệ nghiệm nguyên dương
\begin{align*}
\heva{&5x-4y=1 \\ &3x^2-2y^2-1=0}
&\Leftrightarrow \heva{&y=\dfrac{5x-1}{4} \\ & 3x^2-2\left(\dfrac{5x-1}{4}\right)^2-1=0}    
\\&\Leftrightarrow \heva{&y=\dfrac{5x-1}{4} \\ & (x-1)(x-9)=0} \\&
\Leftrightarrow
\left[\begin{aligned}
     &\heva{x&=1 \\ y&=1}  \\
     &\heva{x&=9 \\ y&=11.} 
\end{aligned}\right.
\end{align*}
Tới đây, ta xét các trường  hợp sau.
\begin{enumerate}
    \item Với $x=1,$ ta được $n=0.$ Lúc này $2n+9=9$ không là số nguyên tố.
    \item Với $x=9,$ ta được $n=40.$ Lúc này $2n+9=89$ là số nguyên tố. 
\end{enumerate}
Như vậy, giá trị duy nhất của $n$ thỏa mãn là $n=40.$}
\end{gbtt}

\begin{gbtt}
Cho 2 số hữu tỉ $a, b$ thoả mãn điều kiện $a^{3}+4 a^{2} b=4 a^{2}+b^{4}$. Chúng minh rằng $\sqrt{a}-1$ là bình phương của 1 số hữu tỉ.
\nguon{Polish Mathematical Olympiad 2010}
\loigiai{
Ta viết lại giả thiết bài toán thành 
\begin{align*}
    a^{3}+4 a^{2} b+4 a b^{2}=4 a^{2}+4 a b^{2}+b^{4} &\Leftrightarrow a(a+2 b)^{2}=\left(2 a+b^{2}\right)^{2}\\&
    \Leftrightarrow a=\dfrac{\left(2 a+b^{2}\right)^{2}}{(a+2 b)^{2}}\\&
    \Leftrightarrow \sqrt{a}=\dfrac{2 a+b^{2}}{a+2 b}.
\end{align*}
Do đó ta có $\sqrt{a} \in \mathbb{Q} .$
Mặt khác ta lại có 
\[\sqrt{a}-1=\dfrac{2 a+b^{2}}{a+2 b}-1=\dfrac{a-2 b+b^{2}}{a+2 b}=\dfrac{\left(a-2 b+b^{2}\right)(a+2 b)}{(a+2 b)^{2}}.\tag{*}\]
Vì vậy ta cần chỉ ra $\left(a-2 b+b^{2}\right)(a+2 b)$ là bình phương của một số hữu tỉ, hay 
$$\tron{a-2 b+b^{2}}(a+2 b)a$$
là bình phương một số hữu tỉ. Thật vậy ta có $$\left(a-2 b+b^{2}\right)(a+2 b) a=a^{2} b^{2}+a^{3}+2 b^{3} a-4 a b^{2}.$$ 
Thay $a^{3}=4 a^{2}+b^{4}-4 a^{2} b$ vào ta có
$$\left(a-2 b+b^{2}\right)(a+2 b) a=a^{2} b^{2}+4 a^{2}+b^{4}-4 a^{2} b+2 b^{3} a-4 a b^{2}=\left(a b+b^{2}-2 a\right)^{2}.$$
Vậy $\left(a-2 b+b^{2}\right)(a+2 b)a$ là bình phương của 1 số hữu tỉ, lại do $a$ là bình phương 1 số hữu tỉ nên 
$$\left(a-2 b+b^{2}\right)(a+2 b)$$
cũng là bình phương một số hữu tỉ. Lúc này từ (*) ta suy ra điều phải chứng minh.}
\end{gbtt}

\begin{gbtt}
Chứng minh rằng nếu $n$ là tổng của ba số chính phương thì $3n$ được viết dưới dạng tổng của bốn số chính phương.
\nguon{Titu Andresscu}
\loigiai{
Giả sử $n = a^2 + b^2 + c^2$. Ta có
\begin{align*}
   3n & = 3a^2 + 3b^2 + 3c^2 \\
      & = \left( a^2 + b^2 + c^2 + 2ab + 2bc + 2ca \right) + \left( a^2 - 2ab + b^2 \right) + \left( b^2 - 2bc + c^2 \right) + \left( c^2 - 2ca + a^2 \right) \\
      & = (a + b + c)^2 + (a - b)^2 + (b - c)^2 + (c - a)^2.
\end{align*}
Bài toán được chứng minh. }
\end{gbtt}

\begin{gbtt}
 Cho các số nguyên $a$, $b$, $c$, $d$ thỏa mãn $a + b = c + d$. Chứng minh rằng $a^2 + b^2 + c^2 + d^2$ là tổng của ba số chính phương.
 \loigiai{
Từ giả thiết ta có
\begin{align*}
   a^2 + b^2 + c^2 + d^2 & = a^2 + b^2 + c^2 + d^2 + 2a(a + b - c - d) \\
   & = 3a^2 + b^2 + c^2 + d^2 + 2ab - 2ac - 2ad \\
   	& = \left( a^2 + 2ab + b^2 \right) + \left( a^2 - 2ac + c^2 \right) + \left( a^2 - 2ad + d^2 \right) \\
   & = (a + b)^2 + (a - c)^2 + (a - d)^2.
  \end{align*}
Bài toán được chứng minh.}
\end{gbtt}

\begin{gbtt}
 Cho các số nguyên ${a}, {b}$ và số nguyên tố ${p}$ thỏa mãn $\dfrac{{a}^{2}+{b}^{2}}{{p}} \in \mathbb{Z}$. Cho biết ${p}$ là
tổng của hai số chính phương. Chứng minh rằng $\dfrac{{a}^{2}+{b}^{2}}{{p}}$ cũng là tổng của hai số chính phương.
\nguon{Chọn học sinh giỏi tỉnh Thanh Hóa}
\loigiai{
Từ giả thiết, ta có thể đặt ${p}={c}^{2}+{d}^{2}$ với ${c}, {d}$ là các số nguyên. Phép đặt trên cho ta $$\dfrac{a^{2}+b^{2}}{p}=\dfrac{\left(a^{2}+b^{2}\right)\left(c^{2}+d^{2}\right)}{p^{2}}=\left(\dfrac{a d+b c}{p}\right)^{2}+\left(\dfrac{a c-b d}{p}\right)^{2}.$$
Một cách tương tự, ta cũng có $$\dfrac{a^{2}+b^{2}}{p}=\dfrac{\left(a^{2}+b^{2}\right)\left(c^{2}+d^{2}\right)}{p^{2}}=\left(\dfrac{a d-b c}{p}\right)^{2}+\left(\dfrac{a c+b d}{p}\right)^{2}.$$
Hơn thế nữa, ta xét tích dưới đây
$$({ac}+{bd})({ac}-{bd})={a}^{2} {c}^{2}-{b}^{2} {d}^{2}={a}^{2}\left({c}^{2}+{d}^{2}\right)-{d}^{2}\left({a}^{2}+{b}^{2}\right).$$
Do ${p}={c}^{2}+{d}^{2}$ và $p\mid \left({{a}^{2}+{b}^{2}}\right)$ nên $p\mid (a c+b d)(a c-b d).$ Như vậy, hoặc $p$ là ước của $ac+bd,$ hoặc $p$ là ước của $ac-bd.$ Đối chiếu từng trường hợp với một trong hai biến đổi trên, ta thu được điều phải chứng minh.}
\begin{luuy}
\nx \\
Biến đổi ở câu trên đều có liên quan đến định thức $Brahmagupta - Fibonacci$
\[\left(a^2+b^2\right)\left(x^2+y^2\right)=\left(ax+by\right)^2+\left(ay-bx\right)^2.\]
\end{luuy}
\end{gbtt}

\section{Ứng dụng của đồng dư thức}
\subsection*{Lí thuyết}
Trong mục này, chúng ta sẽ tìm hiểu về các ứng dụng về đồng dư thức đối với các bài toán chứa yếu tố số chính phương, lập phương. Tác giả sẽ phát biểu lại lại các kiến thức đã học ở \chu{chương I} dưới dạng khác.

\begin{enumerate}
    \item Một số chính phương bất kì chỉ có thể
    \begin{multicols}{2}
            \begin{itemize}
            \item Đồng dư với $0$ hoặc $1$ theo modulo $3.$
            \item Đồng dư với $0$ hoặc $1$ theo modulo $4.$
            \item Đồng dư với $0,1$ hoặc $4$ theo modulo $5.$
            \item Đồng dư với $0,1,2$ hoặc $4$ theo modulo $7.$
            \item Đồng dư với $0,1$ hoặc $4$ theo modulo $8.$
            \item Đồng dư với $0,1,3,4,5,9$ theo modulo $11.$            
        \end{itemize}
    \end{multicols}
    \item Một số lập phương bất kì chỉ có thể
        \begin{itemize}
            \item Đồng dư với $-1,0$ hoặc $1$ theo modulo $7.$
            \item Đồng dư với $-1,0$ hoặc $1$ theo modulo $9.$
        \end{itemize} 
    \item Một lũy thừa bậc bốn bất kì chỉ có thể 
        \begin{itemize}
            \item Đồng dư với $0$ hoặc $1$ theo modulo $5.$
            \item Đồng dư với $0$ hoặc $1$ theo modulo $16.$
        \end{itemize}
\end{enumerate}

\subsection*{Ví dụ minh họa}
\begin{bx}
Chứng minh các số sau đây không thể là số chính phương
\begin{multicols}{2}
\begin{enumerate}[a,]
    \item $A=2020^{2021}+2021^{2020}.$
    \item $B=3^{57}+7^{53}.$
\end{enumerate}
\end{multicols}
\loigiai{
\begin{enumerate}[a,]
    \item Xét trong hệ đồng dư modulo $3,$ ta có
    \begin{align*}
        2020^{2021}+2021^{2020}
        &=(3\cdot673+1)^{2021}+(3\cdot674-1)^{2020}
        \equiv 1^{2021}+(-1)^{2020}
        \equiv 2\pmod{3}.  
    \end{align*}
    Không có số chính phương nào đồng dư $2$ theo modulo $3,$ thế nên $A$ không là số chính phương.
    \item Xét trong hệ đồng dư modulo $7,$ ta có
    \begin{align*}
        3^{57}+7^{53}
        &=27\cdot\left(3^6\right)^9+7^{53}
        \equiv 27\cdot(7\cdot104+1)^9
        \equiv 27\equiv 6\pmod{7}.    
    \end{align*}
    Không có số chính phương nào đồng dư $6$ theo modulo $7,$ thế nên $B$ không là số chính phương.    
    \end{enumerate}}
\end{bx}

\begin{bx}
Chứng minh rằng không tồn tại số tự nhiên $a$ sao cho $a^2+a=2010^{2009}.$
\nguon{Phổ thông Năng khiếu}
\loigiai{Giả sử rằng tồn tại số tự nhiên $a$ thỏa mãn. Giả sử này cho ta
$$4a^2+4a=4\cdot2010^{2009}\Leftrightarrow \tron{2a+1}^2= 4\cdot2010^{2009}+1.$$
Xét trong hệ đồng dư modulo $7$, ta có
$$4\cdot2010^{2009}+1\equiv4\cdot1^{2009}+1\equiv4+1\equiv5\pmod{7}\Rightarrow\tron{2a+1}^2\equiv5\pmod{7}.$$
Không có số chính phương nào đồng dư $5$ theo modulo $7.$ Giả sử sai. Bài toán được chứng minh.}
\end{bx}

\begin{bx}
Tìm tất cả các số nguyên tố $p,q$ thỏa mãn $p^2-q^2-1$ là số chính phương.
\nguon{Adrian Andreescu}
\loigiai{Giả sử tồn tại các số nguyên tố $p,q$ thỏa mãn đề bài. Ta xét các trường hợp sau.
\begin{enumerate}
    \item Với $p>q\ge 3,$ do $p$ và $q$ cùng lẻ, nên $p^2\equiv q^2\equiv 1\pmod{4}.$ Ta có
    $$p^2-q^2-1\equiv 1-1-1=-1\equiv 3 \pmod{4}.$$ 
    Trong trường hợp này, $p^2-q^2-1$ không chính phương.
    \item Với $q=2,$ ta đặt $k^2=p^2-q^2-1.$ Ta lần lượt suy ra
    $$k^2=p^2-5\Rightarrow 5=(p-k)(p+k)\Rightarrow \left\{\begin{matrix} p+k=5\\ p-k=1 \end{matrix}\right. \Rightarrow p=3\Rightarrow q=2.$$
\end{enumerate}
    Kiểm tra trực tiếp, ta thấy $(p,q)=(3,2)$ là cặp số duy nhất thỏa mãn đề bài.}
\end{bx}

\begin{bx}
Tìm số nguyên dương $n>2$ nhỏ nhất thỏa mãn tính chất:
\begin{it}
Tồn tại số $n$ số chính phương liên tiếp có tổng cũng là một số chính phương.
\end{it}
\nguon{Allessandro Ventullo}
\loigiai{Chú ý rằng $18^{2}+19^{2}+\cdots+28^{2}=77^{2}$, vì thế, ta sẽ đi chứng minh $n\ge 11.$ Ta xét các trường hợp sau.
\begin{enumerate}
    \item[i,] $(k-1)^{2}+k^{2}+(k+1)^{2}=3 k^{2}+2\equiv 2 \pmod{3},$
    \item[ii,] $(k-1)^{2}+k^{2}+(k+1)^{2}+(k+2)^{2}=4k^{2}+4 k+6 \equiv 2 \pmod{4},$ 
    \item[iii,] $(k-2)^{2}+(k-1)^{2}+\cdots+(k+2)^{2}+(k+3)^{2}=6k(k+1)+19 \equiv 3 \pmod{4},$     
    \item[iv,] $(k-3)^{2}+(k-2)^{2}+\cdots+(k+4)^{2}=8k(k+1)+44 \equiv 12 \pmod{16}.$
\end{enumerate}
Các đồng dư thức trên chứng tỏ $n$ khác $3,4,6,8.$ Mặt khác
\begin{enumerate}
    \item[v,] $(k-2)^{2}+(k-1)^{2}+\cdots+(k+2)^{2}=5\left(k^{2}+2\right)$ chia hết cho $5$ nhưng không chia hết cho $25,$
    \item[vi,] $(k-3)^{2}+(k-2)^{2}+\cdots+(k+3)^{2}=7\left(k^{2}+4\right)$ chia hết cho $7$ nhưng không chia hết cho $49,$
    \item[vii,] $(k-4)^{2}+(k-3)^{2}+\cdots+(k+4)^{2}=3\left(3 k^{2}+20\right)$ chia hết cho $3$ nhưng không chia hết cho $9,$
    \item[viii,] $(k-4)^{2}+(k-3)^{2}+\cdots+(k+5)^{2}=5\left(2 k^{2}+2 k+17\right)$ chia hết cho $5$ nhưng không cho $25.$
\end{enumerate}
Các lập luận trên chứng tỏ $n$ khác $5,7,9,10.$ Như vậy $n=11$ là đáp số bài toán.}
\end{bx}

\begin{bx}
Tìm các số nguyên dương $n$ thỏa mãn $2^n+12^n+2011^n$ là số chính phương.
\nguon{Adrian Andreescu, Turkey Mathematical Olympiad 2006}
\loigiai{Bằng kiểm tra trực tiếp, ta chỉ ra $n=1$ là đáp số. Ta sẽ chứng minh mọi $n\ge 2$ không thỏa mãn đề bài. 
\begin{enumerate}
    \item Với $n$ chẵn, ta đặt $n=2k,$ trong đó $k$ nguyên dương. Ta có
    \begin{align*}
        2^n+12^n+2011^n&=4^k+12^{n}+\left(3\cdot670+1\right)^{n} \\&\equiv 1+0+1 
        \\&=2 \pmod{3}
    \end{align*}
    nên $2^n+12^n+2011^n$ không chính phương.
    \item Với $n$ lẻ, ta đặt $n=2k+1,$ trong đó $k$ nguyên dương. Ta có
    \begin{align*}
        2^n+12^n+2011^n&=4\cdot 2^{n-2}+12^{n}+(4\cdot 503-1)^{2k+1} \\&\equiv (-1)^{2k+1}\\&\equiv 3 \pmod{4}
    \end{align*}
    nên $2^n+12^n+2011^n$ không chính phương.    
\end{enumerate}
Như vậy, $n=1$ là đáp số bài toán.}
\end{bx}

\subsection*{Bài tập tự luyện}

\begin{btt}
Cho số tự nhiên $n$ thỏa mãn $n(n+1)+6$ không chia hết cho $3$.\\ Chứng minh rằng $2n^{2} +n+8$ không phải là số chính phương.
\nguon{Chuyên Toán Hà Nội 2013}
\end{btt}

\begin{btt}
Tồn tại không số nguyên dương $n$ để $n^5-n+2$ là số chính phương?
\end{btt}

\begin{btt}
Cho số tự nhiên $N=1\cdot3\cdot5\cdot7\cdots2107$. Chứng minh rằng trong $3$ số nguyên liên tiếp $2N-1,  2N$ và $2N+1,$ không có số nào là số chính phương.
\end{btt}

\begin{btt}
Tìm tất cả số tự nhiên $n$ sao cho $12\cdot n!+11^n+2$ là số chính phương.
\end{btt}

\begin{btt}
Tìm số tự nhiên $n \ge 1$ sao cho tổng $1!+2!+3!+\cdots+n!$ là một số chính phương.
\end{btt}

\begin{btt}
Cho số nguyên không âm $A.$ Hãy xác định $A$ biết rằng trong $3$ mệnh đề $\mathcal{P}, \mathcal{Q}, \mathcal{R}$ dưới đây có 2 mệnh đề đúng và 1 mệnh đề sai
\begin{itemize}
    \item $\mathcal{P}:$ "$A+51$ là số chính phương".
    \item $\mathcal{Q}:$ "Chữ số tận cùng của $A$ là $1$".
    \item $\mathcal{R}:$ "$A-38$ là số chính phương".
\end{itemize}
\nguon{Chuyên Toán Phổ thông Năng khiếu 2000}
\end{btt}

\begin{btt}
Giả sử tồn tại hai số nguyên $x,y$ sao cho $x^2-4y$ và $x^2+4y$ đều là số chính phương. Chứng minh rằng $y$ chia hết cho $6.$
\end{btt}

\begin{btt}
Tìm số nguyên tố $p$ để $2p^4-p^2+16$ là số chính phương
\nguon{Leningrad Mathematical Olympiad 1980}
\end{btt}

\begin{btt}
Chứng minh rằng với mọi số nguyên tố $p>3,$ ta có thể biểu diễn $\dfrac{p^6-7}{3}+2p^2$ thành tổng của hai số lập phương.
\nguon{Titu Andreescu}
\end{btt}

\begin{btt}
Tồn tại không các số nguyên tố $p,q,r$ thỏa mãn $\left(p^2-7\right)\left(q^2-7\right)\left(r^2-7\right)$
là số chính phương?
\nguon{Titu Andreescu}
\end{btt}

\begin{btt}
Tìm các số nguyên tố $p,q,r$ thỏa mãn $p+q+r$ không chia hết cho $3,$ đồng thời cả $p+q+r$ và $pq+qr+rp+3$ đều là số chính phương.
\nguon{Turkey Junior Balkan Mathematical Olympiad 2013}
\end{btt}

\begin{btt}
Tồn tại hay không các số nguyên $a,b$ sao cho $$a^5b+3\text{ và }ab^5+3$$
là các số lập phương?
\nguon{USA Junior Mathematical Olympiad 2013}
\end{btt}

\begin{btt}
Tồn tại hay không các số nguyên $a,b,c$ sao cho $$ab^2c+2,\: bc^2a+2\text{ và }ca^2b+2$$
là các số chính phương?
\nguon{China Western Mathematical Olympiad 2013}
\end{btt}

\begin{btt}
Cho tập các số nguyên $S=\{n;n+1;\cdots;n+5\}.$ Chứng minh rằng không thể chia $S$ thành hai tập $A$ và $B$ giao nhau khác rỗng, sao cho tích các phần tử trong $A$ bằng tích các phần tử trong $B.$
\nguon{Cao Đình Huy}
\end{btt}

\begin{btt}
Tìm tất cả các số nguyên $a,b$ sao cho $a^4+b^4+(a+b)^4$ là số chính phương.
\nguon{Vũ Hữu Bình}
\end{btt}

\begin{btt}
Cho $a,b$ là các số nguyên dương. Đặt $A=\tron{a+b}^2-2a^2$ và $B=\tron{a+b}^2-2b^2.$ Chứng minh rằng $A$ và $B$ không thể đồng thời là số chính phương.
\nguon{Chuyên Đại học Sư phạm Hà Nội}
\end{btt}

\begin{btt}
Cho $d$ là một số nguyên dương khác $2,5,13.$ Chứng minh rằng ta có thể tìm được hai số nguyên phân biệt $a,b$ trong tập $\{2;5;13;d\}$ sao cho $ab-1$ không phải là số chính phương.
\end{btt}

\begin{btt}
Với mọi số nguyên dương $n,$ chứng minh rằng $$2020^{4n}+2021^{4n}+2022^{4n}+2023^{4n}$$ không phải là số chính phương.
\end{btt}

\begin{btt}
Tìm tất cả các cặp số tự nhiên $(m,n)$ sao cho $2^m3^n-1$ là số chính phương.
\nguon{Tạp chí Toán học và Tuổi trẻ, tháng 7 năm 2017}
\end{btt}

\begin{btt}
Tìm các số nguyên dương $m,n$ thỏa mãn $A=6^m+2^n+2$ là số chính phương.
\nguon{Croatian Mathematical Olympiad 2009}
\end{btt}

\begin{btt}
Tìm tất cả các số nguyên tố $p$ thỏa mãn $2^p+5\cdot 3^p$ là số chính phương.
\end{btt}

\begin{btt}
Tìm tất cả các số nguyên tố $p$ thỏa mãn $7^p+9^p+259$ là số lập phương.

\end{btt}

\subsection*{Hướng dẫn bài tập tự luyện}

\begin{gbtt}
Cho số tự nhiên $n$ thỏa mãn $n(n+1)+6$ không chia hết cho $3$.\\ Chứng minh rằng $2n^{2} +n+8$ không phải là số chính phương.
\nguon{Chuyên Toán Hà Nội 2013}
\loigiai{
Từ giả thiết $n(n+1)+6$ không chia hết cho $3,$ ta có $n\equiv 1\pmod{3}.$ Như vậy
$$2n^2+n+8\equiv 2+1+8\equiv 2\pmod{3}.$$ 
Không có số chính phương nào chia $3$ dư $2,$ và bài toán được chứng minh.}
\end{gbtt}

\begin{gbtt}
Tồn tại không số nguyên dương $n$ để $n^5-n+2$ là số chính phương?
\loigiai{
Ta nhận thấy
$n^5-n+2=n\left(n^4-1\right)+2.$
Ta đã biết, một lũy thừa mũ $4$ khi chia cho $5$ chỉ có thể dư $0$ hoặc $1.$ Ta xét các trường hợp dưới đây.
\begin{enumerate}
    \item Nếu $n$ chia hết cho $5,$ hiển nhiên $n\left(n^4-1\right)$ chia hết cho $5.$
    \item Nếu $n$ không chia hết cho $5,$ ta có $n^4\equiv 1 \pmod{5},$ và vì thế $5\mid n\left(n^4-1\right).$
\end{enumerate}
Hai lập luận trên giúp ta chỉ ra
$n\left(n^4-1\right)+2\equiv 2\pmod{5}.$
Tuy nhiên, không có số chính phương nào đồng dư $2$ theo modulo $5.$ Câu trả lời là phủ định.}
\end{gbtt}

\begin{gbtt}
Cho số tự nhiên $N=1\cdot3\cdot5\cdot7\cdots2107.$
Chứng minh rằng trong $3$ số nguyên liên tiếp $2N-1,  2N$ và $2N+1,$ không có số nào là số chính phương.
\loigiai{
    Do $N$ là số lẻ, ta có $2N\equiv 2\pmod{4},$ thế nên $2N$ không là số chính phương, và khi đó
    $$2N+1\equiv 3\pmod{4}.$$
    Không có số chính phương nào chia cho $4$ dư $3,$ thế nên $2N+1$ cũng không phải số chính phương. \\
    Hơn thế nữa, rõ ràng $N$ chia hết cho $3,$ và theo đó
    $$2N-1\equiv -1\equiv 2\pmod{3}.$$
    Không có số chính phương nào chia cho $3$ dư $2,$ và $2N-1$ cũng không phải số chính phương.
    \\Bài toán được chứng minh.}
\end{gbtt}

\begin{gbtt}
Tìm tất cả số tự nhiên $n$ sao cho $12\cdot n!+11^n+2$ là số chính phương.
\loigiai{
Với $n\ge 5,$ ta có $12\cdot n!$ chia hết cho $5$. Suy ra \[
12\cdot n!+11^n+2\equiv 0+1+2\equiv 3\pmod{5}.\] 
Điều này chứng tỏ $12\cdot n!+11^n+2$ không là số chính phương với mọi $n\ge 5.$\\
Với $n=0,1,2,3,4,$ ta lập bảng giá trị sau đây.
\begin{center}
    \begin{tabular}{c|c|c|c|c|c}
        $n$ &  $0$ & $1$ & $2$ & $3$ & $4$\\
        \hline
         $12\cdot n!+11^n+2$ & $15$ & $25$ & $147$ & $1405$ & $14931$
    \end{tabular}
\end{center}
Căn cứ vào bảng, ta nhân thấy $n=1$ là số tự nhiên duy nhất thỏa mãn yêu cầu.}
\end{gbtt}

\begin{gbtt}
Tìm số tự nhiên $n \ge 1$ sao cho tổng $1!+2!+3!+\cdots+n!$ là một số chính phương.
\loigiai{
Trong bài toán này, ta xét các trường hợp sau đây.
\begin{enumerate}
	\item Nếu $n=1,$ ta có $1!=1=1^2$ là số chính phương.
	\item Nếu $n=2,$ ta có $1!+2!=3$ không là số chính phương.
	\item Nếu $n=3,$ ta có  $1!+2!+3!=1+1\cdot2+1\cdot2\cdot3=9=3^2$ là số chính phương.
    \item Nếu $n \ge 4,$ do $m!$ chia hết cho $5$ với mọi $n\ge 5$ nên $$1!+2!+3!+\cdots+n!\equiv 1!+2!+3!+4!=33\equiv 3\pmod{5}.$$ 
    Không có số chính phương nào chia cho $5$ được dư là $3.$ Trường hợp này không xảy ra.
\end{enumerate}
Kết luận, $n=1$ và $n=3$ là hai giá trị của $n$ thỏa mãn đề bài.
}
\end{gbtt}

\begin{gbtt}
Cho số nguyên không âm $A.$ Hãy xác định $A$ biết rằng trong $3$ mệnh đề $\mathcal{P}, \mathcal{Q}, \mathcal{R}$ dưới đây có 2 mệnh đề đúng và 1 mệnh đề sai
\begin{itemize}
    \item $\mathcal{P}:$ "$A+51$ là số chính phương".
    \item $\mathcal{Q}:$ "Chữ số tận cùng của $A$ là $1$".
    \item $\mathcal{R}:$ "$A-38$ là số chính phương".
\end{itemize}
\nguon{Chuyên Toán Phổ thông Năng khiếu 2000}
\loigiai{
Giả sử mệnh đề $\mathcal{Q}$ đúng. Xét trong hệ đồng dư modulo $5$, ta thu được
$$A+1 \equiv 2\pmod{5}, \qquad A-38\equiv -7\equiv3\pmod{5}.$$
Không có số chính phương nào đồng dư với $2$ hoặc $3$ theo modulo $5$. Điều này trái với mệnh đề $\mathcal{P}$ và  $\mathcal{R}$. Do đó $2$ cặp mệnh đề $\tron{\mathcal{Q},\mathcal{P}}$ và $\tron{\mathcal{Q},\mathcal{R}}$ không thể cùng đúng. Từ đây, ta suy ra  $\mathcal{P},\mathcal{R}$ là $2$ mệnh đề đúng. Đặt $A+51=x^2$ và $A-38=y^2.$ Trừ theo vế, ta được
$$x^2-y^2=89\Leftrightarrow\tron{x-y}\tron{x+y}=89.$$
Do $0<x-y<x+y$ nên bắt buộc $x-y=1$ và $x+y=89.$ Ta chỉ ra $x=45$ và $A=1974$ từ đây.}
\end{gbtt}

\begin{gbtt}
Giả sử tồn tại hai số nguyên $x,y$ sao cho $x^2-4y$ và $x^2+4y$ đều là số chính phương. Chứng minh rằng $y$ chia hết cho $6.$
\loigiai{
Ta chia bài toán thành các bước làm sau.
\begin{enumerate}[\color{tuancolor}\bf\sffamily Bước 1.]
    \item Ta chứng minh $y$ là số chẵn.\\ Giả sử rằng $y$ là số lẻ. Khi đó do $x^2\equiv 0,1,4\pmod{8}$ nên
    $$x^2+4y\equiv 4,5,0\pmod{8}.$$
    Trường hợp $x^2+4y\equiv 5\pmod{8}$ bị loại. Theo đó, $x$ là số chẵn. Ta đặt $x=2z,$ thế thì
    $$x^2-4y=4\tron{z^2-y},\quad x^2+4y=4\tron{z^2+y}$$
    đều là các số chính phương, và $z^2-y,z^2+y$ cũng như vậy. Ta xét bảng đồng dư modulo $4.$
    \begin{center}
        \begin{tabular}{c|c|c|c|c}
          $\quad z^2\quad $ & $0$ & $1$ & $0$ & $1$ \\
          \hline
          $\quad y\quad $  & $1$ & $1$ & $3$ & $3$ \\
          \hline
          $\quad z^2-y\quad$ & $3$ & $0$ & $1$ & $2$ \\
          \hline
          $\quad z^2+y\quad$ & $1$ & $2$ & $3$ & $0$  \\
        \end{tabular}
    \end{center}
    Ta dễ dàng thấy giả sử phản chứng bị sai từ đây. Ta có $y$ chẵn.
    \item Ta chứng minh $y$ chia hết cho $3.$\\ Ta giả sử phản chứng rằng
    $y\equiv 1,2\pmod{3}.$ Ta lập bảng đồng dư theo modulo $3$
    \begin{center}
        \begin{tabular}{c|c|c|c|c}
          $\quad x^2\quad $ & $0$ & $1$ & $0$ & $1$ \\
          \hline
          $\quad y\quad $  & $1$ & $1$ & $2$ & $2$ \\
          \hline
          $\quad x^2-4y\quad$ & $2$ & $0$ & $1$ & $2$ \\
          \hline
          $\quad x^2+4y\quad$ & $1$ & $2$ & $2$ & $0$  
        \end{tabular}
    \end{center}   
    Ta dễ dàng thấy giả sử phản chứng bị sai từ đây. Ta có $y$ chia hết cho $3.$  
\end{enumerate}
Như vậy, $y$ chia hết cho $[2,3]=6.$ Bài toán được chứng minh.}
\end{gbtt}

\begin{gbtt}
Tìm số nguyên tố $p$ để $2p^4-p^2+16$ là số chính phương
\nguon{Leningrad Mathematical Olympiad 1980}
\loigiai{
Thử trực tiếp với $p=2$ và $p=3,$ ta thấy $p=3$ thỏa mãn. \\
Với $p>3,$ ta có $p^2\equiv 1 \pmod{3}$ do lúc này $p$ không là bội của $3.$ Nhận xét này cho ta
$$2p^{4}-p^{2}+16= 2\left(p^2\right)^2-p^2+16 \equiv 2-1+16=17\equiv 2\pmod{3}.$$ 
Không có số chính phương nào đồng dư $2$ theo modulo $3.$ Trường hợp này không xảy ra. \\
Như vậy, $p=3$ là số nguyên tố cần tìm.}
\end{gbtt}

\begin{gbtt}
Chứng minh rằng với mọi số nguyên tố $p>3,$ ta có thể biểu diễn $\dfrac{p^6-7}{3}+2p^2$ thành tổng của hai số lập phương.
\nguon{Titu Andreescu}
\loigiai{
Theo lí thuyết đã học, ta có $p^2\equiv 0,1 \pmod{3},$ nhưng do $p>3$ nên $p^2\equiv 1 \pmod{3}.$\\
Vì lẽ đó, ta có thể đặt $p^2=3n+1,$ ở đây $n$ là số nguyên dương. Phép đặt này cho ta
\begin{align*}
\dfrac{p^{6}-7}{3}+2 p^{2}&=\dfrac{\left(3n+1\right)^3-7}{3}+2(3n+1) 
\\&=9n^3+9n^2+9n
\\&=(n-1)^3+(2n+1)^3.
\end{align*}
Thông qua biểu diễn trên, bài toán được chứng minh.}
\end{gbtt}

\begin{gbtt}
Tồn tại không các số nguyên tố $p,q,r$ thỏa mãn $\left(p^2-7\right)\left(q^2-7\right)\left(r^2-7\right)$
là một số chính phương?
\nguon{Titu Andreescu}
\loigiai{Ta giả sử phản chứng rằng, tồn tại bộ ba $(p,q,r)$ thỏa mãn đề bài và cả $p\le q\le r.$ Rõ ràng $p^2-7<0$ khi và chỉ khi $p=2,$ và ngược lại, $p^2-7>0$ khi và chỉ khi $p\ge 3.$ Ta xét các trường hợp sau. 
\begin{enumerate}
    \item Với $p=2,$ do $\left(2^{2}-7\right)\left(q^{2}-7\right)\left(r^{2}-7\right)$ nguyên dương nên bắt buộc $q=2,r\ge 3.$ Lúc này 
    $$\left(p^{2}-7\right)\left(q^{2}-7\right)\left(r^{2}-7\right)=9(r^2-7)$$
    là số chính phương, và $r^2-7$ cũng là số chính phương. Ta đặt $k^2=r^2-7,$ khi đó 
    $$(r-k)(r+k)=7\Rightarrow r-k=1, r+k=7.$$
    Ta tìm được $r=4,$ trái với giả thiết $r$ nguyên tố.
    \item Với $p\ge 3,$ xuất phát từ tính chất quen thuộc
    $p^2,q^2,r^2\equiv 1 \pmod{8},$
    ta chỉ ra rằng
    $$p^2-7,q^2-7,r^2-7\equiv 2 \pmod{8},$$   
    Ta đặt $p^2-7=8a+2,q^2-7=8b+2,r^2-7=8c+2.$ Ta có
    $$\tron{p^2-7}\tron{q^2-7}\tron{r^2-7}=8(4a+1)(4b+1)(4c+1).$$
    Số kể trên chia hết cho $8$ nhưng không chia hết cho $16,$ do vậy đây không là số chính phương.
\end{enumerate}
Như vậy, giả sử phản chứng là sai. Bài toán được chứng minh.}
\end{gbtt}

\begin{gbtt}
Tìm các số nguyên tố $p,q,r$ thỏa mãn $p+q+r$ không chia hết cho $3,$ đồng thời cả $p+q+r$ và $pq+qr+rp+3$ đều là số chính phương.
\nguon{Turkey Junior Balkan Mathematical Olympiad 2013}
\loigiai{
Không mất tổng quát, ta giả sử $p\ge q\ge r.$ Dựa vào tính chất của đồng dư thức, ta sẽ lần lượt đi tìm $r,q,p.$ 
\begin{enumerate}[\color{tuancolor}\bf\sffamily Bước 1.]
    \item Ta chứng minh $r=2.$\\ Trong trường hợp $r>2,$ cả $p,q$ và $r$ đều lẻ. Một cách không mất tổng quát, ta xét bảng đồng dư theo modulo $4$ sau.
            \begin{center}
            \begin{tabular}{c|c|c|c|c}
            $p$ & $1$ & $1$ & $1$ & $3$ \\
            \hline
            $q$ & $1$ & $1$ & $3$ & $3$ \\
            \hline
            $r$ & $1$ & $3$ & $3$ & $3$ \\
            \hline
            $pq+qr+rp+3$ & $2$ & $2$ & $2$ & $2$
            \end{tabular}
        \end{center}
    Dựa theo bảng, ta được $pq+qr+rp+3$ chia cho $4$ dư $2,$ do đó $pq+qr+rp+3$ không là số chính phương, mâu thuẫn với giả thiết. Mâu thuẫn này chứng tỏ $r=2.$
    \item Ta chứng minh $q=3.$\\ Với $r=2$ chứng minh được ở trên, ta có $p+q+2$ và $pq+2p+2q+3$ đồng thời là các số chính phương.
    \begin{itemize}
        \item Nếu $q=2,$ ta nhận thấy
        $$pq+2p+2q+3=4p+7\equiv 3\pmod{4}$$
        thế nên $pq+2p+2q+3$ không thể là số chính phương.
        \item Nếu $q>3,$ một cách không mất tổng quát, ta xét bảng đồng dư theo modulo $3$ sau.
            \begin{center}
            \begin{tabular}{c|c|c|c}
            $p$ & $1$ & $1$ & $2$ \\
            \hline
            $q$ & $1$ & $2$ & $2$ \\
            \hline
            $p+q+2$ & $1$ & $2$ & $0$ \\
            \hline
            $pq+2p+2q+3$ & $2$ & $2$ & $0$
            \end{tabular}
        \end{center}
        Dựa theo bảng, ta thấy hai điều kiện $p+q+2$ không chia hết cho $3$ và $pq+2p+2q+3$ chính phương không đồng thời xảy ra, một điều mâu thuẫn. Như vậy $q=3.$
    \end{itemize}
    \item Ta tiếp tục tìm $p$ dựa trên $r=2,q=3.$\\ Với $q=3$ chứng minh được ở trên, ta có $p+5$ và $5p+9$ đồng thời là các số chính phương. Ta đặt $5p+9=a^2,$ với $a$ nguyên dương. Khi đó
    $$5p=a^2-9\Rightarrow 5p=(a-3)(a+3).$$
    Do các ước nguyên dương của $5p$ chỉ có thể là $1,5,p,5p$ nên ta xét các trường hợp sau.
    \begin{itemize}
        \item Nếu $a-3=1,$ ta có $a=4$ nhưng không tìm được $p.$
        \item Nếu $a-3=5,$ ta có $a=8$ và $p=11.$
        \item Nếu $a-3=p$ và $a+3=5,$ ta có $p=-1,$ vô lí.
        \item Nếu $a-3=5p$ và $a+3=1,$ ta có $a=-2,$ vô lí.
    \end{itemize}
\end{enumerate}
Kết quả, tất cả các bộ $(p,q,r)$ cần tìm là $(11,3,2)$ và các hoán vị của nó.}
\end{gbtt}

\begin{gbtt}
Tồn tại hay không các số nguyên $a,b$ sao cho $$a^5b+3\text{ và }ab^5+3$$
là các số lập phương?
\nguon{USA Junior Mathematical Olympiad 2013}
\loigiai{Ta giả sử tồn tại các số tự nhiên $a,b$ thỏa mãn đề bài. Ta có
$$\heva{& a^5b+3 \equiv -1,0,1 \pmod{9} \\& ab^5+3 \equiv -1,0,1 \pmod{9}}\Rightarrow \heva{& a^5b \equiv -2,-3,-4 \pmod{9} \\& ab^5 \equiv -2,-3,-4 \pmod{9}.}$$
Ta xét các trường hợp sau.
\begin{enumerate}
    \item Với $ab^5\equiv -3 \pmod{9},$ ta có $ab^5$ chia hết cho $3.$ Tuy nhiên, nếu $b$ chia hết cho $3$ thì $ab^5$ chia hết cho $3^5,$ vô lí. Mâu thuẫn này chứng tỏ $a$ chia hết cho $3,$ và như vậy
    $$a^5b\equiv 0 \pmod{9},$$
    trái với việc $a^5b \equiv -2,-3,-4 \pmod{9}.$ 
    \item Với $a^5b\equiv -3 \pmod{9},$ ta chỉ ra điều vô lí bằng cách lập luận tương tự trường hợp đầu tiên.
    \item Với $\heva{& a^5b \equiv -2,-4 \pmod{9} \\& ab^5 \equiv -2,-4 \pmod{9}},$ lấy tích theo vế, ta được
    $$a^6b^6\equiv 4,8,16 \pmod{9}. $$
    Vì $a^6b^6$ là số lập phương nên chỉ trường hợp $a^6b^6\equiv 8 \pmod{9}$ là có thể xảy ra, và như vậy thì $a^6b^6$ chia $3$ dư $2,$ vô lí do $a^6b^6$ cũng là số chính phương. 
\end{enumerate}
Các trường hợp trên đều tồn tại mâu thuẫn. Giả sử sai, và chứng minh hoàn tất.}
\end{gbtt}

\begin{gbtt}
Tồn tại hay không các số nguyên $a,b,c$ sao cho $$ab^2c+2,\: bc^2a+2\text{ và }ca^2b+2$$
là các số chính phương?
\nguon{China Western Mathematical Olympiad 2013}
\loigiai{
Ta giả sử tồn tại các số nguyên $a,b,c$ thỏa mãn đề bài. Ta có
$$\heva{ab^2c+2\equiv0,1\pmod{4}\\bc^2a+2\equiv0,1\pmod{4}\\ca^2b+2\equiv0,1\pmod{4}}\Rightarrow \heva{ab^2c&\equiv2,3\pmod{4}\\bc^2a&\equiv2,3\pmod{4}\\ca^2b&\equiv2,3\pmod{4}.}$$
Ta xét các trường hợp sau.
\begin{enumerate}
    \item Nếu $ab^2c\equiv2\pmod{4},$ ta có $ab^2c$ chia hết cho $2.$ Tuy nhiên nếu $b$ chia hết cho $2$ thì $ab^2c$ chia hết cho $2^2$, vô lí. Mâu thuẫn này chứng tỏ $a$ hoặc $c$ chia hết cho $2$. Vì $a,c$ có vai trò ngang nhau, ta giả sử $a$ chia hết cho $2$, và như vậy
    $$ca^2b\equiv 0\pmod{4},$$
    trái với việc $ca^2b\equiv2,3 \pmod{4}.$
    \item Nếu $bc^2a\equiv2\pmod{4},$ ta chỉ ra điều vô lí bằng lập luận tương tự trường hợp đầu tiên.
    \item Nếu $ca^2b\equiv2\pmod{4},$ ta chỉ ra điều vô lí bằng lập luận tương tự trường hợp đầu tiên.
    \item Nếu $ab^2c\equiv bc^2a\equiv ca^2b\equiv3\pmod{4},$ lấy tích theo vế, ta nhận được
    $$a^4b^4c^4\equiv3\pmod{4}.$$
    Vì $a^4b^4c^4$ là số chính phương nên $a^4b^4c^4$ chia $4$ chỉ dư $0$ hoặc $1$, mâu thuẫn với điều trên.
\end{enumerate}
Các trường hợp trên đều tồn tại mâu thuẫn. Giả sử sai, và chứng minh hoàn tất.}
\end{gbtt}

\begin{gbtt}
Cho tập các số nguyên $S=\{n;n+1;\cdots;n+5\}.$ Chứng minh rằng không thể chia $S$ thành hai tập $A$ và $B$ giao nhau khác rỗng, sao cho tích các phần tử trong $A$ bằng tích các phần tử trong $B.$
\nguon{Cao Đình Huy}
\loigiai{
Giả sử phản chứng rằng có thể chia $S$ thành hai tập $A,B$ như đề bài. Theo đó, tích các phần tử trong $S$ (gọi là $P$) phải là một số chính phương. Ta xét các trường hợp sau đây.
\begin{enumerate}
    \item Nếu trong $S$ có số $n+k$ chia hết cho $7,$ ta nhận thấy nó là số duy nhất trong $S$ chia được cho $7.$ Theo đó, trong $A$ và $B,$ có một tập có tích các số chia hết cho $7,$ nhưng tập còn lại thì không, mâu thuẫn.
    \item Nếu trong $S$ không có số nào chia hết cho $7,$ số dư của các số ấy khi chia cho $7$ bắt buộc là $1,2,3,4,5,6$ theo một thứ tự nào đó. Như vậy
    $$P\equiv 1\cdot2\cdot3\cdots6=720\equiv6\pmod{7}.$$
    Không có số chính phương nào đồng dư $6$ theo modulo $7.$ Trường hợp này không xảy ra.
\end{enumerate}
Dựa theo các mâu thuẫn chỉ ra, bài toán đã cho được chứng minh.}
\begin{luuy}
Bạn đọc có thể tự sáng tạo thêm các kết quả thú vị hơn cho bài toán trên bằng cách thay modulo $7$ thành modulo một số nguyên tố có dạng $4k+3.$
\end{luuy}
\end{gbtt}

\begin{gbtt}
Tìm tất cả các số nguyên $a,b$ sao cho $a^4+b^4+(a+b)^4$ là số chính phương.
\nguon{Vũ Hữu Bình}
\loigiai{Ta nhận thấy $a=b=0$ thỏa mãn. Phần còn lại của bài toán, ta xét trường hợp $a,b$ khác $0.$ Đặt $d=(a,b),$ khi đó tồn tại các số nguyên dương $x,y$ sao cho $(x,y)=1$ và $a=dx,b=dy.$ Phép đặt này cho ta
$$a^4+b^4+(a+b)^4=d^4\left[x^4+y^4+(x+y)^4\right].$$
Do $d^4$ là chính phương khác $0$ và giả thiết $a^4+b^4+(a+b)^4$  chính phương, ta suy ra $$x^4+y^4+(x+y)^4$$ cũng là số chính phương. Ta đã biết, một lũy thừa mũ $4$ chỉ có thể đồng dư theo $0$ hoặc $1$ theo modulo $16,$ phụ thuộc vào tính chẵn lẻ của chính lũy thừa đó. Ta xét bảng đồng dư theo modulo $16$ sau
    \begin{center}
        \begin{tabular}{c|c|c|c|c}
            $x^4$ & $0$ & $0$ & $1$ & $1$\\
            \hline
            $y^4$ & $0$ & $1$ & $0$ & $1$\\
            \hline
            $(x+y)^4$ & $0$ & $1$ & $1$ & $0$\\ 
            \hline
            $x^4+y^4+(x+y)^4$ & $0$ & $2$ & $2$ & $2$
        \end{tabular}
    \end{center}
Một số chính phương không thể đồng dư $2$ theo modulo $16,$ thế nên đối chiếu từng cột của bảng, ta chỉ ra $x,y$ cùng chẵn. Tuy nhiên, điều này là không thể do điều kiện phép đặt $(x,y)=1.$ \\
Như vậy, chỉ có cặp $(a,b)=(0,0)$ là thỏa mãn đề bài.}
\end{gbtt}

\begin{gbtt}
Cho $a,b$ là các số nguyên dương. Đặt $A=\tron{a+b}^2-2a^2$ và $B=\tron{a+b}^2-2b^2.$ Chứng minh rằng $A$ và $B$ không thể đồng thời là số chính phương.
\nguon{Chuyên Đại học Sư phạm Hà Nội}
\loigiai{
Giả sử $A$ và $B$ đồng thời là số chính phương. Đặt $d=(a,b),a=dm,b=dn,$ với $(m,n)=1.$ Ta có
$$(m+n)^2-2m^2,\quad (m+n)^2-2n^2$$
đều là các số chính phương. Ta xét các trường hợp sau đây.
\begin{enumerate}
    \item Nếu $m,n$ cùng là số lẻ thì $(m+n)^2$ chia hết cho $4,$ còn $2n^2\equiv 2\pmod{4}.$ Vì thế
    $$(m+n)^2-2n^2\equiv 4-2\equiv 2\pmod{4}.$$
    Không tồn tại số chính phương nào chia $4$ dư $2.$ Trường hợp này không xảy ra.
    \item Nếu $m,n$ khác tính chẵn lẻ, không mất tổng quát, ta giả sử $m$ chẵn và $n$ lẻ. Lúc này
    $$(m+n)^2-2n^2\equiv 1-2\equiv -1\equiv3\pmod{4}.$$
    Không tồn tại số chính phương nào chia $4$ dư $3.$ Trường hợp này cũng không xảy ra.    
\end{enumerate}
Như vậy, giả sử phản chứng ban đầu là sai. Bài toán được chứng minh.}
\end{gbtt}

\begin{gbtt}
Cho $d$ là một số nguyên dương khác $2,5,13.$ Chứng minh rằng ta có thể tìm được hai số nguyên phân biệt $a,b$ trong tập $\{2;5;13;d\}$ sao cho $ab-1$ không phải là số chính phương.
\loigiai{Vì $2\cdot 5-1=3^2,\ 2\cdot 13-1=5^2$ và $5\cdot 13-1=8^2$ nên hai số $a,b$ không thể là hai trong ba số $2,5,13$ trong tập hợp $\{2;5;13;d\}.$ Do đó, trong hai số $a,b$ có một số là $d$ và số kia là một trong ba số $2,5,13.$ Ta giả sử rằng cả $2d-1,5d-1,13d-1$ đều là số chính phương. Khi đó
$$2d-1\equiv 0,1,4,9\pmod{16}.$$
Từ đây ta suy ra $d\equiv 1,5,9,13\pmod{8}.$ Ta xét các trường hợp kể trên.
\begin{enumerate}
    \item Nếu $d\equiv 1\pmod{16}$ thì $13d-1\equiv 12\pmod{16},$ và khi ấy $13d-1$ không là số chính phương.
    \item Nếu $d\equiv 5\pmod{16}$ thì $5d-1\equiv 8\pmod{16},$ và khi ấy $5d-1$ không là số chính phương.
    \item Nếu $d\equiv 9\pmod{16}$ thì $5d-1\equiv12\pmod{16},$ và khi ấy $5d-1$ không là số chính phương.
    \item Nếu $d\equiv 13\pmod{16}$ thì $13d-1\equiv 8\pmod{16},$ và khi ấy $13d-1$ không là số chính phương.    
\end{enumerate}
Như vậy, giả sử phản chứng là sai, và bài toán được chứng minh.}
\end{gbtt}

\begin{gbtt}
Với mọi số nguyên dương $n,$ chứng minh rằng $$2020^{4n}+2021^{4n}+2022^{4n}+2023^{4n}$$ không phải là số chính phương.
\loigiai{
Xét trong hệ đồng dư modulo $4,$ ta có
\begin{align*}
    &2020^{4n}=(4\cdot505)^{4n}\equiv 0\pmod{4},
    \\&2021^{4n}=(4\cdot505+1)^{4n}\equiv 1\pmod{4},
    \\&2022^{4n}=(4\cdot505+2)^{4n}\equiv 2^{4n}=16^n\equiv 0\pmod{4},
    \\&2023^{4n}=(4\cdot506-1)^{4n}\equiv (-1)^{4n}=1\pmod{4}.    
\end{align*}
Các nhận xét trên chỉ ra cho ta
$$2020^{4n}+2021^{4n}+2022^{4n}+2023^{4n}\equiv 0+1+0+1 =2 \pmod{4}.$$
Một số chính phương chỉ có thể đồng dư với $0$ hoặc $1$ theo modulo $4.$ Như vậy $$2020^{4n}+2021^{4n}+2022^{4n}+2023^{4n}$$ không thể là số chính phương. Bài toán được chứng minh.}
\end{gbtt}

\begin{gbtt}
Tìm tất cả các cặp số tự nhiên $(m,n)$ sao cho $2^m3^n-1$ là số chính phương.
\nguon{Tạp chí Toán học và Tuổi trẻ, tháng 7 năm 2017}
\loigiai{
Trong bài toán này, ta xét các trường hợp sau.
\begin{enumerate}
    \item Với $m\ge 2,$ ta có $2^m3^n$ chia hết cho $4,$ và vì thế
    $$2^m3^n-1\equiv -1\equiv  3\pmod{4}.$$ 
    Không có số chính phương nào chia $4$ dư $3.$ Trường hợp này không xảy ra.
    \item Với $m=1,$ ta có $2\cdot 3^n-1$ là số chính phương. Nếu $n\ge 1$ thì
    $$2\cdot 3^n-1\equiv -1\equiv 2\pmod{3}.$$
    Không có số chính phương nào chia $3$ dư $2,$ thế nên $n=0.$ Thử với $m=1,\, n=0,$ ta thấy thoả mãn.
    \item Với $m=0,$ ta có $3^n-1$ là số chính phương. Nếu $n\ge 1$ thì
    $$3^n-1\equiv -1\equiv 2\pmod{3}.$$
    Không có số chính phương nào chia $3$ dư $2,$ thế nên $n=0.$ Thử với $m=n=0,$ ta thấy thoả mãn.    
\end{enumerate}
Như vậy có hai cặp $(m,n)$ thoả mãn đề bài là $(0,0)$ và $(1,0).$}
\end{gbtt}

\begin{gbtt}
Tìm các số nguyên dương $m,n$ thỏa mãn $A=6^m+2^n+2$ là số chính phương.
\nguon{Croatian Mathematical Olympiad 2009}
\loigiai{
Ta có nhận xét $6^{m}+2^{n}+2=2\left(3^{m}\cdot 2^{m-1}+2^{n-1}+1\right).$ Ta nghĩ đến việc xét các trường hợp sau.
\begin{enumerate}
    \item Với $m\ge 2$ và $n\ge 2,$ do cả $2^{m-1}$ và $2^{n-1}$ đều chẵn nên
    $$3^{m}\cdot 2^{m-1}+2^{n-1}+1\equiv 1\pmod{4}.$$
    Điều này giúp ta suy ra $A\equiv 2 \pmod{4}.$ Không có số chính phương nào đồng dư $2$ theo modulo $4,$ thế nên trường hợp này không xảy ra.
    \item Với $m=1,$ ta có $A=4\left(2^{n-2}+2\right)$ là số chính phương.
    \begin{itemize}
        \item Với $n=1,2,3,$ bằng kiểm tra trực tiếp, ta thấy chỉ có $n=3$ thỏa mãn.
        \item Với $n\ge 4,$ ta có $2^{n-2}+2$ là số chính phương chia cho $4$ dư $2.$ Điều này là không thể xảy ra.
    \end{itemize}
    \item Với $n=1,$ ta có $A=6^m+4$ là số chính phương. Ngoài ra, do
    $$6^m+4\equiv (-1)^m+4\equiv 3,5 \pmod{7}.$$
    nên $A$ đồng dư với $3$ hoặc $5$ theo modulo $7.$ \\Tuy nhiên, theo lí thuyết đã học, không có số chính phương nào như vậy.
\end{enumerate}
Vậy cặp số $(m, n)=(1,3)$ là cặp số duy nhất thỏa mãn đề bài.}
\end{gbtt}

\begin{gbtt}
Tìm tất cả các số nguyên tố $p$ thỏa mãn $2^p+5\cdot 3^p$ là số chính phương.
\loigiai{
Nếu $p=2,$ số đã cho chính phương. Nếu $p>2,$ ta đặt $p=2k+1$ và ta có
$$2^p+5\cdot3^p=2\cdot4^k+15\cdot9^k\equiv 3\pmod{4}.$$
Không có số chính phương nào chia $4$ dư $3.$ Kết quả bài toán chỉ có $p=2.$}
\end{gbtt}

\begin{gbtt}
Tìm tất cả các số nguyên tố $p$ thỏa mãn $7^p+9^p+259$ là số lập phương.

\loigiai{Bằng kiểm tra trực tiếp, ta chỉ ra $p=3$ là đáp số. Ta sẽ chứng minh mọi $p\ne 3$ không thỏa mãn đề bài.
\begin{enumerate}
    \item Với $p$ là số nguyên tố dạng $3k+1,$ ta có
    \begin{align*}
        7^p+9^p+259&=7^{3k+1}+9^{3k+1}+259\\&=7.342^k+9^{3k+1}+259\\&\equiv \ 7+0+7 \\&\equiv  5 \pmod{9}
    \end{align*}
    nên $7^p+9^p+259$ không là số lập phương.
    \item Với $p$ là số nguyên tố dạng $3k+2,$ ta có  
    \begin{align*}
        7^p+9^p+259&=7^{3k+2}+9^{3k+2}+259\\&=49.342^k+9^{3k+2}+259\\&\equiv \ 49+0+7 \\&\equiv  2 \pmod{9}
    \end{align*} 
    nên $7^p+9^p+259$ không là số lập phương.
    \end{enumerate}
Như vậy, $p=3$ là đáp số bài toán.}
\end{gbtt}

\section{Phương pháp kẹp lũy thừa}

\subsection*{Lí thuyết}
  Giữa hai lũy thừa số mũ $n$ liên tiếp, không tồn tại một lũy thừa cơ số $n$ nào. Hệ quả, với mọi số nguyên $a$ 
    \begin{enumerate}
        \item Không có số chính phương nào nằm giữa $a^2$ và $\left(a+1\right)^2.$
        \item Số chính phương duy nhất nằm giữa $a^2$ và $\left(a+2\right)^2$ là $\left(a+1\right)^2.$    
        \item Có $k-1$ số chính phương nằm giữa $a^2$ và $\left(a+k\right)^2,$ bao gồm \[\left(a+1\right)^2,\left(a+2\right)^2,\ldots,\left(a+k-1\right)^2.\]  
    \end{enumerate}
    Về các kết quả tương tự với số mũ khác, mời bạn đọc tự nghiên cứu và phát biểu.
\subsection*{Ví dụ minh họa}

\begin{bx}
Tìm tất cả các số nguyên dương $n$ sao cho $n^2+5n-5$ là số chính phương. 
\end{bx}
\nx{Muốn có thể sử dụng phương pháp kẹp lũy thừa, ta chọn hai tham số $a,b$ để cho
$$(n+a)^2\le n^2+5n-5\le(n+b)^2.$$
Do $(n+a)^2=n^2+2an+a^2$ và $(n+b)^2=n^2+2bn+b^2$ nên ta sẽ chọn $a,b$ sao cho khoảng cách giữa các số $2a,5,2b$ không quá lớn. Chẳng hạn, ta chọn $a=2,b=3.$ \\
Tương ứng với kiểu chọn tham số này, ta có hai cách làm như sau cho bài toán.\\}
\loigiai{
\begin{enumerate}[\sffamily\color{tuancolor}\bfseries Cách 1. ]
    \item  Ta xét các hiệu sau đây
\begin{align*}
    n^2+5n-5-(n+2)^2=n-9,\:
    (n+3)^2-\left(n^2+5n-5\right)=4n+4.
\end{align*}
Với $n\ge 10,$ cả hai hiệu trên đều dương, thế nên
$$(n+2)^2<n^2+5n-5<(n+3)^2.$$
Áp dụng phần lí thuyết đã biết, ta suy ra không tồn tại $n\ge 10$ sao cho $n^2+5n-5$ là số chính phương.
Như vậy $n\le 9.$ Thử trực tiếp với $1\le n\le 9,$ ta tìm được $n=1,\ n=2$ và $n=9.$ 
    \item  Ta xét các hiệu sau đây
\begin{align*}
    n^2+5n-5-(n-1)^2&=7n-6\ge 7-6=1> 0,
    \\(n+3)^2-\left(n^2+5n-5\right)&=4n+4\ge 4+4=8>0.
\end{align*}
Các hiệu trên đều dương, chứng tỏ rằng
$$(n-1)^2<n^2+5n-5<(n+3)^2.$$
Do $n^2+5n-5$ là số chính phương nên ta xét các trường hợp sau
\begin{itemize}
    \item\chu{Trường hợp 1.} Với $n^2+5n-5=n^2,$ ta có
    $n^2+5n-5=n^2,$ hay $n=1.$
    \item\chu{Trường hợp 2.} Với $n^2+5n-5=(n+1)^2,$ ta có
    $n^2+5n-5=n^2+2n+1,$ hay $n=2.$   
    \item\chu{Trường hợp 3.} Với $n^2+5n-5=(n+2)^2,$ ta có
    $n^2+5n-5=n^2+4n+4,$ hay $n=9.$
\end{itemize}
Như vậy, có $3$ giá trị của $n$ thỏa mãn đề bài là $n=1,\ n=2$ và $n=9.$
\end{enumerate}}

\begin{bx}
Tìm tất cả các số nguyên dương $n$ sao cho $n^4-9n^3+33n^2-63n+54$ là số chính phương.
\loigiai{Đặt $A=n^4-9n^3+33n^2-63n+54.$ Xét phân tích
$$A=(n-3)^2\left(n^2-3n+6\right).$$
Với $n=3,$ ta có $A=0$ là số chính phương. Còn với $n\ne 3,$ ta có $n^2-3n+6$ là số chính phương. \\
Trong trường hợp này, ta nhận xét được
    \begin{align*}
        n^2-3n+6-(n-2)^2&=n+2>0,
        \\(n+1)^2-\left(n^2-3n+6\right)&=5n-5\ge 5-5\ge 0.
    \end{align*}
    Các nhận xét trên cho ta
    $(n-2)^2<n^2-3n+6\le (n+1)^2.$ \\Do $n^2-3n+6$ là số chính phương, ta xét các trường hợp sau.
\begin{enumerate}
    \item Với $n^2-3n+6=(n-1)^2,$ ta có $n=5.$
    \item Với $n^2-3n+6=n^2,$ ta có $n=2.$ 
    \item Với $n^2-3n+6=(n+1)^2,$ ta có $n=1.$ 
\end{enumerate}
Như vậy, có $4$ giá trị của $n$ thỏa mãn đề bài là $n=1,\ n=2,\ n=3$ và $n=5.$}
\end{bx}

\begin{bx}
Tìm tất cả các số nguyên $n$ sao cho $A=n^3-2n^2+5n+25$ là một số lập phương.
\end{bx}
\nx{Với điều kiện $n$ nguyên, khoảng kẹp của chúng ta cần phải rộng hơn một chút. Tương tự như các bài kẹp số chính phương, ta sẽ chọn các số thực $a,b$ sao cho
$$(n+a)^3\le n^3-2n^2+5n+25\le(n+b)^3.$$
Điều này hướng ta nghĩ đến việc xét các hiệu sau đây
\begin{align*}
    (n+b)^3-\tron{n^3-2n^2+5n+25}&=\tron{3b+2}n^2+\tron{3b^2-5}n+\tron{b^3-25},\\
    \tron{n^3-2n^2+5n+25}-(n+a)^3&=\tron{-3a-2}n^2+\tron{-3a^2+5}n+\tron{-a^3+25}.
\end{align*}
Muốn $\tron{3b+2}n^2+\tron{3b^2-5}n+\tron{b^3-25}$ luôn nhận giá trị không âm với mọi $n$ nguyên, ta cần phải có
$$\heva{&3b+2\ge 0\\&\tron{3b^2-5}^2-4\tron{3b+2}\tron{b^3-25}\le 0.}$$
Do $3b+2\ge 0,$ ta sẽ thử các giá trị $b=0,1,2,\cdots$ rồi kiểm tra xem $$\tron{3b^2-5}^2-4\tron{3b+2}\tron{b^3-25}\le 0$$ kể từ khi $b$ bằng bao nhiêu. Kết quả ở đây là $b=4.$ Một cách tương tự, số $a$ cũng phải thỏa mãn
$$\heva{&3a+2\le 0\\&\tron{3a^2-5}^2-4\tron{3a+2}\tron{a^3-25}\le 0.}$$
Bằng cách thử $a=-1,-2,-3,\cdots,$ ta thấy ngay $a=-1$ thỏa mãn. Dưới đây là lời giải cho bài toán.}
\loigiai{
Ta xét các hiệu sau đây
\begin{align*}
    n^3-2n^2+5n+25-(n-1)^3&=n^2+2n+26\\&=(n+1)^2+25,\\
    (n+4)^3-\tron{n^3-2n^2+5n+25}&=14n^2+43n+39\\&=\dfrac{1}{56}\tron{28n+43}^2+\dfrac{335}{56}.
\end{align*}
Các hiệu trên đều dương, chứng tỏ
$$(n-1)^3<n^3-2n^2+5n+25<(n+4)^3.$$
Do $n^3-2n^2+5n+25$ là số lập phương, ta xét các trường hợp sau.
\begin{enumerate}
    \item Với $n^3-2n^2+5n+25=n^3,$ ta có $(n-5)(2n+5)=0.$ Do $n$ nguyên nên $n=5.$
    \item Với $n^3-2n^2+5n+25=(n+1)^3,$ ta có $(n+2)(5n-12)=0.$ Do $n$ nguyên nên $n=-2.$
    \item Với $n^3-2n^2+5n+25=(n+2)^3,$ ta có $8n^2+7n-17=0.$ Ta không tìm được $n$ nguyên từ đây.
    \item Với $n^3-2n^2+5n+25=(n+3)^3,$ ta có $11n^2+22n+2=0.$ Ta không tìm được $n$ nguyên từ đây.     
\end{enumerate}
Như vậy, $n=-2$ và $n=5$ là tất cả các giá trị của $n$ thỏa mãn đề bài.}


\begin{bx}
Tìm tất cả các số nguyên dương $n$ sao cho $n^4+4n^3-3n^2-n+3$ là số chính phương.
\end{bx}
\nx{Ta không thể phân tích $n^4+4n^3-3n^2-n+3$ thành các đa thức nhân tử hệ số nguyên. Do đó, trong bài toán này, ta nghĩ đến việc chọn các tham số $a,b,c,d$ để chắc chắn rằng
\[\left(n^2+an+b\right)^2\le n^4+4n^3-11n^2-n+11\le\left(n^2+cn+d\right)^2.\tag{*}\]
Đối với những bài toán kẹp bậc bốn như trên, ta nên cố gắng kẹp sao cho \chu{làm mất hệ số bậc ba}. Theo đó, do hệ số bậc ba trong đa thức ở các vế của (*) lần lượt là $2a,4,2c$ nên ta chọn $a=c=2.$ \\
Đặt biểu thức đã cho là $A.$ Tiếp đó, ta xét các hiệu sau
\begin{align*}
    A-\left(n^2+2n+b\right)^2&=(-7-2b)n^2+(-4b-1)n+\left(-b^2+3\right),
    \\\left(n^2+2n+d\right)^2-A&=(7+2d)n^2+(4d+1)n+\left(d^2-3\right).
\end{align*}
Hệ số bậc hai trong các hiệu trên phải dương. Các hệ số bậc hai cũng phải dương để đánh giá được tốt, thế nên để phần kẹp của chúng ta được "sát" nhất, ta nên chọn $b=-4$ và $d=0.$}
\loigiai{Đặt $A=n^4+4n^3-3n^2-n+3.$ Ta nhận xét được rằng là
\begin{align*}
A-\left(n^2+2n-4\right)^2&=n^2+15n-13\ge 1+15-13>0,
\\(n^2+2n)^2-A&=7n^2+n-3\ge 7+1-3>0.
\end{align*}
Các hiệu trên đều dương, chứng tỏ
$$\left(n^2+2n-4\right)^2<A<(n^2+2n)^2.$$
Do $A$ là số chính phương, ta xét các trường hợp sau.
\begin{enumerate}
    \item Với $A=\left(n^2+2n-3\right)^2,$ ta có 
    $$n^4+4n^3-3n^2-n+3=n^4+4n^3-2n^2-12n+9\Leftrightarrow n^2-11n-6=0.$$
    Phương trình trên vô nghiệm nguyên do nó có $\Delta=11^2+4.6=145$ không chính phương.
    \item Với $A=\left(n^2+2n-2\right)^2,$ ta có 
    $$n^4+4n^3-3n^2-n+3=n^4+4n^3-8n+4\Leftrightarrow 3n^2-7n+1=0.$$
    Phương trình trên vô nghiệm nguyên do nó có $\Delta=7^2-4.3=37$ không chính phương.   
    \item Với $A=\left(n^2+2n-1\right)^2,$ ta có 
    \begin{align*}
        n^4+4n^3-3n^2-n+3=n^4+4n^3+2n^2-4n+1
        &\Leftrightarrow 5n^2-3n-2=0.
    \end{align*}
    Do $n$ nguyên, ta chọn $n=1.$    
\end{enumerate}
Như vậy, $n=1$ là giá trị duy nhất của $n$ thỏa mãn đề bài.}

\begin{bx}
Tìm tất cả các số nguyên $n$ sao cho $n^4+n^3+1$ là số chính phương.
\loigiai{Từ giả thiết, ta suy ra $4n^4+4n^3+4$ cũng là số chính phương. Với mọi số nguyên $n,$ ta có
\begin{align*}
    4n^4+4n^3+4-\left(2n^2+n-1\right)^2&=3n^2+2n+3>0,
    \\\left(2n^2+n+3\right)^2-\left(4n^4+4n^3+4\right)&=13n^2+6n+5>0.
\end{align*}
Các đánh giá theo hiệu trên cho ta
$$\left(2n^2+n-1\right)^2<4n^4+4n^3+4<\left(2n^2+n+3\right)^2.$$
Do $4n^4+4n^3+4$ là số chính phương nên ta xét các trường hợp sau.
\begin{enumerate}
    \item Với $4n^4+4n^3+4=\left(2n^2+n\right)^2,$ ta có
    $$4n^4+4n^3+4=4n^4+4n^3+n^2\Leftrightarrow n^2=4\Leftrightarrow n=\pm 2.$$
    \item Với $4n^4+4n^3+4=\left(2n^2+n+1\right)^2,$ ta có
    \begin{align*}
        4n^4+4n^3+4=4n^4+4n^3+5n^2+2n+1&\Leftrightarrow 5n^2+2n-3=0\Leftrightarrow (n+1)(5n-3)=0.
    \end{align*}
    Do $n$ nguyên, ta chọn $n=1.$
    \item Với $4n^4+4n^3+4=\left(2n^2+n+2\right)^2,$ ta có
    $$4n^4+4n^3+4=4n^4+4n^3+9n^2+4n+4\Leftrightarrow 9n^2+4n=0\Leftrightarrow n(9n+4)=0.$$
    Do $n$ nguyên, ta chọn $n=0.$    
\end{enumerate}
Như vậy, có $4$ giá trị của $n$ thỏa mãn đề bài là $n=-2,n=0,n=1$ và $n=2.$}
\end{bx}

\begin{bx}
Tìm các số nguyên $n$ thỏa mãn $9n+16$ và $16n+9$ đều là các số chính phương.
\nguon{Titu Andreescu}
\loigiai{Do $16n+9$ là số chính phương nên $16n+9\ge 0$ hay $n\ge 0.$ Ta có 
$$(9 n+16)(16n+9)=144n^{2}+337n+144$$
cũng là số chính phương. Bởi vì
$$(12 n+12)^{2} \leq 144n^{2}+337n+144<(12 n+15)^{2}.$$
nên $144n^{2}+337n+144$ bằng $(12 n+12)^{2},(12 n+13)^{2},$ hoặc $(12 n+14)^{2}$.\\ Kiểm tra trực tiếp, ta thu được $n=0,\ n=1$ và $n=52$ là các giá trị thỏa mãn đề bài của $n.$}
\end{bx}

%nguyệt anh
\begin{bx}
Tìm tất cả các số nguyên dương $n$ sao cho $n+2$ và $n^2-n-3$ là số lập phương.
\loigiai{
Giả sử rằng tồn tại số nguyên dương $n$ thỏa mãn đề bài. Ta dễ dàng chỉ ra $n\ge 3.$\\ Rõ ràng $(n+2)(n^2-n-3)=n^3+n^2-5n-6$ cũng là số lập phương. Ta có đánh giá
$$(n-2)^3< n^3+n^2-5n-6<(n+1)^3.$$
với mọi $n\ge 3.$ Đánh giá trên cho ta $n^3+n^2-5n-6$ bằng $(n-1)^{3}$ hoặc $n^{3}.$\\ Kiểm tra trực tiếp, ta thu được $n=6$ là giá trị thỏa mãn đề bài của $n.$}
\end{bx}

\begin{bx}
Tìm tất cả các số nguyên dương $x,y$ thỏa mãn
\[x^3+3x^2+2x+9=y^3+2y.\]
\loigiai{
Giả sử tồn tại các số $x,y$ thỏa yêu cầu. Ta xét các hiệu sau đây
\begin{align*}
    \tron{x^3+3x^2+2x+9}-\tron{x^3+2x}&=3x^2+9,\\
    \tron{(x+2)^3+2(x+2)}-\tron{x^3+3x^2+2x+9}&=3x^2+12x+3.
\end{align*}
Các hiệu trên đều dương, chứng tỏ
$$x^3+2x<x^3+3x^2+2x+9<(x+2)^3+2(x+2).$$
Do $x^3+3x^2+2x+9=y^3+2y$ nên \[x^3+2x<y^3+2y<(x+2)^3+2(x+2).\tag{*}\label{kepdongdangne}\]
Ta sẽ chứng minh $y=x+1.$ Thật vậy, cả hai trường hợp $y\ge x+2$ và $y\le x$ đều cho ta mâu thuẫn với (\ref{kepdongdangne}). Với $y=x+1,$ thay trở lại phương trình ban đầu, ta được
\[x^3+3x^2+2x+9=(x+1)^3+2(x+1).\]
Ta tìm ra $x=2$ từ đây, và đáp số bài toán là $(x,y)=(2,3).$}
\begin{luuy}
Trong bài toán trên, ta đã xác định hai số $a,b$ sao cho
$$(x+a)^3+2(x+a)<x^3+3x^2+2x+9<(x+b)^3+2(x+b).$$
Phương pháp kẹp như trên được gọi là phương pháp \chu{kẹp đồng dạng}.
\end{luuy}
\end{bx}

\begin{bx}
Cho $x, y$ là những số nguyên lớn hơn 1 sao cho $4x^2y^2-7x+7y$ là số chính
phương. Chứng minh rằng ${x}={y}$.
\nguon{Chuyên Khoa học Tự nhiên 2014}
\loigiai{
Đặt $A=4 x^{2} y^{2}-7 x+7 y.$ Ta xét các hiệu sau
\begin{align*}
&A-(2 x y-1)^{2}=4 x y-7 x+7 y-1,
\\&(2 x y+1)^{2}-A=4 x y+7 x-7 y+1.
\end{align*}
Ta sẽ chứng minh các hiệu trên đều dương. Từ giả thiết, ta có $x\ge 2$ và $y\ge 2.$ Hai đánh giá này cho ta 
\begin{align*}
4xy-7x+7y-1&=(4y-7)x+6y+(y-1)>0,
\\4xy+7x-7y+1&=(4x-7)y+7x+1>0.
\end{align*}
Như vậy, $(2 {xy}-1)^{2}<{A}<(2 {xy}+1)^{2}.$ Ta có $A=4x^2y^2,$ tức là $x=y.$ Bài toán được chứng minh.}
\end{bx}

\begin{bx}
Tìm các cặp số nguyên dương $(m, n)$ sao cho $m^2+5n$ và $n^2+5m$ đều là số chính phương.
\nguon{Titu Andreescu}
\loigiai{Không mất tính tổng quát, ta giả sử $m\le n.$ Khi đó
$$
n^{2}<n^{2}+5 m\le n^{2}+5 n<(n+3)^{2} .
$$
Đánh giá trên hướng ta đến việc xét các trường hợp sau.
\begin{enumerate}
    \item Nếu $n^2+5m=(n+1)^2$ hay $5m=2n+1,$ ta chứng minh được $n$ chia $5$ dư $2.$ Thật vậy
    $$2n\equiv -1\pmod{5}\Rightarrow 6n\equiv -3\pmod{5}\Rightarrow n\equiv 2\pmod{5}.$$
    Đặt $n=5k+2$ thì $m=2k+1.$ Khi đó $m^{2}+5 n=(2 k+1)^{2}+5(5 k+2)$ là số chính phương. Ta có
    $$(2 k+4)^{2}<(2 k+1)^{2}+5(5 k+2)=4 k^{2}+29 k+11<(2 k+8)^{2}$$
    nên $(2 k+1)^{2}+5(5 k+2)$ nhận một trong các giá trị
    $$(2k+5)^2,\quad (2k+6)^2,\quad (2k+7)^2.$$
    Ta tìm ra $k=5$ và $k=38.$ Trường hợp này cho ta các cặp $(m,n)=(11,27)$ và $(m,n)=(77,192).$
    \item Nếu $n^2+5m=(n+2)^2$ hay $5m=4n+4,$ ta chứng minh được $n$ chia $5$ dư $4.$ Thật vậy
    $$4n\equiv -4\pmod{5}\Rightarrow -n\equiv -4\pmod{5}\Rightarrow n\equiv 4\pmod{5}.$$    
    Đặt $n=5k-1$ thì $m=4k.$ Khi đó $m^{2}+5n=(4k)^{2}+5(5k-1)$ là số chính phương. Ta có
    $$(4k+1)^{2}<(4k)^{2}+5(5k-1)=(4k)^{2}+5(5k-1)<(4k+4)^{2}$$    
    nên $(4k)^{2}+5(5k-1)$ nhận một trong các giá trị
    $$(4k+2)^2,\quad (4k+3)^2.$$    
    Ta tìm ra $k=1$ và $k=14.$ Trường hợp này cho ta các cặp $(m,n)=(4,4)$ và $(m,n)=(56,69).$
\end{enumerate}
Kết luận, có tất cả cả $7$ cặp $(m,n)$ thỏa yêu cầu là
$$(11,27),\ (27,11),\ (77,192),\ (192,77),\ (4,4),\ (56,69),\ (69,56).$$}
\end{bx}

\begin{bx}
Tìm tất cả các bộ số $\left(a, b, c\right)$ là ba cạnh một tam giác thỏa mãn $$a^{2}-3 a+b+c,\quad  b^{2}-3 b+c+a, \quad c^{2}-3 c+a+b$$ đều là các số chính phương. 
\nguon{Titu Andreescu}
\loigiai{Giả sử tồn tại bộ ba $(a,b,c)$ thỏa mãn đề bài. \\
Theo bất đẳng thức tam giác, $b+c>a$, và do $a,b,c$ nguyên nên $b+c \ge a+1.$ Ta có 
$$
a^{2}-3 a+b+c \geq a^{2}-2 a+1=(a-1)^{2}
$$
Không mất tổng quát, ta giả sử $a\ge b\ge c.$ Giả sử này cho ta 
$$(a-1)^{2} \leq a^{2}-3 a+b+c \leq a^{2}-a<a^{2},$$
và như vậy, $a^{2}-3 a+b+c$ chỉ có thể là số chính phương nếu $b+c=a+1$. \\
Tiếp theo, ta cũng có $b^2-3b+c+a=(b-1)^{2}+2(c-1)$ là số chính phương, thế nhưng do
$$(b-1)^2\le(b-1)^{2}+2(c-1)<b^2-1<b^2$$
nên rõ ràng $c=1.$ Kết hợp với đẳng thức $b+c=a+1,$ ta chỉ ra $a=b.$ \\
Cuối cùng, ta có $c^2-3c+a+b=2(a-1)$ là số chính phương, thế nên tồn tại số tự nhiên $m$ sao cho $a=2m^2+1.$ Kiểm tra trực tiếp các bộ số $\left(2 m^{2}+1,2 m^{2}+1,1\right),$ ta thấy chúng đều thỏa mãn đề bài. Đây chính là kết quả bài toán.}
\end{bx}

\begin{bx}
Cho các số nguyên dương $x,y.$ Chứng minh rằng nếu $x^2+2y$ là một số chính phương thì $x^2+y$ biểu diễn được thành tổng của hai số chính phương.
\loigiai{
Do $y$ là số nguyên dương, ta có $x^{2}+2y>x^2$. Theo đó, ta có thể đặt $x^{2}+2 y=(x+t)^{2},$ ở đây $t$ là số nguyên dương. Xét biến đổi sau
$$x^{2}+2 y=(x+t)^{2}\Leftrightarrow x^{2}+2 y=x^{2}+2 x t+t^{2} \Leftrightarrow t^{2}=2(y-x t).$$
Biến đổi trên cho ta $t$ chẵn, đồng thời $y-xt$ cũng là số chẵn. Ta tiếp tục đặt ${t}=2{m},$ với $m$ nguyên dương.\\
Phép đặt này cho ta 
$2y=t^2+2xt=4m^2+4mx,$
và như vậy
$$x^2+y=x^2+\dfrac{4m^2+4mx}{2}=x^2+2mx+2m^2=\left(x+m\right)^2+m^2.$$
Ta biểu diễn được $x^{2}+y$ thành tổng của hai số chính phương. Chứng minh hoàn tất.}
\end{bx}

\begin{bx}
Tìm tất cả các số tự nhiên $n$ sao cho $n^2+7n+4$ là lũy thừa số mũ tự nhiên của $2.$
\loigiai{
Ta đặt $n^2+7n+4=2^m, $ trong đó $m$ là số tự nhiên. Với mọi số tự nhiên $n,$ ta có nhận xét
$$n^2+7n+4\equiv 0,1\pmod{3}.$$
Do $2^m$ không thể chia hết cho $3,$ chỉ trường hợp $2^m$ chia $3$ dư $1$ là thỏa mãn, và ngoài ra, trường hợp này còn cho ta $m$ chẵn. Theo đó, $n^2+7n+4$ là số chính phương. Với mọi số tự nhiên $n$ thì
$$(n+2)^2\le n^2+7n+4<(n+4)^2.$$
Ta suy ra $n^2+7n+4$ hoặc bằng $(n+2)^2,$ hoặc bằng $(n+3)^2.$\\ Các trường hợp này cho ta các kết quả là $n=0$ và $n=5.$}
\end{bx}

\begin{bx}
Tìm tất cả các số nguyên dương $n$ sao cho $3^{2n}+3 n^2+7$ là một số chính phương.
\loigiai{Ta đặt $3^{2n}+3 n^2+7=m^2,$ ở đây $m$ là một số nguyên dương. \\
Do $m^2>3^{2n}$ và $3^{2n}+3 n^2+7=m^2$ là số chính phương, ta suy ra
$$
3^{2n}+3n^2+7 \geq\left(3^n+1\right)^2=3^{2 n}+2\cdot3^n+1.
$$
Đánh giá trên cho ta 
$2\cdot3^{n} \leq 3 n^{2}+6.$
Với $n\ge 3,$ bất đẳng thức trên đảo chiều. Theo đó, ta cần chứng minh bất đẳng thức sau với mọi $n\ge 3$ bằng phương pháp quy nạp.
\[2\cdot3^n>3 n^2+6.\tag{*}\]
Hiển nhiên (*) đúng với $n=3.$ Giả sử (*) đúng với $n=3,4,\ldots,k,$ thế thì
$$2\cdot3^{k+1}=3\cdot 2\cdot 3^k>3\left(3k^2+6\right)>3(k+1)^2+6.$$
Theo nguyên lí quy nạp, (*) được chứng minh với mọi $n\ge 3.$ Điều này đồng nghĩa với việc chỉ tồn các trường hợp $n=1,n=2$ và $n=3.$ Thử trực tiếp, ta thấy $n=2$ là giá trị duy nhất thỏa mãn đề bài.} 
\begin{luuy}
\nx{Thông thường, khi $n$ đủ lớn, hàm số mũ sẽ lớn hơn hàm đa thức. Nguyên lí này chính là cơ sở để ta nghĩ đến việc chứng minh bất đẳng thức (*) như bài trên.}
\end{luuy}
\end{bx}

\begin{bx}
Tìm tất cả các số nguyên dương $n$ sao cho $7^n+8n+67$ là số chính phương.
\loigiai{Nếu $n$ là số lẻ, ta có đánh giá đồng dư sau đây
$$7^n+8n+67\equiv (-1)^n+3\equiv 2\pmod{4}.$$
Không có số chính phương nào đồng dư $2$ theo modulo $4,$ chứng tỏ $n$ chẵn. Ta đặt $n=2m,$ khi đó
$$7^n+8n+67=7^{2m}+16m+67.$$
Do $7^{2m}+16m+67>7^{2m}$ và $7^{2m}+16m+67$ là số chính phương, ta suy ra
$$
7^{2m}+16m+67 \geq\left(7^m+1\right)^2=7^{2 m}+2\cdot 7^m+1.
$$
Đánh giá trên cho ta 
$2\cdot7^m \leq 16m+66,$
hay là
$7^m\le 8m+33.$
Với $m\ge 2,$ bất đẳng thức trên đảo chiều. Theo đó, ta cần chứng minh bất đẳng thức sau với mọi $m\ge 2$ bằng phương pháp quy nạp
\[7^m>8m+8.\tag{*}\]
Hiển nhiên (*) đúng với $m=2.$ Giả sử (*) đúng với $m=2,3,4,\ldots,k,$ thế thì
$$7^{k+1}=7\cdot7^k>7(8k+33)>8(k+1)+33.$$
Theo nguyên lí quy nạp, (*) được chứng minh với mọi $m\ge 2.$ Điều này đồng nghĩa với việc chỉ tồn các trường hợp $n=1$ và $n=2.$ Thử trực tiếp, ta thấy $m=2,$ và $n=2m=4$ là giá trị duy nhất thỏa mãn đề bài.} 
\end{bx}


\subsection*{Bài tập tự luyện}

\begin{btt}
Tìm tất cả các số nguyên $n$ sao cho $8n^3+7n^2+2n+4$ là số lập phương.
\end{btt}

\begin{btt}
Tìm tất cả các số nguyên dương $n$ sao cho $n^4+3n^3+3n^2+7$ là số chính phương.
\end{btt}

\begin{btt}
Tồn tại hay không số nguyên dương $n$ thỏa mãn $3^{6 n-3}+3^{3 n-1}+1$ là số lập phương?
\nguon{Titu Andreescu}
\end{btt}

\begin{btt}
Tìm tất cả các số nguyên dương $n$ sao cho
$$A=4\tron{1+2+\cdots+n}^2+3n^3-10n+1$$
là lũy thừa bậc bốn của một số tự nhiên.
\end{btt}

\begin{btt}
Tìm tất cả các số nguyên dương $n$ sao cho
$$A=24\tron{1^2+2^2+\cdots+n^2}+n+6$$
là một số lập phương.
\end{btt}

\begin{btt}
Tìm tất cả các cặp số nguyên dương $m,n$ thỏa mãn
\[m^6+5n^2=m+n^3.\]
\end{btt}


\begin{btt}
Tìm tất cả các số tự nhiên $n$ sao cho $2n+7$ và $18n+22$ là số chính phương.
\end{btt}

\begin{btt}
Tìm tất cả các số nguyên dương $n$ sao cho $n+1$ và $n^3+n^2-2n+6$ là số chính phương.
\end{btt}

\begin{btt}
Tìm các số nguyên dương $x,y$ sao cho $4x^2+9y+3$ và $4y^2+9x+3$ là số chính phương.
\end{btt}

\begin{btt}
Tìm tất cả các bộ ba số nguyên dương $(a,b,c)$ sao cho $(a+b+c)^2-2a+2b$ là số chính phương.
\nguon{Chuyên Toán Vĩnh Phúc 2021}
\end{btt}

\begin{btt}
Tìm tất cả các số nguyên dương $x,y$ thỏa mãn $x^2y^4-y^3+1$ là số chính phương.
\end{btt}

\begin{btt}
Tìm tất cả các số nguyên dương $a,b$ sao cho $a^3b^3+4a^2-3b$ là số lập phương.
\end{btt}

\begin{btt}
Tìm tất cả các số nguyên dương $a, b, c$ sao cho cả ba số
$$4a^2+5b,\quad 4b^2+5c,\quad 4c^2+5a$$ 
đều là số chính phương.
\nguon{Chuyên Khoa học Tự nhiên 2020}
\end{btt}

\begin{btt}
Tìm tất cả các bộ số tự nhiên \(\left ( a,b,c \right )\) thỏa mãn
\[a^2+2b+c,\quad b^2+2c+a,\quad c^2+2a+b\]
đều là các số chính phương.
\end{btt}

\begin{btt}
Tìm tất cả các số tự nhiên $n$ sao cho $n^4+3n^3+n^2+5$ là lũy thừa cơ số $7$ của một số tự nhiên.
\end{btt}

\begin{btt}
Tìm tất cả các số tự nhiên $n$ sao cho $13^n+7n+13$ là số chính phương.
\end{btt}

\begin{btt}
Tìm tất cả các số tự nhiên $n$ sao cho $4^n+3n+7$ là số lập phương.
\end{btt}

\begin{btt}
Cho $x, y$ là các số nguyên dương. Chứng minh $x^2+y+1$ và $y^2+4x+3$ không đồng thời là số chính phương.
\end{btt}

\begin{btt}
Cho các số nguyên dương $x,y.$ Chứng minh rằng $x^3+y^2+5x+2$ và $y^3+xy+y^2+3$ không cùng là số lập phương.
\end{btt}

\begin{btt}
Tìm tất các các số nguyên tố $p$ sao cho tổng các ước nguyên dương của $p^4$ là số chính phương.
\end{btt}

\begin{btt}
Cho \(n\) là một số nguyên dương. Tìm tất cả các ước nguyên dương \(d\) của \(3n^2\) thỏa mãn \(n^2+d\) là bình phương của một số nguyên.
\end{btt}

\begin{btt}
Tìm tất cả các số nguyên dương $a$ thỏa mãn với mọi số nguyên dương $n,$ ta có $4\left(a^n+1\right)$ là số lập phương.
\nguon{Iran Team Selection Test 2008}
\end{btt}

\begin{btt}
Tìm tất cả các số nguyên dương $n$ sao cho $n^4+8n+11$ có thể viết được thành tích ít nhất hai số nguyên dương liên tiếp.
\nguon{Junior Balkan Mathematical Olympiad 2008}
\end{btt}

\begin{btt}
Với mỗi số thực $a$ ta gọi phần nguyên của $a$ là số nguyên lớn nhất không vượt quá $a$ và ký hiệu là $[{a}]$. Chứng minh rằng vói mọi số nguyên dương ${n}$, biểu thức
$${n}+\left[\sqrt[3]{{n}-\dfrac{1}{27}}+\dfrac{1}{3}\right]^{2}$$ không biểu diễn được dưới dạng lập phương của một số nguyên dương.
\nguon{Chuyên Khoa học Tự nhiên 2011}
\end{btt}

\begin{btt}
Tìm tất cả các số nguyên dương $x,y$ thỏa mãn $4^x+4^y+1$ là một số chính phương.
\nguon{Korean Mathematical Olympiad 2007}
\end{btt}

\begin{btt}
Cho $x$ là một số thực thỏa mãn $4x^5-7$ và $4x^{13}-7$ đều là các số chính phương.
\begin{enumerate}[a,]
    \item Chứng minh rằng $x$ là số nguyên dương.
    \item Tìm tất cả các giá trị có thể của $x.$
\end{enumerate}
\nguon{Trại hè Hùng Vương 2018,
German Mathematical Olympiad 2018}
\end{btt}

\subsection*{Hướng dẫn bài tập tự luyện}

\begin{gbtt}
Tìm tất cả các số nguyên $n$ sao cho $8n^3+7n^2+2n+4$ là số lập phương.
\loigiai{
Ta xét các hiệu sau đây
\begin{align*}
    8n^3+7n^2+2n+4-(2n)^3&=7n^2+2n+4=7\tron{x+\dfrac{1}{7}}^2+\dfrac{27}{7},\\
    (2n+4)^3-\tron{8n^3+7n^2+2n+4}&=41\tron{x+\dfrac{47}{41}}^2+\dfrac{251}{41}.
\end{align*}
Các hiệu trên đều dương, chứng tỏ
$$(2n)^3<8n^3+7n^2+2n+4<(2n+4)^3.$$
Do $8n^3+7n^2+2n+4$ là số lập phương, ta xét các trường hợp sau.
\begin{enumerate}
    \item Với $8n^3+7n^2+2n+4=(2n+1)^3,$ ta có $(5n+4)n=3.$ Ta không tìm được $n$ nguyên từ đây.
    \item Với $8n^3+7n^2+2n+4=(2n+2)^3,$ ta có $17n^2+22n+4=0.$ Ta không tìm được $n$ nguyên từ đây.
    \item Với $8n^3+7n^2+2n+4=(2n+3)^3,$ ta có $(n+1)(29n+23)=0.$ Do $n$ nguyên nên $n=-1.$
\end{enumerate}
Như vậy, $n=-1$ là giá trị duy nhất của $n$ thỏa yêu cầu.}
\end{gbtt}

\begin{gbtt}
Tìm tất cả các số nguyên dương $n$ sao cho $n^4+3n^3+3n^2+7$ là số chính phương.
\loigiai{
Ta xét các hiệu sau đây
\begin{align*}
    4\tron{n^4+3n^3+3n^2+7}-\tron{2n^2+3n}^2&=3n^2+28,\\
    \tron{2n^2+3n+3}^2-4\tron{n^4+3n^3+3n^2+7}&=9n^2+18n-19\\&=8n^2+\tron{n^2-1}+18\tron{n-1}.
\end{align*}
Các hiệu trên đều dương, chứng tỏ
$$\tron{2n^2+3n}^2<4\tron{n^4+3n^3+3n^2+7}<\tron{2n^2+3n+3}^2.$$
Do $4\tron{n^4+n^3+3n^2+7}$ là số chính phương, ta xét các trường hợp sau.
\begin{enumerate}
    \item Với $4\tron{n^4+3n^3+3n^2+7}=(2n^2+3n+1)^2,$ ta có $(n-3)(n+9)=0,$ và $n=3.$
    \item Với $4\tron{n^4+3n^3+3n^2+7}=(2n^2+3n+2)^2,$ ta có $$5n^2+12n-24=0.$$ Ta không tìm được $n$ nguyên dương từ đây.
\end{enumerate}
Như vậy, $n=3$ là giá trị duy nhất của $n$ thỏa yêu cầu.}
\end{gbtt}

\begin{gbtt}
Tồn tại hay không số nguyên dương $n$ thỏa mãn $3^{6n-3}+3^{3n-1}+1$ là số lập phương?
\nguon{Titu Andreescu}
\loigiai{
Ta đặt $3^{n-1}=x,$ khi đó
$3^{6 n-3}+3^{3 n-1}+1=27x^6+9x^3+1.$
Bằng tính toán trực tiếp, ta chỉ ra
$$
\left(3x^2+1\right)^3>27x^6+9x^3+1>\left(3x^2\right)^3.
$$
Theo lí thuyết đã học, $27x^6+9x^3+1$ không là số lập phương. Câu trả lời của bài toán là phủ định.}
\end{gbtt}

\begin{gbtt}
Tìm tất cả các số nguyên dương $n$ sao cho
$$A=4\tron{1+2+\cdots+n}^2+3n^3-10n+1$$
là lũy thừa bậc bốn của một số tự nhiên.

\loigiai{
Trước tiên, ta tính được
\begin{align*}
    4\tron{1+2+\cdots+n}^2+3n^3-10n+1&=4\tron{\dfrac{n(n+1)}{2}}^2+3n^3-10n+1
    \\&=n^2(n+1)^2+3n^3-10n+1
    \\&=n^4+5n^3+n^2-10n+1.
\end{align*}
Do $A$ dương nên $n^4+5n^3+n^2-10n+1>0,$ và ta suy ra $n\ge 2.$ Với mọi $n\ge 2,$ ta có
$$\tron{2n^2+5n-6}^2<4\tron{n^4+5n^3+n^2-10n+1}<\tron{2n^2+5n-4}^2.$$
Do $4\tron{n^4+5n^3+n^2-10n+1}$ là số chính phương nên ta có $$4\tron{n^4+5n^3+n^2-10n+1}=\tron{2n^2+5n-5}^2.$$ Ta tìm được $n=7$ và $n=3$ từ đây. Thử trực tiếp, chỉ trường hợp $n=7$ cho $A$ là lũy thừa mũ bốn của một số nguyên, và đây là đáp số bài toán.}
\end{gbtt}

\begin{gbtt}
Tìm tất cả các số nguyên dương $n$ sao cho
$$A=24\tron{1^2+2^2+\cdots+n^2}+n+6$$
là một số lập phương.

\loigiai{
Trước hết, ta sẽ tìm cách tính tổng $1^2+2^2+\cdots+n^2.$ Ta đã biết
$$k^2=\tron{\dfrac{1}{3}(k+1)^3-\dfrac{1}{2}(k+1)^2+\dfrac{1}{6}(k+1)}-\tron{\dfrac{1}{3}k^3-\dfrac{1}{2}k^2+\dfrac{1}{6}k},$$
với mọi số nguyên dương $k.$ Vì thế, ta tính được
\begin{align*}
    1^2+2^2+\cdots+n^2
    &=\tron{\dfrac{1}{3}(n+1)^3-\dfrac{1}{2}(n+1)^2+\dfrac{1}{6}(n+1)}-\tron{\dfrac{1}{3}-\dfrac{1}{2}+\dfrac{1}{6}}
    \\&=\dfrac{n(n+1)(2n+1)}{6}.
\end{align*}
Như vậy $A=4n(n+1)(2n+1)+n+6=8n^3+12n^2+5n+6.$ Với mọi $n\ge 1,$ ta có nhận xét
$$(2n)^3<8n^3+12n^2+5n+6<(2n+2)^3.$$
Do $8n^3+12n^2+5n+6$ là số lập phương nên $8n^3+12n^2+5n+6=(2n+1)^3,$ hay là $n=5.$}
\end{gbtt}

\begin{gbtt}
Tìm tất cả các cặp số nguyên dương $m,n$ thỏa mãn
\[m^6+5n^2=m+n^3.\]
\loigiai{
Giả sử tồn tại các số $m,n$ thỏa yêu cầu. Ta viết lại phương trình đã cho thành
$$m^6-m=n^3-5n^2.$$
Tiếp đó, ta xét các hiệu sau đây
\begin{align*}
    \tron{m^6-m}-\bigg(\tron{m^2}^3-5m^2\bigg)&=5m^2-m=m\tron{5m-1},\\
    \bigg(\tron{m^2+1}^3-5\tron{m^2+1}\bigg)-\tron{m^6-m}&=3m^4-2m^2+m-4\\&=(m-2)\tron{3m^3+6m^2+10m+21}+38.
\end{align*}
Với $m\ge 2,$ các hiệu trên điều dương, và khi ấy
\[\tron{m^2}^3-5m^2<m^6-m<\tron{m^2+1}^3-5\tron{m^2+1}.\tag{*}\label{kepdongrang}\]
Bây giờ, ta sẽ đi chứng minh hàm số
$f\tron{x}=x^3-5x^2$
đồng biến trên tập các số thực không nhỏ hơn $2.$ Thật vậy, lấy hai số $x_1,\ x_2$ bất kì thỏa mãn $x_1>x_2\ge 2,$ và ta có
$$\dfrac{f\tron{x_1}-f\tron{x_2}}{x_1-x_2}=\dfrac{x_1^3-x_2^3-5x_1+5x_2}{x_1-x_2}=x_1^2+x_1x_2+x_2^2-5\ge  2^2+2\cdot 2+2^2-5>0.$$
Tính đồng biến của hàm $f(x)$ trên tập các số thực không nhỏ hơn $2$ kết hợp với (\ref{kepdongrang}) giúp ta chỉ ra $m<n<m+1,$ mâu thuẫn do $m,n$ đều nguyên. Như vậy $m=1,$ và thay ngược lại ta tìm ra $n=5.$ \\
Cặp số duy nhất thỏa yêu cầu là $(m,n)=(1,5).$}
\end{gbtt}

\begin{gbtt}
Tìm tất cả các số tự nhiên $n$ sao cho $2n+7$ và $18n+22$ là số chính phương.
\loigiai{
Từ giả thiết, ta có số 
$(2n+7)(18n+22)=36n^2+170n+154$
là số chính phương.\\ Rõ ràng $n=1$ không thỏa yêu cầu. Với mọi số nguyên $n\ge 2,$ ta có
\begin{align*}
    36n^2+170n+154-\tron{6n+13}^2&=14n-15>0,\\
    \tron{6n+15}^2-\tron{36n^2+170n+154}&=10n+71>0,    
\end{align*}
thế nên $36n^2+170n+154$ là số chính phương chỉ khi $36n^2+170n+154=\tron{6n+14}^2.$ \\Ta tìm ra $n=21$ từ đây. Thử lại, ta thấy thỏa mãn.}
\end{gbtt}

\begin{gbtt}
Tìm tất cả các số nguyên dương $n$ sao cho $n+1$ và $n^3+n^2-2n+6$ là số chính phương.
\loigiai{
Với số nguyên $n$ thỏa yêu cầu, ta dễ dàng chỉ ra $n\ge 1$ và
$$(n+1)(n^3+n^2-2n+6)=n^4 +2n^3-n^2+4n+6$$ 
cũng là số chính phương. Ta có đánh giá sau đây
$$\tron{n^2+n-1}^{2} \leq n^4 + 2 n^3 - n^2 + 4 n + 6<\tron{n^2+n+2}^{2}.$$
Đánh giá trên cho ta $n^4 + 2 n^3 - n^2 + 4 n + 6$ bằng $\tron{n^2+n}^{2}$ hoặc $\tron{n^2+n+1}^{2}$.\\ Kiểm tra trực tiếp, ta thu được $n=3$ là giá trị thỏa mãn đề bài của $n.$}
\end{gbtt}

\begin{gbtt}
Tìm tất cả các số nguyên dương $x,y$ sao cho $4x^2+9y+3$ và $4y^2+9x+3$ là số chính phương.
\loigiai{
Do vai trò của $x,y$ như nhau nên không mất tổng quát, ta giả sử $x\ge y.$ \\
Với mọi số nguyên dương $x,$ ta dễ dàng có được nhận xét
$$(2x)^2<4x^2+9y+3\le 4x^2+9x+3<(2x+3)^2.$$
Do $4x^2+9y+3$ là số chính phương, ta xét các trường hợp sau.
\begin{enumerate}
    \item Nếu $4x^2+9y+3=(2x+1)^2,$ ta có $9y+2=4x.$ Xét trong hệ đồng dư modulo $9$ thì $x$ chia $9$ dư $5.$ Ta đặt $x=9k+5,$ trong đó $k$ là số tự nhiên, thế thì $y=4k+2.$ Đồng thời, phép đặt này cho ta
    $$4y^2+9x+3=4(4k+2)^2+9(9k+5)+3=64k^2+145k+64.$$
    Với mọi số tự nhiên $k,$ ta luôn có
    $$(8k+8)^2\le 64k^2+145k+64<(8k+10)^2.$$
    Do $64k^2+145k+64$ là số chính phương nên $64k^2+145k+64\in\{(8k+8)^2;(8k+9)^2\}.$
     \begin{itemize}
\item \chu{Trường hợp 1.} Với $64k^2+145k+64=(8k+8)^2,$ ta tìm ra $k=0,$ từ đây $x=5$ và $y=2.$
\item \chu{Trường hợp 2.} Với $64k^2+145k+64=(8k+9)^2,$ ta tìm ra $k=17,$ từ đây $x=158$ và $y=70.$
\end{itemize}
Trường hợp này cho ta $(x,y)=(5,2)$ và $(x,y)=(158,70).$
    \item Nếu $4x^2+9y+3=(2x+2)^2,$ ta có $9y=8x+1.$\\ Tương tự trường hợp trước, ta có thể đặt $x=9k+1,y=8k+1.$ Phép đặt này cho ta
    $$4y^2+9x+3=4(8k+1)^2+9(9k+2)+3=256k^2+145k+25.$$
    Với mọi số tự nhiên $k,$ ta luôn có
    $$(16k+4)^2<256k^2+145k+25\le (16k+5)^2.$$
    Do $256k^2+145k+25$ là số chính phương nên $256k^2+145k+25= (16k+5)^2.$ \\Thử trực tiếp, ta tìm được $k=0$ thỏa mãn trường hợp này, và khi đó $x=1,y=1.$
\end{enumerate}
Kết luận, có năm cặp $(x,y)$ thỏa yêu cầu là $(1,1),\ (2,5),\ (5,2),\ (70,158)$ và $(158,70).$}
\end{gbtt}

\begin{gbtt}
Tìm tất cả các bộ ba số nguyên dương $(a,b,c)$ sao cho $(a+b+c)^2-2a+2b$ là số chính phương.
\nguon{Chuyên Toán Vĩnh Phúc 2021}
\loigiai{
Ta xét các hiệu sau
    \begin{align*}
        (a+b+c+1)^2-\left[(a+b+c)^2-2a+2b\right]&=2a+2b+2c+1+2a-2b\\&=4a+2c+1>0,\\
        \left[(a+b+c)^2-2a+2b\right]-(a+b+c-1)^2&=2a+2b+2c-1-2a+2b\\&=4b+2c-1>0.  
    \end{align*}
    Các đánh giá trên cho ta
    $$(a+b+c-1)^2<(a+b+c)^2-2a+2b<(a+b+c+1)^2.$$
    Như vậy, $(a+b+c)^2-2a+2b$ là số chính phương khi và chỉ khi
    $$(a+b+c)^2-2a+2b=(a+b+c)^2,$$
    tức là $a=b.$ Nói cách khác, tất cả các bộ $(a,b,c)$ thỏa mãn đề bài là $(k,k,t),$ với $k,t$ là các số nguyên dương.}
\end{gbtt}

\begin{gbtt}
Tìm tất cả các số nguyên dương $x,y$ thỏa mãn $x^2y^4-y^3+1$ là số chính phương.
\loigiai{
Từ giả thiết, ta suy ra $4x^4y^4-4x^2y^3+4x^2$ cũng là số chính phương. Ta nhận thấy rằng
\begin{align*}
    4x^4y^4-4x^2y^3+4x^2-\left(2x^2y^2-y-1\right)^2&=4x^2(y^2+1)-(y+1)^2\\
    &\ge 4\left(y^2+1\right)-(y+1)^2 \\
    &\ge 3y^2-2y+3\\&\ge 3\\&>0
    \\
    4x^4y^4-4x^2y^3+4x^2-\left(2x^2y^2-y+1\right)^2&=-4x^2\left(y^2-1\right)-(y-1)^2\\&\le 0.
\end{align*}
Như vậy, $\left(2x^2y^2-y-1\right)^2\le 4x^4y^4-4x^2y^3+4x^2\le\left(2x^2y^2-y+1\right)^2.$ Ta xét hai trường hợp sau đây.
\begin{enumerate}
    \item Với $4x^4y^4-4x^2y^3+4x^2=\left(2x^2y^2-y+1\right)^2,$ ta có $$-4x^2\left(y^2-1\right)-(y-1)^2=0\Rightarrow (y-1)\left(4x^2y+4x^2-y+1\right)=0\Rightarrow y=1.$$
    \item Với $4x^4y^4-4x^2y^3+4x^2=\left(2x^2y^2-y\right)^2,$ ta có $$4x^2=y^2\Rightarrow (2x-y)(2x+y)=0\Rightarrow 2x=y.$$    
\end{enumerate}
Kết luận, các cặp $(x,y)$ thỏa mãn đề bài bao gồm $(x,1)$ và $(x,2x),$ trong đó $x$ là một số nguyên dương tùy ý.}
\end{gbtt}

\begin{gbtt}
Tìm tất cả các số nguyên dương $a,b$ sao cho $a^3b^3+4a^2-3b$ là số lập phương.
\loigiai{
Trong bài toán này, ta xét các trường hợp sau.
\begin{enumerate}
    \item Nếu $a\ge 2$ và $b\ge 2,$ ta xét các hiệu
    \begin{align*}
        (ab+1)^3-\tron{a^3b^3+4a^2-3b}&=a^2\tron{3b^2-4}+3ab+3b+1>0,\\
        \tron{a^3b^3+4a^2-3b}-(ab-1)^3&=3b\tron{a^2b-a-1}+4a^2+1
        \\&>3b(4a-a-1)+4a^2+1
        \\&>0.
    \end{align*}
    Các đánh giá theo hiệu trên chứng tỏ
    $$(ab-1)^3<a^3b^3+4a^2-3b<(ab+1)^3.$$
    Do $a^3b^3+4a^2-3b$ là số lập phương nên $a^3b^3+4a^2-3b=a^3b^3,$ hay $4a^2=3b.$ \\
    Ta có $a$ chia hết cho $3.$ Đặt $a=3k$ với $k$ nguyên dương, ta tìm được $b=12k^2.$
    \item Nếu $a=1,$ ta có $a^3b^3+4a^2-3b=b^3-3b+4$ là số lập phương. Do
    $$(b-1)^3<b^3-3b+4<(b+1)^3$$
    nên $b^3-3b+4=b^3.$ Ta không tìm được $b$ nguyên từ đây.
    \item Nếu $b=1,$ ta có $a^3b^3+4a^2-3b=a^3+4a^2-3$ là số lập phương. Do
    $$a^3<a^3+4a^2-3<(a+2)^3$$
    nên $a^3+4a^2-3=(a+1)^3.$ Ta tìm ra $a=4$ từ đây.
\end{enumerate}
Kết luận, tất cả các cặp $(a,b)$ thỏa mãn đề bài gồm $(4,1)$ và dạng tổng quát
$$\tron{3k,12k^2},\text{ với }k\text{ là số nguyên dương tùy ý}.$$
}
\end{gbtt}


\begin{gbtt}
Tìm tất cả các số nguyên dương $a, b, c$ sao cho cả ba số
$$4a^2+5b,\quad 4b^2+5c,\quad 4c^2+5a$$ 
đều là số chính phương.
\nguon{Chuyên Khoa học Tự nhiên 2020}
\loigiai{
Không mất tính tổng quát, giả sử $a=\max\left\{a,b,c\right\}$. Giả sử kể trên cho ta
$$4a^2<4a^2+5b\le 4a^2+5a<(2a+2)^2.$$ 
Do  $4a^2+5b$ chính phương, bắt buộc $4a^2+5b=(2a+1)^2$ hay $5b=4a+1.$ Xét các số dư của $a$ khi chia cho $5,$ ta chỉ ra $a$ chia $5$ dư $1.$ Đặt $a=5k+1,$ ta có $b=4k+1.$ Từ kết quả này, ta thu được
$$4b^2+5c=4\tron{4k+1}^2+5c.$$
Với $k=0,$ ta tìm ra $a=b=c=1.$ Với $k\ge 1,$ do $c\le 5k+1$ nên là
$$(8k+2)^2<4\tron{4k+1}^2+5c\le 4\tron{4k+1}^2+5(5k+1)\le \tron{8k+4}^2.$$
Dựa theo đánh giá bên trên, ta chỉ ra
$$4\tron{4k+1}^2+5(5k+1)\in\left\{\tron{8k+3}^2;\tron{8k+4}^2\right\}.$$
Hai trường hợp trên lần lượt cho ta $k=0$ và $k=-1,$ mâu thuẫn với điều kiện $k\ge 1.$\\ Như vậy, bộ $3$ số nguyên dương $(a,b,c)$ duy nhất thỏa mãn là $(1,1,1).$}
\end{gbtt}

\begin{gbtt}
Tìm tất cả các bộ số tự nhiên \(\left ( a,b,c \right )\) thỏa mãn
\[a^2+2b+c,\quad b^2+2c+a,\quad c^2+2a+b\]
đều là các số chính phương.
\loigiai{
Không mất tính tổng quát, ta giả sử $a=\max\{a;b;c\}.$ Ta sẽ có
$$a^2<a^2+2b+c<a^2+2a+a<(a+2)^2.$$
Do $a^2+2b+c$ là số chính phương nên $a^2+2b+c=(a+1)^2,$ hay là 
\[a=\dfrac{2b+c-1}{2}.\tag{1}\label{elmo13.1}\]
Tới đây, ta sẽ xét các trường hợp sau.
\begin{enumerate}
    \item Nếu $b\ge c,$ ta có nhận xét
    \begin{align*}
        b^2<b^2+2c+a&=b^2+2c+\dfrac{2b+c-1}{2}\\&\le b^2+2b+\dfrac{2b+b-1}{2}
        \\&\le b^2+\dfrac{7}{2}b-\dfrac{1}{2}
        \\&<(b+2)^2.
    \end{align*}
    Do $b^2+2c+a$ là số chính phương nên $b^2+2c+a=(b+1)^2,$ hay là 
    \[2c=2b-a+1.\tag{2}\label{elmo13.2}\]
    Từ (\ref{elmo13.1}) và (\ref{elmo13.2}), ta có $6b=5a+1,3c=a+2.$ Ta đặt $a=6n+1,b=5n+1,c=2n+1,$ khi đó
    $$c^2+2a+b=(2n+1)^2+2(6n+1)+5n+1=4n^2+21n+4.$$
    Tới đây, áp dụng so sánh $(2n+2)^2\le 4n^2+21n+4<(2n+6)^2$ để tìm được $n=0$ và $n=21.$ Trường hợp này cho ta các cặp $(a,b,c)$ là 
    $(1,1,1)\text{ và }(127,106,43).$
    \item Nếu $c\ge b,$ ta có nhận xét
    \begin{align*}
        c^2<c^2+2a+b&=c^2+(2b+c-1)+b\\&\le c^2+(2c+c-1)+b
        \\&\le c^2+4c-1
        \\&<(c+2)^2.
    \end{align*}
    Do $c^2+2a+b$ là số chính phương nên $c^2+2a+b=(c+1)^2,$ hay là 
    \[2c=2a+b-1.\tag{3}\label{elmo13.3}\]
    Từ (\ref{elmo13.3}), ta suy ra $2c=2a+b-1\ge 2a,$ thế nên $c\ge a.$ Do giả sử của chúng là là $a=\max \{a;b;c\}$ nên bắt buộc dấu bằng $a=b=c=1$ phải xảy ra.
\end{enumerate}
Kết luận, có bốn cặp $(a,b,c)$ thỏa yêu cầu bài toán là
$$(1,1,1),\ (127,106,43),\ (106,43,127),\ (43,127,106).$$}
\end{gbtt}

\begin{gbtt}
Tìm các số tự nhiên $n$ sao cho $n^4+3n^3+n^2+5$ là lũy thừa cơ số $7$ của một số tự nhiên.
\loigiai{
Ta dễ dàng chứng minh được $n$ chẵn. Khi đó
$$n^4+3n^3+n^2+5\equiv 1\pmod{4}.$$
Một lũy thừa của $7$ chia $4$ dư $1$ chỉ khi đây là lũy thừa số mũ chẵn. Như vậy, $n^4+3n^3+n^2+5$ là một số chính phương. Với mọi số nguyên $n\ge 9,$ ta có
$$\tron{2n^2+3n-2}^2<4\tron{n^4+3n^3+n^2+5}<\tron{2n^2+3n-1}^2.$$
Do $n^4+3n^3+n^2+5$ là số chính phương nên trường hợp $n\ge 9$ không xảy ra.
\\Đối với $n\le 8,$ thử trực tiếp, ta có $n=2.$}
\end{gbtt}

\begin{gbtt}
Tìm tất cả các số tự nhiên $n$ sao cho $13^n+7n+13$ là số chính phương.
\loigiai{
Nếu $n$ là số lẻ, ta có đánh giá đồng dư sau đây
$$13^n+7n+13\equiv (-1)^n+6\equiv 5\pmod{7}.$$
Không có số chính phương nào đồng dư $5$ theo modulo $7,$ chứng tỏ $n$ chẵn. Ta đặt $n=2m,$ khi đó
$$13^n+7n+13=13^{2m}+14m+13.$$
Do $13^{2m}+14m+13>13^{2m}$ và $13^{2m}+14m+13$ là số chính phương, ta suy ra
$$13^{2m}+14m+13\ge\left(13^m+1\right)^2=13^{2m}+2\cdot 13^m+1.$$
Đánh giá trên cho ta 
$2\cdot13^m \leq 14m+12,$
hay là
$13^m\le 7m+6.$
Với $m\ge 2,$ bất đẳng thức trên đảo chiều. Theo đó, ta cần chứng minh bất đẳng thức sau với mọi $m\ge 2$ bằng phương pháp quy nạp
\[13^m>7m+6.\tag{*}\]
Hiển nhiên (*) đúng với $m=2.$ Giả sử (*) đúng với $m=2,3,4,\ldots,k,$ thế thì
$$13^{k+1}=13\cdot13^k>13(7k+6)>7(k+1)+6.$$
Theo nguyên lí quy nạp, (*) được chứng minh với mọi $m\ge 2.$ Điều này đồng nghĩa với việc chỉ tồn trường hợp $m=1.$ Thử trực tiếp, ta thấy đây là giá trị duy nhất thỏa mãn đề bài.}
\end{gbtt}

\begin{gbtt}
Tìm tất cả các số tự nhiên $n$ sao cho $4^n+3n+7$ là số lập phương.
\loigiai{Trước hết, ta sẽ chỉ ra $n$ chia hết cho $3.$ Thật vậy
\begin{enumerate}
    \item Nếu $n=3k+1,$ xét trong hệ đồng dư modulo $9$ ta có
    $$4^n+3n+7=4^{3k+1}+3\tron{3k+1}+7=4\cdot64^k+9k+10\equiv 4+0+1\equiv 5\pmod{9}.$$
    Không có số lập phương nào chia $9$ dư $5.$ Trường hợp này không xảy ra.
    \item Nếu $n=3k+2,$ xét trong hệ đồng dư modulo $9$ ta có
    $$4^n+3n+7=4^{3k+2}+3\tron{3k+2}+7=16\cdot64^k+9k+13\equiv 7+0+4\equiv 2\pmod{9}.$$
    Không có số lập phương nào chia $9$ dư $2.$ Trường hợp này không xảy ra.    
\end{enumerate}
Các mâu thuẫn trên chứng tỏ $n$ chia hết cho $3.$ Ta đặt $n=3m,$ khi đó
$$4^n+3n+7=4^{3m}+9m+7.$$
Do $4^{3m}+9m+7>4^{3m}$ và $4^{3m}+9m+13$ là số lập phương, ta suy ra
$$4^{3m}+9m+7\ge\left(4^m+1\right)^3=4^{3m}+3\cdot4^{2m}+3\cdot4^m+1.$$
Đánh giá trên cho ta $3m+2\ge 4^{2m}+4^m.$ Ta sẽ đi chứng minh bất đẳng thức sau với mọi $m\ge 1$
\[4^{2m}>3m+2.\tag{*}\label{mod99}\]
Hiển nhiên (\ref{mod99}) đúng với $m=1.$ Giả sử (\ref{mod99}) đúng với $m=2,3,4,\ldots,k,$ thế thì
$$4^{2(m+1)}=16\cdot4^{2m}>16(3m+2)>3(m+1)+2.$$
Theo nguyên lí quy nạp, (*) được chứng minh với mọi $m\ge 1.$ Điều này đồng nghĩa với việc trường hợp $m\ge 1$ không thỏa. Thử trực tiếp với $m=0,$ ta tìm được đáp số bài toán là $n=0.$}
\end{gbtt}

\begin{gbtt}
Cho $x,y$ là các số nguyên dương. Chứng minh $x^2+y+1$ và $y^2+4x+3$ không đồng thời là số chính phương.
\loigiai{
Ta giả sử phản chứng rằng, $x^2+y+1$ và $y^2+4 x+3$ đều là số chính phương. Ta có nhận xét
$$x^{2}+y+1>x^2.$$
Do $x^{2}+y+1$ là số chính phương, ta suy ra $x^2+y+1\ge (x+1)^2,$ tức là $y\ge 2x.$ Ta lại có nhận xét
$$y^2+4x+3\ge y^2.$$
Do $y^2+4x+3$ là số chính phương, ta suy ra $y^2+4x+3\ge (y+1)^2,$ tức là $2x+1\ge y.$ \\
Sử dụng các kết quả vừa thu được, ta có
$y\in \{2x;2x+1\}.$
Ta sẽ đi xét các trường hợp kể trên.
\begin{enumerate}
    \item Nếu $y=2x,$ ta có $y^2+4x+3=4x^2+4x+3=(2x+1)^2+2$ không là số chính phương do
    $$(2x+1)^2+2\equiv 3\pmod{4}.$$
    \item Nếu $y=2x+1,$ ta có $x^2+y+1=x^2+2x+2=(x+1)^2+1.$ Do
    $$(x+1)^2<(x+1)^2+1<(x+2)^2$$
    nên $(x+1)^2+1$ không là số chính phương, với $x$ nguyên dương.
\end{enumerate}
Các mâu thuẫn trên chứng tỏ giả sử phản chứng là sai. Bài toán được chứng minh.}
\end{gbtt}

\begin{gbtt}
Cho các số nguyên dương $x,y.$ Chứng minh rằng $x^3+y^2+5x+2$ và $y^3+xy+y^2+3$ không cùng là số lập phương.

\loigiai{
Ta giả sử tồn tại các số nguyên dương $x,y$ thỏa mãn $x^3+y^2+5x+2$ và $y^3+xy+y^2+3$ cùng là số lập phương. Do $x^3+y^2+5x+2>x^3$ nên $x^3+y^2+5x+2\ge (x+1)^3,$ hay là
\[y^2\ge 3x^2-2x-1.\tag{1}\label{kep.chau.ba.1}\]
Chứng minh tương tự, ta có $y^3+xy+y^2+3\ge (y+1)^3,$ hay là
\[xy\ge 2y^2+3y-2.\tag{2}\label{kep.chau.ba.2}\]
Tới đây, ta xét các trường hợp sau.
\begin{enumerate}
    \item Với $x=1,$ ta có $y^3+xy+y^2+3=y^3+y^2+y+3$ là số lập phương. Dựa theo nhận xét
    $$y^3<y^3+y^2+y+3<(y+1)^3,$$
    ta chỉ ra mâu thuẫn trong trường hợp này.
    \item Với $x\ge 2,$ ta có $x^2-2x-1>0,$ kết hợp với (\ref{kep.chau.ba.1}) thì
    $$y^2\ge 2x^2+\tron{x^2-2x-1}>2x^2.$$
    Do cả $x$ và $y$ nguyên dương nên $y> x\sqrt{2}.$ Đánh giá này kết hợp với (\ref{kep.chau.ba.2}) cho ta
    $$x\ge \dfrac{2y^2+3y-2}{y}>\dfrac{2y^2}{y}=2y>2x\sqrt{2}.$$
    Ta cũng chỉ ra mâu thuẫn trong trường hợp này.
\end{enumerate}
Như vậy, giả sử phản chứng là sai. Bài toán được chứng minh.}
\end{gbtt}

\begin{gbtt}
Tìm tất các các số nguyên tố $p$ sao cho tổng các ước nguyên dương của $p^4$ là số chính phương.

\loigiai{
Tập các ước của $p^4$ là $ \left\{1; p; p^2; p^3; p^4\right\}$. Tổng các số trong tập này là một số chính phương, vậy nên tồn tại số tự nhiên $n$ sao cho
$4p^4+4p^3+4p^2+4p+4=4n^2.$
Dựa vào đánh giá
$$\left(2p^2+p\right)^2<4p^4+4p^3+4p^2+4p+4\le\left(2p^2+p+2\right)^2,$$
ta chỉ ra $4p^4+4p^3+4p^2+4p+4=\left(2p^2+p+1\right)^2$, hay là $p=3.$ \\Đây là số nguyên tố duy nhất thỏa yêu cầu.}
\end{gbtt}

\begin{gbtt}
Cho \(n\) là một số nguyên dương. Tìm tất cả các ước nguyên dương \(d\) của \(3n^2\) thỏa mãn \(n^2+d\) là bình phương của một số nguyên.
\loigiai{
Nếu \(d\) là ước của \(3n^2\), tồn tại các số nguyên dương \(k\) và \(m\) thỏa mãn $3n^{2}=dk$ và $n^2+d=m^2$.\\
Với phép đặt như vậy, ta lần lượt suy ra
$$n^{2}+\frac{3n^{2}}{k}=m\Rightarrow(mk)^{2}=n^{2}\left(k^{2}+3 k\right).$$
Rõ ràng, $k^{2}+3k$ phải là một số chính phương. Do
$$k^{2}<k^{2}+3 k<(k+2)^{2},$$
nên ta chỉ ra $k^{2}+3 k=(k+1)^{2},$ nghĩa là $k=1.$ \\
Với $k=1,$ ta tìm được $d=3n^2$ thỏa yêu cầu đề bài.}
\end{gbtt}
%anh vẫn đang soát sách à???
\begin{gbtt}
Tìm tất cả các số nguyên dương $a$ thỏa mãn với mọi số nguyên dương $n,$ ta có $4\left(a^n+1\right)$ là số lập phương.
\nguon{Iran Team Selection Test 2008}
\loigiai{Từ giả thiết, ta suy ra cả $4\left(a^3+1\right)$ và $4\left(a^9+1\right)$ đều là số lập phương. Ta có nhận xét
$$4\left(a^9+1\right)=4\left(a^3+1\right)\left(a^6-a^3+1\right).$$
Nhận xét trên kết hợp với việc $a^3+1>0$ cho ta $a^6-a^3+1$ là một số lập phương. Xét hiệu, ta chỉ ra
$$\left(a^3-1\right)^2\le a^6-a^3+1\le a^6.$$
Tóm lại, $a^6-a^3+1$ bằng $\left(a^3-1\right)^2$ hoặc bằng $a^6.$ \\
Ta tính được $a=1$ từ đây, và đó là kết quả bài toán.}
\end{gbtt}

\begin{gbtt}
Tìm tất cả các số nguyên dương $n$ sao cho $n^4+8n+11$ có thể viết được thành tích ít nhất hai số nguyên dương liên tiếp.
\nguon{Junior Balkan Mathematical Olympiad 2008}
\loigiai{Trước hết, ta sẽ lập bảng đồng dư sao theo modulo $3$ của $n$
\begin{center}
    \begin{tabular}{c|c|c|c}
       $n$  &  $0$ & $1$ & $2$\\
       \hline
        $n^4+8n+11$ &  $2$ & $2$ & $1$\\
    \end{tabular}
\end{center}
Căn cứ vào bảng, $n^4+8n+11$ không thể chia hết cho $3,$ và do đó nó chỉ có thể là tích tối đa hai số nguyên dương liên tiếp. Ta đặt $n^4+8n+11=m(m+1),$ trong đó $m$ là số nguyên dương. Ta có
$$4n^4+32n+45=(2m+1)^2.$$
Theo lập luận kể trên, $4n^4+32n+45$ là số chính phương. Thử với $n=1,n=2,$ ta thấy $n=1$ thỏa mãn. Với $n\ge 3,$ ta chứng minh được
$$\tron{2n^2}^2<4n^4+32n+45<\tron{2n^2+4}^2,$$
do vậy $4n^4+32n+45\in\left\{\tron{2n^2+1}^2;\tron{2n^2+2}^2;\tron{2n^2+3}^2\right\}.$\\
Ta không tìm được $n$ nguyên từ đây. Nói tóm lại, đáp số bài toán là $n=1.$}
\end{gbtt}

\begin{gbtt}
Với mỗi số thực $a$ ta gọi phần nguyên của $a$ là số nguyên lớn nhất không vượt quá $a$ và ký hiệu là $[{a}]$. Chứng minh rằng vói mọi số nguyên dương ${n}$, biểu thức
$${n}+\left[\sqrt[3]{{n}-\dfrac{1}{27}}+\dfrac{1}{3}\right]^{2}$$ không biểu diễn được dưới dạng lập phương của một số nguyên dương.
\nguon{Chuyên Khoa học Tự nhiên 2011}
\loigiai{
Từ giả thiết, ta có thể đặt $\left[\sqrt[3]{n-\dfrac{1}{27}}+\dfrac{1}{3}\right]=a.$ Dựa vào tính chất của phần nguyên, ta có
\begin{align*}
a \leq \sqrt[3]{n-\dfrac{1}{27}}+\dfrac{1}{3}<a+1 &\Rightarrow a^{3}-a^{2}+\dfrac{4 a}{3} \leq n<a^{3}+2 a^{2}+\dfrac{7 a}{3}+\dfrac{1}{3}
\\&\Rightarrow a^{3}+\dfrac{4 a}{3} \leq n+a^{2}<a^{3}+3 a^{2}+\dfrac{7 a}{3}+\dfrac{1}{3}.
\end{align*}
Bằng biến đổi đại số trực tiếp, ta chỉ ra 
$$a^3<a^3+\dfrac{4a}{3}<a^3+3a^2+\dfrac{7a}{3}+\dfrac{1}{3}<\left(a+1\right)^3$$ 
với mọi $a$ nguyên dương. Đánh giá này cho ta
$a^3<n+a^2<\left(a+1\right)^3.$\\
Theo lí thuyết đã học, $n+a^2$ không là số lập phương. Bài toán được chứng minh.}
\end{gbtt}

\begin{gbtt}
Tìm tất cả các số nguyên dương $x,y$ thỏa mãn $4^x+4^y+1$ là một số chính phương.
\nguon{Korean Mathematical Olympiad 2007}
\loigiai{
Không mất tính tổng quát, ta giả sử $x\ge y.$ Đầu tiên, ta đặt $4^x+4^y+1=z^2,$ ở đây $z$ là số nguyên dương. Ta dễ dàng chứng minh $z$ lẻ, vậy nên ta xét biến đổi
\begin{align*}
    4^x+4^y=z^2-1
    &\Rightarrow 4^y\left(4^{x-y}+1\right)=(z-1)(z+1) 
    \\&\Rightarrow 4^{y-1}\left(4^{x-y}+1\right)=\left(\dfrac{z-1}{2}\right)\left(\dfrac{z+1}{2}\right).
    \tag{*}
\end{align*}
Do hai số $\dfrac{z-1}{2}$ và $\dfrac{z+1}{2}$ là hai số nguyên liên tiếp, chỉ một trong chúng chia hết cho $4^{y-1}.$ 
\begin{enumerate}
    \item Nếu $\dfrac{z-1}{2}$ chia hết cho $4^{y-1},$ ta suy ra $\dfrac{z-1}{2}\ge 4^{y-1}.$
    Kết hợp với $(*)$, ta được
    \begin{align*}
        4^{y-1}\left(4^{x-y}+1\right)\ge 4^{y-1}\left(4^{y-1}+1\right)
        &\Rightarrow 4^{x-y}+1\ge 4^{y-1}+1
        \\&\Rightarrow x-y\ge y-1
        \\&\Rightarrow x\ge 2y-1.
    \end{align*}
    Nhận xét này cho phép ta chỉ ra
    $$\left(2^x\right)^2<2^{2x}+2^{2y}+1\le 2^{2x}+2^{x+1}+1=\left(2^x+1\right)^2.$$
    Theo như phần lí thuyết đã học ta có $2y=x+1.$ Các bộ $(x,y)$ thỏa mãn trường hợp này có dạng
    $$(x,y)=\left(y,2y-1\right).$$
    \item Nếu $\dfrac{z+1}{2}$ chia hết cho $4^{y-1},$ ta suy ra $\dfrac{z+1}{2}\ge 4^{y-1}.$
    Kết hợp với (*), ta được
    \begin{align*}
        4^{y-1}\left(4^{x-y}+1\right)\ge 4^{y-1}\left(4^{y-1}-1\right)
        &\Rightarrow 4^{x-y}+1\ge 4^{y-1}-1
    \end{align*}    
    Nếu như $x-y<y-1,$ ta sẽ có
    $$4^{x-y}+1\ge 4^{y-1}-1\ge 4^{x-y+1}-1=4\cdot4^{x-y}-1.$$
    Chuyển vế, ta được $4^{x-y}\le \dfrac{2}{3},$ vô lí. Do đó $x-y\ge y-1.$ Tương tự trường hợp trước, ta tìm ra $$(x,y)=(y,2y-1).$$
\end{enumerate}
Tổng kết lại, tất cả các bộ $(x,y)=(y,2y-1)$ đều thỏa yêu cầu bài toán.}
\begin{luuy}
Dạng bài tập bất đẳng thức trong chia hết này đã từng xuất hiện ở \chu{chương I}. Kết hợp thêm công cụ sử dụng \chu{phương pháp kẹp lũy thừa}, bài toán được chứng minh hoàn toàn.
\end{luuy}
\end{gbtt}

\begin{gbtt}
Cho $x$ là một số thực thỏa mãn $4x^5-7$ và $4x^{13}-7$ đều là các số chính phương.
\begin{enumerate}[a,]
    \item Chứng minh rằng $x$ là số nguyên dương.
    \item Tìm tất cả các giá trị có thể của $x.$
\end{enumerate}
\nguon{Trại hè Hùng Vương 2018}
\loigiai{
\begin{enumerate}[a,]
    \item Từ giả thiết, ta có phép đặt sau với $a,b$ là số nguyên dương
    \begin{align*}
        4x^5-7&=a^2\tag{1}\\
        4x^{13}-7&=b^2\tag{2}
    \end{align*}
    Dựa vào phép đặt này, ta có $x^5=\dfrac{a^2+7}{4}$ và $x^{13}=\dfrac{b^2+7}{4}$ đều là số hữu tỉ, thế nên
    $$x=\dfrac{\left(x^5\right)^8}{\left(x^{13}\right)^3}=\dfrac{\left(\dfrac{a^2+7}{4}\right)^8}{\left(\dfrac{b^2+7}{4}\right)^3}.$$
    là số hữu tỉ. Ta tiếp tục đặt $x=\dfrac{p}{q},$ với $(p,q)=1.$ Phép đặt này cho ta
    $$4\left(\dfrac{p}{q}\right)^5=a^2+7.$$
    Ta được $q^5\mid 4p^5,$ nhưng do điều kiện phép đặt là $(p,q)=1$ nên $q^5\mid 4,$ và thế thì $q=1.$ Chứng minh này, hiển nhiên cho ta $x$ nguyên dương.
    \item Rõ ràng, $a$ và $x$ cùng tính chẵn lẻ. Với $a$ là số lẻ, ta có
    $$a^2=4x^5-7\equiv 5\pmod{8}.$$
    Đồng dư thức trên không xảy ra, chứng tỏ $a$ là số chẵn. Lấy tích theo vế của (1) và (2), ta được
    $$\left(4x^5-7\right)\left(4x^{13}-7\right)=(ab)^2.$$
    Ta xét các hiệu
    \begin{align*}
        \left(4x^5-7\right)\left(4x^{13}-7\right)-\left(4x^9-\dfrac{7x^4}{2}-1\right)^2&=8x^9-\dfrac{49x^8}{8}-28x^5-7x^4+48,\\
        \left(4x^9-\dfrac{7x^4}{2}\right)^2-\left(4x^5-7\right)\left(4x^{13}-7\right)&=\dfrac{49x^8}{4}+28x^5-49.
    \end{align*}
    Với mọi số thực $x\ge 4,$ ta có
    \begin{align*}
        8x^9-\dfrac{49x^8}{8}-28x^5-7x^4+48> 8x^9-\dfrac{49x^8 \cdot x}{8 \cdot 4}-\dfrac{28x^5 \cdot x^3}{4^3}-\dfrac{7x^4 \cdot x^4}{4^4}&>0, \\  
        \dfrac{49x^8}{4}+28x^5-49\ge 49 \cdot4^7+28 \cdot4^5-49&>0.
    \end{align*}    
    Theo như phần lí thuyết đã học, ta suy ra $\left(4x^5-7\right)\left(4x^{13}-7\right)$ không thể là số chính phương với $x\ge 4.$ Đối với $x=2,$ thử lại, ta thấy đây là giá trị duy nhất của $x$ thỏa mãn đề bài.
\end{enumerate}}
\end{gbtt}